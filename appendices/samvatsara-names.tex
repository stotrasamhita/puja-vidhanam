% !TeX program = XeLaTeX
% !TeX root = ../pujavidhanam.tex
\section{संवत्सर-नामानि}
\label{app:samvatsara_names}

\twolineshloka
{प्रभवो विभवः शुक्लः प्रमोदोऽथ प्रजापतिः}
{अङ्गिराः श्रीमुखो भावः युवा धाता तथैव च}

\twolineshloka
{ईश्वरो बहुधान्यश्च प्रमाथी विक्रमो विषुः}
{चित्रभानुः स्वभानुश्च तारणः पार्थिवो व्ययः}

\twolineshloka
{सर्वजित्सर्वधारी च विरोधी विकृतिः खरः}
{नन्दनो विजयश्चैव जयो मन्मथदुर्मुखौ}

\twolineshloka
{हेविलम्बी विलम्बी च विकारः शर्वरी प्लवः}
{शुभकृच्छोभनः क्रोधी विश्वावसुपराभवौ}

\twolineshloka
{प्लवङ्गः कीलकः सौम्यः साधारणविरोधिकृत्}
{परिधावी प्रमादी च आनन्दो राक्षसो नलः}

\twolineshloka
{पिङ्गलः कालसिद्धार्थौ रौद्रिर्वै दुर्मतिस्तथा}
{दुन्दुभी रुधिरोद्गारी रक्ताक्षः क्रोधनोऽक्षयः}

\begin{multicols}{2}
\begin{enumerate}
\item प्रभव %1
\item विभव %2
\item शुक्ल %3
\item प्रमोद %4
\item प्रजापति %5
\item आङ्गिरस %6
\item श्रीमुख %7
\item भाव %8
\item युव %9
\item धातृ %10
\item ईश्वर %11
\item बहुधान्य %12
\item प्रमाथी %13
\item विक्रम %14
\item वृष %15
\item चित्रभानु %16
\item स्वभानु %17
\item तारण %18
\item पार्थिव %19
\item व्यय %20
\item सर्वजित् %21
\item सर्वधारी %22
\item विरोधी %23
\item विकृति %24
\item खर %25
\item नन्दन %26
\item विजय %27
\item जय %28
\item मन्मथ %29
\item दुर्मुखी %30
\item हेविलम्बी %31
\item विलम्बी %32
\item विकारी %33
\item शर्वरी %34
\item प्लव %35
\item शुभकृत् %36
\item शोभकृत् %37
\item क्रोधी %38
\item विश्वावसु %39
\item पराभव %40
\item प्लवङ्ग %41
\item कीलक %42
\item सौम्य %43
\item साधारण %44
\item विरोधिकृति %45
\item परिधावी %46
\item प्रमादी %47
\item आनन्द %48
\item राक्षस %49
\item नल %50
\item पिङ्गल %51
\item कालयुक्ति %52
\item सिद्धार्थी %53
\item रौद्र %54
\item दुर्मति %55
\item दुन्दुभि %56
\item रुधिरोद्गारी %57
\item रक्ताक्षी %58
\item क्रोधन %59
\item अक्षय %60
\end{enumerate}
\end{multicols}