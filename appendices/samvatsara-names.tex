% !TeX program = XeLaTeX
% !TeX root = ../pujavidhanam.tex
\chapt{संवत्सर-नामानि}
\label{app:samvatsara_names}

\twolineshloka
{प्रभवो विभवः शुक्लः प्रमोदोऽथ प्रजापतिः}
{अङ्गिराः श्रीमुखो भावो युवा धाता तथैव च}

\twolineshloka
{ईश्वरो बहुधान्यश्च प्रमाथी विक्रमो वृषः}
{चित्रभानुः सुभानुश्च तारणः पार्थिवो व्ययः}

\twolineshloka
{सर्वजित्सर्वधारी च विरोधी विकृतिः खरः}
{नन्दनो विजयश्चैव जयो मन्मथदुर्मुखौ}

\twolineshloka
{हेमलम्बो विलम्बोऽथ विकारी शार्वरी प्लवः}
{शुभकृच्छोभनः क्रोधी विश्वावसुपराभवौ}

\twolineshloka
{प्लवङ्गः कीलकः सौम्यः साधारणविरोधिकृत्}
{परिधावी प्रमादी च आनन्दो राक्षसो नलः}

\twolineshloka
{पिङ्गलः कालयुक्तश्च सिद्धार्थी रौद्रदुर्मती}
{दुन्दुभी रुधिरोद्गारी रक्ताक्षी क्रोधनः क्षयः}


\begin{multicols}{2}
\begin{enumerate}
\item प्रभवः 
\item विभवः 
\item शुक्लः 
\item प्रमोदः 
\item प्रजापतिः 
\item अङ्गिराः 
\item श्रीमुखः 
\item भावः 
\item युवा 
\item धाता 
\item ईश्वरः 
\item बहुधान्यः 
\item प्रमाथी 
\item विक्रमः 
\item वृषः 
\item चित्रभानुः 
\item सुभानुः 
\item तारणः 
\item पार्थिवः 
\item व्ययः 
\item सर्वजित् 
\item सर्वधारी 
\item विरोधी 
\item विकृतिः 
\item खरः 
\item नन्दनः 
\item विजयः 
\item जयः 
\item मन्मथः 
\item दुर्मुखः 
\item हेमलम्बः 
\item विलम्बः 
\item विकारी 
\item शार्वरी 
\item प्लवः 
\item शुभकृत् 
\item शोभनः 
\item क्रोधी 
\item विश्वावसुः 
\item पराभवः 
\item प्लवङ्गः 
\item कीलकः 
\item सौम्यः 
\item साधारणः 
\item विरोधिकृत् 
\item परितापी 
\item प्रमादी 
\item आनन्दः 
\item राक्षसः 
\item नलः 
\item पिङ्गलः 
\item कालयुक्तिः 
\item सिद्धार्थी 
\item रौद्रः 
\item दुर्मतिः 
\item दुन्दुभिः 
\item रुधिरोद्गारी 
\item रक्ताक्षः 
\item क्रोधनः 
\item क्षयः
\end{enumerate}
\end{multicols}

\sect{संवत्सर-श्लोका देवताश्च}

[१] संवत्सरः—प्रभवः, देवता—कमलजः\\

\fourlineindentedshloka
{अक्षाद्यङ्कि-पाणि-पद्ममभयं चिन्मुद्रिका हस्तयोः}
{बिभ्राणं चतुराननं प्रविलसत् पद्मासने सुस्थितम्}
{वक्षोभाग-विराजमान-विशद-श्री-ब्रह्मसूत्रोज्ज्वम्}
{वन्देऽहं प्रभवाभिधं कमलजं श्रेयोऽभिवृद्धिप्रदम्}

[२] संवत्सरः—विभवः, देवता—रमेशः\\

\fourlineindentedshloka
{भास्वत् किरीटं दरचक्र-शार्ङ्ग-}
{गदाभिरामैः सहितं चतुर्भिः}
{करैरनन्तासन-सन्निविष्टम्}
{ध्यायेद् रमेशं विभवाभिधानम्}

[३] संवत्सरः—शुक्लः, देवता—शूलपाणिः\\

\fourlineindentedshloka
{जटिलम् उरग-भूषाभूषिताङ्गं त्रिनेत्रम्}
{शशिधर-मकुटाग्रं व्याघ्र-चर्मोत्तरीयम्}
{वृषवरकृतवाहं शुक्लसंज्ञं गिरीशम्}
{नमदमरमुनीन्द्रं शूलपाणिं भजेऽहम्}

[४] संवत्सरः—प्रमोदः, देवता—गणेशः\\

\twolineshloka
{करीन्द्रवक्त्रं कनदेकदन्तोज्ज्वलं प्रमोदेति कृताभिधानम्}
{प्रत्यूह-शान्तिप्रदमाखुसंस्थं ध्यायेद्धृदब्जे सततं गणेशम्} %॥४॥

[५] संवत्सरः—प्रजापतिः, देवता—शक्तिः\\

\fourlineindentedshloka
{दन्तीन्द्र-वक्त्राब्ज-गतां च शक्तिम्}
{चतुर्भुजां चन्द्र-कलावतंसाम्}
{भक्तेष्टपाथोनिधि-चन्द्रभासम्}
{वन्दे प्रजापत्यभिधान-देवीम्}

[६] संवत्सरः—अङ्गिराः, देवता—षण्मुखः\\

\twolineshloka
{शक्त्युज्ज्वलैकबाहुं शशिधरमकुटं मयूरमारूढम्}
{अङ्गिरसं नाम्ना तं षण्मुखमीडे अमरेन्द्र-सेनान्यम्} %॥६॥

[७] संवत्सरः—श्रीमुखः, देवता—वल्ली\\

\twolineshloka
{चन्द्रमुखीं चारुदृशं कुक्कुटवाहां गुहादृतम् अनिशम्}
{श्रीमुखसंज्ञा वल्लीं श्रेयो वृद्धिप्रदां कलये} %॥७॥

[८] संवत्सरः—भावः, देवता—गौरी\\

\twolineshloka
{गौरी गिरीन्द्रतनयां वृषवरसंस्थां कलाधरां शशिनः}
{भावाख्याम् अहमीडे भवकुतुकाम्भोधि-पूर्ण-चन्द्र-कलाम्} %॥८॥

[९] संवत्सरः—युवा, देवता—ब्राह्मी\\

\twolineshloka
{युव संज्ञाम् अहमीडे ब्राह्मीं ब्रह्मादि वन्दितां शक्तिम्}
{हंसवरारूढां तां शम्भु-मनोहारि-रूप-सौभाग्याम्} %॥९॥

[१०] संवत्सरः—धाता, देवता—माहेश्वरी\\

\twolineshloka
{माहेश्वरी हृदब्जे विलसतु सततं च धातु संज्ञां मे}
{चन्द्रावतंसमहिता श्रेयोवृद्धिप्रदा शम्भोः} %॥१०॥

[११] संवत्सरः—ईश्वरः, देवता—कौमारी\\

\twolineshloka
{कौमारीम् अहमीडे सम्पद्दात्रीं सदा शुकारूढाम्}
{तामीश्वराभिधानां शङ्कर-तोष-प्रदां स्वरूपेण} %॥११॥

[१२] संवत्सरः—बहुधान्यः, देवता—वैष्णवी\\

\twolineshloka
{बहुधान्य नामधेयां सततं हृदयेन वैष्णवीं कलये}
{शङ्कर-वामाङ्कस्थां नमताम् इष्टार्थ-दायिनीम् अनिशम्} %॥१२॥

[१३] संवत्सरः—प्रमाथी, देवता—वाराही\\

\twolineshloka
{वाराहीं प्रणमामो धूर्जटि-हर्षाब्धि-पूर्ण-चन्द्र-कलाम्}
{श्रेयोभिवृद्धि कर्त्रीं कृतनिजनाम्नीं प्रमाथीति} %॥१३॥

[१४] संवत्सरः—विक्रमः, देवता—माहेन्द्री\\

\twolineshloka
{माहेन्द्रीं कलयामो विक्रम-संज्ञां वृषासनारूढाम्}
{भासुर-चन्द्र-किरीटां भक्ताभीष्टप्रदां शान्त्यै} %॥१४॥

[१५] संवत्सरः—वृषः, देवता—चामुण्डा\\

\twolineshloka
{चामुण्डां चारुतनूम् अहमिह वन्दे वृषासनारूढाम्}
{वृषनाम्नीं सुरवन्द्यां दैत्यकुल-ध्वंसकारिणीम् अनिशम्} %॥१५॥

[१६] संवत्सरः—चित्रभानुः, देवता—आरोगः\\

\twolineshloka
{आरोगाह्वयमीडेऽहं चित्रभानुं सदा मुदे}
{तुरङ्गमवरारूढं मकुटोज्ज्वल मस्तकम्} %॥१६॥
(तुरङ्गवरमारूढं?)

[१७] संवत्सरः—सुभानुः, देवता—भ्राजः\\

\twolineshloka
{आरोग्य-सिद्धिदं नॄणां स्वभानुं भ्राज-संज्ञितम्}
{गजारूढं लसत्-खड्ग-खेट-हस्तं नमाम्यहम्} %॥१७॥

[१८] संवत्सरः—तारणः, देवता—पटरः\\

\twolineshloka
{अन्तर्हृदब्जे सततं भावये पटराह्वयम्}
{तारणं सूर्यनामानं शूलोद्भासि-कराम्बुजम्} %॥१८॥

[१९] संवत्सरः—पार्थिवः, देवता—पतङ्गः\\

\twolineshloka
{पतङ्गं पदभारूढं पार्थिवाह्वयम् आश्रये}
{शङ्ख-चक्र-धरं दिव्य-पीताम्बर-महोज्ज्वलम्} %॥१९॥

[२०] संवत्सरः—व्ययः, देवता—स्वर्णरः\\

\twolineshloka
{स्वर्णरं कोकिलारूढं व्यय-संज्ञकम् आश्रये}
{चन्द्रावतंसं जटिलं त्रिनेत्रं शूलभासुरम्} %॥२०॥

[२१] संवत्सरः—सर्वजित्, देवता—ज्योतिषीमान्\\

\twolineshloka
{सर्वजित् संज्ञकं वन्दे ज्योतिमन्तमिष्टदम्}
{भुसुण्ठी पाशहस्ताब्जं व्याघ्रारूढं त्रिलोचनम्} %॥२१॥

[२२] संवत्सरः—सर्वधारी, देवता—विभासः\\

\twolineshloka
{विभास-नामकं भानुं सर्वधारि-कृताह्वयम्}
{इष्टप्राप्त्यै सदा वन्दे सप्ताश्व-रथमास्थितम्} %॥२२॥

[२३] संवत्सरः—विरोधी, देवता—कश्यपः\\

\twolineshloka
{मेरु-शृङ्ग-समारूढं विरोधिकृत-नामकम्}
{कश्यपं पञ्चवदनं दशहस्तं नमाम्यहम्} %॥२३॥

[२४] संवत्सरः—विकृतिः, देवता—रविः\\

\twolineshloka
{रविं विकृतनामानं त्रिशीर्षं षड्भुजं भजे}
{पन्नगेश्वरमारूढं चारु-पीताम्बरावृतम्} %॥२४॥

[२५] संवत्सरः—खरः, देवता—सूर्यः\\

\twolineshloka
{सूर्यं खराभिधं वन्दे चतुर्वक्त्राष्ट-हस्तकम्}
{खगवर्यं समारूढं नीलाम्बर-समावृतम्} %॥२५॥

[२६] संवत्सरः—नन्दनः, देवता—भानुः\\

\twolineshloka
{भानुं नन्दन-नामानं हरन्तं चक्षुरोजसा}
{मृगेन्द्र-वाहनं पञ्च-वक्त्रं दशभुजं भजे} %॥२६॥

[२७] संवत्सरः—विजयः, देवता—खगः\\

\twolineshloka
{षण्मुखं द्वादश-भुजं व्याघ्र-वाहनमाश्रितम्}
{व्याघ्र-चर्माम्बरं वन्दे खगं विजय-संज्ञकम्} %॥२७॥

[२८] संवत्सरः—जयः, देवता—पूषा\\

\twolineshloka
{पूषणं जय-नामानं जयदं भक्त-सन्ततेः}
{शङ्ख-चक्राङ्कित-कर-द्वन्द्वं हृदि समाश्रये} %॥२८॥

[२९] संवत्सरः—मन्मथः, देवता—हिरण्यगर्भः\\

\twolineshloka
{हंसवर्याधिरूढं तं वीणामण्डित-हस्तकम्}
{हिरण्यगर्भं वन्देऽहं मन्मथाह्वयम् इष्टदम्} %॥२९॥

[३०] संवत्सरः—दुर्मुखः, देवता—मरीचिः\\

\twolineshloka
{मरीचिं दुर्मुखाख्यानं मन्यु मर्कटमाश्रितम्}
{विवृतास्यं जगद्भीति-दायकं संश्रये मुदे} %॥३०॥

[३१] संवत्सरः—हेमलम्बः, देवता—आदित्यः\\

\twolineshloka
{आदित्यं तेजसां स्थानं हेमलम्ब-कृताह्वयम्}
{सप्त-सप्ति-समारूढं जगतां नेत्रमाश्रये} %॥३१॥

[३२] संवत्सरः—विलम्बः, देवता—सविता\\

\twolineshloka
{विलम्ब-संज्ञं मनसा सवितारं स्मराम्यहम्}
{चाप-बाणधरं दोर्भ्यां तुरङ्गवरवाहनम्} %॥३२॥

[३३] संवत्सरः—विकारी, देवता—अर्कः\\

\twolineshloka
{अर्कं विकारि-नामानं कर्कशं क्रकचायुधम्}
{क्रूर-व्याघ्र-समारूढम् एकनेत्रं नमाम्यहम्} %॥३३॥

[३४] संवत्सरः—शार्वरी, देवता—भास्करः\\

\twolineshloka
{भास्करं शार्वराभिख्यं भासा पूरितदिङ्मुखम्}
{चक्रोज्ज्वलकरं वन्दे भास्वद्रथ-समाश्रयम्} %॥३४॥

[३५] संवत्सरः—प्लवः, देवता—अग्निः\\

\twolineshloka
{अग्निं नमामि सततं ज्वालामालं तमोपहम्}
{अजारूढं चतुर्हस्तं द्विशीर्षं प्लव-संज्ञकम्} %॥३५॥

[३६] संवत्सरः—शुभकृत्, देवता—जातवेदसः\\

\twolineshloka
{जातवेदसमीडेऽहं शुभकृन्नामकं सदा}
{आन्दोलिका-समारूढं खड्ग-खेटक-धारिणम्} %॥३६॥

[३७] संवत्सरः—शोभनः, देवता—सहोजसः\\

\twolineshloka
{सहोजसं शोभकृतं नृणाम् इष्टदमाश्रये}
{शिबिका-वाहनारूढं चामरद्वय-पाणिकम्} %॥३७॥

[३८] संवत्सरः—क्रोधी, देवता—अजिराप्रभुः\\

\twolineshloka
{अजिराप्रभुमीडेऽहं क्रोधीति कृतनामकम्}
{गोरथारूढम् अनिशं कुन्तोज्ज्वल-कराम्बुजम्} %॥३८॥

[३९] संवत्सरः—विश्वावसुः, देवता—वैश्वानरः\\

\twolineshloka
{वैश्वानरं भजे नित्यं विश्वावसु-कृताह्वयम्}
{नर-वाहनमारूढं खड्ग-खेट-धरं विभुम्} %॥३९॥

[४०] संवत्सरः—पराभवः, देवता—नर्यापसः\\

\twolineshloka
{पराभवाह्वयं वन्दे नर्यापसम् उदर्चिषम्}
{उत्तुङ्ग-तुरगारूढं परशूज्ज्वल-पाणिकम्} %॥४०॥

[४१] संवत्सरः—प्लवङ्गः, देवता—पङ्क्तिराधसः\\

\twolineshloka
{पङ्क्तिराधसमीडेऽग्निं प्लवङ्गाख्यं बहुश्रुतम्}
{श्वेताश्व-वाहनारूढं बाण-बाणास-धारिणम्} %॥४१॥

[४२] संवत्सरः—कीलकः, देवता—विसर्पिणः\\

\twolineshloka
{विसर्पिणं हुतवहं कीलकाह्वयम् आश्रये}
{उत्तुङ्ग-गजमारूढं शूलोज्ज्वल-कराञ्चलम्} %॥४२॥

[४३] संवत्सरः—सौम्यः, देवता—मत्स्यमूर्तिः हरिः\\

\twolineshloka
{मत्स्य-मूर्तिं हरिं वन्दे शिरसा सौम्य-संज्ञकम्}
{शङ्ख-चक्रोज्ज्वल-करं भक्ताभीष्ट-प्रदायकम्} %॥४३॥

[४४] संवत्सरः—साधारणः, देवता—कूर्ममूर्तिः हरिः\\

\twolineshloka
{कूर्माकृतिं हरिं वन्दे मन्दराचलधारिणम्}
{कौमोदकी-शार्ङ्ग-हस्तं साधारण-कृताह्वयम्} %॥४४॥

[४५] संवत्सरः—विरोधिकृत्, देवता—आदिवराहः\\

\twolineshloka
{दंष्ट्रारूढ-महीभाजम् आदिक्रोडमभीष्टदम्}
{वराभयकरं वन्दे विरोधिकृदिति श्रुतम्} %॥४५॥

[४६] संवत्सरः—परितापी, देवता—नरसिंहः\\

\twolineshloka
{सकेसरिणं वन्दे हरिं स्तम्भ-विदारिणम्}
{प्रह्लाद-हर्ष-दातारं परिधाविकृताह्वयम्} %॥४६॥
(परितापिकृताह्वयम्?)

[४७] संवत्सरः—प्रमादी, देवता—वामनः\\

\twolineshloka
{कमण्डलु-छत्त्रधरं हरिं वामन-रूपिणम्}
{बलि-प्प्रतारकं वन्दे प्रमादीति कृताह्वयम्} %॥४७॥

[४८] संवत्सरः—आनन्दः, देवता—परशुरामः\\

\twolineshloka
{कुठार-प्रोज्ज्वलकरं जटामण्डल-मण्डितम्}
{भार्गवं राममीडेऽहम् आनन्दाह्वयमद्भुतम्} %॥४८॥

[४९] संवत्सरः—राक्षसः, देवता—श्रीरामः\\

\twolineshloka
{जगदानन्दजनकं सीता-वल्ल्लभमाश्रये}
{रामं राक्षस-नामानं धृतकोदण्डमार्गणम्} %॥४९॥

[५०] संवत्सरः—नलः, देवता—बलरामः\\

\twolineshloka
{सीरोद्भासि कराम्भोजं नीलाम्बर-समावृतम्}
{बलरामं मुसलिनं नलाह्वयमहं भजे} %॥५०॥

[५१] संवत्सरः—पिङ्गलः, देवता—श्रीकृष्णः\\

\twolineshloka
{कृष्णं-पिङ्गल-नामानं बर्हिबर्होज्ज्वलाङ्गकम्}
{नवनीतोज्ज्वलकरं हृदि पीताम्बरं भजे} %॥५१॥

[५२] संवत्सरः—कालयुक्तिः, देवता—कल्किः\\

\twolineshloka
{म्लेच्छावलि-कृतद्वेषं कालयुक्त-कृताह्वयम्}
{आश्रयेऽश्व-समारूढं धनुर्बाण-धरं सदा} %॥५२॥
(कालयुक्ति-कृताह्वयम्?)

[५३] संवत्सरः—सिद्धार्थी, देवता—बुद्धः\\

\twolineshloka
{सिद्धार्थकृत-नामानं बुद्धाकृतिम् अहं भजे}
{नग्नं सुरूप-वपुष स्त्री-प्रधर्षण-कारणम्} %॥५३॥
(सिद्धार्थि-कृत?)

[५४] संवत्सरः—रौद्रः, देवता—दुर्गा\\

\twolineshloka
{दुर्गां विपद्विधुतये रौद्रनाम्नीं समाश्रये}
{सर्व-दैवत्य-निहन्त्रीं तां रक्तास्वादन-तत्पराम्} %॥५४॥

[५५] संवत्सरः—दुर्मतिः, देवता—यातुधानः\\

\twolineshloka
{यातुधानं वक्रमुखम् एकाक्षं दारुणाकृतिम्}
{वन्दे दुर्मति-नामानं सर्वारिष्ट-निवृत्तये} %॥५५॥

[५६] संवत्सरः—दुन्दुभिः, देवता—भैरवः\\

\twolineshloka
{देवं दुन्दुभि-नामानं समस्त-रिपु-भैरवम्}
{भैरवाश्व-वरारूढं दिगम्बरमुपाश्रये} %॥५६॥

[५७] संवत्सरः—रुधिरोद्गारी, देवता—हनुमान्\\

\twolineshloka
{हनूमन्तं वायुसुतं रुधिरोगारि-संज्ञकम्}
{समुद्र-लङ्घन-पटुं सर्व-रक्षोहरं भजे} %॥५७॥

[५८] संवत्सरः—रक्ताक्षः, देवता—सरस्वती\\

\twolineshloka
{सरस्वती भासमान-वीणोज्ज्वल-कराम्बुजाम्}
{रक्ताक्ष-संज्ञिकां देवीं प्रपद्ये वाग्विभूतिदाम्} %॥५८॥

[५९] संवत्सरः—क्रोधनः, देवता—दाक्षायणी\\

\twolineshloka
{दाक्षायणीं क्रोधनाख्या दक्ष-क्रतु-हर-प्रियाम्}
{शुकारूढां शूलधरीं भजेऽभीष्टस्य सिद्धये} %॥५९॥

[६०] संवत्सरः—क्षयः, देवता—लक्ष्मीः\\

\twolineshloka
{क्षय-नाम्नीं प्रपद्येऽहं लक्ष्मीं क्षीराब्धि-कन्यकाम्}
{हंसारूढाम् इन्दुमुखीं पाणिद्वय-धृता-म्बुजाम्} %॥६०॥
