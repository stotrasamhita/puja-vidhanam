% !TeX program = XeLaTeX
% !TeX root = ../pujavidhanam.tex
\chapt{मास-नामानि}
\label{app:masa_names}

\sect{चान्द्र-मासाः}

\addtocounter{shlokacount}{27}
\twolineshloka
{मासश्चैत्रोऽथ वैशाखो ज्येष्ठ आषाढसंज्ञकः}
{ततस्तु श्रावणो भाद्रपदाऽऽश्विनसंज्ञकः}

\onelineshloka
{कार्त्तिको मार्गशीर्षश्च पौषो माघोऽथ फाल्गुनः} ॥ २९ ॥
—मुहूर्तगणपतिः

\begin{longtable}{lllll}
  ऋतुमासाः & सौरमासाः & ऋतुः    & चान्द्रमासाः & \textsf{\normalsize Gregorian} \\
  \multicolumn{5}{c}{\small (सौरमासेन ऋतुमासस्य चान्द्रमासस्य च अन्तयोगः, आधुनिकस्य आदियोगः)}\\\hline\endhead
  मधुः     & मेषः      & \multirow{2}{*}{वसन्तः}  & चैत्रः       & \textsf{\normalsize April}    \\*
  माधवः   & वृषभः     &   & वैशाखः      & \textsf{\normalsize May}       \\\hline
  शुक्रः    & मिथुनम्    & \multirow{2}{*}{ग्रीष्मः} & ज्यैष्ठः      & \textsf{\normalsize June}    \\*
  शुचिः    & कटकः     &  & आषाढः      & \textsf{\normalsize July}      \\\hline
  नभाः    & सिंहः     & \multirow{2}{*}{वर्षाः}  & श्रावणः     & \textsf{\normalsize August}     \\*
  नभस्यः   & कन्या     &   & भाद्रपदः    & \textsf{\normalsize September} \\\hline
  इषः     & तुला      & \multirow{2}{*}{शरत्}    & आश्वयुजः     & \textsf{\normalsize October}\\*
  ऊर्जः    & वृश्चिकः   &     & कार्त्तिकः   & \textsf{\normalsize November}  \\\hline
  सहाः    & धनुः      & \multirow{2}{*}{हेमन्तः}  & मार्गशीर्षः  & \textsf{\normalsize December}  \\*
  सहस्यः   & मकरः     &   & पौषः       & \textsf{\normalsize January}   \\\hline
  तपाः    & कुम्भः     & \multirow{2}{*}{शिशिरः} & माघः       & \textsf{\normalsize February}  \\*
  तपस्यः   & मीनः     &  & फाल्गुनः     & \textsf{\normalsize March}     \\\hline
\end{longtable}
