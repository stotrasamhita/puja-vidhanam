\sect{सिद्धिविनायक-चतुर्थी-व्रत-कथा}
\centerline{\small{(मूलम्—श्री-व्रतराजः)}}

\twolineshloka
{शौनकाद्या ऋषिगणा नैमिषारण्यवासिनः}
{सूतं पौराणिकं श्रेष्ठमिदमूचुर्वचस्तदा}%॥१॥

\uvacha{ऋषय ऊचुः}

\twolineshloka
{निर्विघ्ने तु कार्याणि कथं सिध्यन्ति सूतज}
{अर्थसिद्धिः कथं नॄणां पुत्रसौभाग्यसम्पदः}%॥२॥

\twolineshloka
{दम्पत्योः कलहे चैव बन्धुभेदे तथा नृणाम्}
{उदासीनेषु लोकेषु कथं सुमुखता भवेत्}%॥३॥

\twolineshloka
{विद्यारम्भे तथा नॄणां वाणिज्ये च कृषौ तथा}
{नृपतेः परचक्रं च जयसिद्धिः कथं भवेत्}%॥४॥

\twolineshloka
{कां देवतां नमस्कृत्य कार्यसिद्धिर्भवेन्नृणाम्}
{एतत् समस्तं विस्तार्य ब्रूहि मे सूत पृच्छतः}%॥५॥

\uvacha{सूत उवाच}
\twolineshloka
{सन्नद्धयोः पुरा विप्राः कुरुपाण्डवसेनयोः}
{पृष्टवान् देवकीपुत्रं कुन्तीपुत्रो युधिष्ठिरः}%॥६॥ 

\uvacha{युधिष्ठिर उवाच}

\twolineshloka
{निर्विघ्नेन जयं मह्यं वद त्वां देवकीसुत}
{कां देवतां नमस्कृत्य सम्यग्राज्यं लभेमहि}%॥७॥

\uvacha{कृष्ण उवाच}

\twolineshloka
{पूजयस्व गणाध्यक्षमुमा-मल-समुद्भवम्}
{तस्मिन् सम्पूजिते देवे ध्रुवं राज्यमवाप्स्यसि}%॥८॥

\uvacha{युधिष्ठिर उवाच}

\twolineshloka
{देव केन विधानेन पूजनीयो गणाधिपः}
{पूजितस्तु तिथौ कस्यां सिद्धिदो गणपो भवेत्}%॥९॥

\uvacha{कृष्ण उवाच}

\twolineshloka
{मासि भाद्रपदे शुक्ले चतुर्थ्यां पूजयेन्नृप}
{मासि माघे श्रावणे वा मार्गशीर्षेऽथवा भवेत्}%॥१०॥

\twolineshloka
{गजवक्त्रं तु शुक्लायां चतुर्थ्यां पूजयेन्नृप}
{यदा चोत्पद्यते भक्तिस्तदा पूज्यो गणाधिपः}%॥११॥

\twolineshloka
{प्रातः शुक्लतिलैः स्नात्वा मध्याह्ने पूजयेन्नृप}
{निष्कमात्रसुवर्णेन तदर्धार्धेन वा पुनः}%॥१२॥

\twolineshloka
{स्वशक्त्या गणनाथस्य स्वर्णरौप्यमयाकृतिम्}
{अथवा मृन्मयीं कुर्याद् वित्तशाठ्यं न कारयेत्}%॥१३॥

\twolineshloka
{एकदन्तं शूर्पकर्णं गजवक्त्रं चतुर्भुजम्}
{पाशाङ्कुशधरं देवं ध्यायेत् सिद्धिविनायकम्}%॥१४॥

\twolineshloka
{ध्यात्वा चानेन मन्त्रेण स्नाप्य पञ्चामृतैः पृथक्}
{गणाध्यक्षेति नाम्ना वै गन्धं दद्याच्च भक्तितः}%॥१५॥

\twolineshloka
{आवाहनार्थे पाद्यं च दत्त्वा पश्चात् प्रयत्नतः}
{रक्तवस्त्रयुगं सर्वप्रदं दद्याच्च भक्तितः}%॥१६॥

\twolineshloka
{विनायकेति पुष्पाणि धूपं चोमासुताय च}
{दीपं रुद्रप्रियायेति नैवेद्यं विघ्ननाशिने}%॥१७॥

\twolineshloka
{किञ्चित् सुवर्णं पूजां च ताम्बूलं च समर्पयेत्}
{ततो दूर्वाङ्कुरान् गृह्य विंशतिं चैकमेव हि}%॥१८॥

\twolineshloka
{पूजनीयः प्रयत्नेन एभिर्नामपदैः पृथक्}
{गणाधिप नमस्तेऽस्तु उमापुत्राघनाशन}%॥१९॥

\twolineshloka
{विनायकेशपुत्रेति सर्वसिद्धिप्रदायक}
{एकदन्तेभवक्रेति तथा मूषकवाहन}%॥२०॥

\twolineshloka
{कुमारगुरवे तुभ्यं पूजनीयः प्रयत्नतः}
{दूर्वायुग्मं गृहीत्वा तु गन्धपुष्पाक्षतैर्युतम्}%॥२१॥

\twolineshloka
{एकैकेन तु नाम्ना वै दत्त्वैकं सर्वनामभिः}
{अथैकविंशतिं गृह्य मोदकान् घृतपाचितान्}%॥२२॥

\twolineshloka
{स्थापयित्वा गणाध्यक्षसमीपे कुरुनन्दन}
{दश विप्राय दातव्याः स्वयं ग्राह्यास्तथा दश}%॥२३॥

\twolineshloka
{एकं गणाधिपे दद्यात् सनैवेद्यं नृपोत्तम}
{विनायकस्य प्रतिमां ब्राह्मणाय निवेदयेत्}%॥२४॥

\twolineshloka
{विनायकस्य प्रतिमां वस्त्रयुग्मेन वेष्टिताम्}
{तुभ्यं सम्प्रददे विप्र प्रीयतां मे गजाननः}%॥२५॥

\twolineshloka
{विनायक गणेश त्वं सर्वदेवनमस्कृत}
{पार्वतीप्रिय विघ्नेश मम विघ्नं विनाशय}%॥२६॥

\twolineshloka
{गणेशः प्रतिगृह्णाति गणेशो वै ददाति च}
{गणेशस्तारकोभाभ्यां गणेशाय नमो नमः}%॥२७॥

\twolineshloka
{कृत्वा नैमित्तिकं कर्म पूजयेदिष्टदेवताम्}
{ब्राह्मणान् भोजयेत् पश्चाद् भुञ्जीयात् तैलवर्जितम्}%॥२८॥

\twolineshloka
{एवं कृते धर्मराज गणनाथस्य पूजने}
{विजयस्ते भवेन्नूनं सत्यं सत्यं मयोदितम्}%॥२९॥

\twolineshloka
{त्रिपुरं हन्तुकामेन पूजितः शूलपाणिना}
{शक्रेण पूजितः पूर्वं वृत्रासुरवधेच्छया}%॥३०॥

\twolineshloka
{अन्वेषयन्त्या भर्तारं पूजितोऽहल्यया पुरा}
{नलस्यान्वेषणार्थाय दमयन्त्या पुराऽऽर्चितः}%॥३१॥

\twolineshloka
{रघुनाथेन तद्वच्च सीतायान्वेषणे पुरा}
{द्रष्टुं सीतां महाभागां वीरेण च हनूमता}%॥३२॥

\twolineshloka
{भगीरथेन तद्वच्च गङ्गामानयता पुरा}
{अमृतोत्पादनार्थाय तथा देवासुरैरपि}%॥३३॥

\twolineshloka
{अमृतं हरता पूर्वं वैनतेयेन पक्षिणा}
{आराधितो गणाध्यक्षो ह्यमृतं च हृतं बलात्}%॥३४॥

\twolineshloka
{रुक्मिणीहेतुकामेन पूजितोऽसौ मया प्रभुः}
{तस्य प्रसादाद्राजेन्द्र रुक्मिणीं प्राप्तवानहम्}%॥३५॥

\twolineshloka
{यदा पूर्वं हि दैत्येन हृतो रुक्मिणिनन्दनः}
{आराधितो मया तद्वद् रुक्मिण्या सहितेन च}%॥३६॥

\twolineshloka
{कुष्ठव्याधियुतेनाथ साम्बेनाऽऽराधितः पुरा}
{जयकामस्तथा शीघ्रं त्वमाराधय शाङ्करिम्}%॥३७॥

\twolineshloka
{विद्याकामो लभेद् विद्यां धनकामो धनं तथा}
{जयं च जयकामस्तु पुत्रार्थी विन्दते सुतान्}%॥३८॥

\twolineshloka
{पतिकामा च भर्तारं सौभाग्यं च सुवासिनी}
{विधवा पूजयित्वा तु वैधव्यं नाप्नुयात्क्वचित्}%॥३९॥

\twolineshloka
{वैष्णव्याद्यासु दीक्षासु आदौ पूज्यो गणाधिपः}
{तस्मिन् सम्पूजिते विष्णुरीशो भानुस्तथा ह्युमा}%॥४०॥ 

\twolineshloka
{हव्यवाहमुखा देवाः पूजिताः स्युर्न संशयः}
{चण्डिकाद्या मातृगणाः परितुष्टा भवन्ति च}%॥४१॥

\twolineshloka
{तस्मिन्सम्पूजिते विप्रा भक्त्या सिद्धिविनायके}
{एवं कृते धर्मराज गणनाथस्य पूजने}%॥४२॥

\twolineshloka
{प्राप्स्यसि त्वं स्वकं राज्यं हत्वा शत्रून् रणाजिरे}
{सिध्यन्ति सर्वकार्याणि नात्र कार्या विचारणा}%॥४३॥

\twolineshloka
{एवमुक्तस्तु कृष्णेन सानुजः पाण्डुनन्दनः}
{पूजयामास देवस्य पुत्रं त्रिपुरघातिनः}%॥४४॥

\onelineshloka*
{शत्रुसङ्घं निहत्यासौ प्राप्तवान्राज्यमोजसा}

\uvacha{सूत उवाच}
\onelineshloka
{यः पूजयेन्मन्दभाग्यो गणेशं सिद्धिदायकम्}%॥४५॥

\twolineshloka
{सिध्यन्ति तस्य कार्याणि मनसा चिन्तितान्यपि}
{ख्यातिं गमिष्यते तेन नाम्ना सिद्धिविनायकः}%॥४६॥

\twolineshloka
{य इदं शृणुयान्नित्यं श्रावयेद् वा समाहितः}
{सिध्यन्ति सर्वकार्याणि विनायकप्रसादतः}%॥४७॥  

॥इति सिद्धिविनायकव्रतं भविष्योक्तं सम्पूर्णम्॥
