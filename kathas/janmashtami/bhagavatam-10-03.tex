\sect{श्रीमद्भागवते महापुराणे दशमस्कन्धे पूर्वार्धे तृतीयोऽध्यायः}

\uvacha{श्रीशुक उवाच}

\twolineshloka
{अथ सर्वगुणोपेतः कालः परमशोभनः}
{यर्ह्येवाजनजन्मर्क्षं शान्तर्क्षग्रहतारकम्} %1

\twolineshloka
{दिशः प्रसेदुर्गगनं निर्मलोडुगणोदयम्}
{मही मङ्गलभूयिष्ठ पुरग्रामव्रजाकरा} %2

\twolineshloka
{नद्यः प्रसन्नसलिला ह्रदा जलरुहश्रियः}
{द्विजालिकुलसन्नाद स्तवका वनराजयः} %3

\twolineshloka
{ववौ वायुः सुखस्पर्शः पुण्यगन्धवहः शुचिः}
{अग्नयश्च द्विजातीनां शान्तास्तत्र समिन्धत} %4

\twolineshloka
{मनांस्यासन्प्रसन्नानि साधूनामसुरद्रुहाम्}
{जायमानेऽजने तस्मिन्नेदुर्दुन्दुभयः समम्} %5

\twolineshloka
{जगुः किन्नरगन्धर्वास्तुष्टुवुः सिद्धचारणाः}
{विद्याधर्यश्च ननृतुरप्सरोभिः समं मुदा} %6

\twolineshloka
{मुमुचुर्मुनयो देवाः सुमनांसि मुदान्विताः}
{मन्दं मन्दं जलधरा जगर्जुरनुसागरम्} %7

\threelineshloka
{निशीथे तम उद्भूते जायमाने जनार्दने}
{देवक्यां देवरूपिण्यां विष्णुः सर्वगुहाशयः}
{आविरासीद्यथा प्राच्यां दिशीन्दुरिव पुष्कलः}

\twolineshloka
{तमद्भुतं बालकमम्बुजेक्षणं चतुर्भुजं शङ्खगदाद्युदायुधम्}
{श्रीवत्सलक्ष्मं गलशोभिकौस्तुभं पीताम्बरं सान्द्रपयोदसौभगम्} %9

\twolineshloka
{महार्हवैदूर्यकिरीटकुण्डल त्विषा परिष्वक्तसहस्रकुन्तलम्}
{उद्दामकाञ्च्यङ्गदकङ्कणादिभिर्विरोचमानं वसुदेव ऐक्षत} %10

\twolineshloka
{स विस्मयोत्फुल्लविलोचनो हरिं सुतं विलोक्यानकदुन्दुभिस्तदा}
{कृष्णावतारोत्सवसम्भ्रमोऽस्पृशन्मुदा द्विजेभ्योऽयुतमाप्लुतो गवाम्} %11

\twolineshloka
{अथैनमस्तौदवधार्य पूरुषं परं नताङ्गः कृतधीः कृताञ्जलिः}
{स्वरोचिषा भारत सूतिकागृहं विरोचयन्तं गतभीः प्रभाववित्} %12

\uvacha{श्रीवसुदेव उवाच}


\twolineshloka
{विदितोऽसि भवान्साक्षात्पुरुषः प्रकृतेः परः}
{केवलानुभवानन्द स्वरूपः सर्वबुद्धिदृक्} %13

\twolineshloka
{स एव स्वप्रकृत्येदं सृष्ट्वाग्रे त्रिगुणात्मकम्}
{तदनु त्वं ह्यप्रविष्टः प्रविष्ट इव भाव्यसे} %14

\twolineshloka
{यथेमेऽविकृता भावास्तथा ते विकृतैः सह}
{नानावीर्याः पृथग्भूता विराजं जनयन्ति हि} %15

\twolineshloka
{सन्निपत्य समुत्पाद्य दृश्यन्तेऽनुगता इव}
{प्रागेव विद्यमानत्वान्न तेषामिह सम्भवः} %16

\twolineshloka
{एवं भवान्बुद्ध्यनुमेयलक्षणैर्ग्राह्यैर्गुणैः सन्नपि तद्गुणाग्रहः}
{अनावृतत्वाद्बहिरन्तरं न ते सर्वस्य सर्वात्मन आत्मवस्तुनः} %17

\twolineshloka
{य आत्मनो दृश्यगुणेषु सन्निति व्यवस्यते स्वव्यतिरेकतोऽबुधः}
{विनानुवादं न च तन्मनीषितं सम्यग्यतस्त्यक्तमुपाददत्पुमान्} %18

\fourlineindentedshloka
{त्वत्तोऽस्य जन्मस्थितिसंयमान्विभो}
{वदन्त्यनीहादगुणादविक्रियात्}
{त्वयीश्वरे ब्रह्मणि नो विरुध्यते}
{त्वदाश्रयत्वादुपचर्यते गुणैः} %19

\fourlineindentedshloka
{स त्वं त्रिलोकस्थितये स्वमायया}
{बिभर्षि शुक्लं खलु वर्णमात्मनः}
{सर्गाय रक्तं रजसोपबृंहितं}
{कृष्णं च वर्णं तमसा जनात्यये} %20

\twolineshloka
{त्वमस्य लोकस्य विभो रिरक्षिषुर्गृहेऽवतीर्णोऽसि ममाखिलेश्वर}
{राजन्यसंज्ञासुरकोटियूथपैर्निर्व्यूह्यमाना निहनिष्यसे चमूः} %21

\fourlineindentedshloka
{अयं त्वसभ्यस्तव जन्म नौ गृहे}
{श्रुत्वाग्रजांस्ते न्यवधीत्सुरेश्वर}
{स तेऽवतारं पुरुषैः समर्पितं}
{श्रुत्वाधुनैवाभिसरत्युदायुधः} %22

\uvacha{श्रीशुक उवाच}


\twolineshloka
{अथैनमात्मजं वीक्ष्य महापुरुषलक्षणम्}
{देवकी तमुपाधावत्कंसाद्भीता सुविस्मिता} %23

\uvacha{श्रीदेवक्युवाच}


\fourlineindentedshloka
{रूपं यत्तत्प्राहुरव्यक्तमाद्यं}
{ब्रह्म ज्योतिर्निर्गुणं निर्विकारम्}
{सत्तामात्रं निर्विशेषं निरीहं}
{स त्वं साक्षाद्विष्णुरध्यात्मदीपः} %24

\twolineshloka
{नष्टे लोके द्विपरार्धावसाने महाभूतेष्वादिभूतं गतेषु}
{व्यक्तेऽव्यक्तं कालवेगेन याते भवानेकः शिष्यतेऽशेषसंज्ञः} %25

\fourlineindentedshloka
{योऽयं कालस्तस्य तेऽव्यक्तबन्धो}
{चेष्टामाहुश्चेष्टते येन विश्वम्}
{निमेषादिर्वत्सरान्तो महीयांस्}
{तं त्वेशानं क्षेमधाम प्रपद्ये} %26

\twolineshloka
{मर्त्यो मृत्युव्यालभीतः पलायन्लोकान्सर्वान्निर्भयं नाध्यगच्छत्}
{त्वत्पादाब्जं प्राप्य यदृच्छयाद्य सुस्थः शेते मृत्युरस्मादपैति} %27

\twolineshloka
{स त्वं घोरादुग्रसेनात्मजान्नस्त्राहि त्रस्तान्भृत्यवित्रासहासि}
{रूपं चेदं पौरुषं ध्यानधिष्ण्यं मा प्रत्यक्षं मांसदृशां कृषीष्ठाः} %28

\twolineshloka
{जन्म ते मय्यसौ पापो मा विद्यान्मधुसूदन}
{समुद्विजे भवद्धेतोः कंसादहमधीरधीः} %29

\twolineshloka
{उपसंहर विश्वात्मन्नदो रूपमलौकिकम्}
{शङ्खचक्रगदापद्म श्रिया जुष्टं चतुर्भुजम्} %30

\twolineshloka
{विश्वं यदेतत्स्वतनौ निशान्ते यथावकाशं पुरुषः परो भवान्}
{बिभर्ति सोऽयं मम गर्भगोऽभूदहो नृलोकस्य विडम्बनं हि तत्} %31

\uvacha{श्रीभगवानुवाच}


\twolineshloka
{त्वमेव पूर्वसर्गेऽभूः पृश्निः स्वायम्भुवे सति}
{तदायं सुतपा नाम प्रजापतिरकल्मषः} %32

\twolineshloka
{युवां वै ब्रह्मणादिष्टौ प्रजासर्गे यदा ततः}
{सन्नियम्येन्द्रियग्रामं तेपाथे परमं तपः} %33

\twolineshloka
{वर्षवातातपहिम घर्मकालगुणाननु}
{सहमानौ श्वासरोध विनिर्धूतमनोमलौ} %34

\twolineshloka
{शीर्णपर्णानिलाहारावुपशान्तेन चेतसा}
{मत्तः कामानभीप्सन्तौ मदाराधनमीहतुः} %35

\twolineshloka
{एवं वां तप्यतोस्तीव्रं तपः परमदुष्करम्}
{दिव्यवर्षसहस्राणि द्वादशेयुर्मदात्मनोः} %36

\twolineshloka
{तदा वां परितुष्टोऽहममुना वपुषानघे}
{तपसा श्रद्धया नित्यं भक्त्या च हृदि भावितः} %37

\twolineshloka
{प्रादुरासं वरदराड्युवयोः कामदित्सया}
{व्रियतां वर इत्युक्ते मादृशो वां वृतः सुतः} %38

\twolineshloka
{अजुष्टग्राम्यविषयावनपत्यौ च दम्पती}
{न वव्राथेऽपवर्गं मे मोहितौ देवमायया} %39

\twolineshloka
{गते मयि युवां लब्ध्वा वरं मत्सदृशं सुतम्}
{ग्राम्यान्भोगानभुञ्जाथां युवां प्राप्तमनोरथौ} %40

\twolineshloka
{अदृष्ट्वान्यतमं लोके शीलौदार्यगुणैः समम्}
{अहं सुतो वामभवं पृश्निगर्भ इति श्रुतः} %41

\twolineshloka
{तयोर्वां पुनरेवाहमदित्यामास कश्यपात्}
{उपेन्द्र इति विख्यातो वामनत्वाच्च वामनः} %42

\twolineshloka
{तृतीयेऽस्मिन्भवेऽहं वै तेनैव वपुषाथ वाम्}
{जातो भूयस्तयोरेव सत्यं मे व्याहृतं सति} %43

\twolineshloka
{एतद्वां दर्शितं रूपं प्राग्जन्मस्मरणाय मे}
{नान्यथा मद्भवं ज्ञानं मर्त्यलिङ्गेन जायते} %44

\twolineshloka
{युवां मां पुत्रभावेन ब्रह्मभावेन चासकृत्}
{चिन्तयन्तौ कृतस्नेहौ यास्येथे मद्गतिं पराम्} %45

\uvacha{श्रीशुक उवाच}


\twolineshloka
{इत्युक्त्वासीद्धरिस्तूष्णीं भगवानात्ममायया}
{पित्रोः सम्पश्यतोः सद्यो बभूव प्राकृतः शिशुः} %46

\fourlineindentedshloka
{ततश्च शौरिर्भगवत्प्रचोदितः}
{सुतं समादाय स सूतिकागृहात्}
{यदा बहिर्गन्तुमियेष तर्ह्यजा}
{या योगमायाजनि नन्दजायया} %47

\fourlineindentedshloka
{तया हृतप्रत्ययसर्ववृत्तिषु}{द्वाःस्थेषु पौरेष्वपि शायितेष्वथ}
{द्वारश्च सर्वाः पिहिता दुरत्यया}{बृहत्कपाटायसकीलशृङ्खलैः} %48

\fourlineindentedshloka
{ताः कृष्णवाहे वसुदेव आगते}{स्वयं व्यवर्यन्त यथा तमो रवेः}
{ववर्ष पर्जन्य उपांशुगर्जितः}{शेषोऽन्वगाद्वारि निवारयन्फणैः} %49

\fourlineindentedshloka
{मघोनि वर्षत्यसकृद्यमानुजा}{गम्भीरतोयौघजवोर्मिफेनिला}
{भयानकावर्तशताकुला नदी}{मार्गं ददौ सिन्धुरिव श्रियः पतेः} %50

\fourlineindentedshloka
{नन्दव्रजं शौरिरुपेत्य तत्र तान्}
{गोपान्प्रसुप्तानुपलभ्य निद्रया}
{सुतं यशोदाशयने निधाय तत्}
{सुतामुपादाय पुनर्गृहानगात्} %51

\twolineshloka
{देवक्याः शयने न्यस्य वसुदेवोऽथ दारिकाम्}
{प्रतिमुच्य पदोर्लोहमास्ते पूर्ववदावृतः} %52

\twolineshloka
{यशोदा नन्दपत्नी च जातं परमबुध्यत}
{न तल्लिङ्गं परिश्रान्ता निद्रयापगतस्मृतिः}


॥इति श्रीमद्भागवते महापुराणे पारमहंस्यां संहितायां दशमस्कन्धे पूर्वार्धे तृतीयोऽध्यायः॥

