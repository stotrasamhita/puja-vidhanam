\sect{हरिवंशे पञ्चत्रिंशोऽध्यायः --- कृष्णजन्मानुकीर्तनम्}

\uvacha{वैशम्पायन उवाच}


\twolineshloka
{याः पत्न्यो वसुदेवस्य चतुर्दश वराङ्गनाः}
{पौरवी रोहिणी नाम इन्दिरा च तथा वरा} %॥१॥

\twolineshloka
{वैशाखी च तथा भद्रा सुनाम्नी चैव पञ्चमी}
{सहदेवा शान्तिदेवा श्रीदेवा देवरक्षिता} %॥२॥

\twolineshloka
{वृकदेव्युपदेवी च देवकी चैव सप्तमी}
{सुतनुर्वडवा चैव द्वे एते परिचारिके} %॥३॥

\twolineshloka
{पौरवी रोहिणी नाम बाह्लिकस्याऽऽत्मजाऽभवत्}
{ज्येष्ठा पत्नी महाराज दयिताऽऽनकदुन्दुभेः} %॥४॥

\twolineshloka
{लेभे ज्येष्ठं सुतं रामं सारणं शठमेव च}
{दुर्दमं दमनं श्वभ्रं पिण्डारकमुशीनरम्} %॥५॥

\twolineshloka
{चित्रां नाम कुमारीं च रोहिणीतनया दश}
{चित्रा सुभद्रेति पुनर्विख्याता कुरुनन्दन} %॥६॥

\twolineshloka
{वसुदेवाच्च देवक्यां जज्ञे शौरिर्महायशाः}
{रामाच्च निशठो जज्ञे रेवत्यां दयितः सुतः} %॥७॥

\twolineshloka
{सुभद्रायां रथी पार्थादभिमन्युरजायत}
{अक्रूरात् काशिकन्यायां सत्यकेतुरजायत} %॥॥

\twolineshloka
{वसुदेवस्य भार्यासु महाभागासु सप्तसु}
{ये पुत्रा जज्ञिरे शूरा नामतस्तान् निबोध मे} %॥९॥

\twolineshloka
{भोजश्च विजयश्चैव शान्तिदेवासुतावुभौ}
{वृकदेवः सुनामायां गदश्चास्ता सुतावुभौ} %॥1.35.१०॥

\twolineshloka
{उपासङ्गवरं लेभे तनयं देवरक्षिता}
{अगावहं महात्मानं वृकदेवी व्यजायत} %॥११॥

\twolineshloka
{कन्या त्रिगर्तराजस्य भर्ता वै शैशिरायणः}
{जिज्ञासां पौरुषे चक्रे न चस्कन्देऽथ पौरुषम्} %॥१२॥

\twolineshloka
{कृष्णायससमप्रख्यो वर्षे द्वादशमे तथा}
{मिथ्याभिशप्तो गार्ग्यस्तु मन्युनाभिसमीरितः} %॥१३॥

\twolineshloka
{गोपकन्यामुपादाय मैथुनायोपचक्रमे}
{गोपाली त्वप्सरास्तस्य गोपस्त्रीवेषधारिणी} %॥१४॥

\twolineshloka
{धारयामास गार्ग्यस्य गर्भं दुर्धरमच्युतम्}
{मानुष्यां गार्ग्यभार्यायां नियोगाच्छूलपाणिनः} %॥१५॥

\twolineshloka
{स कालयवनो नाम जज्ञे राजा महाबलः}
{वृषपूर्वार्धकायास्तमवहन् वाजिनो रणे} %॥१६॥

\twolineshloka
{अपुत्रस्य स राज्ञस्तु ववृधेऽन्तःपुरे शिशुः}
{यवनस्य महाराज स कालयवनोऽभवत्} %॥१७॥॥

\twolineshloka
{स युद्धकामी नृपतिः पर्यपृच्छद् द्विजोत्तमान्।}
{वृष्ण्यन्धककुलं तस्य नारदोऽकथयद् विभुः} %॥१८॥

\twolineshloka
{अक्षौहिण्या तु सैन्यस्य मथुरामभ्ययात्तदा}
{दूतं सम्प्रेषयामास वृष्ण्यन्धकनिवेशनम्} %॥१९॥

\twolineshloka
{ततो वृष्ण्यन्धकाः कृष्णं पुरस्कृत्य महामतिम्।}
{समेता मन्त्रयामासुर्यवनस्य भयात् तदा} %॥२०॥।

\twolineshloka
{कृत्वा च निश्चयं सर्वे पलायनपरायणाः}
{विहाय मथुरां रम्यां मानयन्तः पिनाकिनम्} %॥२१॥

\threelineshloka
{कुशस्थलीं द्वारवतीं निवेशयितुमीप्सवः}
{इति कृष्णस्य जन्मेदं यः शुचिर्नियतेन्द्रियः}
{पर्वसु श्रावयेद् विद्वाननृणः स सुखी भवेत्} %॥२२॥

॥इति श्रीमहाभारते खिलभागे हरिवंशे हरिवंशपर्वणि श्रीकृष्णजन्मानुकीर्तनं नाम पञ्चत्रिंशोऽध्यायः॥३५॥