\sect{कथा {[भविष्योत्तरपुराणम्]}}

\uvacha{सूत उवाच}

\twolineshloka
{कैलासशिखरे रम्ये नानागणनिषेविते}
{मन्दारपीठे विद्रान्ते नानामणिविभूषिते} %॥१॥

\twolineshloka
{रत्नपीठे सुखासीनं शङ्करं लोकशङ्करम्}
{पप्रच्छ गौरी सन्तुष्टा परानुग्रहकाम्यया} %॥२॥

\uvacha{पार्वत्युवाच}

\twolineshloka
{भगवन् सर्वलोकेश सर्वभूतहिते रत}
{यद्रहस्यमिदं पुण्यं तदाचक्ष्व ममानघ} %॥३॥

\onelineshloka*
{वरलक्ष्मीव्रतं चास्ति तन्मे ब्रूहि जगत्प्रभो}

\uvacha{शङ्कर उवाच}

\onelineshloka
{व्रतानामुत्तमं नाम सर्वसौभाग्यदायकम्} %॥४॥

\threelineshloka
{सर्वसम्पत्करं शीघ्रं पुत्रपौत्रप्रवर्धनम्}
{शुक्ले श्रावणके मासे पौर्णिमास्यां तु भार्गवे} %॥५॥
{तदारभ्य व्रतं नार्या महालक्ष्मीं च पूजयेत्}

\uvacha{पार्वत्युवाच}

\twolineshloka
{विधिना केन कर्तव्यं तत्र का नाम देवता}
{कथमाराधिता पूर्वं साभूत् सन्तुमनेनसा} %॥६॥

\uvacha{ईश्वर उवाच}

\twolineshloka
{वरलक्ष्मीव्रतं पुण्यं वक्ष्यामि शृणु पार्वति}
{कथं त्वं च चकोरोक्षितदधीना भविष्यसि} %॥७॥

\twolineshloka
{कौण्डण्य नाम नगरे सर्वमण्डनमण्डिते}
{हेमप्राकारसहिते चामीकरगृहेज्वले} %॥८॥

\threelineshloka
{तत्र सा ब्राह्मणी काचित् चारुनामेति विश्रुता}
{पतिभक्तिरता साध्वी श्वश्रूश्वशुरयोर्मता}
{कलानिधिसमारूपा सततं मञ्जुभाषिणी} %॥९॥

\twolineshloka
{तस्याः प्रसन्नचित्तेन लक्ष्मीः स्वप्नङ्गता तदा}
{एहि कल्याणभर्द्रं ते वरलक्ष्मीप्रसादतः} %॥१०॥

\twolineshloka
{नभोमासे पौर्णिमास्यां नातिक्रान्ते भृगोर्दिने}
{आरब्धव्यं व्रतं तत्र महालक्ष्म्या यतात्मभिः} %॥११॥

\twolineshloka
{सुवर्णप्रतिमां कुर्याच्चतुर्भुजसमन्विताम्}
{पूर्वगृहमलङ्कृत्य तोरणै रङ्गवल्लिकैः} %॥१२॥

\twolineshloka
{तद्दिने भार्गवे वारे नवभाण्डसमन्वितम्}
{गृहं च पूर्वदिग्भागे ईशान्यां च विशेषतः} %॥१३॥

\twolineshloka
{गोधूमान् प्रस्थसङ्ख्याकान् भूमौ निक्षिप्य पूजयेत्}
{संस्थाप्य कलशं तत्र तन्दुलैर्वाससा भरेत्} %॥१४॥

\twolineshloka
{पुष्पाणि च विनिक्षिप्य सुवर्णं प्रक्षिपंस्ततः}
{पल्लवांश्च विनिक्षिप्य वस्त्रेणाऽऽच्छाद्य यत्नतः} %॥१५॥

\twolineshloka
{प्रतिमांस्थापयेत् तत्र पूजयंश्च यथाविधि}
{पञ्चामृतेन स्नपनं कारयेन्मन्त्रितः सुधीः} %॥१६॥

\twolineshloka
{शुद्धस्नानं ततः कृत्वा देवीसूक्तेन वै ततः}
{अष्टगन्धैः समभ्यर्च्य पल्लवांश्च समर्पयेत्} %॥१७॥

\twolineshloka
{अश्वत्थवटविल्वादिचूतदाडिममल्लिकाः}
{तुलसी करवीरैश्च केतकैश्चम्पकैस्तथा} %॥१८॥

\twolineshloka
{एतेषां पत्रमादाय एकविंशतिसङ्ख्यया}
{नानाविधानि पुष्पाणि मालत्यादीनि वै ततः} %॥१९॥

\twolineshloka
{धूपदीपैर्महालक्ष्मीं पूजयेत् सर्वकामदाम्}
{पायसं सर्वमन्नञ्च सर्वभक्ष्येश्वसंयुतम्} %॥२०॥

\twolineshloka
{एकविंशतिसङ्ख्याकैरपूपैश्च न्यवेदयेत्}
{पुनः पञ्चैवते तत्र लक्ष्म्यर्थं तु विनिक्षिपेत्} %॥२१॥

\twolineshloka
{उपचारैर्बहुविधैर्नानासन्मानकैस्तथा}
{देव्यै सर्वं समर्प्याथ वरमिष्टं च याचयेत्} %॥२२॥

\twolineshloka
{नृत्यगीतादिसहितं देवीं सम्प्रार्थयेच्छ्रियम्}
{उमा सरस्वती धात्री शची च प्रियवादिनि} %॥२३॥

\twolineshloka
{एताभिश्च कृतं सम्यग् व्रतं सर्वसमृद्धिदम्}
{मत्पूजा तत्र कर्तव्या वरं दास्यामि काङ्क्षितम्} %॥२४॥

\uvacha{चारुमतिरुवाच}

\twolineshloka
{नमामि वरलक्ष्मीं त्वामागतां परमेश्वरीम्}
{नमस्ते सर्वलोकानां जनन्यै पुण्यमूर्तये} %॥२५॥

\twolineshloka
{शरण्ये त्रिजगद्वन्द्ये विष्णुवक्षस्थलस्थिते}
{त्वया विलोकिता प्रीत्या मुक्ता सा सद्यः सङ्कटात्} %॥२६॥

\twolineshloka
{जन्मान्तरसहस्रेषु किं मया सुकृतं कृतम्}
{यतस्त्वत्पादयुगुलं पश्यामि हरिवल्लभे} %॥२७॥

\twolineshloka
{एवं स्तुता सा कमला प्रहास्य च बहुन् वरान्}
{दत्वा चारुमतिस्तत्र स्वप्नादुत्थाय साक्षगात्} %॥२८॥

\twolineshloka
{तत्सर्वं कथयामास बन्धूनां पुरतस्तथा}
{श्रुत्वा ते बान्धवः सर्वे साधु साध्विति चाब्रुवन्} %॥२९॥

\twolineshloka
{तथैव करवामेति तदागमनकाङ्क्षिणी}
{भाग्योदये न सम्प्राप्तं वरलक्ष्मीदिनं तदा} %॥३०॥

\twolineshloka
{स्त्रियः प्रसन्नवदना निर्मलाश्चित्रवाससः}
{नूतने तन्दुलैः पूर्णे कुम्भे च पदवल्लभे} %॥३१॥

\twolineshloka
{पद्मासने पद्मकरे सर्वलोकैकपूजिते}
{नारायणप्रिये देवि सुप्रीता भव सर्वदा} %॥३२॥

\twolineshloka
{मन्त्रेणानेककलशे उपचाराननुक्रमैः}
{त्यक्त्वा च दक्षिणे हस्ते वरसुत्रं ददुः श्रियः} %॥३३॥

\twolineshloka
{अन्नदानरता नित्यं बन्धुपोषणतत्परा}
{पुत्रपौत्रैः परिवृता धनधान्यसमन्विता} %॥३४॥

\twolineshloka
{ततो देवीसमीपे तु तिष्ठती कृतमङ्गला}
{शिवदेव्यः प्रसादेन मुक्ताहारविभूषिता} %॥३५॥

\twolineshloka
{स्वपदं समयाजग्मुर्हत्त्यश्वरथसङ्कुला}
{अन्योन्याः कथयामास प्रीत्या चारुमतिस्तदा} %॥३६॥

\twolineshloka
{इदं गुह्यमिदं सत्यं नरो भद्राणि पश्यति}
{स्वयं चारुमतिर्मुख्यानुपलब्धा मनोरथान्} %॥३७॥

\twolineshloka
{पूज्या चारुमतिश्चैव भूत्वा भाग्यवतीश्चिरम्}
{एषा चारुमती साध्वी दृष्टा सा मित्रयोषिताम्} %॥३८॥

\twolineshloka
{इह मानुषलोके हि व्रतं कार्यं सुविस्तरम्}
{व्रतं पुण्यकरं चैव कुर्याद् भक्तिपुरःसरम्} %॥३९॥

\twolineshloka
{भक्त्या करोति विपुलान्}
{भोगान् प्राप्य श्रियं व्रजेत्} %॥४०॥

\twolineshloka
{व्रतानामुत्तमं पुण्यं वरलक्ष्मीव्रतं शुभम्}
{तत्कृतेन नरो नारी पद्भ्यां स्वर्गं गमिष्यति} %॥४१॥

\twolineshloka
{य इदं शृणुयान्नित्यं वाचयेद् वा समाहितः}
{धनधान्यं समाप्नोति वरलक्ष्मीप्रसादतः} %॥४२॥

॥इति श्रीभविष्योत्तरपुराणे ईश्वरपार्वतीसंवादे वरलक्ष्मीव्रतकथा समाप्ता॥
