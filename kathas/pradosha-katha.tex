।। ऋषय ऊचुः ।। ।।
यदुक्तं भवता सूत महदाख्यानमद्भुतम् ।।
शम्भोर्माहात्म्यकथनमशेषाघहरं परम् ।। १ ।।
भूयोपि श्रोतुमिच्छामस्तदेव सुसमाहिताः ।।
प्रदोषे भगवाञ्छंभुः पूजितस्तु महात्मभिः ।। २ ।।
संप्रयच्छति कां सिद्धिमेतन्नो ब्रूहि सुव्रत ।।
श्रुतमप्यसकृत्सूत भूयस्तृष्णा प्रवर्धते ।। ३ ।।
।। सूत उवाच ।। ।।
साधु पृष्टं महाप्राज्ञा भवद्भिर्लोकविश्रुतैः ।।
अतोऽहं संप्रवक्ष्यामि शिवपूजाफलं महत् ।। ४ ।।
त्रयोदश्यां तिथौ सायं प्रदोषः परिकीर्तितः ।।
तत्र पूज्यो महादेवो नान्यो देवः फलार्थिभिः ।। ५ ।।
प्रदोषपूजामाहात्म्यं को नु वर्णयितुं क्षमः ।।
यत्र सर्वेऽपि विबुधास्तिष्ठंति गिरिशांतिके ।। ६ ।।
प्रदोषसमये देवः कैलासे रजतालये ।।
करोति नृत्यं विबुधैरभिष्टुतगुणोदयः ।। ७ ।।
अतः पूजा जपो होमस्तत्कथास्तद्गुणस्तवः ।।
कर्तव्यो नियतं मर्त्यैश्चतुर्वर्गफला र्थिभिः ।। ८ ।।
दारिद्यतिमिरांधानां मर्त्यानां भवभीरुणाम् ।।
भवसागरमग्नानां प्लवोऽयं पारदर्शनः ।। ९ ।।
दुःखशोकभयार्तानां क्लेशनिर्वाणमिच्छताम् ।।
प्रदोषे पार्वतीशस्य पूजनं मंगलायनम् ।। ३.३.६.१० ।।
दुर्बुद्धिरपि नीचोपि मन्दभाग्यः शठोऽपि वा ।।
प्रदोषे पूज्य देवेशं विपद्भ्यः स प्रमुच्यते ।। ११ ।।
शत्रुभिर्हन्यमानोऽपि दश्यमानोपि पन्नगैः ।।
शैलैराक्रम्यमाणोऽपि पतितोऽपि महांबुधौ ।। १२ ।।
आविद्धकालदण्डोऽपि नानारोगहतोऽपि वा ।।
न विनश्यति मर्त्योऽसौ प्रदोषे गिरिशार्चनात् ।। १३ ।।
दारिद्र्यं मरणं दुःखमृणभारं नगोपमम् ।।
सद्यो विधूय संपद्भिः पूज्यते शिवपूजनात् ।। १४ ।।
अत्र वक्ष्ये महापुण्यमितिहासं पुरातनम् ।।
यं श्रुत्वा मनुजाः सर्वे प्रयांति कृतकृत्यताम् ।। १५ ।।
आसीद्विदर्भविषये नाम्ना सत्यरथो नृपः ।।
सर्वधर्मरतो धीरः सुशीलः सत्यसंगरः ।। १६ ।।
तस्य पालयतो भूमिं धर्मेण मुनिपुंगवाः ।।
व्यतीयाय महान्कालः सुखेनैव महामतेः ।। १७ ।।
अथ तस्य महीभर्तुर्बभूवुः शाल्वभूभुजः ।।
शत्रवश्चोद्धतबला दुर्मर्षणपुरोगमाः ।। १८ ।।
कदाचिदथ ते शाल्वाः संनद्धबहुसैनिकाः ।।
विदर्भनगरीं प्राप्य रुरुधुर्विजिगीषवः ।। १९ ।।
दृष्ट्वा निरुद्ध्यमानां तां विदर्भाधिपतिः पुरीम् ।।
योद्धुमभ्याययौ तूर्णं बलेन महता वृतः ।। ३.३.६.२० ।।
तस्य तैरभवयुद्धं शाल्वैरपि बलोद्धतैः ।।
पाताले पन्नगेन्द्रस्य गन्धर्वैरिव दुर्मदैः ।। २१ ।।
विदर्भनृपतिः सोऽथ कृत्वा युद्धं सुदारुणम् ।।
प्रनष्टोरुबलैः शाल्वैर्निहतो रणमूर्धनि ।। २२ ।।
तस्मिन्महारथे वीरे निहते मंत्रिभिः सह ।।
दुद्रुवुः समरे भग्ना हतशेषाश्च सैनिकाः ।। २३ ।।
अथ युद्धेभिविरते नदत्सु रिपुमंत्रिषु ।।
नगर्यां युद्ध्यमानायां जाते कोलाहले रवे ।। २४ ।।
तस्य सत्यरथस्यैका विदर्भाधिपतेः सती ।।
भूरिशोकसमाविष्टा क्वचिद्यत्नाद्विनिर्ययौ ।। २५ ।।
सा निशासमये यत्नादंतर्वत्नी नृपांगना ।।
निर्गता शोक संतप्ता प्रतीचीं प्रययौ दिशम् ।। २६ ।।
अथ प्रभाते मार्गेण गच्छन्ती शनकैः सती ।।
अतीत्य दूरमध्वानं ददर्श विमलं सरः ।। २७ ।।
तत्रागत्य वरारोहा तप्ता तापेन भूयसा ।।
विलसंतं सरस्तीरे छायावृक्षं समाश्रयत् ।। २८ ।।
तत्र दैववशाद्राज्ञी विजने तरुकुट्टिमे ।।
असूत तनयं साध्वी मूहूर्ते सद्गुणान्विते ।। २९ ।।
अथ सा राजमहिषी पिपासाभिहता भृशम् ।।
सरोऽवतीर्णा चार्वंगी ग्रस्ता ग्राहेण भूयसा ।। ३.३.६.३० ।।
जातमात्रः कुमारोऽपि विनष्टपितृमातृकः ।।
रुरोदोच्चैः सरस्तीरे क्षुत्पिपासार्दितोऽबलः ।। ३१ ।।
तस्मिन्नेवं क्रन्दमाने जातमात्रे कुमारके ।।
काचिदभ्याययौ शीघ्रं दिष्ट्या विप्रवरांगना ।। ३२ ।।
साप्येकहायनं बालमुद्वहन्ती निजात्मजम् ।।
अधना भर्तृरहिता याचमाना गृहेगृहे ।। ३३ ।।
एकात्मजा बंधुहीना याञ्चामार्गवशंगता।।
उमानाम द्विजसतीददर्श नृपनंदनम्।। ३४ ।।
सा दृष्ट्वा राजतनयं सूर्यबिंवमिव च्युतम्।।
अनाथमेनं क्रंदंतं चिंतयामास भूरिशः ॥ ३५॥
अहो सुमहदाश्चर्यमिदं दृष्टं मयाधुना ॥
अच्छिन्ननाभिसूत्रोऽयं शिशुर्माता क्व वा गता ॥ ३६ ॥
पिता नास्ति न चान्योस्ति नास्ति बंधुजनोऽपि वा ॥
अनाथः कृपणो बालः शेते केवल भूतले ॥ ३७ ॥
एष चांडालजो वाऽपि शूद्रजो वैश्यजोपि वा ॥
विप्रात्मजो वा नृपजो ज्ञायते कथमर्भकः ॥ ३८॥
शिशुमेनं समुद्धृत्य पुष्णाम्यौरसवद्ध्रुवम् ॥
किं त्वविज्ञातकुलजं नोत्सहे स्प्रष्टुमुत्तमम् ॥ ३९ ॥
इति मीमांसमानायां तस्यां विप्रवरस्त्रियाम् ॥ ३.३.६.४० ॥
कश्चित्समाययौ भिक्षुः साक्षाद्देवः शिवः स्वयम् ॥
तामाह भिक्षुवर्योथ विप्रभामिनि मा खिदः ॥ ११ ॥
रक्षैनं बालकं सुभ्रुर्विसृज्य हृदि संशयम्॥
अनेन परमं श्रेयः प्राप्स्यसे ह्यचिरादेिह ॥ ४२ ॥
एतावदुक्त्वा त्वरितो भिक्षुः कारुणिको ययौ ॥
अथ तस्मिन्गते भिक्षौ विश्रब्धा विप्रभामिनी ॥ ४३ ॥
तमर्भकं समादाय निजमेव गृहं ययौ ॥
भिक्षुवाक्येन विश्रब्धा सा राज तनयं सती ॥ ४४ ॥
आत्मपुत्रेण सदृशं कृपया पर्यपोषयत्॥
एकचक्राह्वये रम्ये ग्रामे कृतनिकेतना ॥ ४९ ॥
स्वपुत्रं राजपुत्रं च भिक्षान्नेन व्यवर्धयत् ॥
ब्राह्मणीतनयश्चैव स राजतनयस्तथा ॥ १६ ॥
ब्राह्मणैः कृतसंस्कारौ ववृधाते सुपूजितौ।॥
कृतोपनयनौ काले बालकौ नियमे स्थितौ ॥ ॥ ४७ ॥
भिक्षार्थं चेरतुस्तत्र मात्रा सह दिनेदिने ॥
ताभ्यां कदाचिद्बालाभ्यां सा विप्रवनिता सह ॥ ४८ ॥
भैक्ष्यं चरंती दैवेन प्रविष्टा देवतालयम्॥
तत्र वृद्धैः समाकीर्णे मुनिभिर्देवतालये ॥ १९ ॥
तौ दृष्ट्वा बालकौ धीमाञ्छांडिल्यो मुनिरब्रवीत् ॥
अहो दैवबलं चित्रमहो कर्म दुरत्ययम् ॥ ३.३.६.५० ॥
एष बालोऽन्यजननीं श्रितो भैक्ष्येण जीवति ॥
इमामेव द्विजवधूं प्राप्य मातरमुत्तमाम् ॥ ५१ ॥
सहैव द्विजपुत्रेण द्विजभावं समाश्रितः ॥
इति श्रुत्वा मुनेर्वाक्यं शांडिल्यस्य द्विजांगना ॥ ५२॥
सा प्रणम्य सभामध्ये पर्यपृच्छत्सविस्मया ॥
ब्रह्मन्नेषोर्भको नीतो मया भिक्षोर्गिरा गृहम् ॥ ५३ ॥
अविज्ञातकुलोद्यापि सुतवत्परिपोष्यते ॥
कस्मिन्कुले प्रसूतोऽयं का माता जनकोस्य कः ॥ ५४ ॥
सर्वं विज्ञातुमिच्छामि भवतो ज्ञानचक्षुषः ॥.५५ ।॥
इति पृष्टो मुनिः सोथ ज्ञानदृष्टिर्द्विजस्त्रियां ॥
आचख्यौ तस्य बालस्य जन्म कर्म च पौर्विकम् ॥ ५६ ॥
विदर्भराजपुत्रस्तु तत्पितुः समरे मृतिम्॥
तन्मातुर्नक्रहरणं साकल्येन न्यवेदयत् ॥ ५७ ॥
अथ सा विस्मिता नारी पुनः प्रपच्छ तं मुनिम् ॥
स राजा सकलान्भोगान्हित्वा युद्धे कथं मृतः ॥ ५८ ॥
दारिद्र्यमस्य बालस्य कथं प्राप्तं महामुने ॥
दारिद्र्यं पुनरुद्धूय कथं राज्यमवाप्स्यति ॥ ॥ ५९ ॥
अस्यापि मम पुत्रस्य भिक्षान्नेनैव जीवतः ।
दारिद्र्यशमनोपायमुपदेष्टुं त्वमर्हसि ॥३.३.६.६०॥
॥ शांडिल्य उवाच ॥॥
अमुष्य बालस्य पिता स विदर्भमहीपतिः ।।
पूर्वजन्मनि पांड्येशो बभूव नृपसत्तमः ।। ६१ ।।
स राजा सर्वधर्मज्ञः पालयन्सकलां महीम् ।।
प्रदोषसमये शंभुं कदा चित्प्रत्यपूजयत् ।। ६२ ।।
तस्य पूजयतो भक्त्या देवं त्रिभुवनेश्वरम् ।।
आसीत्कलकलारावः सर्वत्र नगरे महान् ।। ६३ ।।
श्रुत्वा तमुत्कटं शब्दं राजा त्यक्तशिवार्चनः ।।
निर्ययौ राजभवनान्नगरक्षोभशंकया ।। ६४ ।।
एतस्मिन्नेव समये तस्यामात्यो महाबलः ।।
शत्रुं गृहीत्वा सामंतं राजांतिकमुपागमत् ।। ६५ ।।
अमात्येन समानीतं शत्रुं सामंतमुद्धतम् ।।
दृष्ट्वा क्रोधेन नृपतिः शिरच्छेदमकारयत् ।। ६६ ।।
स तथैव महीपालो विसृज्य शिवपूजनम् ।।
असमाप्तात्मनियमश्चकार निशि भोजनम् ।। ६७ ।।
तत्पुत्रोपि तथा चक्रे प्रदोषसमये शिवम् ।।
अनर्चयित्वा मूढात्मा भुक्त्वा सुष्वाप दुर्मदः ।। ६८ ।।
जन्मांतरे स नृपतिर्विदर्भक्षितिपोऽभवत् ।।
शिवार्चनांतरायेण परैर्भोगांतरे हतः ।। ६९ ।।
तत्पुत्रो यः पूर्वभवे सोस्मिञ्जन्मनि तत्सुतः ।।
भूत्वा दारिद्र्यमापन्नः शिवपूजाव्यतिक्रमात् ।। ३.३.६.७० ।।
अस्य माता पूर्वभवे सपत्नीं छद्मनाहनत् ।।
तेन पापेन महता ग्राहेणास्मिन्भवे हता ।। ७१ ।।
एषा प्रवृत्तिरेतेषां भवत्यै समुदाहृता ।।
अनर्चितशिवा मर्त्याः प्राप्नुवंति दरिद्रताम् ।। ।। ७२ ।।
सत्यं ब्रवीमि परलोकहितं ब्रवीमि सारं ब्रवीम्युपनिषद्धृदयं ब्रवीमि ।।
संसारमुल्बणमसारमवाप्य जंतोः सारो यमीश्वरपदांबुरुहस्य सेवा ।। ७३ ।।
ये नार्चयंति गिरिशं समये प्रदोषे ये नार्चितं शिवमपि प्रणमंति चान्ये ।।
एतत्कथां श्रुतिपुटैर्न पिबंति मूढास्ते जन्मजन्मसु भवंति नरा दरिद्राः ।। ७४ ।।
ये वै प्रदोषसमये परमेश्वरस्य कुर्वंत्यनन्यमनसोंऽघ्रिसरोजपूजाम् ।।
नित्यं प्रवृद्धधन धान्यकलत्रपुत्रसौभाग्यसंपदधिकास्त इहैव लोके ।। ७५ ।।
कैलासशैलभवने त्रिजगजनित्रीं गौरीं निवेश्य कनकांचितरत्नपीठे ।।
नृत्यं विधातु मभिवाञ्छति शूलपाणौ देवाः प्रदोषसमयेऽनुभजंति सर्वे ।। ७६ ।।
वाग्देवी धृतवल्लकी शतमखो वेणुं दधत्पद्मजस्तालोन्निद्रकरो रमा भगवती गेयप्रयोगान्विता ।।
विष्णुः सांद्रमृदंगवादनपटुर्देवाः समंतात्स्थिताः सेवंते तमनु प्रदोषसमये देवं मृडानीपतिम् ।। ७७ ।।
गंधर्वयक्षपतगोरगसिद्ध साध्या विद्याधरामरवराप्सरसां गणाश्च ।।
येऽन्ये त्रिलोकनिलयाः सह भूतवर्गाः प्राप्ते प्रदोषसमये हरपार्थसंस्थाः ।। ७८ ।।
अतः प्रदोषे शिव एक एव पूज्योऽथ नान्ये हरिपद्मजाद्याः ।।
तस्मिन्महेशे विधिनेज्यमाने सर्वे प्रसीदंति सुराधिनाथाः ।। ७९ ।।
एष ते तनयः पूर्वजन्मनि ब्राह्मणोत्तमः ।।
प्रतिग्रहैर्वयो निन्ये न यज्ञाद्यैः सुकर्मभिः।।३.३.६.८०।।
अतो दारिद्र्यमापन्नः पुत्रस्ते द्विजभामिनि।।
तद्दोष परिहारार्थं शरणं यातु शंकरम् ।। ८१ ।।
इति श्रीस्कांदे महापुराण एकाशीतिसाहस्र्यां संहितायां तृतीये ब्रह्मोत्तरखण्डे प्रदोषमाहात्म्यवर्णनंनाम षष्ठोऽध्यायः ।। ६ ।। ।। छ ।।

।। सूत उवाच ।। ।।
इत्युक्ता मुनिना साध्वी सा विप्रवनिता पुनः ।।
तं प्रणम्याथ पप्रच्छ शिवपूजाविधेः क्रमम् ।। १ ।।
।। शांडिल्य उवाच ।। ।।
पक्षद्वये त्रयोदश्यां निराहारो भवेद्यदा ।।
घटीत्रयादस्तमयात्पूर्वं स्नानं समाचरेत् ।। २ ।।
शुक्लांबरधरो धीरो वाग्यतो नियमान्वितः।।
कृतसंध्याजपविधिः शिवपूजां समारभेत् ।। ३ ।।
देवस्य पुरतः सम्यगुपलिप्य नवांभसा ।।
विधाय मंडलं रम्यं धौतवस्त्रादिभिर्बुधः ।। ४।।
वितानाद्यैरलंकृत्य फलपुष्पनवांकुरैः ।।
विचित्रपद्ममुद्धृत्य वर्णपंचकसंयुतम् ।। ५ ।।
तत्रोपविश्य सुशुभे भक्तियुक्तः स्थिरासने ।।
सम्यक्संपादिताशेष पूजोपकरणः शुचिः ।। ६ ।।
आगमोक्तेन मंत्रेण पीठमामंत्रयेत्सुधीः ।।
ततः कृत्वात्मशुद्धिं च भूतशुद्ध्यादिकं क्रमात् ।। ७ ।।
प्राणायामत्रयं कृत्वा बीजवर्णैः सबिंदुकैः ।।
मातृका न्यस्य विधिवद्ध्यात्वा तां देवतां पराम् ।। ८ ।।
समाप्य मातृका भूयो ध्यात्वा चैव परं शिवम् ।।
वामभागे गुरुं नत्वा दक्षिणे गणपं नमेत् ।। ९ ।।
अंसोरुयुग्मे धर्मादीन्न्यस्य नाभौ च पार्श्वयोः ।।
अधर्मादीननंतादीन्हृदि पीठे मनुं न्यसेत् ।। ।। 3.3.7.१० ।।
आधारशक्तिमारभ्य ज्ञानात्मानमनुक्रमात् ।।
उक्तक्रमेण विन्यस्य हृत्पद्मे साधुभाविते ।। ११ ।।
नवशक्तिमये रम्ये ध्यायेद्देवमुमापतिम्।।
चन्द्रकोटिप्रतीकाशं त्रिनेत्रं चन्द्रशेखरम् ।। १२ ।।
आपिंगलजटाजूटं रत्नमौलिविराजितम् ।।
नीलग्रीवमुदारांगं नागहारोपशोभितम् ।। १३ ।।
वरदाभयहस्तं च धारिणं च परश्वधम् ।।
दधानं नागवलयकेयूरांगदमुद्रिकम् ।। १४ ।।
व्याघ्रचर्मपरीधानं रत्नसिंहासने स्थितम् ।।
ध्यात्वा तद्वाम भागे च चिंतयेद्गिरिकन्यकाम् ।। १५ ।।
भास्वज्जपाप्रसूनाभामुदयार्कसमप्रभाम् ।।
विद्युत्पुंजनिभां तन्वीं मनोनयननंदिनीम् ।। १६ ।।
बालेंदु शेखरां स्निग्धां नीलकुंचितकुन्तलाम् ।।
भृंगसंघातरुचिरां नीलालकविराजिताम् ।। १७ ।।
मणिकुंडलविद्योतन्मुखमंडलविभ्रमाम् ।।
नवकुम्कुमपंकांक कपोलदलदर्पणाम् ।।१८।।
मधुरस्मितविभ्राजदरुणाधरपल्लवाम् ।।
कंबुकंठीं शिवामुद्यत्कुचपंकजकुड्मलाम् ।। १९ ।।
पाशांकुशाभयाभीष्टविल सत्सु चतुर्भुजाम् ।।
अनेकरत्नविलसत्कंकणांकितमुद्रिकाम् ।। 3.3.7.२० ।।
वलित्रयेण विलसद्धेमकांचीगुणान्विताम् ।।
रक्तमाल्यांबरधरां दिव्यचंदनच र्चिताम् ।। २१ ।।
दिक्पालवनितामौलिसन्नतांघ्रिसरोरुहाम् ।।
रत्नसिंहासनारूढां सर्पराजपरिच्छदाम् ।। २१ ।।
एवं ध्यात्वा महादेवं देवीं च गिरि कन्यकाम् ।।
न्यासक्रमेण संपूज्य देवं गंधादिभिः क्रमात् ।। २३ ।।
पंचभिर्ब्रह्मभिः कुर्यात्प्रोक्तस्थानेषु वा हृदि ।।
पृथक्पुष्पांजलिं देहे मूलेन च हदि त्रिधा ।। २४ ।।
पुनः स्वयं शिवो भूत्वा मूलमंत्रेण साधकः ।।
ततः संपूजयेद्देवं बाह्यपीठे पुनः क्रमात् ।। २५ ।।
संकल्पं प्रवदेत्तत्र पूजारंभे समाहितः ।।
कृतांजलिपुटो भूत्वा चिंतयेद्धृदि शंकरम् ।। २६ ।।
ऋणपातकदौर्भाग्यदारिद्र्यविनिवृत्तये ।।
अशेषाघविनाशाय प्रसीद मम शंकर ।। २७ ।।
दुःखशोकाग्निसंतप्तं संसारभयपीडितम् ।।
बहुरोगाकुलं दीनं त्राहि मां वृषवाहन ।। २८ ।।
आगच्छ देवदेवेश महादेवाभयंकर ।।
गृहाण सह पार्वत्या तव पूजां मया कृताम् ।। २९।।
इति संकल्प्य विधिवद्ब्राह्मपूजां समाचरेत् ।।
गुरुं गणपतिं चैव यजेत्सव्यापसव्ययोः ।। 3.3.7.३० ।।
क्षेत्रेशमीशकोणे तु यजेद्वास्तोष्पतिं क्रमात् ।।
वाग्देवीं च यजेत्तत्र ततः कात्यायनीं यजेत् ।। ३१ ।।
धर्मं ज्ञानं च वैराग्यमैश्वर्यं च नमोंऽतकैः ।।
स्वरैरीशादिकोणेषु पीठपादाननुक्रमात् ।।
आभ्यां बिंदुविसर्गाभ्यामधर्मादीन्प्रपूजयेत् ।। ३२ ।।
सत्त्वरूपैश्चतुर्दिक्षु मध्येऽनंतं सतारकम् ।।
सत्त्वादींस्त्रिगुणांस्तं तु रूपान्पीठेषु विन्यसेत् ।। ३३ ।।
अत ऊर्ध्वच्छदे मायां सह लक्ष्म्या शिवेन च ।। ३४ ।।
तदंते चांबुजं भूयः सकलं मंडलत्रयम् ।।
पत्रकेसरकिंजल्कव्याप्तं ताराक्षरैः क्रमात् ।। ३५ ।।
पद्मत्रयं तथाभ्यर्च्य मध्ये मंडलमादरात् ।।
वामां ज्येष्ठां च रौद्रीं च भागाद्यैर्दिक्षु पूजयेत् ।। ।। ३६ ।।
वामाद्या नव शक्तीश्च नवस्वरयुता यजेत् ।।
हृदि बीजत्रयाद्येन पीठमंत्रेण चार्चयेत् ।।३७।।
आवृत्तैः प्रथमांगैश्च पंचभिर्मूर्तिशक्तिभिः ।।
त्रिशक्तिमूर्तिभिश्चान्यैर्निधिद्वयसमन्वितैः ।। ३८ ।।
अनंताद्यैः परीताश्च मातृभिश्च वृषादिभिः ।।
सिद्धिभिश्चाणिमाद्याभिरिंद्राद्यैश्च सहायुधैः ।। ।। ३९ ।।
वृषभक्षेत्रचंडेशदुर्गाश्च स्कंदनंदिनौ ।।
गणेशः सैन्यपश्चैव स्वस्वलक्षणलक्षिताः ।। 3.3.7.४० ।।
अणिमा महिमा चैव गरिमा लघिमा तथा ।।
ईशित्वं च वशित्वं च प्राप्तिः प्राकाम्यमेव च ।। ४१ ।।
अष्टैश्वर्याणि चोक्तानि तेजोरूपाणि केवलम् ।।
पंचभिर्ब्रह्मभिः पूर्वं हृल्लेखाद्यादिभिः क्रमात ।। ४२ ।।
अंगैरुमाद्यैरिंद्राद्यैः पूजोक्ता मुनिभिस्तु तैः ।।
उमाचंडेश्वरादींश्च पूजयेदुत्तरादितः ।। ४३ ।।
एवमावरणैर्युक्तं तेजोरूपं सदाशिवम् ।।
उमया सहितं देवमुपचारैः प्रपूजयेत् ।। ४४ ।।
सुप्रतिष्ठितशंखस्य तीर्थैः पंचामृतैरपि ।।
अभिषिच्य महादेवं रुद्रसूक्तैः समाहितः ।। ४५ ।।
कल्पयेद्विविधैर्मंत्रैरासनाद्युपचारकान्।।
आसनं कल्पयेद्धैमं दिव्यवस्त्रसमन्वितम् ।। ४६ ।।
अर्घ्यमष्टगुणोपेतं पाद्यशुद्धोदकेन च ।।
तेनैवाचमनं दद्यान्मधुपर्कं मधूत्तरम् ।। ४७ ।।
पुनराचमनं दत्त्वा स्नानं मंत्रै प्रकल्पयेत् ।।
उपवीतं तथा वासो भूषणानि निवेदयेत् ।।
गंधमष्टांगसंयुक्तं सुपूतं विनिवेदयेत् ।। ।। ४८ ।।
ततश्च बिल्वमंदारकह्लारसरसीरुहम् ।।
धत्तूरकं कर्णिकारं शणपुष्पं च मल्लिकाम् ।। ४९ ।।
कुशापामार्गतुलसीमाधवीचंपकादिकम् ।।
बृहतीकरवीराणि यथालब्धानि साधकः ।। 3.3.7.५० ।।
निवेदयेत्सुगंधीनि माल्यानि विविधानि च ।।
धूपं कालागरूत्पन्नं दीपं च विमलं शुभम् ।। ।।५१।।
विशेषकम् ।।
अथ पायसनैवेद्यं सघृतं सोपदंशकम् ।।
मोदकापूपसंयुक्तं शर्करागुडसंयुतम् ।। ५२ ।।
मधुनाक्तं दधियुतं जलपानसमन्वितम् ।।
तेनैव हविषा वह्नौ जुहुयान्मंत्रभाविते ।। ५३ ।।
आगमोक्तेन विधिना गुरुवाक्यनियंत्रितः ।।
नैवेद्यं शंभवे भूयो दत्त्वा तांबूलमुत्तमम् ।। ५४ ।।।
धूपं नीराजनं रम्यं छत्रं दर्पणमुत्तमम् ।।
समर्पयित्वा विधिवन्मंत्रैर्वेदिकतांत्रिकैः ।। ५५ ।।
यद्यशक्तः स्वयं निःस्वो यथाविभवमर्चयेत् ।।
भक्त्त्या दत्तेन गौरीशः पुष्पमात्रेण तुष्यति ।। ५६ ।।
अथांगभूतान्सकलान्गणेशादीन्प्रपूजयेत् ।।
स्तवैर्नानाविधैः स्तुत्वा साष्टांगं प्रणमेद्बुधः ।। ५७ ।।
ततः प्रदक्षिणीकृत्य वृषचंडेश्वरादिकान् ।।
पूजां समर्प्य विधिवत्प्रार्थयेद्गिरिजापतिम् ।। ५८ ।।
जय देव जगन्नाथ जय शंकर शाश्वत ।।
जय सर्व सुराध्यक्ष जय सर्वसुरार्चित ।। ५९ ।।
जय सर्वगुणातीत जय सर्ववरप्रद ।।
जय नित्य निराधार जय विश्वंभराव्यय ।। 3.3.7.६० ।।
जय विश्वैकवेद्येश जय नागेंद्रभूषण ।।
जय गौरीपते शंभो जय चंद्रार्धशेखर ।। ६१ ।।
जय कोट्यर्कसंकाश जयानंतगुणाश्रय ।। ६२ ।।
जय रुद्र विरूपाक्ष जयाचिंत्य निरंजन ।।
जय नाथ कृपासिंधो जय भक्तार्तिभञ्जन ।।
जय दुस्तरसंसारसागरोत्तारण प्रभो ।। ६३ ।।
प्रसीद मे महादेव संसारार्तस्य खिद्यतः ।।
सर्वपापभयं हृत्वा रक्ष मां परमेश्वर ।। ६४ ।।
महादारिद्र्यमग्नस्य महापापहतस्य च ।।
महाशोकविनष्टस्य महारोगातुरस्य च ।। ६५ ।।
ऋणभारपरीतस्य दह्यमानस्य कर्मभिः ।।
ग्रहैः प्रपीड्यमानस्य प्रसीद मम शंकर ।। ६६ ।।
दरिद्रः प्रार्थयेदेवं पूजांते गिरिजापतिम् ।।
अर्थाढ्यो वाऽपि राजा वा प्रार्थयेद्देवमीश्वरम् ।। ६७ ।।
दीर्घमायुः सदारोग्यं कोशवृद्धिर्बलोन्नतिः ।।
ममास्तु नित्यमानन्दः प्रसादात्तव शंकर ।। ६८ ।।
शत्रवः संक्षयं यांतु प्रसीदन्तु मम ग्रहाः ।।
नश्यन्तु दस्यवो राष्ट्रे जनाः संतु निरापदः ।। ६९ ।।
दुर्भिक्षमारीसंतापाः शमं यांतु महीतले ।।
सर्वसस्यसमृद्धिश्च भूयात्सुखमया दिशः ।। 3.3.7.७० ।।
एवमाराधयेद्देवं प्रदोषे गिरिजापतिम् ।।
ब्राह्मणान्भोजयेत्पश्चाद्दक्षिणाभिश्च तोषयेत् ।। ७१ ।।
सर्वपापक्षयकरी सर्वदारिद्र्यनाशिनी ।।
शिवपूजा मया ख्याता सर्वाभीष्टवरप्रदा ।। ७२ ।।
महापातकसंघातमधिकं चोपपातकम् ।।
शिवद्रव्यापहरणादन्यत्सर्वं निवारयेत् ।। ७३ ।।
ब्रह्महत्यादिपापानां पुराणेषु स्मृतिष्वपि ।।
प्रायश्चित्तानि दृष्टानि न शिवद्रव्यहारिणाम् ।। ७४ ।।
बहुनात्र किमुक्तेन श्लोकार्धेन ब्रवीम्यहम् ।।
ब्रह्महत्याशतं वाऽपि शिवपूजा विनाशयेत् ।। ७५ ।।
मया कथितमेतत्ते प्रदोषे शिवपूजनम् ।।
रहस्यं सर्वजंतूनामत्र नास्त्येव संशयः ।। ७६ ।।
एताभ्यामपि बालाभ्यामेवं पूजा विधीयताम् ।।
अतः संवत्सरादेव परां सिद्धिमवाप्स्यथ ।। ७७ ।।
इति शांडिल्यवचनमाकर्ण्य द्विजभामिनी ।।
ताभ्यां तु सह बालाभ्यां प्रणनाम मुनेः पदम् ।। ७८ ।। ।।
।। विप्रस्त्र्युवाच ।। ।।
अहमद्य कृतार्थास्मि तव दर्शनमात्रतः ।।
एतौ कुमारौ भगवंस्त्वामेव शरणं गतौ ।। ७९ ।।
एष मे तनयो ब्रह्मञ्छुचिव्रत इतीरितः ।।
एष राजसुतो नाम्ना धर्मगुप्तः कृतो मया ।। 3.3.7.८० ।।
एतावहं च भगवन्भवच्चरणकिंकराः ।।
समुद्धरास्मिन्पतितान्घोरे दारिद्र्यसागरे ।। ८१ ।।
इति प्रपन्नां शरणं द्विजांगनामाश्वास्य वाक्यैरमृतोपमानैः ।।
उपादिदेशाथ तयोः कुमारयोर्मुनिः शिवाराधनमंत्र विद्याम् ।। ८२ ।।
अथोपदिष्टौ मुनिना कुमारौ ब्राह्मणी च सा ।।
तं प्रणम्य समामंत्र्य जग्मुस्ते शिवमंदिरात् ।। ८३ ।।
ततः प्रभृति तौ बालौ मुनिवर्योपदेशतः ।।
प्रदोषे पार्वतीशस्य पूजां चक्रतुरंजसा ।। ८४ ।।
एवं पूजयतोर्देवं द्विजराजकुमारयोः ।।
सुखेनैव व्यतीयाय तयोर्मासचतुष्टयम् ।। ८५ ।।
कदाचिद्राजपुत्रेण विनासौ द्विजनंदनः ।।
स्नातुं गतो नदीतीरे चचार बहुलीलया ।। ८६ ।।
तत्र निर्झरनिर्घातनिर्भिन्ने वप्र कुट्टिमे ।।
निधानकलशं स्थूलं प्रस्फुरंतं ददर्श ह ।। ८७ ।।
तं दृष्ट्वा सहसागत्य हर्षकौतुकविह्वलः ।।
दैवोपपन्नं मन्वानो गृहीत्वा शिरसा ययौ ।। ।। ८८ ।।
ससंभ्रमं समानीय निधाय कलशं बलात् ।।
निधाय भवनस्यांते मातरं समभाषत ।। ८९ ।।
मातर्मातरिमं पश्य प्रसादं गिरिजापतेः ।।
निधानं कुम्भरूपेण दर्शितं करुणात्मना ।। 3.3.7.९० ।।
अथ सा विस्मिता साध्वी समाहूय नृपात्मजम् ।।
स्वपुत्रं प्रतिनंद्याह मानयन्ती शिवार्चनम् ।। ।। ९१ ।।
शृणुतां मे वचः पुत्रौ निधानकलशीमिमाम् ।।
समं विभज्य गृह्णीतं मम शासनगौरवात् ।। ९२ ।।
इति मातुर्वचः श्रुत्वा तुतोष द्विज नंदनः ।।
प्रत्याह राजपुत्रस्तां विस्रब्धः शंकरार्चने ।। ९३ ।।
मातस्तव सुतस्यैव सुकृतेन समागतम् ।।
नाहं ग्रहीतुमिच्छामि विभक्तं धनसंच यम् ।। ९४ ।।
आत्मनः सुकृताल्लब्धं स्वयमेव भुनक्त्वसौ ।।
स एव भगवानीशः करिष्यति कृपां मयि ।। ९५ ।।
एवमर्चयतोः शंभुं भूयोपि परया मुदा ।।
संवत्सरो व्यतीयाय तस्मिन्नेव गृहे तयोः ।। ९६ ।।
अथैकदा राजसूनुः सह तेन द्विजन्मना ।।
वसंतसमये प्राप्ते विजहार वनां तरे ।। ९७ ।।
अथ दूरं गतौ क्वापि वने द्विजनृपात्मजौ ।।
गन्धर्वकन्याः क्रीडंती शतशस्तावपश्यताम् ।। ९८ ।।
ताः सर्वाश्चारुसर्वांग्यो विहरंत्यो मनोहरम् ॥
दृष्ट्वा द्विजात्मजो दूरादुवाच नृपनंदनम् ॥ ९९ ॥
इतः पुरो न गंतव्यं विहरंत्यग्रतः स्त्रियः॥
स्त्रीसंन्निधानं विबुधास्त्यजंति विमलाशयाः॥ 3.3.7.१०० ॥
एताः कैतवकारिण्यो धनयौवनदुर्मदाः ।
मोहयन्त्यो जनं दृष्ट्वा वाचानुनयकोविदाः ॥ १ ॥
अतः परित्यजेत्त्स्त्रीणां सन्निधिं सहभाषणम् ।
निज़धर्मरतो विद्वन्ब्रह्मचारी विशेषतः ॥ ३॥
अतोहं नोत्सहे गन्तुं क्रीडास्थानं मृगीदृशाम् ॥
इत्युक्त्वा द्विजपुत्रस्तु निवृत्तो दूरतः स्थितः ॥ ३ ॥
अथासौ राजपुत्रस्तु कौतुकाविष्टमानसः ॥
तासां विहारपदवीमेक एवाभयो ययौ ॥ १ ॥
तत्र गंधर्वकन्यानां मध्ये त्वेका वरानना ॥
दृष्ट्वाऽऽयांतं राजपुत्रं चिंतयामास चेतसा ॥ ९ ॥
अहो कोयमुदारांगो युवा सर्वांगसुन्दरः ॥
मत्तमातंग गमनो लावण्यामृतवारिधिः ॥ ६॥
लीलालोलविशालाक्षी मधुरस्मितपेशलः ॥
मदनोपमरूपश्रीः सुकुमारांगलक्षणः ॥ ७॥
इत्याश्चर्ययुता बाला दूराद्दृष्ट्वा नृपात्मजम् ॥
सर्वाः सखीः समालोक्य वचनं चेदमब्रवीत् ॥ ८ ॥
इतो विदूरे हे सख्यो वनमस्त्येकमुत्तमम् ॥
विचित्रचंपकाशोकपुन्नागबकुलैर्युतम् ॥ ९॥
तत्र गत्वा वनं सर्वाः संचीय कुसुमोत्करम् ॥
भवत्यः पुनरायांतु तावत्तिष्ठाम्यहं त्विह ॥ 3.3.7.११० ॥
इत्यादिष्टः सखीवर्गो जगाम विपिनांतरम् ॥
सापि गंधर्वजा तस्थौ न्यस्तदृष्टिर्नृपात्मजे ॥ ११ ॥
तां समालोक्य तन्वंगीं नवयौवनशालिनीम् ॥
बालां स्वरूपसंपत्त्या परैिभूततिलोत्तमाम् ॥ १२ ॥
राजपुत्रः समागम्य कौतुकोत्फुल्ललोचनः ॥
अवाप दैवयोगेन मदनस्य शरव्यथाम् ॥ १३ ॥
गन्धर्वतनया सापि प्राप्ताय नृपसूनवे ॥
उत्थाय तरसा तस्मै प्रददौ पल्लवासनम् ॥ १४ ॥
कृतोपचारमासीनं तमासाद्य सुमध्यमा ॥
पप्रच्छ तद्रूपगुणैर्ध्वस्तधैर्याकुलेंद्रिया ॥ १९ ।
कस्त्वं कमलपत्राक्ष कस्माद्देशादिहागतः ॥
कस्य पुत्र इति प्रेम्णा पृष्टः सर्वं न्यवेदयत् ॥ १६ ॥
विदर्भ राजतनयं विध्वस्तपितृमातृकम् ॥
शत्रुभिश्च हृतस्थानमात्मानं परराष्ट्रगम् ॥ १७ ॥
सर्वमावेद्य भूयस्तां पप्रच्छ नृपनंदनः ॥
का त्वं वामोरु किं चात्र कार्यं ते कस्य चात्मजा ॥ १८ ॥
किमवध्यायसि हृदा किं वा वक्तुमिहेच्छसि ॥
इत्युक्ता सा पुनः प्राह शृणु राजेंद्रसत्तम ॥ १९ ॥
अस्त्येको द्रविकोनाम गंधर्वाणां कुलाग्रणीः ॥
तस्याहमस्मि तनया नाम्ना चांशुमती स्मृता ॥ 3.3.7.१२० ॥
त्वामायांतं विलोक्याहं त्वत्संभाषण लालसा ॥
त्यक्त्वा सखीजनं सर्वमेकैवास्मि महामते ॥ २१ ॥
सर्वसंगीतविद्यासु न मत्तोऽन्यास्ति काचन ॥
मम योगेन तुष्यंति सर्वा अपि सुरस्त्रियः ॥ २२ ॥
साहं सर्वकलाभिज्ञा ज्ञातसर्वजनेंगिता ॥
तवाहमीप्सितं वेद्मि मयि ते संगतं मनः ॥ २३ ॥
तथा ममापि चौत्सुक्यं दैवेन प्रतिपादितम्।।
आवयोः स्नेहभेदोऽत्र नाभिभूयादितः परम् ।। २४ ।।
इति संभाष्य तेनाशु प्रेम्णा गन्धर्वनंदिनी ।।
मुक्ताहारं ददौ तस्मै स्वकुचांतरभूषणम्।। २५ ।।
तमादायाद्भुतं हारं स तस्याः प्रणयाकुलः ।।
गाढहर्षभरोत्सिक्तामिदमाह नृपात्मजः ।। २६ ।।
सत्यमुक्तं त्वया भीरु तथाप्येकं वदाम्यहम् ।।
त्यक्तराज्यस्य निःस्वस्य कथं मे भवसि प्रिया ।। २७ ।।
सा त्वं पितृमती बाला विलंघ्य पितृशासनम् ।।
स्वच्छंदा चरणं कर्तुं मूढेव कथमर्हसि ।। २८ ।।
इति तस्य वचः श्रुत्वा तं प्रत्याह शुचिस्मिता ।।
अस्तु नाम तथैवाहं करिष्ये पश्य कौतुकम् ।। २९ ।।
गच्छ स्वभवनं कांत परश्वः प्रातरेव तु ।।
आगच्छ पुनरत्रैव कार्यमस्ति च नो मृषा ।। 3.3.7.१३० ।।
इत्युक्त्वा तं नृपसुतं सा संगतसखीजना ।।
अपाक्रामत चार्वंगी स चापि नृपनन्दनः ।। ३१ ।।
स समभ्येत्य हर्षेण द्विजपुत्रस्य सन्निधिम् ।।
सर्वमाख्याय तेनैव सार्धं स्वभवनं ययौ ।। ।। ३२ ।।
तां च विप्रसतीं भूयो हर्षयित्वा नृपात्मजः ।।
परश्वो द्विजपुत्रेण सार्धं तेन वनं ययौ ।। ३३ ।।
स तया पूर्वनिर्दिष्टं स्थानं प्राप्य नृपात्मजः ।।
गन्धर्वराजमद्राक्षीत्स्वदुहित्रा समन्वितम् ।। ३४ ।।
स गंधर्वपतिः प्राप्तावभिनंद्य कुमारकौ ।।
उपवेश्यासने रम्ये राजपुत्रमभाषत ।। ।। ३५ ।।
।। गंधर्व उवाच ।। ।।
राजेंद्रपुत्र पूर्वेद्युः कैलासं गतवानहम् ।।
तत्रापश्यं महादेवं पार्वत्या सहितं प्रभुम् ।। ३६ ।।
आहूय मां स देवेशः सर्वेषां त्रिदिवौकसाम् ।।
सन्निधावाह भगवान्करुणामृतवारिधिः । ३७ ।।
धर्मगुप्ताह्वयः कश्चिद्राजपुत्रोऽस्ति भूतले ।।
अकिञ्चनो भ्रष्टराज्यो हृतदेशश्च शत्रुभिः ।। ३८ ।।
स बालो गुरुवाक्येन मदर्च्चायां रतः सदा ।।
अद्य तत्पितरः सर्वे मां प्राप्तास्तत्प्रभावतः ।। ३९ ।।
तस्य त्वमपि साहाय्यं कुरु गन्धर्वसत्तम ।।
अथासौ निजराज्यस्थो हतशत्रुर्भविष्यति ।। 3.3.7.१४० ।।
इत्याज्ञप्तो महेशेन संप्राप्तो निजमदिरम् ।।
अनया मद्दुहित्रा च बहुशोऽभ्यर्थितस्तथा ।। ४१ ।।
ज्ञात्वेमं सकलं शंभोर्नियोगं करुणात्मनः ।।
आदायेमां दुहितरं प्राप्तोऽस्मीदं वनांतरम् ।।४२।।
अत एनां प्रयच्छामि कन्यामंशुमतीं तव ।।
हत्वा शत्रून्स्वराष्टे त्वां स्थापयामि शिवाज्ञया ।। ४३ ।।
तस्मिन्पुरे त्वमनया भुक्त्वा भोगान्यथेप्सितान्।।
दशवर्षसहस्रान्ते गन्तासि गिरिशालयम् ।।
तत्रापि मम कन्येयं त्वामेव प्रतिपत्स्यते ।।
अनेनैव स्वदेहेन दिव्येन शिवसन्निधौ ।। ४५ ।।
इति गन्धर्वराजस्तमाभाष्य नृपनन्दनम् ।।
तस्मिन्वने स्वदुहितुः पाणिग्रहमकारयत् ।। ४६ ।।
पारिबर्हमदात्तस्मै रत्नभारान्महोज्ज्वलान् ।।
चूडामणिं चन्द्रनिभं मुक्ताहारांश्च भासुरान्।।४७।।
दिव्यालंकारवासांसि कार्तस्वरपरिच्छदान् ।।
गजानामयुतं भूयो नियुतं नीलवाजिनाम्।।४८।।
स्यन्दनानां सहस्राणि सौवर्णानि महांति च ।।
पुनरेकं रथं दिव्यं धनुश्चेन्द्रायुधोपमम् ।। ४९ ।।
अस्त्राणां च सहस्राणि तूणी चाक्षय्यसायकौ ।।
अभेद्यं वर्म सौवर्णं शक्तिं च रिपुमर्दिनीम् ।। 3.3.7.१५० ।।
दुहितुः परिचर्यार्थं दासीपञ्चसहस्रकम् ।।
ददौ प्रीतमनास्तस्मै धनानि विविधानि च ।। ।। ५१ ।।
गंधर्वसैन्यमत्युग्रं चतुरंगसममन्वितम् ।।
पुनश्च तत्सहायार्थे गन्धर्वाधिपतिर्ददौ ।। ५२ ।।
इत्थं राजेन्द्रतनयः संप्राप्तः श्रियमुत्तमाम् ।।
अभीष्टजायासहितो मुमुदे निजसंपदा ।। ५३ ।।
कारयित्वा स्वदुहितुर्विवाहं समयोचितम् ।।
ययौ विमानमारुह्य गंधर्वाधिपतिर्दिवम् ।। ५४ ।।
धर्मगुप्तः कृतोद्वाहः सह गंधर्वसेनया ।।
पुनः स्वनगरं प्राप्य जघान रिपुवाहिनीम् ।। ५५ ।।
दुर्धर्षणं रणे हत्वा शक्त्या गंधर्वसेनया ।।
निः शेषितारातिबलः प्रविवेश निजं पुरम् ।। ५६ ।।
ततोभिषिक्तः सचिवैर्ब्राह्मणैश्च महोत्तमैः ।।
रत्नसिंहासनारूढश्चक्रे राज्यमकण्टकम् ।। ५७ ।।
या विप्रवनिता पूर्वं तमपुष्णात्स्वपुत्रवत् ।।
सैव माताभवत्तस्य स भ्राता द्विजनन्दनः ।। ५८ ।।
गंधर्वतनया जाया विदर्भनगरेश्वरः ।।
आराध्य देवं गिरिशं धर्मगुप्तो नृपोऽभवत् ।। ५९ ।।
एवमन्ये समाराध्य प्रदोषे गिरिजापतिम् ।।
लभंतेभीप्सितान्कामान्देहांते तु परां गतिम् ।। 3.3.7.१६० ।।
।। सूत उवाच ।। ।।
एतन्महाव्रतं पुण्यं प्रदोषे शंकरार्चनम् ।।
धर्मार्थकाममोक्षाणां यदेतत्साधनं परम् ।। ६१ ।।
य एतच्छृणुयात्पुण्यं माहात्म्यं परमाद्भुतम् ।।
प्रदोषे शिवपूजांते कथयेद्वा समाहितः ।। ६२ ।।
भवेन्न तस्य दारिद्र्यं जन्मान्तरशतेष्वपि ।।
ज्ञानैश्वर्यसमायुक्तः सोन्ते शिवपुरं व्रजेत् ।। ६३ ।।
ये प्राप्य दुर्लभतरं मनुजाः शरीरं कुर्वंति हंत परमेश्वरपादपूजाम् ।।
धन्यास्त एव निजपुण्यजितत्रिलोकास्तेषां पदांबुजरजो भुवनं पुनाति ।। १६४ ।। ।।
इति श्रीस्कान्दे महापुराण एकाशीतिसाहस्र्यां संहितायां तृतीये ब्रह्मोत्तरखण्डे प्रदोषमहिमावर्णनं नाम सप्तमो ऽध्यायः ।। ७ ।।
