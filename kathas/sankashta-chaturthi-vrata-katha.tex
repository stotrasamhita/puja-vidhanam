\sect{सङ्कष्ट-चतुर्थी-व्रत-कथा}
\centerline{\small{(मूलम्—श्री-व्रतराजः)}}

\uvacha{ऋषय ऊचुः}

\twolineshloka
{दारिद्र्यशोककष्टाद्यैः पीडितानां च वैरिभिः}
{राज्यभ्रष्टैर्नृपैः सर्वैः क्रियते किं शुभार्थिभिः} %१


\twolineshloka
{धनहीनैर्नरैः स्कन्द सर्वोपद्रवपीडितैः}
{विद्यापुत्रगृहभ्रष्टै रोगयुक्तैः शुभार्थिभिः} %२

\onelineshloka*
{कर्तव्यं किं वदोपायं पुनः क्षेमार्थसिद्धये}

\uvacha{स्कन्द उवाच}
\onelineshloka
{शृणुध्वं मुनयः सर्वे व्रतानामुत्तमं व्रतम्} %३


\twolineshloka
{सङ्कष्टतरणं नामामुत्रेह सुखदायकम्}
{येनोपायेन सङ्कष्टं तरन्ति भुवि देहिनः} %४


\twolineshloka
{यद्व्रतं देवकीपुत्रः कृष्णो धर्माय दत्तवान्}
{अरण्ये क्लिश्यमानाय पुनः क्षेमार्थसिद्धये} %५


\twolineshloka
{यथा कथितवान् पूर्वं गणेशो मातरं प्रति}
{तथा कथितवाञ्छ्रीशो द्वापरे पाण्डवान्प्रति} %६

\uvacha{ऋषय ऊचुः}

\twolineshloka
{कथं कथितवानम्बां पार्वतीं श्रीगणेश्वरः}
{यथा पृच्छन्ति मुनयो लोकानुग्रहकाङ्क्षिणः} %७

\uvacha{स्कन्द उवाच}

\twolineshloka
{पुरा कृतयुगे पुण्ये हिमाचलसुता सती}
{तपस्तप्तवती भूरि तेनालब्धः शिवः पतिः\footnotemark}\footnotetext{न दृष्टः शङ्करः पतिरित्यपि पाठः} %८


\twolineshloka
{तदाऽस्मरत्सा हेरम्बं गणेशं पूर्वजं सुतम्}
{तत्क्षणादागतं दृष्ट्वा गणेशं परिपृच्छति} %९

\uvacha{पार्वत्युवाच}

\twolineshloka
{तपस्तप्तं मया घोरं दुश्चरं लोमहर्षणम्}
{न प्राप्तः स मया कान्तो गिरीशो मम वल्लभः} %१०


\twolineshloka
{सङ्कष्टतरणं दिव्यं व्रतं नारद उक्तवान्}
{त्वदीयं यद्व्रतं तावत् कथयस्व पुरातनम्} %११


\twolineshloka
{तच्छ्रुत्वा पार्वतीवाक्यं सङ्कष्टतरणं व्रतम्}
{प्रीत्या कथितवान् देवो गणेशो ज्ञानसिद्धिदः} %१२

\uvacha{गणेश उवाच}

\twolineshloka
{श्रावणे बहुले पक्षे चतुर्थ्यां तु विधूदये}
{गणेशं पूजयित्वा तु चन्द्रायार्घ्यं प्रदापयेत्} %१३

\uvacha{पार्वत्युवाच}

\twolineshloka
{क्रियते केन विधिना किं कार्य किं च पूजनम्}
{उद्यापनं कदा कार्यं मन्त्राः के स्युस्तु पूजने} %१४

\onelineshloka*
{किं ध्यानं श्रीगणेशस्य गणेश वद विस्तरात्}

\uvacha{गणेश उवाच}
\onelineshloka
{चतुर्थ्यां प्रातरुत्थाय दन्तधावनपूर्वकम्} %१५


\twolineshloka
{ग्राह्यं व्रतमिदं पुण्यं सङ्कष्टतरणं शुभम्}
{कर्तव्यमिति सङ्कल्प्य व्रतेऽस्मिन् गणपं स्मरेत्} %१६


स्वीकारमन्त्रः—\hfill 
\twolineshloka
{निराहारोऽस्मि देवेश यावच्चन्द्रोदये भवेत्}
{भोक्ष्यामि पूजयित्वाऽहं सङ्कष्टात्तारयस्व माम्} %१७


\twolineshloka
{एवं सङ्कल्प्य राजेन्द्र स्नात्वा कृष्णतिलैः शुभैः}
{आह्निकं तु विधायैव पश्चात्पूज्यो गणाधिपः} %१८


\twolineshloka
{त्रिभिर्माषैस्तदर्धेन तृतीयांशेन वा पुनः}
{यथाशक्त्या तु वा हैमी प्रतिमा क्रियते मम} %१९


\twolineshloka
{हेमाभावे तु रौप्यस्य ताम्रस्यापि यथासुखम्}
{सर्वथैव दरिद्रेण क्रियते मृन्मयी शुभा} %२०


\twolineshloka
{वित्तशाठ्यं न कर्तव्यं कृते कार्यं विनश्यति}
{जलपूर्णं वस्त्रयुतं कुम्भं तदुपरि न्यसेत्} %२१


\twolineshloka
{पूर्णपात्रं तत्र पद्मं लिखेदष्टदलं शुभम्}
{देवतां तत्र संस्थाप्य गन्धपुष्पैः प्रपूजयेत्} %२२


\twolineshloka
{एवं व्रतं प्रकर्तव्यं प्रतिमासं त्वयाऽद्रिजे}
{यावज्जीवं तु वा वर्षाण्येकविंशतिमेव वा} %२३


\twolineshloka
{अशक्तोऽप्येकवर्षं वा प्रतिवर्षमथापि वा}
{उद्यापनं तु कर्तव्यं चतुर्थ्यां श्रावणेऽसिते} %२४


\twolineshloka
{स्वीकारश्च तथा कार्यः सङ्कष्टहरणे तिथौ}
{गाणपत्यं तथाऽऽचार्यं सर्वशास्त्रविशारदम्} %२५


\twolineshloka
{श्रद्धया प्रार्थयेदादौ तेनोक्तं विधिमाचरेत्}
{एकविंशतिविप्रांश्च वस्त्रालङ्कारभूषणैः} %२६


\twolineshloka
{पूजयेद्गोहिरण्याद्यैर्हुत्वाऽग्नौ विधिपूर्वकम्}
{होमद्रव्यं मोदकाश्च तिलयुक्ता घृतप्लुताः} %२७


\twolineshloka
{अष्टोत्तरसहस्रं वा नोचेदष्टोत्तरं शतम्}
{अष्टाविंशतिसङ्ख्याकान्मोदकान्वा सशर्करान्} %२८


\twolineshloka
{अशक्तोऽष्टौ शुभान् स्थूलाञ्जुहुयाज्जातवेदसि}
{वैदिकेन च मन्त्रेण आगमोक्तेन वा तथा} %२९


\twolineshloka
{अथवा नाममन्त्रेण होमं कुर्याद्यथाविधि}
{पुष्पमण्डपिका कार्या गणेशाह्लादकारिणी} %३०


\twolineshloka
{पूजयेत्तत्र गणपं भक्तसङ्कष्टनाशनम्}
{गीतवादित्रनिनदैर्भक्तिभावपुरस्कृतैः} %३१


\twolineshloka
{पुराणवेदनिर्घोषैस्तोषयेच्च गणेश्वरम्}
{एवं जागरणं कार्यं शक्त्या दानादिकं तथा} %३२


\twolineshloka
{सपत्नीकमथाऽऽचार्यं तोषयेद्वस्त्रभूषणैः}
{उपानच्छत्रगोदानकमण्डलुफलादिभिः} %३३


\twolineshloka
{शय्यावाहनभूदानं धनधान्यगृहादिभिः}
{यथाशक्त्या प्रकर्तव्यं दारिद्र्याभावमिच्छता} %३४


\twolineshloka
{एकविंशतिविप्रांश्च भोजयेन्नामभिर्मम}
{गजास्यो विघ्नराजश्च लम्बोदरशिवात्मजौ} %३५


\twolineshloka
{वक्रतुण्डः शूर्पकर्णः कुब्जश्चैव विनायकः}
{विघ्ननाशो हि विकटो वामनः सर्वदैवतः} %३६


\twolineshloka
{सर्वार्तिनाशी भगवान् विघ्नहर्ता च धूम्रकः}
{सर्वदेवाधिदेवश्च सर्वे षोडश वै स्मृताः} %३७


\twolineshloka
{एकदन्तः कृष्णपिङ्गो भालचन्द्रो गणेश्वरः}
{गणपश्चैकविंशश्च सर्व एते गणेश्वराः} %३८

\twolineshloka
{दुर्गोपेन्द्रश्च रुद्रश्च कुलदेव्याधिकं भवेत्}
{विशेषेणाष्टसङ्ख्याकैर्मोदकैर्हवनं स्मृतम्} %३९


\twolineshloka
{एवं कृते विधानेन प्रसन्नोऽहं न संशयः}
{ददामि वाञ्छितान् कामांस्तद्व्रतं मत्प्रियं कुरु} %४०

\uvacha{श्रीकृष्ण उवाच}

\twolineshloka
{एवं तु कथितं सर्वं गणेशेन स्वयं नृप}
{पार्वत्या तत्कृतं राजन् व्रतं सङ्कष्टनाशनम्} %४१


\twolineshloka
{व्रतेनानेन सा प्राप महादेवं पतिं स्वकम्}
{तत्कुरुष्व महाराज व्रतं सङ्कष्टनाशनम्} %४२


\twolineshloka
{चतुर्थी सङ्कटा नाम स्कन्देन कथिता ऋषीन्}
{ऋषिभिर्लोककामैस्तैर्लोके ततमिदं व्रतम्} %४३

\uvacha{सूत उवाच}

\twolineshloka
{कृतं युधिष्ठिरेणैतद्राज्यकामेन वै द्विज}
{तेन शत्रून्निहत्याऽऽजौ स्वराज्यं प्राप्तवान् स्वयम्} %४४


\twolineshloka
{तस्मात्सर्वप्रयत्नेन व्रतं कार्यं विचक्षणैः}
{येन धर्मार्थकामाश्च मोक्षश्चापि भवेत्किल} %४५


\twolineshloka
{यः करोति व्रतं विप्राः सर्वकामार्थसिद्धिदम्}
{स वाञ्छितफलं प्राप्य पश्चाद्गणपतां व्रजेत्} %४६


\twolineshloka
{यदा यदा परं विप्रा नरः प्राप्नोति सङ्कटम्}
{तदा तदा प्रकर्तव्यं व्रतं सङ्कष्टनाशनम्} %४७


\twolineshloka
{त्रिपुरं हन्तुकामेन कृतं देवेन शूलिना}
{त्रैलोक्यभूतिकामेन महेन्द्रेण तथा कृतम्} %४८


\twolineshloka
{रावणेन कृतं पूर्वं वालिबन्धनसङ्कटे}
{स्वकीयं प्राप्तवान्राज्यं गणेशस्य प्रसादतः} %४९


\twolineshloka
{सीतान्वेषणकामेन कृतं वायुसुतेन च}
{सङ्कल्प्य दृष्टवान्सोऽयं सीतां रामप्रियां पुरा} %५०


\threelineshloka
{दमयन्त्या कृतं पूर्वं नलान्वेषणकारणात्}
{सा पतिं नैषधं लेभे पुण्यश्लोकं द्विजोत्तमाः}
{अहल्याऽपि पतिं लेभे गौतमं प्राणवल्लभम्} %५१

\twolineshloka
{विद्यार्थी लभते विद्यां धनार्थी धनमाप्नुयात्}
{पुत्रार्थी पुत्रमाप्नोति रोगी रोगात्प्रमुच्यते} %५२

\centerline{॥इति श्रीस्कन्दपुराणोक्तं सङ्कष्टचतुर्थीव्रतम्॥} %


