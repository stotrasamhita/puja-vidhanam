\sect{धात्रीतुलस्युत्पत्तिवर्णनम्}

\uvacha{नारद उवाच}
\twolineshloka
{क्षिप्तेभ्यस्तत्र बीजेभ्यो वनस्पत्यस्त्रयोऽभवन्}
{धात्री च मालती चैव तुलसी च नृपोत्तम} %१

\twolineshloka
{धात्र्युद्भवा स्मृता धात्री माभवा मालती स्मृता}
{गौरीभवा च तुलसी तमःसत्त्वरजोगुणाः} %२

\twolineshloka
{स्त्रीरूपिण्यौ वनस्पत्यौ दृष्ट्वा विष्णुस्तदा नृप}
{उत्तस्थौ सम्भ्रमाद्वृन्दा रूपातिशयविभ्रमः} %३

\twolineshloka
{दृष्ट्वा च याचते मोहात्कामासक्तेन चेतसा}
{तं चापि तुलसीधात्र्यौ रागेणैव व्यलोकताम्} %४

\twolineshloka
{यच्च लक्ष्म्या पुरा बीजमीर्ष्ययैव समर्पितम्}
{तस्मात्तदुद्भवा नारी तस्मिन्नीर्ष्यापराऽभवत्} %५

\twolineshloka
{अतः सा बर्बरीत्याख्यामवापाध विगर्हिताम्}
{धात्रीतुलस्यौ तद्रागात्तस्य प्रीतिप्रदे सदा} %६

\twolineshloka
{ततो विस्मृतदुःखोऽसौ विष्णुस्ताभ्यां सहैव तु}
{वैकुण्ठमगमद् धृष्टः सर्वदेवनमस्कृतः} %७

\twolineshloka
{कार्तिकोद्यापने विष्णोस्तस्मात्पूजा विधीयते}
{तुलसीमूलदेशेऽस्य प्रीतिदा सा यतः स्मृता} %८

\twolineshloka
{तुलसीकाननं राजन्गृहे तस्यावतिष्ठते}
{तद्गृहं तीर्थरूपं तु नाऽऽयान्ति यमकिङ्कराः} %९

\twolineshloka
{सर्वपापहरं नित्यं कामदं तुलसीवनम्}
{रोपयन्ति नराः श्रेष्ठास्ते न पश्यन्ति भास्करिम्} %१०

\twolineshloka
{दर्शनं नर्मदायास्तु गङ्गास्नानं तथैव च}
{तुलसीवनसंसर्गः सममेव त्रयं स्मृतम्} %११

\twolineshloka
{रोपणात्पालनात्सेकाद्दर्शनात्स्पर्शनान्नृणाम्}
{तुलसी दहते पापं वाङ्मनःकायसञ्चितम्} %१२

\twolineshloka
{तुलसीमञ्जरीभिर्यः कुर्याद्धरिहरार्चनम्}
{न स गर्भगृहं याति मुक्तिभागी न संशयः} %१३

\twolineshloka
{पुष्कराद्यानि तीर्थानि गङ्गाद्याः सरितस्तथा}
{वासुदेवादयो देवास्तिष्ठन्ति तुलसीदले} %१४

\twolineshloka
{तुलसीमञ्जरीयुक्तो यस्तु प्राणान्विमुञ्चति}
{यमोऽपि नेक्षितुं शक्तो युक्तं पापशतैरपि} %१५

\twolineshloka
{विष्णोः सायुज्यमाप्नोति सत्यं सत्यं नृपोत्तम}
{तुलसीकाष्ठजं यस्तु चन्दनं धारयेन्नरः} %१६

\twolineshloka
{तद्देहं न स्पृशेत्पापं क्रियमाणमपीह यत्}
{तुलसीविपिनच्छाया यत्र यत्र भवेन्नृप} %१७

\twolineshloka
{तत्र श्राद्धं प्रकर्तव्यं पितॄणां दत्तमक्षयम्}
{धात्रीफलविमिश्रैश्च तुलसीपत्रमिश्रितैः} %१८

\twolineshloka
{जलैः स्नाति नरस्तस्य गङ्गास्नानफलं स्मृतम्}
{देवार्चनं नरः कुर्याद्धात्रीपत्रैः फलैस्तथा} %१९

\twolineshloka
{सुवर्णमणिमुक्तौघैरर्चनस्याऽऽप्नुयात्फलम्}
{तीर्थानि मुनयो देवा यज्ञाः सर्वेऽपि कार्तिके} %२०

\twolineshloka
{नित्यं धात्रीं समाश्रित्य तिष्ठन्त्यर्के तुलास्थिते}
{द्वादश्यां तुलसीपत्रं धात्रीपत्रं तु कार्तिके} %२१

\threelineshloka
{लुनाति स नरो गच्छेन्निरयानतिगर्हितान्}
{धात्रीतुलस्योर्माहात्म्यमपि देवश्चतुर्मुखः}
{न समर्थो भवेद्वक्तुं यथा देवस्य शार्ङ्गिणः} %२२

\fourlineindentedshloka
{धात्रीतुलस्युद्भवकारणं यः}
{शृणोति यः श्रावयते च भक्त्या}
{विधूतपाप्मा सह पूर्वजैः स्वैः}
{स्वर्गं व्रजत्यग्र्यविमानसंस्थैः} %२३


॥इति श्रीस्कान्दे महापुराण एकाशीतिसाहस्र्यां संहितायां\\ द्वितीये वैष्णवखण्डे कार्तिकमासमाहात्म्ये\\धात्रीतुलस्युत्पत्तिवर्णनं नाम त्रयोविंशोऽध्यायः॥२३॥