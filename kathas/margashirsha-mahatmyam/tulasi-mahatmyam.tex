\sect{तुलसी-माहात्म्यम्}

\uvacha{ब्रह्मोवाच}

\twolineshloka
{श्रीमत्तुलसिमाहात्म्यं यथावद्वर्णय प्रभो}
{यस्याः सन्निधिमात्रेण प्रीतिर्भवति तेऽधिका} %॥१॥

\uvacha{श्रीभगवानुवाच}

\twolineshloka
{मणिकांचनपुष्पाणि तथा मुक्तामयानि च}
{तुलसीपत्रदानस्य कलां नार्हंति षोडशीम्} %॥२॥

\twolineshloka
{तुलसीमंजरीभिर्यः कुर्याद्वै मम पूजनम्}
{न स गर्भगृहं यायान्मुक्तिभागी भवेन्नरः} %॥३॥

\twolineshloka
{आरोप्य तुलसीं वत्स पूजयेत्तद्दलैश्च माम्}
{दिवि संमोदमानः स श्वेतद्वीपे च मे गृहे} %॥४॥

\fourlineindentedshloka
{श्रीमत्तुलस्यार्चयते सकृद्धि मां}
{पत्रैः सुगन्धैर्विमलैरखंडितैः}
{यस्तस्य पापं पटसंस्थितं तदा}
{निरीक्षयित्वा परिमार्जयेद्यमः} %॥५॥

\fourlineindentedshloka
{तुलसी न येषां मम पूजनार्थं}
{संपादितैकादशिपुण्यवासरे}
{धिग्यौवनं जीवितमर्थसन्ततिस्-}
{तेषां सुखं नेह च दृश्यते परे} %॥६॥

\twolineshloka
{लिंगमभ्यर्चितं दृष्ट्वा सहोमासे च मामकम्}
{तुलसीपत्रनिकरैर्मुच्यते ब्रह्महत्यया} %॥७॥

नित्यमभ्यर्चयेद्यो वै तुलस्या मां रमेश्वरम् ॥
महापापानि नश्यंति किं पुनश्चोपपातकम् ॥८॥

\twolineshloka
{वर्ज्यं पर्युषितं पुष्पं वर्ज्यं पर्युषितं जलम्}
{न वर्ज्यं तुलसीपत्रं न वर्ज्यं जाह्नवीजलम्} %॥९॥

\twolineshloka
{तावद्गर्जंति पुष्पाणि मालत्यादीनि भोः सुत}
{यावन्न प्राप्यते पुण्या तुलसी मम वल्लभा} %॥2॥

\twolineshloka
{सकृदभ्यर्चयेद्यो मां बिल्वपत्रेण मानवः}
{मुक्तिभागी निरातंको मम पार्श्वगतो भवेत्} %॥११॥

\twolineshloka
{बिल्वपत्राच्छमीपत्राज्जातीपत्रात्सरोरुहात्}
{वल्लभं तुलसीपत्रं कौस्तुभादधिकं मम} %॥१२॥

\twolineshloka
{अभिन्नपत्रा तुलसी हृद्या मंजरिसंयुता}
{क्षीरोदार्णवसंभूता पद्मेवेयं सदा मम} %॥१३॥

\twolineshloka
{अकृष्णाऽप्यथवा कृष्णा तुलसी मम वल्लभा}
{सिता वाऽप्यसिता वापि द्वादशी वल्लभा यथा} %॥१४॥

\twolineshloka
{गृहीत्वा तुलसीपत्रं भक्त्या यो मां समर्चयेत्}
{अर्चितं तेन सकलं सदेवासुरमानुषम्} %॥१५

\twolineshloka
{तावद्गर्जंति रत्नानि कौस्तुभादीन्यनन्तशः}
{यावन्न प्राप्यते कृष्ण तुलसीकृष्णमंजरी} %॥१६॥

\twolineshloka
{कृष्णं कृष्णतुलस्या हि यो भक्त्या पूजयेन्नरः}
{स याति भुवनं शुभ्रं यत्र विष्णुः श्रिया सह} %॥१७॥

\twolineshloka
{ममाऽर्चनार्थं भिक्षूणां यच्छंति तुलसीदलम्}
{अन्येषामपि भक्तानां यांति ते पदमव्ययम्} %॥१८॥

\twolineshloka
{तुलसी कृष्णगौरा या तया यो मां समर्चयेत्}
{नरो याति तनुं त्यक्त्वा वैष्णवीं शाश्वतीं गतिम्} %॥१९॥

[इतः पश्चात् दीपमाहात्म्यवर्णनम्]

॥इति श्रीस्कान्दे महापुराण एकाशीतिसाहस्र्यां संहितायां द्वितीये वैष्णवखण्डे ब्रह्मविष्णुसंवादे मार्गशीर्षमाहात्म्ये दीपमाहात्म्यवर्णनं नाम अष्टमोऽध्यायः॥८॥

