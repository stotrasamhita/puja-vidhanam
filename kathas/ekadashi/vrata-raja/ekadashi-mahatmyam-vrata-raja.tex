\sect{मार्गशीर्ष-कृष्ण-एकादशी-माहात्म्यम्}
\label{sec:vrata-raja-margashirsha-krishna}

\uvacha{सूत उवाच}

\twolineshloka
{एवं प्रीत्या पुरा विप्राः श्रीकृष्णेन परं व्रतम्}
{माहात्म्यविधिसंयुक्तमुपदिष्टं विशेषतः} %॥१॥

\twolineshloka
{उत्पत्तिं यः शृणोत्येवमेकादश्यां द्विजोत्तम}
{भुक्त्वा भोगाननकांस्तु विष्णुलोकं प्रयाति सः} %॥२॥

\uvacha{पार्थ उवाच}

\twolineshloka
{उपवासस्य नक्तस्य एकभक्तस्य च प्रभो}
{किं पुण्यं किं विधानं हि बेहि सर्व जनार्दन} %॥३॥

\uvacha{श्रीकृष्ण उवाच}

\twolineshloka
{हेमन्ते चैव सम्प्राप्ते मासि मार्गशिरे शुभे}
{शुक्लपक्षे तथा पार्थ एकादश्या मुपोषयेत्} %॥४॥

\twolineshloka
{नक्तं दशम्यां कुर्यात्तु दन्तधावनपूर्वकम्}
{दिवसस्याष्टमे भागे अन्दीभूते दिवाकरे} %॥५॥

\twolineshloka
{तत्र नक्तं विजानीयान नक्तं निशिभोजनम्}
{ततः प्रभातसमये लइल्प नियनश्चरेत्} %॥६॥

\twolineshloka
{मध्याहे च तथा पार्थ शुचिः स्नानः समानः}
{नद्यां नडागे वाप्यां वा घुत्तमं मध्यम त्वधः} %॥७॥

\twolineshloka
{क्रमाज्ज्ञेयं तथा कूपे तदनावे प्रशस्यत}
{अश्वक्रान्ने रथक्रान्तं विष्णुकान्ते वसुन्धरे} %॥८॥

\twolineshloka
{मृत्तिके हर मे पापं यन्मया पूर्वसञ्चितम्}
{क्या हतेन पापेन गच्छामि परमां गतिम्} %॥९॥

\twolineshloka
{अनेन मृत्तिकास्नान विदध्यातु व्रती नरः}
{नालपेत्पतितैश्चोरैस्तथा पाखण्डिभिः सह} %॥१०॥

\twolineshloka
{मिथ्यापवादिनो देववेदब्राह्मणनिन्दकान्}
{अन्यांश्चैव दुराचारानगम्यागामिनस्तथा} %॥११॥

\twolineshloka
{परद्रव्यापहर्तृश्च देवद्रव्यापहारिणः}
{न सम्भाषेत दृष्ट्वाऽपि भास्करं चावलोकयेत्} %॥१२॥

\twolineshloka
{ततो गोविन्दमभ्यर्च्य नैवेद्यादिभिरादरात्}
{दीपं दद्याद्गृहे चैव भक्तियुक्तेन चेतसा} %॥१३॥

\twolineshloka
{तदिने वर्जयेत्पार्थ निद्रां मैथुनमेव च}
{गीतशास्त्रविनोदेन दिवारानं नयेद्रती} %॥१४॥

\twolineshloka
{रात्रौ जागरणं कृत्वा भक्तियुक्तेन चेतसा}
{विप्रेभ्यो दक्षिणां दत्त्वा प्रणिपत्य क्षमापयेत्} %॥५॥

\twolineshloka
{यथा शुक्ला तथा कृष्णा मान्या वै धर्मतत्परैः}
{एकादश्योयो राजन्विभेदं नैव कारयत्} %॥१६॥

\twolineshloka
{एवं हि कुरुते यस्तु शृणु तस्यापि यत्फलम्}
{शङ्खोद्धारे नरः स्नात्वा दृष्ट्वा देवं गदाधरम्} %॥१७॥

\twolineshloka
{एकादश्युपवासस्य कलां नाप्नोति षोडशीम्}
{व्यतीपाते च दानस्य लक्षमें फलं स्मृतम्} %॥१८॥

\twolineshloka
{सङ्क्रान्ति चतुर्लक्षं दानस्य च धनञ्जय}
{कुरुक्षेत्रे च यत्पुण्यं ग्रहणे चन्द्रसूर्ययोः} %॥१९॥

\twolineshloka
{तत्सर्व लभते यस्तु टङ्कादश्यामुपोषितः}
{अश्वमेधस्य यज्ञस्य करणाद्यत्फलं लभत्} %॥२०॥

\twolineshloka
{ततः शतगुणं पुण्यमेकादश्युपवासतः}
{तपस्विनो गृहे नित्यं लक्ष यस्य च भुञ्जते} %॥२१॥

\twolineshloka
{षष्टिवर्षसहस्राणि तस्य पुण्यं च यद्भवेत्}
{एकादश्युपवासेन फलं प्राप्नोति मानवः} %॥२२॥

\twolineshloka
{गोसहस्रे च यत्पुण्यं दत्ते वेदाङ्गपारगे}
{तस्मात्पुण्यं दशगुणमेकादश्युपवासिनाम्} %॥२३॥

\twolineshloka
{नित्यं च भुञ्जते यस्य गृहे दश द्विजोत्तमाः}
{यत्पुण्यं तद्दशगुणं भोजने ब्रह्मचारिणः} %॥२४॥

\twolineshloka
{एतत्सहस्रं भूदाने कन्यादाने तु तत्स्मृतम्}
{तस्मादशगुणं प्रोक्त विद्यमाने तथैव च} %॥२५॥

\twolineshloka
{विद्यादशगुणं चान्नं यो ददाति बुभुक्षिते}
{अन्नदानसमं दानं न भूतं न भविष्यति} %॥२६॥

\twolineshloka
{तृप्तिमायान्ति कौन्तेय स्वर्गस्थाः पितृदेवताः}
{एकादश्या व्रतस्यापि पुण्यसङ्ख्या न विद्यते} %॥२७॥

\twolineshloka
{एतत्पुण्यप्रभावश्च यत्सुरैरपि दुर्लभः}
{नक्तस्याद्रफलं तस्य एकभक्तस्य सत्तम} %॥२८॥

\twolineshloka
{एकभक्तं च नक्तं च उपवासस्तथैव च}
{एतेष्वन्यतमं वाऽपि व्रतं कुर्याद्धरोर्दन} %॥२९॥

\twolineshloka
{तावद्गर्जन्ति तीर्थानि दानानि नियमा यमाः}
{एकादशी न सम्प्राप्ता यावत्तावन्मखा अपि} %॥३०॥

\twolineshloka
{तस्मादेकादशी सर्वैरुपोष्या भवभीरुभिः}
{न शङ्खन पिवेत्तोयं न खादेन्मत्स्यसूकरौ} %॥३१॥

\twolineshloka
{एकादश्यां न भुञ्जीत यन्मां त्वं पृच्छसेऽर्जुन}
{एतत्ते कथितं सर्व व्रतानामुत्तमं व्रतम्} %॥३२॥

\onelineshloka*
{एकादशीसमं नास्ति कृत्वा यज्ञसहस्रकम्}

\uvacha{अर्जुन उवाच}

\onelineshloka
{उक्ता त्वया कथं देव पुण्येयं सर्वतस्तिथिः} %॥३३॥

\onelineshloka*
{सर्वभ्योऽपि पवित्रेयं कथं होकादशी तिथिः}

\uvacha{श्रीकृष्ण उवाच}

\onelineshloka
{पुरा कृतयुगे पार्थ मुरनामा हि दानवः} %॥३४॥

\twolineshloka
{अत्यद्भुतो महारौद्रः सर्वदेवभयङ्करः}
{इन्द्रो विनिर्जितस्तेन ह्याद्यो देवः पुरन्दरः} %॥३५॥

\twolineshloka
{आदित्या वसवो ब्रह्मा वायुरग्निस्तथैव च}
{देवता निर्जितास्तेन अत्युप्रेण च पाण्डव} %॥३६॥

\twolineshloka
{इन्द्रेण कथितः सर्वो वृत्तान्तः शङ्कराय वै}
{स्वर्गलोकपरिभ्रष्टा विचरामो महीतले} %॥३७॥

\onelineshloka*
{उपायं ब्रूहि मे देव अमराणां तु का गतिः}

\uvacha{ईश्वर उवाच}

\onelineshloka
{गच्छ गच्छ सुरश्रेष्ठ यत्राम्नि गरुडध्वजः} %॥३८॥

\twolineshloka
{शरण्यश्च जगन्नाथः परित्राणपरायणः}
{ईशस्य वचनं श्रुत्वा देवराजो महामनाः} %॥३९॥

\twolineshloka
{त्रिदशैः सहितः सर्वैर्गतस्तत्र धनञ्जय}
{यत्र देवो जगन्नाथः प्रसुप्तो हि जनार्दनः} %॥४०॥

\twolineshloka
{जलमध्ये प्रसुप्तं तु दृष्ट्वा देवं जगत्पतिम्}
{कृताञ्जलिपुटो भूत्वा इदं स्तोत्रमुदीरयत} %॥४१॥

\twolineshloka
{ओं नमो देवदेवाय देवदेवः सुवन्दित}
{दैत्यारे पुण्डरीकाक्ष त्राहि नो मधुसूदन} %॥४२॥

\twolineshloka
{दैत्यभीता इमे देवा मया सह समागताः}
{शरणं त्वं जगन्नाथ त्वं कर्ता त्वं च कारकः} %॥४३॥

\twolineshloka
{त्वं माता सर्वलोकानां त्वमेव जगतः पिता}
{त्वं स्थितिस्त्वं तथोत्पत्तिस्त्वं च संहारकारकः} %॥४४॥

\twolineshloka
{सहायस्त्वं च देवानां त्वं च शान्तिकरः प्रभो}
{त्वं धरा च त्वमाकाशः सर्वविश्वोपकारकः} %॥४५॥

\twolineshloka
{भवस्त्वं च स्वयं ब्रह्मा त्रैलोक्यप्रतिपालक}
{त्वं रविस्त्वं शशाङ्कश्च त्वं च देवो हुताशनः} %॥४६॥

\twolineshloka
{हव्यं होमो हुतस्त्वं च मन्त्रतन्त्रविजो जपः}
{यजमानश्च यज्ञस्त्वं फलभोक्ता त्वमीश्वरः} %॥४७॥

\twolineshloka
{न त्वया रहितं किञ्चित्रलोक्ये सचराचरे}
{भगवन्देवदेवेश शरणागतवत्सल} %॥४८॥

\twolineshloka
{त्राहि त्राहि महायोगिन्भीतानां शरणं भव}
{दानवैर्विजिता देवाः स्वर्गभ्रष्टाः कृता विभो} %॥४९॥

\twolineshloka
{स्थानभ्रष्टा जगन्नाथ विचरन्ति महीतले}
{इन्द्रस्य वचनं श्रुत्वा विष्णुर्वचनमब्रवीत्} %॥५०॥

\uvacha{श्रीभगवानुवाच}

\twolineshloka
{कोऽसौ दैत्यौ महा यो देवा येन विनिर्जिताः}
{किं स्थानं तस्य किं नाम किं बलं कस्तदाश्रयः} %॥५१॥

\onelineshloka*
{एतत्सर्व समाचक्ष्व मघवन्निर्भयो भव}

\uvacha{इन्द्र उवाच}

\onelineshloka
{भगवन्देवदेवेश भक्तानुग्रहकारक} %॥५२॥

\twolineshloka
{दैत्यः पूर्व महानासीन्नाडीजव इति स्मृतः}
{ब्रह्मवंशसमुद्भूतो महोग्रः सुरसूदनः} %॥५३॥

\twolineshloka
{तस्य पुत्रोऽतिविख्यातो मुरनामा महासुरः}
{तस्य चन्द्रवतीनाम नगरी च गरीयसी} %॥५४॥

\twolineshloka
{तस्यां वसन्स दुष्टात्मा विश्व निर्जित्य वीर्यवान्}
{तुरान्स्ववशमानिन्ये निराकृत्य त्रिविष्टपात्} %॥५५॥

\twolineshloka
{इन्द्राग्नियमवाय्वीशसोमानिऋतिपाशिनाम्}
{पदेषु स्वयमेवासीत्सूर्यो भूत्वा तपत्यपि} %॥५६॥

\twolineshloka
{पर्जन्यः स्वयमेवासीदजेयः सर्वदेवतैः}
{जहि तं दानवं विष्णो सुराणां जयमावह} %॥५७॥

\twolineshloka
{तस्य तद्वचनं श्रुत्वा कोपाविष्टो जनादनः}
{उवाच शत्रु देवेन्द्र हनिष्ये तं महाबलम्} %॥५८॥

\twolineshloka
{प्रयान्तु सहिताः सर्वे चन्द्रवत्यां महाबलाः}
{इत्युक्ताः प्रययुः सर्वे पुरस्कृत्य हरिं सुराः} %॥५९॥

\twolineshloka
{दृष्टो देवैस्तु दैत्येन्द्रो गर्जमानस्तु दानवैः}
{असङ्ख्यातसहस्रेस्तु दिव्यप्रहरणायुधैः} %॥६०॥

\twolineshloka
{हन्यमानास्तदा देवा असुरैर्बाहुशालिभिः}
{सङ्ग्रामं ते समुत्सृज्य पलायन्त दिशो दश} %॥६१॥

\twolineshloka
{ततो दृष्ट्वा हृषीकेशं सङ्ग्रामे समुपस्थितम्}
{अन्वधावन्नाभिकुद्धा विविधायुधपाणयः} %॥६२॥

\twolineshloka
{अथ तान्प्रगुतान्दृष्ट्वा शङ्खचक्रगदाधरः}
{विव्याध सर्वगात्रेषु शरैराशीविषोपमै} %॥६३॥

\twolineshloka
{तेनाहतास्ते शनशो दानवा निधनं गताः}
{एकाङ्गो दानवः स्थित्वा युध्यमानो मुहुर्मुहुः} %॥६४॥

\twolineshloka
{तस्योपरि हृषीकेशो यद्यदायुधमुत्सृजत्}
{पुष्पवत्तत्समभ्येति कुण्ठितं तस्य तेजसा} %॥६५॥

\twolineshloka
{शस्त्रास्त्रविध्यमानोऽपि यदा जेतुं न शक्यते}
{युयोध च तदा क्रुद्धो बाहुभिः परिघोपमैः} %॥६६॥

\twolineshloka
{वाहुयुद्धं कृतं तेन दिव्यवर्ष लहाकन्द}
{तेन श्रान्तः स भगवान् गतो बदरिकाश्रमम्} %॥६७॥

\twolineshloka
{तत्र हैमवती नाम्नी गुहा परमशोजना}
{तां प्राविशन्महायोगी शयनार्थ जगत्पतिः} %॥६८॥

\twolineshloka
{योजनबादशायामः पत्रद्वारा धनञ्जय}
{अहं तत्र प्रसुप्तोस्मि भयभीतो न संशयः} %॥६९॥

\twolineshloka
{महायुद्धेन तेनैव श्रान्तोऽहं गाडुनन्दन}
{दानवः पृष्ठतो लग्नः प्रविवेश स तां गुहाम्} %॥७०॥

\twolineshloka
{प्रसुप्तं मां तदा दृष्ट्वाऽचिन्नयदानो हृदि}
{हरिमेनं हनिष्येऽहं दानवानां क्षयावहम्} %॥७१॥

\twolineshloka
{एवं सुदुनेम्नाय व्यवसायं व्यवस्य च}
{समुद्भूता ममाङ्गेभ्यः कन्यैका च महाप्रभा} %॥७२॥

\twolineshloka
{दिव्यप्रहरणा देवी युद्धाय समुपस्थिता}
{मुरेण दानवेन्द्रेण ईक्षिता पाण्डुनन्दन} %॥७३॥

\twolineshloka
{युद्धं समीरितं तेन स्त्रिया तत्र प्रयाचितम्}
{तेनायुध्यत सा नित्यं तां दृष्ट्वा विस्मयं गतः} %॥७४॥

\twolineshloka
{केनेयं निर्मिता रौद्रा अत्युप्राशनिपातिनी}
{इत्युक्त्वा दानवेन्द्रोऽसौ युयुधे कन्यया तया} %॥७५॥

\twolineshloka
{ततस्तया महादेव्या त्वरया दानवो बली}
{छित्त्वा सर्वाणि शस्त्राणि क्षणेन विरथः कृतः} %॥७६॥

\twolineshloka
{बाहुप्रहरणोपेतो धावमानो महाबलात्}
{तलेनाहत्यहृदये तया देव्या निपातितः} %॥७॥

\twolineshloka
{पुनरुत्थाय सोचावकन्याह ननकाङ्क्षया}
{दानवं पुनरायान्तं रोषणाहत्य तच्छिरः} %॥७८॥

\twolineshloka
{क्षणानिपातयामास भूमौ तच्च समुज्ज्वलत्}
{दैत्यः कृत्तशिराः सोथ ययौ वैवस्वतालयम्} %॥७९॥

\twolineshloka
{शेषा भयादिना दीनाः पातालं विविशुद्धिषः}
{तत्तः समुत्थितो देवः पुरो दृष्ट्वाऽसुरं हतम्} %॥८०॥

\twolineshloka
{कन्यां पुरः स्थिता चापि कृताञ्जलिपुटां नताम्}
{विस्मयोत्फुल्लनयनः प्रोवाच जगतां पतिः} %॥८१॥

\twolineshloka
{केनायं निहतः सङ्ख्ये दानवो दुष्टमानसः}
{येन देवाः सगन्धर्वाः सेन्द्राश्च समस्हणाः} %॥८२॥

\twolineshloka
{सनागाः सहलोकेशा लीलयैव विनिर्जिताः}
{येनाहं निर्जितो भीतः श्रान्तः सुप्तो गुहामिमाम्} %॥८३॥

\onelineshloka*
{केन कारुण्यभावेन राक्षतोऽहं पलायितः}

\uvacha{कन्योवाच}

\onelineshloka
{मया विनिहतो दैत्यस्त्वदंशोद्भूतया प्रभो} %॥८४॥

\twolineshloka
{दृष्ट्वा सुप्तं हरे त्वां तु दैत्यो हन्तुं समुद्यतः}
{त्रैलोक्यकण्टकस्येत्यं व्यवसाय प्रबुध्य च} %॥८५॥

\twolineshloka
{हतो मया दुरात्माऽसौ देवता निर्भयाः कृताः}
{तवैवाहं महाशक्तिः सर्वशत्रुभयङ्करी} %॥८६॥

\twolineshloka
{त्रैलोक्यरक्षणार्थाय हतो लोकभयङ्करः}
{निहतं दानवं दृष्ट्वा किमाश्चर्य वद प्रभो} %॥८७॥

\uvacha{श्रीभगवानुवाच}

\twolineshloka
{निहते दानवेन्द्रेऽस्मिन्सन्तुष्टोऽहं त्वयानघे}
{हृष्टाः पुष्टाश्च वै देवा आनन्दः समजायत} %॥८८॥

\twolineshloka
{आनन्दस्त्रिषु लोकेषु देवानां यस्त्वया कृतः}
{प्रसन्नोस्म्यनचे तुभ्यं वरं वरय सुव्रते} %॥८९॥

\onelineshloka*
{ददामि तन्न सन्देहो यत्सुरैरपि दुर्लभम्}

\uvacha{कन्योवाच}

\onelineshloka
{यदि तुष्टोऽसि मे देव यदि दयो वरो मम} %॥९०॥

\twolineshloka
{तारयेहं महापापादुपवासपरं नरम्}
{उपवालस्य यत्पुण्यं तस्यार्ध नक्तभोजने} %॥९१॥

\twolineshloka
{तदर्ध च भवेत्तस्य एकभुक्तं करोति यः}
{यः करोति व्रतं भक्त्या दिने मम जितेन्द्रियः} %॥९२॥

\twolineshloka
{स गत्वा वैष्णवं स्थानं कल्पकोटिशतानि च}
{भुञ्जानो विविधान्भोगानुपवासी जितेन्द्रियः} %॥९३॥

\twolineshloka
{भगवंस्त्वत्प्रसादेन भवत्वेष वरो मम}
{उपवासं च नक्तं च एकभुक्तं करोति यः} %॥९४॥

\onelineshloka*
{तस्य धर्म च वित्तं च मोक्षं देहि जनार्दन}

\uvacha{श्रीभगवानुवाच}

\onelineshloka
{यत्त्वं वदसि कल्याणि तत्सर्वं च भविष्यति} %॥९५॥

\twolineshloka
{मम भक्ताश्च ये लोकास्तव भक्ताश्च ये नराः}
{त्रिषु लोकेषु विख्याताः प्राप्स्यन्ति मम सनिधिम्} %॥९६॥

\twolineshloka
{एकादश्यां समुत्पन्ना मम शक्तिः परा यतः}
{अत एकादशीत्येवं तव नाम भविष्यति} %॥९॥

\twolineshloka
{दग्ध्वा पापानि सर्वाणि दास्यामि पदमव्ययम्}
{तृतीया चाष्टमी चैव नवमी च चतुर्दशी} %॥९८॥

\twolineshloka
{एकादशी विशेषेण तिथयो मे महाप्रियाः}
{सर्वतीर्थाधिकं पुण्यं सर्वदानाधिकं फलम्} %॥९९॥

\twolineshloka
{सर्वव्रताधिक चैव सत्यं सत्यं वदामि ते}
{एवं दत्त्वा वरं तस्यास्तत्रैवान्तरधीयत} %॥१००॥

\twolineshloka
{दृष्टा तुष्टा तु सा जाता तदा एकादशीतिथिः}
{इमामेकादशी पार्थ करिष्यन्ति नरास्तु ये} %॥१॥

\twolineshloka
{तेषां शत्रु हनिष्यामि दास्यामि परमां गतिम्}
{अन्येऽपि ये करिष्यन्ति एकादश्या महाव्रतम्} %॥२॥

\twolineshloka
{हरामि तेषां विनांश्च सर्वसिद्धिं ददामि च}
{एवमुक्का समुत्पत्तिरेकादश्याः पृथासुत} %॥३॥

\twolineshloka
{इयमेकादशी नित्या सर्वपापक्षयङ्करी}
{एकैव च महापुण्या सर्वपापनिषूदनी} %॥४॥

\twolineshloka
{उदिता सर्वलोकेषु सर्वसिद्धिकरी तिथिः}
{शुक्का वाप्यथवा कृष्णा इति भेदं न कारयेत्} %॥५॥

\twolineshloka
{कर्तव्ये तु उभे पार्थ न तुल्या द्वादशीतिथिः}
{अन्तरं नैव कर्तव्यं समस्तैर्वतकारिभिः} %॥६॥

\twolineshloka
{तिथिरेका भवेत्सर्वा पक्षयोरुभयोरपि}
{ते यान्ति परमं स्थानं यत्राले गरुडध्वजः} %॥७॥

\twolineshloka
{धन्यान्ते मानवा लोके विष्णुभक्तिपरायणाः}
{एकादश्यास्तु माहात्म्यं सर्वकालं तु यः पठेत्} %॥८॥

\twolineshloka
{अश्वमेधस्य यत्पुण्यं तदाप्नोति न संशय}
{यः शृणोनि दिवारात्रौ नरो विष्णुपरायणः} %॥९॥

\twolineshloka
{तद्भक्तमुखनिष्पन्नां कथां विष्णोः सुमङ्गलाम्}
{कुलकोटिसमायुक्तो विष्णुलोके महीयते} %॥१०॥

\twolineshloka
{एकादश्याश्च माहात्म्यं पादमेकं शृणोति यः}
{ब्रह्महत्यादिकं पापं नदयते नात्र संशयः} %॥११॥

\twolineshloka
{विष्णुधर्मः समो नास्ति गीतार्थेन धनञ्जय}
{एकादशी समं नास्ति व्रतं नाम सनातनम्} %॥११२॥

॥इति श्रीकृष्णार्जुनसंवादे मार्गशीर्षकृष्णैकादशीमाहात्म्यं सम्पूर्णम्॥


\hyperref[sec:ekadashi_mahatmyam_vrata_raja]{\closesub}
\clearpage

\sect{मार्गशीर्ष-शुक्ल-मोक्षदा-एकादशी-माहात्म्यम्}
\label{sec:vrata-raja-margashirsha-shukla-mokshada}

\uvacha{युधिष्ठिर उवाच}

\twolineshloka
{वन्दे विष्णु प्रभु साक्षाल्लोकत्रयसुखप्रदम्}
{विश्वेशं विश्वकर्तारं पुराणं पुरुषोत्तमम्} %॥१॥

\twolineshloka
{पृच्छामि देवदेवेश संशयोऽस्ति महान्मम}
{लोकानां तु हितार्थाय पापानां च क्षयाय च} %॥२॥

\twolineshloka
{मार्गशीर्षे सिते पक्षे किन्नामैकादशी भवेत्}
{कीदृशश्च विधिस्तस्याः को देवस्तत्र पूज्यते} %॥३॥

\onelineshloka*
{एतदाबश्व मे स्वामिविस्तरेण यथातथम्}

\uvacha{श्रीकृष्ण उवाच}

\onelineshloka
{सम्यक् पृष्टं त्वया राजन् साधु ते विमला मतिः} %॥४॥

\twolineshloka
{कथयिष्यामि राजेन्द्र हरिवासरमुत्तमम्}
{उत्पन्ना सा सिते पक्षे द्वादशी मम वल्लभा} %॥५॥

\twolineshloka
{मार्गशीर्षे समुत्पन्ना मम देहान्नराधिप}
{मुरस्य च वधार्थाय प्रख्याता मम वल्लभा} %॥६॥

\twolineshloka
{कथिना सा मया चैव त्वद्ने राजसत्तम}
{पूर्वमेकादशी राजन् त्रैलोक्ये सचराचरे} %॥७॥

\twolineshloka
{मार्गशीर्षेसिते पक्षे चोत्पत्तिरिति नामतः}
{अतः पर प्रवक्ष्यामि मार्गशीर्षलितां तथा} %॥८॥

\twolineshloka
{मोक्षाना-नातिविख्यातां सर्वपापहरां पराम्}
{देव दामोदरं तस्या पूजयेञ्च प्रयत्नतः} %॥९॥

\twolineshloka
{गन्धपुष्पादिभिश्चैव गीतनृत्यः सुमङ्गलैः}
{शृणु राजेन्द्र वक्ष्यामि कथा पौराणिकी शुभाम्} %॥१०॥

\twolineshloka
{यस्याः श्रवणमात्रेण वाजपेयफलं लभेत्}
{अधोगतिं गता ये व पितृमातृसुतादयः} %॥११॥

\twolineshloka
{अस्याः पुण्यप्रभावेण स्वर्ग यान्ति न संशयः}
{एतत्मात्कारणाद्राजन्महिमान शृणुष्व तम्} %॥१२॥

\twolineshloka
{पुरा वै नगरे रम्ये गोकुलै न्यबसन्नृपः}
{वैखानसेति राजर्षिः पुत्रवत्पालयन्प्रजाः} %॥१३॥

\twolineshloka
{द्विजाश्च न्यवसंस्तत्र चतुर्वेदपरायणाः}
{एवं स राज्यं कुर्वाणो रात्रौ तु स्वप्नमध्यतः} %॥१४॥

\twolineshloka
{ददर्श जनकं स्वं तु अधोयोनिगतं नृपः}
{एवं दृष्ट्वा तु तं तन्त्र विस्मयो फुललोचनः} %॥१५॥

\onelineshloka*
{कथयामास वृत्तान्तं द्विजाने स्वप्नसम्भवम्}

\uvacha{राजो वाच}

\onelineshloka
{मया तु स्वपिता दृष्टो नरके पतितो द्विजाः} %॥१६॥

\twolineshloka
{तारयस्वेति मां तात अधोयोनिगतं सुत}
{इति ब्रुवाणः स तदा मया दृष्टः पिता स्वयम्} %॥१८॥

\twolineshloka
{तदाप्रभृति भो विप्रा नाहं शर्म लभाम्यहो}
{एतद्राज्यं मम महदसह्यमसुखं तथा} %॥१८॥

\twolineshloka
{अश्वा गजा रथाश्चैव न मां रोचन्ति सर्वथा}
{न कोशोऽपि सुखायालं न किञ्चित्सुखदं मम} %॥१९॥

\twolineshloka
{न दारा न सुता मह्यं रोचन्ते द्विसत्तम}
{किं करोमि व गच्छामि शरीरं मे तु दह्यते} %॥२०॥

\twolineshloka
{दानं व्रतं तपो योगो येनैव मम पूर्वजाः}
{मोक्षमायान्ति विप्रेन्द्रास्तदेव कययन्तु मे} %॥२१॥

\twolineshloka
{किं तेन जीवता लोके सुपुत्रेण बलीयला}
{पिता तु यस्य नरके तस्य जन्म निरर्थकम्} %॥२२॥

\uvacha{ब्राह्मणा ऊचुः}

\twolineshloka
{पर्वतस्य मुनेरत्र आश्रमो निकटे नृप}
{गम्यतां राजशार्दूल भूतं भव्यं विजानतः} %॥२३॥

\twolineshloka
{तेषां श्रुत्वा ततो वाक्यं विषण्णो राजसत्तमः}
{जगाम नत्र यत्रासौ आश्रमे पर्वतो मुनिः} %॥२४॥

\twolineshloka
{ब्राह्मणैर्वेष्टितः शान्तैः प्रजाभिश्च समन्ततः}
{आश्रमो विपुलस्तस्य मुनिभिः सन्निषेवितः} %॥२५॥

\twolineshloka
{ऋग्वेदिभिर्याजुषैश्च सामावणकोविदः}
{वेष्टितो मुनिभिस्तत्र द्वितीय इव पद्मजः} %॥२६॥

\twolineshloka
{दृष्ट्वा तं मुनिशार्दूलं राजा वैखानसस्तदा}
{जगाम चावनिं मूर्ना दण्डवत् प्रणनाम च} %॥२७॥

\twolineshloka
{पप्रच्छ कुशलं तस्य सप्तस्वङ्गेष्वसौ मुनिः}
{राज्ये निष्कण्टकत्वं च राजसौख्यसमन्वितम्} %॥२८॥

\uvacha{राजोवाच}

\twolineshloka
{तव प्रसादान्कुशलमङ्गेषु मम सप्ततु}
{विभवेष्वनुकूलेषु कश्चिद्विन्न उपस्थितः} %॥२९॥

\twolineshloka
{एवं मे संशयं ब्रह्मन् प्रष्टुं त्वामहमागतः}
{एवं श्रुत्वा नृपवचः पर्वतो मुनिसत्तमः} %॥३०॥

\twolineshloka
{ध्यानस्तिमितनेत्रोऽसौ भूतं भव्यं व्यचिन्तयत}
{मुहूर्तमेकं ध्यात्वा च प्रत्युवाच नृपोत्तमम्} %॥३१॥

\uvacha{मुनिरुवाच}

\twolineshloka
{जानेऽहं तव राजेन्द्र पितुः पापं विकमणः}
{पूर्वजन्मनि ते पित्रा स्वपत्नीद्रयमध्यतः} %॥३२॥

\twolineshloka
{कामासक्तेन चैकस्या ऋतुभङ्गः कृतः स्त्रियः}
{त्राहि देहीति जल्पत्या अन्यस्याश्च नराधिप} %॥३३॥

\onelineshloka*
{कर्मणा तेन सततं नरके पतितो ह्ययम्}

\uvacha{राजोवाच}

\onelineshloka
{केन व्रतेन दानेन मोक्षस्तस्य भवेन्मुने} %॥३४॥

\onelineshloka*
{निरयात्पापसंयुक्तातन्ममाचक्ष्व पृच्छतः}

\uvacha{मुनिरुवाच}

\onelineshloka
{मार्गशीर्षे सिते पक्षे मोक्षानाम्नी हरेस्तिथिः} %॥३५॥

\twolineshloka
{सर्वैस्तु तद्वतं कृत्वा पित्रे पुण्यं प्रदीयताम्}
{तस्य पुण्यप्रभावेण मोक्षस्तस्य भविष्यति} %॥३॥

\twolineshloka
{मुनेर्वाक्यं ततः श्रुत्वा नृपः स्वगृहमागतः}
{आग्रहायणिकी शुक्ला प्राप्ता भरतसत्तम} %॥३७॥

\twolineshloka
{अन्तःपुरचरैः सर्वैः पुत्ररस्तदा नृपः}
{व्रतं कृत्वा विधानेन पित्रे पुण्यं ददौ नृपः} %॥३८॥

\twolineshloka
{तस्मिन्दत्ते तदा पुण्ये पुष्पवृष्टिरभूदिवः}
{वैखानसपिता तेन गतः स्वर्ग स्तुतो गणैः} %॥३९॥

\twolineshloka
{राजानमन्तरिक्षाच शुद्धां गिरमभाषत}
{स्वस्त्यस्तु ते पुत्र सदेत्यथ स त्रिदिवं गतः} %॥४०॥

\twolineshloka
{एवं यः कुरुते राजन् मोक्षामेकादशीमिमाम्}
{तस्य पापं क्षयं याति मृतो मोक्षमवाप्नुयात्} %॥४१॥

\twolineshloka
{नातः परतरा काचिन्मोझदा विमला शुभा}
{पुण्यसङ्ख्यां तु तेषां वै न जानेऽहं तु यः कृता} %॥४२॥

\twolineshloka
{पठनाच्छ्वणाचास्या वाजपेयफलं लभेत्}
{चिन्तामणिसमा ह्येषा स्वर्गमोक्षप्रदायिनी} %॥४३॥

॥इति श्रीब्रह्माण्डपुराणे मार्गशीर्षशुक्लैकादश्या मोक्षानाम्न्या माहात्म्यं सम्पूर्णम्॥


\hyperref[sec:ekadashi_mahatmyam_vrata_raja]{\closesub}
\clearpage

\sect{पौष-कृष्ण-सफला-एकादशी-माहात्म्यम्}
\label{sec:vrata-raja-pausha-krishna-saphala}

\uvacha{युधिष्ठिर उवाच}

\twolineshloka
{पौषस्य कृष्णपक्षे तु द्वादशी या भवेत् प्रभो}
{किन्नाम को विधिस्तस्याः को देवस्तत्र पूज्यते} %॥१॥

\onelineshloka*
{एतदाचक्ष्व मे स्वामिन्विस्तरेण जनार्दन}

\uvacha{श्रीकृष्ण उवाच}

\onelineshloka
{कथयिष्यामि राजेन्द्र भवतः स्नेहकारणात्} %॥२॥

\twolineshloka
{तथा तुष्टिर्न मे राजन् ऋतुभिश्चाप्तदक्षिणैः}
{यथा तुष्टिर्भवन्मह्यमेकादश्या व्रतेन वै} %॥३॥

\twolineshloka
{तस्मात्सर्वप्रयत्नेन कर्तव्यो हरिवासरः}
{पौषस्य कृष्णपक्षे तु द्वादशी या भवेन्नृप} %॥४॥

\twolineshloka
{तस्याश्चैव च माहात्म्यं शृणुष्वैकापमानसः}
{गदितायाश्च चै राजन्नैकादश्यो भवन्ति हि} %॥५॥

\twolineshloka
{तासामपि हि सर्वासां विकल्पं नैव कारयेत्}
{अतः परं प्रवक्ष्यामि पौधे कृष्णा हि द्वादशी} %॥६॥

\twolineshloka
{तस्या विधिं नृपश्रेष्ठ लोकानां हितकाम्यया}
{पौषस्य कृष्णपक्षे या सफलानाम नामतः} %॥७॥

\twolineshloka
{नारायणोऽधिदेवोऽस्याः पूजयेत्तं प्रयत्नतः}
{पूर्वेण विधिना राजन् कर्तव्यकादशी जनः} %॥८॥

\twolineshloka
{नागानां च यथा शेषः पक्षिणां गरुडो यथा}
{यथाश्वमेधो यज्ञानां नदीनां जाह्नवी यथा} %॥९॥

\twolineshloka
{देवानां च यथाविष्णुर्द्विपदां ब्राह्मणो यथा}
{व्रतानां च तथा राजन् प्रवरैकादशी तिथिः} %॥१०॥

\twolineshloka
{ते जना भरतश्रेष्ठ मम पूज्याश्च सर्वशः}
{हरिवासरसंसक्ता वर्तन्ते ये भृशं नृप} %॥११॥

\twolineshloka
{सफलानाम या प्रोक्ता तस्याःपूजाविध शृणु}
{फलैर्मा पूजयेत्तत्र कालदेशोद्भवैः शुभैः} %॥१२॥

\twolineshloka
{नारिकेलफल शुद्धस्तथा वै बीजपूरकैः}
{जम्बीरैर्दाडिमैश्चैव तथा पूगफलैरपि} %॥१३॥

\twolineshloka
{लवङ्गैविविधैश्चान्यैस्तथा चाम्रफलादिभिः}
{पूजयेदेवदेवेशं धूपै-पैर्यथाक्रमम्} %॥१४॥

\twolineshloka
{सफलायां दीपदानं विशेषेण प्रकीर्तितम्}
{रात्री जागरणं तत्र कर्तव्यं च प्रयत्नतः} %॥१५॥

\twolineshloka
{यावदुन्मिषते नेत्रं तावजागर्ति यो निशि}
{एकाग्रमानसो भूत्वा तस्य पुण्यफलं शृणु} %॥१६॥

\twolineshloka
{तत्समो नास्ति वै यज्ञस्तीर्थ नत्सदृशं न हि}
{तत्समं न व्रतं किञ्चिदिह लोके नराधिप} %॥१७॥

\twolineshloka
{पञ्चवर्षसहस्राणि तपस्तप्त्वा च यत्फलम्}
{तत्फलं समवाप्नोति सफलाजागरेण वै} %॥१८॥

\twolineshloka
{श्रूयतां राजशार्दूल सफलायाः कथानकम्}
{चम्पावतीति विख्याता पुरी माहिष्मतस्य च} %॥१९॥

\twolineshloka
{माहिष्मतव्य राजर्वेश्चन्वारश्वाभवन्सुताः}
{तेषां मध्ये तु यो ज्येष्ठः स महापापसंयुतः} %॥२०॥

\twolineshloka
{परदाराभिगामी च यतवेश्यारतः सदा}
{पितुर्द्रव्यं स पापिष्ठो गमयामास सर्वशः} %॥२१॥

\twolineshloka
{असवृत्तिरतो नित्यं देवताद्विजनिन्दकः}
{वैष्णवानां च देवानां नित्यं निन्दारतः स वै} %॥२२॥

\twolineshloka
{ईदृग्विधं तदा दृष्ट्वा पुत्र माहिष्मतो नृपः}
{राज्यानिष्कासयामास लुम्पकं नाम नामतः} %॥२३॥

\twolineshloka
{राज्यानिष्कासितस्तेन पित्रा चैवापि बन्धुभिः}
{परिवारजनैः सर्वैस्त्यक्तो राज्ञो भयात्तदा} %॥२४॥

\twolineshloka
{लुम्पकोऽपि तदा त्यक्तश्चिन्तयामास चैकलः}
{मयात्र किं प्रकर्तव्यं त्यक्तेन पितृबान्धवैः} %॥२५॥

\twolineshloka
{इति चिन्तापरो भृत्वा मतिं पापे तदाकरोत्}
{मया तु गमनं कार्य बने त्यक्त्वा पुरं पितुः} %॥२६॥

\twolineshloka
{तस्मानापितुः सर्व व्यापयिष्ये पुरं निशि}
{दिवा वने चरिष्यामि रात्रावपि पितुः पुरे} %॥२७॥

\twolineshloka
{इत्येवं स मति कृत्वा लुम्पको दैवपातितः}
{निर्जगाम पुरात्तस्माद्गतोऽसौ गहनं वनम्} %॥२८॥

\twolineshloka
{जीवघातकरो नित्यं नित्यं स्तेयपरायणः}
{सर्व च नगरं तेन मुषितं पापकर्मणा} %॥२९॥

\twolineshloka
{गृहीतश्च परिल्यको लोके राज्ञो भयात्तदा}
{जन्मान्तरीयपापेन राज्यभ्रष्टः स पापकृत्} %॥३०॥

\twolineshloka
{आमिषाभिरतो नित्यं नित्यं वै फलभक्षकः}
{आश्रमस्तस्य दुष्टस्य वासुदेवस्य सम्मतः} %॥३१॥

\twolineshloka
{अश्वत्थो वर्तते तत्र जीर्णो बहुलवार्षिकः}
{देवत्वं तस्य वृक्षस्य वर्तते तद्ने महत्} %॥३२॥

\twolineshloka
{तत्रैव न्यबसचासौ लुम्पकः पापबुद्धिमान्}
{एवं कालक्रमेणैव वसतस्तस्य पापिनः} %॥३३॥

\twolineshloka
{दुष्कर्मनिरतस्यास्य कुर्वतः कर्म निन्दितम्}
{पौषस्य कृष्णपक्षे तु पूवस्मिन् सफलादिनात्} %॥३४॥

\twolineshloka
{दशमीदिवसे राजन्निशायां शीतपीडितः}
{लुम्पको वस्त्रहीनो वै निश्चेष्टो ह्यभवत्तदा} %॥३५॥

\twolineshloka
{पीडयमानस्तु शीतेन अश्वत्थस्य समीपगः}
{न निद्रा न सुखं तस्य गतप्राण इवाभवत्} %॥३६॥

\twolineshloka
{पीडयन्दशनैर्दन्तानवं सोऽगमयन्निशाम्}
{भानूदयेऽपि तस्याथ न सञ्ज्ञा समजायत} %॥३७॥

\twolineshloka
{लुम्पको गतसञ्ज्ञस्तु सफलादिवसे ततः}
{मध्याह्नसमये प्राप्ते सञ्ज्ञा लेभे स पार्थिव} %॥३८॥

\twolineshloka
{प्राप्तसञ्ज्ञो मुहूर्तेन चोत्थितोसौ तदासनात्}
{प्रस्खलंश्च पदस्यास्पैः पडूगुवञ्चलितो मुहुः} %॥३९॥

\twolineshloka
{वनमध्ये गतस्तत्र क्षुत्तृषापीडितोऽभवत्}
{न शक्तिविघातस्य लुप्पकस्य दुरात्मनः} %॥४०॥

\twolineshloka
{फलानि भूमौ पतितान्याहृत्य च स लुपकः}
{यावत्स चागतस्तत्र तावदस्तमगाद्रविः} %॥४१॥

\twolineshloka
{किं भविष्यति तातेति विललापातिदुःखितः}
{फलानि तानि सर्वाणि वृक्षमूले निवेदयन्} %॥४२॥

\twolineshloka
{इत्युवाच फलैरेभिः प्रीयतां भगवान् हरिः}
{उपविष्टो लुपकश्च निद्रां लेभे न वै निशि} %॥४३॥

\twolineshloka
{तेन जागरणं मेने भगवान्मधुसूदनः}
{फलश्च पूजनं मेने सफलायां तथानघ} %॥४४॥

\twolineshloka
{कृतमेवं लुपकेन ह्यकस्माद्वतमुत्तमम्}
{तेन व्रतप्रभावेण प्राप्त राज्यमकण्टकम्} %॥४५॥

\twolineshloka
{पुण्याकुरोदयाद्राजन् यथाप्राप्तं तथा शृणु}
{रवेरुदयवेलायां दिव्योऽश्वश्चाजगाम ह} %॥४६॥

\twolineshloka
{दिव्यवस्तुपरीवारो लुम्पकस्य समीपतः}
{तस्थौ स तुरगो राजन वागुवाचाशरीरिणाम्} %॥४७॥

\twolineshloka
{प्राप्तुहि त्वं नृपसुत स्वराज्यं हतकण्टकम्}
{वासुदेवप्रसादेन सफलायाः प्रभावतः} %॥४८॥

\twolineshloka
{पितुः समीपं गच्छ त्वं भुंश्व राज्यमकण्टकम्}
{तथेत्युक्त्वा त्वसौ तत्र दिव्यरूपधरोऽभवत्} %॥४९॥

\twolineshloka
{कृष्णे मतिश्च तस्यासीत्परमा वैष्णवी तथा}
{दिव्याभरणशोभाढयस्तातं नत्वा स्थितो गृहे} %॥५०॥

\twolineshloka
{वैष्णवाय ततो दत्तं पित्रा राज्यमकण्टकम्}
{कृतं राज्यं तु तेनैव वर्षाणि सुबहून्यपि} %॥५१॥

\twolineshloka
{हरिवासरसलीनो विष्णुभक्तिरतः सदा}
{मनोज्ञास्त्वस्य पुत्राः स्युर्दाराः कृष्णप्रसादतः} %॥५२॥

\twolineshloka
{ततः स वार्ध के प्राप्त राज्यं पुत्रे निवेश्य च}
{वनं गतः संयतात्मा विष्णुभक्तिपरायणः} %॥५३॥

\twolineshloka
{साधयित्वा तथात्मानं विष्णुलोकं जगाम ह}
{एवं ये वै प्रकुर्वन्ति सफलैकादशीव्रतम्} %॥५४॥

\twolineshloka
{इह लोके यशः प्राप्य मोक्षं यास्यत्यसंशयम्}
{धन्यास्ते मानवा लोके सफलातकारिणः} %॥५५॥


\threelineshloka
{तस्मिन्नन्मनि ते मोक्षं लभन्ते नात्र संशयः}
{सफलायाश्च माहात्म्यश्रवणाद्धि विशाम्पते}
{राजसूयफलं प्राप्य वसेत्स्वर्गे च मानवः} %॥५६॥

॥इति पौषकृष्ण-एकादश्याः सफलानाम्न्या माहात्म्यं सम्पूर्णम्॥


\hyperref[sec:ekadashi_mahatmyam_vrata_raja]{\closesub}
\clearpage

\sect{पौष-शुक्ल-पुत्रदा-एकादशी-माहात्म्यम्}
\label{sec:vrata-raja-pausha-shukla-putrada}

\uvacha{युधिष्ठिर उवाच}

\twolineshloka
{कथिता वै त्वया कृष्ण सफलैकादशी शुभा}
{कथयस्व प्रसादेन शुक्ला पौषस्य या भवेत्} %॥१॥

\twolineshloka
{किन्नाम को विधिस्तस्याः को देवस्तत्र पूज्यते}
{कस्मै तुष्टो हृषीकेश त्वमेव पुरुषोत्तम} %॥२॥

\uvacha{श्रीकृष्ण उवाच}

\twolineshloka
{शृणु राजन् प्रवक्ष्यामि शुक्ला पौषस्य या भवेत्}
{तस्या विधि महारान लोकानां च हिताय वै} %॥३॥

\twolineshloka
{पूर्वेण विधिना राजन् कर्तव्यैषा प्रयत्नतः}
{पुत्रदेति च नाम्नासौ सर्वपापहरा वरा} %॥४॥

\twolineshloka
{नारायणोऽधिदेवोऽम्याः कामदः सिद्विदायकः}
{नातःपरतरा काचित्रलोक्ये सचराचरे} %॥५॥

\twolineshloka
{विद्यावन्तं यशस्वन्तं लक्ष्मीवन्तं करोत्यसौ}
{शृणु राजन् प्रवक्ष्यामि कथां पापहरां पराम्} %॥६॥

\twolineshloka
{पुरी भद्रावती नाम्नी राजा तत्र सुकेतुमान्}
{तस्य राज्ञोऽथ राज्ञी च शैब्या नाम्नीति विश्रुता} %॥७॥

\twolineshloka
{पुत्रहीनेन राज्ञा च कालो नीतो मनोरथैः}
{नैवात्मजं नृपो लेभे वंशकर्तारमेव च} %॥८॥

\twolineshloka
{तेनैव राज्ञा धर्मेण चिन्तितं बहुकालतः}
{किं करोमि छ गच्छामि सुतप्रातिः कथं भवेत्} %॥९॥

\twolineshloka
{न राष्ट्र न पुरे सौख्यं लेभे राजा सुकेतुमान्}
{शैव्यया कान्तया साई प्रत्यहं दुःखितोऽभवत्} %॥१०॥


\threelineshloka
{तावुभौ दम्पती नित्यं चिन्ताशोकपरायणौ}
{पितरोऽन्य जलं दत्तं कवोष्णमुपभुनते}
{राजः पश्चान्न पश्यामो योऽस्मान् सन्तर्पयिष्यति} %॥११॥

\twolineshloka
{इत्येवं संस्मरन्तोऽस्य पितरो दुःखिनोभवन्}
{न बान्धवा न मित्राणि नामात्याः सुहृदस्तथा} %॥१२॥

\twolineshloka
{रोचन्ते तस्य भूपस्य न गजाश्वपदातयः}
{नैराश्यं भूपतेस्तस्य मनस्येवमजायत} %॥१३॥

\twolineshloka
{नरस्य पुत्रहीनस्य नास्ति वै जन्मनः फलम्}
{अपुत्रस्य गृहं शून्यं हृदयं दुःखितं सदा} %॥१४॥

\twolineshloka
{पितृदेवमनुष्याणां नानृगित्वं सुतं विना}
{तस्मात्सर्वप्रयत्नेन सुतमुत्पादयेन्नरः} %॥१५॥

\twolineshloka
{इहलोके यशस्तेषां परलोके शुभा गतिः}
{येषां तु पुण्यकर्तृणां पुत्रजन्म गृहे भवेत्} %॥१६॥

\twolineshloka
{आयुरारोग्यसम्पत्तिस्तेषां गेहे प्रवर्तते}
{पुत्राः पौत्राश्च लोकाश्च भवेयुः पुण्यकर्मणाम्} %॥१७॥

\twolineshloka
{पुण्यं विना न च प्राप्तिर्विष्णुभक्तिं विना तथा}
{पुत्राणां सम्पदो वाऽपि विद्यापाश्चेति मे मतिः} %॥१८॥

\twolineshloka
{एवं चिन्तयमानोसो राजा शर्म न लब्धवान्}
{प्रत्यूषेऽचिन्तयद्राजा निशीथेऽचिन्तयत्तथा} %॥१९॥

\twolineshloka
{ततश्चात्मविनाशं व विचार्याथ सुकेतुमान्}
{आत्मघाते दुर्गातं च चिन्तयित्वा तदा नृपः} %॥२०॥

\twolineshloka
{दृष्ट्वान्मदेई प्रक्षीणमपुत्रत्वं तथैव च}
{पुनर्विचार्यात्मबुद्धचा ह्यात्मनो हितकारणम्} %॥२१॥

\twolineshloka
{अश्वारूढस्ततो राजा जगाम गहनं वनम्}
{पुरोहितादयः सर्वे न जानन्ति गतं नृपम्} %॥२२॥

\twolineshloka
{गम्भीरे विपिन राजा मृगपक्षिनिषेविते}
{विचचार तदा तस्मिन्वनवृक्षान्विलोकयन्} %॥२३॥

\twolineshloka
{वटानश्वत्थविल्वांश्च खजूरापनसांस्तथा}
{बकुलांश्च सदापर्णास्तिन्दुकांस्तिलकानपि} %॥२४॥

\twolineshloka
{शालांस्तालास्तमालांश्च ददर्श सरलानृपः}
{इगुदीककुभांश्चैव श्लेष्मातकविभीतकान्} %॥२५॥

\twolineshloka
{शल्लकीकरमर्दाश्च पाटलान् स्खदिरानपि}
{शाकांश्चैव पलाशांश्च शोभितान् ददृशे पुनः} %॥२६॥

\twolineshloka
{मृगव्याध्रवराहांश्च सिंहाशाखामृगानपि}
{गवयान् कृष्णसारांश्च सृगालाशशकानपि} %॥२७॥

\twolineshloka
{वनमार्जार कान् क्रूराञ्शल्लकांश्चमरानपि}
{ददर्श भुजगान् राजा वल्मीकादभिनिन्मृतान्} %॥२८॥

\twolineshloka
{तथा वनगजान्मत्तान्कलभैः सह सङ्गतान्}
{यूथपांश्च चतुर्दन्तान्करिणीगणमध्यगान्} %॥२९॥

\twolineshloka
{तान् दृष्ट्वा चिन्तयामास ह्यात्मनः स गजान्नृपा}
{तेषां स विचरन्मध्ये राजा शोभामवाप ह} %॥३०॥

\twolineshloka
{महदाश्चर्यसंयुक्तं ददर्श विपिनं नृपः}
{कञ्चिच्छिवारुतं शृण्वन्तुलूकविरुतं तथा} %॥३१॥

\twolineshloka
{तांस्तान् क्षिमृगान् पश्यन्बभ्राम वनमध्यगः}
{एवं ददर्श गहनं नृपो मध्यङ्गते रवौ} %॥३२॥

\twolineshloka
{क्षुत्तृभ्यां पीडितो राजा इतश्चेतश्च धावति}
{चिन्तयामास नृपतिः संशुष्कगलकन्धरः} %॥३३॥

\twolineshloka
{मया तु किं कृतं कर्म प्राप्तं दुःखं यदीदृशम्}
{मया वै तोविता देवा यज्ञैः पूजाभिरेव च} %॥३४॥

\twolineshloka
{तथैव ब्राह्मगा दानस्तोषिता मृष्टभोजनैः}
{प्रजाश्चैव यथाकालं पुत्रवन्परिपालिताः} %॥३५॥

\twolineshloka
{कस्मादुःखं मया प्रातमीदृशं दारुणं महत्}
{इति चिन्तापरो राजा जगामाथागतो वनम्} %॥३६॥

\twolineshloka
{सुकृतस्य प्रभावेण सरो दृष्टं मनोरमम्}
{मानसेन स्पर्धमानं पद्मिनी परिशोभितम्} %॥३७॥

\twolineshloka
{कारण्डवैश्चक्रवाकै राजहंसैश्च नादितम्}
{मकरहुभिर्मत्स्यैरन्य जलचरैर्युतम्} %॥३८॥

\twolineshloka
{समीपे सरसस्तत्र मुनीनामाश्रमान् बहून्}
{ददर्श राजा लक्ष्मीवानिमित्तैः शुभशंसिभिः} %॥३९॥

\twolineshloka
{सव्यात्परतरं चक्षुरपसव्यस्तथा करः}
{प्रास्फुरन्नृपतेस्तस्य कथयञ्शोभनं फलम्} %॥४०॥

\twolineshloka
{तस्य तीरे मुनीन् दृष्ट्वा कुर्वाणान्नैगमं जपम्}
{अवतीर्य हयात्तस्मान्मुनीनामग्रतः स्थितः} %॥४१॥

\twolineshloka
{पृथक् पृथग्ववन्दे स मुनीस्तान् संशितव्रतान्}
{कृताञ्जलिपुटो भूत्वा दण्डवञ्च प्रणम्य सः} %॥४२॥

\twolineshloka
{हर्षेण महताविष्टो बभूव नृपसत्तमः}
{तमूचुस्तेऽपि मुनयः प्रसन्नाः स्मो वयं तव} %॥४३॥

\onelineshloka*
{कथयस्वाद्य वै राजन्यत्ते मनसि वर्तते}

\uvacha{राजोवाच}

\onelineshloka
{के यूयमुग्रतपसः का आख्या भवतामपि} %॥४४॥

\onelineshloka*
{किमर्थ सङ्गता यूयं वदन्तु मम तत्त्वतः}

\uvacha{मुनय ऊचुः}

\onelineshloka
{विश्वेदेवा वयं राजन् स्नानार्थमिह चागताः} %॥४५॥

\twolineshloka
{माघो निकटमायात एतस्मात्पञ्चमेहनि}
{अद्य ह्येकादशी राजन पुत्रदा नाम नामतः} %॥४६॥

\onelineshloka*
{पुत्रं ददात्यसौ शुक्ला पुत्रदा पुत्रमिच्छताम्}

\uvacha{राजोवाच}

\onelineshloka
{ममापि यत्नो मुनयः सुतस्योत्पादने महान्} %॥४७॥

\onelineshloka
{यदि तुष्टा भवन्तो मे पुत्रो धै दीयतां शुभः}

\uvacha{मुनय ऊचुः}

\onelineshloka
{अस्मिन्नेव दिने राजन् पुत्रदा नाम वर्तते} %॥४८॥

\twolineshloka
{एकादशी तिथिः ख्याता क्रियतां व्रतमुत्तमम्}
{आशीर्वादेन चास्माकं केशवस्य प्रसादतः} %॥४९॥

\twolineshloka
{अवश्यं तव राजेन्द्र पुत्रप्राप्तिर्भविष्यति}
{इत्येवं वचनात्तेषां कृतं राज्ञा व्रतं शुभम्} %॥५०॥


\twolineshloka
{द्वादश्यां पारणं कृत्वा मुनीनत्वा पुनः पुनः}
{आजगाम गृहं राजा राज्ञी गर्भ समादधे} %॥५१॥

\twolineshloka
{मुनीनां वचनेनैव पुत्रदायाः प्रसादतः}
{पुत्रो जातस्तथा काले तेजस्वी पुण्यकर्मकृत्} %॥५२॥

\twolineshloka
{पितरं तोषयामास प्रजापालो बभूव सः}
{एतस्मात्कारणाद्राजन्कर्तव्यं पुत्रदावतम्} %॥५३॥

\twolineshloka
{लोकानां च हितार्थाय तवाने कथितं मया}
{एतद्वतं तु ये माः कुर्वन्ति पुत्रदाभिधम्} %॥५४॥

\twolineshloka
{पुत्र प्राप्येह लोके तु मृतास्ते स्वर्गगामिनः}
{पठनाच्छ्रवणाद्राजन्नश्वमेधफलं लभेत्} %॥५५॥

॥इति श्रीभविष्योत्तरपुराणे पौषशुक्लैकादश्याः पुत्रदानाम्न्या माहात्म्यं सम्पूर्णम्॥


\hyperref[sec:ekadashi_mahatmyam_vrata_raja]{\closesub}
\clearpage

\sect{माघ-कृष्ण-षट्तिला-एकादशी-माहात्म्यम्}
\label{sec:vrata-raja-magha-krishna-shattila}

\uvacha{दाल्भ्य उवाच}

\twolineshloka
{मर्त्यलोके तु सम्प्राप्ताः पापं कुर्वन्ति जन्तवः}
{ब्रह्महत्यादिपापैश्च ह्यन्यैश्च विविधैर्युताः} %॥१॥

\twolineshloka
{परद्रव्यापहर्तारः परव्यसनमोहिताः}
{कथं नायान्ति नरकान्ब्रह्मस्तवहि तत्त्वतः} %॥२॥

\twolineshloka
{अनायासेन भगवन् दानेनाल्पेन केनचित्}
{पापं प्रशममायाति येन तद्वक्तुमर्हसि} %॥३॥

\uvacha{पुलस्त्य उवाच}

\twolineshloka
{साधु साधु महाभाग गुह्यमेतत्तुदुर्लभम्}
{यन्न कस्यचिदाख्यातं ब्रह्मविष्ण्विन्द्रदैवतैः} %॥४॥

\twolineshloka
{तदहं कथयिष्यामि त्वया पृष्टो द्विजोत्तम}
{पौषमासे तु सम्प्राप्ते शुचिः स्नातो जितेन्द्रियः} %॥५॥

\twolineshloka
{कामक्रोधाभिमानेालोभपैशुन्यवर्जितः}
{देवदेवं च संस्मृत्य पादौ प्रक्षाल्य वारिणा} %॥६॥

\twolineshloka
{पुष्यक्षेण तु सङ्गृह्य गोमयं तत्र मानकः}
{तिलान्प्रक्षिप्य कार्यासं पिण्डकांश्चैव कारयेत्} %॥७॥

\twolineshloka
{अष्टोत्तरशतं होमो नात्र कार्या विचारणा}
{माघमासे तु सम्प्राप्त ह्याषाढः भवेद्यदि} %॥८॥

\twolineshloka
{मूलं वा कृष्णपक्षस्य द्वादश्यां नियमं ततः}
{गृहीयात्पुण्यफलदं विधानं तस्य मे शृणु} %॥९॥

\twolineshloka
{देवदेवं समभ्यर्च सुनातः प्रयतः शुचिः}
{कृष्णनामानि सङ्कीर्त्य एकादश्यामपोषितः} %॥१०॥

\twolineshloka
{रात्री जागरणं कुर्याद्रात्रौ होमं च कारयेत्}
{अर्चयेद्देवदेवशं द्वितीयेदि पुनहरिम्} %॥११॥

\twolineshloka
{चन्दनागुरुकर्परेनैवेद्यं कृसरं तथा}
{संस्तुत्य नाम्ना तेनैव कृष्णाख्येन पुनः पुनः} %॥१२॥

\twolineshloka
{कूष्माण्डेनारिकेले व ह्यथवा बीजपूरकैः}
{सर्वाभावे तु विप्रेन्द्र शस्तपूगीफलैर्युतम्} %॥१३॥

\twolineshloka
{अर्थ दद्याद्विधानेन पूजयित्वा जनादनम्}
{कृष्ण कृष्ण कृपालुस्त्वमगतीनां गतिर्भव} %॥१४॥

\twolineshloka
{संसारार्णवमन्नानां प्रसीद परमेश्वर}
{नमस्ते पुण्डरीकाक्ष नमस्ते विश्वभावन} %॥१५॥

\twolineshloka
{सुब्रह्मण्य नमस्तेऽस्तु महापुरुषपूर्वज}
{गृहाणायं मया दत्तं लक्ष्म्या सह जगत्पते} %॥१६॥

\twolineshloka
{ततस्तु पूजयेद्विप्रनुदकुम्भं प्रदापयेत्}
{छत्रोपानयुगैः सार्ध कृष्णो मे प्रीयतामिति} %॥१७॥

\twolineshloka
{कृष्णा धेनुः प्रदातव्या यथाशक्त्या द्विजोत्तम}
{तिलपात्र द्विजश्रेष्ठ दात्तत्र विचक्षणः} %॥१८॥

\twolineshloka
{स्नानप्राशनयोः शस्ताः श्वेताः कृष्णास्तिला मुने}
{तान्प्रदद्यात्प्रयत्नेन यथाशक्त्या द्विजोत्तम} %॥१९॥

\twolineshloka
{तिलप्ररोहजाः क्षेत्रे यावत्सङ्ख्यास्तिला द्विज}
{तावद्वर्षसहस्राणि स्वर्गलोके महीयते} %॥२०॥

\twolineshloka
{तिलस्नायी तिलोद्वौ तिलहोमी तिलोदकी}
{तिलभुक् तिलदाता च षट्तिलाः पापनाशकाः} %॥२१॥

[इयमेव पन्तलाख्या।]

\uvacha{नारद उवाच}

\twolineshloka
{कृष्ण कृष्ण महाबाहो नमस्ते विश्वभावन}
{षटतिलकादशीभूतं कीदृशं फलमश्नुते} %॥२२॥

\onelineshloka*
{सोपाख्यानं मम ब्रूहि यदि तुष्टोसि यादव}

\uvacha{श्रीकृष्ण उवाच}

\onelineshloka
{शृणु ब्रह्मन् यथावृत्तं दृष्टं तत्कथयामि ते} %॥२३॥

\twolineshloka
{मृत्युलोके पुरा ह्यासीद्राह्मण्येका च नारद}
{व्रतचर्यारता नित्यं देवपूजारता सदा} %॥२४॥

\twolineshloka
{मासोपंवासन्निरता मम भक्ता च सर्वदा}
{कृष्णोपवाससंयुक्ता मम पूजापरायणा} %॥२५॥

\twolineshloka
{शरीरं क्लेशितं नित्यमुपवासैस्तया द्विज}
{दीनानां ब्राह्मणानाञ्च कुमारीणां च भक्तितः} %॥२६॥

\twolineshloka
{गृहादिकं प्रयच्छन्ती सर्वकालं महामतिः}
{अतिकृच्छ्रता सा तु सर्वकालेषु वै द्विजा} %॥२७॥

\twolineshloka
{ब्राह्मणा नान्नदानेन तर्पिता देवता न च}
{ततःकालेन महता मया ये चिन्तितं द्विज} %॥२८॥

\twolineshloka
{शुद्धमस्या शरीरं हि व्रतः कृच्छर्न संशयः}
{जितो वैष्णवो लोकः कायक्लेशेन वै तया} %॥२९॥

\twolineshloka
{न दत्तमन्नदानं हि येन तृप्तिःपरा भवेत्}
{विचिन्त्यैवं मया ब्रह्मन् मृत्युलोकमुपेत्य च} %॥३०॥

\onelineshloka*
{कापालं रूपमास्थाय भिक्षा पात्रेण याचिता}

\uvacha{ब्राह्मण्युवाच}

\onelineshloka
{कस्मात्त्वमागतो ब्रह्मन् वद सत्यं ममाप्रतः} %॥३१॥

\twolineshloka
{पुनरेव मयाप्रोक्तं दहि भिक्षां च सुन्दरि}
{तया कोपेन महता मृत्पिण्डस्ताम्रभाजने} %॥३२॥

\twolineshloka
{क्षिप्तो यावदहं ब्रह्मन् पुनः स्वर्गः गतो द्विज}
{ततः कालेन महता तापसी तुमहाव्रता} %॥३३॥

\twolineshloka
{सदेहा स्वर्गमायाता व्रतचर्याप्रभावतः}
{मृत्पिण्डस्य प्रभावेण गृहं प्राप्त मनोरमम्} %॥३४॥

\twolineshloka
{परं तच्चैव विप्रर्षे धान्यकोशविवर्जितम्}
{गृहं यावत्प्रविश्यैषा न किञ्चितत्र पक्ष्यति} %॥३५॥

\twolineshloka
{तावद्गृहाद्विनिष्क्रम्य ममान्ते चागता द्विज}
{क्रोधेन महताविष्टा इदै वचनमब्रवीत्} %॥३६॥

\twolineshloka
{मया वतैश्च कृछैच ह्युपवासैरनेकशः}
{पूजयाराधितो देवः सर्वलो कल्य भावनः} %॥३७॥

\twolineshloka
{न तत्र दश्यते किश्चिद्गृहे मम जनार्दन}
{ततश्चोक्ता मया सा तु गृह गच्छ यथागतम्} %॥३८॥

\twolineshloka
{आगमिष्यन्ति सुतरां कौतूहलसमन्विताः}
{द्रष्टुं त्वां देवपत्न्यस्तु दिव्यरूपसमन्विताः} %॥३९॥


\threelineshloka
{द्वारं नोद्धाटय विना षट्तिलापुण्यवाचनात्}
{एवमुक्ता गता सा तु याव? मानुषी गृहम्}
{अवान्तरे समायाता देवपत्त्यश्च नारद} %॥४०॥

\twolineshloka
{ताभिश्च कथितं तत्र त्वां द्रष्टुं हि समागताः}
{द्वारमुद्घाटय त्वं च पश्यामस्त्वां शुभानने} %॥४१॥

\uvacha{मानुष्युवाच}

\twolineshloka
{यदि द्रष्टुं समायाताः सत्यं वाच्यं विशेषतः}
{षट्तिलाया व्रतं पुण्यं द्वारोद्धाटनकारणात्} %॥४२॥

\twolineshloka
{एकापि नावदत्तत्र षट्तिलैकादशीव्रतम्}
{अन्यया कथितं तत्र द्रष्टव्या मानुषी मया} %॥४३॥

\twolineshloka
{ततो द्वारं समुदाय दृष्टा ताभिश्च मानुषी}
{न देवी न च गन्धर्वी नासुरी न च पन्नगी} %॥४४॥

\twolineshloka
{दृष्टा पूर्व तथा नारी यादृशीयं द्विजर्षभ}
{देवीनामुपदेशेन षट्तिलाया व्रतं कृतम्} %॥४५॥

\twolineshloka
{मानुष्या सत्यवतया भुक्तिमुक्तिफलप्रदम्}
{रूपकान्तिसमायुक्ता क्षणेन समवाप सा} %॥४६॥

\twolineshloka
{धनं धान्यं च वस्त्रादि सुवर्ण रौप्यमेव च}
{भवनं सर्वसम्पन्नं षटतिलायाः प्रसादतः} %॥४७॥

\twolineshloka
{अतितृष्णा न कर्तव्या वित्तशाठयं विवर्जयेत्}
{आत्मवित्तानुसारेण तिलान् वस्त्रादि दापयेत्} %॥४८॥

\twolineshloka
{लभते चैवमारोग्यं ततो जन्मनि जन्मनि}
{दारिद्यं न च कष्टं च न च दौर्भाग्यमेव च} %॥४९॥

\twolineshloka
{न भवढे द्विजश्रेष्ठ पतिलायामुपोषणात्}
{अनेन विधिना ब्रह्मस्तिलदानान्न संशयः} %॥५०॥

\threelineshloka
{मुच्यते पातकैः सर्वैर्नात्र कार्या विचारणा}
{दानं च विधिना सम्यक् सर्वपापप्रणाशनम्}
{नानर्थः कश्चिन्नायासः शरीरे मुनिसत्तम} %॥५१॥

॥इति श्रीभविष्योत्तरपुराणे माघकृष्णैकादश्याः षट्तिलानाम्न्या माहात्म्यं सम्पूर्णम्॥


\hyperref[sec:ekadashi_mahatmyam_vrata_raja]{\closesub}
\clearpage

\sect{माघ-शुक्ल-जया-एकादशी-माहात्म्यम्}
\label{sec:vrata-raja-magha-shukla-jaya}

\uvacha{युधिष्ठिर उवाच}

\twolineshloka
{कृष्ण कृष्णाप्रमेयात्मवादिदेव जगत्पते}
{स्वेदजा अण्ड जाश्चैव उद्भिजाश्च जरायुजाः} %॥१॥

\twolineshloka
{तेषां कर्ता विकर्ता त्वं पालकः क्षयकारकः}
{माघस्य कृष्णपक्षे तु षट्तिला कथिता त्वया} %॥२॥

\twolineshloka
{शुक्ले यैकादशी तां च कथयस्व प्रसादतः}
{किन्नामा कोविधिस्तस्याः को देवस्तत्र पूज्यते} %॥३॥

\uvacha{श्रीकृष्ण उवाच}

\twolineshloka
{कथयिष्यामि राजेन्द्र शुक्ले माघन्य या भवेत्}
{जयानाम्रीति विख्याता सर्वपापहरा परा} %॥४॥


\threelineshloka
{पवित्रा पापहन्त्री च कामदा मोक्षदा नृणाम्}
{ब्रह्महत्यापहन्त्री च पिशाचत्वाविनाशिनी}
{नैव तस्या ते चीणे प्रेतत्वं जायते नृणाम्} %॥५॥

\twolineshloka
{नातः परतरा काचिपापन्नी मोक्षदायनी}
{एतस्मात्कारणाद्राजन् कर्तव्येयं प्रयत्नतः} %॥६॥

\twolineshloka
{श्रूयतां राजशार्दूल कथा पौराणिकी शुभा}
{पङ्कजाख्यपुराणेऽस्या महिमा कथितो मया} %॥७॥

\twolineshloka
{एकदा नाकलोके वै इन्द्रो राज्यं चकार ह}
{देवाश्च तत्र सौख्येन निवसन्ति मनोरमे} %॥८॥

\twolineshloka
{पीयूष पाननिरता ह्यप्सरोगणसेविताः}
{नन्दनं तु वनं तत्र पारिजातोपशोभितम्} %॥९॥

\twolineshloka
{रमयन्ति रमन्त्यत्र ह्यप्सरोभिर्दिवौकसः}
{एकदा रममाणोऽसौ देवेन्द्रः स्वेच्छया नृप} %॥१०॥

\twolineshloka
{नर्तयामास हर्षात्स पञ्चाशत्कोटिनायिकाः}
{गन्धर्वास्तत्र गायन्ति गन्धर्वः पुष्पदन्तकः} %॥११॥

\twolineshloka
{चित्रसनश्च नत्रैव चित्रसेनसुता तथा}
{मालिनीति च नाना तु चित्रसेनस्य कामिनी} %॥१२॥

\twolineshloka
{मालिन्यां तु समुत्पन्नः पुष्पवानिति नामतः}
{तस्य पुष्पवतः पुत्रो माल्यवान्नाम नामतः} %॥१३॥

\twolineshloka
{गन्धर्वी पुष्पवत्याख्या माल्यवत्यतिमोहिता}
{कामस्य च शरैस्तीक्ष्णैर्विद्धाङ्गी सा बभूव ह} %॥१४॥

\twolineshloka
{तया भावकटाक्षश्च माल्यवांस्तु वशीकृतः}
{लावण्यरूपसम्पत्या तस्या रूपं नृप शृणु} %॥१५॥

\twolineshloka
{बाहू तस्यास्तु कामेन कण्ठपाशौ कृताविव}
{चन्द्रबद्वदनं तस्या नयने श्रवणायते} %॥१६॥

\twolineshloka
{कर्णी तु शोभितौ तस्याः कुण्डलाभ्यां नृपोत्तम}
{कण्ठो प्रैवेयसंयुक्तो दिव्याभरणभूषितः} %॥१७॥

\twolineshloka
{पीनोन्नतौ कुचौ तस्यास्तौ हेमकलशाविव}
{अतिक्षामं तदुदर मुष्टिमात्रं च मध्यमम्} %॥१८॥

\twolineshloka
{नितम्बौ विपुलौ तस्या विस्तीर्ण जघनस्थलम्}
{चरणौ शोभमानौ तौ रक्तोत्पलसमद्युती} %॥१९॥

\twolineshloka
{ईदृश्यां पुष्पवत्यां स माल्यवानपि मोहितः}
{शक्रस्य परितोषाय नृत्यार्थ तौ समागतौ} %॥२०॥

\twolineshloka
{गायमानौ च तो तत्र ह्यप्सरोगणसङ्गतौ}
{न शुद्धगानं गायेतां चित्तभ्रमसमन्वितौ} %॥२१॥

\twolineshloka
{बद्धदृष्टी तथान्योन्यं कामबाणवशं गतौ}
{ज्ञात्वा लेखर्षभस्तत्र सङ्गतं मानसं तयोः} %॥२२॥

\twolineshloka
{कालक्रियाणां संलोपात्तथा गीतावभञ्जनात्}
{चिन्तयित्वा तु मघवानवज्ञानं तथात्मनः} %॥२३॥

\twolineshloka
{कुपितश्च तयोरित्थं शापं दास्यन्निदं जगौ}
{धिग्वां पापगतौ मूढावाजाभङ्गकरौ मम} %॥२४॥

\twolineshloka
{युवां पिशाचौ भवतं दम्पतीरूपधारिणौ}
{मृत्युलोकमनुप्राप्तौ भुनानौ कर्मणः फलम्} %॥२५॥

\twolineshloka
{एवं मघवता शप्तावुभौ दुःखितमानसौ}
{हिमवन्तमनुप्राप्ताविन्द्रशापविमोहितौ} %॥२६॥

\twolineshloka
{उभी पिशाचतां प्राप्तौ दारुणं दुःखमेव च}
{सन्तप्तमानसौ तत्र महाकृच्छ्रगतावुभौ} %॥२७॥

\twolineshloka
{गन्धं रसं च स्पर्श च न जानीतो विमोहितौ}
{पीडयमानौ तु दाहेन देहपातकरेण च} %॥२८॥

\twolineshloka
{तो ननिद्रामुखं प्राप्तौ कर्मणा तेन पीडितौ}
{परस्परं खादमानौ चरेतुर्गिरिगह्वरम्} %॥२९॥

\twolineshloka
{पीडयमानौ तु शीतेन तुषारप्रभवेण तौ}
{दन्तघर्ष प्रकुर्वाणौ रोमाश्चितवपुर्धरौ} %॥३०॥

\twolineshloka
{ऊचे पिशाचः शीतार्तः स्वपत्नी तु पिशाचिकाम्}
{किमावाभ्यां कृतं पापमत्यन्तं दुःखदायकम्} %॥३१॥

\twolineshloka
{येन प्राप्तं पिशाचत्वं स्वेन दुष्कृतकर्मणा}
{नरकं दारुणं मन्ये पिशाचत्वं च गर्हितम्} %॥३२॥

\twolineshloka
{तस्मात्सर्वप्रयत्नेन पापं नैव समाचरेत्}
{इति चिन्तापरौ तत्र ह्यास्तां दुःखेन कर्शितौ} %॥३३॥

\twolineshloka
{दैवयोगात्तयोः प्राप्ता माघस्यैकादशी सिता}
{जया नाम्नीति विख्याता तिथीनामुत्तमा तिथिः} %॥३४॥

\twolineshloka
{तस्मिन्दिने तु सम्प्राप्ते तावाहारविवर्जितौ}
{आसाते तत्र नृपते जलपानविवर्जितो} %॥३५॥

\twolineshloka
{न कृतो जीवघातश्च न पेत्रफलभक्षणम्}
{अश्वत्थस्य समीपे तु पतितौ दुःखसंयुतो} %॥३६॥

\twolineshloka
{रविरस्तं गतो राजस्तथैव स्थितयोस्तयोः}
{प्राप्ता चैव निशा घोरा दारुणा शीतकारिणी} %॥३७॥

\twolineshloka
{वेपमानौ तु तो तत्र हिमेन च जडीकृतौ}
{परस्परेण संलग्नौ गात्रयो(जयोरपि} %॥३८॥

\twolineshloka
{न निद्रां न रतिं तत्र न तौ सौस्यमविन्दताम्}
{एवं तौ राजशार्दूल शापेनेन्द्रस्य पीडितौ} %॥३९॥

\twolineshloka
{इत्थं तयोर्दुःखितयोर्निर्जमाम तदा निशा}
{जयायास्तु व्रते चीर्ण रात्री जागरणे कृते} %॥४०॥

\twolineshloka
{तयोव्रतप्रभावेण यथा ह्यासीत्तथा शृणु}
{द्वादशीदिवसे प्राप्ते ताभ्यां चीर्णे जयाव्रते} %॥४१॥

\twolineshloka
{विष्णोः प्रभावान्नृपते पिशाचत्वं तयोर्गतम्}
{पुष्पवनीमाल्यवांश्च पूर्वरूपौ बभूवतुः} %॥४२॥

\twolineshloka
{पुरातनस्नेहयुतौ पूर्वालङ्कारसंयुनौ}
{विमानमधिरूढौ तावप्सरोगणसेवितो} %॥४३॥

\twolineshloka
{स्तूयमानौ तु गन्धर्वैस्तुम्बुरुप्रमुम्बैस्तथा}
{हावभावसमायुनौ गतौ नाके मनोरमे} %॥४४॥

\twolineshloka
{देवेन्द्रस्याग्रतो गत्वा प्रणाम चक्रतुर्मुदा}
{तथाविधौ तु तो दृष्ट्रा मघवा विस्मितोऽब्रवीत्} %॥४५॥

\uvacha{इन्द्र उवाच}

\twolineshloka
{वदतं केन पुण्यन पिशाचत्वं विनिर्गतम}
{मम शापवशं प्राप्तौ केन देवेन मोचितौ} %॥४६॥

\uvacha{माल्यवानुवाच}

\twolineshloka
{वासुदेवप्रसादेन जयायाः सुव्रतेन च}
{पिशाचत्वं गतं स्वामिन्सत्यं भक्तिप्रभावतः} %॥४७॥

\twolineshloka
{इति श्रुत्वा वचस्तस्य प्रत्युवाच सुरेश्वरः}
{पवित्रौ पावनौ जातो वन्दनीयौ ममापि च} %॥४८॥

\twolineshloka
{हरिवासरकारी विष्णुभक्तिपरायणौ}
{हरिभक्तिरता ये च शिवभक्तिरतास्तथा} %॥४९॥

\twolineshloka
{अस्माकमपि ते माः पूज्या वन्द्या न संशयः}
{विहरस्व यथासौख्यं पुष्पवत्या सुरालये} %॥५०॥

\twolineshloka
{एतस्मात्कारणाद्राजन् कर्तव्यो हरिवासरः}
{जया नामेति राजेन्द्र ब्रह्महत्यापहारकः} %॥५१॥

\twolineshloka
{सर्वदानानि दत्तानि यज्ञास्तेन कृता नृप}
{सर्वतीर्थेषु सुनातः कृतं येन जयाव्रतम्} %॥५२॥

\twolineshloka
{य करोति नरो भक्त्या श्रद्धायुक्तो जयावतम्}
{कल्पकोटिशतं यावद्वैकुण्ठे मोदते ध्रुवम्} %॥५३॥

\onelineshloka
{पठनाच्यणाद्राजन्नाग्निष्टोमफलं लभेत्} %॥५४॥

॥इति श्रीभविष्योत्तरपुराणे माघशुक्क्लैकादश्या जयाया माहात्म्यं सम्पूर्णम्॥


\hyperref[sec:ekadashi_mahatmyam_vrata_raja]{\closesub}
\clearpage

\sect{फाल्गुन-कृष्ण-विजया-एकादशी-माहात्म्यम्}
\label{sec:vrata-raja-phalguna-krishna-vijaya}

\uvacha{युधिष्ठिर उवाच}

\twolineshloka
{फाल्गुनस्यासिते पक्षे किन्नामैकादशी भवेत्}
{वासुदेव कृपासिन्धो कथयस्व प्रसादतः} %॥१॥

\uvacha{श्रीकृष्ण उवाच}

\twolineshloka
{कथयिष्यामि राजेन्द्र कृष्णा या फाल्गुनी भवेत्}
{विजयेति च सा प्रोक्ता कर्तृणां जयदा सदा} %॥२॥

\twolineshloka
{तस्याश्च व्रतमाहात्म्यं सर्वपापहरं परम्}
{नारदः परिपप्रच्छ ब्रह्माणं कमलासनम्} %॥३॥

\twolineshloka
{फाल्गुनस्यासिते पक्षे विजयानाम या तिथिः}
{तस्या व्रतं सुरश्रेष्ठ कथयस्व प्रसादतः} %॥४॥

\onelineshloka*
{इति पृष्टो मारदेन प्रत्युवाच पितामहः}

\uvacha{ब्रह्मोवाच}

\onelineshloka
{शृणु नारद वक्ष्यामि कथां पापहरां पराम्} %॥५॥

\twolineshloka
{पुरातनं व्रतं ह्येतत्पवित्रं पापनाशनम्}
{यन्न कस्यचिदाख्यातं मयैतद्विजयाव्रतम्} %॥६॥

\twolineshloka
{जयं ददाति विजया नृणां चैवन संशयः}
{रामस्तपोवनं यातो वर्षाण्येव चतुर्दश} %॥७॥

\twolineshloka
{न्यवसत्पञ्चवट्यां तु ससीतश्च सल क्ष्मणः}
{तत्रैव वसतस्तस्य राघवस्य महात्मनः} %॥८॥

\twolineshloka
{रावणेन हृता भार्या सीतानाम्नी तपस्विनी}
{तेन दुःखेन रामोऽसौ मोहमभ्यागतस्तदा} %॥९॥

\twolineshloka
{भ्रमञ्जटायुषं तत्र ददर्श विगतायुषम्}
{कवन्धो निहतः पश्चाद्रमतारण्यमध्यतः} %॥१०॥

\twolineshloka
{राजे विज्ञाप्य तत्सर्व सोऽपि मृत्युवशं गतः}
{सुग्रीवेण समं सख्यमजयं समजायत} %॥११॥

\twolineshloka
{वानराणामनीकानि रामार्थ सङ्गतानि वै}
{ततो हनूमता दृष्टा लोद्याने तु जानकी} %॥१२॥

\twolineshloka
{रामसञ्ज्ञापनं तस्यै दत्तं कर्म महत्कृतम्}
{समेत्य रामेण पुनः सर्व तत्र निवेदितम्} %॥१३॥

\twolineshloka
{अथ श्रुत्वा रामचन्द्रो वाक्यं चैव हनूमतः}
{सुग्रीवानुमतेनैव प्रस्थानं समरोचयत्} %॥१४॥

\twolineshloka
{स गत्वा वानरैः सार्ध तीरं नदनदीपतेः}
{दृष्ट्वाब्धि दुस्तरं रामो विस्मितोऽभूत्कपिप्रियः} %॥१५॥

\twolineshloka
{प्रोत्फुल्ललोचनो भूत्वा लक्ष्मणं वाक्यमब्रवीत्}
{सौमित्र केन पुण्येन तीर्यते वरुणालयः} %॥१६॥

\twolineshloka
{अगाधसलिलैः पूर्णो नीमैः समाकुलः}
{उपायं नैव पश्यामि येनैव सुतरो भवेत्} %॥१७॥

\uvacha{लक्ष्मण उवाच}

\twolineshloka
{आदिदेवस्त्वमेवासि पुराणपुरुषोत्तम}
{बकदाल्भ्यो मुनिश्चात्र वर्तते द्वीपमध्यतः} %॥१८॥

\twolineshloka
{अस्मात्स्थानाद्योजनार्धमाश्रमस्तस्य राघव}
{अनेन दृष्टा ब्रह्माणो बहवो रघुनन्दन} %॥१९॥

\twolineshloka
{तं पृच्छ गत्वा राजेन्द्र पुराणमृषिपुङ्गवम्}
{इति वाक्यं ततः श्रुत्वा लक्ष्मणस्यातिशोभनम्} %॥२०॥

\twolineshloka
{जगाम राघवो द्रष्टुं बदाल्भ्यं महामुनिम्}
{त्रण नाम मुनि मूर्ना रामो विष्णुमिवामराः} %॥२१॥

\twolineshloka
{मुनित्विा ततो राम पुराणपुरुषोत्तमम्}
{केनापि कारणे व प्रविष्टं मानुषी तनुम्} %॥२२॥

\onelineshloka*
{उवाच स ऋषिस्तत्र कुतो राम तवागमः}

\uvacha{राम उवाच}

\onelineshloka
{त्वत्प्रसादादहो विप्र वरुणालयसनिधिम्} %॥२३॥

\twolineshloka
{आगतोऽस्मि ससैन्योऽत्र लङ्कां जेतुं सराक्षसाम्}
{भवतश्चानुकूल्येन तीर्यतेन्धिर्यथा मया} %॥२४॥

\twolineshloka
{तमुपायं वद मुने प्रसादं कुरु सुव्रत}
{एतस्मात्कारणादेव द्रष्टुं त्वाहमुपागतः} %॥२५॥

\uvacha{मुनिरुवाच}

\twolineshloka
{कथयिष्याम्यहं राम बतानामुत्तमं व्रतम्}
{कृतेन येन सहसा विजयस्ते भविष्यति} %॥२६॥

\twolineshloka
{लड़ां जित्वा राक्षसांश्च दीवी कीर्तिमवाप्स्यसि}
{एकाप्रमानसो भूत्वा प्रतमेतत्समाचर} %॥२७॥

\twolineshloka
{फाल्गुनस्यासिते पक्षे विजयैकादशी भवेत्}
{तस्या व्रते कृते राम विजयस्ते भविष्यति} %॥२८॥

\twolineshloka
{निःसंशयं समुद्रं च तरिष्यसि सवानरः}
{विधिस्तु श्रूयतां राम व्रतस्यास्य फलप्रदः} %॥२९॥

\twolineshloka
{दशमीदिवसे प्राप्ते कुम्भमेकं च कारयेत्}
{हेमं वा राजतं वाऽपि तानं वाप्यथ मृन्मयम्} %॥३॥

\twolineshloka
{स्थापयेत्स्थण्डिले कुम्भं जलपूर्ण सपल्लवम्}
{सतधान्यान्य वस्तस्य यवातुपरि विन्यसेत्} %॥३१॥

\twolineshloka
{तस्योपरि न्यसेदेवं हेमं नारायणं प्रभुम्}
{एकादशीदिने प्राते प्रातःस्नानं समाचरेत्} %॥३२॥

\twolineshloka
{निश्चले स्थापित कुम्भे गन्धमाल्यातुलपिते}
{गन्धैर्धपैस्तथा दीपनैवेद्युविविधैरपि} %॥३३॥

\twolineshloka
{दाडिमालिकेरैश्च पूजयेच्च विशेषतः}
{कुम्भाने तहिन राम नेतव्यं भक्तिभावतः} %॥३४॥

\twolineshloka
{रात्रौ जागरणं तब तस्याने कारयेद्बुधः}
{द्वादशीदिवसे प्राप्त मार्तण्डस्योदये नृप} %॥३५॥

\twolineshloka
{नीत्वा कुम्भं जलोद्देशे नद्यां प्रस्रवणे तथा}
{तडागे स्थापयित्वा वा पूजयित्वा यथाविधि} %॥३६॥

\twolineshloka
{दद्यात्सदैवतं कुम्भं ब्राह्मणे वेदपारगे}
{कुम्भेन सह राजेन्द्र महादानानि दापयेत्} %॥३७॥

\twolineshloka
{अनेन विधिना राम यूथपैः सह सङ्गतः}
{कुरु व्रतं प्रयत्नेन विजयस्ते भविष्यति} %॥३८॥

\twolineshloka
{इति श्रुत्वा वचो रामो यथोक्तमकरोत्तथा}
{कृते व्रते स विजयी बभूव रघुनन्दनः} %॥३९॥

\twolineshloka
{अनेन विधिना राजन्ये कुर्वन्ति नरा व्रतम्}
{इहलोके जयस्तेषां परलोकस्तथाऽक्षयः} %॥४०॥


\threelineshloka
{एतस्मात्कारणात्पुत्र कर्तव्यं विजयावतम्}
{विजयायाश्च माहात्म्यं सर्वकिल्बिषनाशनम्}
{पठनाच्छ्रवणात्तस्य वाजपेयफलं लभेत्} %॥४१॥

॥इति श्रीस्कन्दपुराणे फाल्गुनकृष्णैकादश्या विजयानाम्न्या माहात्म्यं सम्पूर्णम्॥


\hyperref[sec:ekadashi_mahatmyam_vrata_raja]{\closesub}
\clearpage

\sect{फाल्गुन-शुक्लामलकी-एकादशी-माहात्म्यम्}
\label{sec:vrata-raja-phalguna-shuklamalaki}

\uvacha{मान्धातोवाच}

\twolineshloka
{वद ब्रह्मन्महाभाग येन श्रेयो भवेन्मम}
{कृपया तब्रह्मयोने यधनुग्राह्यतो मयि} %॥१॥

\onelineshloka*
{सरहस्यं सेतिहासं प्रतानामुत्तमं व्रतम्}

\uvacha{वसिष्ठ उवाच}

\onelineshloka
{कथयाम्यधुना तुभ्यं सर्ववतफलप्रदम्} %॥२॥

\twolineshloka
{, आमलक्या व्रतं राजन् महापातकनाशनम्}
{मोक्षदं सर्वलोकानां गोसहस्नफलप्रदम्} %॥३॥

\twolineshloka
{अत्रयोदाहरन्तीममितिहासं पुरातनम्}
{यथामुक्तिमनुप्राप्तो व्याधो हिंसासमन्वितः} %॥४॥

\twolineshloka
{वैदिशं नाम नगरं हृष्टपुष्टजनावृतम्}
{ब्राह्मणैः क्षत्रियैश्यः शूद्रेश्च समलङ्कृतम्} %॥५॥

\twolineshloka
{रुचिरं नृपशार्दूल ब्रह्मघोषनिनादितम्}
{न नास्तिको दुष्कृतिकस्तस्मिन्पुरवरे सदा} %॥६॥

\twolineshloka
{तत्र सोमान्वयो राजा विख्यातः शशिबिन्दवः}
{राजा चैत्ररथो नाम धर्मात्मा सत्यसङ्गरः} %॥७॥

\twolineshloka
{नागायुतबलः श्रीमाञ्छस्त्रशास्त्रार्थपारगः}
{तस्मिन्छासति धर्मज्ञे धर्मात्मनि धरां प्रभो} %॥८॥

\twolineshloka
{कृपणो नैव कुत्रापि दृश्यते नैव निर्धनः}
{सुकालः क्षेममारोग्यं न दुर्भिक्षं न चेतयः} %॥९॥

\twolineshloka
{विष्णुभक्तिरता लोकास्तस्मिन्पुरवरे सदा}
{हरिपूजारताश्चैव राजा चापि विशेषतः} %॥१०॥

\twolineshloka
{न शुक्लां नैव कृष्णां च द्वादशी भुञ्जते जनाः}
{सर्वधर्मान्परित्यज्य हरिभक्तिपरायणाः} %॥११॥

\twolineshloka
{एवं संवत्सरा जग्मुर्बहवो राजसत्तम}
{जनस्य सौख्ययुक्तस्य हरिभक्तिरतस्य च} %॥१२॥

\twolineshloka
{अथ कालेन सम्प्राप्ता द्वादशी पुण्यसंयुता}
{फाल्गुनस्य सिते पक्षे नाम्ना ह्यामलकी स्मृता} %॥१३॥

\twolineshloka
{तामवाप्य जनाः सर्वे बालकाः स्थविरा नृप}
{नियमं चोपवासं च सर्वे चकुर्नरा विभो} %॥१४॥

\twolineshloka
{प्रहालं व्रतं ज्ञात्वा स्नानं कृत्वा नदीजल}
{नत्र देवालये राजा लोकयुक्तो महाप्रभुः} %॥१५॥

\twolineshloka
{पूर्णकुम्भमवस्थाप्य छोपानह संयुतम्}
{पञ्चरत्नसमायुतं दिव्यगधाधिवालितम्} %॥६॥

\twolineshloka
{दीपमालावितं चैव जामदग्न्यसमन्वितम्}
{पूजयामातुरव्यमा धात्री च मुनिभिर्जनाः} %॥१७॥

\twolineshloka
{जामदग्न्य नमस्तेऽस्तु रेगुकानन्दवर्धन}
{मलकीतच्छाय भुक्तिमुक्तिवरप्रद} %॥१८॥

\twolineshloka
{धात्रि धातृसमुद्भूते सर्वपातकनाशिनि}
{आमलकि नमस्तुभ्यं सहामा यो दकं मम} %॥१९॥

\twolineshloka
{धात्रि ब्रह्म स्वरूपासि त्वं तु रामे ग पूजिता}
{प्रदक्षिण विधानेन सर्वनापहरा भव} %॥२०॥

\twolineshloka
{तत्र जागरणं चक्रुर्जनः सर्वे स्वभक्तिः}
{एतस्मिन्नेव काले तु व्याधस्तत्र समागतः} %॥२१॥

\twolineshloka
{क्षुधाश्रमपरिव्यातो महामारण पीडितः}
{कुटुम्बार्थ जीवनाती सर्वधर्मबहि कृतः} %॥२२॥

\twolineshloka
{जागरं तत्र सोऽपश्यदामलक्यां झुधान्वितः}
{दीपमालाकुलं दृष्ट्वा तत्रैव निषसाद सः} %॥२३॥

\twolineshloka
{किमेतदिनि सञ्चिन्त्य प्रातबाविस्मयं भृशम्}
{ददर्श कुम्भं तत्रस्थं देवं दामोदरं तथा} %॥२४॥

\twolineshloka
{ददर्शामलकीवृक्षं तत्रस्थाश्चैव दीपकान्}
{वैष्णवं च तथाऽख्यानं शुश्राव पठतां नृणाम्} %॥२५॥

\twolineshloka
{एकादश्याश्च माहात्म्यं शुश्राव क्षुधितोऽपि सन्}
{जाग्रतस्तस्य सा रात्रिर्गता विस्मितचेतसः} %॥२६॥

\twolineshloka
{ततः प्रभातसमये विविशुनगरं जनाः}
{व्याधोऽपि गृहमा गत्य बुजे प्रीतमानसः} %॥२७॥

\twolineshloka
{ततः कालेन महता व्यायः पञ्चत्वमागतः}
{एकादश्याः प्रभावेण रात्री जागरणेन च} %॥२८॥

\twolineshloka
{राज्यं प्रपेदे सुमहच्चतुरङ्गवलाधिनम्}
{जय-नीनाम नगरी तब राजा विदूरथः} %॥२९॥

\twolineshloka
{तस्मात्स तनयो जज्ञ नाम्ना वतुरथो बली}
{चतुरङ्गबलोपेतो धनधान्यसमन्वितः} %॥३०॥

\twolineshloka
{दशायुतानि ग्रामागां बुभुजे भयवर्जिनः}
{तेजसादित्यसदृशः कान्त्या चन्द्रसमप्रभः} %॥३१॥

\twolineshloka
{पराक्रमे विष्णुप्तमः क्षमया पृथिवीलमः}
{धार्मिकः सत्यवादी शक्तिपरायणः} %॥३२॥

\twolineshloka
{ब्रह्मज्ञः कर्मशीलश्च प्रजापालनतत्परः}
{यजते विविधान कमान राजा परदर्पहा} %॥३३॥

\twolineshloka
{दानानि विविधान्येव प्रददाति च सर्वदा}
{एकदा मृगयां यानो देवान्मार्गपरिच्युतः} %॥३४॥

\twolineshloka
{न दिशो नैव विदिशो वेत्ति तत्र महीपतिः}
{उपधाय च दोलनकाकी गहने वने} %॥३५॥

\twolineshloka
{श्रान्तश्च क्षुधितोऽत्यन्तं संविवेश महीपतिः}
{अवान्तरे न पर्वतान्तरवासभाक्} %॥३६॥

\twolineshloka
{आययौ तत्र यत्रास्ते राजा परबलार्दनः}
{कृतवैरास्ते राज्ञा सर्वदेवोपतापिताः} %॥३७॥

\twolineshloka
{परिवार्य ततस्तस्थू राजानं भूरिदक्षिणम्}
{हन्यता हाला वायं पूर्व वरविरुद्धधीः} %॥३८॥

\twolineshloka
{अनेन निहताः पूर्व पितरौ भ्रातरः सुताः}
{पौवाश्च व मातुलाश्च निपातिताः} %॥३९॥


\threelineshloka
{निष्कासिताश्च स्वस्थानाद्विक्षिप्ताश्च दिशो दश}
{एनपदुःस्वा ते सर्वे तत्रैनं हन्तुमुद्यताः}
{पाशश्च पट्टिशैः खङ्गैर्बाणैर्धनुषि संस्थितैः} %॥४०॥

\twolineshloka
{शस्त्राणि समापतन्ति न वै शरीरे प्रविशन्ति}
{तस्यातिचापि सर्वे हतशस्त्रसङ्घा म्लेच्छा अरिजीवदेहाः} %॥४१॥

\twolineshloka
{यदापि चलितुं तत्र न शेकुल्तेरयो भृशम्}
{शस्त्रागि कुण्ठता जग्गः तवेषां हतचेतसाम्} %॥४२॥

\twolineshloka
{दीना बभूवुस्ते सर्वे ये तं हन्तुं समागताः}
{एतस्मिन्नेव काले तु तस्य राज्ञः शरीरतः} %॥४३॥

\onelineshloka
{निन्मृता प्रमदा ह्येका सर्वावयवशोभना} %॥४४॥

\twolineshloka
{दिव्य पता दिव्याभरणभूषिता}
{दिव्यमाल्याम्बरधरा भृकुटीकुटिलानना} %॥४५॥

\twolineshloka
{स्फुलि गाभ्यो च नेत्राभ्यां पावकं वमती बहु}
{चक्रोद्यतकरा चैव कालरात्रिरिवापरा} %॥४६॥

\twolineshloka
{अभ्यधामा सङ्कुद्धा म्लेच्छानत्यन्तदुःखितान्}
{निहताश्च यदा म्लेच्छास्ते विकर्मरतास्तथा} %॥४॥

\twolineshloka
{ततो राजा विबुद्धः सन् ददर्श महदद्भतम्}
{हतान् म्लेच्छगणान् दृष्ट्वा राजा हर्षमवाप सः} %॥४८॥

\twolineshloka
{इह न हता म्लेच्छा अत्यन्तं वैरिणो मम}
{केन चेदं महत्कर्म कृतमस्मद्धितार्थिना} %॥४९॥

\twolineshloka
{समय काले तु वागुवाचाशरीरिणी}
{तं स्थितं नृपतिं दृष्ट्वा निकामं विस्मयान्वितम्} %॥५०॥

\twolineshloka
{शरणं केशवादन्यो नास्ति कोपि द्वितीयकः}
{इति श्रुत्वाकाशवाणी विस्मयोत्तर} %॥५१॥

\twolineshloka
{वनात्तस्मात्स कुशली समायातः स भूमिभुक्}
{राज्यं चकार धर्मात्मा धरायो तेशवत्} %॥५२॥

\uvacha{वसिष्ठ उवाच}

\twolineshloka
{तस्मादामलकी राजन् ये कुर्वन्ति नरोत्तमाः}
{ते यान्ति के लोकं नात्र कार्या विचारणा} %॥५३॥

॥इति श्रीब्रह्माण्डपुराणे आमलक्याख्यफाल्गुनशुक्लैकादशीव्रतम्॥


\hyperref[sec:ekadashi_mahatmyam_vrata_raja]{\closesub}
\clearpage

\sect{चैत्र-कृष्ण-पापमोचनी-एकादशी-माहात्म्यम्}
\label{sec:vrata-raja-chaitra-krishna-papamochani}

\uvacha{युधिष्ठिर उवाच}

\twolineshloka
{फाल्गुनस्य सिते पक्षे श्रुता साऽमलकी मया}
{चैत्रस्य कृष्णपक्षेन किं नामैकादशी भवेत्} %॥१॥

\twolineshloka
{को विधिः किं फलं तस्या ब्रूहि कृष्ण ममाननः}
{श्रीकृष्ण शृणु राजेन्द्र वक्ष्यामि पापमोचनिकाव्रतम्} %॥२॥

\onelineshloka*
{यल्लोमशोऽब्रवीत्पृष्टो मान्धात्रा चक्र}

\uvacha{मान्धातोवाच}

\onelineshloka
{भगवञ्छोतुमिच्छामि लोकानां हितकाम्यया} %॥३॥

\twolineshloka
{चैत्रमास्यसिते पक्ष नामैकादशी भवेत्}
{को विधिः किं फलं तस्याः कथयस्व प्रसादतः} %॥४॥

\uvacha{लोमश उवाच}

\twolineshloka
{चैत्रमास्यसिते पक्षे नाम्ना वै पापमोचनी}
{एकादशी समाख्याता पिशाचत्वविनाशिनी} %॥५॥

\twolineshloka
{शृणु तस्याः प्रवक्ष्यामि क मदां सिद्धिदां नृप}
{कथां विचित्रां शुभदा पापी धर्मका} %॥६॥

\twolineshloka
{पुरा चैत्ररथोद्देशे अप्सरोगणसेविते}
{वसन्तसमये प्राते पुष्पैराकुलिते वने} %॥७॥

\twolineshloka
{गन्धर्वकन्यास्तत्रैव रमन्ति सह किन्नरैः}
{पाकशालनमुख्याश्च क्रीडन्ते च दिवौकसः} %॥८॥

\twolineshloka
{नापरं सुन्दर किश्चिदनाच्चैत्ररथानम्}
{तस्मिन्वने तु मुनयस्त पन्ति बहुलं तपः} %॥९॥

\twolineshloka
{वैस्तु मघवा रमते मधुमाधवौ}
{एको मुनिवरस्तव मेधावी नाम नामतः} %॥१०॥

\twolineshloka
{ॐ मुनिवरं मोहनायोपचक्रमे}
{मजुयोति विख्याना भावं तस्य विचिन्वती} %॥११॥

\twolineshloka
{क्रोशमात्र स्थिता तस्य भयदाश्रमसन्निधौ}
{गायन्ती मधुरं साधु पीडयन्ती विपश्चिकाम्} %॥१२॥

\twolineshloka
{गायानी तोमथालोक्य पुष्पचन्दनवेष्टिताम्}
{कामोऽपि विजयाकाङ्क्षी शिवभक्तं मुनीश्वरम्} %॥१३॥

\twolineshloka
{तस्याः शरीरसंसर्ग शिववैरमनुस्मरन्}
{कृत्वा भ्रुवौ धनुष्कोटी गुणं कृत्वा कटाक्षकम्} %॥१४॥

\twolineshloka
{मार्गणो नयने कृत्वा पक्षयुक्तौ यथाक्रमम्}
{कुचौ कृत्वा पटकुटी विजयायोपसंस्थितः} %॥१५॥

\twolineshloka
{मञ्जुघोषाभवत्तत्र कामस्येव वरूथिनी}
{मेधाविनं मुनिं दृष्ट्वा सापि कामेन पीडिता} %॥१६॥

\twolineshloka
{यौवनोदिनदेहोऽसौ मेधाव्यतिविराजते}
{सितोपवीतसहितो दण्डी स्मर इवापरः} %॥१७॥

\twolineshloka
{मञ्जुघोषा स्थिता तत्र दृष्ट्वा तं मुनिपुङ्गवम्}
{मदनस्य वशं प्राप्ता मन्दं मन्दमगायत} %॥१८॥

\twolineshloka
{रणद्रलयसंयुक्तां शिञ्जपुरमेखलाम्}
{गायन्ती भावसंयुक्तां विलोक्य मुनिपुङ्गवः} %॥१९॥

\twolineshloka
{मदनेन ससैन्येन नीतो मोहवशं बलात्}
{मञ्जुघोषा समागम्य मुनिं दृष्ट्वा तथाविधम्} %॥२०॥

\twolineshloka
{हावभावकटाक्षैस्तु मोहयामास चाङ्गना}
{अधः संस्थाप्य वीणां सा सस्वजे तं मुनीश्वरम्} %॥२१॥

\twolineshloka
{वल्लीवाकुलिता वृक्षं वातवेगेन वेपिता}
{सोऽपि रेमे तया सार्ध मेधावी मुनिपुङ्गवः} %॥२२॥

\twolineshloka
{तस्मिन्नेव वनोदेशे दृष्ट्वा तदेहमुत्तमम्}
{शिवतत्त्वं स विस्मृत्य कामतत्त्ववशं गतः} %॥२३॥

\twolineshloka
{न निशां न दिनं सोऽपि रमजानाति कामुकः}
{बहुलश्च गतः कालो मुनराचारलोपकः} %॥२४॥

\twolineshloka
{मजुघोषा देवलोकगमनायोपचक्रमे}
{गच्छन्ती प्रत्युवाचाथ रमन्तं मुनिपुङ्गवम्} %॥२५॥

\onelineshloka*
{आदशो दीयतां ब्रह्मन् स्वधामगमनाय मे}

\uvacha{मेधाव्युवाच}

\onelineshloka
{अव त्वं समायाता प्रदोषादौ वरानने} %॥२६॥

\twolineshloka
{यावत्प्रभातसन्ध्या स्यात्तावत्तिष्ठं ममान्तिके}
{इति श्रुत्वा मुने वाक्यं भयभीता बभूव सा} %॥२७॥

\twolineshloka
{पुनर्वै रमयामास तं मुनि नृपसत्तम}
{मुनिशापभयागीता बहुलान्परिवत्सरान्} %॥२८॥

\twolineshloka
{वर्षाणि सप्तपञ्चाशनवमासान् दिनत्रयम्}
{सा रेमे सुनिना तस्य निशार्धमिव चाभवत्} %॥२९॥

\twolineshloka
{सा तं पुनरुवाचाथ तस्मिन्काले गत मुनिम्}
{आदेशो दीयता ब्रह्मन गन्तव्यं स्वगृहे मया} %॥३०॥

\uvacha{मेधाव्युवाच}

\twolineshloka
{प्रातःकालोऽधुनेवारते श्रूयतां वचनं मम}
{कुर्वे सन्ध्यां दिनं यावत्तावत्त्वं सुस्थिरा भव} %॥३१॥

\twolineshloka
{इति वाक्यं मुनेः श्रुत्वा भयानन्दसमाकुलम्}
{स्मितं कृत्वा तु सा किश्चित्प्रत्युवाच सुविस्मिता} %॥३२॥

\uvacha{अप्सरा उवाच}

\twolineshloka
{कियत्प्रमाणा विप्रेन्द्र तव सन्ध्या गताः किल}
{मयि प्रसादं कृत्वा तु गतः कालो विचार्यताम्} %॥३३॥

\twolineshloka
{इति तस्या वचः श्रुत्वा विस्मयोत्फुल्ललोचनः}
{सध्यात्वा हृदि विप्रेन्द्रःप्रणाममकरोत्तदा} %॥३४॥

\twolineshloka
{समाश्च सप्तपञ्चाशद्गता मम तया सह}
{नेत्राभ्यां विस्फुल्लिङ्गान्स मुश्चमानोऽतिकोपनः} %॥३५॥

\twolineshloka
{कालरूपां च तां दृष्ट्वा तपसः क्षयकारिणीम्}
{दुःखार्जितं मम तपो नीतं तदनया क्षयम्} %॥३६॥

\twolineshloka
{विचार्येत्थं स कम्पोष्ठो मुनिस्तु व्याकुलेन्द्रियः}
{तां शशाप च मेधावी त्वं पिशाची भवेति हि} %॥३७॥

\twolineshloka
{धिक्त्वा पापे दुराचारे कुलटे पातकप्रिये}
{तस्य शापेन सा दग्धा विनयावनता स्थिता} %॥३८॥

\twolineshloka
{उवाच वचनं सुभूः प्रसादं वाञ्छती मुनिम्}
{कृत्वा प्रसादं विप्रेन्द्र शापस्योपशमं कुरु} %॥३९॥

\twolineshloka
{सतां सङ्गेहि भवति मित्रत्वं सप्तमे पद}
{त्वया सह मम ब्रह्मन् गताः सुबहवः समाः} %॥४०॥

\onelineshloka*
{एतस्मात्कारणात्स्वामिन् प्रसादं कुरु सुव्रत}

\uvacha{मुनिरुवाच}

\onelineshloka
{शृणु मे वचनं भद्रे शापानुग्रहकारकम्} %॥४१॥

\twolineshloka
{किं करोमि त्वया पापे क्षयं नीतं महत्तपः}
{चैत्रस्य कृष्णपक्षे या भवेदेकादशी शुभा} %॥४२॥

\twolineshloka
{पापमोचनिका नाम सर्वपापक्षयङ्करी}
{तस्या व्रते कृते सुभ्रु पिशाचत्वं प्रयास्यति} %॥४३॥

\twolineshloka
{इत्युक्त्वा तां स मधावी जगाम पितुराश्रमम्}
{तमागतं समालोक्य च्यवनःप्रत्युवाच ह} %॥४४॥

\onelineshloka*
{किमेतद्विहितं पुत्र त्वया पुण्यक्षयः कृतः}

\uvacha{मेधाव्युवाच}

\onelineshloka
{पापं कृतं महत्तात रमिता चाप्सरा मया} %॥४५॥

\onelineshloka*
{प्रायश्चित्तं ब्रूहि मम येन पापक्षयो भवेत्}

\uvacha{च्यवन उवाच}

\onelineshloka
{चैत्रस्य चासिते पक्ष नाम्रा वै पापमोचनी} %॥४६॥

\twolineshloka
{अस्या व्रते कृते पुत्र पापराशिः क्षयं व्रजेत्}
{इति श्रुत्वा पितुः वाक्यं कृतं तेन व्रतोत्तमम्} %॥४७॥

\twolineshloka
{गतं पापं क्षयं तस्य पुण्ययुक्तो बभूव सा}
{साप्येवं मञ्जुघोषा च कृत्वा तद्रूतमुत्तमम्} %॥४८॥

\twolineshloka
{पिशाचत्वविनिर्मुक्ता पापमोचनिकाव्रतात्}
{दिव्यरूपधरा भूत्वा गता नाकं वराप्सराः} %॥४९॥

\uvacha{लोमश उवाच}

\twolineshloka
{इत्थम्भूतप्रभावं हि पापमोचनिकाव्रतम्}
{पापमोचनिको राजन् ये कुर्वन्तीह मानवाः} %॥५०॥

\twolineshloka
{तेषां पापं च यत्किञ्चित्तत्सर्व क्षीणतां व्रजेत्}
{पठनाच्छ्रवणादस्या गोसहस्रफलं लभेत्} %॥५१॥


\threelineshloka
{ब्रह्महा हेमहारी च सुरापो गुरुतल्पगः}
{व्रतस्य चास्य करणात् पापमुक्ता भवन्ति ते}
{बहुपुण्यप्रदं ह्येतत्करणागतमुत्तमम्} %॥५२॥

॥इति श्रीभविष्यपुराणे पापमोचनिकाख्यचैत्रकृष्णैकादशीमाहात्म्यं सम्पूर्णम्॥


\hyperref[sec:ekadashi_mahatmyam_vrata_raja]{\closesub}
\clearpage

\sect{चैत्र-शुक्ल-कामदा-एकादशी-माहात्म्यम्}
\label{sec:vrata-raja-chaitra-shukla-kamada}

\uvacha{युधिष्ठिर उवाच}

\twolineshloka
{वासुदेव नमस्तुभ्यं कथयस्व ममाग्रतः}
{चैत्रस्य शुक्लपक्षे तु किन्नामकादशी भवेत्} %॥१॥

\uvacha{श्रीकृष्ण उवाच}

\twolineshloka
{शृणुष्वैकमना राजन् कथामेको पुरातनीम्}
{वसिष्ठो यामकथयत्प्राग्दिलीपाय पृच्छते} %॥२॥

\uvacha{दिलीप उवाच}

\twolineshloka
{भगकछोतुमिच्छामि कथयस्व प्रसा दतः}
{चैत्रे मासि सिते पक्षे किन्नामैकादशी भवेत्} %॥३॥

\uvacha{वसिष्ठ उवाच}

\twolineshloka
{साधु पृष्टं नृपश्रेष्ठ कथयामि तवाग्रतः}
{चत्रस्य शुक्लपक्षे तु कामदा नाम नामतः} %॥४॥

\twolineshloka
{एकादशी पुण्यतमा पापेन्धनदवानलः}
{शृणु राजन् कथामेतां पापन्नी पुत्रदायिनीम्} %॥५॥

\twolineshloka
{पुरा भोगिपुरे रम्ये हेमरत्नविभूषिते}
{पुण्डरीकमुखा नागा निवसान्त मदोत्कटाः} %॥६॥

\twolineshloka
{तस्मिन्पुरे पुण्डरीको राजा राज्यं करोति च}
{गन्धर्वैः किन्नरैश्चैव ह्यप्सरोभिः स सेव्यते} %॥७॥

\twolineshloka
{वराप्सरा तु ललिता गन्धर्वो ललितस्तथा}
{उभौ रागेण संयुक्तौ दम्पती कामपीडितौ} %॥८॥

\twolineshloka
{रेमाते स्वगृहे रम्ये धनधान्ययुते सदा}
{ललितायास्तु हृदये पतिर्वसति सर्वदा} %॥९॥

\twolineshloka
{हृदयें तस्य ललिता नित्यं वसति भामिनी}
{एकदा पुण्डरीकाद्याः क्रीडन्तः सदसि स्थिताः} %॥१०॥

\twolineshloka
{गीतगानं प्रकुरुते ललितो दयितां विना}
{पदबन्धे स्खलजिह्वो बभूव ललितां स्मरन्} %॥११॥

\twolineshloka
{मनोभावं विदित्वाऽस्य कोटो नागसत्तमः}
{पदबन्धच्युतिं तस्य पुण्डरीके न्यवेदयत्} %॥१२॥

\twolineshloka
{क्रोधसंरक्तनयनः पुण्डरीकोऽभवत्तदा}
{शशाप ललितं तत्र मन्दनातुरचेतसम्} %॥१३॥

\twolineshloka
{राक्षसो भव दुर्बुद्धे क्रव्यादः पुरुषादकः}
{यतः पत्नीवशो जातो गायश्चैव ममाग्रतः} %॥१४॥

\twolineshloka
{वचनात्तस्य राजेन्द्र रक्षोरूपो बभूव ह}
{रौद्राननो विरूपाक्षो दृष्टमात्रो भयङ्करः} %॥१५॥

\twolineshloka
{बाहू योजनविस्तीर्णी मुखकन्दरसन्निभम्}
{चन्द्रसूर्यनिभे नेत्रे ग्रीवा पर्वतसन्निभा} %॥१६॥

\twolineshloka
{नासारन्ध्रे तु विवरे चाधरौ योजनार्धकौ}
{शरीरं तस्य राजेन्द्र उत्थितं योजनाष्टकम्} %॥१७॥

\twolineshloka
{ईदृशो राक्षसः सोऽभूद्भुञ्जानः कर्मणः फलम्}
{ललिता तमथालोक्य स्वपति विकृताकृतिम्} %॥१८॥

\twolineshloka
{चिन्तयामास मनसा दुःखेन महतार्दिता}
{किं करोमि व गच्छामि पतिः शापेन पीडितः} %॥१९॥

\twolineshloka
{इति संस्मृत्य मनसा न शर्म लभते तु सा}
{चचार पतिना सार्ध ललिता गहने वन} %॥२०॥

\twolineshloka
{बभ्राम विपिने दुर्गे कामरूपः स राक्षसः}
{निर्वृणः पापनिरतो विरूपः पुरुषदकः} %॥२१॥

\twolineshloka
{न सुखं लभते रात्रौ न दिवा तापपीडितः}
{ललिता दुःखितातीव पतिं दृष्ट्वा तथाविधम्} %॥२२॥

\twolineshloka
{भ्रमन्ती तेन सार्ध सा रुदती गहने वने}
{कदाचिदगमद्विन्ध्यशिखरे बहुकौतुके} %॥२३॥

\twolineshloka
{ऋष्यशृङ्गमुनेस्तत्र दृष्ट्वाश्रमपदं शुभम्}
{शीघ्रं जगाम ललिता विनयावनता स्थिता} %॥२४॥

\twolineshloka
{प्रत्युवाच मुनिदृष्ट्वा का त्वं कस्य सुता शुभम्}
{किमर्थ त्वमिहायाता सत्यं वद ममाप्रतः} %॥२६॥

\uvacha{ललितोवाच}

\twolineshloka
{वीरधन्तेति गन्धर्वः सुतां तस्य महात्मनः}
{ललितां नाम मां विद्धि पत्यर्थमिह चागताम्} %॥२६॥

\twolineshloka
{भर्ता में शापदोषेण राक्षसोऽभून्महामुने}
{रौद्ररूपो दुराचारस्तं दृष्ट्वा नास्ति मे सुखम्} %॥२७॥

\twolineshloka
{साम्प्रतं शाधि मां ब्रह्मन् प्रायश्चित्तं करोमि तत्}
{येन पुण्येन मे भर्ता राक्षसत्वाद्विमुच्यते} %॥२८॥

\uvacha{ऋषिरुवाच}

\twolineshloka
{चैत्रमासस्य रम्भोरु शुक्लपक्षेऽस्ति साम्प्रतम्}
{कामदैकादशी नाम्ना या कृता कामदा नृणाम्} %॥२९॥

\twolineshloka
{कुरुष्व तद्वतं भद्रे विधिपूर्व मयोदितम्}
{तस्य व्रतस्य यत्पुग्यं तत्स्वभत्रै प्रदीयताम्} %॥३०॥

\twolineshloka
{दत्ते पुण्ये क्षणात्तस्य शापदोषः प्रशाम्यति}
{इति श्रुत्वा मुनेर्वाक्यं ललिता हर्विपाभवत्} %॥३१॥

\twolineshloka
{उपोष्यैकादशी राजन्द्वादशी दिवसे तदा}
{विप्रस्यैव समीपे तु वासुदेवाग्रतः स्थिता} %॥३२॥

\twolineshloka
{वाक्यमूचे तु ललिता स्वपत्युत्तारणाय वै}
{मया तु यद्वतं चीर्ण कामदाया उपोषणम्} %॥३३॥

\twolineshloka
{तस्य पुण्यप्रभावेण गच्छत्वस्य पिशाचता}
{ललितावचनादेवं वर्तमानोपि तत्क्षगे} %॥३४॥

\twolineshloka
{गतपापः सललितो दिव्य देहो बभूव ह}
{राक्षसत्वं गतं तस्य प्राप्तो गन्धर्षतां पुनः} %॥३५॥

\twolineshloka
{हेमरत्नसमाकीणों रेमे ललितया सह}
{तौ विधानं समारूढौ पूर्वरूपाधिकावुभौ} %॥३६॥

\twolineshloka
{दम्पती चापि शोभेता कामदायाः प्रभावतः}
{इति ज्ञात्वा नृपश्रेष्ठ कर्तव्यैषा प्रयत्नतः} %॥३७॥

\twolineshloka
{लोकानां च हितार्थाय तबाने कथिता मया}
{ब्रह्महत्यादिपापनी पिशाचत्वविनाशिनी} %॥३८॥

\twolineshloka
{नातः परतरा काचित्रैलोक्ये सचराचरे}
{पठनाच्छ्रवणाद्वाऽपि वाजपेयफलं लभेत्} %॥३९॥

॥इति श्रीवाराहपुराणे कामदा नाम चैत्रशुक्लैकादशीमाहात्म्यं सम्पूर्णम्॥


\hyperref[sec:ekadashi_mahatmyam_vrata_raja]{\closesub}
\clearpage

\sect{वैशाख-कृष्ण-वरूथिनी-एकादशी-माहात्म्यम्}
\label{sec:vrata-raja-vaishakha-krishna-varuthini}

\uvacha{युधिष्ठिर उवाच}

\twolineshloka
{वैशाखस्यासिते पक्षे किन्नामैकादशी भवेत्}
{महिमानं कथय मे वासुदेव नमोऽस्तु ते} %॥१॥

\uvacha{श्रीकृष्ण उवाच}

\twolineshloka
{सौभाग्यदायिनी राजनिह लोके परत्र च}
{वैशाखकृष्णपक्षे तु नाम्ना चैव वरूथिनी} %॥२॥

\twolineshloka
{वरूथिन्या व्रतेनैव सौख्यं भवति सर्वदा}
{पापहानिश्च भवति सौभाग्यप्राप्तिरेव च} %॥३॥

\twolineshloka
{दुर्भगा या करोत्येना सा स्त्री सौभाग्यमाप्नुयात्}
{लोकानां चैव सर्वेषां भुक्तिमुक्तिप्रदायिनी} %॥४॥

\twolineshloka
{सर्वपापहरा नृणां गर्भवासनिकृन्तनी}
{वरूथिन्या व्रतेनैव मान्धाता स्वर्गतिं गतः} %॥५॥

\twolineshloka
{धुन्धुमारादयश्चान्ये राजानो बहवस्तथा}
{ब्रह्मकपालनिर्मुक्तो बभूव भगवान्भवः} %॥६॥

\twolineshloka
{दशवर्षसहस्राणि तपस्तप्यति यो नरः}
{तचुल्यं फलमाप्नोति वरूथिन्या व्रतादपि} %॥७॥

\twolineshloka
{कुरुक्षेत्रे रविग्रहे स्वर्णभारं ददाति यः}
{तत्तुल्यं फलमाप्नोति वरूथिन्या व्रतान्नरः} %॥८॥

\twolineshloka
{श्रद्धावान्यस्तु कुरुते वरूथिन्या व्रतं नरः}
{वाञ्छितं लभते सोऽपि इह लोके परत्र च} %॥९॥

\twolineshloka
{पवित्रा पावनी ह्येषा महापातकनाशिनी}
{भुक्तिमुक्तिप्रदा चापि कर्मणां नृपसत्तम} %॥१०॥

\twolineshloka
{अश्वदानान्नृपश्रेष्ठ गजदानं विशिष्यते}
{गजदानाद्भूमिदानं तिलदानं ततोऽधिकम्} %॥११॥

\twolineshloka
{ततः सुवर्णदानं तु अन्नदानं ततोऽधिकम्}
{अन्नदानात्पर दानं न भूतं न भविष्यति} %॥१२॥

\twolineshloka
{पितृदेवमनुष्याणां तृप्तिरनेन जायते}
{तत्सम कविभिः प्रोक्तं कन्यादानं नृपोत्तम} %॥१३॥

\twolineshloka
{धेनुदानं च तत्तुल्यमित्याह भगवान् स्वयम्}
{प्रोक्तेभ्यः सर्वदानेभ्यो विद्यादानं विशिष्यते} %॥१४॥

\twolineshloka
{तत्फलं समवाप्नोति नरः कृत्वा वरूथिनीम्}
{कन्यावित्तेन जीवन्ति ये नराः पापमोहिताः} %॥१५॥

\twolineshloka
{ते नरा नरकं यान्ति यावदाभूतसम्प्लवम्}
{तस्मात्सर्वप्रयत्नेन न ग्राह्यं कन्यकाधनम्} %॥१६॥

\twolineshloka
{यञ्च गृहाति लोभेन कन्या क्रीत्वा च तद्धनम्}
{'सोन्यजन्मनि राजेन्द्र ओतुर्भवति निश्चि} %॥१७॥

\twolineshloka
{कन्यां वित्तेन यो दद्याद्यथाशक्ति स्वलङ्कृताम्}
{तत्पुण्यसङ्ख्यां कर्तुं हि चित्रगुप्तो न वेत्यलम्} %॥१८॥


\threelineshloka
{तत्फलं समवाप्नोति नरः कृत्वा वरूथिनीम्}
{कांस्यं मांसं मसूरानं चणकान् कोद्रवांस्तथा}
{शाकं मधु परान्नं च पुनर्भोजनमैथुने} %॥१९॥

\twolineshloka
{वैष्णवव्रतकर्ता च दशम्यां दश वर्जयेत्}
{द्यूतक्रीडां च निद्रां च ताम्बूलं दन्तधावनम्} %॥२०॥

\twolineshloka
{परापवादं पैशुन्यं पतितैः सह भाषणम्}
{क्रोध चैवानृतं वाक्यमेकादश्यां विवर्जयेत्} %॥२१॥

\twolineshloka
{कांस्यं मांसं मसूरांश्च क्षौद्रं वितथभाषणम्}
{व्यायामश्च प्रयासं च पुनर्भोजनमैथुने} %॥२२॥


\threelineshloka
{क्षौर तैलं परानं च द्वादश्यां परिवर्जयेत्}
{अनेन विधिना राजन्विहिता यैर्वरूथिनी}
{सर्वपापक्षयं कृत्वा दद्यात्प्रान्तेऽक्षयां गतिम्} %॥२३॥

\twolineshloka
{रात्री जागरणं कृत्वा पूजितो यर्जनार्दनः}
{सर्वपापविनिर्मुक्तास्ते यान्ति परमो गतिम्} %॥२४॥

\twolineshloka
{तस्मात्सर्वप्रयत्नेन कर्तव्या पापभीरुभिः}
{क्षपारितनयादीतैर्नरदेव वरूथिनीम्} %॥२५॥

\twolineshloka
{पठनाच्छ्रवणाद्राजन् गोसहस्रकलं लभेत्}
{सर्वपापविनिर्मुक्तो विष्णुलोके महीयते} %॥२६॥

॥इति श्रीभविष्यपुराणे वैशाखकृष्णैकादश्या वरूथिन्याख्याया माहात्म्यं सम्पूर्णम्॥


\hyperref[sec:ekadashi_mahatmyam_vrata_raja]{\closesub}
\clearpage

\sect{वैशाख-शुक्ल-मोहिनी-एकादशी-माहात्म्यम्}
\label{sec:vrata-raja-vaishakha-shukla-mohini}

\uvacha{युधिष्ठिर उवाच}

\twolineshloka
{वैशाखशुक्लपक्षे तु किन्नामैकादशी भवेत्}
{किं फलं को विधिस्तस्याः कथयस्व जनार्दन} %॥१॥

\uvacha{श्रीकृष्ण उवाच}

\twolineshloka
{कथयामि कथामेतां शृणु त्वं धर्मनन्दन}
{वसिष्ठो यामकथयत्पुरा रामाय पृच्छते} %॥२॥

\uvacha{राम उवाच}

\twolineshloka
{भगवन् श्रोतुमिच्छामि व्रतानामुत्तमं व्रत}
{सर्वपापक्षयकरं सर्वदुःखनिकृन्तनम्} %॥३॥

\twolineshloka
{मया दुःखानि भुक्तानि सीताविरहजानि व}
{ततोऽहं भयभीतोऽस्मि पृच्छामि त्वां महामुने} %॥४॥

\uvacha{वसिष्ठ उवाच}

\twolineshloka
{साधु पृष्टं त्वया राम तवैषा नैष्ठिकी मतिः}
{त्वन्नामग्रहणेनैव पूतो भवति मानवः} %॥५॥

\twolineshloka
{तथापि कथयिष्यामि लोकानां हितकाम्यया}
{पवित्रं प्रावनानां च व्रतानामुत्तमं व्रतम्} %॥६॥

\twolineshloka
{वैशाखस्य सिते पक्षे द्वादशी राम या भवेत्}
{मोहिनीनाम सा प्रोक्ता सर्व पापहरा परा} %॥७॥

\twolineshloka
{मोहजालात्प्रमुच्येत पातकानां समूहन्तः}
{अस्या व्रतप्रभावेण सत्यं सत्यं वदाम्यहम्} %॥८॥

\twolineshloka
{अतस्तु कारणाद्राम कर्तव्यैषा भवादशैः}
{पातकानां क्षयकरी महादुःखविनाशिनी} %॥९॥

\twolineshloka
{शृणुष्वैकमना राम कथां पुण्यप्रदां शुभाम्}
{यस्याः श्रवणमात्रण महापापं प्रणश्यति} %॥१०॥

\twolineshloka
{सरस्वत्यास्तटे रम्ये पुरी भद्रावती शुभा}
{द्युतिमानाम नृपतिस्तत्र राज्यं करोति वै} %॥११॥

\twolineshloka
{सोमवंशोद्भवो राम धृतिमान्सत्यसङ्गरः}
{तत्र वैश्यो निवसति धनधान्यसमृद्धिमान्} %॥१२॥

\twolineshloka
{धनपाल इति ख्यात पुण्यकर्मप्रवर्तकः}
{प्रपासवाद्यायतनतडागारामकारकः} %॥१३॥

\twolineshloka
{विष्णुभक्तिपरः शान्तस्तस्यांसपञ्चपुत्रकाः}
{सुमना द्युतिमांश्चैव मेधावी सुकृती तथा} %॥१४॥

\twolineshloka
{पञ्चमो धृष्टबुद्धिश्च महापापरतः सदा}
{वारस्त्रीसङ्गनिरतो विटगोष्ठीविशारदः} %॥१५॥

{द्यूतादिव्यसनासक्तः परस्त्रीरतिलालसः}
{न देवांश्चातिथीन्वृद्धान्पितॄश्चा विजानपि} %॥१६॥

\twolineshloka
{अन्यायकर्ता दुष्टात्मा पितृद्रव्यक्षयङ्करः}
{अभक्ष्यभक्षकः पाप सुरापानरतः सदा} %॥१७॥

\twolineshloka
{वेश्याकण्ठक्षिप्तवाहुर्धमदृष्टिश्चतुष्पथे}
{पित्रा निष्कासितो गेहात्परित्यक्तश्च बान्धवैः} %॥१८॥

\twolineshloka
{स्वदेहभूषणान्येवं क्षयं नीतानि तेन वै}
{मणिकाभिः परित्यक्तो निन्दितश्च धनक्षयात्} %॥१९॥

\twolineshloka
{ततश्चिन्तापरो जातो वस्त्रहीनः क्षुधार्दितः}
{किं करोमि क गच्छानि केनोपायेन जीव्यते} %॥२०॥

\twolineshloka
{तस्करत्वं समारब्धं तत्रैव नगरे ततः}
{गृहीतो राजपुरुषैर्मुक्तश्च पितृगौरवात्} %॥२१॥

\twolineshloka
{पुनर्बद्धः पुनर्मुक्तः पुनर्मुक्तः स वै भटैः}
{धृष्टबुद्धिर्दुराचारो निबद्धो निगडैद्वैः} %॥२२॥

\twolineshloka
{कशाघातस्ताडितश्च पीडितश्च पुनः पुनः}
{न स्थातव्यं हि मन्दात्मंस्त्वया मद्देश गोचरे} %॥२३॥

\twolineshloka
{एवमुक्त्वा ततो राज्ञा मोचितो दृढबन्धनात्}
{निर्जगाम भयात्तस्य गतोऽसौ गहनं वनम्} %॥२४॥

\twolineshloka
{क्षुत्तृषापीडितश्चायमितश्चेतश्च धावति}
{सिंहवनिजघानासौ मृगसूकरचित्तलान्} %॥२५॥

\twolineshloka
{आमिषाहारनिरतो वने तिष्ठति सर्वदा}
{शरासने शरं कृत्वा निषङ्गं पृष्ठसङ्गतम्} %॥२६॥

\twolineshloka
{अरण्यचारिणो हन्ति दक्षिणश्च चतुष्पदान्}
{चकोरांश्च मयूरांश्च कास्तित्तिरिमूषकान्} %॥२७॥

\twolineshloka
{एतानन्यान् हन्ति नित्यं धृष्टबुद्धिः स निर्वृणः}
{पूर्वजन्मकृतैः पापैनिमग्नः पापकर्दमे} %॥२८॥

\twolineshloka
{दुःखशोकसमाविष्टश्चिन्तयन् सोऽप्यहर्निशम्}
{कौण्डिन्यस्याश्रमं प्राप्तः कस्माञ्चित्पुण्यगौरवात्} %॥२९॥

\twolineshloka
{माधवे मासि जाहल्या कृतस्नानं तपोधनम्}
{आससाद धृष्टबुद्धिः शोकभारेण पीडितः} %॥३०॥

\twolineshloka
{तद्वस्त्रविन्दुस्पर्शन गतपाप्मा हताशुभः}
{कौण्डिन्यस्याप्रतः स्थित्वा प्रत्युवाच कृताञ्जलिः} %॥३१॥

\uvacha{धृष्टबुद्धिरुवाच}

\twolineshloka
{प्रायश्चित्तं वद ब्रह्मन्विना वित्तेन यद्भवेत्}
{आजन्मकृतपापस्य नास्ति वित्तं ममाधुना} %॥३२॥

\uvacha{ऋषिरुवाच}

\twolineshloka
{शृणुष्वकमना भूत्वा येन पापक्षयस्तव}
{वैशाखस्य सिते पक्षे मोहिनी नाम नामतः} %॥३३॥

\twolineshloka
{एकादशीवतं तस्याः कुरु मद्वाक्यनोदितः}
{मेरुतुल्यानि पापानि क्षयं नयति देहिनाम्} %॥३४॥

\twolineshloka
{बहुजमार्जितान्येषा मोहिनी समुपोषिता}
{इति वाक्यं मुनेः श्रुत्वा धृष्टबुद्धिः प्रसन्नहृत्} %॥३५॥

\twolineshloka
{व्रतं चकार विधिवत्कौण्डिन्यस्योपदेशतः}
{कृते व्रते नृपश्रेष्ठ हतपापो बभूव सः} %॥३६॥

\twolineshloka
{दिव्यदेहस्ततो भूत्वा गरुडोपरि संस्थितः}
{जगाम वैष्णवं लोकं सर्वोपद्रववर्जितम्} %॥३७॥

\twolineshloka
{इतीदृशं रामचन्द्र तमोमोहनिकृन्तनम्}
{नातः परतरं किश्चित्रैलोक्ये सचराचरे} %॥३८॥

\twolineshloka
{यज्ञादितीर्थदानानि कलां नाहन्ति षोडशीम्}
{पठनाच्छ्रवणाद्राजन् गोसहस्रफलं लभेत्} %॥३९॥

॥इति श्रीकूर्मपुराणे मोहिन्याख्यवैशाखशुक्लैकादशीमाहात्म्यं सम्पूर्णम्॥


\hyperref[sec:ekadashi_mahatmyam_vrata_raja]{\closesub}
\clearpage

\sect{ज्येष्ठ-कृष्णापरा-एकादशी-माहात्म्यम्}
\label{sec:vrata-raja-jyeshtha-krishnapara}

\uvacha{युधिष्ठिर उवाच}

\twolineshloka
{ज्येष्ठस्य कृष्णपक्षे तु किन्नामैकादशी भवेत्}
{श्रोतुमिच्छामि माहात्म्य तद्वदस्व जनार्दन} %॥१॥

\uvacha{श्रीकृष्ण उवाच}

\twolineshloka
{साधु पृष्टं त्वया राजैल्लोकानां हितकाम्यया}
{बहुपुण्यप्रदा ह्येषा महापातकनाशिनी} %॥२॥

\twolineshloka
{अपरा नाम राजेन्द्र अपारफलदायिनी}
{लोक प्रसिद्धतां याति अपरां यस्तु सेवते} %॥३॥

\twolineshloka
{ब्रह्महत्याभिभूतोऽपि गोत्रहा भ्रूणहा तथा}
{परापवादवादी च परस्त्रीरसिकोपि च} %॥४॥

\twolineshloka
{अपरासेवनाद्राजन्विपाप्मा भवति ध्रुवम्}
{कूटसाक्ष्य मानकूटं तुलाकूटं करोति यः} %॥५॥

\twolineshloka
{कूटवेदं पठेद्विप्रः कूटशास्त्रं तथैव च}
{ज्योतिषी कूटगणकः कूटायुर्वेदको भिषक्} %॥६॥

\twolineshloka
{कूट साक्षिसमा ह्येते विज्ञेया नरकौकसः}
{अपरासेवनादाजन् पापमुक्ता भवन्ति ते} %॥७॥

\twolineshloka
{क्षत्रियः क्षात्रधर्म यस्त्यक्त्वा युद्धात्पलायते}
{स याति नरकं घोरं स्वीयधर्मबहिष्कृतः} %॥८॥

\twolineshloka
{अपरासेवनात्सोपि पापं त्यक्त्वा दिवं व्रजेत्}
{विद्यामधीत्य यः शिष्यो गुरुनिन्दा करोति च} %॥९॥

\twolineshloka
{महापातकसंयुक्तो निरयं याति दारुणम्}
{अपरासेवनात्सोपि सद्गतिं प्राप्नुयानरः} %॥१०॥

\twolineshloka
{पुष्करत्रितये स्नात्वा कार्तिक्यां यत्फलं लभेत्}
{मकरस्थे रवौ माघे प्रयागे यत्फलं नृणाम्} %॥११॥

\twolineshloka
{काश्यां यत्प्राप्यते पुण्यं शिवरारुपोषणात्}
{गयायां पिण्डदानेन यत्फलं प्राप्यते नृभिः} %॥१२॥

\twolineshloka
{सिंहस्थिते देवगुरौ गौतमीस्नानतो नरः}
{यत्फलं समवाप्नोति कुम्भे केदारदर्शनात्} %॥१३॥

\twolineshloka
{बदर्याश्रमयात्रायास्तत्तीर्थसेवनादपि}
{यत्फलं समवाप्नोति कुरुक्षेत्रे रविग्रहे} %॥१४॥

\twolineshloka
{गजाश्वहेमदानेन यज्ञे कृत्स्नसुवर्णदः}
{तत्फलं समवाप्नोति अपराया व्रतान्नरः} %॥१५॥

\twolineshloka
{अर्धप्रसूतां गां दत्त्वा सुवर्ण वसुधां तथा}
{नरो यत्फलमानोति अपराया व्रतेन तत्} %॥१६॥

\twolineshloka
{पापद्रुमकुठारोऽयं पापेन्धनदवानलः}
{पापान्धकारसूर्योऽयं पापसारङ्गकेसरी} %॥१७॥

\twolineshloka
{अपरैकादशी राजन् कर्तव्या पापभीरुभिः}
{बुदबुदा इव तोयेषु पुत्तिका इव जन्तुषु} %॥१८॥

\twolineshloka
{जायन्ते मरणायैव एकादश्या व्रतं विना}
{अपरां समुपोष्यैव पूजयित्वा त्रिविक्रमम्} %॥१९॥


\threelineshloka
{सर्वपापविनिर्मुक्तो विष्णुलोकं व्रजेन्नरः}
{लोकानां च हितार्थाय तवाने कथितं मया}
{पठनाच्छ्रवणाद्राजन् सर्वपापैः प्रमुच्यते} %॥२०॥

॥इति ब्रह्माण्डपुराणे ज्येष्ठकृष्णापराख्यकादशीमाहात्म्यं सम्पूर्णम्॥


\hyperref[sec:ekadashi_mahatmyam_vrata_raja]{\closesub}
\clearpage

\sect{ज्येष्ठ-शुक्ल-निर्जला-एकादशी-माहात्म्यम्}
\label{sec:vrata-raja-jyeshtha-shukla-nirjala}

\uvacha{भीमसेन उवाच}

\twolineshloka
{पितामह महाबुद्धे शृणु मे परमं वचः}
{युधिष्ठिरश्च कुन्ती च तथा द्रुपद नन्दिनी} %॥१॥

\twolineshloka
{अर्जुनो नकुलश्चैव सहदेवस्तथैव च}
{एकादश्यां न भुनन्ति कदाचिदपि सुव्रत} %॥२॥

\twolineshloka
{ते मां ब्रुवन्ति वै नित्यं मा भुक्ष्व त्वं वृकोदर}
{अहं तानत्रुवं तात बुभुक्षा दुःसहा मम} %॥३॥

\twolineshloka
{दानं दास्यामि विधिवत्पूजयिष्यामि केशवम्}
{विनोपवासं लभ्येत कथमेकादशीव्रतम्} %॥४॥

\onelineshloka*
{भीमसेनवचः श्रुत्वा व्यासो वचनमब्रवीत्}

\uvacha{व्यास उवाच}

\onelineshloka
{यदि स्वर्गोत्यभीष्टस्ते नरकोनिष्ट एव च} %॥५॥

\onelineshloka*
{एकादश्यां न भोक्तव्यं पक्षयोरुभयोरपि}

\uvacha{भीमसेन उवाच}

\onelineshloka
{पितामह महाबुद्धे कथयामि तवाग्रतः} %॥६॥

\twolineshloka
{एकभक्तं न शक्तोऽहमुपवासः कुतो मुने}
{वृको नामास्ति यो वन्द्विः स सदा जठरे मम} %॥७॥

\twolineshloka
{अतीवानं यदानामि तदा समुपशाम्यति}
{एकं शक्तोस्म्यहं कर्तु चोपवासं महामुने} %॥८॥

\onelineshloka*
{तदेकं वद निश्चित्य येन श्रेयोइमाप्नुयाम्}

\uvacha{व्यास उवाच}

\onelineshloka
{श्रुतास्ते मानवा धर्मा वैदिकाश्च श्रुतास्त्वया} %॥९॥

\twolineshloka
{कलौ युगे न शक्यन्ते ते वै कर्तु नराधिप}
{सुखोपायं चाल्पधनमल्पक्लेशं महाफलम्} %॥१०॥

\twolineshloka
{पुराणानां च सर्वेषां सारभूतं वदामि ते}
{एकादश्यां न भुञ्जीत पक्षयोरुभयोरपि} %॥११॥

\twolineshloka
{एकादश्या न भुङ्क्त यो न याति नरकं तु सः}
{व्यासस्य वचनं श्रुत्वा कम्पितोऽश्वत्थपत्रवत्} %॥१२॥

\onelineshloka*
{भीमसेनो महाबाहुर्भातो वाक्यमभाषत}

\uvacha{भीमसेन उवाच}

\onelineshloka
{पितामह न शक्तोऽहमुपवासे करोमि किम्} %॥१३॥

\onelineshloka*
{ततो बहुफलं ब्रूहि व्रतमेक मम प्रभो}

\uvacha{व्यास उवाच}

\onelineshloka
{वृषस्थ मिथुनस्थे के शुक्का यैकादशी भवेत्} %॥१४॥

\twolineshloka
{ज्येष्ठमासे प्रयत्नेन सोपोष्या जलवर्जिता}
{स्नाने चाचमने चैव वर्जयित्वोदकं बुधः} %॥१५॥

\twolineshloka
{उपयुञ्जीत नैवान्यगतभङ्गोऽन्यथा भवेत}
{उदयादुदयं यावद्वर्जयित्वा जलं बुधः} %॥१६॥

\twolineshloka
{अप्रयत्नादवाप्नोति द्वादशद्वादशीफलम्}
{ततः प्रभाते विमले द्वादश्यां स्नानमाचरेत्} %॥१७॥

\twolineshloka
{जलं सुवर्ण दत्त्वा च द्विजातिभ्यो यथाविधि}
{भुञ्जीत कृतकृत्यस्तु ब्राह्मणैः सहितो वशी} %॥१८॥

\twolineshloka
{एवं कृते तु यत्पुण्यं भीमसेन शृणुष्व तत्}
{संवत्स रस्य या मध्ये एकादश्यो भवन्ति वै} %॥१९॥

\twolineshloka
{तासां फलमवाप्नोति अत्र मे नास्ति संशयः}
{इति मां केशवः प्राह शङ्खचक्रगदाधरः} %॥२०॥

\twolineshloka
{एकादश्यां सिते पक्षे ज्येष्ठस्यौदकवर्जितम्}
{उपोष्य फलमाप्नोति तच्छृणुष्व वृतोदर} %॥२१॥

\twolineshloka
{सर्वतीर्थेषु यत्पुण्यं सर्वदानेषु यत्फलम्}
{यत्फेलं समवाप्नोति इमां कृत्वा वृकोदर} %॥२२॥

\twolineshloka
{संवत्सरस्य यावन्त्यः शुक्लाः कृष्णा वृकोदर}
{उपोपितास्ताः सर्वाः स्युरेकादश्यो न संशयः} %॥२३॥

\twolineshloka
{धनधान्यवहाः पुण्याः पुत्रारोग्यफलप्रदाः}
{उपोषिता नरव्याघ्र इति सत्यं वदामि ते} %॥२४॥

\twolineshloka
{यमदूता महाकायाः करालाः कृष्णपिङ्गलाः}
{दण्डपाशधरा रौद्रा मरणे दृष्टिगोचरम्} %॥२५॥

\twolineshloka
{न प्रयान्ति नरव्याघ्र एकादश्यामुपोषणात्}
{पीताम्बरधराः सौम्याश्चक्रहस्ता मनोजवाः} %॥२६॥

\twolineshloka
{अन्नकाले नयन्त्येव मानवं वैष्णवीं पुरीम्}
{तस्मात्सर्वप्रयत्नेन सोपोष्योदकवर्जिता} %॥२७॥

\twolineshloka
{जलधेनुं ततो दत्त्वा सर्वपापैः प्रमुच्यते}
{इति श्रुत्वा तदा चकुः पाण्डवा जनमेजय} %॥२८॥

\twolineshloka
{ततःप्रभृति भीमेन कृतेयं निर्जला शुभा}
{पाण्डवद्वादशोनाम्ना लोके ख्याता बभूव ह} %॥२९॥

\twolineshloka
{तथा त्वमपि भूपाल सोपवासार्चनं हरेः}
{कुरु त्वं च प्रयत्नेन सर्वपापप्रशान्तये} %॥३०॥

\twolineshloka
{करिष्याम्यद्य देवेश जलवर्जमुपोषणम्}
{भोक्ष्ये परेऽति देवेश ह्यन्नं च तव वासरात्} %॥३१॥

\twolineshloka
{इत्युच्चार्य ततो मन्त्रमुपवासपरो भवेत्}
{सर्वपापविनाशाय श्रद्धादमसमन्वितः} %॥३२॥

\twolineshloka
{मेरुमन्दरमानं तु त्रियाथ पुरुषस्य यत्}
{पापं तद्भस्मतां याति एकादश्याः प्रभावतः} %॥३३॥

\twolineshloka
{न शक्रोति च यो दातुं जलधेतुं नराधिप}
{सकाश्चनो घटस्तेन देयो वस्त्रेण संवृतः} %॥३४॥

\twolineshloka
{तोयस्य नियमं योस्यां कुरुते व पुण्यभाक्}
{पलकोटिनुवर्गस्य यामेयामेऽश्नुते फलम्} %॥३५॥

\twolineshloka
{स्नानं दानं जपं होमं यदस्यां कुरुते नरः}
{तत्सर्व चाक्षयं प्रोक्तमेतत्कृष्णस्य भाषितम्} %॥३६॥

\twolineshloka
{किं वापरण धर्मेण निर्जलैकादशीं नृप}
{उपोष्य च नरो भक्त्या वैष्णवं पदमाप्नुयात्} %॥३७॥

\twolineshloka
{सुवर्णमन्नं वासांसि यदस्यां सम्प्रदीयते}
{तदस्य च कुरुश्रेष्ठ सर्वमप्यक्षयं भवेत्} %॥३८॥

\twolineshloka
{एकादशीदिने योऽनं भुक्ते पापं भुनक्ति सः}
{इह लोके स चाण्डालो मृतः प्राप्नोति दुर्गतिम्} %॥३९॥

\twolineshloka
{ये प्रदास्यन्ति दानानि द्वादशी समुपोष्य च}
{ज्येष्ठे मासि सिते पक्षे प्राप्स्यन्ति परमं पदम्} %॥४०॥

\twolineshloka
{ब्रह्महा मद्यपः स्तेनो गुरुद्वेष्टा सदाऽनृती}
{मुच्यन्ते पातकै सर्वैर्निर्जला यरुपोषिता} %॥४१॥

\twolineshloka
{विशेषं शृणु राजेन्द्र निर्जलैकादशीदिने}
{यत्कर्तव्यं नरैः स्त्रीभिः श्रद्धादमसमन्वितैः} %॥४२॥

\twolineshloka
{जलशायी तु सम्पूज्यो देया धेनुश्च तन्मयी}
{प्रत्यक्षा वा नृपश्रेष्ठ घृतधेनुरथापि वा} %॥४३॥

\twolineshloka
{दक्षिणाभिश्च श्रेष्ठाभिदृष्टान्नैश्च पृथग्विधैः}
{तोषणीया प्रयत्नेन द्विजा धर्मभृतां वर} %॥४४॥

\twolineshloka
{तुष्टो भवति वै क्षिप्रं तैस्तुष्टैर्मोक्षदो हरिः}
{आत्मद्रोहः कृतस्तैस्तु यैषा समुपोषिता} %॥४५॥

\twolineshloka
{पापात्मानो दुराचारा दुष्टास्ते नात्र संशयः}
{कुलानां च शतं सायमनाचाररतं सदा} %॥४६॥

\twolineshloka
{आत्मना सह तैनीतं वासुदेवस्य मन्दिरम्}
{शान्तैनपरैश्चैव अर्चद्भिश्च तथा हरिम्} %॥७॥

\twolineshloka
{कुर्वद्भिर्जागरं रात्रौ यैरेषा समुपोषिता}
{अन्नं पानं तथा गावो वस्त्रं शय्यासनं शुभम्} %॥४८॥

\twolineshloka
{कमण्डलु तथा छत्रं दातव्यं निर्जलादिन}
{उपानहीं च यो दद्यात्पात्रभूते द्विजोत्तमे} %॥४९॥

\twolineshloka
{स सौवर्णेन यानेन स्वर्गलोकं व्रजेयुवम्}
{यश्चेमां शृणुयाद्भक्त्या यश्चापि परिकीर्तयेत्} %॥५०॥

\twolineshloka
{उभौ तौ स्वर्गतौ स्याता नात्र कार्या विचारणा}
{यत्फलं सन्निहत्यायां राहुप्रस्ते दिवाकरे} %॥५१॥


\threelineshloka
{कृत्वा श्राद्धं लभेन्मयम्तदस्याः श्रवणादपि}
{एवं यः कुरुते पुण्यां द्वादशी पापनाशिनीम्}
{सर्वपापविनिमुक्तः पदं गच्छत्यनामयम्} %॥५२॥

॥इति श्रीभारतपद्मयोरुक्तं ज्येष्ठशुक्लनिर्जलैकादशीमाहात्म्यं सम्पूर्णम्॥


\hyperref[sec:ekadashi_mahatmyam_vrata_raja]{\closesub}
\clearpage

\sect{आषाढ-कृष्ण-योगिनी-एकादशी-माहात्म्यम्}
\label{sec:vrata-raja-ashadha-krishna-yogini}

\uvacha{युधिष्ठिर उवाच}

\twolineshloka
{ज्येष्ठशुक्ने निर्जलाया माहात्म्यं व श्रुतं मया}
{आषाढकृष्णपक्षे तु किनामैकादशी भवेत्} %॥१॥

\onelineshloka*
{कथयस्व प्रसादेन ममाने मधुसूदन}

\uvacha{श्रीकृष्ण उवाच}

\onelineshloka
{व्रतानामुत्तमं राजन्कथयामि तवाप्रतः} %॥२॥

\twolineshloka
{सर्वपापक्षयकरं भुक्तिमुक्तिप्रदायकम्}
{आषाढस्यासिते पक्षे योगिनीनाम नामतः} %॥३॥

\twolineshloka
{एकादशी नृपश्रेष्ठ महापातकनाशिनी}
{संसारार्णवमन्नानां पोतरूपा सनातनी} %॥४॥

\twolineshloka
{जगत्रये सारभूता योगिनीति नराधिप}
{कथयामि, कथा, तस्याः पौराणी पापहारिणीम्} %॥५॥

\twolineshloka
{अलकाधिपतिर्नाम्ना कुबेरः शिवपूजकः}
{तस्यासीत्पुष्पबदुको हेममालीति नामतः} %॥६॥

\twolineshloka
{तस्य पत्नी सुरूपासीद्विशालाक्षीति नामतः}
{स तस्यां स्नेहसंयुक्तः कामपाशवशं गतः} %॥७॥

\twolineshloka
{मानसात्पुष्पनिचयमानीय स्वगृहे स्थितः}
{पत्नीप्रेमसमायुक्तो न कुबेरालयं गतः} %॥८॥

\twolineshloka
{कुबेरो देवतदने करोति शिवपूजनम्}
{मध्याह्नसमये राजन् पुष्पाणि प्रसमीक्षते} %॥९॥

\twolineshloka
{हेममाली स्वभवने रमते कान्तया सह}
{यक्षराट् प्रत्युवाचाथ कालातिक्रमकोपितः} %॥१०॥

\twolineshloka
{कस्मान्नायाति भो यक्षा हेममाली दुरात्मवान्}
{निश्चयः क्रियतामस्य प्रत्युवाच पुनः पुनः} %॥११॥

\uvacha{यक्षा उचुः}

\twolineshloka
{वनिताकामुको गेहे रमते स्वेच्छया नृप}
{तेषां वाक्यं समाकर्ण्य कुबेरः कोपपूरितः} %॥१२॥

\twolineshloka
{आह्वयामास तं तूर्ण बटुक हेममालिनम्}
{ज्ञात्वा कालात्ययं सोपि भयव्याकुललोचनः} %॥१३॥

\twolineshloka
{आजगाम नमस्कृत्य कुबेरस्याप्रतःस्थितः}
{तं दृष्ट्वा धनदः क्रुद्धः कोपसंरक्तलोचनः} %॥१४॥

\onelineshloka*
{प्रत्युवाच रुषाविष्टः कोपाद्विस्फुरिताधरः}

\uvacha{धनद उवाच}

\onelineshloka
{रे पाप दुष्ट दुर्वृत्त कृतवान् देवहेलनम्} %॥१५॥

\twolineshloka
{अतो भव चित्रयुक्तो वियुक्तः कान्तया सदा}
{अस्मात्स्थानादपध्वस्तो गच्छ स्थानमथाधमम्} %॥१६॥

\twolineshloka
{इत्युक्ते वचने तेन तस्मात्स्था नात्पपातु सः}
{महादुःखाभिभूतश्च कुष्ठपीडितविग्रहः} %॥१७॥

\twolineshloka
{न वै तोयं न भक्ष्यं च बनेरौद्रे लभत्यसौ}
{न सुखं दिवसे तस्य न निद्रा लभते निशि} %॥१८॥

\twolineshloka
{छायायां पीडिततनुर्निदाघेऽत्यन्तपीडितः}
{शिवपूजाप्रभावेण स्मृतिस्तस्य न गच्छति} %॥१९॥

\twolineshloka
{पातकेन्नाभिभूतोऽपि कर्म पूर्वमनुस्मरन्}
{भ्रममाणस्ततोऽगच्छद्धिमाद्रि पर्वतोत्तमम्} %॥२०॥

\twolineshloka
{तत्रापश्यन्मुनिवरं मार्कण्डेयं तपोनिधिम्}
{यस्यायुर्विद्यते राजन् ब्रह्मणो दिनसप्तकम्} %॥२१॥

\twolineshloka
{आश्रम स गतस्तस्य ऋषेब्रह्मसदः समम्}
{ववन्दे चरणौ तस्य दूरतः पापकर्मकृत्} %॥२२॥

\twolineshloka
{मार्कण्डेयो मुनिवरो दृष्ट्वा तं कुष्ठिनं तदा}
{परोपकरणार्थाय समाहूयदमब्रवीत्} %॥२३॥

\uvacha{मार्कण्डेय उवाच}

\twolineshloka
{कस्मात् कुष्ठाभिभूतस्त्वं कुतो निन्द्यतरो ह्यसि}
{इत्युक्तः प्रत्युवाचाथ मार्कण्डेयेन धीमता} %॥२४॥

\uvacha{हेममाल्युवाच}

\twolineshloka
{यक्षराजस्यानुचरो हेममालीति नामतः}
{मानसात्पुष्पनिचयमानीय प्रत्यहं मुने} %॥२५॥

\twolineshloka
{शिवपूजनवेलायां कुबेराय समर्पये}
{एकस्मिन् दिवसे काललोपश्च विहितो मया} %॥२६॥

\twolineshloka
{पत्नीसौख्यप्रसक्तन कामव्याकुलचेतसा}
{ततःक्रुद्धेन शप्तोऽहं राजराजन वै मुन} %॥२७॥

\twolineshloka
{कुष्ठाभिभूतः सञ्जातो वियुक्तः कान्तया सह}
{अधुना तव सान्निध्यं प्राप्तोऽस्मि शुभकर्मणा} %॥२८॥

\twolineshloka
{सतां स्वभावतश्चित्तं परोपकरणक्षमम्}
{इति ज्ञात्वा मुनिश्रेष्ठ शाधि मां च कृतैनसम्} %॥२९॥

\uvacha{मार्कण्डेय उवाच}

\twolineshloka
{त्वया सत्यमिह प्रोक्तं नासत्यं भाषितं यतः}
{अतो व्रतोपदेशं ते करिष्यामि शुभप्रदम्} %॥३०॥

\twolineshloka
{आषाढ कृष्णपक्षे त्वं योगिनीव्रतमाचर}
{अस्य व्रतस्य पुण्येन कुष्ठात्त्वं मुच्यसे ध्रुवम्} %॥३१॥

\twolineshloka
{इति वाक्यं मुनेः श्रुत्वा दण्डवत्पतितो भुवि}
{उत्थापितश्च मुनिना बभूवातीव हर्षितः} %॥३२॥

\twolineshloka
{मार्कण्डेयोपदेशेन कृतं तेन व्रतोत्तमम्}
{तद्वतस्य प्रभावेण देवरूपो बभूव सः} %॥३३॥

\twolineshloka
{संयोग कान्तया लेभे बुभुजे सौख्यमुत्तमम्}
{ईटग्विधं नृपश्रेष्ठ कथितं योगिनीव्रतम्} %॥३४॥

\twolineshloka
{अष्टाशीतिसहस्राणि द्विजान् भोजयते तु यः}
{तत्फलं समवानोति योगिनीव्रतकृन्नरः} %॥३५॥

\twolineshloka
{महापापप्रशमनी महापुण्यफलप्रदा}
{शुचिकृष्णकादशी ते कथिता योगिनी नृप} %॥३६॥

॥इति श्रीब्रह्मवैवर्तपुराणे आषाढकृष्णयोगिन्याख्यैकादशीमाहात्म्यं सम्पूर्णम्॥


\hyperref[sec:ekadashi_mahatmyam_vrata_raja]{\closesub}
\clearpage

\sect{आषाढ-शुक्ल-शयनी-एकादशी-माहात्म्यम्}
\label{sec:vrata-raja-ashadha-shukla-shayani}

\uvacha{युधिष्ठिर उवाच}

\twolineshloka
{आषाढस्य सिते पक्षे किन्नामकादशी भवेत्}
{को देवः को विधिस्तस्या एतदाख्याहि केशव} %॥१॥

\uvacha{श्रीकृष्ण उवाच}

\twolineshloka
{कथयामि महीपाल कथामाश्चर्यकारिणीम्}
{थयामास यां ब्रह्मा नारदाय महात्मने} %॥२॥

\uvacha{नारद उवाच}

\twolineshloka
{कथयस्व प्रसादेन विष्णोराराधनाय मे}
{आषाढशुक्लपक्षे तु किनामैकादशी भवेत्} %॥३॥

\uvacha{ब्रह्मोवाच}

\twolineshloka
{वैष्णवोऽप्ति मुनि श्रेष्ठ साधु पृष्ट कलिप्रिय}
{नातः परतरं लोके पवित्रं हरिवासरात्} %॥४॥

\twolineshloka
{कर्तव्यं तु प्रयत्नेन सर्वपापापनुत्तये}
{तस्मात्तेऽहं प्रवक्ष्यामि शुक्ल एकादशीव्रतम्} %॥५॥

\twolineshloka
{एकादश्या व्रतं पुण्यं पापघ्नं सर्वकामदम्}
{न कृतं यैर्नरैलोक ते नरा निरयेषिणः} %॥६॥

\twolineshloka
{पद्मानामेति विख्याता शुचौ ोकादशी सिता}
{हृषीकेशप्रीतये तु कर्तव्यं व्रतमुत्तमम्} %॥७॥

\twolineshloka
{कथयामि तवाग्रेऽहं कथा पौराणिकी शुभाम्}
{यस्याः श्रवणमात्रेण महापापं प्रणश्यति} %॥८॥

\twolineshloka
{मान्धाता नाम राजर्षिर्विवस्वदंशसम्भवः}
{बभूव चक्रवर्ती स सत्यसन्धः प्रतापवान्} %॥९॥

\twolineshloka
{धर्मतः पालयामास प्रजाः पुत्रानिवौरसान्}
{ने तस्य राज्ये दुर्भिक्षं नाधयो व्याधयस्तथा} %॥१०॥

\twolineshloka
{निरातङ्काः प्रजास्तस्य धनधान्यसमन्विताः}
{नान्यायोपार्जितं द्रव्यं कोशे तस्य महीपतेः} %॥११॥

\twolineshloka
{तस्यैवं कुर्वतो राज्य बहुवर्षगणो गतः}
{अथो कदाचित्सम्प्राप्ते विपाके पापकर्मणः} %॥१२॥

\twolineshloka
{वर्षत्रयं तद्विषये न ववर्ष बलाहकः}
{तेनोद्विग्नाः प्रजास्तत्र बभूवुः क्षुधयार्दिताः} %॥१३॥

\twolineshloka
{स्वाहास्वधावषट्कारवेदाध्ययनवर्जिता}
{गावभूबुर्विषयास्तस्य सस्याभावेन पीडिताः} %॥१४॥

\twolineshloka
{अथ प्रजाः समागत्य राजानमिदमब्रुवन्}
{श्रूयतां वचनं राजन् प्रजानां हितकारकम्} %॥१५॥

\twolineshloka
{आपो नारा इति प्रोक्ताः पुराणेषु मनीषिभिः}
{अयनं ता भङ्गंवतस्तेन नारायणः स्मृतः} %॥१६॥

\twolineshloka
{पर्जन्यरूपो भगवान्विष्णुः सर्वगतः सदा}
{स एव कुरुते वीष्ट वृष्टेरन्नं ततः प्रजाः} %॥१७॥

\twolineshloka
{तदभावेन नृपते क्षयं गच्छन्ति वै प्रजाः}
{तथा कुरु नृपश्रेष्ठ योगक्षेमो यथा भवेत्} %॥१८॥

\uvacha{राजोवाच}

\twolineshloka
{सत्यमुक्तं भवद्भिश्च न मिथ्याभिहितं वचः}
{अन्नं ब्रह्ममयं प्रोक्तमन्ने सर्व प्रतिष्ठितम्} %॥१९॥

\twolineshloka
{अन्नाद्भवन्ति भूतानि जगदनेन वर्तते}
{इत्येवं श्रूयते लोके पुराणे बहुविस्तरे} %॥२०॥

\twolineshloka
{नृपाणामपचारेण प्रजानां पीडनं भवेत्}
{नाहं पश्याम्यात्मकृतं दोषं बुद्ध्या विचारयन्} %॥२१॥

\twolineshloka
{तथापि प्रयतिष्यामि प्रजानां हितकाम्यया}
{इति कृत्वा मति राजा परिमेयबलान्वितः} %॥२२॥

\twolineshloka
{नमस्कृत्य विधातारं जगाम गहनं वनम्}
{चचारि मुनिमुख्यानामाश्रमांस्तपसैधितान्} %॥२३॥

\twolineshloka
{शाददर्शाथ ब्रह्मसुतमृषिमगिरसं नृपः}
{तेजसा द्योतितदिशं द्वितीयमिव पद्माजम्} %॥२४॥

\twolineshloka
{तं दृष्ट्वा हर्षितो राजा अवतीर्य च वाहनात्}
{नमश्चक्रेस्य चरणौ कृताञ्जलिपुटो वशी} %॥२५॥

\twolineshloka
{मुनिस्तमभिनन्द्याथ स्वस्तिवाचनपूर्वकम्}
{पप्रच्छ कुशलं राज्ये सप्तस्वङ्गेषु भूपतेः} %॥२६॥

\twolineshloka
{निवेदयित्वा कुशलं पप्रच्छानामयं नृपः}
{ततश्च मुनिना राजो पृष्टागमनकारणः} %॥२७॥

\onelineshloka*
{अब्रवीन्मुनिशार्दूलं स्वस्यागमनकारणम्}

\uvacha{राजोवाच}

\twolineshloka
{भगवन् धर्मविधिना मम पालयतो महीम्}
{अनावृष्टिः सम्प्रवृत्ता नाहं वेदयत्र कारणम्} %॥२८॥

\twolineshloka
{संशयच्छेदनार्थेऽत्र ह्यागतोऽहं तवान्तिकम्}
{योगक्षेमविधानेन प्रजानां निर्वृतिं कुरु} %॥२९॥

\uvacha{ऋषिरुवाच}

\twolineshloka
{एतत्कृतयुगं राजन् युगानामुत्तमं स्मृतम्}
{अत्र ब्रह्मोत्तरा लोका धर्मश्चात्र चतुष्पदः} %॥३०॥

\twolineshloka
{अस्मिन्युगे तपोयुक्ता ब्राह्मणा नेतरे जनाः}
{विषये तव राजेन्द्र वृषलो यत्तपस्यति} %॥३१॥

\twolineshloka
{अकार्यकरणात्तस्य न वर्षति बलाहकः}
{कुरु तस्य वधे यत्नं येन दोषः प्रशाम्यति} %॥३२॥

\uvacha{राजोवाच}

\twolineshloka
{नाहमेनं वधिष्यामि तपस्यन्तमनागसम्}
{धर्मोपदेशं कथय उपसर्गविनाशने} %॥३३॥

\uvacha{ऋषिरुवाच}

\twolineshloka
{यद्येवं तर्हि नृपते कुरुष्वैकादशीव्रतम्}
{शुचिमासे सिते पक्षे पद्मानामेति विश्रुता} %॥३४॥

\twolineshloka
{तस्या व्रतप्रभावेण सुवृष्टिर्भविता ध्रुवम्}
{सर्वसिद्धिप्रदा ह्येषा सर्वोपद्रवनाशिनी} %॥३५॥

\twolineshloka
{अस्या व्रतं कुरु नृप सप्रजः सपरिच्छदः}
{इति वाक्यं मुनेः श्रुत्वा राजा स्वगृहमागतः} %॥३६॥

\twolineshloka
{आषाढमासे सम्प्राप्ते पद्माव्रतमथाकरोत्}
{प्रजाभिः सह सर्वाभिश्चातुर्वर्ण्यसमन्वितः} %॥३७॥

\twolineshloka
{एवं कृते व्रते राजन्प्रववर्ष बलाहकः}
{जलेन प्लाविता भूमिरभवत्सस्यमालिनी} %॥३८॥

\twolineshloka
{हृषीकेशप्रसादेन जनाः सौख्यं प्रपेदिरे}
{एतस्मात्कारणादेव कर्तव्यं व्रतमुत्तमम्} %॥३९॥

\twolineshloka
{भुक्तिमुक्तिप्रदं चैव लोकानां सुखदायकम्}
{पठनाच्छ्रवणादस्याः सर्वपापैः प्रमुच्यते} %॥४०॥

[इयमेव शयन्याख्या। एतस्यां विष्णुशयनव्रतं चातुर्मास्यव्रतग्रहणं चोक्तं भविष्ये।]

॥इति श्रीवाराहपुराणे आषाढशुक्लपद्माख्यैकादशीव्रतमाहात्म्यम्॥


\hyperref[sec:ekadashi_mahatmyam_vrata_raja]{\closesub}
\clearpage

\sect{आषाढ-शुक्ल-शयनी-एकादशी-माहात्म्यम् (भविष्य-पुराणम्)}
\label{sec:vrata-raja-ashadha-shukla-shayani-bhavishya}

\uvacha{कृष्ण उवाच}

\twolineshloka
{इयमेकादशी राजञ्छयनीत्यभिधीयते}
{विष्णोः प्रसादसिद्धयर्थमस्यां च शयनव्रतम्} %॥१॥

\twolineshloka
{कर्तव्यं राजशार्दूल जनैर्मोक्षेच्छुमिः सदा}
{चातुर्मास्यव्रतारम्भोऽप्यस्यामेव विधीयते} %॥२॥

\uvacha{युधिष्ठिर उवाच}

\twolineshloka
{कथं कृष्ण प्रकर्तव्यं श्रीविष्णोः शयनव्रतम्}
{तब्रूहि कृपया देव चातुर्मास्यप्रतानि च} %॥३॥

\twolineshloka
{श्रीकृष्ण उवाचशृणु पार्थ प्रवक्ष्यामि गोविन्दशयनव्रतम्}
{चातुर्मास्ये च यान्युतान्यासंस्तानि व्रतानि च} %॥४॥

\twolineshloka
{कर्कराशिगते सूर्ये शुचौ रुक्के तु पक्षके}
{एकादश्यां जगनाथं स्वापयेन्मधुसूदनम्} %॥५॥

\twolineshloka
{तुलाराशिस्थिते तस्मिन् पुनरुत्थापयेद्धरिम्}
{आषाढल्य सिते पक्षे एकादश्यामुपोषितः} %॥६॥

\twolineshloka
{चातुर्मास्यव्रतानां तु कुर्वीत नियमं ततः}
{स्थापयेत् प्रतिमा विष्णो शङ्खचक्रगदाधराम्} %॥७॥

\twolineshloka
{पीताम्बरधरां सौम्यां पर्यङ्के वै सिते शुभे}
{सितवस्त्रसमाच्छन्ने सोपधाने युधिष्ठिर} %॥८॥

\twolineshloka
{इतिहासपुराणज्ञो ब्राह्मणो वेदपारगः}
{नापयित्वा दधिक्षीरघृतक्षौद्रसिताजलैः} %॥९॥

\twolineshloka
{समालेप्य शुभैर्गन्धैधूपैर्दीपैश्च भूरिशः}
{पूजयेत्कुसुमैः शस्तैमन्त्रेणानेन पाण्डव} %॥१०॥

\twolineshloka
{सुप्ते त्वाय जगन्नाथे जगत्सुप्तं चराचरम्}
{विबुद्धे त्वयि बुध्येत जगत्सर्व चराचरम्} %॥११॥

\twolineshloka
{एवं तां प्रतिमां विष्णोः पूजयित्वा युधिष्ठिर}
{प्रभाषेताग्रतो विष्णोः कृताञ्जलिपुटो नरः} %॥१२॥

\twolineshloka
{चतुरो वार्षिकान्मासान्देवस्योत्थापनावधि}
{ग्रहीष्ये नियमाञ्छुद्वानिर्विघ्रान्कुरु मे प्रभो} %॥१३॥

\twolineshloka
{इति सम्प्रार्थ्य देवेशं प्रहः संशुद्धमानसः}
{स्त्री वा नरो वा मद्भक्तो धर्मार्थ च धृतव्रतः} %॥१४॥

\twolineshloka
{गृहीयानियमानेतान् दन्तधावन पूर्वकम्}
{व्रतप्रारम्भकालास्तु प्रोक्ताः पञ्चैव विष्णुना} %॥१५॥

\twolineshloka
{एकादशी द्वादशी च पौर्णिमा च तथाष्टमी}
{कर्कटाख्या च सङ्क्रान्तिस्तेषु कुर्याद्यथाविधि} %॥१६॥

\twolineshloka
{चतुर्धा गृह्य वै चीर्ण चातुर्मास्यव्रतं नरः}
{कार्तिक शुक्लपक्षे तु द्वादश्यां तत्समापयत्} %॥१७॥

\twolineshloka
{न शैशवं च मौव्यं च शुक्रगुवोर्ने वा तिथेः}
{खण्डत्वं चिन्तयेदादौ चातुर्मास्यविधौ नरः} %॥१८॥

\twolineshloka
{अशुचिर्वा शुचिर्वाऽपि यदि स्त्री यदि वा पुमान्}
{व्रतमेकं नरः कृत्वा मुच्यते सर्वपातकैः} %॥१९॥

\twolineshloka
{प्रतिवर्षे तु यः कुर्याद्रतं वै संस्मरन् हरिम्}
{देहान्तेऽतिप्रदीप्तेन विमानेनार्कतेजसा} %॥२०॥

\twolineshloka
{मोदते विष्णुलोकेऽसौ यावदाभूतसम्प्लवम्}
{तेषां फलानि वक्ष्यामि कर्तृणां त पृथक्पृथक्} %॥२१॥

\twolineshloka
{देवतायतने नित्यं मार्जनं जलसेचनम्}
{प्रलेपनं गोमयेन रङ्गवल्ल्यादिकं तथा} %॥२२॥

\twolineshloka
{यः करोति नरश्रेष्ठश्चातुर्मास्यमतन्द्रितः}
{समाप्तौ च यथाशक्त्या कुर्याद्राह्मणभोजनम्} %॥२३॥

\twolineshloka
{सप्तजन्मसु विप्रेन्द्रः सत्यधर्मपरो भवेत्}
{दना क्षीरेण चाज्येन क्षौद्रेण सितया तथा} %॥२४॥

\twolineshloka
{नापयेद्विधिना देवं चातुर्मास्ये जनाधिप}
{स याति विष्णुसारूप्य सुखमक्षय्यमश्नुते} %॥२५॥

\twolineshloka
{नृपो भूमिं प्रदद्याद्यो यथाशक्त्या च काञ्चनम्}
{विप्राय देवमुद्दिश्य सफलं च सदक्षिणम्} %॥२६॥

\twolineshloka
{अक्षयान् लभते भोगान् स्वर्ग इन्द्र इवापरः}
{लोक स समवाप्नोति विष्णोरत्र न संशयः} %॥२७॥

\twolineshloka
{देवाय हेमपद्मं तु दद्यानैवेद्यसंयुतम्}
{गन्धपुष्पाक्षताद्यैर्यो देवब्राह्मणयोरपि} %॥२८॥

\twolineshloka
{पूजां यः कुरुते नित्यं चातुर्मास्ये व्रती नरः}
{अक्षयं सुखमाप्नोति पुरन्दरपुरं व्रजेत्} %॥२९॥

\twolineshloka
{यस्तु वै चतुरो मासांस्तुलस्या हरिमर्चयेत्}
{तुलसी काञ्चनीं कृत्वा ब्राह्मणाय निवेदयेत्} %॥३०॥

\twolineshloka
{काञ्चनेन विमानेन वैष्णवी लभते गतिम्}
{देवाय गुग्गुलं यो वै दीपं चार्पयते नरः} %॥३१॥

\twolineshloka
{समाप्तौ धूपिको दद्यादीपिकां च महामते}
{स भोगी जायते श्रीमांस्तथा सौभाग्यवानपि} %॥३२॥

\twolineshloka
{प्रदक्षिणास्तु यः कुर्यान्नमस्काराविशेषतः}
{अश्वत्थस्याथवा विष्णोः कार्तिक्यवधि स ध्रुवम्} %॥३३॥

\twolineshloka
{विष्णुलोकमवाप्नोति सत्यं सत्यं न संशयः}
{सन्ध्यादीपप्रदो यस्तु प्राङ्गणे द्विजदेवयोः} %॥३४॥

\twolineshloka
{समाप्तौ दीपिका दद्याद्वस्त्रं चैकं च काचनम्}
{वैकुण्ठं समवाप्नोति तेजस्वी स भवेदिह} %॥३५॥

\twolineshloka
{विष्णुपादोदकं यस्तु पिबेच्छ्राद्धासमन्वितः}
{विष्णोर्लोकमवाप्नोति न चास्मिञ्जायते नरः} %॥३६॥

\twolineshloka
{शतमष्टोत्तरं यस्तु गायत्रीजपमाचरेत्}
{त्रिकालं वैष्णवे हम्र्थे न स पापेन लिप्यते} %॥३७॥

\twolineshloka
{पुराणं शृणुयानित्यं धर्मशास्त्रमथापि वा}
{काश्चनेन युतं वस्त्रं पुस्तकं च निवेदयेत्} %॥३८॥

\twolineshloka
{पुण्यवान् धनवान्भोगी सत्यशौचपरायणः}
{ज्ञानवाँल्लोकविख्यातो बहुशिष्यः सुधार्मिकः} %॥३९॥

\twolineshloka
{नाममन्त्रव्रतपरः शम्भोर्वा केशवस्य च}
{समाप्तौ प्रतिमां दद्यात्तस्य देवस्य काश्चनीम्} %॥४०॥

\twolineshloka
{पुण्यवान् दोषनिर्मुक्तः स भवेच्च गुणालयः}
{कृतनित्यक्रियो भूत्वा सूर्यायाय निवेदयेत्} %॥४१॥

\twolineshloka
{सूर्यमण्डलमध्यस्थं देवं ध्यात्वा जनार्दनम्}
{समाप्तौ काश्चनं दद्याद्रक्तवस्त्रं च गां तथा} %॥४२॥

\twolineshloka
{आरोग्यं पूर्णमायुश्च कीर्ति लक्ष्मी बलं लभेत्}
{तिलहोमं तु यः कुर्याचातुर्मास्ये दिनदिने} %॥४३॥

\twolineshloka
{भक्त्या व्याहृतिभिर्मत्रैर्गायच्या वा व्रतान्वितः}
{अष्टोत्तरशतं चाथ अष्टाविंशतिमेव वा} %॥४४॥

\twolineshloka
{तिलपात्रं समाप्तौ तु दद्याद्विप्राय धीमते}
{वाङ्मन कायजनितः पापैर्मुच्येत् सश्चितैः} %॥४५॥

\twolineshloka
{न रोगरभिभूयेत लभेत्सन्ततिमुत्तमाम्}
{अनहोमं तु यः कुर्याचातुर्मास्यमतन्द्रितः} %॥४६॥

\twolineshloka
{समाप्तौ घृतकुम्भं तु दद्यात्सवस्त्रकाञ्चनम्}
{आरोग्यं कान्तिमतुलां पुत्रसौभाग्यसम्पदः} %॥४७॥

\twolineshloka
{शत्रुक्षयं च लभते ब्रह्मगा प्रतिमो भवेत्}
{अश्वत्थसेवां यः कुर्यात्सर्वपापैः प्रमुच्यते} %॥१८॥

\twolineshloka
{विष्णुभक्तो भवेत्पश्चादन्ते वस्त्रं प्रदादयेत्}
{सकाश्चनं ब्राह्मणाय नैव रोगान् स विन्दते} %॥४९॥

\twolineshloka
{तुलसी धारयेद्यस्तु विष्णुप्रीतिकरां शुभाम्}
{विष्णुलोकमवाप्नोति सर्वपापैः प्रमुच्यते} %॥५०॥

\twolineshloka
{ब्राह्मणान्भोजयेत्पश्चाद्विष्णुमुद्दिश्य पाण्डव}
{यस्तु सुप्ते हृषीकेशे दूर्वामृतसम्भवाम्} %॥५१॥

\twolineshloka
{सदा प्रातर्वहेन्मूनि त्वं दूर्वे इति मन्त्रतः}
{व्रतान्ते च कुरुश्रेष्ठ दूर्वी स्वर्णविनिर्मिताम्} %॥५२॥

\twolineshloka
{दद्याद् दक्षिणया साई मन्त्रेणानेन सुव्रत}
{यथाशाखाप्रशाखाभिर्विस्तृतासि महीतले} %॥५३॥

\twolineshloka
{तथा ममापि सन्तानं देहि त्वमजरामरम्}
{नाशुभं प्राप्नुयाजातु पापेभ्यःप्रविमुच्यते} %॥५४॥

\twolineshloka
{भुक्त्वा तु सकलान् भोगान् स्वर्गलोके महीयते}
{गीतं तु देवदेवस्य केशवस्य शिवस्य वा} %॥५५॥

\twolineshloka
{करोति पुरतो नित्यं जागृतेः फलमाप्नुयात्}
{चातुर्मास्यवती दद्याद् घण्टां देवाय सुस्वराम्} %॥५६॥

\twolineshloka
{सरस्वति जगन्नाथे जगजाड्यापहारिणि}
{साक्षाद्ब्रह्मकलयं च विष्णुरुद्रादिभिः स्तुता} %॥५७॥

\twolineshloka
{गुरोरवज्ञया यच्चानध्यायऽध्ययनं कृतम्}
{तन्ममाध्ययनोत्पन्नं जाड्यं हर वरानने} %॥५८॥

\twolineshloka
{घण्टादानेन तुष्टा त्वं ब्रह्माणी लोकपावनी}
{विप्रपादविनिर्मुक्तं तोयं यःप्रत्यहं पिबेत्} %॥५९॥

\twolineshloka
{चातुर्मास्ये नरो भक्त्या मद्रूपं ब्राह्मणं स्मरन्}
{मनोवाकायजनितैर्मुक्तो भवति किल्बिः} %॥६॥

\twolineshloka
{व्याधिभिर्नामिभूयेत श्रीरायुस्तस्य वर्धते}
{समाप्तौ गोयुगं दद्याद्गामेकां वा पयस्विनीम्} %॥६१॥

\twolineshloka
{तत्राप्यशक्तौ राजेन्द्र दद्याद्वासोयुगं व्रती}
{ब्राह्मणं वन्दते यस्तु सर्वदेवमयं श्रुतम्} %॥६२॥

\twolineshloka
{कृतकृत्यो भवेत्सद्यः सर्वपापैः प्रमुच्यते}
{समाप्तौ भोजयेद्विप्रानायुवित्तं च विन्दति} %॥६३॥

\twolineshloka
{संस्पृशेत्कपिलां यो वै नित्यं भक्तिसमन्वितः}
{तामेवालङ्कृतां दद्यात्सवत्सां दक्षिणायुताम् '} %॥६४॥

\twolineshloka
{सार्वभौमो भवेद्राजा दीर्घायुश्च प्रतापवान्}
{स वसेदिन्द्रवत्स्वर्गे वत्सरान् रोमसम्मितान्} %॥६५॥

\twolineshloka
{नमस्करोति यः सूर्य गणेशं वाऽपि नित्यशः}
{आयुरारोग्यमैश्वर्य लभते कान्तिमुत्तमाम्} %॥६६॥

\twolineshloka
{विनराजप्रसादेन प्राप्नुयादीप्सितं फलम्}
{सर्वत्र विजयं चैव नात्र कार्या विचारणा} %॥६७॥

\twolineshloka
{विघ्नेशाकों सुवर्णस्य सिन्दूरारुणसन्निभौ}
{निवेदयेद्राह्मणाय सर्वकामार्थसिद्धये} %॥६८॥

\twolineshloka
{यस्तु रौप्यं शिवप्रीत्यै दद्याद्भक्त्या ऋतुद्वये}
{तानं वा प्रत्यहं दद्यात्स्वशक्त्या शिवतुष्टये} %॥६९॥

\twolineshloka
{सुरूपाल्लभते पुत्रान् रुद्रभक्तिपरायणान्}
{समाप्तौ मधुपूर्ण तु पात्रं राजतमुत्तमम्} %॥७०॥

\twolineshloka
{प्रदद्यात्ताम्रदाने तु ताम्रपात्रं गुडान्वितम्}
{यस्तु सुप्ते हृषीकेशे स्वर्ण दद्याद स्वशक्तितः} %॥७१॥

\twolineshloka
{वस्त्रयुग्मतिलैः सार्ध सर्वपापैः प्रमुच्यते}
{इह भुक्त्वा महाभोगानन्ते शिवपुरं व्रजेत्} %॥७२॥

\twolineshloka
{वस्त्र दानं तु यः कुर्याचातुर्मास्ये द्विजातये}
{अभ्यर्च्य गन्धपुष्पायेविष्णुमें प्रीयतामिति} %॥७३॥

\twolineshloka
{शय्यां दद्यात्समाप्तौ तु वासः काञ्चनपट्टिकाम्}
{अक्षय्यं सुखमाप्नोति धनं स धनदोपमम्} %॥७४॥

\twolineshloka
{यो गोपीचन्दनं दद्यानित्यं वर्षासु मानवः}
{श्रीपतिस्तस्य सन्तुष्टो भुक्तिं मुक्तिं ददाति च} %॥७५॥

\twolineshloka
{समाप्तावपि तदद्यात्तुलापरिमितं शुभम्}
{तदर्ध वा तदर्ध वा सवस्त्रं च सदक्षिणम्} %॥७६॥

\twolineshloka
{यस्तु सुप्ते हृषीकेशे प्रत्यहं तु व्रतान्वितः}
{दद्याद् दक्षिणया सार्ध शर्करमथवा गुडम्} %॥७७॥

\twolineshloka
{एवं व्रते तु सम्पूर्ण कुर्वत्रुद्यापनं बुधः}
{प्रत्येक ताम्रपात्राणि पलाष्टकमितानि तु} %॥७८॥

\twolineshloka
{वित्तशाठयमकुर्वाणश्चतुष्पलमितानि वा}
{अष्टचत्वारि चैक वा शर्करापूरितानि च} %॥७९॥

\twolineshloka
{दक्षिणाफलवासोभिः प्रत्येकं संयुतानि च}
{सह धान्यानि विप्रेभ्यः श्रद्धया प्रतिपादयेत्} %॥८०॥

\twolineshloka
{ताम्रपात्रं सवस्त्रं च शर्कराहेमसंयुतम्}
{सूर्यप्रीतिकरं यस्माद्रोगघ्नं पापनाशनम्} %॥८१॥

\twolineshloka
{पुष्टि कीर्तिदं नृणां नित्यं सन्तानकारकम्}
{सर्वकामप्रदं स्वर्ग्यमायुर्वर्धनमुत्तमम्} %॥८२॥

\twolineshloka
{तस्मादस्य प्रदानेन कीर्तिरस्तु सदा मम}
{एवं व्रतं तु यः कुर्यात्तस्य पुण्यफलं शृणु} %॥८३॥

\twolineshloka
{गन्धर्वविद्यासम्पन्नः सर्वयोषित्प्रियो भवेत्}
{राजापि लभते राज्यं पुत्रार्थी लभते सुतान्} %॥८४॥

\twolineshloka
{अर्थार्थी प्राप्नुयादर्थ निष्कामो मोक्षमाप्तुयात्}
{यस्तु वै चतुरो मासाञ्छाकमूलफलादिकम्} %॥८५॥

\twolineshloka
{नित्यं ददाति विप्रेभ्यः शक्त्या यत्सम्भवेन्नृप}
{व्रतान्ते वस्त्रयुग्मं च शक्त्या दद्यात्सदक्षिणम्} %॥८६॥

\twolineshloka
{सुखीभूत्वा चिरं काल राजयोगी भवेन्नरः}
{सर्वदेवप्रियं यस्माच्छाकं तृप्तिकरं नृणाम्} %॥८७॥

\twolineshloka
{ददामि तेन देवाद्याः सदा कुर्वन्तु मङ्गलम्}
{यस्तु सुप्ते हृषीकेशे प्रत्यहं तु ऋतुद्ये} %॥८८॥

\twolineshloka
{दद्यात्कटुवयं मयों गृहपर्याप्तमादरात्}
{ब्राह्मणाय सुशीलाय दिनेशप्रीतये नघ} %॥८९॥

\twolineshloka
{दक्षिणावस्त्रसहितं मन्त्रेणानेन सुव्रत}
{कटुत्रयमिदं यस्माद्रोगनं सर्वदेहिनाम्} %॥९०॥

\twolineshloka
{तस्मादस्य प्रदानेन प्रीतो भवतु भास्करः}
{एवं कृत्वा व्रतं सम्यकुर्यादुद्यापनं बुधः} %॥९१॥

\twolineshloka
{कृत्वा स्वर्णमयों शुण्ठी मरीचं मागधीमपि}
{सवस्त्रां दक्षिणायुक्तां दद्याद्विप्राय धीमते} %॥९२॥

\twolineshloka
{एवं व्रतं यः कुरुते स जीवेच्छरदां शतम्}
{प्राप्नुयादीप्सितानानन्ते स्वर्ग ब्रजेनृप} %॥९३॥

\twolineshloka
{मुक्ताफलानि यो दद्यानित्यं विप्राय सन्मतिः}
{अन्नवान्कीर्तिमाञ्छीमाञ्जायते वसुधाधिप} %॥९॥

\twolineshloka
{ताम्बूलदानं यःकुर्याद्वर्जयेता जितेन्द्रियः}
{रक्तवस्त्रद्वयं दद्यात्समाप्तौ च सदक्षिणम्} %॥९५॥

\twolineshloka
{महालावण्यमानोति सर्वरोगविवर्जितः}
{मेधावी तुभगःप्राज्ञो रक्तकण्डव जायते} %॥९६॥

\twolineshloka
{गाधर्वत्वमवानोति स्वर्गलोकं च गच्छति}
{ताम्बूलं श्रीकर भद्रं ब्रह्मविष्णुशिवात्मकम्} %॥९७॥

\twolineshloka
{अस्य प्रदानाद्ब्रह्माद्याः श्रियं ददतु पुष्कलाम्}
{चातुर्मासे प्रतिदिनं सुवासिन्यै द्विजाय च} %॥९८॥

\twolineshloka
{नारीवा पुरुषो वाऽपि हरिद्रां सम्प्रयच्छति}
{लक्ष्मीमुद्दिश्य गौरी वा समाप्तौ राजतं नवन्} %॥९९॥

\twolineshloka
{हरिद्रापूरितं कृत्वा तत्पात्र दक्षिणान्वितम्}
{प्रदद्याद्भक्तिसंयुक्तं देवी मे प्रीयतामिति} %॥१००॥

\twolineshloka
{भ; सह मुखं मुङ्क्त नारी नार्या तथा पुमान्}
{सौभाग्यप्रक्षयं धान्यं धनपुत्रसमुन्नतिम्} %॥१०१॥

\twolineshloka
{सम्प्राप्य रूपलावण्ये देवीलोके महीयते}
{उमामहेशमुद्दिश्य चातुर्मास्ये दिने दिने} %॥१०२॥

\twolineshloka
{सम्पूज्य विप्रमिथुनं तस्मै यश्च स्वशक्तितः}
{दद्यात् सदक्षिणं हेम उमेशः प्रीयतामिति} %॥१०३॥

\twolineshloka
{उमेशप्रतिमा हैमी दयादुद्यापने बुधः}
{पञ्चोपचारैः सम्पूज्य धेन्वा च वृषभेण च} %॥१०४॥

\twolineshloka
{भोजयेदपि मिष्टानं तस्य पुण्यफलं शृणु}
{सम्पत्तिरक्षया कीर्तिर्जायते व्रतवैभवात्} %॥१०५॥

\twolineshloka
{इह भुक्त्वाखिलान्कामानन्ते शिवपुरं व्रजेत्}
{फलदानं तु यः कुर्याश्चातुर्मास्यमतन्द्रितः} %॥१०६॥

\twolineshloka
{समाप्तौ कलधौतानि तानि दद्याट्विजातथे}
{सन्निनोरयान्प्राप्य सन्तति चानपायिनीम्} %॥१०७॥

\twolineshloka
{फलदानस्य माहात्म्यान्मोदते नन्दने बने}
{पुष्पदानव्रते चापि स्वर्णपुष्पादि दापयेत्} %॥१०८॥

\twolineshloka
{स सौभाग्यं परं प्राप्य गन्धर्वपदमाप्नुयात्}
{वासुदेवे प्रसुप्ते तु चातुर्मास्यमतन्द्रितः} %॥१०९॥

\twolineshloka
{नित्यं वामनमुद्दिश्य दध्यन्नं स्वादु षड्सैः}
{भोजये इथवः दद्यादेकादश्यां न भोजयेत्} %॥११०॥

\twolineshloka
{दानमेव प्रकुर्वीत ग्रहणादौ तथैव च}
{अशक्तौ नित्यदाने तु कुर्यात्तश्च पर्वतु} %॥१११॥

\twolineshloka
{भूताष्टम्याममायां च पूर्णिमायां तथैव च}
{प्रत्यकवारमथवा प्रतिभार्गववासरम्} %॥११२॥

\twolineshloka
{एवं कृत्वा समाप्तौ तु यथाशक्ति महीं ददेव}
{अशक्तौ भूमिदाने तु धेतुं दद्यादलङ्कृताम्} %॥११३॥

\twolineshloka
{तत्राप्यशक्तौ वासश्च सरुक्मे पादुके तथा}
{अक्षय्यमनमामोति पुत्रपौत्रादिसम्पदम्} %॥११४॥

\twolineshloka
{सुस्थिरां विष्णुभक्तिं च प्रयाति हरिमन्दिरम्}
{नित्यं पयस्विनी दद्यात्सालङ्कारों शुभावहाम्} %॥११५॥

\twolineshloka
{सवत्सां दक्षिणोपेतां स सर्वज्ञानवान् भवेत्}
{न परप्रेष्यतां याति ब्रह्मलोकं च मच्छति} %॥११६॥

\twolineshloka
{अक्षय्यं सुखमाप्नोति पितृभिः सहितो नरः}
{वार्षिकांश्चतुरो मासान् प्राजापत्यं चरेन्नरः} %॥११७॥

\twolineshloka
{समाप्तौ गोयुग दत्त्वा कृत्वा ब्राह्मणभोजनम्}
{सर्वपापविशुद्धात्मा याति ब्रह्म सनातनम्} %॥११८॥

\twolineshloka
{एकान्तरोपवासे तु सीराण्यष्टौ प्रदापयेत्}
{वस्त्रकाञ्चनयुक्तानि बलीवर्दयुतानि च} %॥११९॥

\twolineshloka
{अनछुट्टयसंयुक्तं लाल कर्षणक्षमम्}
{सर्वोपस्करसंयुक्तं ददामि प्रीतये हरेः} %॥१२०॥

\twolineshloka
{शाकमूलफलैर्वाऽपि चातुर्मास्यं नयेन्नरः}
{समाप्तौ गोप्रदानेन स गच्छेद्विष्णुमन्दिरम्} %॥१२१॥

\twolineshloka
{पयोव्रती तथाप्नोति ब्रह्मलोक सनातनम्}
{व्रतान्त च तथा दद्याद्गामेकां च पयस्विनीम्} %॥१२२॥

\twolineshloka
{नित्यं रम्भापलाशे च ये भुङ्क्त तु ऋतुद्वये}
{वस्त्रयुग्मं च कांस्यं च शक्स्या दत्त्वा सुखी भवेत्} %॥१२३॥

\twolineshloka
{कांस्यं ब्रह्मा शिवो लक्ष्मीः कास्यमेव विभावसुः}
{कांस्यं विष्णुमयं यस्मादतः शान्ति प्रयच्छ मे} %॥१२४॥

\twolineshloka
{नित्यं पलाशभोजी चेत्तैलाभ्यङ्गविवर्जितः}
{स निहन्त्यतिपापानि तूलराशिमिवानलः} %॥१२५॥

\twolineshloka
{ब्रह्मन्नश्च सुरापश्च बालघातकरश्च यः}
{असत्यवादिनो ये च स्त्रीधातिव्रतघातकाः} %॥१२६॥

\twolineshloka
{अगम्यागामिनश्चैव विधवागामिनस्तथा}
{चाण्डालीगामिनश्चैव विप्रस्त्रीगामिनस्तथा} %॥१२७॥

\twolineshloka
{ते सर्वे पापनिर्मुक्ता भवन्त्येतद्रतेन च}
{समाप्तौ कांस्यपात्रं तु चतुःषष्टिपलैर्युतम्} %॥१२८॥

\twolineshloka
{सवत्सां गां च वै दद्यात्सालङ्कारां पयस्विनीम्}
{अलङ्कृताय विदुषे सुवस्त्राय सुत्रेषिणे} %॥१२९॥

\twolineshloka
{भूमौ विलीप्य यो भुक्ते देव नारायणं स्मरन्}
{दद्याद्भूमि यथाशक्ति कृष्यां बहुजलान्वितान्} %॥१३०॥

\twolineshloka
{आरोग्य पुत्रसम्पन्नो राजा भवति धार्मिकः}
{शत्रोभयं न लभते विष्णुलोकं स गच्छति} %॥१३१॥

\twolineshloka
{अयाचिते त्वनड्वाहं सहिरण्यं सचन्दनम्}
{षडसं भोजनं दद्यात्स याति परमां गतिम्} %॥१३२॥

\twolineshloka
{यस्तुसुप्ते हृषीकेशे नक्तं च कुरुते व्रतम्}
{ब्राह्मणान्भोजयेत्पश्चाच्छिक्लो महीयते} %॥१२३॥

\twolineshloka
{एकभक्तं नरः कृत्वा मिताशी च दृढव्रतः}
{योर्चयेचतुरो मासान्वासुदेवं स नाकभाक्} %॥१३४॥

\twolineshloka
{समाप्तौ भोजयेदिप्राञ्छक्त्या दद्याच्च दक्षिणाम्}
{यस्तु सुप्ते हृषीकेशे क्षितिशायी भवेन्नरः} %॥१३५॥

\twolineshloka
{शय्यां सोपस्करां दद्याच्छिवलोके महीयते}
{पादाभ्यनं नरो यस्तु वर्जयेच्च ऋतुद्वये} %॥१३६॥

\twolineshloka
{समाप्तौ च यथाशक्ति कुर्याद्राह्मणभोजनन्}
{दत्त्वा च दक्षिणां शक्त्या स गच्छेद्विष्णुमन्दिरम्} %॥१३७॥

\twolineshloka
{आषाढादिचतुर्मासान्वर्जयेन्नखकृन्तनम्}
{आरोग्यपुत्रसम्पन्नो राजा भवति धार्मिकः} %॥१३८॥

\twolineshloka
{पायसं लवणं चैव मधुसर्पिः फलानि च}
{चातुर्मास्ये वर्जयेद्यो गौरीशङ्करतुष्टये} %॥१३९॥

\twolineshloka
{कार्तिक्यां च पुनस्तानि ब्राह्मणाय निवेदयेत्}
{स रुद्रलोकमाप्नोति रुद्रव्रतनिषेवणात्} %॥१४०॥

\twolineshloka
{यवान्नं भक्षयेद्यस्तु शुभं शाल्यन्नमेव वा}
{पुत्रपौत्रादिभिः सार्ध शिवलोके महीयते} %॥१४१॥

\twolineshloka
{तैलाभ्यङ्गपरित्यागी विष्णुभक्तः सदा व्रती}
{वर्षासु विष्णुमभ्यर्च्य वैष्णवी लभते गतिम्} %॥१४२॥

\twolineshloka
{समाप्तौ कांस्यपात्रं च सुवर्णेन समन्वितम्}
{तैलेन पूरितं कृत्वा ब्राह्मणाय निवेदयेत्} %॥१४३॥

\twolineshloka
{वार्षिकांश्चतुरो मासाञ्छाकानि परिवर्जयेत्}
{व्रतान्ते हरिमुद्दिश्य पात्रं राजतमेव हि} %॥१४४॥

\twolineshloka
{वस्त्रेण वेष्टितं शाकदशकेन प्रपूरितम्}
{समभ्यर्च्य यथाशक्त्या ब्राह्मणान्वेदपारगान्} %॥१४५॥

\twolineshloka
{तेभ्यो दद्यादक्षिणया व्रतसम्पूर्तिहेतवे}
{शिवसायुज्यमाप्नोति प्रसादाच्छूलपाणिनः} %॥१४६॥

\twolineshloka
{गोधूमवर्जनं कृत्वा भोजनव्रतमाचरेत्}
{कार्तिके स्वर्णगोधूमान वस्त्रं दत्त्वाऽश्वमेधकृत्} %॥१४७॥

\twolineshloka
{गोधूमाः सर्वजन्तूनां बलपुष्टिविवर्धनाः}
{मुख्याश्च हव्यकव्येषु तस्मान्मे ददतु श्रियम्} %॥१४८॥

\twolineshloka
{आषाढादिचतुर्मासान्वृन्ताकं वर्जयेन्नरः}
{कारवेल्लफलं वाऽपि तथालाबू पटोलकम्} %॥१४९॥

\twolineshloka
{यद्यत्फलं प्रियतरं तच्चापि परिवर्जयेत्}
{चातुर्मास्ये ततो वृत्ते रौप्याण्येतानि कारयेत्} %॥१५०॥

\twolineshloka
{मध्ये विद्रुमयुक्तानि ह्ययित्वा तु शक्तितः}
{दद्यादक्षिणया सार्ध ब्राह्मणायातिभक्तितः} %॥१५१॥

\twolineshloka
{अभिष्ट देवमुद्दिश्य देवो मे प्रीयतामिति}
{स दीर्घमायुरारोग्यं पुत्रपौत्रान्सुरूपताम्} %॥१५२॥

\twolineshloka
{अक्षय्यो सन्तात कीर्ति लब्ध्वा स्वर्गे महीयते}
{श्रावणे वर्जयेच्छाकं दधि भाद्रपदे तथा} %॥१५३॥

\twolineshloka
{दुग्धमाश्वयुजे मासि कातक द्विदलं त्यजेत्}
{चत्वार्येतानि नित्यानि चातुराश्रमवर्तिनाम्} %॥१५४॥

\twolineshloka
{कूष्माण्डं राजमाषांश्च मूलकं गृञ्जनं तथा}
{करमर्द चक्षुदण्डं चातुर्मास्ये त्यजन्नरः} %॥१५५॥

\twolineshloka
{मसूरं बहुबीजं च वृन्ताङ्क चैव वर्जयेत्}
{नित्यान्येतानि विप्रेन्द्र व्रतान्याहुर्मनीषिणः} %॥१५६॥

\twolineshloka
{विशेषाद्वदरी धात्रीमलाबु चिश्चिणी त्यजेत्}
{वार्षिकांश्चतुरो मासान्प्रसुप्ते च जनार्दने} %॥१५७॥

\twolineshloka
{मञ्चखट्या दिशयनं वर्जयेद्भक्तिमानरः}
{अनृतौ वर्जयेद्भार्यामृतौ गच्छन्न दुष्यति} %॥१५८॥

\twolineshloka
{मधुवल्ली च शिग्रं च चातुर्मास्ये त्यजेनरः}
{वृन्ताकं च कलिङ्गं च बिल्वोदुम्बरभिस्सष्टाः} %॥१५९॥

\twolineshloka
{उदरे यस्य जीर्यन्ते तस्य दूरतरो हरिः}
{उपवासं तथा नक्तमेकभक्तमयाचितम्} %॥११६॥

\twolineshloka
{अशक्तस्तु यथाकुर्यात्सायम्प्रातरखण्डितम्}
{स्नानपूजादिकं यस्तु स नरो हरिलोकभाक्} %॥१६१॥

\twolineshloka
{गीतवाद्यपरो विष्णोर्गान्धर्व लोकमाप्नुयात्}
{मधुत्यामी भवेहाजा पुरुषो गुडवर्जनात्} %॥१६२॥

\twolineshloka
{लभेच्च सन्तति दीर्घा पुत्रपौत्रादिवर्धिनीम्}
{तैलस्य वर्जनाद्राजन् सुदर्शाङ्गः प्रजायते} %॥१६३॥

\twolineshloka
{कौसुम्भतैलसन्त्यागाच्छत्रुनाशमवाप्नुयात्}
{मधूकतेलत्यागाच सुसौभाग्यफलं लभेत्} %॥१६४॥

\twolineshloka
{कटुतिक्ताम्लमधुरकषायलवणान् रसान्}
{वर्जयेत्स च वैरूप्यं दौर्गन्ध्यं नाप्नुयात्सदा} %॥१६५॥

\twolineshloka
{पुष्पादिभोगत्यागेन स्वर्गे विद्याधरो भवेत्}
{योगाभ्यासी भवेद्यस्तु स ब्रह्मपदवीमियात्} %॥१६६॥

\twolineshloka
{ताम्बूलवर्जनाद्रोगी सद्योमुक्तामयो भवेत्}
{पादाभ्यङ्गपरित्यागाच्छिरोऽभ्यङ्गस्य पार्थिव} %॥१६७॥

\twolineshloka
{दीप्तिमान्दीप्तकरणो यक्षद्रव्यपतिर्भवेत्}
{दधिदुग्धपरित्यागी गोलोकं लभते नरः} %॥१६८॥

\twolineshloka
{इन्द्रलोकमवाप्नोति स्थालीपाकविवर्जनात्}
{एकान्तरोपवासेन ब्रह्मलोके महीयते} %॥१६९॥

\twolineshloka
{चतुरो वार्षिकामासानखरोमाणि धारयेत्}
{कल्पस्थायी भवेद्राजन्स नरो नात्र संशयः} %॥१७०॥

\twolineshloka
{नमो नाराय गायेति जपित्वानन्तकं फलम्}
{विष्णुपादाम्बुजस्पर्शात्कृत्यकृत्यो भवेन्नरः} %॥१७१॥

\twolineshloka
{लक्षदक्षिगाभिः सेवते हरिमव्ययम्}
{हंसयुक्तविमानेन स याति वैष्णवीं पुरीम्} %॥१७२॥

\twolineshloka
{त्रिरात्रभोजनत्यागान्मोदते दिवि देववत्}
{परानवर्जनाद्राजन्देवो वै मानुषो भवेत्} %॥१७३॥

\twolineshloka
{प्राजापत्यं चरेद्यो वै चातुर्मास्ये व्रतं नरः}
{मुच्यते पातकै सवैत्रिविधैर्नात्र संशयः} %॥१७४॥

\twolineshloka
{तप्तकृच्छातिकृच्छ्राभ्यां यः क्षिपच्छयनं हरेः}
{स याति परमं स्थानं पुनरावृत्तिवर्जितम्} %॥१७५॥

\twolineshloka
{चान्द्राय गेन यो राजन्मिासचतुष्टयम्}
{दिव्यदेहो भवत्सोऽथ शिवलोकं च गच्छति} %॥१७६॥

\twolineshloka
{चातुर्मास्ये नरो यो वै त्यजेदन्नादिभक्षणम्}
{स गच्छेद्धरिसायुज्यं न भूयस्तु प्रजायते} %॥१७७॥

\twolineshloka
{भिक्षाभोजी नरो यो हि स भवेद्वेदपारगा}
{पयोव्रतेन यो राजन्क्षिपेन्मासचतुष्टयम्} %॥१७८॥

\twolineshloka
{तस्य वंशसमुच्छेदः कदाचिन्नोपपद्यते}
{पञ्चगव्याशनः पार्थ चान्द्रायणफलं लभेत्} %॥१७९॥

\twolineshloka
{दिनत्रयं जलत्यागान्न रोगैरभिभूयते}
{एवमादिवतेः पार्थ तुष्टिमायाति केशवः} %॥१८०॥

\fourlineindentedshloka
{दुग्धाब्धिवीचिशयने भगवाननन्तो}
{यस्मिन्दिने स्वपिति चाथ विबुध्यते च}
{तस्मिन्ननन्यमनसामुपवासमाजां}
{पुंसां ददाति च गतिं गरुडासनोऽसौ} %॥१८१॥

॥इति श्रीभविष्यपुराणे विष्णोः शयन्येकादशीमाहात्म्यं सम्पूर्णम्॥


\hyperref[sec:ekadashi_mahatmyam_vrata_raja]{\closesub}
\clearpage

\sect{श्रावण-कृष्ण-कामिका-एकादशी-माहात्म्यम्}
\label{sec:vrata-raja-shravana-krishna-kamika}

\uvacha{युधिष्ठिर उवाच}

\twolineshloka
{आषाढशुक्लपक्षे तु यदेवशयनव्रतम्}
{तन्मया श्रुतपूर्व हि पुराणे बहुविस्तरम्} %॥१॥

\twolineshloka
{श्रावणे कृष्णपक्षे तु किन्नामैकादशी भवेत्}
{एतत्कथय गोविन्द वासुदेव नमोऽस्तु ते} %॥२॥

\uvacha{श्रीकृष्ण उवाच}

\twolineshloka
{शृणु राजन् प्रवक्ष्यामि व्रतं पापप्रणाशनम्}
{नारदाय पुरा राजन् पृच्छते च पितामहः} %॥३॥

\onelineshloka*
{परं यदुलवांस्तात तदहं ते वदामि च}

\uvacha{नारद उवाच}

\onelineshloka
{भगवञ्छ्रोतुमिच्छामि त्वत्तोऽहं कमलासन} %॥४॥

\twolineshloka
{श्रावणस्यासिते पक्षे किनामैकादशी भवेत्}
{को देवः को विधिस्तस्याः किं पुण्यं कथय प्रभो} %॥५॥

\onelineshloka*
{इति तस्य वचः श्रुत्वा ब्रह्मा वचनमब्रवीत्}

\uvacha{ब्रह्मोवाच}

\onelineshloka
{शृणु नारद ते वच्मि लोकानां हितकाम्यया} %॥६॥

\twolineshloka
{श्रावणैकादशी कृष्णा कामिकेति च नामतः}
{तस्याः श्रवणमात्रेण वाजपेयफलं लभेत्} %॥७॥

\twolineshloka
{तस्यां यः पूजयेद्देवं शङ्खचक्रगदाधरम्}
{श्रीधराख्यं हारं विष्णुं माधवं मधुसूदनम्} %॥८॥

\twolineshloka
{यजते ध्यायतेऽथो वै तस्य पुण्यफलं शृणु}
{न गङ्गायां न काश्यां वै नैमिष न च पुष्करे} %॥९॥

\twolineshloka
{तत्फलं समवाप्नोति यत्फलं विष्णुपूजनात्}
{केदारे च कुरुक्षेत्रे राहुप्रस्ते दिवाकरे} %॥१०॥

\twolineshloka
{न तत्फलमवाप्नोति यत्फलं कृष्णपूजनात्}
{गोदावर्या गुरौ सिंहे व्यतीपाते च गण्डके} %॥११॥

\twolineshloka
{म तकलमवाप्नोति यत्फलं कृष्णपूजनात्}
{ससागरवनोपेतां यो ददाति वसुन्धराम्} %॥१२॥

\twolineshloka
{कामिकानतकारी च ह्युभौ समकलौ स्मृतौ}
{प्रलूयमानां यो धेनुं दद्यात्सोपस्करां नरः} %॥१३॥

\twolineshloka
{तत्फलं समवाप्नोति कामिकाबलकारकः}
{श्रावणे श्रीधरं देवं पूजयेद्यो नरोत्तमः} %॥१४॥

\twolineshloka
{तेनैव पूजिता देवा गन्धवारगपन्नगाः}
{तस्मात्सर्वप्रयत्नेन कामिकादिवसे हरिः} %॥१५॥

\twolineshloka
{पूजनीयो यथाशक्त्या मनुष्यः पापभीरूभिः}
{संसारार्णवमना ये पारपासनाकुलाः} %॥१६॥

\twolineshloka
{तेषामुद्धरणार्थाय कामिकावतमुत्तमम्}
{नातः परतरा काचित्पवित्रा पापहारिणी} %॥१७॥

\twolineshloka
{एवं नारद जानीहि स्वयमाह पुरा हरिः}
{अध्यात्मविद्यानिरर्यकलं प्राप्यते नरैः} %॥१८॥

\twolineshloka
{ततो बहुलरं विद्धि कामिकात्रतसेवनात्}
{रात्रौ जागरणं कुर्यात्कामिकावनकुन्नरः} %॥१९॥

\twolineshloka
{न पश्यति यम रौद्रं नैव पश्यति दुर्गतिम्}
{न गच्छनि कुयोनि च कामिकाव्रतसेवनात्} %॥२०॥

\twolineshloka
{कामिकाया व्रतेनैव कैवल्यं योगिनो गताः}
{तस्मात्सर्वप्रयलेन कर्नव्या नियतात्मभिः} %॥२१॥

\twolineshloka
{तुलसीप्रभवैः पत्रों नरः पूजयेद्धरिम्}
{न व स लिप्यते पापैः पद्मपत्रमिवाम्भसा} %॥२२॥

\twolineshloka
{सुवर्णभारमेकं तु रजतं च चतुर्गुणम्}
{तत्फलं समवाप्नोति तुलसीदलपूजनात्} %॥२३॥

\twolineshloka
{रत्नमौक्तिकवैदूर्यप्रवालादिमिर्चितः}
{न तुष्यति तथा विष्णुस्तुल लीपूजनाद्यथा} %॥२४॥

\twolineshloka
{तुलसीमनरीभिस्तु पूजितो येन केशवः}
{आजन्मकृतपापस्य तेन सम्मार्जिता लिपिः} %॥२५॥

\fourlineindentedshloka
{या दृष्टा निखिलाघसंवशमनी स्पृष्टा घः पावनी}
{रोगाणामभिवन्दिता निरलिनी सिकान्तकत्रासिनी}
{प्रत्यासतिविधाधिनी भगवतः कुष्गस्य संरोपिना}
{न्यस्ता तचर विनुकिकलदा तस्यै तुलस्यै नमः} %॥२६॥

\twolineshloka
{दीपं ददाति यो भयो दिवारात्रौ हरेर्दिने}
{तस्य पुण्यस्य सङ्ख्यानं चित्रगुप्तोऽपि वेचि न} %॥२७॥

\twolineshloka
{कृष्णाने दीको यस्य ज्वले देकादशीदिने}
{पितरस्तस्य तृप्यन्ति अमृतेन दिवि स्थिताः} %॥२८॥

\twolineshloka
{घृतेन दीप प्रज्वाल्य तिलतलेन वा पुनः}
{प्रयाति सूर्यलोकेऽसौ दीपकोटिशतैर्वृतः} %॥२९॥

\twolineshloka
{अयं तवाग्रे कथितः कामिकामहिमा मया}
{अतो नरेः प्रकर्तव्या सर्वपातकहारिणी} %॥३०॥

\twolineshloka
{ब्रह्महत्यापहरणी भ्रूणहत्याविनाशिनी}
{त्रिदिवस्थानदात्री च महापुण्यफलप्रदा} %॥३१॥

\twolineshloka
{श्रुत्वा माहात्म्यमेतस्या नरः श्रद्धासमन्वितः}
{विष्णुलोकमवाप्नोति सर्वपापैः प्रमुच्यते} %॥३२॥

॥इति श्रीब्रह्मवैवर्तपुराणे श्रावणकृष्णैकादश्याः कामिकाया माहात्म्यं सम्पूर्णम्॥


\hyperref[sec:ekadashi_mahatmyam_vrata_raja]{\closesub}
\clearpage

\sect{श्रावण-शुक्ल-पुत्रदा-एकादशी-माहात्म्यम्}
\label{sec:vrata-raja-shravana-shukla-putrada}

\uvacha{युधिष्ठिर उवाच}

\twolineshloka
{श्रावणस्य सिते पक्षे किनामैकादशी भवेत्}
{कथयस्व प्रसादेन ममाने मधुसूदन} %॥१॥

\uvacha{श्रीकृष्ण उवाच}

\twolineshloka
{शृणुष्वावहितो राजन् कथा पापहरा पराम्}
{यस्याः श्रवणमात्रेण वाजपेयफलं लभेत्} %॥२॥

\twolineshloka
{द्वापरल्य युगस्यादौ पुरा माहिष्मतीपुरे}
{राजा महीजिदाख्यातो राज्यं पालयति स्वकम्} %॥३॥

\twolineshloka
{पुत्रहीनस्य तस्यैव न तद्राज्यं सुखप्रदम्}
{अपुत्रस्य सुखं नास्ति इहलोके परत्र च} %॥४॥

\twolineshloka
{यततोऽस्य सुतप्राप्तौ कालो बहुतरो गतः}
{न प्राप्तश्च मुतो राज्ञा सर्वसौख्यप्रदो नृणाम्} %॥५॥

\twolineshloka
{दृष्ट्वात्मानं प्रवयसं राजा चिन्तापरोऽभवत्}
{सदोगतः प्रजामध्य इदं वचनमब्रवीत्} %॥६॥

\twolineshloka
{इहजन्मनि भो लोका न मया पातकं कृतम्}
{अन्यायोपार्जितं वित्तं क्षिप्तं कोशे मया न हि} %॥७॥

\twolineshloka
{ब्रह्मस्वं देवद्रविणं न गृहीतं मया क्वचित्}
{न्यासापहाग्ने न कृतः परस्य बहुपापदः} %॥८॥

\twolineshloka
{सुतवत्पालिता लोका धर्मेण विजिता मही}
{दुष्टेषु पातितो दण्डो बन्धुपुत्रोपमेष्वपि} %॥९॥


\threelineshloka
{शिष्टाः सुपूजिता लोका द्वेष्याश्चापि महाजनाः}
{इत्येवं व्रजते मार्गे धर्मयुक्ते द्विजोत्तमाः}
{कस्मान्मम गृहे पुत्रो न जातस्तद्विचार्यताम्} %॥१०॥

\twolineshloka
{इति वाक्य द्विजाः श्रुत्वा समजाः सपुरोहिताः}
{मन्त्रयित्वा नृपहितं जग्मुस्ते गहनं वनम्} %॥११॥

\twolineshloka
{इतस्ततश्च पश्यन्तश्चाश्रमानुषिसेवितान्}
{नृपतेहितमिच्छन्तो ददृशुर्मुनिसत्तमम्} %॥१२॥

\twolineshloka
{तप्यमानं तपो घोरं चिदानन्दं निरामयम्}
{निराहारं जितात्मानं जितक्रोध सनातनम्} %॥१३॥

\twolineshloka
{लोमशं धर्मतत्त्वज्ञं सर्वशास्त्रविशारदम्}
{दीर्घायुषं महात्मानमनेकब्रह्मसम्मितम्} %॥१४॥

\twolineshloka
{कल्पे गते यस्यैकस्मिन्नेक लोम विशीर्यते}
{अतो लोमशनामानं त्रिकालज्ञ महामुनिम्} %॥१५॥

\twolineshloka
{तं दृष्ट्वा हर्षिताः सर्वे आजग्मुस्तस्य सन्निधिम्}
{यथान्यायं यथार्ह ते नमश्चकुर्यथोदितम्} %॥१६॥

\twolineshloka
{विनयावनताः सर्वे ऊचुश्चैव परस्परम्}
{अस्मद्भाग्यवशादेव प्राप्तोऽयं मुनिसत्तमः} %॥१७॥

\onelineshloka*
{तांस्तथा प्रणतान्दृष्ट्वा ह्युवाच मुनिसत्तमः}

\uvacha{लोमश उवाच}

\onelineshloka
{किमर्थामह सम्प्राप्ताः कथयध्वं च कारणम्} %॥१८॥

\twolineshloka
{मद्दर्शनालादगिरा भवन्तः स्तुवते किमु}
{असंशयं करिष्यामि भवतां यद्वितं भवेत्} %॥१९॥

\onelineshloka*
{परोपकृतये जन्म मादृशानां न संशयः}

\uvacha{जना ऊचुः}

\onelineshloka
{श्रूयताममिधास्यामो वयमागमकारणम्} %॥२०॥

\twolineshloka
{संशयच्छेदनार्थाय तव सन्निधिमागताः}
{पद्मयोनेः परतरस्त्वत्तः श्रेष्ठो न विद्यते} %॥२१॥

\twolineshloka
{अतः कार्यवशात्प्राप्ताः समीपं भवतो वयम्}
{महीजिनाम राजासौ पुत्रहीनोस्ति साम्प्रतम्} %॥२२॥

\twolineshloka
{वयं तस्य प्रजा ब्रह्मन् पुत्रवत्तेन पालिताः}
{तं पुत्ररहितं दृष्ट्वा तस्य दुःखेन दुःखिताः} %॥२३॥

\twolineshloka
{तपः कर्तुमिहायाता मतिं कृत्वा तु नैष्ठिकीम्}
{तस्य भाग्यवशादृष्टस्त्वमस्माभिर्द्विजोत्तम} %॥२४॥

\twolineshloka
{महतां दर्शनेनैव कार्यसिद्धिर्भवेन्नृणाम्}
{उपदेशं वद मुने राज्ञः पुत्रो यथा भवेत्} %॥२५॥

\twolineshloka
{इति तेषां वचः श्रुत्वा मुहूर्त ध्यानमास्थितः}
{प्रत्युवाच मुनिख़त्वा तस्य जन्म पुरातनम्} %॥२६॥

\uvacha{लोमश उवाच}

\twolineshloka
{पूर्वजन्मनि वैश्योऽयं धनहीनो नृशंसकृत्}
{वाणिज्यकर्मनिरतो मामाद् प्रामान्तरं भ्रमन्} %॥२७॥

\twolineshloka
{ज्येष्ठे मासि सिते पक्ष द्वादशीदिवसे तथा}
{मध्याह्ने ह्युमणौ प्राप्ते ग्रामसीन्नि तृषाकुलः} %॥२८॥

\twolineshloka
{रम्यं जलाशयं दृष्ट्वा जलपाने मनो दधौ}
{सद्यःसूता सवत्सा च धेनुस्तत्र समागता} %॥२९॥

\twolineshloka
{तृषातुरा निदाघार्ता तस्य चापः पपौ तु सा}
{पिबन्ती वारयित्वा तामसी तोयं स्वयं पपौ} %॥३०॥

\twolineshloka
{कर्मणस्तस्य पाकेन पुत्रहीनो नृपोऽभवत्}
{पूर्वजन्मकृतात्पुण्यात्प्राप्तं राज्यमकण्टकम्} %॥३१॥

\uvacha{जना ऊचुः}


\twolineshloka
{पुग्यात्पापं क्षयं याति पुराणे श्रूयते मुने}
{पुण्योपदेशं कथय येन पापक्षयो भवेत्} %॥३२॥

\onelineshloka*
{यथा भवत्प्रसादेन पुत्रोऽस्य भविता तथा}

\uvacha{लोमश उवाच}

\onelineshloka
{श्रावणे शुक्लपक्षे तु पुत्रदानाम विश्रुता} %॥३३॥

\twolineshloka
{एकादशीतिथिश्चास्ति कुरुध्वं तद्वतं जनाः}
{यथाविधि यथान्याय्य यथोक्तं जागरान्वितम्} %॥३४॥

\twolineshloka
{तस्याः पुण्यं सुविमलं देयं नृपतये जनाः}
{एवं कृते सुनिया राजः पुत्रो भविष्यति} %॥३५॥

\threelineshloka
{श्रुत्वा तु लोमशवचस्तं प्रणम्य द्विजोत्तमम्}
{प्रजग्मुः स्वगृहान् सर्वे हषर्कोत्फुल्लविलोचनाः}
{श्रावणं तु समासाद्य स्मृत्वा लोमशभाषितम्} %॥३६॥

\twolineshloka
{राज्ञा सह व्रतं चक्रुः सर्वे श्रद्धासमन्विताः}
{द्वादशीदिवसे पुण्यं ददुपतये जनाः} %॥३७॥

\twolineshloka
{दत्ते पुण्ये तु सा राज्ञी गर्भमाधत्त शोभनम्}
{प्राप्ते प्रसवकाले सा सुषुवे पुत्रमूर्जितम्} %॥३८॥

\twolineshloka
{एवमेषा नृपश्रेष्ठ पुत्रदानाम विश्रुता}
{कर्तव्या सुखमिच्छद्भिरिह लोके परत्र च} %॥३९॥

\twolineshloka
{श्रुत्वा माहात्म्यमेतस्याः सर्वपापैः प्रमुच्यते}
{इह पुत्रसुखं प्राप्य परत्र स्वर्गतिं लभेत्} %॥४०॥

॥इति श्रीभविष्योत्तरपुराणे पुत्रदाख्यश्रावणशुक्कैकादशीमाहात्म्यं सम्पूर्णम्॥


\hyperref[sec:ekadashi_mahatmyam_vrata_raja]{\closesub}
\clearpage

\sect{भाद्रपद-कृष्णाजा-एकादशी-माहात्म्यम्}
\label{sec:vrata-raja-bhadrapada-krishnaja}

\uvacha{युधिष्ठिर उवाच}

\twolineshloka
{भाद्रस्य कृष्णपक्षे तु किन्नामैकादशी भवेत्}
{एनदिच्छाम्यहं श्रोतुं कथयस्व जनार्दन} %॥१॥

\uvacha{श्रीकृष्ण उवाच}

\twolineshloka
{शृणुष्वैकमना राजन् कथयिष्यामि विस्तरात्}
{अजेति नाम्ना विख्याता सर्वपापप्रणाशिनी} %॥२॥

\twolineshloka
{पूजयित्वा हृषीकेशं व्रतं तस्याः करोति यः}
{पापानि तस्य नश्यन्ति व्रतस्य श्रवणादपि} %॥३॥

\twolineshloka
{नातः परतरा राजैल्लोकद्वयहितावहा}
{सत्यमुक्तं मया ह्येतेन्नासत्यं भाषितं मम} %॥४॥

\twolineshloka
{हरिश्चन्द्र इति ख्यातो बभूव नृपतिः पुरा}
{चक्रवर्ती सत्यसन्धः समस्ताया भुवः पतिः} %॥५॥

\twolineshloka
{कस्यापि कर्मणो योगाद्राज्यभ्रष्टो बभूव सः}
{विक्रीय वनितां पुत्रं स चकारात्मविक्रयम्} %॥६॥

\twolineshloka
{पुल्कप्तस्य च दासत्वं गतो राजा स पुण्यकृत्}
{सत्यमालम्ब्य राजेन्द्र मृतचैलापहारकः} %॥७॥

\twolineshloka
{सोऽभवन्नृपतिश्रेष्ठो न सत्याञ्चलितस्तथा}
{एवं गतस्य नृपतेर्बहवो वत्सरा गताः} %॥८॥

\twolineshloka
{ततश्चिन्तापरो राजा बभूवात्यन्तदुःखितः}
{किं करोमि क्व गच्छामि निष्कृतिमें कथं भवेत्} %॥९॥

\twolineshloka
{इति चिन्तयतस्तस्य मनस्य वृजिनार्णवे}
{आजगाम मुनिः कैश्चिज्ज्ञात्वा राजानमातुरम्} %॥१०॥

\twolineshloka
{परोपकरणार्थाय निर्मितो ब्रह्मणा द्विजः}
{स तं दृष्ट्वा द्विजवरं ननाम नृपसत्तमः} %॥११॥

\twolineshloka
{कृताञ्जलिपुटो भूत्वा गौतमस्याग्रतः स्थितः}
{कथयामास वृत्तान्तमात्मनो दुःखसंयुतम्} %॥१२॥

\twolineshloka
{श्रुत्वा नृपतिवाक्यानि गौतमो विस्मयान्वितः}
{उपदेशं नृपतये व्रतस्यास्य मुनिर्ददौ} %॥१३॥

\twolineshloka
{मासि भाद्रपदे राजन् कृष्णपक्षे तु शोभना}
{एकादशी समाख्याता अजानाम्रातिपुण्यदा} %॥१४॥

\twolineshloka
{तस्याः कुरु व्रतं राजपापनाशो भविष्यति}
{तव भाग्यवशादेषा सप्तमेऽति समागता} %॥१५॥

\twolineshloka
{उपवासपरो भूत्वा रात्री जागरणं कुरु}
{एवं तस्या व्रत चीर्णे सर्वपापक्षयो भवेत्} %॥१६॥

\twolineshloka
{तव पुण्यप्रभावेण चागतोऽहं नृपोत्तम}
{इत्येवं कथयित्वा तु मुनिरन्तरधीयत} %॥१७॥

\twolineshloka
{मुनिवाक्यं नृपः श्रुत्वा चकार व्रतमुत्तमम्}
{कृते तस्मिन्नते राज्ञः पापस्यान्तोऽभवत्क्षणात्} %॥१८॥

\twolineshloka
{श्रूयतां राजशार्दूल प्रभावोऽस्य व्रतस्य च}
{यदुःखं बहुभिर्वर्षेभोक्तव्यं तत्क्षयो भवेत्} %॥१९॥

\twolineshloka
{निस्तीर्ण दुःखो राजासीद्वतस्यास्य प्रभावतः}
{पत्न्या सह समायोगं पुत्रजीवनमाप सः} %॥२०॥

\twolineshloka
{देवदुन्दुभयो नेदुः पुष्पवर्षमभद्दिवः}
{एकादश्याः प्रभावेण प्राप्त राज्यमकण्टकम्} %॥२१॥

\twolineshloka
{स्वर्ग लेभे हरिश्चन्द्रः सपुरः सपरिच्छदः}
{ईदृग्विधं व्रतं राजन् ये कुर्वन्ति द्विजोत्तमाः} %॥२२॥

\twolineshloka
{सर्वपापविनिर्मुक्तास्त्रिदिवं यान्ति ते ध्रुवम्}
{पठनाच्छ्रवणाद्राजन्नश्वमेधफलं भवेत्} %॥२३॥

॥इति श्रीब्रह्माण्डपुराणे भाद्रपदकृष्णाया अजानाम्न्या एकादश्या माहात्म्यं सम्पूर्णम्॥


\hyperref[sec:ekadashi_mahatmyam_vrata_raja]{\closesub}
\clearpage

\sect{भाद्रपद-शुक्ल-पद्मा-एकादशी-माहात्म्यम्}
\label{sec:vrata-raja-bhadrapada-shukla-padma}

\uvacha{युधिष्ठिर उवाच}

\twolineshloka
{नभस्य सितपक्ष तु किन्नामैकादशी भवेत्}
{को देवः को विधिस्तस्याः किं पुण्यं च वदस्व नः} %॥१॥

\uvacha{श्रीकृष्ण उवाच}

\twolineshloka
{कथयामि महापुण्यां स्वर्गमोक्षदायिनीम्}
{वामनैकादशी राजन्सर्वपापहारां पराम्} %॥२॥

\twolineshloka
{इमामेव जयन्त्याख्यां प्राहुरेकादशीं नृप}
{यस्य श्रवणमात्रेण सर्वपापक्षयो भवेत्} %॥३॥

\twolineshloka
{पापिनां पापशमनं जयन्तीव्रतमुत्तमम्}
{नातः पर तरा राजन्न वै मोक्षप्रदायिनी} %॥४॥

\twolineshloka
{एतस्मात्कारणाद्राजन्कर्तव्या गतिमिच्छता}
{वैष्णव मम भक्तैस्तु मनुजैर्मत्परायणैः} %॥५॥

\twolineshloka
{नभस्ये वामनो यैस्तु पूजितस्तैर्जगत्रयम्}
{पूजितं नाव सन्देहस्ते यान्ति हरिसन्निधिम्} %॥६॥

\twolineshloka
{वामनः पूजितो येन कमलैः कमलेक्षणः}
{नभस्यसितपक्षे तु जयन्त्येकादशीदिने} %॥७॥

\twolineshloka
{तेनार्चितं जगत्सर्व वयो देवाः सनातनाः}
{एतस्मात्कारणाद्राजन्कर्तव्यो हरिवासरः} %॥८॥

\twolineshloka
{अस्मिन्कृते न कर्तव्यं किञ्चिदस्ति जगत्रये}
{अस्या प्रसुप्तो भगवानेत्यङ्गपरिवर्तनम्} %॥९॥

\onelineshloka*
{तस्मादेनां जनाः सर्वे वदन्ति परिवर्तिनीम्}

\uvacha{युधिष्ठिर उवाच}

\onelineshloka
{संशयोऽस्ति महान्मह्यं श्रूयतां च जनार्दन} %॥१०॥

\twolineshloka
{कथं सुप्तोऽसि देवेश कथं यास्यङ्गवर्तनम्}
{किमर्थ देवदेवेश बलिर्बद्धस्त्वयासुरः} %॥११॥

\twolineshloka
{सन्तुष्टाः पृथिवीदेवाः किमकुर्वञ्जनार्दन}
{को विधिः किं व्रतं चैव चातुर्मास्यमुपासताम्} %॥१२॥

\twolineshloka
{त्वयि मुप्ते जगनाथ किं कुर्वन्ति जनाः प्रभो}
{एतद्विस्तरतो ब्रूहि संशयं हर मे प्रभो} %॥१३॥

\uvacha{श्रीकृष्ण उवाच}

\twolineshloka
{श्रूयतां राजशार्दूल कथां पापहरी पराम्}
{बालवै दानवः पूर्वमासीत्रेतायुगे नृप} %॥१४॥

\twolineshloka
{अपूजयच्च मां नित्यं मद्भक्तो मत्परायणः}
{जपैस्तु विविधैः सूक्तैर्यजते मांस नित्यशः} %॥१५॥

\twolineshloka
{द्विजानां पूजको नित्यं यज्ञकर्मकृताशयः}
{परन्त्विन्द्रकृतद्वेषो देवलोकमजीजयत्} %॥१६॥

\twolineshloka
{महत्तमिह लोकश्च जितस्तेन महात्मना}
{विलोक्य च ततः सर्वे देवाः संहत्य मन्त्र यन्} %॥१७॥

\twolineshloka
{सवर्मिलित्वा गन्तव्यं देवं विज्ञापितुं प्रभुम्}
{ततश्च देवऋषिभिः साकमिन्द्रो गतः प्रभुम्} %॥१८॥

\twolineshloka
{शिरसा ह्यवनीं गत्वा स्तुत इन्द्रेण सूक्तिभिः}
{गुरुणा देवतैः सार्ध बहुधा पूजितो ह्यहम्} %॥१९॥

\twolineshloka
{ततो वामनरूपेण ह्यवतीर्णश्च पञ्चमः}
{अत्युग्ररूपेण तदा सर्वब्रह्माण्डरूपिणा} %॥२०॥

\onelineshloka*
{बालकेन जितः सोऽथ सत्यमालम्ब्य तस्थिवान्}

\uvacha{युधिष्ठिर उवाच}

\onelineshloka
{त्वया वामनरूपेण सोऽसुरश्च जितः कथम्} %॥२१॥

\onelineshloka*
{एतत्कथय देवेश मह्यं भक्ताय विस्तरात}

\uvacha{श्रीकृष्ण उवाच}

\onelineshloka
{मयाऽलीकेन स बालः प्रार्थितो बटुरूपिणा} %॥२२॥

\twolineshloka
{पदत्रयमिता भूमि देहि मे भुवनत्रयम्}
{दत्तं भवति ते राजन्नात्र कार्या विचारणा} %॥२३॥

\twolineshloka
{इत्युक्तश्च मया राजा दत्तवास्त्रिपदां भुवम्}
{सङ्कल्पमात्राद्विवृधे देहविक्रमः परम्} %॥२४॥

\twolineshloka
{भूलोके तु कृतौ पादौ भुवोङ्के तु जानुनी}
{स्वलोङ्के तु कटिं न्यस्य महर्लोके तथोदरम्} %॥२५॥

\twolineshloka
{जनलोके तु हृदयं तपोलोके च कण्ठकम्}
{सत्यलोके मुखं स्थाप्य उत्तमाङ्ग तथोर्ध्वतः} %॥२६॥

\twolineshloka
{चन्द्रसूर्यग्रहाश्चैव भगणो योगसंयुतः}
{सेन्द्राश्चैव तदा देवा नागाः शेषादयः परे} %॥२७॥

\twolineshloka
{अस्तुवन्वेदसम्भूतैः सूक्तैश्च विविधैस्तु माम्}
{करे गृहीत्वा तु बलिमब्रुवं वचनं तदा} %॥२८॥

\twolineshloka
{एकेन पूरिता पृथ्वी द्वितीयेन त्रिविष्टपम्}
{तृतीयस्य तु पादस्य स्थानं देहि ममानघ} %॥२९॥

\twolineshloka
{एवमुक्ते मया सोऽपि मस्तके दत्तवान्बलिः}
{ततो वै मस्तके ह्येकं पदं दत्तं मया तदा} %॥३०॥

\twolineshloka
{क्षिप्तो रसातले राजन्दानवो मम पूजकः}
{विनयावनतं दृष्ट्वा प्रसन्नोऽस्मि जनार्दनः} %॥३१॥

\twolineshloka
{बले वसामि सततं सन्निधौ तव मानद}
{इत्यवोचं महाभागं बालि वैरोचनि तदा} %॥३२॥

\twolineshloka
{नमस्यशुक्लपक्षे तु परिवर्तिनि वासरे}
{ममैका तत्र मूर्तिश्च बलिमाश्रित्य तिष्ठति} %॥३३॥

\twolineshloka
{द्वितीया शेषपृष्ठे वै क्षीराब्धौ सागरोत्तमे}
{सुप्ता भवति भो भूप यावच्चायाति कार्तिकी} %॥३४॥

\twolineshloka
{एतस्मात्कारणाद्राजन्कर्तव्यैषा प्रयत्नतः}
{एकादशी महापुण्या पवित्रा पापहारिणी} %॥३५॥

\twolineshloka
{अस्यां प्रसुप्तो भगवानेत्यङ्गपरिवर्तनम्}
{एतस्यां पूजयेदेवं त्रैलोक्यस्य पितामहम्} %॥३६॥

\twolineshloka
{दधिदानं प्रकर्तव्यं रौप्यतण्डुलसंयुतम्}
{रात्री जागरणं कृत्वा मुक्तो भवति मानवः} %॥३७॥

\twolineshloka
{एवं यः कुरुते राजनेकादश्या व्रतं शुभम्}
{सर्वपापहरं चैव भुक्तिमुक्तिप्रदायकम्} %॥३८॥

\threelineshloka
{स देवलोकं सम्प्राप्य भ्राजते चन्द्रमा यथा}
{शृणुयाच्चैव यो मर्त्यः कथा पापहरां पराम्}
{अश्वमेधसहस्रस्य फलं प्राप्नोति मानवः} %॥३९॥

॥इति श्रीस्कन्दपुराणे भाद्रपदशुक्लायाः परिवर्तिनीनामैकादश्या माहात्म्यं सम्पूर्णम्॥


\hyperref[sec:ekadashi_mahatmyam_vrata_raja]{\closesub}
\clearpage

\sect{आश्विन-कृष्णेन्दिरा-एकादशी-माहात्म्यम्}
\label{sec:vrata-raja-ashvina-krishnendira}

\uvacha{युधिष्ठिर उवाच}

\twolineshloka
{कथयस्व प्रसादेन ममाग्रे मधुसूदन}
{आश्विने कृष्णपक्षे तु किन्नामकादशी भवेत्} %॥१॥

\uvacha{श्रीकृष्ण उवाच}

\twolineshloka
{आश्विनस्यासिते पक्षे इन्दिरानाम नामतः}
{तस्या व्रतप्रभावेण महापापं प्रणश्यति} %॥२॥

\twolineshloka
{अधोयोनिगतानां च पितॄणां गतिदायिनी}
{शृणुष्वावहितो राजन्कथां पापहरी पराम्} %॥३॥

\twolineshloka
{यस्याः श्रवणमात्रेण वाजपेयफलं लभेत्}
{पुरा कृतयुगे राजा बभूव रिपुसूदनः} %॥४॥

\twolineshloka
{इन्द्रसेन इति ख्यातः पुरी माहिष्मती प्रति}
{सराज्यं पालयामास धर्मेण यशसान्वितः} %॥५॥

\twolineshloka
{पुत्रपौत्रसमायुक्तो धनधान्यसमन्वितः}
{माहिष्मत्यधिपो राजा विष्णुभक्तिपरायणः} %॥६॥

\twolineshloka
{जपन् गोविन्दनामानि मुक्तिदानि नराधिपः}
{ध्यानेन कालं नयति नित्यमध्यात्मचिन्तकः} %॥७॥

\twolineshloka
{एकस्मिन् दिवसे राज्ञि सुखासीने सदोगते}
{अवतीर्यागमद्धीमानम्बरानारदो मुनिः} %॥८॥

\twolineshloka
{तमागतमभिप्रेक्ष्य प्रत्युत्थाय कृताञ्जलिः}
{पूजयित्वाविधिना चासने सन्न्यवेशयत्} %॥९॥

\twolineshloka
{सुखोपविष्टः सः मुनिः प्रत्युवाच नृपोत्तमम्}
{कुशलं तव राजेन्द्र सप्तस्वङ्गेषु वर्तते} %॥१०॥

\twolineshloka
{धर्मे मतिर्वर्तते ते विष्णुभाक्तिरतिस्तथा}
{इति वाक्यं तु देवर्षेः श्रुत्वा राजा तमब्रवीत्} %॥११॥

\uvacha{राजोवाच}

\twolineshloka
{त्वत्प्रसादान्मुनिश्रेष्ठ सर्वत्र कुशलं मम}
{अद्य क्रतुक्रियाः सर्वाः सफलास्तव दर्शनात्} %॥१२॥

\twolineshloka
{प्रसादं कुरु विप्रर्षे ब्रह्मागमनकारणम्}
{इति राज्ञो वचः श्रुत्वा देवर्षिर्वाक्यमब्रवीत्} %॥१३॥

\uvacha{नारद उवाच}

\twolineshloka
{श्रूयतां राजशार्दूल मद्वचो विस्मयप्रदम्}
{ब्रह्मलोकादहं प्राप्तो यमलोकं द्विजोत्तम} %॥१४॥

\twolineshloka
{शमनेनार्चितो भक्त्या उपविष्टो वरासने}
{धर्मशील सत्यवांस्तु भास्करिं समुपासते} %॥१५॥

\twolineshloka
{बहुपुण्यप्रकर्ता च व्रतवैकल्यदोषतः}
{सभायां श्राद्धदेवस्य मया दृष्टः पिता तव} %॥१६॥

\twolineshloka
{कथितस्तेन सन्देशस्त निबोध जनेश्वर}
{इन्द्रसन इति ख्यातो राजा माहिष्मतीप्रभुः} %॥१७॥

\twolineshloka
{तस्याने कथय ब्रह्मन् स्थितं मां यमसन्निधौ}
{केनलि चान्तराये गर्वजन्मोद्वेन वै} %॥१८॥

\twolineshloka
{स्वर्ग प्रेषय मां पुत्र इन्दिरावतदानतः}
{इत्युकोऽहं समायातः समीरं तव पार्थिव} %॥१९॥

\twolineshloka
{पितुः स्वर्गतये राजनिन्दिरावतमाचर}
{तेन व्रतमभावेग स्वर्ग यास्यति ते पिता} %॥२०॥

\uvacha{राजोवाच}

\twolineshloka
{कथयस्व प्रसादेन भगवनिन्दिराव्रतम्}
{विधिना केन कर्तव्यं कस्मिन्पक्षे तिथौ तथा} %॥२१॥

\uvacha{नारद उवाच}

\twolineshloka
{शृगु राजन् हितं वच्मि व्रतस्यास्य विधिं शुभम्}
{आश्विनस्यासिते पक्षे दशमीदिवसे शुभे} %॥२२॥

\twolineshloka
{प्रातः स्नानं प्रकुर्वीत श्रद्वायु तेन चेतसा}
{ततो मध्याह्नसमये स्नानं कृत्वा बहिर्जल} %॥२३॥

\twolineshloka
{पितॄणां प्रीतये श्राद्धं कुर्याच्छ्रद्धासमविन्तः}
{एकभक्तं ततः कृत्वा रात्री भूमौ शयीत च} %॥२४॥

\twolineshloka
{प्रभाते विमले जाते प्राप्ते चैकादशीदिने}
{मुखप्रक्षालनं कुर्यादन्तधावनपूर्वकम्} %॥२५॥

\twolineshloka
{उपवासस्य नियम गृहीयाद्भक्तिभावतः}
{अद्य स्थित्वा निराहारः सर्वभोगविवर्जितः} %॥२६॥

\twolineshloka
{श्वो भोक्ष्ये पुण्डरीकाक्ष शरणं मे भवाच्युत}
{इत्येवं नियमं कृत्वा मध्याह्नसमये तथा} %॥२७॥

\twolineshloka
{शालग्रामशिला तु श्राद्धं कृत्वा यथाविधि}
{भोजयित्वा द्विजाछुद्वान्दक्षिणाभिः सुपूजितान्} %॥२८॥

\twolineshloka
{पितृशेष समावाय गवे दद्याद्विचक्षणः}
{पूजयित्वा हृषीकेशं धूपगन्धादिभिस्तथा} %॥२९॥

\twolineshloka
{रात्री जागरणं कुर्यात्केशवस्य समीपतः}
{ततः प्रभातसमये सम्प्राप्ते द्वादशीदिने} %॥३०॥

\twolineshloka
{अर्चयित्वा हरि भक्त्या भोजयित्वा द्विजानथ}
{बन्धुदौहित्रपुत्राद्यैः स्वयं भुञ्जीत वाङ्ग्यतः} %॥३१॥

\twolineshloka
{अनेन विधिना राजकुरु व्रतमतन्द्रितः}
{विष्णुलो के प्रयास्यन्ति पितरस्तव भूपते} %॥३२॥

\twolineshloka
{इत्युक्त्वा नृपतिं राजन् मुनिरन्तरधीयत}
{यथोक्तविधिना राजा चकार व्रतमुत्तमम्} %॥३३॥

\twolineshloka
{अन्तःपुरेण सहितः पुत्रभृत्यसमन्वितः}
{कृते व्रते तु कौन्तेय पुष्पवृष्टिरभूदिवः} %॥३४॥

\twolineshloka
{तत्पिता गरुडारूढो जगाम हरिमन्दिरम्}
{इन्द्रसेनौऽपि राजर्षिः कृत्वा राज्यमकण्टकम्} %॥३५॥

\twolineshloka
{राज्ये निवेश्य तनयं जगाम विदिवं स्वयम्}
{इन्दिरावतमाहात्म्यं तवाग्रे कथितं मया} %॥३६॥

\twolineshloka
{पठनाच्छ्रवणाचास्य सर्वपापैः प्रमुच्यते}
{भुक्त्वेह निखिलान्भोगाविष्णुलोके वसच्चिरम्} %॥३७॥

॥इति श्रीब्रह्मवैवर्तपुराणे आश्विनकृष्णैकादश्या इन्दिरानाम्न्या माहात्म्यं सम्पूर्णम्॥


\hyperref[sec:ekadashi_mahatmyam_vrata_raja]{\closesub}
\clearpage

\sect{आश्विन-शुक्ल-पाशाङ्कुशा-एकादशी-माहात्म्यम्}
\label{sec:vrata-raja-ashvina-shukla-pashankusha}

\uvacha{धुधिष्ठिर उवाच}

\twolineshloka
{कथयस्व प्रसादेन भगवन् मधुसूदन}
{इषस्य शुक्लपक्षे तु किन्नामैकादशी भवेत्} %॥१॥

\uvacha{श्रीकृष्ण उवाच}

\twolineshloka
{शृणु राजेन्द्र वक्ष्यामि माहात्म्यं पापनाशनम्}
{शुक्लपक्षे चाश्वयुजि भवेदकादशी तु या} %॥२॥

\twolineshloka
{पाशाङ्कुशति विख्याता सर्वपापहरा परा}
{पद्मनाभाभिधानं तु पूजयेत्तत्र मानवः} %॥३॥

\twolineshloka
{सर्वाभीष्टफलप्रात्प्यै स्वर्गमोक्षप्रदं नृणाम्}
{तपस्तप्त्वा नरस्ती चिरं सुनियतेन्द्रियः} %॥४॥

\twolineshloka
{यत्फलं समवाप्नोति तं नत्वा गरुडध्वजम्}
{कृत्वाऽपि बहुशः पापं नरो मोहसमन्वितः} %॥५॥

\twolineshloka
{न याति नरकं घोरं नत्वा पापहरं हरिम्}
{पृथिव्यां यानि तीर्थानि पुण्यान्यायतनानि च} %॥६॥

\twolineshloka
{तानि सर्वाण्यवाप्नोति विष्णोर्नामानुकीर्तनात्}
{देवं शार्ङ्गधरं विष्णु ये प्रपन्ना जनार्दनम्} %॥७॥

\twolineshloka
{न तेषां यमलोकश्च नृणां वै जायते कचित्}
{उपोष्यैकादशीमेको प्रसङ्गनापि मानवाः} %॥८॥

\twolineshloka
{न यान्ति यातनां यामी पापं कृत्वाऽपि दारुणम्}
{वैष्णवः पुरुषो भूत्वा शिवनिन्दा करोति यः} %॥९॥

\twolineshloka
{यो निन्देवैष्णवं लोके स याति नरकं ध्रुवम्}
{अश्वमेधसहस्राणि राजसूयशतानि च} %॥१०॥

\twolineshloka
{एकादश्युपवासस्य कलां नाहन्ति षोडशीम्}
{एकादशीसमं पुण्यं किञ्चिल्लोके न विद्यते} %॥११॥

\twolineshloka
{नेदृशं पावनं किञ्चित्रिषु लोकेषु विद्यते}
{यादृशं पद्मनाभस्य दिन पातकहानिदम्} %॥१२॥


\threelineshloka
{तावत्पापानि तिष्ठन्ति दहेऽस्मिन् मनुजाधिप}
{यावनोपोष्यते भक्त्या एद्मनाभदिनं शुभम्}
{व्याजेनोपोषितमपि न दर्शयति भास्करिम्} %॥१३॥

\twolineshloka
{स्वर्गमोक्षप्रदा ह्येषा शरीरारोग्यदायिनी}
{सुकलप्रदा ह्येषा धनधान्यप्रदायिनी} %॥१४॥

\twolineshloka
{न गङ्गा न गया राजन काशी न च पुष्करम्}
{न चापि कौरवं क्षेत्रं पुण्यं भूप हरेर्दिनात्} %॥१५॥

\twolineshloka
{रागे जागरणं कृत्वा समुपोष्य हरेर्दिनम्}
{अनायासेन भूपाल प्राप्यते वैष्णवं पदम्} %॥१६॥

\twolineshloka
{दश वै मातृके पक्षे दश राजेन्द्र पैतृके}
{प्रियाया दश पक्षे तु पुरुषानुद्धरेनरः} %॥१७॥

\twolineshloka
{चतुर्भुजा दिव्यरूपा नागारिकृतकेतनाः}
{स्रग्विणः पीतवस्त्राश्च प्रयान्ति हरिमन्दिरम्} %॥१८॥

\twolineshloka
{बालत्वे यौवने चैव वृद्धत्वेऽपि नृपोत्तम}
{उपोष्य द्वादशी नूनं नैति पापोऽपि दुर्गतिम्} %॥१९॥

\twolineshloka
{पाशाकुशामुपोष्यैव आश्विने चासितेतरे}
{सर्वपापविनिर्मुक्तो हरिलोकं स गच्छति} %॥२०॥

\twolineshloka
{दत्त्वा हेमतिलान् भूमि गामनमुदकं तथा}
{उपानद्वस्वच्छत्रादि न पश्यति यमं नरः} %॥२१॥

\twolineshloka
{यस्य पुण्यविहीनानि दिनान्यपगतानि च}
{स लोहकारभस्नेव श्वसनपि न जीवति} %॥२२॥

\twolineshloka
{अवन्ध्यं दिवसं कुर्यादरिद्रोऽपि नृपोत्तम}
{समाचरन्यथाशक्ति स्नानदानादिकाः क्रियाः} %॥२३॥

\twolineshloka
{तडागारामसौधानां सत्राणां पुण्यकर्मणाम्}
{कर्तारो नैव पश्यन्ति धीरास्तां यमयातनाम्} %॥२४॥

\twolineshloka
{दीर्घायुषो धनाढयाश्च कुलीना रोगवर्जिताः}
{दृश्यन्ते मारवा लोके पुण्यकर्तार ईदृशाः} %॥२५॥

\twolineshloka
{किमत्र बहुमोक्तेन यान्त्यधर्मेण दुर्गतिम्}
{आरोहन्ति दिवं धर्मेात्र कार्या विचारणा} %॥२६॥

\twolineshloka
{इति ते कथितं राजन् यत्पृष्टोऽहं त्वयानघ}
{पाशाङ्कुशाया माहात्म्यं किमन्यच्छोतुमिच्छसि} %॥२७॥

॥इति श्रीब्रह्माण्डपुराणे आश्विनशुक्लैकादश्याः पाशाङ्कुशाख्याया माहात्म्यं सम्पूर्णम्॥


\hyperref[sec:ekadashi_mahatmyam_vrata_raja]{\closesub}
\clearpage

\sect{कार्त्तिक-कृष्ण-रमा-एकादशी-माहात्म्यम्}
\label{sec:vrata-raja-karttika-krishna-rama}

\uvacha{युधिष्ठिर उवाच}

\twolineshloka
{कथयस्व प्रसादेन मम स्नेहाजनार्दन}
{कार्तिकस्यासिते पक्षे किन्नामेकादशी भवेत्} %॥१॥

\uvacha{श्रीकृष्ण उवाच}

\twolineshloka
{श्रूयतां राजशार्दूल कथयामि तवाग्रतः}
{कार्तिके कृष्णपक्षे तु रमानाम्नी सुशोभना} %॥२॥

\twolineshloka
{एकादशी समाख्याता महापापहरा परा}
{अस्याः प्रसअतो राजन् माहात्म्य प्रवदामि ते} %॥३॥

\twolineshloka
{मुचुकुन्द इति ख्यातो बभूव नृपतिः पुरा}
{देवेन्द्रेण समं यस्य मित्रत्वमभवनृप} %॥४॥

\twolineshloka
{यमेन वरुणेनैव कुबेरेण समं तथा}
{विभीषणेन चैतस्य सखित्वमभवत्सह} %॥५॥

\twolineshloka
{विष्णुभक्तः सत्यसन्धो बभूव नपतिः सदा}
{तस्यैव शासतो राजन राज्यं निहतकण्टकम्} %॥६॥

\twolineshloka
{बभूव दुहिता गेहे चन्द्रभागा सरिद्वरा}
{शोभनाय च सा दत्ता चन्द्रसेनसुताय वै} %॥७॥

\twolineshloka
{स कदाचित्समायानः श्वशुरस्य गृहे नृप}
{एकादशीव्रतमिदं समायातं सुपुण्यदम्} %॥८॥

\twolineshloka
{समागते व्रतदिने चन्द्रभागा त्वचिन्तयत्}
{किं भविष्यति देवेश मम भर्तातिदुर्बलः} %॥९॥

\twolineshloka
{क्षुधां सोढुं न शक्नोति पिता चैवोप्रशासनः}
{पटहस्तायने यस्य सम्प्राप्ते दशमीदिने} %॥१०॥

\twolineshloka
{न भोक्तव्यं न भोक्तव्यं न भोक्तव्यं हरेर्दिने}
{श्रुत्वा पटहनियों शोभनस्त्वब्रवीत्प्रियाम्} %॥११॥

\twolineshloka
{किं कर्तव्यं मया कान्ते ब्रूझुपायं सुशोभने}
{कृतेन येन में सम्यग्जीवितं न विनश्यति} %॥१२॥

\uvacha{चन्द्रभागोवाच}

\twolineshloka
{मत्पितुर्वेश्मनि विभो भोक्तव्यं नापि केनचित्}
{गजैरश्वैस्तथा चोष्ट्रैरन्यैः पशुभिरेत च} %॥१३॥

\twolineshloka
{तृणमन्नं तथा वारि न भोक्तव्य हरेर्दिने}
{मानवैश्च कुतः कान्त भुज्यते हरिवासरे} %॥१४॥

\twolineshloka
{यदि त्वं भोक्ष्यसे कान्त ततो गेहात्प्रयास्यताम्}
{एवं विचार्य मनसा सुदृढं मानसं कुरु} %॥१५॥

\uvacha{शोभन उवाच}

\twolineshloka
{सत्यमेतत्त्वया चोक्तं करिष्येऽहमुपोषणम्}
{देवेन विहितं यद्वै तत्तथैव भविष्यति} %॥१६॥

\twolineshloka
{इति दिष्टे मति कृत्वा चकार व्रतमुत्तमम्}
{क्षुत्तृषापीडितततुः स बभूवातिदुःखितः} %॥१७॥

\twolineshloka
{एवं व्याकुलित तस्मिन्त्रादित्योऽस्तमगा द्विरिम्}
{वैष्णवानां नराणां सा निशा हर्षविवर्धिनी} %॥१८॥

\twolineshloka
{हरिपूजारतानां च जागरासक्तचेतसाम्}
{बभूव नृपशार्दूल शोभनस्यातिदुःसहा} %॥१९॥

\twolineshloka
{रवेरुदयवेलायां शोभनः पञ्चतां गतः}
{दाहयामास राजा तं राजयोग्यैश्च दारुभिः} %॥२०॥

\twolineshloka
{चन्द्रभागा नात्मदेहं ददाह पिटवारिता}
{कृत्वौर्ध्वदेहिकं तस्य तस्थौ जनकवेश्मनि} %॥२१॥

\twolineshloka
{शोभनेन नृपश्रेष्ठ रमाव्रतप्रभावतः}
{प्राप्तं देवपुरं रम्यं मन्दराचलसानुनि} %॥२२॥

\twolineshloka
{अत् त्तममनाधृष्यमसङ्ख्येयगुणान्वितम्}
{हेमस्तम्भमयैः सौधे रत्नवैदूर्यमण्डितैः} %॥२३॥

\twolineshloka
{स्फाटिकार्वविधाकारोर्वचित्रैरुपशोभितम्}
{सिंहासनसमारूढः सुश्वेतच्छत्रचामरः} %॥२४॥

\twolineshloka
{किरीटकुण्डलयुतो हारकेयूरभूषितः}
{स्तूयमानश्च गन्धर्वैरप्सरोगणसवितः} %॥२५॥

\twolineshloka
{शोभः शोभते तत्र देवराडपरो यथा}
{सोमशमॆति विख्यातो मुचुकुन्दपुरे वसन्} %॥२६॥

\twolineshloka
{तीर्थयात्राप्रसङ्गेन भ्रमन् विप्रो ददर्श तम्}
{नृपजामातरं ज्ञात्वा तत्समीपं जगाम सः} %॥२७॥


\threelineshloka
{आसनादुत्थितः शीघ्रं नम चक्रे द्विजोत्तमम्}
{चकार कुशलप श्रेश्वशुरस्य नपश्य च}
{कान्तायाश्चन्द्रभागायास्तथैव नगरस्य च} %॥२८॥

\uvacha{सोमशर्मोवाच}

\twolineshloka
{कुशलं वर्तते राजञ्छशुरस्य गृहे तव}
{चन्द्रभागा कुशलिनी सर्वतः कुशलं पुरे} %॥२९॥

\twolineshloka
{स्ववृतं कथ्यतां राजन्नाश्चर्य परमं मम}
{पुरं विचित्रं रुचिरं न दृष्ट केनचित्क्वचित्} %॥३०॥

\onelineshloka*
{एतदाचक्ष्व नृपते कुतः प्राप्तमिदं त्वया}

\uvacha{शोभन उवाच}

\onelineshloka
{कार्तिकस्यासिते पक्षे नाना चैकादशी रमा} %॥३१॥

\twolineshloka
{तामुपोष्प मया प्राप्तं द्विजेन्द्रपुरमध्रुवम्}
{ध्रुवं भवति येनैव तत्कुरुष्व द्विजोत्तम} %॥३२॥

\uvacha{द्विजेन्द्र उवाच}

\twolineshloka
{कथमध्रुवमेतद्धि कथं हि भवति ध्रुवम्}
{तत्त्वं कथय राजेन्द्र तत्करिष्यामि नान्यथा} %॥३३॥

\uvacha{शोभन उवाच}

\twolineshloka
{मयतद्विहितं विप्र श्रद्धाहीनं व्रतोत्तमम्}
{तेनेदमधुवम् मन्ये ध्रुवं भवति तच्छृणु} %॥३४॥

\twolineshloka
{मुचुकुन्दस्य दुहिता चन्द्रभागा सुशोभना}
{तस्यै कथय वृत्तान्तं ध्रुवमेतद्भविष्यति} %॥३५॥

\twolineshloka
{तच्छ्रुत्वाऽथ द्विजवरस्तस्यै सर्व न्यवेदयत्}
{श्रुत्वाऽथ सा द्विजवचो विस्मयोत्फुल्ललोचना} %॥३६॥

\onelineshloka*
{प्रत्यक्षमथवा स्वप्नस्त्वयैतत्कथ्यते द्विज}

\uvacha{सोमशर्मोवाच}

\onelineshloka
{प्रत्यक्षं पुत्रि ते कान्तो मया दृष्टो महावने} %॥३७॥

\twolineshloka
{देवतुल्यमनाधृष्यं दृष्टं तस्य पुरं मया}
{अध्रुवं तेन तत्प्रोक्तं ध्रुवं भवति तत्कुरु} %॥३८॥

\uvacha{चन्द्रभागोवाच}

\twolineshloka
{तत्र मां नय विप्रर्षे पतिदर्शनलालसाम्}
{आत्मनो व्रतपुण्येन करिप्यामि पुरं ध्रुवम्} %॥३९॥

\twolineshloka
{आवयोर्द्विज संयोगो यथा भवति तत्कुरु}
{प्राप्यते हि महत्पुण्यं कृतं योगे विमुक्तयोः} %॥४०॥

\twolineshloka
{इति श्रुत्वा सह तया सोमशर्मा जगाम ह}
{आश्रमं वामदेवस्य मन्दराचलसन्निधौ} %॥४१॥

\twolineshloka
{वामदेवोऽशृणोत्सर्व वृत्तान्तं कथितं तयोः}
{अभ्यषिञ्चञ्चन्द्रभागा वेदमन्त्ररैथोज्ज्वलाम्} %॥४२॥

\twolineshloka
{ऋषिमन्त्रप्रभावेण विष्णुवासरसेवनात्}
{दिव्यदेहा बभूवासो दिव्यां गतिमवाप ह} %॥४३॥

\twolineshloka
{पत्युः समीपमगमत्प्रहर्षोत्फुल्ललोचना}
{सहर्षः शोभनोऽतीव दृष्ट्वा कान्तां समागताम्} %॥४४॥

\twolineshloka
{समाहूय स्वके वामे पार्श्वे तो सन्न्यवेशयत्}
{सा चोवाच मिय हर्षाञ्चन्द्रभागा प्रियं वचः} %॥४५॥

\twolineshloka
{शृणु कान्त हितं वाक्यं यत्पुण्यं विद्यते मयि}
{अष्टधिका जाता यदाहं पितृवेश्मनि} %॥४६॥

\twolineshloka
{मया ततःप्रभृति च कृतमेकादशीव्रतम्}
{यथोक्तविधिसंयुक्तं श्रद्धायुक्तेन चेतसा} %॥४७॥

\twolineshloka
{तेन पुण्यप्रभावेण भविष्यति पुरं ध्रुवम्}
{सर्वकामसमृद्धं च यावदाभूतसम्प्लवम्} %॥४८॥

\twolineshloka
{एवं सा नृपशार्दूल रमते पतिना सह}
{दिव्यभोगा दिव्यरूपा दिव्याभरणभूषिता} %॥४९॥

\twolineshloka
{शोभनोऽपि तया सार्ध रमते दिव्यविग्रहः}
{रमावतप्रभावेण मन्दराचलसानुनि} %॥५०॥

\twolineshloka
{चिन्तामणिसमा ह्येषा कामधेनुसमाथवा}
{रमाभिधाना नृपते तवाने कथिता मया} %॥५१॥

\twolineshloka
{ईदृशं च व्रतं राजन् ये कुर्वन्ति नरोत्तमाः}
{ब्रह्महत्यादिपापानि नाशं यान्ति न संशयः} %॥५२॥

\twolineshloka
{एकादश्या रमाख्याया माहात्म्यं शृणुयानरः}
{सर्वपापविनिर्मुक्तो विष्णुलोके महीयते} %॥५३॥

॥इति श्रीब्रह्माण्डपुराणे कार्तिककृष्णाया रमाख्याया माहात्म्यम्॥


\hyperref[sec:ekadashi_mahatmyam_vrata_raja]{\closesub}
\clearpage

\sect{कार्त्तिक-शुक्ल-प्रबोधिनी-एकादशी-माहात्म्यम्}
\label{sec:vrata-raja-karttika-shukla-prabodhini}

\uvacha{ब्रह्मोवाच}

\twolineshloka
{प्रबोधिन्याश्च माहात्म्यं पापघ्नं पुण्यवर्धनम्}
{मुक्तिप्रदं सुबुद्धीनां शृणुष्व मुनिसत्तम} %॥१॥

\twolineshloka
{तावद्गर्जति विप्रेन्द्र गङ्गा भागीरथी क्षितौ}
{यावन्नायाति पापघ्नी कार्तिके हरिबोधिनी} %॥२॥

\twolineshloka
{तावद्गर्जन्ति तीर्थानि ह्यासमुद्रं सरांसि च}
{यावत्प्रबोधिनी विष्णोस्तिथि याति कार्तिकी} %॥३॥

\twolineshloka
{अश्वमेधसहस्राणि राजसूयशतानि च}
{एकेनेवोपवासेन प्रबोधिन्या लभेन्नरः} %॥४॥

\uvacha{नारद उवाच}

\twolineshloka
{एकभक्ते च किं पुण्यं किं पुण्यं नक्तभोजने}
{उपवासे च किं पुण्यं तन्मे ब्रूहि पितामह} %॥५॥

\uvacha{ब्रह्मोवाच}

\twolineshloka
{एकभक्तेन जन्मोत्थं नक्तेन द्विजनुर्भवम्}
{सप्तजन्मभवं पापमुपवासेन नश्यति} %॥६॥

\twolineshloka
{यदुर्लभं यदप्राप्यं त्रैलोक्ये न तु गोचरम्}
{तदप्यप्रार्थितं पुत्रं ददाति हरिबोधिनी} %॥७॥

\twolineshloka
{मेरुमन्दरमात्राणि पापान्युप्राणि यानि तु}
{एकेनैवोपवासेन दहते पापहारिणी} %॥८॥

\twolineshloka
{पूर्वजन्मसहस्रेस्तु य कर्म ह्युपार्जितम्}
{जागरस्तत्प्रबोधिन्यां दहते चूलराशिवत्} %॥९॥

\twolineshloka
{उपवास प्रबोधिन्यां यः करोति स्वभावतः}
{विधिवन्मुनिशार्दूल यथोक्तं लमते फलम्} %॥१०॥

\twolineshloka
{यथोक्तं सुकृतं यस्तु विधिवत्कुरुते नरः}
{स्वल्पं मुनिवर श्रेष्ठ मेरुतुल्यं भवेञ्च तत्} %॥११॥

\twolineshloka
{विधिहीनं तु यः कुर्यात्सुकृतं मेरुमात्रकम्}
{अणुमात्रं न चाप्नोति फलं धर्मस्म नारद} %॥१२॥

\twolineshloka
{ये ध्यायन्ति मनोवृत्त्या करिष्यामः प्रबोधिनीम्}
{तेषां विलीयते पापं पूर्वजन्मशतोद्भवम्} %॥१३॥

\twolineshloka
{समतीतं भविष्यं च वर्तमानं कुलायुतम्}
{विष्णुलोकं नयत्याश प्रबोधिन्यां तु जागरात} %॥१४॥

\twolineshloka
{वसन्ति पितरो हृष्टा विष्णुलोकेत्यलङ्कृताः}
{विमुक्ता नारकेदुखेः पूर्वकर्मसमुद्भवैः} %॥१५॥

\twolineshloka
{कृत्वा तु पातकं घोरं ब्रह्महत्यादिकं नरः}
{कृत्वा तु जागरं विष्णोधौतपापो भवेन्मुने} %॥१६॥

\twolineshloka
{दुष्प्राप्यं यत्फलं विप्रैरश्वमेधादिभिर्मखैः}
{प्राप्यते तत्सुखेनैव प्रबोधिन्यां तु जागरात्} %॥१७॥

\twolineshloka
{आप्लुत्य सर्वतीर्थेषु दत्त्वा गाः काञ्चनं महीम्}
{न तत्फलमवाप्नोति यत्कृत्वा जागरं हेरेः} %॥१८॥

\twolineshloka
{जातः स एवं सुकृती कुलं तेनैव पावितम्}
{कार्तिके मुनिशार्दूल कृता येन प्रबोधिनी} %॥१९॥

\twolineshloka
{यानि कानि च तीर्थानि त्रैलोक्ये सम्भवन्ति च}
{तानि तस्य गृहे सम्यग्यः करोति प्रबोधिनीम्} %॥२॥

\twolineshloka
{सर्वकृत्यं परित्यज्य तुष्टयर्थ चक्रपाणिनः}
{उपोष्यकादशी रम्यां कार्तिके हरिबोधिनीम्} %॥२१॥

\twolineshloka
{स ज्ञानी स च योगी च स तपस्वी जितेन्द्रियः}
{विष्णुप्रियतरा ह्येषा धर्मसारस्य दायिनी} %॥२२॥

\twolineshloka
{सकृदेनामुपोष्यैव मुक्तिभाक् च भवेन्नरः}
{प्रबोधिनीमुपोषित्वा न गर्भ विशते नरः} %॥२३॥

\twolineshloka
{कर्मणा मनसा वाचा पाएं यत्समुपार्जितम्}
{तत्क्षालयति गोविन्दः प्रबोधिन्यां तु जागरात्} %॥२४॥

\twolineshloka
{स्नानं दानं जपो होमः समुद्दिश्य जनार्दनम्}
{नरैर्यत् क्रियते वत्स प्रबोधिन्यां तदक्षयम्} %॥२५॥

\twolineshloka
{बनानेन देवेशं परितोष्य जनार्दनम्}
{विराजयन्दिशः सर्वाः प्रयाति भवनं हरेः} %॥२६॥

\twolineshloka
{बाल्ये यच्चाजितं वत्स यौवने वार्धके तथा}
{शतजन्मकृतं पापं स्वल्पं वा यदि वा बहु} %॥२७॥


\threelineshloka
{तत्क्षालयति गोविन्दो ह्यस्यामभ्यर्चितो मुने}
{चन्द्रसूर्योपरागे च यत्फलं परिकीर्तितम्}
{तत्सहस्रगुणं प्रोक्तं प्रबोधिन्यां तु जागरात} %॥२८॥

\twolineshloka
{जन्मप्रभृति यत्पुण्यं नरेणासादितं भवेत्}
{वृथा भवति तत्सर्वमकृते कार्तिकवते} %॥२९॥

\twolineshloka
{अकृत्वा नियमं विष्णोः कार्तिकं यः क्षिपेन्नरः}
{जन्मार्जितस्य पुण्यस्य फलं नाप्नोति नारद} %॥३०॥

\twolineshloka
{तस्मात्त्वया प्रयत्नेन देवदेवो जनार्दनः}
{उपासनीयो विप्रेन्द्र सर्वकामफलप्रदः} %॥३१॥

\twolineshloka
{परान्नं वर्जयेद्यस्तु कार्तिके विष्णुतत्परः}
{अवश्यं स नरो वत्स चान्द्रायणफलं लभेत्} %॥३२॥

\twolineshloka
{न तथा तुष्यते यज्ञैर्न दानैर्मुनिसत्तम}
{यथा शास्त्रकथालापैः कार्तिके मधुसूदनः} %॥३३॥

\twolineshloka
{ये कुर्वन्ति कथां विष्णोर्ये शृण्वन्ति समाहिताः}
{श्लोकाई श्लोकमेकं वा कार्तिके गोशतं फलम्} %॥३४॥

\twolineshloka
{श्रेयसे लोभबुद्धया वा यः करोति हरेः कथाम्}
{कार्तिके मुनिशार्दूल कुलानां तारयेच्छतम्} %॥३५॥

\twolineshloka
{नियमेन नरो यस्तु शृणुते वैष्णवीं कथाम्}
{कार्तिके तु विशेषेण गोसहस्रफलं लभेत्} %॥३६॥

\twolineshloka
{प्रबोधवासरे विष्णोः कुरुते यो हरेः कथाम्}
{सप्तद्वीपवतीदान फलं स लभते मुने} %॥३७॥

\twolineshloka
{कृत्वा विष्णुकथां दिव्यां येऽर्चयन्ति कथाविदम्}
{स्वशक्त्या सुनिशार्दूल तेषां लोकाः सनातनाः} %॥३८॥

\onelineshloka*
{ब्रह्मणो वचनं श्रुत्वा नारदः पुनरबवीत्}

\uvacha{नारद उवाच}

\onelineshloka
{विधानं ब्रूहि मे स्वामिन्नेकादश्याः सुरोत्तम} %॥३९॥

\twolineshloka
{चीर्णन येन भगवन्यादृशं फलमाप्नुयात्}
{नारदस्य वचः श्रुत्वा ब्रह्मा वचनमब्रवीत्} %॥४०॥

\uvacha{ब्रह्मोवाच}

\twolineshloka
{ब्राह्म मुहूर्ते चोत्थाय ह्येकादश्यां द्विजोत्तम}
{स्मानं चैव प्रकर्तव्यं दन्तधावनपूर्वकम्} %॥४१॥

\twolineshloka
{नद्या तडागे कूपे वा वाप्यां गेहे तथैव च}
{नियमार्थे महाभाग इमं मन्त्रमुदीरयेत्} %॥४२॥

\twolineshloka
{एकादश्यां निराहारः स्थित्वाऽहनि परे ह्यहम्}
{भोक्ष्यामि पुण्डरीकाक्ष शरणं मे भवाच्युत} %॥४३॥

\twolineshloka
{गृहीत्वानेन नियमं देवदेवं च चक्रिणम्}
{सम्पूज्य भक्त्या तुष्टात्मा झुपवासं समाचरेत्} %॥४४॥

\twolineshloka
{रात्री जागरणं कुर्यादेवदेवस्य सविधौ}
{गीतं नृत्यं च वाद्यं च तथा कृष्णकां मुने} %॥४५॥

\twolineshloka
{बहुपुष्पैर्बहुफलैः कर्पूरागुरुकुङ्कुमैः}
{हरेः पूजा विधातव्या कार्तिक्या बोधवासरे} %॥४६॥

\twolineshloka
{वित्तशाष्ठचं न कर्तव्यं सम्प्राप्त हरिवासरे}
{फलैर्नानाविर्दिव्यैः प्रबोधिन्यां तु भक्तितः} %॥४७॥

\twolineshloka
{शङ्खतोयं समादाय ह्यों देयो जनार्दने}
{यत्फलं सर्वतीर्थेषु सर्वदाने यत्कलम्} %॥४८॥

\twolineshloka
{तत्फलं कोटिगुणितं दत्तेऽधे बोधवासरे}
{अगस्त्य कुसुमैर्देवं पूजयेद्यो जनार्दनम्} %॥४९॥

\twolineshloka
{देवेन्द्रोऽपि तदने च करोति करसम्पुटम्}
{न तत्करोति विप्रेन्द्र तपसा तोषितो हरिः} %॥५०॥

\twolineshloka
{यत् करोति हृषीकेशो मुनिपुष्परलकृतः}
{बिल्वपत्रैश्च ये कृष्णं कार्तिके कलिवर्धन} %॥५१॥

\twolineshloka
{पूजयन्ति महाभक्त्या मुक्तिस्तेषां मयोदिता}
{तुलसीदलपुष्पैर्ये पूजयन्ति जनार्दनम्} %॥५२॥

\twolineshloka
{कार्तिके स दहेत्तेषां पापं जन्मायुतोद्भवम्}
{दृष्टा स्पृष्टाथवा ध्याता कीर्तिता नमिता स्तुता} %॥५३॥

\twolineshloka
{रोपिता सचिता नित्यं पूजिता तुलसी शुभा}
{नवधा सेविता भक्त्या कार्तिके यर्दिनदिने} %॥५४॥

\twolineshloka
{युगकोटिसहस्राणि ते वसन्ति हरेहे}
{रोपिता तुलसी यैस्तु व ते वसुधातले} %॥५५॥

\twolineshloka
{कुले तेषां तु ये जाता ये भविष्यन्ति ये गताः}
{आकल्पयुगसाहस्रं तेषां वासो हरेहे} %॥५६॥

\twolineshloka
{कदम्बकुसुमैर्देवं येऽर्चयन्ति जनार्दनम्}
{तेषां यमालयो नैव प्रसादाचकपाणिनः} %॥५७॥

\twolineshloka
{दृष्ट्वा कदम्बकुसुमं प्रीतो भवति केशवः}
{किं पुनः पूजितो विप्र सर्वकामप्रदो हरिः} %॥५८॥

\twolineshloka
{यःपुनः पाटलापुष्पैः कार्तिके गरुडध्वजम्}
{अर्चयेत्परया भक्त्या मुक्तिभागी भवेद्धि सः} %॥५९॥

\twolineshloka
{बकुलाशोककुसुक्र्येऽर्चयन्ति जगत्पतिम्}
{विशोकास्ते भविष्यन्ति यावचन्द्रदिवाकरौ} %॥६०॥

\twolineshloka
{येऽर्चयन्ति जगन्नाथं करवीरैः सितासितैः}
{तेषां सदा तु विप्रेन्द्र प्रीतो भवति केशवः} %॥६॥

\twolineshloka
{मञ्जरी सहकारस्य केशवोपरि ये नराः}
{यच्छन्ति ते महाभागा गोकोटिफलभागिनः} %॥६॥

\twolineshloka
{दूर्वाकुरेहरेर्यस्तु पूजाकाले प्रयच्छति}
{पूजाफलं शतगुणं सम्यगामोति मानवः} %॥६३॥

\twolineshloka
{शमीपत्रैस्तु ये देवं पूजयन्ति सुखप्रदम्}
{यममार्गो महाघोरो निस्तीर्णस्तैस्तु नारद} %॥६४॥

\twolineshloka
{वर्षाकाले तु देवेश कुसुमैश्चम्पकोद्भवैः}
{येऽर्चयन्ति न ते माः संसरेयुः पुनर्भवे} %॥६५॥

\twolineshloka
{सुवर्णकेतकीपुष्पं यो ददाति जनार्दने}
{कोटिजन्मार्जितं पापं दहते गरुडध्वजः} %॥६६॥

\twolineshloka
{कुङ्कुमारुणवर्णा च गन्धाच्या शतुपत्रिकाम्}
{यो ददाति जगनाथे श्वेतदीपालये वसेत्} %॥६७॥

\twolineshloka
{एवं सम्पूज्य रात्रौ च केशवं भुक्तिमुक्तिदम्}
{प्रातरुत्थाय च ब्रह्मन् गत्वा तु सजला नदीम्} %॥६८॥

\twolineshloka
{तत्र स्नात्वा जपित्वा च कृत्वा पौर्वाहिकी क्रियाः}
{गृहं गत्वा च सम्पूज्यः केशवो विधिवन्नरैः} %॥६९॥

\twolineshloka
{व्रतस्य पूरणार्थाय ब्राह्मणान्भोजयेत्सुधीः}
{क्षमापयेत्सुवचसा भक्तियुक्तेन चेतसा} %॥७०॥

\twolineshloka
{गुरुपूजा ततः कार्या भोजनाच्छादनाभिः}
{दक्षिणा गौश्च दातव्या तुष्टयर्थ चक्रपाणिनः} %॥७१॥

\twolineshloka
{भूयसी चैव दातव्या ब्राह्मणेभ्यः प्रयत्नतः}
{नियमश्चैव सन्त्याज्यो ब्राह्मणाने प्रयत्नतः} %॥७२॥

\twolineshloka
{कथयित्वा द्विजेभ्यस्तु दद्याच्छक्त्या च दक्षिणाम्}
{नक्तभोजी नरो राजन् ब्राह्मणान् भोजयेच्छुभान्} %॥७३॥

\twolineshloka
{अयाचिते बलीव सहिरण्यं प्रदापयेत्}
{अमांसाशी नरो यस्तु प्रददेद्रां सदक्षिणाम्} %॥७४॥

\twolineshloka
{धात्रीस्नायी मरो दद्यादधि माक्षिकमेव च}
{फिलानां नियमे राजन् फलदान समाचरेत्} %॥७५॥

\twolineshloka
{तेलस्थाने घृतं देयं घृतस्थाने पयः स्मृतम्}
{धान्यानां नियमे राजन् दीयन्ते शालिगण्डुलाः} %॥७६॥

\twolineshloka
{दद्याद्र्शयने शय्यां सतूला सपरिच्छदाम्}
{पत्रभोजी नरो दद्याद्भाजनं वृतसंयुतम्} %॥७७॥

\twolineshloka
{मौने घण्टा तिलांश्चैव सहिरण्यं प्रदापयेत्}
{धारणे तु स्वकेशानामादर्श दारयेद्बुधः} %॥७८॥

\twolineshloka
{उपानहौ प्रदातव्यावुपामत्परिवर्जनात्}
{लवणस्य च सन्त्यागे शर्करां च प्रदापयेत्} %॥७९॥

\twolineshloka
{नित्यं दीपप्रदो यस्तु विष्णोर्वा विबुधालये}
{सदीपं सवृतं तानं काञ्चनं वा दशायुतम्} %॥८०॥

\twolineshloka
{प्रदद्याद्विष्णुभक्ताय व्रतसम्पूर्तिहेतवे}
{एकान्तरोपवासे तु कुम्भानष्टौ प्रदापयेत्} %॥८१॥

\twolineshloka
{सवनाकाञ्चनोपेतान् सर्वान् सालङ्कृताञ्छुभान्}
{यथोक्तकरणे शक्तिर्यदि न स्यात्तदा मुने} %॥८२॥

\twolineshloka
{द्विजवाक्यं स्मृतं राजन् सम्पूर्णवतसिद्धिदम्}
{नत्वा विसर्जयेद्विप्रास्ततो भुनीत च स्वयम्} %॥८३॥

\twolineshloka
{यत्त्यक्तं चतुरो मासान् समाप्तिं तस्य चाचरेत्}
{एवं य आचरेत्पार्थ सोनन्तफलमाप्नुयात्} %॥८४॥

\twolineshloka
{अवसाने तु राजेन्द्र वासुदेवपुरं व्रजेत्}
{यश्चाविन्नं समाप्यैवं चातुर्मास्यव्रतं नृप} %॥८५॥

\twolineshloka
{स भवेत्कृतकृत्यस्तु न पुनर्मानुषो भवेत्}
{एतत्कृत्वा महीपाल परिपूर्ण व्रतं भवेत्} %॥८६॥

\threelineshloka
{व्रतवैकल्यमासाद्य ह्यन्धः कुष्ठी प्रजायते}
{एतत्ते सर्वमाख्यातं यत्पृष्टोऽहमिह त्वया}
{पठनाच्छ्रवणाद्वाऽपि लभेगोदान फलम्} %॥८७॥

॥इति श्रीस्कान्दे महापुराणे कार्त्तिकशुक्लप्रबोधिन्येकादाशीमाहात्म्यं सम्पूर्णम्॥


\hyperref[sec:ekadashi_mahatmyam_vrata_raja]{\closesub}
\clearpage

\sect{पुरुषोत्तम-मासस्य शुक्ल-कामदा-एकादशी-माहात्म्यम्}
\label{sec:vrata-raja-purushottama-shukla-kamada}

\uvacha{युधिष्ठिर उवाच}

\twolineshloka
{मलिन्लुचस्य मासस्य का वा एकादशी भवेत्}
{किं नाम को विधिस्तस्याः कथयस्व जनार्दन} %॥१॥

\uvacha{श्रीकृष्ण उवाच}

\twolineshloka
{मलमासस्य या पुण्या पोका नाम्ना च पशिनी}
{सोपोषिता प्रयत्नेन पद्मनाभपुरं नयेत्} %॥२॥

\twolineshloka
{मलमासे महापुण्या कीर्तिता कल्मषापहा}
{तस्याः फलं कथयितुं न शक्तश्चतुराननः} %॥३॥

\twolineshloka
{नारदाय पुरा प्रोक्तं विधिना व्रतमुत्तमम्}
{पद्मिन्याः पापराशिनं भुक्तिमुक्तिफलप्रदम्} %॥४॥

\twolineshloka
{श्रुत्वा वाक्यं मुरारेस्तु प्रोवाचातिमुदान्वितः}
{युधिष्ठिरो जगन्नाथं विधि पप्रच्छ धर्मवित्} %॥५॥

\twolineshloka
{श्रुत्वा राज्ञस्तु क्चनमुवाच मधुसूदनः}
{शृणु राजन्प्रवक्ष्यामि मुनीनामप्यगोचरम्} %॥६॥

\twolineshloka
{दशमीदिवसे प्राप्ते व्रतारम्भो विधीयते}
{कांस्यं मांसं मसूरांश्च चणकान्कोद्रवांस्तथा} %॥७॥

\twolineshloka
{शाकं मधु परान्नं च दशम्यामष्ट वर्जयेत्}
{हविष्यान्नं च भुञ्जीत अक्षारलवणं तथा} %॥८॥

\twolineshloka
{भूमिशायी ब्रह्मचारी भवेच दशमीदिने}
{एकादशीदिने प्राप्ते प्रातरुत्थाय सादरम्} %॥९॥

\twolineshloka
{विधाय च मलोत्सर्ग न कुर्यादन्तधावनम्}
{कृत्वा द्वादशगण्डूषाञ्छुचिर्भूत्वा समाहितः} %॥१०॥

\twolineshloka
{सूर्योदये शुभे तीर्थे स्नानार्थ प्रवजेत्तुधीः}
{गोमयं मृत्तिकां गृह्य तिलान्दर्भाञ्छुचिस्तथा} %॥११॥

\twolineshloka
{चूर्णैरामलकीभूतेविधिना स्नानमाचरेत्}
{उद्धृतासि वराहेण कृष्णेन शतबाहुना} %॥१२॥

\twolineshloka
{मृत्तिके ब्रह्मदत्तासि काश्यपेनाभिमन्त्रिता}
{हरिपूजनयोग्यं मां मृत्तिके कुरु ते नमः} %॥१३॥

\twolineshloka
{सर्वोषधिसमुत्पन्नं गवोदरमधिष्ठितम्}
{पवित्रकरणं भूमी पावयतु गोमयम्} %॥१४॥

\twolineshloka
{ब्रह्मष्ठीवनसम्भूता धात्री भुवनपावनी}
{संस्पृष्टा पावयाङ्ग मे निर्मलं कुरु ते नमः} %॥१५॥

\twolineshloka
{देव देव जगन्नाथ शङ्खचक्रगदाधर}
{देहि विष्णो ममानुज्ञा तव तीर्थावगाहने} %॥१६॥

\twolineshloka
{वारुणांश्च जपेन्मन्त्रान् स्नान कुर्याः द्विधानतः}
{गङ्गादितीर्थ संस्मृत्य यत्र कुत्र जलाशय} %॥१७॥

\twolineshloka
{पश्चात्सम्मार्जयेगात्रं विधिना नृपसत्तम}
{परिधायाहतं वासः शुक्लं शुचि ह्यखण्डितम्} %॥१८॥

\twolineshloka
{सन्ध्यामुपास्य विधिना तर्पयित्वा पितॄन्सुरान्}
{हरेर्मन्दिरमागम्य पूजयेत्कमलापतिम्} %॥१९॥

\twolineshloka
{स्वर्गमाषकृतं देवं राधिकासहितं हरिम्}
{पार्वत्या सहितं शम्भु पूजयेद्विधिपूर्वकम्} %॥२०॥

\twolineshloka
{धान्योपरि न्यसेत्कुम्भ तानं मुन्मयमेव वा}
{दिव्यवस्त्रसमायुक्तं दिव्यगन्धानुवासितम्} %॥२१॥

\twolineshloka
{तस्योपरि न्यसेत् पात्र तानं रौप्यं हिरण्मयम्}
{तस्मिन्संस्थापयेदेवं विधिना पूजयेत्ततः} %॥२२॥

\twolineshloka
{सन्त्राप्य सलिलै श्रेष्ठेर्गन्धधूपाधिवासितैः}
{चन्दनागुरुकरैः पूजयेदेवमीश्वरम्} %॥२३॥

\twolineshloka
{नानाकुसुमकस्तूरीकुङ्कुमन सिताम्बुजैः}
{तत्कालजातैः कुसुमैः पूजयेत्परमेश्वरम्} %॥२४॥

\twolineshloka
{नैवेद्यैर्विविधैः शक्त्या तथा नीराजनादिभिः}
{धूपर्दी पैः सकपूरैः पूजयेत्केशवं शिवम्} %॥२५॥

\twolineshloka
{नृत्यं गीतं तदने तु कुर्याद्भक्तिपुरःसरम्}
{नालपेत्पतितान्पापांस्तस्मिन्नहनि न स्पृशेत्} %॥२६॥

\twolineshloka
{नानृतं हि वदेद्वाक्यं सत्यपूतं वचो वदेत्}
{रजस्वला न स्पृशेच न निन्देवाह्मणं गुरुम्} %॥२७॥

\twolineshloka
{पुराणं पुरतो विष्णोः शृणुयात्सह वैष्णवैः}
{निर्जला सा प्रकर्तव्या या च शुक्के मलिम्लुचे} %॥२८॥

\twolineshloka
{जलपानेन वा कुर्याद् दुग्धाहारण नान्यथा}
{रात्री जागरणं कुर्याद्गीतवादित्रसंयुतम्} %॥२९॥

\twolineshloka
{प्रहरे प्रहरे पूजा कार्या विष्णोः शिवस्य च}
{प्रथमे प्रहर इद्यानारिकेलार्घमुत्तमम्} %॥३०॥

\twolineshloka
{द्वितीये श्रीकलैश्चैव तृतीये बीजपूरकैः}
{चतुर्थप्रहरे पूगै रिङ्गैश्च विशेषतः} %॥३१॥

\twolineshloka
{प्रथम प्रहरे पुण्यमाग्निष्टोनस्य जायते}
{द्वितीये वाजपेयस्य तृतीये हयमेधजम्} %॥३२॥

\twolineshloka
{चतुथें राजसूयस्य जाप्रतो जायते फलम्}
{नातः परतरं पुण्यं नातः परतरा मखाः} %॥३३॥

\twolineshloka
{नातः परतरा विद्या नातः परतरं तपः}
{पृथिव्यां यानि तीर्थानि क्षेत्राण्यायतनानि च} %॥३४॥

\twolineshloka
{तेन स्नातानि दृष्टानि येनाकारि हरेवतम्}
{एवं जागरणं कुर्याद्यावत्सूर्योदयो भवेत्} %॥३५॥

\twolineshloka
{सूर्योदये शुभे तीर्थे गत्वा स्नानं समाचरेत्}
{स्नात्वा चामत्य भवनं पूजयेदेवमीश्वरम्} %॥३६॥

\twolineshloka
{पूर्वोदितन विधिना भोजयेद्ब्राह्मणाञ्छुभान्}
{कुम्भादिकं च यत्सर्व प्रतिमा केशवस्य च} %॥३७॥

\twolineshloka
{पूजयित्वा विधानेन ब्राह्मणाय समर्पयेत्}
{एवंविधं व्रतं यो वै कुरुते भुवि मानवः} %॥३८॥

\twolineshloka
{सफलं जायते जन्म तस्य मुक्तिफलप्रदम्}
{एतत्ते सर्वमाख्यातं यत्पृष्टोऽहं त्वयानघ} %॥३९॥

\twolineshloka
{व्रतानि तेन चीर्णानि सर्वाणि नृपनन्दन}
{पद्मिन्याः प्रीतियुक्तो यः कुरुते व्रतमुत्तमम्} %॥४०॥

\twolineshloka
{अब ते कथयिष्यामि कथामेकां मनोरमाम्}
{नारदाय पुलस्त्येन विस्तरैण निवोदिताम्} %॥४१॥

\twolineshloka
{कार्तवीर्येण कारायां निक्षिप्तं वीक्ष्य रावणम्}
{विमोचितः पुलस्त्येन याचयित्वा महीपतिम्} %॥४२॥

\twolineshloka
{तदाश्चर्य तदा श्रुत्वा नारदो दिव्यदर्शनः}
{पप्रच्छ च यथाभक्त्या पुलस्त्यं मुनिपुङ्गवम} %॥४३॥

\uvacha{नारद उवाच}

\twolineshloka
{दशाननेन विजिताः सर्वे देवाः सवासवाः}
{कार्तवीर्येण विजिताः कथं रणविशारदः} %॥४४॥

\onelineshloka*
{नारदस्य वचः श्रुत्वा पुलस्त्यो मुनिरब्रवीत्}

\uvacha{पुलस्त्य उवाच}

\onelineshloka
{शृणु वत्स प्रवक्ष्यामि कार्तवीर्यसमुद्भवम्} %॥४५॥

\twolineshloka
{पुरा त्रेतायुगे राजन्माहिष्मत्यां बृहत्तरः}
{हैहयानां कुले जातः कृतवीर्यों महीपतिः} %॥४६॥

\twolineshloka
{सहस्रं प्रमदास्तस्य नृपस्य प्राणवल्लभाः}
{न तासां तनयं काचिल्लेभे राज्यधुरन्धरम्} %॥४७॥

\twolineshloka
{यजन् देवानिपतृन्सिद्धान्प्रातपूज्य महत्तरान्}
{कुर्वस्तदुदितं सर्व लब्धवांस्तनयं न सः} %॥४८॥

\twolineshloka
{सुतं विना तदा राज्यं न सुखाय महीपतेः}
{क्षुधितस्य यथा भोगा न भवन्ति सुखप्रदाः} %॥४९॥

\twolineshloka
{विचार्य चित्ते नृपतिस्तपस्तप्तुं मनो दधे}
{तपसैव सदा सिद्धिर्जायते मनसेप्सिता} %॥५०॥

\twolineshloka
{इत्युक्त्वा स हि धर्मात्मा चीरवासा जटाधरः}
{तपस्तप्तुं गतः सद्यो गृहे न्यस्य सुमन्त्रिणम्} %॥५१॥

\twolineshloka
{निर्गतं नृपतिं वीक्ष्य पद्मिनी प्रमदोत्तमा}
{हरिश्चन्द्रस्य तनया तपस्तप्तुं कृतोद्यमम्} %॥५२॥

\twolineshloka
{भूषणादि परित्यज्य चीरमेकं समाश्रयत्}
{जगाम पतिना सार्ध पर्वते गन्धमादने} %॥५३॥

\twolineshloka
{गत्वा तत्र तपस्तेपे वर्षाणामयुतं नृपः}
{न लेभेऽथापि तनयं ध्यायन्देवं गदाधरम्} %॥५४॥

\twolineshloka
{अस्थिस्नायुमयं कान्तं दृष्ट्वा सा प्रमदोत्तमा}
{अनसूयां महासाध्वीं पप्रच्छ विनयान्विता} %॥५५॥

\twolineshloka
{भर्तुः प्रतपतः साध्वि वर्षाणामयुतं गतम्}
{तथापि न प्रसन्नोऽभूत्केशवः कष्टनाशनः} %॥५६॥

\twolineshloka
{व्रतं मम महाभागे कथयस्व यथातथम्}
{येन प्रसन्नो भगवान्भविष्यति सदा मयि} %॥५७॥

\twolineshloka
{येन जायेत मे पुत्रश्चक्रवर्ती महत्तरः}
{श्रुत्वा तस्यास्तु वचनं पतिव्रतपरायणा} %॥५८॥

\twolineshloka
{तदा प्रोवाच संहृष्टा पद्मिनी पद्मलोचनाम् मासो}
{मलिम्लुचः सुभ्र मासद्वादशकाधिकः} %॥५९॥

\twolineshloka
{द्वात्रिंशद्भिर्गतै सरा याति स शुभानने}
{तन्मध्ये द्वादशीयुग्मं पद्मिनी परमा तथा} %॥६०॥

\twolineshloka
{उपोष्य तत्प्रकर्तव्यं विधिना जागरैः समम्}
{शीघ्र प्रसन्नो भगवान् भविष्यति सुतप्रदः} %॥६१॥

\twolineshloka
{इत्युक्त्वाकथयत् सर्व मया पूर्वोदितं नृप}
{विधिव्रतस्य विधिवत्प्रसन्ना कर्दमाङ्गजा} %॥६२॥

\twolineshloka
{श्रुत्वा व्रतविधिसर्व यथोक्तमनसूयया}
{चक्रे राज्ञी च तत्सर्व पुत्रप्राप्तिमभीप्सती} %॥६३॥

\twolineshloka
{एकादश्यां निराहारा सदा जाता च निर्जला}
{जागरेण युता रात्रौ गीतनृत्यसमन्विता} %॥६४॥

\twolineshloka
{पूर्णे व्रते च वै शीघ्र प्रसन्नः केशवः स्वयम्}
{बभाषे गरुडारूढो वरं वरय शोभने} %॥६५॥

\twolineshloka
{श्रुत्वा वाक्यं जगद्धातुः स्तुत्वा प्रीत्या शुचिस्मिता}
{ययाचे द्य वरं देहि मम भर्तुर्वृहत्तरम्} %॥६६॥

\twolineshloka
{पद्मिन्या स्तद्वचः श्रुत्वा प्रत्युवाच जनार्दनः}
{यथा मालम्लुचो मासो नान्यो मे प्रीतिदायकः} %॥६॥

\twolineshloka
{तन्मध्यैकादशी रम्या मम प्रीतिविवर्धनी}
{सा त्वयोपोषिता सुभ्र यथोक्तविधिना शुभे} %॥६८॥

\twolineshloka
{तेन त्वया प्रसन्नोऽहं कृतोऽस्मि सुभगानने}
{तव भर्तुः प्रदास्यामि वरं यन्मनसेप्सितम्} %॥६९॥

\twolineshloka
{इत्युक्त्वा नृपति प्राह विष्णुर्विश्वार्तिनाशनः}
{वरं वरय राजेन्द्र यत्ते मनसि काङ्क्षितम्} %॥७०॥

\twolineshloka
{सन्तोषितोऽहं प्रियया तव सिद्धिचिकीर्षया}
{श्रुत्वा तद्वचनं विष्णोः प्रसन्नो नृपसत्तमः} %॥७१॥

\twolineshloka
{पत्रे सुतं महाबाहुं सर्वलोकनमस्कृतम्}
{न देवैर्मानुषैर्ना गर्दैत्यदानवराक्षसैः} %॥७२॥

\twolineshloka
{जेतुं शक्यो जगन्नाथ विना त्वां मधुसूदन}
{इत्युक्तो भगवान् बाहमित्युक्त्वान्तरधीयत} %॥७३॥

\twolineshloka
{नृपोऽपि सुप्रसनात्मा हृष्टः पुष्टः प्रियायुतः}
{समायात् स्वपुरं रम्यं नरनारीमनोरमम्} %॥७४॥

\twolineshloka
{स पद्मिन्यां सुतं लेभे कार्तवीर्य महाबलम्}
{न तेन सदृशः कश्चित्रिषु लोकेषु मानवः} %॥७५॥

\twolineshloka
{तस्मात्पराजितःसङ्ख्ये रावणो दशकन्धरः}
{न तं जेतुं समर्थोऽस्ति त्रिषु लोकेषु कश्चन} %॥७६॥

\twolineshloka
{विना नारायणं देवं चक्रपाणि गदाधरम्}
{न त्वया विस्मयः कार्यो रावणस्य पराजये} %॥७७॥

\twolineshloka
{मलिम्लुचप्रसादेन पद्मिन्याश्चाप्युपोषणात्}
{दत्तो देवाधिदेवेन कार्तवीर्यो महाबलः} %॥७८॥

\onelineshloka*
{इत्युक्त्वा प्रययौ विप्रेः प्रसनेनान्तरात्मना}

\uvacha{श्रीकृष्ण उवाच}

\onelineshloka
{एतत्ते सर्वमाख्यातं यत्पृष्टोऽहं त्वयानघ} %॥७९॥

\twolineshloka
{मलिम्लुचस्य मासस्य शुक्लाया व्रतमुत्तमम्}
{ये करिष्यन्ति मनुजास्ते यास्यन्ति हरेः पदम्} %॥८०॥

\twolineshloka
{त्वमेवं कुरु राजेन्द्र यदि चेष्टमभीप्ससि}
{केशवस्य वचः श्रुत्वा धर्मराजोऽतिहर्षितः} %॥८१॥

\onelineshloka*
{चक्रे व्रतं विधानेन बन्धुमिः परिवारितः}

\uvacha{सूत उवाच}

\twolineshloka
{एतत्ते सर्वमाख्यातं यत्पृष्टोऽहं पुरा द्विज}
{पुण्यं पवित्रं परमं किं भूयः श्रोतुमिच्छसि} %॥८२॥

\twolineshloka
{एवंविधं येपि व्रतं मनुष्या भक्त्या करिष्यन्ति मलिम्लुचस्य}
{उपोष्य शुक्कामतिसौख्यदात्रीमे कादशी ते भुवि शन्यन्याः} %॥८३॥

\twolineshloka
{श्रोष्यन्ति ये तस्य विधि समयं तेऽप्यशभाजो मतुजा भवन्ति}
{ये वै पठिष्यन्ति कथां समयां ते वै गमिष्यन्ति हरेनिवासम्} %॥८४॥

॥इत्यधिकमासस्य शुक्लैकादशीमाहात्म्यं सम्पूर्णम्॥

\hyperref[sec:ekadashi_mahatmyam_vrata_raja]{\closesub}
\clearpage

\sect{पुरुषोत्तम-मासस्य कृष्ण-कमला-एकादशी-माहात्म्यम्}
\label{sec:vrata-raja-purushottama-krishna-kamala}

\uvacha{युधिष्ठिर उवाच}

\twolineshloka
{मलिम्लुचस्य मासस्य कृष्णा का कथ्यते विभो}
{किं नाम को विधिस्तम्याः कथयस्व जगत्पते} %॥१॥

\uvacha{श्रीकृष्ण उवाच}

\twolineshloka
{परमेति समाख्याता पवित्रा पापहारिका}
{भुक्तिमुक्ति दा नृणां स्त्रीणां चापि युधिष्ठिर} %॥२॥

\twolineshloka
{पूर्वोक्तविधिना कार्या कृष्णापि भुवि मानवैः}
{सम्पूज्य परया भक्त्या नाम्ना देवं नरोत्तम} %॥३॥

\twolineshloka
{अव ते कथयिष्यामि कथामेतां मनोरमाम्}
{काम्पिल्यनगरे जातां मुनीनामग्रतः श्रुताम्} %॥४॥

\twolineshloka
{आसीविजवरः कश्चित्तुमेधानाम धार्मिकः}
{तस्य पत्नी पवित्राख्या पातिव्रत्यपरायणा} %॥५॥

\twolineshloka
{कर्मणा केनचिद्विप्रो धनधान्यविवर्जितः}
{न क्वापि लभते भिक्षा याचनपि नरान्बहून्} %॥६॥

\twolineshloka
{न भोज्यं लभते तादृङ्न वस्त्रं नैव मण्डनम्}
{रूपयौवनमाधुर्या नारी शुश्रूषते पतिम्} %॥७॥

\twolineshloka
{अतिथि भोजयित्वा सा क्षुषितापि स्वयं गृहे}
{तिष्ठत्येव विशालाक्षी ह्यम्लानमुखपङ्कजा} %॥८॥

\twolineshloka
{न भर्तारं क्वचिदपि नास्त्यन्नमिति भाषते}
{विलोक्य भार्या सुदती कर्षती स्वकलेवरम्} %॥९॥

\twolineshloka
{विचार्य ब्राह्मणश्चित्ते भार्यायाः प्रेमबन्धनम्}
{निन्दन्भाग्यं स्वकं खिन्नः प्रोचे वाक्यं प्रियंवदाम्} %॥१०॥

\twolineshloka
{कान्ने करोमि किं कार्य न मया लभ्यते धनम्}
{याचामि च नरान्मव्यान्न यच्छन्ति च मे धनम्} %॥११॥

\twolineshloka
{किं करोमि व गच्छामि तन्मे कथय शोभने}
{विना धनेन सुश्रोणि गृहकार्य न सियति} %॥१२॥

\twolineshloka
{देशाज्ञां परदेशाय गच्छामि धनलब्धये}
{यस्मिन्देशे च यत्प्राप्यं भोग्यं तत्रैव लभ्यते} %॥१३॥

\twolineshloka
{उद्यमेन धिन सिद्धिः कर्मणां नोपलभ्यते}
{तस्माद्बुधाः प्रशंसन्ति सर्वथैव शुभोद्यमम्} %॥१४॥

\twolineshloka
{श्रुत्वा कान्तस्य वचनं लाश्रुनेत्रा विचक्षणा}
{प्रोवाच प्राञ्जलिर्भूत्वा विनयानतकन्धरा} %॥१५॥

\twolineshloka
{त्वत्तो नास्ति सुविज्ञाता त्वयाज्ञप्ता ब्रवीम्यहम्}
{हितैषिणो नरा ब्रूयुः शश्वत्साधु ह्यसाध्वपि} %॥१६॥

\twolineshloka
{पूर्वदत्तं हि लल्गेत यत्र कुत्र महीतले}
{विना दानं न लभ्येत मेरो कनकपर्वते} %॥१७॥

\twolineshloka
{पूर्वदत्ता हि या विद्या पूर्वदत्तं हि यद्धनम्}
{पूर्वइत्ता हि या भूमिरिह जन्मनि लभ्यते} %॥१८॥

\twolineshloka
{यात्रा लिखितं भाले तत्तथैव हि लभ्यते}
{विना दानेन तु क्वापि लभ्यते नैव किश्चन} %॥१९॥

\twolineshloka
{पूर्वजन्मानि विप्रेन्द्र न मया म त्वया क्वचित्}
{सत्पात्राणां करे दत्तं स्वरूपं भूर्यपि सद्धनम्} %॥२०॥

\twolineshloka
{इह देशे परे वाऽपि दत्तं सर्वत्र लभ्यते}
{अन्नमात्र तु विश्वेशो विना दत्तेन यच्छति} %॥२१॥

\twolineshloka
{तस्मादत्रैव विप्राग्य स्थातव्यं भवता मया}
{त्वां विनाहं न तिष्ठामि क्षणमात्र महामुने} %॥२२॥

\twolineshloka
{न माता न पिता भ्राता न श्वश्रूः श्वशुरो जनः}
{न सत्कुर्वन्ति केऽपि स्त्री स्वजनाश्च परे कुतः} %॥२३॥

\twolineshloka
{भी वियुक्तां निन्दन्ति दुर्भगेति वदन्ति च}
{तस्मादत्र स्थिरो भूत्वा विहरस्व यथासुखम्} %॥२४॥

\twolineshloka
{भवतो भाग्ययोगेन प्राप्तिश्चात्र भविष्यति}
{श्रुत्वा तस्यास्तु वचनं स्थितस्तत्र विचक्षणः} %॥२५॥

\twolineshloka
{तावत्तत्र समायातः कौण्डिन्यो मुनिसत्तमः}
{दृष्ट्वा समागतं हृष्टः सुमेधा द्विजसत्तमः} %॥२६॥

\twolineshloka
{सभार्यः सहसोत्थाय ननाम शिरसाऽसकृत्}
{धन्योऽप्यनुगृहीतोऽस्मि सफलं जीवितं मम} %॥२७॥

\twolineshloka
{यदृष्टोसि महाभाग्यादित्युवाच मुनीश्वरम्}
{दत्त्वा सुविष्ठरं तस्मै पूजयामास तं द्विजम्} %॥२८॥

\twolineshloka
{भोजयित्वा विधानेन पप्रच्छ प्रमदोत्तमा}
{विद्वन्केन प्रकारेण दारिद्यस्य क्षयो भवेत्} %॥२९॥

\twolineshloka
{विना दत्तं कथं लभ्येद्धनं विद्या कुटुम्बिनी}
{मां मे भर्ता परित्यज्य गन्तुकामोऽद्य वर्तते} %॥३०॥

\twolineshloka
{अन्यदेशं परॉल्लोकान्याचितुं परपत्तने}
{सम्प्रार्थ्य तु मया विद्वन् हेतुवाक्यमहत्तरैः} %॥३१॥

\twolineshloka
{नादत्तं लभ्यते किञ्चिदित्युक्त्वा स निवारितः}
{मम भाग्यान्मुनीन्द्राद्य त्वमत्रैव समागतः} %॥३२॥

\twolineshloka
{दारिद्यं त्वत्प्रसादान्मे शीघ्रं नश्यत्यसंशयम्}
{केनोपायेन विपेन्द्र दारिद्यं नश्यति ध्रुवम्} %॥३३॥

\twolineshloka
{कथयस्व कृपासिन्धो व्रतं तीर्थ तपादिकम्}
{श्रुत्वा तस्याः सुशीलाया भाषितं मुनिपुङ्गवः} %॥३४॥

\twolineshloka
{प्रोवाच प्रवरं चित्ते विचार्य व्रतमुत्तमम्}
{सर्वपापौघशमनं दुःखदारिद्यनाशनम्} %॥३५॥

\twolineshloka
{परमानाम विख्याता विष्णोस्तिथिरनुत्तमा}
{मलिम्लुचे तु या कृष्णा भुक्तिमुक्तिफलप्रदा} %॥३६॥

\twolineshloka
{तस्यामुपोषणं कृत्वा धनधान्ययुतो भवेत्}
{विधिना जागरैः साकं गीतवादित्रसंयुतम्} %॥३७॥

\twolineshloka
{धनदेन यदाचीर्ण व्रतमेतत्सुशोभनम्}
{तदा हृष्टेन रुद्रेण धनानामधिपः कृतः} %॥३८॥

\twolineshloka
{हरिश्चन्द्रेण च कृतं पुरा क्रीतसुतेन वै}
{पुनः प्राप्ता प्रिया तेन राज्यं निहतकण्टकम्} %॥३९॥

\twolineshloka
{तस्मात्कुरु विशालाक्षि व्रतमेतत्सुशोभनम्}
{यथोक्तविधिना भद्रे समं जागरणेन च} %॥४०॥

\twolineshloka
{इत्युक्त्वा तद्विधि सर्व कथयामास वाडवः}
{पुनः प्रोवाच तं विप्रं पञ्चरात्रिव्रतं शुभम्} %॥४१॥

\twolineshloka
{यस्यानुष्ठानमात्रेण भुक्तिर्मुक्तिश्च प्राप्यते}
{परमादिवसे प्रातः कृत्वा पौर्वाहिकं विधिम्} %॥४२॥

\twolineshloka
{कुर्यात् मुनियमाञ्छक्त्या पञ्चराविव्रतादरात्}
{प्रातः स्नात्वा निराहारो यस्तिष्ठेदिनपञ्चकम्} %॥४३॥

\twolineshloka
{स गच्छेद्वैष्णवं स्थानं पितृमातृप्रियायुतः}
{एकाशनस्तु यो भूयादिनानां पञ्चकं नरः} %॥४४॥

\twolineshloka
{सर्वपापविनिर्मुक्तः स्वर्गलोके महीयते}
{स्नात्वा यो भोजयेद्विप्रं दिनानां पञ्चकं नरः} %॥४५॥

\twolineshloka
{भोजितं तेन हि जगत्सदेवासुरमानुषम्}
{पूर्ण कुम्भं सुतोयेन यो ददाति द्विजातये} %॥४६॥

\twolineshloka
{दत्तं तेनैव सकलं ब्रह्माण्डं सचराचरम्}
{तिलपात्रं तु यो दद्याद्राह्मणाय विपश्चिते} %॥४७॥

\twolineshloka
{तिलसङ्ख्यासमाः साध्वि स वसेन्नाकमण्डले}
{घृतपात्रं तु यो दद्यात्स्नात्वा पञ्चदिनं नरः} %॥४८॥

\twolineshloka
{स भुक्त्वा विपुलाभोगान्सूर्यलोके महीयते}
{ब्रह्मचर्येण यस्तिष्ठेदिनानां पञ्चकं नरः} %॥४९॥

\twolineshloka
{भुनक्ति स स्वर्गभोगान्स्वर्वेश्याभिः समं मुदा}
{एवंविधं व्रतं साध्वि कुरु त्वं पतिना शुभे} %॥५०॥

\twolineshloka
{धनधान्ययुता भूत्वा स्वर्ग यास्यसि सुव्रते}
{इत्युक्ता सा व्रतं चक्रे कौण्डिन्येन यथोदितम्} %॥५१॥

\twolineshloka
{भर्ना समं भावयुता स्नात्वा मासि मलिम्लुचे}
{पञ्चरात्रव्रते पूर्णे परायाः प्रियसंयुता} %॥५२॥

\twolineshloka
{सापश्यद्राजभवनादायान्तं नृपनन्दनम्}
{स दत्त्वा नव्यभवनं भव्यवस्तुसमन्वितम्} %॥५३॥

\twolineshloka
{वासयामास विधिना विधिना प्रेरितः स्वयम्}
{दत्त्वा ग्रामं वृत्तिकरं ब्राह्मणाय सुमेधसे} %॥५४॥

\twolineshloka
{प्रसन्नस्तपसा राजा तं स्तुत्वा स्वगृहं ययौ}
{मलिम्लुचस्य मासस्य पराख्यायाः परादरात्} %॥५५॥

\twolineshloka
{उपोषणात्स कृष्णायाः पञ्चरात्रव्रतेन च}
{सर्वपापविनिर्मुक्तः सर्वसौख्यसमन्वितः} %॥५६॥

\onelineshloka*
{भुक्त्वा भोगास्त्रिया सार्धमन्ते विष्णुपुरं ययौ}

\uvacha{श्रीकृष्ण उवाच}

\onelineshloka
{पञ्चरात्रभवं पुण्यं मया वक्तुं न शक्यते} %॥५७॥

\twolineshloka
{तथापि किञ्चिद्वक्ष्यामि येन चीर्ण पराव्रतम्}
{स्नातानि पुष्कराद्यानि गङ्गाद्याः सरितस्तथा} %॥५८॥

\twolineshloka
{धेनुमुख्यानि दानानि तेन चीर्णानि सर्वथा}
{गयाश्राद्धं कृतं तेन पितरः परितोषिताः} %॥५९॥

\twolineshloka
{व्रतानि तेन चीर्णानि व्रतखण्डोदितानि वै}
{द्विपदां ब्राह्मणः श्रेष्ठो गौर्वरिष्ठा चतुस्पदाम्} %॥६०॥

\twolineshloka
{देवानां वासवः श्रेष्ठस्तथा मासो मलिम्लुचः}
{मलिम्लुचे पञ्चरात्रं महापापहरं स्मृतम्} %॥६१॥

\twolineshloka
{पञ्चरात्रे च परमा पद्मिनी पापशोषिणी}
{सैकाप्यशक्तैः कर्तव्याऽवश्यं भक्त्या विचक्षणैः} %॥६२॥

\twolineshloka
{मानुषं जनुरासाद्य न स्नातो यैर्मलिम्लुचः}
{ते जन्मघातिनो नूनं नोपोष्य हरिवासरे} %॥६३॥

\twolineshloka
{योनीभ्रमद्भिश्चवरशीतिलक्षाणि मानवैः}
{प्राप्यते मानुषं जन्म दुर्लभं पुण्यसञ्चयः} %॥६४॥

\onelineshloka*
{तस्मात्कार्य प्रयलेन परमाया व्रतं शुभम्}

\uvacha{श्रीकृष्ण उवाच}

\onelineshloka
{एतत्ते सर्वमाख्यातं यत्पृष्टोऽहं त्वयानघ} %॥६५॥

\twolineshloka
{मलिम्लुचस्य मासस्य परमायाः शुभं व्रतम्}
{तत्सर्व ते समाख्यातं कुरुप्यावहितो नृप} %॥६६॥

\twolineshloka
{ये त्वेवं भुवि परमा व्रतं चरन्ति सद्भक्त्या शुभविधिना मलिम्लुचे वै}
{ते भुक्त्वा दिवि विभवं सुरेन्द्रतुल्यं गच्छेयुत्रिभुवननन्दितस्य गेहम्} %॥६७॥

॥इत्यधिककृष्णैकादश्याः परमाख्याया माहात्म्यं सम्पूर्णम्॥

\hyperref[sec:ekadashi_mahatmyam_vrata_raja]{\closesub}