\sect{उन्मीलनीव्रतम्}
\label{sec:padma-unmilani}


\uvacha{महादेव उवाच}

\twolineshloka
{अतस्त्वां सम्प्रवक्ष्यामि उन्मीलनीमनुत्तमाम्}
{यस्याः श्रवणमात्रेण जन्मसंसारबन्धनात्}% ॥१॥

\twolineshloka
{पापात्मा मुच्यते पापैः स्वर्गलोके महीयते}
{देवताः पितरश्चैव तस्या गतिमवाप्नुयुः}% ॥२॥

\twolineshloka
{विद्यार्थी लभते विद्यां सर्वकामानवाप्नुयात्}
{तस्या व्रतान्न सन्देहः स्वर्गलोके महीयते}% ॥३॥

\twolineshloka
{स्वस्थानं तत्र वै प्राप्तः शिवलोके महीयते}
{अतस्त्वं कुरुषे राजन्वैष्णवानां तु पूजनम्}% ॥४॥

\twolineshloka
{वैष्णवानां तु ये राजन्सेवा कुर्वन्ति नित्यशः}
{तेषां दण्डं च कुरुषे नो वा तेषां नराधिपः}% ॥५॥

\twolineshloka
{भोजनानन्तरं तेषां भोजनं कुरुते नृप}
{तैरेव पूजितो विष्णुर्यैर्भक्त्या तु प्रपूजितः}% ॥६॥

\twolineshloka
{शालग्रामशिलाभूतां दत्त्वा मूर्धनि प्रत्यहम्}
{त्वं धारयसि भूपाल कण्ठे नित्यं सुभक्तितः}% ॥७॥

\twolineshloka
{धूपशेषं तु वै विष्णोर्भक्त्या भजसि भूपते}
{आरार्तिकं सदा कृत्वा भक्तानां वेदयेर्नृप}% ॥८॥

\twolineshloka
{शङ्खोदकं हरेर्मूर्ध्नि भ्रामयित्वा तु भक्तितः}
{नित्यं बिभर्षि शिरसि शेषं यच्छसि वैष्णवान्}% ॥९॥

\twolineshloka
{नैवेद्यं प्रत्यहं कृत्वा सर्वोपस्करसंयुतम्}
{विष्वक्सेनाय दत्वा वै स्वयं भुनक्षि वाडव}% ॥१०॥

\twolineshloka
{विष्णोर्निवेदितं चान्नं वैष्णवैः सह भुज्यते}
{नित्यं नामसहस्रेण भक्त्या स्तौषि जनार्दनम्}% ॥११॥

\twolineshloka
{दीपार्घदानं वै विष्णोः कुरुषे गीतनर्तनम्}
{श्यामाङ्कुरैः पूजयसे पूज्यन्ते नृपसत्तम}% ॥१२॥

\twolineshloka
{श्यामाङ्कुरैः सदा वत्स पूजनं चाति दुर्ल्लभम्}
{पृथ्वीदानसमं पुण्यं दूर्वया पूजने कृते}% ॥१३॥

\twolineshloka
{अतो वै नास्ति लोकेऽस्मिन्दूर्वायाः सदृशं भुवि}
{तया वै पूजनं कार्यं विष्णुसायुज्यमिच्छता}% ॥१४॥

\twolineshloka
{अतस्त्वं कुरुषे नित्यं पूजनं दूर्वया सह}
{यवाक्षतैर्विशेषेण पूजनं कुरुषेन वा}% ॥१५॥

\twolineshloka
{पक्षेपक्षे नृपश्रेष्ठ विधिवद्द्वादशीव्रतम्}
{यत्कृतं तु महाराज महापापप्रणाशनम्}% ॥१६॥

\twolineshloka
{मोक्षदं सुखदं चैव तथायुष्यप्रदं सदा}
{एतद्विष्णुव्रतं प्रोक्तं वैष्णवानां तु मोक्षदम्}% ॥१७॥

\twolineshloka
{गृहस्थानां तु सुखदं यतीनां मुक्तिदायकम्}
{सर्वरोगादिशमनं पवित्रं कायशोधनम्}% ॥१८॥

\twolineshloka
{व्रतमेतच्च कुरुषे नो वा चैव नराधिप}
{दशमीवेधरहितं कुरुषे जागरान्वितम्}% ॥१९॥

\twolineshloka
{तुलसीपत्रनिकरैर्नित्यं पूजयसे हरिम्}
{गोपीचन्दनजं पत्रं भाले वा नृपसत्तम}% ॥२०॥

\twolineshloka
{धारितं सर्वलोकानां पवित्रीकरणं नृप}
{अतस्त्वं च धारयसे गोपीचन्दनसम्भवम्}% ॥२१॥

\twolineshloka
{ब्रह्महा हेमहारी च मद्यपानी तथैव च}
{अगम्यगो महापापी तथा ह्यनृतभाषितः}% ॥२२॥

\twolineshloka
{ते सर्वे मुक्तिमायान्ति तिलकं धारणादृताः}
{बिभर्षि कण्ठे नित्यं त्वं धात्रीफलसमुद्भवाम्}% ॥२३॥

\twolineshloka
{मालां मुख्यायुतसमां तुलसीपत्रसम्भवाम्}
{शालग्रामशिलायुक्तां द्वारकायां समुद्भवाम्}% ॥२४॥

\twolineshloka
{नित्यं पूजयसे भूप भुक्तिमुक्तिफलप्रदाम्}
{पद्मसंज्ञं पुराणं वै पठसे पुरतो हरेः}% ॥२५॥

\twolineshloka
{चरितं दैत्यराज्यस्य प्रह्लादस्य च भूपते}
{वासरं वासुदेवस्य सवेधं कुर्वतो नरान्}% ॥२६॥

\twolineshloka
{निवारयसि भूपाल शास्त्रं दृष्ट्वा प्रयत्नतः}
{सवेधं वासरं विष्णोर्यस्मिन्राष्ट्रे प्रवर्तते}% ॥२७॥

\threelineshloka
{लिप्यते तेन पापेन राजा भवति नारकी}
{वेधं चतुर्विधं त्यक्त्वा समुपोष्य हरेर्दिनम्}
{कुलकोटिं समुद्धृत्य विष्णुलोके महीयते}% ॥२८॥

\uvacha{गौतम उवाच}

\twolineshloka
{शृणु भूपाल वक्ष्यामि वैष्णवाख्यं महाव्रतम्}
{यं श्रुत्वा पापिनः सर्वे मुक्तिमायान्ति तत्क्षणात्}% ॥२९॥

\onelineshloka
{द्वादशीसम्भवं पुण्यं मया ख्यातं न कस्यचित्}% ॥३०॥

\twolineshloka
{वैष्णवोऽसि महाराज भक्तो भागवते नृणाम्}
{वैष्णवं तु महागुह्यं तद्व्रतं त्वं निशामय}% ॥३१॥

\twolineshloka
{उन्मीलनी नाम पुरा भक्त्या मे माधवेन तु}
{कथिता सुप्रसन्नेन तां ते भूप वदाम्यहम्}% ॥३२॥

\twolineshloka
{एकादशी अहोरात्रं प्रभाते घटिका भवेत्}
{उन्मीलनी तु सा ज्ञेया विशेषेण हरिप्रिया}% ॥३३॥

\twolineshloka
{त्रैलोक्ये यानि तीर्थानि पुण्यान्यायतनानि च}
{कोट्यंशे नैव तुल्यानि मखा वेदास्तपांसि च}% ॥३४॥

\twolineshloka
{उन्मीलनी समं किञ्चिन्न भूतं न भविष्यति}
{प्रयागो न कुरुक्षेत्रं न काशी न च पुष्करः}% ॥३५॥

\twolineshloka
{शैलो हिमाचलो नैव न मेरुर्गन्धमादनः}
{शैलो न नीलनिषधो न विन्ध्यो नैव नैमिषम्}% ॥३६॥

\twolineshloka
{गोदावरी न कावेरी चन्द्रभागा न वेदिका}
{न तापी न पयोष्णी च न क्षिप्रा नैव चन्दना}% ॥३७॥

\twolineshloka
{चर्मण्वती च सरयूश्चन्द्रभागा न गण्डिका}
{गोमती च विपाशा च शोणाख्यश्च महानदः}% ॥३८॥

\twolineshloka
{किमत्र बहुनोक्तेन भूयोभूयो नराधिप}
{उन्मीलनीसमं किञ्चिन्न देवः केशवात्परः}% ॥३९॥

\twolineshloka
{उन्मीलनीमनु प्राप्य यैः कृतं केशवार्चनम्}
{पापचक्रसमूहस्य राशयः पतिताः क्षणात्}% ॥४०॥

\twolineshloka
{यस्मिन्मासे महीपाल तिथिरुन्मीलनी भवेत्}
{तन्मासनाम्ना गोविन्दः पूजनीयः प्रयत्नतः}% ॥४१॥

\twolineshloka
{जातरूपमयः कार्यं मासनाम्ना तु माधवः}
{स्वशक्त्या विश्वरूपस्तु श्रद्धाभक्तिसमन्वितः}% ॥४२॥

\twolineshloka
{पवित्रोदकसंयुक्तं पञ्चरत्नसमन्वितम्}
{गन्धपुष्पाक्षतैर्युक्तं कुम्भं स्रग्दामभूषितम्}% ॥४३॥

\twolineshloka
{पात्रं च सोदकं कार्यं गोधूमैश्चापि पूरितम्}
{नानारत्नैश्च संयुक्तं नानागन्धैः प्रपूजितम्}% ॥४४॥

\twolineshloka
{मल्लिकामोदसंयुक्तं जातीपुष्पैः प्रपूजितम्}
{श्वेताख्यैस्तन्दुलैश्चैव पूरणीयः प्रयत्नतः}% ॥४५॥

\twolineshloka
{प्रदद्याद्वस्त्रयुग्मं तु उपवीतं तु सोत्तरम्}
{उपानहौ तु राजर्षे आतपत्रं शिरोपरि}% ॥४६॥

\twolineshloka
{भोजनं जलपात्रं च सप्तधान्यं तिलैः सह}
{रूप्यं चैव तु कार्पासं पायसं मुद्रिका हरेः}% ॥४७॥

\twolineshloka
{धेनुर्वात्र तु दातव्या वत्सालङ्कारसंयुता}
{सुवर्णशृङ्गी रौप्यखुरी ताम्रपृष्ठी तथैव च}% ॥४८॥

\twolineshloka
{कांस्यदोहीं रत्नपुच्छीं दद्याद्वै गुरवे तदा}
{शय्यां सोपस्करां दद्यात्साधवे भक्तिपूर्वकम्}% ॥४९॥

\twolineshloka
{धूपं दीपं तु नैवेद्यं फलपत्रं निवेदयेत्}
{पूजनीयो महाभक्तैर्मन्त्रैरेभिस्तु केशवः}% ॥५०॥

\twolineshloka
{तुलसीपत्रसंयुक्तैः पुष्पैः कालोद्भवैस्तथा}
{मासनाम्नैव चरणौ जानुनी विष्णुरूपिणे}% ॥५१॥

\twolineshloka
{गुह्ये तु गुह्यपतये कटे वै पीतवाससे}
{ब्रह्ममूर्तिभृते नाभावुदरे विश्वयोनये}% ॥५२॥

\twolineshloka
{हृदये ज्ञानगम्याय कण्ठे वैकुण्ठमूर्तये}
{ऊर्ध्वगाय ललाटे तु बाहौ दक्षान्तकारिणे}% ॥५३॥

\twolineshloka
{उत्तमाङ्गे सुरेशाय सर्वाङ्गे सर्वमूर्तये}
{स्वनाम्ना चायुधादीनि पूजनीयानि भक्तितः}% ॥५४॥

\twolineshloka
{अर्घदानं प्रकर्तव्यं नालिकेरादिभिः समम्}
{शङ्खोपरि जले कृत्वा गन्धपुष्पाक्षतान्वितम्}% ॥५५॥

\twolineshloka
{सूत्रेणावेष्टितं कृत्वा दद्यादर्घं विधानतः}
{देवदेव महादेव श्रीकेशव जनार्दन}% ॥५६॥

\twolineshloka
{सुब्रह्मण्य नमस्तेऽस्तु पुण्यराशिविवर्धन}
{शोकमोहमहापापान्मामुद्धर भवार्णवात्}% ॥५७॥

\twolineshloka
{सुकृतं न कृतं किचिज्जन्मकोटिशतैरपि}
{तथापि मां महास्वामिन्मामुद्धर भवार्णवात्}% ॥५८॥

\twolineshloka
{व्रतेनानेन देवेश ये चान्ये मम पूर्वजाः}
{वियोनिं च गताश्चान्ये पापमृत्युवशं गताः}% ॥५९॥

\twolineshloka
{ये भविष्यन्ति येऽतीताः प्रेतलोकान्समुद्धर}
{श्रान्तोस्म्यहमधीनस्य भक्तिरव्यभिचारिणी}% ॥६०॥

\twolineshloka
{दत्तमर्घं मया तुभ्यं भक्त्या गृह्ण गदाधर}
{दत्वार्घं धूपदीपाद्यैर्नैवेद्यैर्विष्णुसम्भवैः}% ॥६१॥

\twolineshloka
{स्तोत्रैर्नीराजनैर्गीतैर्भृत्यैः सन्तोषयेद्धरिम्}
{वस्त्रैर्दानैश्च गोदानैर्भोजनैस्तोषयेद्गुरुम्}% ॥६२॥

\twolineshloka
{तथातथा विधातव्यं गुरुर्वै प्रीतिमाप्नुयात्}
{लोकानां तारणार्थाय धात्रा सृष्टो गुरुर्यतः}% ॥६३॥

\twolineshloka
{अतो वै गुरुपूजा च कर्तव्या वै प्रयत्नतः}
{अहितं यो नाशयति स्वहितं दर्शयेत्सदा}% ॥६४॥

\twolineshloka
{स गुरुः स च विज्ञेयः सर्वधर्मार्थकोविदः}
{अकुर्वन्वित्तशाठ्यं तु गुरवे तं निवेदयेत्}% ॥६५॥

\twolineshloka
{गुरोर्निवेदिते भूप परिपूर्णं भवेद्व्रतम्}
{कृत्वा दिवातनं कर्म भोजनं ब्राह्मणैः सह}% ॥६६॥

\twolineshloka
{कर्तव्यं नृपशार्दूल दिनं नेयं कथानकैः}
{अनेन विधिना यस्तु कुर्यादुन्मीलनीव्रतम्}% ॥६७॥

\onelineshloka
{कल्पकोटिसहस्राणि वसते विष्णुसन्निधौ}% ॥६८॥

॥इति श्रीपाद्मे महापुराणे पञ्चपञ्चाशत्साहस्र्यां संहितायामुत्तरखण्डे उमापतिनारदसंवादान्तर्गतकृष्णयुधिष्ठिरसंवादे उन्मीलनीव्रतं नाम सप्तत्रिंशोऽध्यायः॥३७॥


\hyperref[sec:ekadashi_mahatmyam_padma_puranam]{\closesub}
\clearpage

\sect{पक्षवर्धिनी-एकादशी-माहात्म्यम्}
\label{sec:padma-pakshavardhini}


\uvacha{नारद उवाच}

\twolineshloka
{कीदृशी स्यान्महादेव पक्षवर्धनिसंज्ञका}
{यया वै कृतया जन्तुर्महपापात्प्रमुच्यते}% ॥१॥

\uvacha{श्रीमहादेव उवाच}

\onelineshloka*
{अमा वा यदि वा पूर्णा सम्पूर्णा जायते तदा}

\twolineshloka
{भूत्वा वै नाडिकाषष्ठिर्जायते प्रतिपद्दिने}
{अश्वमेधायुतैस्तुल्या सा भवेत्पक्षवर्धिनी}% ॥२॥

\uvacha{नारद उवाच}

\twolineshloka
{पूजाविधिं तु पृच्छामि साम्प्रतं देवसत्तम}
{यत्कृते तु महादेव महाफलमवाप्नुयात्}% ॥३॥

\uvacha{महादेव उवाच}

\twolineshloka
{पूजाविधिं प्रवक्ष्यामि साम्प्रतं द्विजनन्दन}
{पूजिते चार्चिते विष्णौ फलं प्राप्नोत्यसंशयः}% ॥४॥

\twolineshloka
{येन पूजाविधानेन तुष्टिं प्राप्नोति माधवः}
{अव्रणं जलपूर्णं च कुम्भं चन्दनचर्चितम्}% ॥५॥

\twolineshloka
{पञ्चरत्नसमायुक्तं पुष्पमालाभिवेष्टितम्}
{स्थाप्यं ताम्रमयं पात्रं सगोधूमं घटोपरि}% ॥६॥

\twolineshloka
{सौवर्णं कारयेद्देवं माससंज्ञाभिनामकम्}
{पञ्चामृतेन विधिना स्नपनं सुमनोरमम्}% ॥७॥

\twolineshloka
{कारयेद्देवदेवेशं जगन्नाथं जगत्पतिम्}
{विलेपनं तु कर्तव्यं कुङ्कुमागरुचन्दनैः}% ॥८॥

\twolineshloka
{वस्त्रयुग्मं च दातव्यं छत्रोपानहसंयुतम्}
{पूजयेद्देवताधीशं कुम्भपात्रोपरिस्थितम्}% ॥९॥

\twolineshloka
{पद्मनाभाय वै पादौ जानुनी विश्वमूर्तये}
{ऊरू वै ज्ञानगम्याय कटी दानप्रदाय च}% ॥१०॥

\twolineshloka
{उदरं विश्वनाथाय हृदयं श्रीधराय च}
{कण्ठं कौस्तुभकण्ठाय बाहू क्षत्रान्तकारिणे}% ॥११॥

\twolineshloka
{ललाटं व्योममूर्ध्ने तु शिरो वै सर्वरूपिणे}
{स्वनाम्ना चैव कमलां सर्वाङ्गीं दिव्यरूपिणीम्}% ॥१२॥

\twolineshloka
{एवं विधिवत्सम्पूज्य ततोऽर्घं दापयेत्सुधीः}
{नालिकेरेण शुभ्रेण देवदेवस्य चक्रिणः}% ॥१३॥

\twolineshloka
{अनेनैवार्घदानेन सम्पूर्णं जायते व्रतम्}
{संसारार्णवमग्नं मां समुद्धर जगत्पते}% ॥१४॥

\twolineshloka
{त्वमीशः सर्वलोकानां त्वं साक्षाच्च जगत्पतिः}
{गृहाणार्घं मया दत्तं पद्मनाभ नमोस्तु ते}% ॥१५॥

\twolineshloka
{नैवेद्यानि सुहृद्यानि षड्रसानि विशेषतः}
{देयानि तु विशेषेण केशवाय सुभक्तितः}% ॥१६॥

\twolineshloka
{नागपत्रं सकर्पूरं दद्याद्देवस्य भक्तितः}
{घृतेन दीपकं कुर्यात्तिलतैलेन वा पुनः}% ॥१७॥

\twolineshloka
{कृत्वा सम्यग्विधानेन गुरोः पूजां प्रकारयेत्}
{वस्त्राणि चैव चोष्णीषं कञ्चुकं च प्रदापयेत्}% ॥१८॥

\twolineshloka
{दक्षिणां च यथाशक्त्या गुरुवे सम्प्रदापयेत् }
{भोजनं चैव ताम्बूलं दत्त्वा चार्घं प्रदापयेत्}% ॥१९॥

\twolineshloka
{स्ववित्तस्यानुमानेन यथाशक्त्या तु निर्धनैः}
{कार्या सम्यक्प्रयत्नेन द्वादशी पक्षवर्धिनी}% ॥२०॥

\twolineshloka
{ततो जागरणं कुर्याद्गीतनृत्यसमन्वितम्}
{पुराणपाठसहितं हास्याह्लादसमन्वितम्}% ॥२१॥

\twolineshloka
{स्तुवन्ति च प्रशंसन्ति जागरं चक्रपाणिनः}
{नित्योत्सवो भवेत्तेषां गृहे वै दशजन्मसु}% ॥२२॥

\twolineshloka
{अतो धन्यतमा चेयं कर्तव्या पक्षवर्धनी}
{कृत्वा तु सकलं पुण्यं फलं प्राप्नोत्यसंशयम्}% ॥२३॥

\twolineshloka
{पक्षवर्धनी माहात्म्यं ये शृण्वन्ति मनीषिणः}
{तैः कृतं सत्कृतं सर्वं यावदाभूतसम्प्लवम्}% ॥२४॥

\twolineshloka
{पञ्चाग्निसाधने पुण्यं यच्च स्यात्तीर्थसाधने}
{तत्पुण्यं समवाप्नोति विष्णोर्जागरकारणात्}% ॥२५॥

\twolineshloka
{पक्षवर्धनिका पुण्या पवित्रा पापनाशिनी}
{उपवासकृता विप्र हत्याकोटिविनाशिनी}% ॥२६॥

\twolineshloka
{वसिष्ठेन कृता पूर्वं भारद्वाजेन वै मुने}
{ध्रुवेण चाम्बरीषेण कृतेयं विष्णुवल्लभा}% ॥२७॥

\twolineshloka
{इयं काशीसमा पुण्या इयं वै द्वारका समा}
{उपोषिता च भक्तेन वाञ्छितं च ददात्यसौ}% ॥२८॥

\twolineshloka
{इयं धन्या धन्यतमा हत्यायुतविनाशिनी}
{कर्तव्या तु विशेषेण वैष्णवैर्ज्ञानतत्परैः}% ॥२९॥

\twolineshloka
{अहो सर्वेश्वरो देवः संसेव्यो व्रततत्परैः}
{किमन्यद्बहुनोक्तेन कर्तव्यं व्रतमुत्तमम्}% ॥३०॥

\twolineshloka
{यथा च वर्धते चन्द्रः सिते पक्षे विशेषतः}
{तथा वै वर्धते भक्त कारणात्पक्षवर्धिनी}% ॥३१॥

\twolineshloka
{सूर्योदये यथा ध्वान्तं नश्यते तत्क्षणादपि}
{तथाघं नाशमाप्नोति करणात्पक्षवर्धिनी}% ॥३२॥

॥इति श्रीपाद्मे महापुराणे पञ्चपञ्चाशत्साहस्र्यां संहितायामुत्तरखण्डे उमापतिनारदसंवादान्तर्गतकृष्णयुधिष्ठिरसंवादे पक्षवर्धिनी-एकादशी-माहात्म्यं नाम नामाष्टत्रिंशोऽध्यायः॥३८॥


\hyperref[sec:ekadashi_mahatmyam_padma_puranam]{\closesub}
\clearpage

\sect{द्वादशी-एकादशी-जागरणमहिमावर्णनम्}
\label{sec:padma-jagaranamahima}


\uvacha{महादेव उवाच}

\twolineshloka
{शृणु नारद वक्ष्यामि माहात्म्यं जागरस्य च}
{यच्छ्रुत्वा मुक्तिमाप्नोति महापापी न संशयः}% ॥१॥

\uvacha{नारद उवाच}

\twolineshloka
{अहो विश्वेश्वरो विष्णुः पवित्रीकरणः सदा}
{तस्योपवासमाहात्म्यं श्रुतं हि त्वन्मुखाच्छिव}% ॥२॥

\twolineshloka
{तथापि श्रोतुमिच्छामि माहात्म्यं जागरस्य तु }
{कीदृक्जागरमाहात्म्यं रात्रो भक्तिस्तु  कीदृशी}% ॥३॥

\threelineshloka
{प्रहरेषु च या पूजा वद विश्वेश्वर प्रभो}
{त्वं लोकेषु सदा पूज्यस्त्वं हि देवो जनार्दनः}
{त्वं हि विश्वेश्वरो देवो यतो भक्तिर्जनार्दने}% ॥४॥

\twolineshloka
{सर्वेषां चैव भक्तानां त्वं च श्रेष्ठ उमापतिः}
{लोकेस्मिन्सर्वदा भक्त्या तवाख्या वर्तते सदा}% ॥५॥

\twolineshloka
{अतो येन प्रकारेण लोकानां मुक्तिरेव च}
{विश्वेश्वर वद त्वं तु माहात्म्यं जागरस्य तु}% ॥६॥

\uvacha{महादेव उवाच}

\twolineshloka
{एकादश्यां जनो विष्णुं रात्रौ सम्पूज्य भक्तितः}
{कुर्याज्जागरणं विष्णोः पुरतो वैष्णवैः सह}% ॥७॥

\twolineshloka
{गीतं वाद्यं तथा नृत्यं पुराणपठनं तथा}
{धूपं दीपं च नैवेद्यं पुष्पं गन्धानुलेपनम्}% ॥८॥

\twolineshloka
{फलमर्घं तथा श्रद्धा दानमिन्द्रियसंयमम्}
{सत्यान्वितं च विप्रेन्द्र वचोयुक्तं क्रियान्वितम्}% ॥९॥

\twolineshloka
{विनिद्रं च मुदायुक्तो यः करोति नरः सदा}
{सर्वपापविनिर्मुक्तो जायते विष्णुवल्लभः}% ॥१०॥

\twolineshloka
{रात्रौ जागरणे प्राप्ते निद्रां कुर्वन्ति वैष्णवाः}
{हारितं चोपवासं तैर्व्रतं वै विष्णुसंज्ञकम्}% ॥११॥

\twolineshloka
{ये कुर्वन्ति नराः प्राज्ञ जागरे विष्णुसंज्ञके}
{जागरं कृष्णभावेन न स्वपन्ति कदाचन}% ॥१२॥

\twolineshloka
{कृष्णस्य नाम मनसा वदन्ति च पुनः पुनः}
{ते तु धन्यतमा ज्ञेया अस्यां रात्रौ विशेषतः}% ॥१३॥

\twolineshloka
{क्षणेक्षणे तु गोदानं घट्यां चैव चतुर्गुणम्}
{प्रहरे कोटिगुणितं चतुर्यामेष्वसङ्ख्यकम्}% ॥१४॥

\twolineshloka
{जागरे निमिषार्धे तु केशवाग्रे विशेषतः}
{तत्फलं कोटिगुणितं तस्य सङ्ख्या न विद्यते}% ॥१५॥

\twolineshloka
{नर्तनं कुरुते यस्तु केशवाग्रे नरोत्तमः}
{न फलं हीयते तस्य आजन्ममरणान्तिकम्}% ॥१६॥

\twolineshloka
{साश्चर्यं चैव सोत्साहं पापालापादिवर्जितम्}
{प्रदक्षिणसमायुक्तं नमस्कारपुरःसरम्}% ॥१७॥

\twolineshloka
{नीराजनसमायुक्तमनिर्विण्णेन चेतसा}
{यामेयामे महाभाग कुर्यादारार्तिकं हरेः}% ॥१८॥

\twolineshloka
{षड्विंशगुणसंयुक्तमेकादश्यां च जागरम्}
{यः करोति नरो भक्त्या न पुनर्जायते भुवि}% ॥१९॥

\twolineshloka
{य एवं कुरुते भक्त्या वित्तशाठ्यविवर्जितः}
{जागरं वासरे विष्णोर्लीयते परमात्मनि}% ॥२०॥

\twolineshloka
{धनवान्वित्तशाठ्येन यः करोति प्रजागरम्}
{तेनात्माहारितो नूनं कितवेन दुरात्मना}% ॥२१॥

\twolineshloka
{विष्णुजागरणे प्राप्ते उपहासं करोति यः}
{षष्टिवर्षसहस्राणि विष्ठायां जायते कृमिः}% ॥२२॥

\twolineshloka
{वेदविद्ब्राह्मणो यस्तु नर्तनेन विशेषतः}
{उपहासपरः प्राप्तः स वै चाण्डाल उच्यते}% ॥२३॥

\twolineshloka
{निमिषं निमिषार्धं वा यः करोति प्रजागरम्}
{धर्मार्थकाममोक्षाणां प्राप्नोति पदमव्ययम्}% ॥२४॥

\twolineshloka
{वेदशास्त्ररतो नित्यं नित्यं वै यज्ञयाजकः}
{रात्रौ जागरणे प्राप्ते निन्दां कुर्वन्व्रजत्यधः}% ॥२५॥

\twolineshloka
{मम पूजां प्रकुर्वाणो विष्णुनिन्दासु तत्परः}
{एकविंशत्कुलेनैव नरकं प्रतिपद्यते}% ॥२६॥

\threelineshloka
{विष्णुः शिवः शिवो विष्णुरेकमूर्तिर्द्विधा स्थिताः}
{तस्मात्सर्वप्रकारेण नैव निन्दां प्रकारयेत्}
{दष्टाः कलिभुजङ्गेन स्वपन्ति मधुहाहनि}% ॥२७॥

\twolineshloka
{कुर्वन्ति जागरं नैव मायया ते च मोहिताः}
{प्राप्ता एकादशी येषां कलौ जागरणं विना}% ॥२८॥

\twolineshloka
{ते विनष्टा न सन्देहो यस्माज्जीवितमध्रुवम्}
{उद्धृतं नेत्रयुग्मं तु दत्त्वा वै वैष्णवं पदम्}% ॥२९॥

\twolineshloka
{कृतं ये नैव पश्यन्ति पापिनो हरिजागरम्}
{अभावे वाचकस्याथ गीतं नृत्यं तु कारयेत्}% ॥३०॥

\twolineshloka
{वाचके सति देवर्षे पुराणं प्रथमं पठेत्}
{अश्वमेधसहस्रस्य वाजपेयायुतस्य च}% ॥३१॥

\twolineshloka
{पुण्यं कोटिगुणं वत्स विष्णोर्जागरणे कृते}
{पितृपक्षे मातृपक्षे भार्यापक्षे तु वाडव}% ॥३२॥

\twolineshloka
{कुलानुद्धरते चैतान्कृत्वा जागरणं हरेः}
{उपोषणदिने विद्धे जागरं पूजनं हरेः}% ॥३३॥

\twolineshloka
{वृथादानादिकं सर्वं कृतघ्नेषु कृतं यथा}
{उपोषणदिने विद्धे प्रारब्धे जागरे स्थितिम्}% ॥३४॥

\twolineshloka
{विहाय स्थानं तद्विष्णुः शापं दत्वा प्रगच्छति}
{अविद्धे वासरे विष्णोर्ये कुर्वन्ति प्रजागरम्}% ॥३५॥

\twolineshloka
{तेषां मध्ये तु तुष्टः सन्नृत्यं तु कुरुते हरिः}
{यावद्दिनानि कुरुते जागरं केशवाग्रतः}% ॥३६॥

\twolineshloka
{युगानि तानि तावन्ति विष्णुलोके महीयते}
{यावद्दिनानि वसते विना जागरणं हरेः}% ॥३७॥

\twolineshloka
{तावद्वर्षसहस्राणि रौरवान्न निवर्तते}
{एकदश्यां शयानस्तु विना जागरणं हरेः}% ॥३८॥

\twolineshloka
{मूकवत्तिष्ठते यो वै गानं पाठं न वाचरेत्}
{सप्तजन्मानि मूकत्वं जायतेऽजागरे हरेः}% ॥३९॥

\twolineshloka
{यो न नृत्यति मूढात्मा पुरतो जागरे हरेः}
{पङ्गुत्वं तस्य जानीयात्सप्तजन्मानि वाडव}% ॥४०॥

\twolineshloka
{यः पुनः कुरुते गीतं नृत्यं जागरणं हरेः}
{ब्राह्मं पदं मदीयं च सत्यं वै तस्य वैष्णवम्}% ॥४१॥

\twolineshloka
{यः प्रबोधयते लोकान्विष्णोर्जागरणे रतः}
{वसेच्चिरं तु वैकुण्ठे पितृभिः सह वैष्णवः}% ॥४२॥

\twolineshloka
{मतिं प्रयच्छते यस्तु हरेर्जागरणं प्रति}
{षष्टिवर्षसहस्राणि श्वेतद्वीपे वसेन्नरः}% ॥४३॥

\twolineshloka
{यत्किञ्चित्क्रियते पापं कोटिजन्मनि मानवैः}
{श्रीकृष्णजागरे सर्वं रात्रौ नश्यति वाडव}% ॥४४॥

\twolineshloka
{शालग्रामशिलाग्रे ये कुर्वन्ति प्रतिजागरम्}
{यामेयामे फलं प्रोक्तं कोट्यैन्दवसमुद्भवम्}% ॥४५॥

\twolineshloka
{सम्प्राप्ते वासरे विष्णोर्ये न कुर्वन्ति जागरम्}
{वृथा स्यात्तत्कृतं तेषां वैष्णवानां च निन्दया}% ॥४६॥

\twolineshloka
{कामार्थौ सम्पदः पुत्राः कीर्तिर्लोकाश्च शाश्वताः}
{यज्ञायुतैर्न लभ्यन्ते द्वादशीजागरं विना}% ॥४७॥

\twolineshloka
{मतिर्न जायते यस्य द्वादश्यां जागरं प्रति}
{न हि तस्याधिकारोऽस्ति पूजने केशवस्य हि}% ॥४८॥

\twolineshloka
{यावत्पदानि चलति केशवायतनं प्रति}
{अश्वमेधसमानि स्युः जागरार्थं प्रगच्छतः}% ॥४९॥

\twolineshloka
{पादयोः पतितं यावद्धरण्यां पांशु गच्छताम्}
{तावद्वर्षसहस्राणि जागरो वसते दिवि}% ॥५०॥

\twolineshloka
{तस्माद्गृहात्प्रगन्तव्यं जागरे केशवालये}
{कलौ मलविनाशाय द्वादशीद्वादशीषु च}% ॥५१॥

\twolineshloka
{परापवादसंयुक्तं मनः प्रासाद वर्जितम्}
{शास्त्रहीनमगान्धर्वं तथा दीपविवर्जितम्}% ॥५२॥

\twolineshloka
{शक्त्योपचाररहितमुदासीनं सनिन्दनम्}
{कलियुक्तं विशेषेण जागरं नवधा मतम्}% ॥५३॥

\twolineshloka
{सशास्त्रं जागरं यच्च नृत्यगान्धर्वसंयुतम्}
{सवाद्यं तालासंयुक्तं सदीपं मधुभिर्युतम्}% ॥५४॥

\twolineshloka
{उच्चारैस्तु समायुक्तं यथोक्तैर्भक्तिभावितैः}
{प्रसन्नं तुष्टिजननं सम्मूढं लोकरञ्जनम्}% ॥५५॥

\twolineshloka
{गुणैर्द्वादशभिर्युक्तं जागरं माधवप्रियम्}
{कर्तव्यं तत्प्रयत्नेन पक्षयोः शुक्लकृष्णयोः}% ॥५६॥

\twolineshloka
{किं व्रतैर्बहुभिश्चीर्णैस्तीर्थवासेन तस्य किम्}
{द्वादशीवासरे प्राप्ते न कुर्याज्जागरं हरेः}% ॥५७॥

\twolineshloka
{प्रवासेन त्यजेद्यस्तु पथिस्विन्नोऽपि वाडव}
{जागरं वासुदेवस्य द्वादश्यां तु समे प्रियः}% ॥५८॥

\twolineshloka
{मद्भक्तो न हरेः कुर्याज्जागरं पापमोहितः}
{व्यर्थं मत्पूजनं तस्य मत्पूज्यं यो न पूजयेत्}% ॥५९॥

\twolineshloka
{न शैवो न च सौरोऽसौ न शाक्तो गणसेवकः}
{यो भुङ्क्ते वासरे विष्णोर्ज्ञेयः पश्वधिको हि सः}% ॥६०॥

\twolineshloka
{विप्रियं च कृतं तेन दुष्टेनैव च पापिना}
{मद्भक्तिबलमाश्रित्य यो भुङ्क्ते वै हरेर्दिने}% ॥६१॥

\twolineshloka
{सबाह्याभ्यन्तरं देहं वेष्टितं पापकोटिभिः}
{मुच्यन्ते वासरे विष्णोर्ये कुर्वन्ति प्रजागरम्}% ॥६२॥

\twolineshloka
{कूर्परं यमदूतानां दत्तं तेन यमस्य च}
{कृत्वा जागरणं विष्णोरविद्धं द्वादशीव्रतम्}% ॥६३॥

\twolineshloka
{स्वर्गापेक्षा मुनिश्रेष्ठ मुक्ता ते नैव संशयः}
{वाञ्छितं नारकं सौख्यं विद्धं कृत्वा हरेर्दिनम्}% ॥६४॥

\twolineshloka
{निहताः पितरस्तेन देवानां वै वधः कृतः}
{दत्तं राज्यं तु दैत्यानां कृत्वा विद्धं हरेर्दिनम्}% ॥६५॥

\twolineshloka
{यो नृत्यति प्रहृष्टात्मा कृत्वा वै करताडनम्}
{गीतं कुर्वन्मुखेनापि दर्शयन्कौतुकान्बहून्}% ॥६६॥

\twolineshloka
{पुरतो वासुदेवस्य रात्रौ जागरणे स्थितः}
{पठन्कृष्णचरित्राणि रञ्जयन्वैष्णवान्गणान्}% ॥६७॥

\twolineshloka
{मुखेन कुरुते वाद्यं सम्प्रहृष्टतनूरुहः}
{दर्शयन्विविधान्भृत्यान्स्वेच्छालापान्प्रकारयन्}% ॥६८॥

\twolineshloka
{भावैरेतैर्नरो यस्तु कुरुते जागरे हरेः}
{निमिषेनिमिषे पुण्यं तीर्थकोटिफलं स्मृतम्}% ॥६९॥

\twolineshloka
{अनुद्विग्नमना यस्तु धूपनीराजनं हरेः}
{कुरुते जागरे रात्रौ सप्तद्वीपाधिपो भवेत्}% ॥७०॥

\twolineshloka
{यानि कानि च पापानि ब्रह्महत्यासमानि च}
{कृष्णा ह जागरात्तानि विलयं यान्ति खण्डशः}% ॥७१॥

\twolineshloka
{एकतः क्रतवः सर्वे समाप्तवरदक्षिणाः}
{एकतो देवदेवस्य जागरः कृष्णवल्लभः}% ॥७२॥

\twolineshloka
{तत्र काशी पुष्करं च प्रयागं नैमिषं गया}
{शालग्राममहाक्षेत्रमर्बुदारण्यमेव च}% ॥७३॥

\twolineshloka
{पौष्करं मथुरा तत्र सर्वतीर्थानि चैव हि}
{यज्ञा वेदाश्च चत्वारो व्रजन्ति हरिजागरम्}% ॥७४॥

\twolineshloka
{गङ्गा सरस्वती तापी यमुना च शतद्रुका}
{चन्द्रभागा वितस्ता च नद्यः सर्वास्तु तत्र वै}% ॥७५॥

\twolineshloka
{सरांसि च ह्रदाः सर्वे समुद्राः सर्व एव हि}
{एकादश्यां द्विजश्रेष्ठ गच्छन्ति कृष्णजागरम्}% ॥७६॥

\twolineshloka
{स्पृहणीया हि देवानां ये नराः कृष्णजागरे}
{नृत्यन्ति गीतं कुर्वन्ति वीणावाद्यप्रहर्षिताः}% ॥७७॥

\twolineshloka
{एवं जागरणं कृत्वा सम्पूज्य च महाहरिम्}
{द्वादश्यां पारणं कृत्वा स्वशक्त्या वैष्णवैः सह}% ॥७८॥

\uvacha{महादेव उवाच}

\twolineshloka
{शृणु ब्रह्मन्प्रवक्ष्यामि द्वादशीमाहात्म्यमुत्तमम्}
{द्वादशी तु सदा ज्ञेया पुत्रदा मोक्षदायिनी}% ॥७९॥

\twolineshloka
{प्रातः स्नात्वा हरिं पूज्य उपवासं समर्पयेत्}
{अज्ञानतिमिरान्धस्य व्रतेनानेन केशव}% ॥८०॥

\twolineshloka
{प्रसीद सुमुखो भूत्वा ज्ञानदृष्टिप्रदो भव}
{पारणं च ततः कुर्याद्यथासम्भवमग्रतः}% ॥८१॥

\twolineshloka
{अत ऊर्ध्वं यथेष्टं तु कारयेच्च यथाविधि}
{यदा तु द्वादशी स्वल्पा पारणेन भवेद्दिवज}% ॥८२॥

\twolineshloka
{तदा रात्रौ तु कर्तव्यं पारणं मुक्तिमिच्छता}
{तदा न रात्रिदोषः स्यान्निषिद्धं न भवेत्क्वचित्}% ॥८३॥

\twolineshloka
{यदुक्तं निशि न स्नायान्महानिशि न भोजयेत्}
{तत्पूर्वपरयामाभ्यां दिनवत्कर्म कारयेत्}% ॥८४॥

\twolineshloka
{यदा भवति स्वल्पा तु द्वादशी पारणे दिने}
{उषःकाले  द्वयं कुर्यात्प्रातर्मध्याह्निकं तथा}% ॥८५॥

\twolineshloka
{द्वादशी साधिता येन नरेण भुवि सर्वदा}
{तस्य पुण्यमहं वक्तुं न समर्थो विशेषतः}% ॥८६॥

\twolineshloka
{साधयित्वाखिलान्कामान्प्राप्नुयुश्च महाजनाः}
{अम्बरीषादयः सर्वे ये भक्ता भुवि विश्रुताः}% ॥८७॥

\twolineshloka
{द्वादशीं साधयित्वा तु ते गता विष्णुसद्मनि}
{सत्यं सत्यं पुनः सत्यं यदुक्तं तु मया तव}% ॥८८॥

\twolineshloka
{नास्ति विष्णुसमो देवो न तिथिर्द्वादशीसमा}
{अत्र दत्तं च भुक्तं च तथा पूजादिकं च यत्}% ॥८९॥

\twolineshloka
{तत्सर्वं पूर्णतां याति पूजिते माधवे सति}
{किं पुनर्बहुनोक्तेन भक्तानां वल्लभो हरिः}% ॥९०॥

\twolineshloka
{प्रददात्यखिलान्कामान्यावदाभूतसम्प्लवम्}
{द्वादश्यां चैव यद्दत्तं तत्सर्वं सफलं भवेत्}% ॥९१॥

\twolineshloka
{कुरुक्षेत्रेषु यद्दत्तं निष्फलं नैव जायते}
{तद्वच्च द्वादशीदत्तं भवेद्देवर्षिसत्तम}% ॥९२॥

॥इति श्रीपाद्मे महापुराणे पञ्चपञ्चाशत्साहस्र्यां संहितायामुत्तरखण्डे उमापतिनारदसंवादान्तर्गतकृष्णयुधिष्ठिरसंवादे द्वादशी-एकादशी-जागरणमहिमावर्णनं नाम नामैकोनचत्वारिंशोऽध्यायः॥३९॥


\hyperref[sec:ekadashi_mahatmyam_padma_puranam]{\closesub}
\clearpage

\sect{मार्गशीर्ष-कृष्ण-उत्पन्ना-एकादशी-कृत-मुरवधः}
\label{sec:padma-margashirsha-krishna-utpanna}

\uvacha{महादेव उवाच}

\twolineshloka
{एकस्मिन्समये पुत्र गतोऽहं विष्णुसन्निधौ}
{तत्र पृष्टं मया पूर्वं माहात्म्यं द्वादशीभवम्}% ॥१॥

\uvacha{नारद उवाच}

\twolineshloka
{कीदृशी स्यान्महादेव महाद्वादशिका परा}
{तस्या व्रते फलं कीदृग्वद सर्वेश्वर प्रभो}% ॥२॥

\uvacha{शिव उवाच}

\twolineshloka
{इयमेकादशी ब्रह्मन्महापुण्यफलप्रदा}
{ऋक्षयोगैश्च संयुक्ता कर्तव्या मुनिसत्तमैः}% ॥३॥

\twolineshloka
{जया च विजया चैव जयन्ती पापनाशिनी}
{सर्वपापहराश्चैताः कर्तव्याः फलकाङ्क्षिभिः}% ॥४॥

\twolineshloka
{एकादश्यां यदा ऋक्षं शुक्लपक्षे पुनर्वसुः}
{नाम्ना सा च जया ख्याता तिथीनामुत्तमा तिथिः}% ॥५॥

\twolineshloka
{तामुपोष्य नरः पापान्मुच्यते नात्र संशयः}
{यदा च शुक्लद्वादश्यां नक्षत्रं श्रवणं भवेत्}% ॥६॥

\twolineshloka
{विजया सा समाख्याता तिथीनामुत्तमा तिथिः}
{सहस्रगुणितं दानं यस्यां वै विप्र भोजनम्}% ॥७॥

\twolineshloka
{होमस्तथोपवासश्च सहस्राधिफलप्रदः}
{यदा च शुक्लद्वादश्यां प्राजापत्यं हि जायते}% ॥८॥

\twolineshloka
{जयन्ती नाम सा प्रोक्ता सर्वपापहरा तिथिः}
{सप्तजन्मकृतं पापं स्वल्पं वा यदि वा बहु}% ॥९॥

\twolineshloka
{प्रक्षालयति गोविन्दस्तस्यामभ्यर्चितो ध्रुवम्}
{यदा वै शुक्लद्वादश्यां पुष्यं भवति कर्हिचित्}% ॥१०॥

\twolineshloka
{तदा तु सा महापुण्या भविता पापनाशिनी}
{यो ददाति तिलप्रस्थं नित्यं संवत्सरं प्रति}% ॥११॥

\twolineshloka
{उपवासं च यस्तस्यां करोत्येतत्समं स्मृतम्}
{तस्यां जगत्पतिर्देवस्तुष्टः सर्वेश्वरो हरिः}% ॥१२॥

\twolineshloka
{प्रत्यक्षतां प्रयात्येव तत्रानन्तफलं स्मृतम्}
{सगरेण ककुत्स्थेन नहुषेण च साधितः}% ॥१३॥

\twolineshloka
{तस्यामाराधितः कृष्णो दत्तवानखिलं भुवि}
{वाचिकान्मानसाद्वाऽपि कायजाच्च विशेषतः}% ॥१४॥

\twolineshloka
{सप्तजन्मकृतात्पापान्मुच्यते नात्र संशयः}
{तामेकां समुपोष्याथ पुण्यनक्षत्रसंयुताम्}% ॥१५॥

\twolineshloka
{एकादशीसहस्रस्य फलं प्राप्नोति मानवः}
{स्नानं दानं जपो होमः स्वाध्यायो देवतार्चनम्}% ॥१६॥

\twolineshloka
{यत्तस्यां क्रियते किञ्चित्तदक्षयफलं स्मृतम्}
{तस्मादेषा प्रकर्तव्या यत्नेन फलकाङ्क्षिभिः}% ॥१७॥

\twolineshloka
{पञ्चमेनाश्वमेधेन यदि स्नातो युधिष्ठिरः}
{पर्यपृच्छत धर्मात्मा कृष्णं यदुकुलोद्वहम्}% ॥१८॥

\uvacha{युधिष्ठि रउवाच}

\twolineshloka
{उपवासस्य नक्तस्य त्वेकभुक्तस्य मे प्रभो}
{किं पुण्यं किं फलं तस्य ब्रूहि सर्वं जनार्दन}% ॥१९॥

\uvacha{श्रीभगवानुवाच}
\onelineshloka*
{हेमन्ते चैव सम्प्राप्ते मासे मार्गे च शोभने}

\twolineshloka
{कृष्णपक्षे च या पार्थ द्वादशी तामुपोषयेत्}
{दशम्यां चैकभक्तश्च शुद्धचित्तो दृढव्रतः}% ॥२०॥

\twolineshloka
{नक्तं चैव तथा ज्ञात्वा दशम्यां नियतः सदा}
{दिवसस्याष्टमे भागे मन्दीभूते दिवाकरे}% ॥२१॥

\twolineshloka
{नक्तं च तद्विजानीयान्न नक्तं निशिभोजनम्}
{नक्षत्रदर्शनान्नक्तं गृहस्थस्य विधीयते}% ॥२२॥

\twolineshloka
{यतेर्दिनाष्टमे भागे रात्रौ तस्य निषेधनम्}
{ततः प्रभातसमये कृत्वा च नियमं व्रती}% ॥२३॥

\twolineshloka
{मध्याह्ने च तथा पार्थ स्नानं शुचिः समाचरेत्}
{अधमं कूपके स्नानं वाप्यां स्नानं च मध्यमम्}% ॥२४॥

\twolineshloka
{तडागे चोत्तमं स्नानं नद्यां स्नानं ततः परम्}
{पीड्यन्ते जन्तवो यत्र जलमध्ये व्यवस्थिते}% ॥२५॥

\twolineshloka
{तत्र स्नाने कृते पार्थ पापं पुण्यं समं भवेत्}
{गृहे चैवोत्तमं स्नानं जलं चैव विशोधयेत्}% ॥२६॥

\twolineshloka
{तस्मात्तु पाण्डवश्रेष्ठ गृहे स्नानं समाचरेत्}
{अश्वक्रान्ते रथक्रान्ते विष्णुक्रान्ते वसुन्धरे}% ॥२७॥

\twolineshloka
{मृत्तिके हर मे पापं यन्मया पूर्वसञ्चितम्}
{क्रोधलोभौ परित्यज्य चैकचित्तो दृढव्रतः}% ॥२८॥

\twolineshloka
{नालापेतान्त्यजं चैव तथा पाखण्डिनो नरान्}
{मिथ्यावादरतांश्चैव तथा ब्राह्मणनिन्दकान्}% ॥२९॥

\twolineshloka
{अन्यांश्चैव दुराचारानगम्यागमनेरतान्}
{परद्रव्यापहारांश्च परदाराभिगामिनः}% ॥३०॥

\twolineshloka
{केशवं पूजयित्वा तु नैवेद्यं तत्र कारयेत्}
{दीपं दद्याद्गृहे तत्र भक्तियुक्तेन चेतसा}% ॥३१॥

\twolineshloka
{तद्दिने वर्जयेत्पार्थ निद्रां चैव तु मैथुनम्}
{धर्मशास्त्रविनोदेन दिनं सर्वं निवारयेत्}% ॥३२॥

\twolineshloka
{रात्रौ जागरणं कृत्वा भक्तियुक्तो नृपोत्तम}
{विप्रेभ्यो दक्षिणां दद्यात्प्रणिपत्य क्षमापयेत्}% ॥३३॥

\twolineshloka
{यथा कृष्णा तथा शुक्ला विधिनैवं प्रकारयेत्}
{एकादशीं द्विजः पार्थ विभेदं नैव कारयेत्}% ॥३४॥

\twolineshloka
{एवं हि कुरुते यस्तु शृणु तस्य हि यत्फलम्}
{शङ्खोद्धारे नरः स्नात्वा दृष्ट्वा देवं गदाधरम्}% ॥३५॥

\twolineshloka
{एकादश्युपवासस्य कलां नार्हति षोडशीम्}
{सङ्क्रान्तिषु चतुर्लक्षं यो ददाति नृपोत्तम}% ॥३६॥

\twolineshloka
{एकादश्युपवासस्य कलां नार्हति षोडशीम्}
{प्रभासक्षेत्रे यत्पुण्यं ग्रहणे चन्द्र सूर्ययोः}% ॥३७॥

\twolineshloka
{तत्फलं जायते नूनमेकादश्युपवासिनः}
{केदारे चोदकं पीत्वा पुनर्जन्म न विद्यते}% ॥३८॥

\twolineshloka
{तथा चैकादशी पार्थ गर्भवासक्षयङ्करी}
{अश्वमेधस्य यज्ञस्य पृथिव्यां यत्फलं लभेत्}% ॥३९॥

\twolineshloka
{तस्माच्छतगुणं पुण्यमेकादश्युपवासिनः}
{तपस्विनो गृहे यस्य भुञ्जते च द्विजोत्तमाः}% ॥४०॥

\twolineshloka
{तत्फलं जायते नूनमेकादश्युपवासिनः}
{गोसहस्रेण यत्पुण्यं दत्वा वेदान्तपारगे}% ॥४१॥

\twolineshloka
{तस्माच्छतगुणं पुण्यमेकादश्युपवासिनः}
{येषां देहे त्रयो देवा ब्रह्मविष्णुमहेश्वराः}% ॥४२॥

\twolineshloka
{वसन्ति तेषां ते तुल्या एकादश्युपवासिनः}
{ते नराः पुण्यकर्माणो ये भक्ता हरिपूजकाः}% ॥४३॥

\twolineshloka
{एकादशीव्रतस्यापि पुण्यसङ्ख्या न विद्यते}
{एतत्पुण्यं भवेत्तस्य यत्सुरैरपि दुर्ल्लभम्}% ॥४४॥

\twolineshloka
{एतस्मादर्धपुण्यं तु प्राप्यते नक्तभोजनात्}
{नक्तस्यार्धं भवेत्पुण्यमेकभक्तेन देहिनाम्}% ॥४५॥

\twolineshloka
{तावद्गर्जन्ति तीर्थानि दानानि नियमानि च}
{यावन्नोपोषयेज्जन्तुर्वासरं विष्णुवल्लभम्}% ॥४६॥

\twolineshloka
{तस्मात्त्वं पाण्डवश्रेष्ठ व्रतमेतत्समाचर}
{पुण्यसङ्ख्यां न जानामि यत्त्वं पृच्छसि पाण्डव}% ॥४७॥

\twolineshloka
{एतद्धि कथितं पार्थ यद्गोप्यं व्रतमुत्तमम्}
{एकादशीसमं नास्ति कृत्वा यज्ञसहस्रकम्}% ॥४८॥

\uvacha{युधिष्ठिर उवाच}

\twolineshloka
{उत्पन्ना सा कथं देव पुण्या चैकादशी तिथिः}
{कथं पवित्रा विश्वेऽस्मिन्कथं वै देवताप्रिया}% ॥४९॥

\uvacha{श्रीभगवानुवाच}

\twolineshloka
{पुरा कृतयुगे पार्थ मुरनामेति दानवः}
{अत्यद्भुतो महारौद्र सर्वदेवभयङ्करः}% ॥५०॥

\twolineshloka
{इन्द्रोऽपि निर्जितस्तेन सर्वदेवास्तथा नृप}
{महासुरेण तेनैव मृत्युना च दुरात्मना}% ॥५१॥

\twolineshloka
{स्वर्गान्निराकृतास्तेन विचरन्ति महीतले}
{सशङ्का भयभीताश्च सर्वे गत्वा महेश्वरम्}% ॥५२॥

\twolineshloka
{इन्द्रेण कथितं सर्वमीश्वरस्यापि चाग्रतः}
{स्वर्गलोकपरिभ्रष्टा विचरन्ति  महीतले}% ॥५३॥

\twolineshloka
{मर्त्येषु संस्थिता देवा न शोभन्ते महेश्वर}
{उपायं ब्रूहि मे देव ह्यमरा यान्ति कां गतिम्}% ॥५४॥

\uvacha{महादेव उवाच}

\twolineshloka
{देवराजसुरश्रेष्ठयत्रास्तेगरुडध्वजः}
{शरण्यश्च जगन्नाथः परित्राता परायणः}% ॥५५॥

\twolineshloka
{तत्र गच्छ सुरश्रेष्ठ स वो रक्षां विधास्यति}
{ईश्वरस्य वचः श्रुत्वा देवराजो महामतिः॥५६                          ॥}

\twolineshloka
{त्रिदशैः सहितो यत्र गतस्तत्र युधिष्ठिर}
{जलमध्ये प्रसुप्तं तं दृष्ट्वा देवं गदाधरम्}% ॥५७॥

\onelineshloka
{कृताञ्जलिपुटो भूत्वा इन्द्रः स्तोत्रमुदीरयत्}% ॥५८॥

\uvacha{इन्द्र उवाच}

\twolineshloka
{ॐनमो देवदेवेश देवदानववन्दित}
{दैत्यारे पुण्डरीकाक्ष त्राहि नो मधुसूदन}% ॥५९॥

\twolineshloka
{सुराः सर्वे समायाता भयभीताश्च दानवात्}
{शरणं त्वां जगन्नाथ त्राहि मां भक्तवत्सल}% ॥६०॥

\twolineshloka
{त्राहि नो देवदेवेश त्राहि त्राहि जनार्दन}
{त्राहि वै पुण्डरीकाक्ष दानवानां विनाशक}% ॥६१॥

\twolineshloka
{त्वत्समीपं गताः सर्वे त्वमेव शरणं प्रभो}
{चशरणागतानां देवानां सहायं कुरु वै प्रभो}% ॥६२॥

\twolineshloka
{त्वं पतिस्त्वं मतिर्देव त्वं कर्ता त्वं च कारणम्}
{त्वं माता सर्वलोकानां त्वमेव जगतः पिता}% ॥६३॥

\twolineshloka
{भगवन्देवदेवेश शरणागतवत्सल}
{शरणं तव चायाता भयभीताश्च देवताः}% ॥६४॥


\threelineshloka
{देवता निर्जिताः सर्वाः स्वर्गभ्रष्टाः कृताः प्रभो}
{अत्युग्रेण हि दैत्येन मुरनाम्ना महौजसा}
{इन्द्रस्य वचनं श्रुत्वा विष्णुर्वचनमब्रवीत्}% ॥६५॥

\uvacha{श्रीभगवानुवाच}

\onelineshloka*
{कीदृशो दानवः शक्र किं रूपं कीदृशं बलम्}

\twolineshloka
{क्व स्थानं तस्य दुष्टस्य किं वीर्यं कः पराक्रमः}
{किं वरं तस्य दुष्टस्य ममाख्याहि महामते}% ॥६६॥

\uvacha{इन्द्र उवाच}

\twolineshloka
{पूर्वं बभूव देवेश ब्रह्मवंशसमुद्भवः}
{तालजङ्घस्तु नाम्ना च अत्युग्रोऽपि महासुरः}% ॥६७॥

\twolineshloka
{तस्य पुत्रो हि विख्यातो मुरनामेति दानवः}
{अत्युत्कटो महावीर्यो देवतानां भयङ्करः}% ॥६८॥

\twolineshloka
{पुरी चन्द्रावती नाम्ना तत्र स्थाने वसत्यसौ}
{निर्जिता देवताः सर्वा स्वर्गात्तेन विवासिताः}% ॥६९॥

\twolineshloka
{इन्द्रोऽन्यो रोपितस्तेन वातश्चैव हुताशनः}
{चन्द्रसूर्यो कृतौ चान्यौ वायुर्वरुण एव च}% ॥७०॥

\twolineshloka
{सर्वमात्मकृतं तेन सत्यं सत्यं जनार्दन}
{देवलोकः कृतस्तेन सर्वस्थानविवर्जितः}% ॥७१॥

\twolineshloka
{तस्य तद्वचनं श्रुत्वा कोपमानो जनार्दनः}
{हनिष्ये दानवं दुष्टं देवतानां भयङ्करम्}% ॥७२॥

\twolineshloka
{त्रिदशैः सहितो देवो गतश्चन्द्रावतीं पुरीम्}
{दृष्टो देवैस्तु दैत्येन्द्रो गर्जमानः पुनः पुनः}% ॥७३॥

\twolineshloka
{तेन सर्वे जिता देव गताश्चैव दिशो दश}
{हरिं निरीक्ष्य प्रोवाच तिष्ठतिष्ठेति दानवः}% ॥७४॥

\onelineshloka*
{भगवानब्रवीत्तं च क्रोधसंरक्तलोचनः}

\uvacha{श्रीभगवानुवाच}

\onelineshloka
{रे दानव दुराचार मम बाहुं निरीक्षय}% ॥७५॥

\twolineshloka
{ततस्ते सम्मुखाः सर्वे विष्णुना दुष्टदानवाः}
{हता बाणैः पुनर्दिव्यैर्जाताश्च भयविह्वलाः}% ॥७६॥

\twolineshloka
{चक्रं मुक्तं च कृष्णेन दैत्यसैन्येषु पाण्डव}
{तेनच्छिन्नास्तु शतशो बहवो निधनं गताः}% ॥७७॥

\twolineshloka
{एकोपि दानवस्तत्र युध्यमानो मुहुर्मुहुः}
{नष्टाः सर्वे सुरास्तेन निर्जितो मधुसूदनः}% ॥७८॥

\twolineshloka
{निर्जितं तेन दैत्येन बाहुयुद्धमजायत}
{बाहुयुद्धं कृतं तेन दिव्यं वर्षसहस्रकम्}% ॥७९॥

\twolineshloka
{विष्णुश्चिन्तां प्रपन्नश्च नष्टाः सर्वाश्च देवताः}
{विष्णुश्च निर्जितस्तेन गतो बदरिकाश्रमम्}% ॥८०॥

\twolineshloka
{गुहा सिंहावती नाम तत्र सुप्तो जनार्दनः}
{योजनद्वादशवती एकद्वारा च पाण्डव}% ॥८१॥

\twolineshloka
{तस्यां विष्टः प्रसुप्तश्च दानवो हन्तुमुद्यतः}
{महायुद्धेन तेनैव श्रान्तोऽसौ योगमायया}% ॥८२॥

\twolineshloka
{दानवः पृष्ठतो लग्नो प्रविष्टः स तदा गुहाम्}
{प्रसुप्तं तत्र मां दृष्ट्वा दानवो हर्षमागतः}% ॥८३॥

\twolineshloka
{इत्थं मां निर्जितं मत्वा प्रविष्टं शङ्कया हरिम्}
{निःसन्देहं हनिष्यामि दानवानां भयङ्करम्}% ॥८४॥

\twolineshloka
{निर्गता कन्यका तत्र विष्णुदेहाद्युधिष्ठिर}
{रूपवती सुसौभाग्या दिव्यप्रहरणायुधा}% ॥८५॥

\twolineshloka
{तस्य तेजोंऽशसम्भूता महाबलपराक्रमा}
{दृष्टा सा दानवेन्द्रेण मुरनाम्ना धनञ्जय}% ॥८६॥

\twolineshloka
{युद्धं समाहितं तेन स्त्रिया चैव प्रयाचितम्}
{कन्यका युध्यते तत्र सर्वयुद्धविशारदा}% ॥८७॥

\twolineshloka
{हुङ्कारैर्भस्मसाज्जातो मुर नामा महासुरः}
{निहते दानवे तस्मिंस्तत्र देवस्त्वबुध्यत}% ॥८८॥

\twolineshloka
{पतितं दानवं दृष्ट्वा ततो विस्मयमागतः}
{केनायं च हतो रौद्रो ह्यत्युग्रो मम शात्रवः}% ॥८९॥

\onelineshloka*
{अत्युग्रं च कृतं कर्म मम कारुण्यतः कृतम्}

\uvacha{कन्योवाच}
\onelineshloka
{तेन देवाश्च गन्धर्वा सयक्षोरगराक्षसाः}% ॥९०॥

\twolineshloka
{इन्द्राद्याः सकला जित्वा स्वर्गाच्चैव निराकृताः}
{हरिः सुप्तो मया दृष्टो मुरः पृष्ठे समागतः}% ॥९१॥

\twolineshloka
{संहरिष्यति त्रैलोक्यं सुप्ते चैव जर्नादने}
{तस्यास्तद्वचनं श्रुत्वा विष्णुर्वचनमब्रवीत्}% ॥९२॥
\onelineshloka*
{अहं च निर्जितो येन कथं सोऽपि त्वया जितः}

\uvacha{एकादश्युवाच}
\onelineshloka
{त्वत्प्रसादाच्च भो स्वामिन्महादैत्यो मया हतः}% ॥९३॥

\uvacha{श्रीभगवानुवाच}

\threelineshloka
{आनन्दं त्रिषु लोकेषु मुनयो देवता गताः}
{ब्रूहि त्वं च मया भद्रे यत्ते मनसि रोचते}
{ददामि च न सन्देहो यत्सुरैरपि दुर्ल्लभम्}% ॥९४॥

\uvacha{एकादश्युवाच}

\twolineshloka
{यदि तुष्टोऽसि मे देव सत्यमुक्तं जनार्दन}
{वरमेकं तु वाञ्च्छामि हृदये च जगत्पते}% ॥९५॥

\twolineshloka
{प्रार्थयामि च देवेश ईप्सितं च मया प्रभो}
{यदि सत्यं जगन्नाथ तिस्रो वाचो ददासि मे}% ॥९६॥

\uvacha{श्रीभगवानुवाच}

\twolineshloka
{सत्यं सत्यं मया प्रोक्तं अवश्यं तव सुव्रते}
{तिस्रो वाचो मया दत्ता न चावाक्यं भवेदिह}% ॥९७॥

\uvacha{एकादश्युवाच}

\twolineshloka
{त्रिभुवनेषु च देवेश चतुर्युगेषु साम्प्रतम्}
{त्रिषु लोकेषु सर्वत्र तादृशं कुरु मे प्रभो}% ॥९८॥

\twolineshloka
{सर्वतीर्थप्रधानं हि सर्वविघ्नविनाशिनी}
{सर्वसिद्धिकरी देवी त्वत्प्रसादाद्भवाम्यहम्}% ॥९९॥

\twolineshloka
{मामुपोष्यन्ति ये भक्त्या तव भक्त्या जनार्दन}
{सर्वसिद्धिर्भवेत्तेषां यदि तुष्टोऽसि मे प्रभो}% ॥१००॥

\twolineshloka
{उपवासं च नक्तं च एकभक्तं करोति च}
{तस्य वित्तं च धर्मं च मोक्षं वै देहि माधव}% ॥१०१॥

\uvacha{विष्णुरुवाच}

\twolineshloka
{यत्त्वं वदसि कल्याणि तत्सर्वं च भविष्यति}
{सर्वान्मनोरथान्भद्रे दास्यसि त्वं च नान्यथा}% ॥१०२॥

\twolineshloka
{मम भक्ताश्च ये लोके ये च भक्तास्तु कार्त्तिके}
{चतुर्युगेषु विख्यातास्त्रिषु लोकेषु वै प्रभो}% ॥१०३॥

\twolineshloka
{त्वां च शक्तिमहं मन्ये एकादशीव्रतस्थिताः}
{मम पूजां करिष्यन्ति मोक्षगास्ते न संशयः}% ॥१०४॥

\threelineshloka
{तृतीया चाष्टमी चैव नवमी च चतुर्दशी}
{एकादशी विशेषेण तिथिरेषा हरिप्रिया}
{सर्वतीर्थाधिकं पुण्यं सत्यं सत्यं न संशयः}% ॥१०५॥

\twolineshloka
{इदं दत्त्वा वरं तस्यै तिस्रो वाचो न संशयः}
{हृष्टा पुष्टा च सञ्जाता एकादशी महाव्रता}% ॥१०६॥

\twolineshloka
{शत्रुं हंसि परां तस्य ददासि परमां गतिम्}
{त्वं हंसि सर्वविघ्नानि सर्वसिद्धिवरप्रदा}% ॥१०७॥

\twolineshloka
{उभयोः पक्षयोः पार्थ तुल्या एकादशी शुभा}
{न शुक्ला नैव कृष्णा च विभेदं नैव कारयेत्}% ॥१०८॥

\twolineshloka
{विभेदो नैव कर्तव्यः समस्तव्रतकारिभिः}
{दिवा वा यदि वा रात्रौ शृणोति भक्तितत्परः}% ॥१०९॥

\twolineshloka
{तिथिरेका भवेत्सर्वा पक्षयोरुभयोरपि}
{उभयैकादशी स्वल्पा ह्यन्ते चैव त्रयोदशी}% ॥११०॥

\twolineshloka
{मध्ये तु द्वादशी पूर्णा त्रिस्पृशा सा हरिप्रिया}
{एकामुपोषयेत्तां वै सहस्रैकादशीफलम्}% ॥१११॥

\twolineshloka
{सहस्रगुणितं ह्येवं द्वादश्यां पारणे कृते}
{अष्टम्येकादशी षष्ठी तृतीया च चतुर्दशी}% ॥११२॥

\twolineshloka
{पूर्वविद्धा न कर्तव्या परविद्धामुपोषयेत्}
{एकादशी ह्यहोरात्रं प्रभाते घटिका भवेत्}% ॥११३॥

\twolineshloka
{सा तिथिः परिहर्तव्या उपोष्या द्वादशीयुता}
{एवंविधा मया प्रोक्ता पक्षयोरुभयोरपि}% ॥११४॥

\twolineshloka
{एकादश्यां प्रकुर्वीत ह्युपवासं न संशयः}
{ते यान्ति वैष्णवं स्थानं यत्रास्ते गरुडध्वजः}% ॥११५॥

\twolineshloka
{धन्यास्ते मानवा लोके विष्णुभक्तिपरायणाः}
{एकादश्याश्च माहात्म्यं सर्वकालेषु यः पठेत्}% ॥११६॥

\twolineshloka
{गोसहस्रफलं सोऽपि पुण्यं प्राप्नोति मानवः}
{दिवा वा यदि वा रात्रौ ये वै शृण्वन्ति भक्तितः}% ॥११७॥

\threelineshloka
{ब्रह्महत्यादिपापेभ्यो मुच्यन्ते नात्र संशयः}
{विष्णुधर्मसमं नास्ति गीतार्थं च नृपोत्तम}
{एकादशीसमं नास्ति व्रतं पापप्रणाशनम्}% ॥११८॥

॥इति श्रीपाद्मे महापुराणे पञ्चपञ्चाशत्साहस्र्यां संहितायामुत्तरखण्डे उमापतिनारदसंवादान्तर्गतकृष्णयुधिष्ठिरसंवादे मार्गशीर्ष-कृष्ण-उत्पन्ना-एकादशी-कृत-मुरवधो नाम चत्वारिंशोऽध्यायः॥४०॥


\hyperref[sec:ekadashi_mahatmyam_padma_puranam]{\closesub}
\clearpage

\sect{मार्गशीर्ष-शुक्ल-मोक्षदा-एकादशी-माहात्म्यम्}
\label{sec:padma-margashirsha-shukla-mokshada}


\uvacha{युधिष्ठिर उवाच}

\twolineshloka
{वन्दे विष्णुं विभुं साक्षाल्लोकत्रयसुखावहम्}
{विश्वेशं विश्वकर्तारं पुराणं पुरुषोत्तमम्}% ॥१॥

\twolineshloka
{पृच्छामि देवदेवेश संशयोऽस्ति महान्मम}
{लोकानां च हितार्थाय पापानां क्षयहेतवे}% ॥२॥

\threelineshloka
{मार्गशीर्षेसिते पक्षे भवेदेकादशी तु या}
{किं नाम को विधिस्तस्याः को देवस्तत्र पूज्यते}
{एतदाचक्ष्व मे स्वामिन्विस्तरेण यथातथम्}% ॥३॥

\uvacha{श्रीकृष्ण उवाच}

\twolineshloka
{साधु पृष्टं त्वया राजन्साधु ते विमलं यशः}
{कथयिष्यामि राजेन्द्र हरिवासरमुत्तमम्}% ॥४॥

\twolineshloka
{उत्पन्ना चासिते पक्षे द्वादशी मम वल्लभा}
{मार्गशीर्षोत्पत्तिरिति मम देह समुद्भवा}% ॥५॥

\twolineshloka
{सुरासुरवधार्थाय ह्युत्पन्ना भरतर्षभ}
{कथिता च मया सा वै तवाग्रे राजसत्तम}% ॥६॥

\twolineshloka
{पूर्वा चैकादशी राजंस्त्रैलोक्ये सचराचरे}
{मार्गशीर्षे सिते पक्षे उत्पत्तिरिति नामतः}% ॥७॥

\twolineshloka
{अतः परं प्रवक्ष्यामि मार्गशीर्षे सिता तु या}
{यस्याः श्रवणमात्रेण वाजपेयफलं लभेत्}% ॥८॥

\twolineshloka
{मोक्षा नामेति सा प्रोक्ता सर्वपापहरा परा}
{देवं दामोदरं राजन्पूजयेच्च प्रयत्नतः}% ॥९॥

\twolineshloka
{तुलस्या मञ्जरीभिश्च धूपैर्दीपैः प्रयत्नतः}
{पूर्वेण विधिना चैव दशम्येकादशी तथा}% ॥१०॥

\twolineshloka
{मोक्षा चैकादशी नाम्ना महापातकनाशिनी}
{रात्रौ जागरणं कार्यं नृत्यगीतस्तवैर्मम}% ॥११॥

\twolineshloka
{शृणु राजन्प्रवक्ष्यामि दिव्यां पौराणिकीकथाम्}
{यस्याः श्रवणमात्रेण सर्वपापक्षयो भवेत्}% ॥१२॥

\twolineshloka
{अधोयोनिगतश्चैव पितरो यस्य पापतः}
{अस्याश्च पुण्यदानेन मोक्षं यान्ति न संशयः}% ॥१३॥

\twolineshloka
{चम्पके नगरे रम्ये वैष्णवैश्च विभूषिते}
{वैखानसो नाम नृपः पुत्रवत्पालयेत्प्रजाः}% ॥१४॥

\twolineshloka
{वसन्ति बहवो विप्रा वेदवेदाङ्गपारगाः}
{ऋद्धिमत्यः प्रजास्तस्य राज्ञो वैखानसस्य हि}% ॥१५॥

\twolineshloka
{एवं राज्यं प्रकुर्वाणो रात्रौ स्वप्नस्य मध्यतः}
{स्वकीयपितरो दृष्ट्वा अधोयोनिगता नृप}% ॥१६॥

\twolineshloka
{एवं दृष्ट्वा च तान्सर्वान्विस्मयाविष्टमानसः}
{कथयामास वृत्तान्तं स्वप्नजातं द्विजान्प्रति}% ॥१७॥

\uvacha{राजोवाच}

\twolineshloka
{मया स्वपितरो दृष्ट्वा नरकोपगता द्विजाः}
{तारयेति तनूजत्वमस्मान्निरयसागरात्}% ॥१८॥

\twolineshloka
{एवं ब्रुवाणास्ते नूनं रोदमाना मुहुर्मुहुः}
{मया दृष्टा द्विजश्रेष्ठा एतस्माच्च न मे सुखम्}% ॥१९॥

\twolineshloka
{एतद्राज्यं मम महत्सुखदायि न विद्यते}
{अश्वा गजास्तथा सर्वे रोचन्ते मे न भो द्विजाः}% ॥२०॥

\twolineshloka
{न दारा न सुता मह्यं रोचन्ते द्विजसत्तमाः}
{किं करोमि क्व गच्छामि हृदयं मेऽवरुध्यते}% ॥२१॥

\twolineshloka
{तद्व्रतं तं तपोयोगं येनैव मम पूर्वजाः}
{मोक्षं प्रयान्ति सद्यो वै कथ्यतां च द्विजोत्तमाः}% ॥२२॥

\twolineshloka
{पुत्रे तु जीवितप्राये बलीयसि महात्मनि}
{पितास्ति नरके घोरे तस्य पुत्रस्य किं फलम्}% ॥२३॥

\uvacha{ब्राह्मणा ऊचुः}

\twolineshloka
{पर्वतस्य मुने राजन्निकटे चाश्रमो महान्}
{गम्यतां राजशार्दूल भूतं भव्यं विजानतः}% ॥२४॥

\twolineshloka
{तेषां श्रुत्वा ततो वाक्यं राजा वैखानसो महान्}
{जगाम चाशु तत्रैव चाश्रमं पर्वतस्य च}% ॥२५॥

\twolineshloka
{ब्राह्मणैर्वेष्टितो राजा राजभिश्च समन्वितः}
{आश्रमं विपुलं तस्य सम्प्राप्तो राजसत्तमः}% ॥२६॥

\twolineshloka
{तत्रर्ग्वेद यजुर्वेद सामाध्ययनकोविदैः}
{वेष्टितं मुनिभिश्चैव द्वितीयं ब्रह्मणो यथा}% ॥२७॥

\twolineshloka
{दृष्ट्वा तं मुनिशार्दूलं राजा वैखानसस्तथा}
{दण्डवत्प्रणतिं कृत्वा पस्पर्श चरणौ मुनेः}% ॥२८॥

\twolineshloka
{पप्रच्छ कुशलं तस्य सप्तस्वङ्गेष्वसौ मुनिः}
{राज्ये निष्कण्टकत्वं च राज्ञः सौख्यसमन्वितम्}% ॥२९॥

\uvacha{राजोवाच}

\twolineshloka
{तव प्रसादाद्भो स्वामिन्कुशलं मेऽङ्गसप्तसु}
{भक्ता ये विष्णुविप्रेषु कथं तेषां च विघ्नता}% ॥३०॥

\twolineshloka
{मया स्वपितरो दृष्टाः स्वप्ने च नरके स्थिताः}
{कस्य पुण्यस्य सामर्थ्यान्मोक्षं यान्ति द्विजोत्तम}% ॥३१॥

\twolineshloka
{अयं मे संशयः स्वामिन्प्रष्टुं तं त्वामुपागतः}
{उपायः कश्चिदेवात्र कर्तव्यो मुनिसत्तम}% ॥३२॥

\twolineshloka
{एतद्वाक्यं ततः श्रुत्वा पर्वतो मुनिसत्तमः}
{ध्यानस्तिमितनेत्रोऽभूत्तपस्वी ब्रह्मसन्निभः}% ॥३३॥

\twolineshloka
{मुहूर्तमेकं ध्यानस्थो भूपतिं प्रत्युवाच ह}
{ज्ञातं हि तव राजेन्द्र पितॄणां पूर्वचेष्टितम्}% ॥३४॥

\twolineshloka
{पूर्वजन्मनि तातस्ते क्षत्रियो राज्यगर्वितः}
{सपत्न्या ऋतुकाले तु राजधर्मप्रवर्तितः}% ॥३५॥

\twolineshloka
{गतो ग्रामे तु तां त्यक्त्वा कार्यार्थी निजयोषितम्}
{तव पित्रा तु तस्याश्च न दत्तमृतुदानकम्}% ॥३६॥

\twolineshloka
{तेन पापप्रभावेन नरके पितृभिः सह}
{पतितो राजशार्दूल तव तातः सुदारुणे}% ॥३७॥

\twolineshloka
{ततः पुनरुवाचेदं राजा वैखानसो मुनिम्}
{केन व्रतप्रभावेन मोक्षस्तेषां भवेन्मुने}% ॥३८॥

\uvacha{मुनिरुवाच}

\twolineshloka
{मार्गशीर्षे सिते पक्षे मोक्षा नामेति नामतः}
{सर्वैश्चेतद्व्रतं कार्यं पित्रे पुण्यं प्रदीयताम्}% ॥३९॥

\twolineshloka
{तेन पुण्यप्रभावेन मोक्षस्तेषां भविष्यति}
{सत्यमेतन्महाभाग ब्रह्मणो वचनं यथा}% ॥४०॥

\twolineshloka
{मुनेर्वाक्यं ततः श्रुत्वा स्वगृहं पुनरागतः}
{मार्गशीर्षस्तथा मासः प्राप्तः कष्टेन तेन वै}% ॥४१॥

\twolineshloka
{मुनेर्वाक्येन तत्कृत्वा व्रतं वैखानसो नृपः}
{अददत्पुण्यमखिलैः सार्धं पित्रे स भूमिपः}% ॥४२॥

\twolineshloka
{दत्ते पुण्यक्षणेनैव पुष्पवृष्टिरभूद्दिवि}
{वैखानसस्य तातो वै पितृभिर्मोक्षमाविशत्}% ॥४३॥

\twolineshloka
{राजानं चान्तरिक्षे स गिरं पुण्यामुवाच ह}
{स्वस्तिस्वस्तीति ते पुत्र प्रोच्य चैवं दिवं गतः}% ॥४४॥

\twolineshloka
{एवं यः कुरुते राजन्मोक्षामेकादशीं शुभाम्}
{तस्य पापानि नश्यन्ति मृतो मोक्षमवाप्नुयात्}% ॥४५॥

\twolineshloka
{नातः परतरा काचित्मोक्षदैकादशी भवेत्}
{पुण्यसङ्ख्यां न जानामि राजन्मे प्रियकृद्व्रतम्}% ॥४६॥

\twolineshloka
{चिन्तामणिसमा ह्येषा नृणां मोक्षप्रदायिनी}
{पठनाच्छ्रवणादस्या वाजपेयफलं लभेत्}% ॥४७॥

॥इति श्रीपाद्मे महापुराणे पञ्चपञ्चाशत्साहस्र्यां संहितायामुत्तरखण्डे उमापतिनारदसंवादान्तर्गतकृष्णयुधिष्ठिरसंवादे मार्गशीर्ष-शुक्ल-मोक्षदा-एकादशी-माहात्म्यं नाम एकचत्वारिंशोऽध्यायः॥४१॥


\hyperref[sec:ekadashi_mahatmyam_padma_puranam]{\closesub}
\clearpage

\sect{पौष-कृष्ण-सफला-एकादशी-माहात्म्यम्}
\label{sec:padma-pausha-krishna-saphala}


\uvacha{युधिष्ठिर उवाच}

\threelineshloka
{पौषस्य  कृष्णपक्षे तु किं नामैकादशी भवेत्}
{किं नाम को विधिस्तस्या एतद्विस्तरतो वद}
{एतदाख्याहि भो स्वामिन्को देवस्तत्र पूज्यते}% ॥१॥

\uvacha{श्रीकृष्ण उवाच}

\twolineshloka
{कथयिष्यामि राजेन्द्र भवतः स्नेहबन्धनात्}
{तुष्टिर्मे न तथा राजन्यज्ञैर्बहुलदक्षिणैः}% ॥२॥

\twolineshloka
{यथा मे तुष्टिरायाति ह्येकादशीव्रतेन वै}
{तस्मात्सर्वप्रयत्नेन कर्तव्यो हरिवासरः}% ॥३॥

\twolineshloka
{सत्यमेतन्न वै मिथ्या धर्मिष्ठानां विशारद}
{पौषस्य कृष्णपक्षे या सफला नाम नामतः}% ॥४॥

\twolineshloka
{तस्यां नारायणं देवं पूजयेच्च यथाविधि}
{पूर्वेणैव विधानेन कर्तव्यैकादशी शुभा}% ॥५॥

\twolineshloka
{नागानां च यथा शेषो पक्षिणां पन्नगाशनः}
{देवानां च यथा विष्णुर्द्विपदानां यथा द्विजः}% ॥६॥

\twolineshloka
{व्रतानां च यथा राजन्श्रेष्ठा चैकादशी तिथिः}
{ते जनाश्चैव भो राजन्पूज्या वै सर्वदा मम}% ॥७॥

\twolineshloka
{हरिवासरसंलीनाः कुर्वन्त्येकादशीव्रतम्}
{इहैव धनसंयुक्ता मृता मोक्षं लभन्ति ते}% ॥८॥

\twolineshloka
{सफलायां फलै राजन्पूजयेन्नामतो हरिम्}
{नारिकेलफलैश्चैव क्रमुकैर्बीजपूरकैः}% ॥९॥

\twolineshloka
{जम्बीरैर्दाडिमैश्चैव तथा धात्रीफलैः शुभैः}
{लवङ्गैर्बदरीभिश्च तथाम्रैश्च विशेषतः}% ॥१०॥

\twolineshloka
{पूजयेद्देवदेवेशं धूपदीपैस्तथैव च}
{सफलायां विशेषेण दीपदानं तु कारयेत्}% ॥११॥

\twolineshloka
{रात्रौ जागरणं चैव कर्तव्यं सह वैष्णवैः}
{यावन्निमेषो नेत्रस्य तावज्जागर्ति यो निशि}% ॥१२॥

\twolineshloka
{एकाग्रमनसो राजन्तस्य पुण्यं शृणुष्व हि}
{तत्समो नास्ति यज्ञो वै तीर्थं वा तत्समं नहि}% ॥१३॥

\twolineshloka
{सर्वव्रतानि राजेन्द्र कलां नार्हन्ति षोडशीम्}
{एवं वर्षसहस्राणि तपसा नैव यत्फलम्}% ॥१४॥

\twolineshloka
{तत्फलं समवाप्नोति यः करोति हि जागरम्}
{श्रूयतां राजशार्दूल सफलायाः कथा शुभा}% ॥१५॥

\twolineshloka
{चम्पावतीति विख्याता पुरी माहिष्मतस्य च}
{बभूवुस्तस्य राजर्षेः पुत्राः पञ्च कुमारकाः}% ॥१६॥

\twolineshloka
{तेषां मध्ये तु ज्येष्ठो वै महापापरतः सदा}
{परदाराभिचारी च वेश्यासक्तश्च मद्यपः}% ॥१७॥

\twolineshloka
{पितुर्द्रव्यं तु तेनैव गमितं पापकर्मणा}
{असद्वृत्तिरतो नित्यं भूसुराणां तु निन्दकः}% ॥१८॥

\twolineshloka
{वैष्णवानां च देवानां नित्यं निन्दां करोति सः}
{ईदृशं तु ततो दृष्ट्वा पुत्रं माहिष्मतो नृपः}% ॥१९॥

\twolineshloka
{नाम्ना तु लुम्पक इति राजपुत्रेषु चापठत्}
{राज्यान्निष्कासितस्तेन पित्रा चैव तु बन्धुभिः}% ॥२०॥

\twolineshloka
{स चैवं परिवारैस्तु त्यक्तश्च परिपन्थिवत्}
{लुम्पकोऽपि तथा त्यक्तश्चिन्तयामास वै तदा}% ॥२१॥

\twolineshloka
{त्यक्तोऽहं बान्धवैः पित्रा राज्यान्निष्कासितः किल}
{इति सञ्चिन्त्यमानोऽसौ मतिं पापे तदाकरोत्}% ॥२२॥

\twolineshloka
{मया गन्तव्यमेवास्तु दारुणे गहने वने}
{तस्माच्चैव पुरं सर्वं लुम्पयिष्यामि वै पितुः}% ॥२३॥

\twolineshloka
{इत्येवं स मतिं कृत्वा लुम्पको दैवयोगतः}
{निर्जगाम पुरात्तस्माद्गतोऽसौ गहने वने}% ॥२४॥

\twolineshloka
{जीवघातरतो नित्यं स्तेयद्यूतकलानिधिः}
{सर्वं च नगरं तेन मुषितं पापकर्मणा}% ॥२५॥

\twolineshloka
{स्तेयाभिगामी नगरे गृहीतः स निशाचरैः}
{उवाच तान्सुतोऽहं वै राज्ञो माहिष्मतस्य च}% ॥२६॥

\twolineshloka
{स तैर्मुक्तः पापकर्मा चागतो विपिनं पुनः}
{आमिषाभिरतो नित्यं तरोर्वै फलभक्षणे}% ॥२७॥

\twolineshloka
{आश्रमस्तस्य दुष्टस्य वासुदेवस्य सन्निधौ}
{अश्वत्थो वर्तते तत्र जीर्णश्च बहुवार्षिकः}% ॥२८॥

\twolineshloka
{देवत्वं तस्य वृक्षस्य विपिने वर्तते महत्}
{तत्रैव निवसंश्चैव लुम्पकः पापबुद्धिमान्}% ॥२९॥

\twolineshloka
{गते बहुतिथे काले कस्यचित्पुण्यसञ्चयात्}
{पौषस्य कृष्णपक्षे तु दशम्यां दिवसे तथा}% ॥३०॥

\twolineshloka
{फलानि भुक्त्वा वृक्षाणां रात्रौ शीतेन पीडितः}
{लुम्पको नाम पापिष्ठो वस्त्रहीनो गतेक्षणः}% ॥३१॥

\twolineshloka
{पीड्यमानोऽतिशीतेन हरिवृक्षसमीपतः}
{न निद्रा न सुखं तस्य गतप्राण इवाभवत्}% ॥३२॥

\twolineshloka
{आछाद्य दशनैरास्यमेवं नीता निशाखिला}
{भानूदयेऽपि पापिष्ठो न लेभे चेतनां तदा}% ॥३३॥

\twolineshloka
{लुम्पको गतसंज्ञस्तु सफलाया दिने तथा}
{रवौ मध्यङ्गते चैव संज्ञां लेभे स लुम्पकः}% ॥३४॥

\twolineshloka
{इतस्ततो विलोक्याथ व्यथितश्च तदासनात्}
{स्खलत्पद्भ्यां प्रचलितः खञ्जन्निव मुहुर्मुहुः}% ॥३५॥

\twolineshloka
{वनमध्ये गतस्तत्र क्षुत्क्षामः पीडितोऽभवत्}
{न शक्तिर्जीवघाते तु लुम्पकस्य दुरात्मनः}% ॥३६॥

\twolineshloka
{फलानि च तदा राजन्नाजहार स लुम्पकः}
{यावत्समागतस्तत्र तावदस्तं गतो रविः}% ॥३७॥

\twolineshloka
{किं भविष्यति तातेति स विलापं चकार ह}
{फलानि तत्र भूरीणि वृक्षमूले न्यवेशयत्}% ॥३८॥

\twolineshloka
{इत्युवाच फलैरेभिः श्रीपतिस्तुष्यतां हरिः}
{इत्युक्त्वा लुम्पकश्चैव निद्रां लेभे न वै निशि}% ॥३९॥

\twolineshloka
{रात्रौ जागरणं मेने विष्णुस्तस्य दुरात्मनः}
{फलैस्तु पूजनं मेने सफलायास्तथानघ}% ॥४०॥

\twolineshloka
{अकस्माद्व्रतमेवैतत्कृतवान्वै  स लुम्पकः}
{तेन पुण्यप्रभावेन प्राप्तं राज्यमकण्टकम्}% ॥४१॥

\twolineshloka
{सूर्यस्योदयनं यावत्तावद्विष्णुर्जगाम ह}
{दिवि तत्कालमुत्पन्ना वागुवाचाशरीरिणी}% ॥४२॥

\twolineshloka
{राज्यं प्राप्स्यसि पुत्रत्वं सफलायाः प्रसादतः।}
{तथेत्युक्ते तु वचसि दिव्यरूपधरोऽभवत्}% ॥४३॥

\twolineshloka
{मतिरासीत्ततस्तस्य परमा वैष्णवी नृप}
{दिव्याभरणशोभाढ्यो लेभे राज्यमकण्टकम्}% ॥४४॥

\twolineshloka
{कृतं राज्यं तु तेनैवं वर्षाणि दशपञ्च च}
{मनोदास्तस्यपुत्रास्तु दाराः कृष्णप्रसादतः}% ॥४५॥

\twolineshloka
{आशु राज्यं परित्यज्य पुत्रे चैव समर्प्य च}
{गतः कृष्णस्य सान्निध्यं यत्र गत्वा न शोचति}% ॥४६॥

\twolineshloka
{एवं यः कुरुते राजन्सफलाव्रतमुत्तमम्}
{इहलोके सुखं प्राप्य मृतो मोक्षमवाप्नुयात्}% ॥४७॥

\twolineshloka
{धन्यास्ते मानवा लोके सफलायां च ये रताः}
{तेषां च सफलं जन्म नात्र कार्या विचारणा}% ॥४८॥

\twolineshloka
{पठनाच्छ्रवणाच्चैव करणाच्च विशाम्पते}
{राजसूयस्य यज्ञस्य फलमाप्नोति मानवः}% ॥४९॥

॥इति श्रीपाद्मे महापुराणे पञ्चपञ्चाशत्साहस्र्यां संहितायामुत्तरखण्डे उमापतिनारदसंवादान्तर्गतकृष्णयुधिष्ठिरसंवादे पौष-कृष्ण-सफला-एकादशी-माहात्म्यं नाम द्विचत्वारिंशोऽध्यायः॥४२॥

\hyperref[sec:ekadashi_mahatmyam_padma_puranam]{\closesub}
\clearpage

\sect{पौष-शुक्ल-पुत्रदा-एकादशी-माहात्म्यम्}
\label{sec:padma-pausha-shukla-putrada}


\uvacha{युधिष्ठिर उवाच}

\twolineshloka
{कथिता वै त्वया कृष्ण सफलैकादशी शुभा}
{कथयस्व प्रसादेन शुक्लपक्षस्य या भवेत्}% ॥१॥

\twolineshloka
{किन्नाम को विधिस्तस्याः को देवस्तत्र पूज्यते}
{कस्मै तुष्टो हृषीकेशस्त्वमेव पुरुषोत्तमः}% ॥२॥

\uvacha{श्रीकृष्ण उवाच}

\twolineshloka
{शृणु राजन्प्रवक्ष्यामि शुक्ला पौषस्य या भवेत्}
{कथयामि महाराज लोकानां हितकाम्यया}% ॥३॥

\twolineshloka
{पूर्वेण विधिना राजन्कर्तव्यैषा प्रयत्नतः}
{पुत्रदा नाम नाम्ना सा सर्वपापहरा परा}% ॥४॥

\twolineshloka
{नारायणोऽधिदेवोऽस्या कामदः सिद्धिदायकः}
{नातः परतरा काचित्त्रैलोक्ये सचराचरे}% ॥५॥

\twolineshloka
{विद्यावन्तं यशस्वन्तं करोति च नरं हरिः}
{शृणु राजन्प्रवक्ष्यामि कथां पापहरां पराम्}% ॥६॥

\twolineshloka
{भद्रावत्यां पुरा ह्यासीत्पुर्यां राजा सुकेतुमान्}
{तस्य राज्ञस्तथाराज्ञी चम्पका नाम वर्तते}% ॥७॥

\twolineshloka
{पुत्रहीनेन राज्ञा च काले नीतो मनोरथैः}
{नैवात्मजं नृपो लेभे वंशकर्तारमेव च}% ॥८॥

\twolineshloka
{तेनैव राज्ञा धर्मेण चिन्तितं बहुकालतः}
{किं करोमि क्व गच्छामि सुतप्राप्तिः कथं भवेत्}% ॥९॥

\twolineshloka
{न राष्ट्रे न पुरे सौख्यं लेभे राजा सुकेतुमान्}
{साध्व्य स्वकान्तया सार्धं प्रत्यहं दुःखितोऽभवत्}% ॥१०॥

\twolineshloka
{तावुभौ दम्पती नित्यं चिन्ताशोकपरायणौ}
{पितरोऽस्य जलं दत्तं कवोष्णमुपभुञ्जते}% ॥११॥

\twolineshloka
{राज्ञः पश्चान्न पश्यामो योऽस्मान्सन्तर्पयिष्यति}
{इत्येवं संस्मरन्तोऽस्य दुःखिताः पितरोऽभवन्}% ॥१२॥

\twolineshloka
{न बान्धवा न मित्राणि नामात्याः सुहृदस्तथा}
{रोचयन्त्यस्य भूपस्य न गजाश्वाः पदातयः}% ॥१३॥

\twolineshloka
{नैराश्यं भूपतेस्तस्य नित्यं मनसि वर्तते}
{नरस्य पुत्रहीनस्य नास्ति वै जन्मनः फलम्}% ॥१४॥

\twolineshloka
{अपुत्रस्य गृहं शून्यं हृदयं दुःखितं सदा}
{पितृदेवमनुष्याणां नानृणत्वं सुतं विना}% ॥१५॥

\twolineshloka
{तस्मात्सर्वप्रयत्नेन सुतमुत्पादयेन्नरः}
{इह लोके यशस्तेषां परलोके शुभा गतिः}% ॥१६॥

\twolineshloka
{येषां तु पुण्यकर्तॄणां पुत्रजन्मगृहे भवेत्}
{आयुरारोग्यसम्पत्तिस्तेषां गेहे प्रवर्तते}% ॥१७॥

\twolineshloka
{पुत्रजन्म गृहे येषां लोकानां पुण्यकारिणाम्}
{पुण्यं विना न च प्राप्तिर्विष्णुभक्तिं विना नृप}% ॥१८॥

\twolineshloka
{पुत्राश्च सम्पदो वाऽपि निश्चयादिति मे मतिः}
{एवं विचिन्त्यमानोऽसौ न शर्म लभते नृपः}% ॥१९॥

\twolineshloka
{प्रत्यूषे चिन्तयद्राजा निशीथे चिन्तयत्ततः}
{स्वयमात्मविनाशं च चिन्तयामास केतुमान्}% ॥२०॥

\twolineshloka
{आत्मघाते दुर्गतिं च चिन्तयित्वा तदा नृपः}
{दृष्ट्वात्मदेहं पतितमपुत्रत्वं तथैव च}% ॥२१॥

\twolineshloka
{पुनर्विचार्यात्मबुद्ध्या आत्मनो हितकारणम्}
{अश्वारूढस्ततो राजा जगाम गहनं वनम्}% ॥२२॥

\twolineshloka
{पुरोहितादयः सर्वे न जानन्ति गतं नृपम्}
{गम्भीरे विपिने राजा मृगपक्षिनिषेविते}% ॥२३॥

\twolineshloka
{विचचार तदा राजा वनवृक्षान्विलोकयन्}
{वटानश्वत्थबिल्वांश्च खर्जूरान्पनसांस्तथा}% ॥२४॥

\twolineshloka
{बकुलान्सप्तपर्णांश्च तिन्दुकांस्तिलकानपि}
{शालांस्तालांस्तमालांश्च ददर्श सरलान्नृपः}% ॥२५॥

\twolineshloka
{इङ्गुदी ककुभांश्चैव श्लेष्मातकनगांस्तथा}
{शल्लकान्करमर्दांश्च पाटलान्बदरानपि}% ॥२६॥

\twolineshloka
{अशोकांश्च पलाशांश्च शृगालाञ्शशकानपि}
{वनमार्जारमहिषान्शल्लकांश्चमरानपि}% ॥२७॥

\twolineshloka
{ददर्श भुजगान्राजा वल्मीकादर्धनिःसृतान्}
{तथा वनगजान् मत्तान् कलभैः सह सङ्गतान्}% ॥२८॥

\twolineshloka
{यूथपांश्च चतुर्दन्तान्करिणीयूथमध्यगान्}
{तान्दृष्ट्वा चिन्तयामास आत्मनः स गजान्नृपः}% ॥२९॥

\twolineshloka
{तेषां स विचरन्मध्ये राजा शोभामवाप ह}
{महदाश्चर्यसंयुक्तं ददृशे विपिनं नृपः}% ॥३०॥

\twolineshloka
{मार्गे शिवारुतान्शृण्वन्नुलूकविरुतं तथा}
{तांस्तानृक्षमृगान्पश्यन्बभ्राम वनमध्यतः}% ॥३१॥

\twolineshloka
{एवं ददर्श गहनं नृपो मध्यगते रवौ}
{क्षुत्तृड्भ्यां पीडितो राजा इतश्चेतश्च धावति}% ॥३२॥

\twolineshloka
{नृपतिश्चिन्तयामास संशुष्कगलकन्धरः}
{मया तु किं कृतं कर्म प्राप्तं दुःखं यदीदृशम्}% ॥३३॥

\twolineshloka
{मया वै तोषिता देवा यज्ञैः पूजाभिरेव च}
{तथैव ब्राह्मणा दानैस्तोषिता मिष्टभोजनैः}% ॥३४॥

\twolineshloka
{प्रजाश्चैव सदा कालं पुत्रवत्पालिता भृशम्}
{कस्माद्दुःखं मया प्राप्तमीदृशं दारुणं महत्}% ॥३५॥

\twolineshloka
{इति चिन्तापरो राजा जगामैवाग्रतो वनम्}
{सुकृतस्य प्रभावेन सरो दृष्टमनुत्तमम्}% ॥३६॥

\twolineshloka
{मीनसंस्पर्शमानं च पद्मैश्चापरशोभितम्}
{कारण्डैश्चक्रवाकैश्च राजहंसैश्च शोभितम्}% ॥३७॥

\twolineshloka
{मकरैर्बहुभिर्मत्स्यैरन्यैर्जलचरैर्युतम्}
{समीपे सरसस्तस्य मुनीनामाश्रमान्बहून्}% ॥३८॥

\twolineshloka
{ददर्श राजा लक्ष्मीवान्शकुनैः शुभशंसिभिः}
{दक्षिणं प्रास्फुरन्नेत्रमथ सव्येतरः करः}% ॥३९॥

\twolineshloka
{प्रास्फुरन्नृपतेस्तस्य कथयञ्शोभनं फलम्}
{तस्य तीरे मुनीन्दृष्ट्वा कुर्वाणन्नैगमं जपम्}% ॥४०॥

\twolineshloka
{हर्षेण महताविष्टो बभूव नृपनन्दनः}
{अवतीर्य हयात्तस्मान्मुनीनामग्रतः स्थितः}% ॥४१॥

\twolineshloka
{पृथक्पृथग्ववन्देऽसौ मुनींस्तान्शंसितव्रतान्}
{कृताञ्जलिपुटो भूत्वा दण्डवच्च पुनः पुनः}% ॥४२॥

\onelineshloka*
{प्रत्यूचुस्तेऽपि मुनयः प्रसन्ना नृपते वयम्}

\uvacha{राजोवाच}

\twolineshloka
{के भवन्तोऽत्र कथ्यन्तां का चाख्या भवतामपि}
{किमर्थं सङ्गता यूयं सत्यं वदत मेऽग्रतः}% ॥४३॥

\uvacha{मुनय ऊचुः}

\twolineshloka
{विश्वेदेवा वयं राजन्स्नानार्थमिह चागताः}
{माघो निकटमायात एतस्मात्पञ्चमेऽहनि}% ॥४४॥

\twolineshloka
{अद्य चैकादशी राजन्पुत्रदा नाम नामतः}
{पुत्रं ददात्यसौ विष्णुः पुत्रदा कारिणां नृणाम्}% ॥४५॥

\uvacha{राजोवाच}

\twolineshloka
{एष वै संशयो मह्यं पुत्रस्योत्पादने महान्}
{यदि तुष्टा भवन्तो वै पुत्रं मे दीयतां तदा}% ॥४६॥

\uvacha{मुनिरुवाच}

\twolineshloka
{अद्यैव दिवसे राजन्पुत्रदा नाम वर्तते}
{एकादशीति विख्यातं क्रियतां व्रतमुत्तमम्}% ॥४७॥

\twolineshloka
{अभिषेकात्ततोऽस्माकं केशवस्य प्रसादतः}
{अवश्यं तव राजेन्द्र पुत्रप्राप्तिर्भविष्यति}% ॥४८॥

\twolineshloka
{इत्येवं वचनात्तेषां कृतं राज्ञा व्रतोत्तमम्}
{मुनीनामुपदेशेन पुत्रदा या विधानतः}% ॥४९॥

\twolineshloka
{द्वादश्यां पारणं कृत्वा मुनीन्नत्वा पुनः पुनः}
{आजगाम गृहं राजा राज्ञी गर्भमथादधौ}% ॥५०॥

\twolineshloka
{पुत्रो जातः सूतिकाले तेजस्वी पुण्यकर्मणा}
{पितरं तोषयामास प्रजापालो बभूव सः}% ॥५१॥

\twolineshloka
{तस्माद्राजन्प्रकर्तव्यं पुत्रदा व्रतमुत्तमम्}
{लोकानां तु हितार्थाय तवाग्रे कथितं मया}% ॥५२॥

\threelineshloka
{एकचित्तास्तु ये मर्त्याः कुर्वन्ति पुत्रदा व्रतम्}
{पुत्रान्प्राप्येह लोके तु मृतास्ते स्वर्गगामिनः}
{पठनाच्छ्रवणाद्राजन्नग्निष्टोमफलं लभेत्}% ॥५३॥

॥इति श्रीपाद्मे महापुराणे पञ्चपञ्चाशत्साहस्र्यां संहितायामुत्तरखण्डे उमापतिनारदसंवादान्तर्गतकृष्णयुधिष्ठिरसंवादे पौष-शुक्ल-पुत्रदा-एकादशी-माहात्म्यं नाम त्रिचत्वारिंशोऽध्यायः॥४३॥


\hyperref[sec:ekadashi_mahatmyam_padma_puranam]{\closesub}
\clearpage

\sect{माघ-कृष्ण-षट्तिला-एकादशी-माहात्म्यम्}
\label{sec:padma-magha-krishna-shattila}


\uvacha{युधिष्ठिर उवाच}

\twolineshloka
{साधु कृष्ण जगन्नाथ आदिदेव जगत्पते}
{कथयस्व प्रसादेन कृपां कुरु ममोपरि}% ॥१॥

\twolineshloka
{माघस्य कृष्णपक्षे तु का वा चैकादशी भवेत्}
{किं नाम को विधिस्तस्या  एतद्विस्तरतो वद}% ॥२॥

\uvacha{श्रीभगवानुवाच}

\twolineshloka
{शृणु त्वं नृपशार्दूल कृष्णमाघस्य या भवेत्}
{षट्तिला नाम विख्याता सर्वपापप्रणाशिनी}% ॥१॥

\twolineshloka
{षट्तिलायाः शृणुष्व त्वं कथां पापहरां शुभाम्}
{यां पुलस्त्यो मुनिश्रेष्ठो दालभ्यं प्रति चोक्तवान्}% ॥२॥

\uvacha{दालभ्य उवाच}

\twolineshloka
{मर्त्यलोकमनुप्राप्ताः पापं कुर्वन्ति जन्तवः}
{ब्रह्महत्यादिपापैश्च युक्ता ये विविधादिभिः}% ॥३॥

\twolineshloka
{परद्रव्यापहाराश्च परव्यसनमोहिताः}
{कथं न यान्ति नरकं ब्रह्मंस्तद्ब्रूहि तत्त्वतः}% ॥४॥

\twolineshloka
{अनायासेन भगवन्दानेनाल्पेन केनचित्}
{पापं प्रशमनं याति एतन्मे वक्तुमर्हसि}% ॥५॥

\uvacha{पुलस्त्य उवाच}

\twolineshloka
{साधुसाधु महाभाग गुह्यमेतत्सुदुर्ल्लभम्}
{यन्न कस्यचिदाख्यातं विष्णुब्रह्मेन्द्रदैवतैः}% ॥६॥

\twolineshloka
{तदहं कथयिष्यामि त्वया पृष्टो द्विजोत्तम}
{माघमासे तु सम्प्राप्ते शुचिस्नातो जितेन्द्रियः}% ॥७॥

\twolineshloka
{कामक्रोधाभिमानेर्ष्यालोभ पैशुन्य वर्जितः}
{देवदेवं च संस्मृत्य पादौ प्रक्षाल्य वारिभिः}% ॥८॥

\twolineshloka
{भूमावपतितं गृह्य गोमयं तत्र मानवः}
{तिलान्प्रक्षिप्य कार्पांसं पिण्डिकाश्चैव कारयेत्}% ॥९॥

\twolineshloka
{अष्टोत्तरशतं चैव नात्र कार्या विचारणा}
{ततो माघे च सम्प्राप्ते ह्याषाढर्क्षं भवेद्यदि}% ॥१०॥

\twolineshloka
{मूलं वा कृष्णपक्षस्यैकादशी नियमांस्ततः}
{गृह्णीयात्पुण्यकाले च विधानं तत्र मे शृणु}% ॥११॥

\twolineshloka
{देवदेवं समभ्यर्च्य सुस्नातः प्रयतः शुचिः}
{कृष्णनामानि सङ्कीर्त्य पुनः प्रस्खलितादिषु}% ॥१२॥

\twolineshloka
{रात्रौ जागरणं कुर्यादादौ होमं च कारयेत्}
{अर्चयेद्देवदेवेशं द्वितीयेऽह्नि पुनर्हरिम्}% ॥१३॥

\twolineshloka
{चन्दनागुरुकर्पूरैर्नैवेद्यं कृसरं तथा}
{संस्मृत्य नाम्ना च ततः कृष्णाख्येन पुनः पुनः}% ॥१४॥

\threelineshloka
{कूष्माण्डैर्नारिकेरैश्च ह्यथवा बीजपूरकैः}
{सर्वाभावेऽपि विप्रेन्द्र शस्तपूगफलैर्वृतम्}
{अर्घं दद्याद्विधानेन पूजयित्वा जनार्दनम्}% ॥१५॥

\twolineshloka
{कृष्णकृष्ण कृपालुस्त्वमगतीनां गतिर्भव}
{संसारार्णव मग्नानां प्रसीद पुरुषोत्तम}% ॥१६॥

\threelineshloka
{नमस्ते पुण्डरीकाक्ष नमस्ते विश्वभावन}
{सुब्रह्मण्य नमस्तेऽस्तु महापुरुषपूर्वज}
{गृहाणार्घ्यं मया दत्तं लक्ष्म्या सह जगत्पते}% ॥१७॥
[इत्यर्घमन्त्रः]

\twolineshloka
{ततस्तु पूजयेद्विप्रमुदकुम्भं प्रदापयेत्}
{छत्रोपानहवस्त्रैश्च कृष्णो मे प्रीयतामिति}% ॥१८॥

\twolineshloka
{कृष्णा धेनुः प्रदातव्या यथाशक्ति द्विजोत्तमे}
{तिलपात्रं द्विजश्रेष्ठ दद्यात्पात्रविचक्षणः}% ॥१९॥

\threelineshloka
{स्नाने प्राशनके शस्तास्तथा कृष्णतिला मुने}
{तान्प्रदद्यात्प्रयत्नेन यथाशक्ति द्विजोत्तमे}
{तिलप्ररोहजाः क्षत्रे यावत्सङ्ख्यास्तिला द्विज}% ॥२०॥

\twolineshloka
{तावद्वर्षसहस्राणि स्वर्गलोके महीयते}
{तिलस्नायी तिलोद्वर्ती तिलहोमी तिलोदकी}% ॥२१॥

\onelineshloka*
{तिलदाता च भोक्ता च षट्तिलाः पापनाशनाः}

\uvacha{नारद उवाच}

\onelineshloka
{कृष्ण कृष्ण महाबाहो नमस्ते विश्वभावन}% ॥२२॥

\twolineshloka*
{षट्तिलैकादशीभूतं कीदृशं फलमस्ति वै}
{सोपाख्यानं मम ब्रूहि यदि तुष्टोऽसि यादव}

\uvacha{श्रीकृष्ण उवाच}

\onelineshloka*
{शृणु राजन्यथावृत्तं दृष्टं तत्कथयामि ते}

\twolineshloka
{मर्त्यलोके पुरा ह्यासीद्ब्राह्मण्येका च नारद}
{व्रतचर्यारता नित्यं देवपूजारता सदा}% ॥२३॥

\twolineshloka
{मासोपवासनिरता मम भक्ता च सर्वदा}
{कृष्णोपवाससंयुक्ता मम पूजापरायणा}% ॥२४॥

\twolineshloka
{शरीरं क्लेशितं चैव उपवासैर्द्विजोत्तम}
{देवानां ब्राह्मणानां च कुमारीणां च भक्तितः}% ॥२५॥

\twolineshloka
{गृहादिकं प्रयच्छन्ती सर्वकालं महासती}
{अतिकृच्छ्ररता सा तु सर्वकालं तु वै द्विज}% ॥२६॥

\onelineshloka
{न दत्ता भिक्षुके भिक्षा ब्राह्मणा न च तर्पिताः}% ॥२७॥

\twolineshloka
{ततः कालेन महता मया वै चिन्तितं द्विज}
{शुद्धमस्याः शरीरं हि व्रतैः कृच्छ्रैर्न संशयः}% ॥२८॥

\twolineshloka
{अर्चितो वैष्णवो लोकः कायक्लेशेन  वै तया}
{न दत्तमन्नदानं हि येन तृप्तिः परा भवेत्}% ॥२९॥

\twolineshloka
{एवं ज्ञात्वा अहं ब्रह्मन्मर्त्यलोकमुपागतः}
{कापालं रूपमास्थाय भिक्षापात्रे च याचिता}% ॥३०॥

\twolineshloka
{कस्मात्त्वमागतो ब्रह्मन्क्व यासि वद मेऽग्रतः}
{पुनरेव मया प्रोक्तं देहि भिक्षां च सुन्दरि}% ॥३१॥

\twolineshloka
{तया कोपेन महता मृत्पिण्डस्ताम्रभाजने}
{क्षिप्तो यावदहं ब्रह्मन्पुनः स्वर्गं गतो द्विज}% ॥३२॥

\twolineshloka
{ततः कालेन महता तापसी सुमहाव्रता}
{सदेहा स्वर्गमायाता व्रतचर्या प्रभावतः}% ॥३३॥

\twolineshloka
{मृत्पिण्डिकाप्रदानेन गृहं प्राप्तं मनोरमम्}
{सञ्जातं चैव विप्रर्षे धान्यराशि विवर्जितम्}% ॥३४॥

\twolineshloka
{गृहं यावन्निरीक्षेत न किञ्चित्तत्र पश्यति}
{तावद्गृहाद्विनिष्क्रान्ता ममान्ते चागता द्विज}% ॥३५॥

\twolineshloka
{क्रोधेन महताविष्टमिदं वचनमब्रवीत्}
{मया व्रतैश्च कृच्छ्रैश्च उपवासैरनेकशः}% ॥३६॥

\twolineshloka
{पूजयाराधितो देवः सर्वलोकस्य पालकः}
{न तत्र  दृश्यते किञ्चिद्गृहे मम जनार्दन}% ॥३७॥

\twolineshloka
{ततश्चोक्तं मया तस्यै गृहं गच्छ महाव्रते}
{आगमिष्यन्ति सुतरां कौतूहल समन्विताः}% ॥३८॥

\twolineshloka
{देवपत्न्यो हि द्रष्टुं त्वां विस्मयाभिसमन्विताः}
{द्वारं नोद्घाटय विना षट्तिलापुण्यवाचनात्}% ॥३९॥

\twolineshloka
{एवमुक्ता मया सा तु गता वै मानुषी तदा}
{अत्रान्तरे समायाता देवपत्न्यश्च वाडव}% ॥४०॥

\twolineshloka
{ताभिश्च कथितं तत्र त्वां द्रष्टुं हि समागताः}
{द्वारमुद्घाटयस्वाद्य त्वां प्रपश्याम शोभने}% ॥४१॥

\uvacha{मानुष्युवाच}

\twolineshloka
{यदि मद्दर्शनं कार्यं सत्यं वाच्यं विशेषतः}
{षट्तिलाया व्रतं पुण्यं द्वारोद्घाटनकारणात्}% ॥४२॥

\uvacha{श्रीकृष्ण उवाच}

\twolineshloka
{एकापिनावदत्तत्र षट्तिलैकादशीव्रतम्}
{अन्यया कथितं तत्र द्रष्टव्या मानुषी मया}% ॥४३॥

\twolineshloka
{ततो द्वारं समुद्घाट्य दृष्टा ताभिश्च मानुषी}
{न देवी न च गन्धर्वी नासुरी न च पन्नगी}% ॥४४॥

\twolineshloka
{दृष्टा पूर्वं तथा नारी यादृशीयं द्विजर्षभ}
{देवीनामुपदेशेन षट्तिलाया व्रतं कृतम्}% ॥४५॥

\twolineshloka
{मानुष्या सत्यव्रतया भुक्तिमुक्तिफलप्रदम्}
{रूपकान्तिसमायुक्ता क्षणेन समवाप सा}% ॥४६॥

\twolineshloka
{धनं धान्यं च वस्त्रादि सुवर्णं रौप्यमेव च}
{भवनं सर्वसम्पन्नं षट्तिलायाः प्रभावतः}% ॥४७॥

\onelineshloka
{रूपकान्तिसमायुक्ता क्षणेन समपद्यत}% ॥४८॥

\twolineshloka
{अतितृष्णा न कर्तव्या वित्तशाठ्यं विवर्जयेत्}
{आत्मवित्तानुसारेण तिलान्वस्त्राणि दापयेत्}% ॥४९॥

\twolineshloka
{लभते चैवमारोग्यं नरो जन्मनि जन्मनि}
{न दारिद्र्यं न कष्टत्वं न च दौर्भाग्यमेव च}% ॥५०॥

\twolineshloka
{सम्भवेद्वै द्विजश्रेष्ठ षट्तिलासमुपोषणात्}
{अनेन विधिना भूप तिलदाता न संशयः}% ॥५१॥

\threelineshloka
{मुच्यते पातकैः सर्वैरनायासेन मानवः}
{दानं च विधिवत्पात्रे सर्वपातकनाशनम्}
{नानर्थः कश्चिन्नायासः शरीरे नृपसत्तम}% ॥५२॥


॥इति श्रीपाद्मे महापुराणे पञ्चपञ्चाशत्साहस्र्यां संहितायामुत्तरखण्डे उमापतिनारदसंवादान्तर्गतकृष्णयुधिष्ठिरसंवादे माघ-कृष्ण-षट्तिला-एकादशी-माहात्म्यं नाम चतुश्चत्वारिंशोऽध्यायः॥४४॥
\hyperref[sec:ekadashi_mahatmyam_padma_puranam]{\closesub}
\clearpage

\sect{माघ-शुक्ल-जया-एकादशी-माहात्म्यम्}
\label{sec:padma-magha-shukla-jaya}


\uvacha{युधिष्ठिर उवाच}

\twolineshloka
{साधु कृष्ण त्वया प्रोक्तमादिदेवो भवान्प्रभो}
{स्वेदजा अण्डजाश्चैव उद्भिज्जाश्च जरायुजाः}% ॥१॥

\twolineshloka
{तेषां कर्ता विकर्ता त्वं पालकः क्षयकारकः}
{माघस्य कृष्णपक्षे तु षट्तिला कथिता त्वया}% ॥२॥

\twolineshloka
{शुक्ले च का भवेद्देव कथयस्व प्रसादतः}
{किं नाम को विधिस्तस्याः को देवस्तत्र पूज्यते}% ॥३॥

\uvacha{श्रीकृष्ण उवाच}

\twolineshloka
{कथयिष्यामि राजेन्द्र शुक्ले माघस्य या भवेत्}
{जया नामेति विख्याता सर्वपापहरा परा}% ॥४॥

\twolineshloka
{पवित्रा पापहन्त्री च कामदा मोक्षदा नृणाम्}
{ब्रह्महत्यापहन्त्री च पिशाचत्वविनाशिनी}% ॥५॥

\twolineshloka
{नैव तस्या व्रते चीर्णे प्रेतत्वं जायते नृणाम्}
{नातः परतरा काचित्पापघ्नी मोक्षदायिनी}% ॥६॥

\twolineshloka
{एतस्मात्कारणाद्राजन्कर्तव्या सा प्रयत्नतः}
{श्रूयतां राजशार्दूल कथा पौराणिकी शुभा}% ॥७॥

\twolineshloka
{पङ्कजे च पुराणेऽस्या महिमा कथितो मया}
{एकदा नाकलोके वै इन्द्रो राज्यं चकार ह}% ॥८॥

\twolineshloka
{देवास्तत्र सुखेनैव निवसन्ति मनोरमे}
{पीयूषपाननिरता अप्सरोगणसेविताः}% ॥९॥

\twolineshloka
{नन्दनं तु वनं तत्र पारिजातोपसेवितम्}
{रमयन्ति रमन्तेऽत्र अप्सरोभिर्दिवौकसः}% ॥१०॥

\twolineshloka
{एकदा रममाणोऽसौ देवेन्द्रः स्वेच्छया नृप}
{नर्तयामास वै हर्षात्पञ्चाशत्कोटिनायकः}% ॥११॥

\twolineshloka
{गन्धर्वास्तत्र गायन्ति गन्धर्वः पुष्पदन्तकः}
{चित्रसेनस्तु तत्रैव चित्रसेनसुता तथा}% ॥१२॥

\twolineshloka
{मालिनीति च नाम्ना तु चित्रसेनस्य योषिता}
{मालिन्यास्तु समुत्पन्ना पुष्पदन्ती च नामतः}% ॥१३॥

\twolineshloka
{पुष्पदन्तस्य पुत्रोऽसौ माल्यवान्नाम नामतः}
{पुष्पदन्त्याश्च रूपेण माल्यवानतिमोहितः}% ॥१४॥

\twolineshloka
{तया ह्येवं कटाक्षैश्च माल्यवांश्च वशीकृतः}
{लावण्यं रूपसम्पन्नं तस्या रूपं निशामय}% ॥१५॥

\twolineshloka
{बाहू तस्याश्च कामेन कण्ठपाशौ कृताविव}
{कर्णायते तु नयने रक्तान्ते घूर्णिते तथा}% ॥१६॥

\twolineshloka
{कर्णौ तु शोभनौ तस्याः कुण्डलाभ्यां नृपोत्तम}
{कम्बुग्रीवायुता सैव दिव्याभरणभूषिता}% ॥१७॥

\twolineshloka
{पीनोन्नतौ कुचौ तस्यास्तौ हेमकलशाविव}
{मध्यं क्षामं च चार्वङ्ग्या मुष्टिग्राह्यमनुत्तमम्}% ॥१८॥

\twolineshloka
{नितम्बौ विस्तृतौ चास्या विस्तीर्णा जघनस्थली}
{चरणौ शोभमानौ च रक्तोत्पलसमद्युती}% ॥१९॥

\twolineshloka
{ईदृश्या पुष्पवत्या स माल्यवानतिमोहितः}
{शक्रस्य परितोषाय नृत्यार्थं तौ समागतौ}% ॥२०॥

\twolineshloka
{गायमानौ तु तौ तत्र अप्सरोगणसेवितौ}
{मदनाभिपरीताङ्गौ पुष्पदन्ती च माल्यवान्}% ॥२१॥

\twolineshloka
{परस्परानुरागेण व्यामोहवशमागतौ}
{न शुद्धगानं गायेतां चित्तभ्रमसमन्वितौ}% ॥२२॥

\twolineshloka
{बद्धदृष्टी तथान्योन्यं कामबाणवशं गतौ}
{ज्ञात्वा लेखर्षभस्तत्र सङ्गतं मानसं तयोः}% ॥२३॥

\twolineshloka
{तालक्रियामानलोपात्तथा गीतविसर्जनात्}
{चिन्तयित्वा तु मघवा ह्यवमानं तथात्मनः}% ॥२४॥

\twolineshloka
{कुपितश्च तयोरर्थे शापं दास्यन्निदं जगौ}
{धिग्धिग्वां पतितौ मूढावाज्ञाभङ्ग कृतौ मम}% ॥२५॥

\twolineshloka
{युवां पिशाचौ भवतां दम्पतीभावधारिणौ}
{मर्त्यलोकमनुप्राप्तौ भुञ्जानौ कर्मणः फलम्}% ॥२६॥

\twolineshloka
{एवं मघवता शप्तावुभौ दुःखितमानसौ}
{हिमवन्तं गिरिं प्राप्ताविन्द्रशापाद्विमोहितौ}% ॥२७॥

\twolineshloka
{उभौ पिशाचतां प्राप्तौ दारुणं दुःखमेव च}
{सन्तप्तमानसौ तत्र हिमकृच्छ्रगतावुभौ}% ॥२८॥

\twolineshloka
{गन्धर्वत्वमप्सरस्त्वं न जानीतो विमौहितौ}
{पीड्यमानौ निदाघेन देहपातकजेन च}% ॥२९॥

\twolineshloka
{न निशायां सुखं शान्तिं लभेते कर्मपीडितौ}
{परस्परं वादमानौ चेरतुर्गिरिगह्वरे}% ॥३०॥

\twolineshloka
{पीड्यमानौ तु शीतेन तुषारप्रभवेन तौ}
{दन्तघर्षं प्रकुर्वाणौ रोमाञ्चितवपुर्धरौ}% ॥३१॥

\twolineshloka
{ऊचे पिशाचः स तदा तां पत्नीं स्वां पिशाचिकाम्}
{किमनल्पकृतं पापं दारुणं रोमहर्षणम्}% ॥३२॥

\twolineshloka
{येन प्राप्तं पिशाचत्वं स्वेन दुष्कृतकर्मणा}
{नरकं दारुणं मत्वा पिशाचत्वं च दुःखदम्}% ॥३३॥

\twolineshloka
{तस्मात्सर्वप्रयत्नेन पातकं न समाचरेत्}
{इति चिन्तापरौ तत्र तावास्तां दुःखकर्षितौ}% ॥३४॥

\twolineshloka
{दैवयोगात्तयोः प्राप्ता माघस्यैकादशी तिथिः}
{जयानामेति विख्याता तिथीनामुत्तमा तिथिः}% ॥३५॥

\twolineshloka
{तस्मिन्दिने तु सम्प्राप्ते तावाहारविवर्जितौ}
{आसाते तत्र नृपते जलपानविवर्जितौ}% ॥३६॥

\twolineshloka
{न कृतो जीवघातश्च न पत्रफलभक्षणम्}
{अश्वत्थस्य समीपे तौ सर्वदा दुःखसंयुतौ}% ॥३७॥

\twolineshloka
{रविरस्तङ्गतो राजंस्तथैव स्थितयोस्तयोः}
{प्राप्ता चैव निशा घोरा दारुणा प्राणहारिणी}% ॥३८॥

\twolineshloka
{वेपमानौ ततस्तौ तु ततः सुषुपतः क्षितौ}
{परस्परेण संलग्नौ गात्रयोर्भुजयोरपि}% ॥३९॥

\twolineshloka
{न निद्रा न रतं तत्र न तौ सौख्यमविन्दताम्}
{एवं तौ राजशार्दूल शापेनेन्द्रस्य पीडितौ}% ॥४०॥

\twolineshloka
{इत्थं तयोर्दुःखितयोर्निर्जगाम निशीथिनी}
{मार्तण्ड उदयं प्राप्तो द्वादशी दिवसागमे}% ॥४१॥

\twolineshloka
{मया तु राजशार्दूल तयोर्मुक्तिर्धृता हृदि}
{जयायाः सुव्रतं चीर्णं रात्रौ जागरणं कृतम्}% ॥४२॥

\twolineshloka
{तस्माद्व्रतप्रभावाच्च यथाजातं तथा शृणु}
{द्वादशीदिवसे प्राप्ते तथा चीर्णे जया व्रते}% ॥४३॥

\twolineshloka
{विष्णोः प्रभावान्नृपते पिशाचत्वं तयोर्गतम्}
{पुष्पदन्ती माल्यवतौ पूर्वरूपौ बभूवतुः}% ॥४४॥

\twolineshloka
{पुरातनस्नेहयुतौ  पूर्वालङ्कारधारिणौ}
{विमानमधिरूढौ तौ गतौ नाके मनोरमे}% ॥४५॥

\twolineshloka
{देवेन्द्रस्याग्रतो गत्वा प्रणामं चक्रतुर्मुदा}
{तथाविधौ तु तौ दृष्ट्वा मघवा विस्मितोऽब्रवीत्}% ॥४६॥

\uvacha{इन्द्र उवाच}

\twolineshloka
{वद तं केन पुण्येन पिशाचत्वं हि वां गतौ}
{मम शापं च सम्प्राप्तौ केन देवेन मोचितौ}% ॥४७॥

\uvacha{माल्यवानुवाच}

\twolineshloka
{वासुदेवप्रसादेन जयायास्तु व्रतेन च}
{पिशाचत्वं गतं स्वामिंस्तव भक्तिप्रभावतः}% ॥४८॥

\twolineshloka
{इति श्रुत्वा तु मघवा प्रत्युवाच पुनस्तथा}
{पवित्रौ पावनौ जातौ वन्दनीयौ ममापि च}% ॥४९॥

\twolineshloka
{हरिवासरकर्तारौ विष्णुभक्तिपरायणौ}
{हरिवासरसँल्लीना ये च कृष्णपरायणाः}% ॥५०॥

\twolineshloka
{अस्माकमपि मर्त्यास्ते पूज्याश्चैव न संशयः}
{विहरस्व यथासौख्यं पुष्पदन्त्या सुरालये}% ॥५१॥

\uvacha{कृष्ण उवाच}

\twolineshloka
{एतस्मात्कारणाद्राजन्कर्तव्यो हरिवासरः}
{जया तु राजशार्दूल ब्रह्महत्यापहारिणी}% ॥५२॥

\twolineshloka
{सर्वदानानि तेनैव सर्वयज्ञा अशेषतः}
{दत्तानि कारिताश्चैव जयायास्तु व्रतं कृतम्}% ॥५३॥

\twolineshloka
{कल्पकोटिर्भवेत्तावद्वैकुण्ठे मोदते ध्रुवम्}
{पठनाच्छ्रवणाद्राजन्नग्निष्टोमफलं लभेत्}% ॥५४॥

॥इति श्रीपाद्मे महापुराणे पञ्चपञ्चाशत्साहस्र्यां संहितायामुत्तरखण्डे उमापतिनारदसंवादान्तर्गतकृष्णयुधिष्ठिरसंवादे माघ-शुक्ल-जया-एकादशी-माहात्म्यं नाम पञ्चचत्वारिंशोऽध्यायः॥४५॥


\hyperref[sec:ekadashi_mahatmyam_padma_puranam]{\closesub}
\clearpage

\sect{फाल्गुन-कृष्ण-विजया-एकादशी-माहात्म्यम्}
\label{sec:padma-phalguna-krishna-vijaya}


\uvacha{युधिष्ठिर उवाच}

\twolineshloka
{फाल्गुनस्यासिते पक्षे किन्नामैकादशी भवेत्}
{कथयस्व प्रसादेन वासुदेव ममाग्रतः}% ॥१॥

\uvacha{श्रीकृष्ण उवाच}

\threelineshloka
{नारदः परिपप्रच्छ ब्रह्माणं कमलासनम्}
{फाल्गुनस्यासिते पक्षे विजयानाम नामतः}
{तस्याः पुण्यं द्विजश्रेष्ठ कथयस्व प्रसादतः}% ॥२॥

\uvacha{ब्रह्मोवाच}

\twolineshloka
{शृणु नारद वक्ष्यामि कथां पापहरां पराम्}
{यन्न कस्यचिदाख्यातं मयैतद्विजयाव्रतम्}% ॥३॥

\twolineshloka
{पुरातनं व्रतं ह्येतत्पवित्रं पापनाशनम्}
{जयं ददाति विजया नृपाणां वै न संशयः}% ॥४॥

\twolineshloka
{पुरा रामो वनं यातो वर्षाण्येव चतुर्दश}
{न्यवसत्पञ्चवट्यां तु  सहसीतः सलक्ष्मणः}% ॥५॥

\twolineshloka
{तत्रैव वसतस्तस्य रामस्य विजयात्मनः}
{रावणेन हृता लौल्याद्भार्या सीता यशस्विनी}% ॥६॥

\twolineshloka
{तेन दुःखेन रामोऽपि मोहमभ्यागतस्तदा}
{भ्रमन्जटायुषमथो ददर्श विगतायुषम्}% ॥७॥

\twolineshloka
{कबन्धो निहतः पश्चाद्भ्रमतारण्यमध्यतः}
{सुग्रीवेण समं तस्य सखित्वं समपद्यत}% ॥८॥

\twolineshloka
{वानराणामनीकानि रामार्थंसङ्गतानि च}
{ततो हनुमता दृष्टा लङ्कोद्याने तु जानकी}% ॥९॥

\twolineshloka
{रामसंज्ञापनं तस्यै दत्तं कर्म महत्कृतम्}
{पुनः समेत्य रामेण सर्वं तत्र निवेदितम्}% ॥१०॥

\twolineshloka
{अथ श्रुत्वा रामचन्द्रो वाक्यं चैव हनूमतः}
{सुग्रीवानुमतेनैव प्रस्थानं समरोचयत्}% ॥११॥

\threelineshloka
{सौमित्रेकेन पुण्येन तीर्यते वरुणालयः}
{अगाधो नितरामेष यादोभिश्च समाकुलः}
{उपायं नैव पश्यामि येनासौ सुतरो भवेत्}% ॥१२॥

\uvacha{लक्ष्मण उवाच}

\twolineshloka
{आदिदेवस्त्वमेवासि पुराणपुरुषोत्तमः}
{बकदाल्भ्यो मुनिश्चात्र वर्तते द्वीपमध्यतः}% ॥१३॥

\twolineshloka
{अस्मात्स्थानाद्योजनार्धमाश्रमस्तस्य राघव}
{अन्ये च ब्राह्मणास्तत्र बहवो रघुनन्दन}% ॥१४॥

\twolineshloka
{तं पृच्छ गत्वा राजेन्द्र पुराणमृषिपुङ्गवम्}
{इति वाक्यं ततः श्रुत्वा लक्ष्मणस्यातिशोभनम्}% ॥१५॥

\twolineshloka
{जगाम राघवो द्रष्टुं बकदाल्भ्यं महामुनिम्}
{प्रणनाम मुनिं मूर्ध्ना रामो विष्णुमिवामरः}% ॥१६॥

\twolineshloka
{ज्ञात्वा मुनिस्ततो रामं पुराणं पुरुषोत्तमम्}
{केनापि कारणेनैव प्रविष्टो मानुषीं तनुम्}% ॥१७॥

\onelineshloka
{उवाच स ऋषिस्तुष्टः कुतो राम तवागमः}% ॥१८॥

\uvacha{राम उवाच}

\twolineshloka
{त्वत्प्रसादादहं विप्र तीरं नदनदीपतेः}
{आगतोऽस्मि ससैन्योऽत्र लङ्कां जेतुं सराक्षसाम्}% ॥१९॥

\twolineshloka
{भवतश्चानुकूलत्वात्तीर्यतेऽब्धिर्यथा मया}
{तमुपायं वद मुने प्रसादं कुरु साम्प्रतम्}% ॥२०॥

\twolineshloka
{एतस्मात्कारणादेव द्रष्टुं त्वाहमिहागतः}
{रामस्य वचनं श्रुत्वा बकदाल्भ्यो महामुनिः}% ॥२१॥

\twolineshloka
{उवाच सुप्रसन्नात्मा रामं राजीवलोचनम्}
{कर्तव्यमद्य ते राम व्रतानां व्रतमुत्तमम्}% ॥२२॥

\twolineshloka
{कृतेन येन सहसा विजयस्ते भविष्यति}
{लङ्कां जित्वा राक्षसांश्च स्वच्छां कीर्तिमवाप्स्यसि}% ॥२३॥

\twolineshloka
{एकाग्रमानसो भूत्वा व्रतमेतत्समाचर}
{फाल्गुनस्यासिते पक्षे विजयैकादशी भवेत्}% ॥२४॥

\twolineshloka
{तस्या व्रतेन हे राम विजयस्ते भविष्यति}
{निःसंशयं समुद्रं त्वं तरिष्यसि सवानरः}% ॥२५॥

\twolineshloka
{विधिस्तु श्रूयतां राजन्व्रतस्यास्य फलप्रदः}
{दशम्यां दिवसे प्राप्ते कुम्भमेकं तु कारयेत्}% ॥२६॥

\twolineshloka
{हैमं वा राजतं वाऽपि ताम्रं वाप्यथ मृन्मयम्}
{स्थापयेच्छोभितं चैव जलपूर्णं सपल्लवम्}% ॥२७॥

\twolineshloka
{सप्तधान्यान्यधस्तस्य यवानुपरि विन्यसेत्}
{तस्योपरि न्यसेद्देवं हैमं नारायणं प्रभुम्}% ॥२८॥

\twolineshloka
{एकादशीदिने प्राप्ते प्रातः स्नानं समाचरेत्}
{निश्चलं स्थापयेत्कुम्भं कण्ठमाल्यानुलेपनैः}% ॥२९॥

\twolineshloka
{पूगीफलैर्नालिकेरैः पूजयेच्च विशेषतः}
{गन्धैर्धूपैश्चदीपैश्च नैवेद्यैर्विविधैरपि}% ॥३०॥

\twolineshloka
{कुम्भाग्रे तद्दिनं राम नीयते सत्कथादिभिः}
{रात्रौ जागरणं चैव तस्याग्रे कारयेद्बुधः}% ॥३१॥

\twolineshloka
{प्रकाशयेद्घृतदीपमखण्डव्रतहेतवे}
{द्वादशीदिवसे प्राप्ते मार्तण्डस्योदये सति}% ॥३२॥

\twolineshloka
{नीत्वा कुम्भं जलोद्देशे नद्याः प्रस्रवणे तथा}
{तडागे स्थापयित्वा तं पूजयित्वा यथाविधि}% ॥३३॥

\twolineshloka
{दद्यात्सदेवं तं कुम्भं ब्राह्मणे वेदपारगे}
{कुम्भेन सह राजेन्द्र महादानानि दापयेत्}% ॥३४॥

\twolineshloka
{अनेनविधिना राम यूथपैः सह सङ्गतः}
{कुरु व्रतं प्रयत्नेन विजयस्ते भविष्यति}% ॥३५॥

\twolineshloka
{इति श्रुत्वा ततो रामो यथोक्तमकरोत्तदा}
{कृते व्रते स विजयी बभूव रघुनन्दनः}% ॥३६॥

\twolineshloka
{प्राप्ता सीता जिता लङ्का पौलस्त्यो निहतो रणे}
{अनेनविधिना पुत्र ये कुर्वन्ति नरा व्रतम्}% ॥३७॥

\twolineshloka
{इहलोकजयप्राप्तिः परलोकस्तथाक्षयः}
{एतस्मात्कारणात्पुत्र कर्तव्यं विजयाव्रतम्}% ॥३८॥

\twolineshloka
{विजयायाश्च माहात्म्यं सर्वकिल्बिषनाशनम्}
{पठनाच्छ्रवणाच्चैव वाजपेयफलं लभेत्}% ॥३९॥

॥इति श्रीपाद्मे महापुराणे पञ्चपञ्चाशत्साहस्र्यां संहितायामुत्तरखण्डे उमापतिनारदसंवादान्तर्गतकृष्णयुधिष्ठिरसंवादे फाल्गुन-कृष्ण-विजया-एकादशी-माहात्म्यं नाम षट्चत्वारिंशोऽध्यायः॥४६॥


\hyperref[sec:ekadashi_mahatmyam_padma_puranam]{\closesub}
\clearpage

\sect{फाल्गुन-शुक्लामलकी-एकादशी-माहात्म्यम्}
\label{sec:padma-phalguna-shuklamalaki}


\uvacha{श्रीकृष्ण उवाच}

\twolineshloka
{माहात्म्यं विजयायाश्च श्रुतं कृष्ण महत्फलम्}
{फाल्गुनस्यार्जुने पक्षे यन्नाम्नी तां वदाधुना}% ॥१॥

\uvacha{श्रीकृष्ण उवाच}

\twolineshloka
{धर्मपुत्र महाभाग शृणु वक्ष्यामितेऽधुना}
{योक्ता पृष्टेन मान्धात्रा वसिष्ठेन महात्मना}% ॥२॥

\twolineshloka
{फाल्गुनस्य विशेषेण विशेषः कथितो नृप}
{आमलकीव्रतं पुण्यं विष्णुलोकफलप्रदम्}% ॥३॥

\twolineshloka
{आमलक्यास्तले गत्वा जागरं तत्र कारयेत्}
{कृत्वा जागरणं रात्रौ गोसहस्रफलं लभेत्}% ॥४॥

\uvacha{मान्धातोवाच}

\twolineshloka
{आमलकी कदा ह्येषा उत्पन्ना द्विजसत्तम}
{एतत्सर्वं ममाचक्ष्व परं कौतूहलं हि मे}% ॥५॥

\twolineshloka
{कस्मादियं पवित्रा च कस्मात्पापप्रणाशिनी}
{कस्माज्जागरणं कृत्वा गोसहस्रफलं लभेत्}% ॥६॥

\uvacha{वसिष्ठ उवाच}

\twolineshloka
{कथयामि महाभाग यथेयमभवत्क्षितौ}
{आमलकी महावृक्षः सर्वपापप्रणाशनः}% ॥७॥

\twolineshloka
{एकार्णवे पुरा जाते नष्टे स्थावरजङ्गमे}
{नष्टे देवासुरगणे प्रणष्टोरगराक्षसे}% ॥८॥

\twolineshloka
{तत्र देवादिदेवेशः परमात्मा सनातनः}
{जगाम ब्रह्मपरममात्मनः पदमव्ययम्}% ॥९॥

\twolineshloka
{ततोऽस्य जाग्रतो ब्रह्ममुखाच्छशिसमप्रभः}
{ष्ठीवनाद्बिन्दुरुत्पन्नः स भूमौ निपपात ह}% ॥१०॥

\twolineshloka
{तस्माद्बिन्दोः समुत्पन्नः स्वयं धात्री नगो महान्}
{शाखाप्रशाखाबहुलः फलभारेण नामितः}% ॥११॥

\twolineshloka
{सर्वेषां चैव वृक्षाणामादिरोहः प्रकीर्तितः}
{ब्रह्मणाथ ततः पश्चात्संसृष्टाश्च इमाः प्रजाः}% ॥१२॥

\twolineshloka
{देवदानवगन्धर्वयक्षराक्षसपन्नगान्}
{असृजद्भगवान्देवो महर्षींश्च तथामलान्}% ॥१३॥

\twolineshloka
{आजग्मुस्तत्र देवास्ते यत्र धात्री हरिप्रिया}
{तां दृष्ट्वा ते महाभाग परं विस्मयमागताः}% ॥१४॥

\twolineshloka
{न जानीम इमं वृक्षं चिन्तयन्तोऽभिसंस्थिताः}
{एवं चिन्तयतां तेषां वागुवाचाशरीरिणी}% ॥१५॥

\twolineshloka
{आमलकी नगो ह्येष प्रवरो वैष्णवो मतः}
{अस्य संस्मरणादेव लभेद्गोदानजं फलम्}% ॥१६॥

\twolineshloka
{स्पर्शनाद्द्विगुणं पुण्यं त्रिगुणं धारणात्तथा}
{तस्मात्सर्वप्रयत्नेन सेव्या आमलकी सदा}% ॥१७॥

\twolineshloka
{सर्वपापहरा प्रोक्ता वैष्णवी पापनाशिनी}
{तस्या मूले स्थितो विष्णुस्तदूर्ध्वे च पितामहः}% ॥१८॥

\twolineshloka
{स्कन्धे च भगवान्रुद्रः संस्थितः परमेश्वरः}
{शाखासु मुनयः सर्वे प्रशाखासु च देवताः}% ॥१९॥

\twolineshloka
{पर्णेषु चासते देवाः पुष्पेषु मरुतस्तथा}
{प्रजानां पतयः सर्वे फलेष्वेव व्यवस्थिताः}% ॥२०॥

\twolineshloka
{सर्वदेवमयी ह्येषा धात्री च कथिता मया}
{तस्मात्पूज्यतमा ह्येषा विष्णुभक्तिपरायणैः}% ॥२१॥

\uvacha{ऋषय ऊचुः}

\twolineshloka
{को भवान्न हि जानीमः कस्मात्कारणतां गतः}
{देवो वा यदि वा चान्यः कथयस्व यथातथम्}% ॥२२॥

\uvacha{वागुवाच}

\twolineshloka
{यः कर्ता सर्वभूतानां भुवनानां च सर्वशः}
{विस्मितान्विदुषः प्रेक्ष्य सोऽहं विष्णुः सनातनः}% ॥२३॥

\twolineshloka
{तच्छ्रुत्वा देवदेवस्य भाषितं ब्रह्मणः सुताः}
{अनादिनिधनं देवं स्तोतुं तत्र प्रचक्रमुः}% ॥२४॥

\twolineshloka
{नमो भूतात्मभूताय आत्मने परमात्मने}
{अच्युताय नमो नित्यमनन्ताय नमो नमः}% ॥२५॥

\twolineshloka
{दामोदराय कवये यज्ञेशाय नमो नमः}
{एवं स्तुतस्तु ऋषिभिस्तुतोष भगवान्हरिः}% ॥२६॥

\onelineshloka*
{प्रत्युवाच महर्षींस्तानभीष्टं किं ददामि वः}

{ऋषय ऊचुः}

\onelineshloka
{यदि तुष्टोऽसि भगवन्नस्माकं हितकाम्यया}% ॥२७॥

\twolineshloka
{व्रतं किञ्चित्समाख्याहि स्वर्गमोक्षफलप्रदम्}
{धनधान्यप्रदं पुण्यमात्मनस्तुष्टिकारकम्}% ॥२८॥

\twolineshloka
{अल्पायासं बहुफलं व्रतानामुत्तमं व्रतम्}
{कृतेन येन देवेश विष्णुलोके महीयते}% ॥२९॥

\uvacha{विष्णुरुवाच}

\twolineshloka
{फाल्गुने शुक्लपक्षे तु पुष्येण द्वादशी यदि}
{भवेत्सा च महापुण्या महापातकनाशिनी}% ॥३०॥

\twolineshloka
{विशेषस्तत्र कर्तव्यः शृणुध्वं द्विजसत्तमाः}
{आमलकीं च सम्प्राप्य जागरं तत्र कारयेत्}% ॥३१॥

\threelineshloka
{सर्वपापविनिर्मुक्तो गोसहस्रफलं लभेत्}
{एतद्वः कथितं विप्रा व्रतानां व्रतमुत्तमम्}
{अर्चयित्वाच्युतं तस्यां विष्णुलोकान्न मुच्यते}% ॥३२॥

\uvacha{ऋषय ऊचुः}

\twolineshloka
{व्रतस्यास्य विधिं ब्रूहि परिपूर्णं कथं भवेत्}
{के मन्त्राः के नमस्काराः देवता का प्रकीर्तिता}% ॥३३॥

\twolineshloka
{कथं दानं कथं स्नानं कश्च पूजाविधिः स्मृतः}
{अर्घार्चनस्य मन्त्रं तु कथयस्व यथातथम्}% ॥३४॥

\uvacha{विष्णुरुवाच}

\twolineshloka
{श्रूयतां यो विधिः सम्यग्व्रतस्यास्य द्विजर्षभाः}
{एकादश्यां निराहारः स्थित्वा चैव परेऽहनि}% ॥३५॥

\twolineshloka
{भोक्ष्येऽहं पुण्डरीकाक्ष शरणं मे भवाच्युत}
{इति कृत्वा तु नियमं दन्तधावनपूर्वकम्}% ॥३६॥

\twolineshloka
{नालपेत्पतितांश्चौरांस्तथा पाषण्डिनो नरान्}
{दुर्वृत्तान्भिन्नमर्यादान्गुरुदारप्रधर्षकान्}% ॥३७॥

\twolineshloka
{अपराह्णे ततः स्नानं विधिना कारयेद्बुधः}
{नद्यां तडागे कूपे वा गृहे वा नियतात्मवान्}% ॥३८॥

\twolineshloka
{मृत्तिकालम्भनं पूर्वं ततः स्नानं च कारयेत्}
{अश्वक्रान्ते रथक्रान्ते विष्णुक्रान्ते वसुन्धरे}% ॥३९॥

\onelineshloka
{मृत्तिके हर मे पापं यन्मया दुष्कृतं कृतम्}% ॥४०॥
[इति मृत्तिकामन्त्रः]

\twolineshloka
{त्वमम्बु सर्वभूतानां जीवनं तनुरक्षकम्}
{स्वेदजोद्भिज्जजातीनां रसानां पतये नमः}% ॥४१॥

\twolineshloka
{स्नातोऽहं सर्वतीर्थेषु ह्रदप्रस्रवणेषु च}
{नदीषु देवखातेषु इदं स्नानं तु मे भवेत्}% ॥४२॥
[इति स्नानमन्त्रः]

\twolineshloka
{जामदग्न्यं मुनिं चैव कारयित्वा हिरण्मयम्}
{माषकस्य सुवर्णस्य तदर्धार्धेन वा पुनः}% ॥४३॥

\twolineshloka
{गृहमागत्य पूजायाः पूजाहोमं तु कारयेत्}
{ततश्चामलकीं गच्छेत्सर्वोपस्करसंयुतः}% ॥४४॥

\twolineshloka
{आमलकीं ततो गत्वा परिशोध्य समन्ततः}
{स्थापयेत्सततं कुम्भमव्रणं मन्त्रपूर्वकम्}% ॥४५॥

\twolineshloka
{पञ्चरत्नसमोपेतं दिव्यगन्धाधिवासितम्}
{छत्रोपानद्युगोपेतं सितचन्दनचर्चितम्}% ॥४६॥

\twolineshloka
{स्रग्दामलम्बितग्रीवं सर्वधूपविधूपितम्}
{दीपमालाकुलं कुर्यात्सर्वतः सुमनोहरम्}% ॥४७॥

\twolineshloka
{तस्योपरि न्यसेत्पात्रं दिव्यलाजैः प्रपूरितम्}
{पात्रोपरि न्यसेद्देवं जामदग्न्यं महाप्रभम्}% ॥४८॥

\twolineshloka
{विशोकाय नमः पादौ जानुनी विश्वरूपिणे}
{उग्राय च ततोऽप्यूरू कटी दामोदराय च}% ॥४९॥

\twolineshloka
{उदरं पद्मनाभाय उरः श्रीवत्सधारिणे}
{चक्रिणे वामबाहुं च दक्षिणं गदिने नमः}% ॥५०॥

\twolineshloka
{वैकुण्ठाय नमः कण्ठमास्यं यज्ञमुखाय वै}
{नासां विशोकनिधये वासुदेवाय चाक्षिणी}% ॥५१॥

\twolineshloka
{ललाटं वामनायेति रामायेति भ्रुवौ नमः}
{सर्वात्मने तु तच्छीर्षं नम इत्यभिपूजयेत्}% ॥५२॥
[इति पूजामन्त्रः]

\twolineshloka
{ततो देवाधिदेवाय अर्घं चैव प्रदापयेत्}
{फलेन चैव शुभ्रेण भक्तियुक्तेन चेतसा}% ॥५३॥

\twolineshloka
{ततो जागरणं कुर्याद्भक्तियुक्तेन चेतसा}
{नृत्यैर्गीतैश्च वादित्रैर्धर्माख्यानैः स्तवैरपि}% ॥५४॥

\twolineshloka
{वैष्णवैश्च तथाख्यानैः क्षपयेत्सर्वशर्वरीम्}
{प्रदक्षिणां ततः कुर्याद्धात्र्या वै विष्णुनामभिः}% ॥५५॥

\twolineshloka
{अष्टाधिकं शतं चैव अष्टाविंशतिरेव वा}
{ततः प्रभाते समये कृत्वा नीराजनं हरेः}% ॥५६॥

\twolineshloka
{ब्राह्मणं पूजयित्वा तु सर्वं तस्मै निवेदयेत्}
{जामदग्न्य घटे तत्र वस्त्रयुग्ममुपानहौ}% ॥५७॥

\twolineshloka
{जामदग्न्यस्वरूपेण प्रीयतां मम केशवः}
{ततश्चामलकीं स्पृष्ट्वा कृत्वा चैव प्रदक्षिणाम्}% ॥५८॥

\twolineshloka
{स्नानं कृत्वा विधानेन ब्राह्मणान्भोजयेत्ततः}
{ततश्च स्वयमश्नीयात्कुटुम्बेन समावृतः}% ॥५९॥

\twolineshloka
{एवं कृतेन यत्पुण्यं तत्सर्वं कथयामि ते}
{सर्वतीर्थेषु यत्पुण्यं सर्वदानेषु यत्फलम्}% ॥६०॥

\twolineshloka
{सर्वयज्ञाधिकं चैव लभते नात्र संशयः}
{एतद्वः सर्वमाख्यातं व्रतानामुत्तमं व्रतम्}% ॥६१॥

\twolineshloka
{एतावदुक्त्वा देवेशस्तत्रैवान्तरधीयत}
{ते चापि ऋषयः सर्वे चक्रुः सर्वमशेषतः}% ॥६२॥

\twolineshloka
{तथा त्वमपि राजेन्द्र कर्तुमर्हसि सत्तम}
{व्रतमेतद्दुराधर्षं सर्वपापप्रमोचनम्}% ॥६३॥

॥इति श्रीपाद्मे महापुराणे पञ्चपञ्चाशत्साहस्र्यां संहितायामुत्तरखण्डे उमापतिनारदसंवादान्तर्गतकृष्णयुधिष्ठिरसंवादे फाल्गुन-शुक्लामलकी-एकादशी-माहात्म्यं नाम सप्तचत्वारिंशोऽध्यायः॥४७॥


\hyperref[sec:ekadashi_mahatmyam_padma_puranam]{\closesub}
\clearpage

\sect{चैत्र-कृष्ण-पापमोचनी-एकादशी-माहात्म्यम्}
\label{sec:padma-chaitra-krishna-papamochani}


\uvacha{युधिष्ठिर उवाच}

\twolineshloka
{फाल्गुनस्य सिते पक्षे श्रुता चामलकी तथा}
{चैत्रस्य कृष्णपक्षे तु किं नामैकादशी भवेत्}% ॥१॥

\uvacha{श्रीकृष्ण उवाच}

\twolineshloka
{शृणु राजेन्द्र वक्ष्यामि आख्यानं पापनाशनम्}
{यल्लोमशोऽब्रवीत्पृष्टो मान्धात्रा चक्रवर्तिना}% ॥२॥

\uvacha{मान्धातोवाच}

\threelineshloka
{भगवन्श्रोतुमिच्छामि लोकानां हितकाम्यया}
{चैत्रस्य प्रथमे पक्षे का नामैकादशी भवेत्}
{को विधिः किं फलं तस्याः कथयस्व प्रसादतः}% ॥३॥

\uvacha{लोमश उवाच}

\twolineshloka
{चैत्रमास्यसिते पक्षे नाम्ना वै पापमोचनी}
{एकादशी समाख्याता पिशाचत्वविनाशिनी}% ॥४॥

\twolineshloka
{शृणु तस्याः प्रवक्ष्यामि कामदां सिद्धिदां नृप}
{कथां विचित्रां शुभदां पापघ्नीं धर्मदायिनीम्}% ॥५॥

\twolineshloka
{पुरा चैत्ररथोद्देशे अप्सरोगणसेविते}
{वसन्तसमये प्राप्ते षट्पदाकुलिते वने}% ॥६॥

\twolineshloka
{गन्धर्वकन्यावादित्रै रमन्ति सह किन्नरैः}
{पाकशासनमुख्याश्च क्रीडान्ते त्रिदिवौकसः}% ॥७॥

\twolineshloka
{नापरं सुखदं किञ्चिद्विना चैत्ररथाद्वनम्}
{तस्मिन्वने तु मुनयस्तपन्ति बहुलं तपः}% ॥८॥

\twolineshloka
{मेधावि नामानमृषिं तत्रस्थं ब्रह्मचारिणम्}
{अप्सरास्तं मुनिवरं मोहनायोपचक्रमे}% ॥९॥

\twolineshloka
{मञ्जुघोषेति विख्याता भावं तस्य वितन्वती}
{क्रोशमात्रं स्थिता तस्य भयादाश्रमसन्निधौ}% ॥१०॥

\twolineshloka
{गायन्ती मधुरं साधु पीडयन्ती विपञ्चिकाम्}
{गायन्तीं तामथालोक्य पुष्पचन्दनसेविताम्}% ॥११॥

\twolineshloka
{कामोऽपि विजयाकाङ्क्षी शिवभक्तान्मुनीश्वरान्}
{तस्याः शरीरे संवासमकरोन्मनसः सुतः}% ॥१२॥

\twolineshloka
{कृत्वा भ्रुवौ धनुष्कोटिं गुणं कृत्वा कटाक्षकम्}
{मार्गणौ नयने कृत्वा पक्ष्मयुक्ते यथाक्रमम्}% ॥१३॥

\twolineshloka
{कुचौ कृत्वा पटकुटीं विजयायोपचक्रमे}
{मञ्जुघोषाभवत्तस्य कामस्यैव वरूथिनी}% ॥१४॥

\twolineshloka
{मेधाविनं मुनिं दृष्ट्वा सापि कामेन पीडिता}
{यौवनोद्भिन्नदेहोऽसौ मेधाव्यपि विराजते}% ॥१५॥

\twolineshloka
{सितोपवीतसहितो दृष्टः स्मर इवापरः}
{मेधावी वसते चासौ च्यवनस्याश्रमे शुभे}% ॥१६॥

\twolineshloka
{मञ्जुघोषा स्थितं तत्र दृष्ट्वा सा मुनिपुङ्गवम्}
{मदनस्य वशं प्राप्ता मन्दं मन्दमगायत}% ॥१७॥

\twolineshloka
{रणद्वलयसंयुक्तां शिञ्जन्नूपुरमेखलाम्}
{गायन्तीं तां तथाभूतां विलोक्य मुनिपुङ्गवः}% ॥१८॥

\twolineshloka
{मदनेन ससैन्येन नीतो मोहवशं बलात्}
{मञ्जुघोषा समागम्य मुनिं दृष्ट्वा तथाविधम्}% ॥१९॥

\twolineshloka
{हावभावकटाक्षैस्तं मोहयामास चाङ्गना}
{अधः संस्थाप्य वीणां सा सस्वजे तं मुनीश्वरम्}% ॥२०॥

\twolineshloka
{वलितेव लता वृक्षं वातवेगेन कम्पितम्}
{सोऽपि रेमे तया सार्धं मेधावी मुनिपुङ्गवः}% ॥२१॥

\twolineshloka
{तस्मिन्नेव ततो दृष्ट्वा तस्यास्तं देहमुत्तमम्}
{शिवतत्वं गतं तस्य कामतत्त्ववशं गतः}% ॥२२॥

\twolineshloka
{न निशां न दिनं सोऽपि रमन्जानाति कामुकः}
{बहुवर्षं गतः कालो मुनेराचारलोपतः}% ॥२३॥

\threelineshloka
{मञ्जुघोषा देवलोकगमनायोपचक्रमे}
{गच्छन्ती तं प्रत्युवाच रमन्तं मुनिसत्तमम्}
{आदेशो दीयतां ब्रह्मन्स्वदेशगमनाय मे}% ॥२४॥

\uvacha{मेधाव्युवाच}

\threelineshloka
{अद्यैव त्वं समायाता प्रदोषादौ वरानने}
{यावत्प्रभातसन्ध्या स्यात्तावत्तिष्ठ ममान्तिके}
{इति श्रुत्वा मुनेर्वाक्यं भयभीता बभूव सा}% ॥२५॥

\twolineshloka
{पुनर्वै रमयामास तमृषिं नृपसत्तम}
{मुनेः शापभयाद्भीता बहुलान्परिवत्सरान्}% ॥२६॥

\twolineshloka
{वर्षाणां पञ्चपञ्चाशन्नवमासदिनत्रयम्}
{सा रेमे मुनिना तस्य निशार्धमिव चाभवत्}% ॥२७॥

\twolineshloka
{सा तं पुनरुवाचाथ तस्मिन्काले गते मुनिम्}
{आदेशो दीयतां ब्रह्मन्गन्तव्यं स्वगृहे मया}% ॥२८॥

\uvacha{मेधाव्युवाच}

\twolineshloka
{प्रभातमधुना चास्ते श्रूयतां वचनं मम}
{सन्ध्या यावच्च कुर्वेऽहं तावत्त्वं वै स्थिरा भव}% ॥२९॥

\twolineshloka
{इति वाक्यं मुनेः श्रुत्वा जातानन्दसमाकुला}
{स्मितं कृत्वा तु सा किञ्चित्प्रत्युवाच शुचिस्मिता}% ॥३०॥

\uvacha{अप्सरा उवाच}

\twolineshloka
{कियत्प्रमाणा विप्रेन्द्र तव सन्ध्या गतानघ}
{मयि प्रसादं कृत्वा तु गतकालो विचार्यताम्}% ॥३१॥

\twolineshloka
{इति तस्या वचः श्रुत्वा विस्मयोत्फुल्ललोचनः}
{गतकालस्य विप्रेन्द्र प्रमाणमकरोत्तदा}% ॥३२॥

\twolineshloka
{समाश्च सप्तपञ्चाशद्गतास्तस्य तया सह}
{चुक्रोध सततस्तस्यै ज्वालामाली बभूव ह}% ॥३३॥

\twolineshloka
{नेत्राभ्यां विस्फुलिङ्गान्स मुञ्चमानोऽतिकोपनः}
{कालरूपां तु तां दृष्ट्वा तपसः क्षयकारिणीम्}% ॥३४॥

\twolineshloka
{दुःखार्जितं क्षयं नीतं तपोदृष्ट्वातयासह}
{सकम्पोष्ठो मुनिस्तत्रप्रत्युवाचाकुलेन्द्रियः}% ॥३५॥

\twolineshloka
{तां शशापाथमेधावी त्वं पिशाची भवेति च}
{धिक्त्वां पापे दुराचारे कुलटे पातकप्रिये}% ॥३६॥

\threelineshloka
{तस्य शापेन सा दग्धा विनयावनता स्थिता}
{उवाच वचनं सुभ्रूः प्रसादं वाञ्छती मुनिम्}
{प्रसादं कुरु विप्रेन्द्र शापस्यानुग्रहं कुरु}% ॥३७॥

\threelineshloka
{सतां सङ्गो हि भवति वचोभिः सप्तभिः पदैः}
{त्वया सह मम ब्रह्मन्नीता वै बहुवत्सराः}
{एतस्मात्कारणात्स्वामिन्प्रसादं कुरु सुव्रत}% ॥३८॥

\uvacha{मुनिरुवाच}

\twolineshloka
{शृणु मे वचनं भद्रे शापानुग्रहकारकम्}
{किं करोमि त्वया पापे क्षयं नीतं महत्तपः}% ॥३९॥

\twolineshloka
{चैत्रस्य कृष्णपक्षे तु भवेदेकादशी शुभा}
{पापमोचनिका नाम सर्वपापक्षयङ्करी}% ॥४०॥

\twolineshloka
{तस्या व्रते कृते शुभ्रे पिशाचत्वं प्रयास्यति}
{इत्युक्त्वा सोऽपि मेधावी जगाम पितुराश्रमम्}% ॥४१॥

\twolineshloka
{तमागतं समालोक्य च्यवनः प्रत्युवाचतम्}
{किमेतद्विहितं पुत्र त्वया पुण्यं क्षयं कृतम्}% ॥४२॥

\uvacha{मेधाव्युवाच}

\twolineshloka
{पातकं वै कृतं तात रमिता चाप्सरा मया}
{प्रायश्चित्तं ब्रूहि तात येन पापक्षयो भवेत्}% ॥४३॥

\uvacha{च्यवन उवाच}

\twolineshloka
{चैत्रस्य चासिते पक्षे नाम्ना वै पापमोचनी}
{अस्या व्रते कृते पुत्र पापराशिः क्षयं व्रजेत्}% ॥४४॥

\twolineshloka
{इति श्रुत्वा पितुर्वाक्यं कृतं तेन व्रतोत्तमम्}
{गतं पापं क्षयं तस्य तपोयुक्तो बभूव सः}% ॥४५॥

\threelineshloka
{साप्येवं मञ्जुघोषा च कृत्वैतद्व्रतमुत्तमम्}
{पिशाचत्वाद्विनिर्मुक्ता पापमोचनिकाव्रतात्}
{दिव्यरूपधरा सा वै गता नाके वराप्सराः}% ॥४६॥

\uvacha{लोमश उवाच}

\twolineshloka
{पापमोचनिकां राजन्ये कुर्वन्ति नरोत्तमाः}
{तेषां पापं च यत्किञ्चित्तत्सर्वं च क्षयं व्रजेत्}% ॥४७॥

\twolineshloka
{पठनाच्छ्रवणाद्राजन्गोसहस्रफलं लभेत्}
{ब्रह्महा हेमहारी च सुरापो गुरुतल्पगः}% ॥४८॥

\twolineshloka
{व्रतस्य चास्य करणात्पापमुक्ता भवन्ति ते}
{बहुपुण्यप्रदं ह्येतत्करणाद्व्रतमुत्तमम्}% ॥४९॥

॥इति श्रीपाद्मे महापुराणे पञ्चपञ्चाशत्साहस्र्यां संहितायामुत्तरखण्डे उमापतिनारदसंवादान्तर्गतकृष्णयुधिष्ठिरसंवादे चैत्र-कृष्ण-पापमोचनी-एकादशी-माहात्म्यं नाम अष्टचत्वारिंशोऽध्यायः॥४८॥


\hyperref[sec:ekadashi_mahatmyam_padma_puranam]{\closesub}
\clearpage

\sect{चैत्र-शुक्ल-कामदा-एकादशी-माहात्म्यम्}
\label{sec:padma-chaitra-shukla-kamada}


\uvacha{युधिष्ठिर उवाच}

\twolineshloka
{वासुदेव नमस्तुभ्यं कथयस्व ममाग्रतः}
{चैत्रस्य शुक्लपक्षे तु किं नामैकादशी भवेत्}% ॥१॥

\uvacha{श्रीकृष्ण उवाच}

\twolineshloka
{शृणुष्वैकमना राजन्कथां पुण्यां पुरातनीम्}
{वसिष्ठो यामकथयत्प्राग्दिलीपाय पृच्छते}% ॥२॥

\uvacha{दिलीप उवाच}

\twolineshloka
{भगवञ्छ्रोतुमिच्छामि कथयस्व प्रसादतः}
{चैत्रमासि सिते पक्षे किन्नामैकादशी भवेत्}% ॥३॥

\uvacha{वसिष्ठ उवाच}

\twolineshloka
{साधु पृष्टं त्वया राजन्कथयामि तवाग्रतः}
{चैत्रस्य शुक्लपक्षे तु कामदा नाम नामतः}% ॥४॥

\twolineshloka
{एकादशी पुण्यतमा पापेन्धनदवानलः}
{शृणु राजन्कथामेतां पापघ्नीं पुण्यदायिनीम्}% ॥५॥

\twolineshloka
{पुरा नागपुरे रम्ये हेमरत्नविभूषिते}
{पुण्डरीकमुखा नागा निवसन्ति महोत्कटाः}% ॥६॥

\twolineshloka
{तस्मिन्पुरे पुण्डरीको राजा राज्यं चकार सः}
{गन्धर्वैः किन्नरैश्चैव अप्सरोभिश्च सेव्यते}% ॥७॥

\twolineshloka
{वराप्सरास्तु ललिता गन्धर्वो ललितस्तथा}
{उभौ रागेण संरक्तौ दम्पती कामपीडितौ}% ॥८॥

\twolineshloka
{रेमाते स्वगृहे रम्ये धनधान्ययुते तदा}
{ललितायाश्च हृदये पतिर्वसति सर्वदा}% ॥९॥

\twolineshloka
{हृदये तस्य ललिता नित्यं वसति भामिनी}
{एकदा पुण्डरीकोऽथ क्रीडते सदसि स्थितः}% ॥१०॥

\twolineshloka
{गीतं गानं प्रकुरुते ललितो दयितां विना}
{पदबन्धस्खलज्जिह्वो बभूव ललितां स्मरन्}% ॥११॥

\twolineshloka
{मनोभावं विदित्वास्य कर्कटो नागसत्तमः}
{पदबन्धच्युतिं तस्य पुण्डरीके न्यवेदयत्}% ॥१२॥

\twolineshloka
{श्रुत्वा कर्कोटकवचः पुण्डरीको भुजङ्गराट्}
{क्रोधसंरक्तनयनो बभूवातिभयङ्करः}% ॥१३॥

\twolineshloka
{शशाप ललितं तत्र गायन्तं मदनातुरम्}
{राक्षसो भव दुर्बुद्धे क्रव्यादः पुरुषादकः}% ॥१४॥

\twolineshloka
{यतः पत्नीवशोपेतो गायमानो ममाग्रतः}
{वचनात्तस्य राजेन्द्र रक्षोरूपो बभूव सः}% ॥१५॥

\twolineshloka
{रौद्राननो विरूपाक्षो दृष्टमात्रो भयङ्करः}
{बाहूयोजनविस्तीर्णौ मुखं कन्दरसन्निभम्}% ॥१६॥

\twolineshloka
{चन्द्रसूर्यनिभे नेत्रे ग्रीवापर्वतसन्निभा}
{नासारन्ध्रे तु विवरे अधरौ योजनायतौ}% ॥१७॥

\twolineshloka
{शरीरं तस्य राजेन्द्र उत्थितं योजनाष्टकम्}
{ईदृशो राक्षसो भूत्वा भुञ्जानः कर्मणः फलम्}% ॥१८॥

\twolineshloka
{ललिता तु तथालोक्य स्वपतिं विकृताकृतिम्}
{चिन्तयामास मनसा दुःखेन महतार्दिता}% ॥१९॥

\twolineshloka
{किं करोमि क्व गच्छामि पतिः शापेन पीडितः}
{इति संस्मृत्य संस्मृत्य मनसा शर्म नालभत्}% ॥२०॥

\twolineshloka
{चचार पतिना सार्धं ललिता गहने वने}
{बभ्राम विपिने दुर्गे कामरूपी स राक्षसः}% ॥२१॥

\twolineshloka
{निर्घृणः पापनिरतो विरूपः पुरुषादकः}
{न सुखं लभते रात्रौ न दिवा पापपीडितः}% ॥२२॥

\twolineshloka
{ललिता दुःखितातीव पतिं दृष्ट्वा तथाविधम्}
{बभ्राम तेन सार्धं सा रुदती गहने वने}% ॥२३॥

\twolineshloka
{दृष्ट्वाश्रमपदं रम्यं मुनिं संशान्तविग्रहम्}
{शीघ्रं जगाम ललिता नमस्कृत्याग्रतः स्थिता}% ॥२४॥

\twolineshloka
{तां दृष्ट्वा स मुनिः प्राह दुःखितां हि दयापरः}
{का त्वं कस्मादिहायाता सत्यं वद ममाग्रतः}% ॥२५॥

\uvacha{ललितोवाच}

\twolineshloka
{वीरधन्वेति गन्धर्वः सुता तस्य महात्मनः}
{ललितां नाम मां विद्धि पत्यर्थमिह चागताम्}% ॥२६॥

\twolineshloka
{भर्ता मे पापदोषेण राक्षसोऽभून्महामुने}
{रौद्ररूपो दुराचारस्तं दृष्ट्वा नास्ति मे सुखम्}% ॥२७॥

\twolineshloka
{साम्प्रतं शाधि मां ब्रह्मन्यत्कृत्यं तद्वद प्रभो}
{कुरुष्व तद् व्रतं भद्रे विधिपूर्वं मयोदितम्}% ॥२८॥

\uvacha{ऋषिरुवाच}

\twolineshloka
{चैत्रमासस्य रम्भोरु शुक्लपक्षोऽस्ति साम्प्रतम्}
{कामदैकादशीनाम पापघ्नी ललिते परा}% ॥२९॥

\twolineshloka
{कुरुष्व तद्व्रतं भद्रे विधिपूर्वं मयोदितम्}
{अस्य व्रतस्य यत्पुण्यं तत्स्वभर्त्रे प्रदीयताम्}% ॥३०॥

\twolineshloka
{दत्ते पुण्येक्षणात्तस्य शापदोषः प्रयास्यति}
{इति श्रुत्वा मुनेर्वाक्यं ललिता हर्षिताभवत्}% ॥३१॥

\twolineshloka
{उपोष्यैकादशीं राजन्द्वादशीदिवसे तथा}
{विप्रस्यैव समीपे तद्वासुदेवस्य चाग्रतः}% ॥३२॥

\twolineshloka
{वाक्यमुवाच ललिता स्वपत्युस्तारणाय वै}
{मया तु तद्व्रतं चीर्णं कामदाया उपोषणम्}% ॥३३॥

\twolineshloka
{तस्य पुण्यप्रभावेन गच्छत्वस्य पिशाचता}
{ललितावचनादेव वर्तमानोऽपि तत्क्षणे}% ॥३४॥

\twolineshloka
{गतपापः स ललितो दिव्यदेहो बभूव ह}
{राक्षसत्वं गतं तस्य प्राप्ता गन्धर्वता पुनः}% ॥३५॥

\twolineshloka
{हेमरत्नसमाकीर्णो रेमे ललितया सह}
{विमानवरमारूढौ पूर्वरूपाधिकौ च तौ}% ॥३६॥

\twolineshloka
{दम्पती अत्यशोभेतां कामदायाः प्रभावतः}
{इति ज्ञात्वा नृपश्रेष्ठ कर्तव्यैषा प्रयत्नतः}% ॥३७॥

\twolineshloka
{लोकानां तु हितार्थाय तवाग्रे कथिता मया}
{ब्रह्महत्यादि पापघ्नी पिशाचत्वविनाशनी}% ॥३८॥

\twolineshloka
{नातः परतरा काचित्त्रैलोक्ये सचराचरे}
{पठनाच्छ्रवणाद्राजन्वाजपेयफलं लभेत्}% ॥३९॥

॥इति श्रीपाद्मे महापुराणे पञ्चपञ्चाशत्साहस्र्यां संहितायामुत्तरखण्डे उमापतिनारदसंवादान्तर्गतकृष्णयुधिष्ठिरसंवादे चैत्र-शुक्ल-कामदा-एकादशी-माहात्म्यं नाम एकोनपञ्चाशत्तमोऽध्यायः॥४९॥


\hyperref[sec:ekadashi_mahatmyam_padma_puranam]{\closesub}
\clearpage

\sect{वैशाख-कृष्ण-वरूथिनी-एकादशी-माहात्म्यम्}
\label{sec:padma-vaishakha-krishna-varuthini ekadashi}


\uvacha{युधिष्ठिर उवाच}

\twolineshloka
{वैशाखस्यासिते पक्षे किन्नामैकादशी भवेत्}
{महिमानं कथय मे वासुदेव नमोऽस्तु ते}% ॥१॥

\uvacha{श्रीकृष्ण उवाच}

\twolineshloka
{सौभाग्यदायिनी राजन्निहलोके परत्र च}
{वैशाखकृष्णपक्षे तु नाम्ना चैव वरूथिनी}% ॥२॥

\twolineshloka
{वरूथिन्या व्रतेनैव सौख्यं भवति सर्वदा}
{पापहानिश्च भवति सौभाग्यप्राप्तिरेव च}% ॥३॥

\twolineshloka
{दुर्भगा या करोत्येनां सा स्त्री सौभाग्यमाप्नुयात्}
{लोकानां चैव सर्वेषां भुक्तिमुक्तिप्रदायिनी}% ॥४॥

\twolineshloka
{सर्वपापहरा नॄणां गर्भवासनिकृन्तनी}
{वरूथिन्या व्रतेनैव मान्धाता स्वर्गतिं गतः}% ॥५॥

\twolineshloka
{धुन्धुमारादयश्चान्ये राजानो बहवस्तथा}
{ब्रह्मकपालनिर्मुक्तो बभूव भगवान्भवः}% ॥६॥

\threelineshloka
{दशवर्षसहस्राणि तपस्तप्यति यो नरः}
{कुरुक्षेत्रे  रविग्रहे स्वर्णभारं ददाति यः}
{तत्तुल्यम्फलमाप्नोतिवरूथिन्याव्रतञ्चरन्}% ॥७॥

\twolineshloka
{श्रद्धावान्यस्तु कुरुते वरूथिन्या व्रतं नरः}
{वाञ्छितं लभते सोऽपि इहलोके परत्र च}% ॥८॥

\twolineshloka
{पवित्रा पावनी ह्येषा महापातकनाशिनी}
{भुक्तिमुक्तिप्रदा चैव कर्तॄणां नृपसत्तम}% ॥९॥

\twolineshloka
{अश्वदानान्नृपश्रेष्ठ गजदानं विशिष्यते}
{गजदानाद्भूमिदानं तिलदानं ततोऽधिकम्}% ॥१०॥

\twolineshloka
{तस्माच्च स्वर्णदानं वै अन्नदानं ततोऽधिकम्}
{अन्नदानात्परं दानं न भूतं न भविष्यति}% ॥११॥

\twolineshloka
{पितृदेवमनुष्याणां तृप्तिरन्नेन जायते}
{तत्समं कविभिः प्रोक्तं कन्यादानं नृपोत्तम}% ॥१२॥

\twolineshloka
{धेनुदानं च तत्तुल्यमित्याह भगवान्स्वयम्}
{प्रोक्तेभ्यः सर्वदानेभ्यो विद्यादानं विशिष्यते}% ॥१३॥

\twolineshloka
{तत्फलं समवाप्नोति नरः कृत्वा वरूथिनीम्}
{कन्यावित्तेन जीवन्ति ये नराः पापमोहिताः}% ॥१४॥

\twolineshloka
{पुण्यक्षयं ते गच्छन्ति निरयं यातनामयम्}
{तस्मात्सर्वप्रयत्नेन न ग्राह्यं कन्यकाधनम्}% ॥१५॥

\twolineshloka
{यश्च गृह्णाति लोभेन कन्यां क्रीत्वा च तद्धनम्}
{सोऽन्यजन्मनि राजेन्द्र ओतुर्भवति निश्चितम्}% ॥१६॥

\twolineshloka
{कन्यां पुण्येन यो दद्याद्यथाशक्ति स्वलङ्कृताम्}
{तत्पुण्यसङ्ख्यां नृपते चित्रगुप्तो न शक्नुयात्}% ॥१७॥

\twolineshloka
{तत्तुल्यं फलमाप्नोति नरः कृत्वा वरूथिनीम्}
{कांस्यं मांसं मसूरांश्च चणकान्कोद्रवांस्तथा}% ॥१८॥

\twolineshloka
{शाकं मधु परान्नं च पुनर्भोजन मैथुने}
{वैष्णवो व्रतकर्ता च दशम्यां दश वर्जयेत्}% ॥१९॥

\twolineshloka
{द्यूतं क्रीडां च निद्रां च ताम्बूलं दन्तधावनम्}
{परापवादं पैशुन्यं स्तेयं हिंसां तथा रतिम्}% ॥२०॥

\twolineshloka
{क्रोधं चैवानृतं वाक्यमेकादश्यां विवर्जयेत्}
{कांस्यं मांसं सुरां क्षौद्रं  तैलं पतितभाषणम्}% ॥२१॥

\twolineshloka
{व्यायामं च प्रवासं च पुनर्भोजनमैथुनम्}
{वृषपृष्ठं मसूरान्नं द्वादश्यां परिवर्जयेत्}% ॥२२॥

\threelineshloka
{अनेन विधिना राजन्विहिता यैर्वरूथिनी}
{सर्वपापक्षयं कृत्वा दद्यात्प्रान्तेऽक्षयां गतिम्}
{रात्रौ जागरणं कृत्वा पूजितो मधुसूदनः}% ॥२३॥

\twolineshloka
{सर्वपापविनिर्मुक्तास्ते यान्ति परमां गतिम्}
{तस्मात्सर्वप्रयत्नेन कर्तव्या पापभीरुभिः}% ॥२४॥

\threelineshloka
{क्षपारितनयाद्भीतो नरः कुर्याद्वरूथिनीम्}
{पठनाच्छ्रवणाद्राजन्गोसहस्रफलं लभेत्}
{सर्वपापविनिर्मुक्तो विष्णुलोके महीयते}% ॥२५॥

॥इति श्रीपाद्मे महापुराणे पञ्चपञ्चाशत्साहस्र्यां संहितायामुत्तरखण्डे उमापतिनारदसंवादान्तर्गतकृष्णयुधिष्ठिरसंवादे वैशाख-कृष्ण-वरूथिनी-एकादशी-माहात्म्यं नाम पञ्चाशत्तमोऽध्यायः॥५०॥


\hyperref[sec:ekadashi_mahatmyam_padma_puranam]{\closesub}
\clearpage

\sect{वैशाख-शुक्ल-मोहिनी-एकादशी-माहात्म्यम्}
\label{sec:padma-vaishakha-shukla-mohini}


\uvacha{युधिष्ठिर उवाच}

\twolineshloka
{वैशाख शुक्लपक्षे तु किन्नामैकादशी भवेत्}
{किं फलं को विधिस्तत्र कथयस्व जनार्दन}% ॥१॥

\uvacha{श्रीकृष्ण उवाच}

\twolineshloka
{इदमेव पुरा पृष्टं रामचन्द्रेण धीमता}
{वसिष्ठं प्रति राजेन्द्र यत्त्वं मामनुपृच्छसि}% ॥२॥

\uvacha{राम उवाच}

\twolineshloka
{भगवञ्छ्रोतुमिच्छामि व्रतानामुत्तमं व्रतम्}
{सर्वपापक्षयकरं सर्वदुःखनिकृन्तनम्}% ॥३॥

\twolineshloka
{मया दुःखानि भुक्तानि सीताविरहजानि तु}
{ततोऽहं भयभीतोऽस्मि पृच्छामि त्वां महामुने}% ॥४॥

\uvacha{वसिष्ठ उवाच}

\twolineshloka
{साधु पृष्टं त्वया राम तवैषा नैष्ठिकी मतिः}
{त्वन्नामग्रहणेनैव पूतो भवति मानवः}% ॥५॥

\twolineshloka
{तथापि कथयिष्यामि लोकानां हितकाम्यया}
{पवित्रं पावनानां च व्रतानामुत्तमं व्रतम्}% ॥६॥

\twolineshloka
{वैशाखस्य सिते पक्षे राम चैकादशी भवेत्}
{मोहिनी नाम सा प्रोक्ता सर्वपापहरापराः}% ॥७॥

\twolineshloka
{मोहजालात्प्रमुच्यन्ते पातकानां समूहतः}
{अस्या व्रत प्रभावेन सत्यं सत्यं वदाम्यहम्}% ॥८॥

\twolineshloka
{अतः कारणतो राम कर्तव्यैषा भवादृशैः}
{पातकानां क्षयकरी महादुःखविनाशिनी}% ॥९॥

\twolineshloka
{शृणुष्वैकमना राम कथां पापहरां पराम्}
{यस्याः श्रवणमात्रेण महापापं प्रणश्यति}% ॥१०॥

\twolineshloka
{सरस्वत्यास्तटे रम्ये पुरी भद्रावती शुभा}
{द्युतिमान्नाम नृपतिस्तत्र राज्यं करोति वै}% ॥११॥

\twolineshloka
{चन्द्रवंशोद्भवो नाम धृतिमान्सत्यसङ्गरः}
{तत्र वैश्यो निवसति धनधान्यसमृद्धिमान्}% ॥१२॥

\twolineshloka
{धनपाल इति ख्यातः पुण्यकर्मप्रवर्तकः}
{प्रपा कूप मठाराम तडाग गृहकारकः}% ॥१३॥

\twolineshloka
{विष्णुभक्तिरतः शान्तस्तस्यासन्पञ्चपुत्रकाः}
{सुमना द्युतिमांश्चैव मेधावी सुकृतस्तथा}% ॥१४॥

\twolineshloka
{पञ्चमो धृष्टबुद्धिश्च महापापरतः सदा}
{परस्त्रीसङ्गनिरतो विटगोष्ठी विशारदः}% ॥१५॥

\twolineshloka
{द्यूतादि व्यसनासक्तः परस्त्री रतिलालसः}
{न च देवार्चने बुद्धिर्नपितॄन्न द्विजान्प्रति}% ॥१६॥

\twolineshloka
{अन्यायवर्ती दुष्टात्मा पितुर्द्रव्यक्षयङ्करः}
{अभक्ष्यभक्षकः पापी सुरापाने रतः सदा}% ॥१७॥

\twolineshloka
{वेश्याकण्ठे क्षिप्तबाहुर्भ्रमन्दुष्टश्चतुष्पथे}
{पित्रा निष्कासितो गेहात्परित्यक्तश्च बान्धवैः}% ॥१८॥

\twolineshloka
{स्वदेहभूषणान्येव क्षयं नीतानि तेन वै}
{गणिकाभिः परित्यक्तो निन्दितश्च धनक्षयात्}% ॥१९॥

\twolineshloka
{ततश्चिन्तापरो जातो वस्त्रहीनः क्षुधार्दितः}
{किं करोमि क्वगच्छामि केनोपायेन जीव्यते}% ॥२०॥

\twolineshloka
{तस्करत्वं समारब्धं तत्रैव नगरे पितुः}
{गृहीतो राजपुरुषैर्मुक्तश्च पितृगौरवात्}% ॥२१॥

\twolineshloka
{पुनर्बद्धः पुनस्त्यक्तः पुनर्बद्धः ससम्भ्रमैः}
{धृष्टबुद्धिर्दुराचारो निबध्य निगडै र्दृढैः}% ॥२२॥

\twolineshloka
{कशाघातैस्ताडितश्च पीडितश्च पुनः पुनः}
{न स्थातव्यं हि मन्दात्मंस्त्वया मद्देशगोचरे}% ॥२३॥

\twolineshloka
{एवमुक्त्वा ततो राज्ञा मोचितो दृढबन्धनात्}
{निर्जगाम भयात्तस्य गतोऽसौ गहनं वनम्}% ॥२४॥

\twolineshloka
{क्षुत्तृषापीडितश्चायमितश्चेतश्च धावति}
{सिंहवन्निजघानासौ मृग शूकर चित्रलान्}% ॥२५॥

\twolineshloka
{आमिषाहार निरतो वने तिष्ठति सर्वदा}
{करे शरासनं कृत्वा निषङ्गं पृष्ठ सङ्गतम्}% ॥२६॥

\twolineshloka
{अरण्यचारिणो हन्ति पक्षिणश्च पदाचरन्}
{चकोरांश्च मयूरांश्च कङ्क तित्तिर मूषिकान्}% ॥२७॥

\twolineshloka
{एतानन्यान्हिनस्त्यन्धो धृष्टबुद्धिस्तु निर्घृणः}
{पूर्वजन्मकृतैः पापैर्निमग्नः पापकर्दमे}% ॥२८॥

\twolineshloka
{दुःखशोकसमाविष्टः पीड्यमानो दिवानिशम्}
{कौण्डिन्यस्याश्रमपदं प्राप्तः पुण्यागमात्क्वचित्}% ॥२९॥

\twolineshloka
{माधवे मासि जाह्नव्याः कृतस्नानं तपोधनम्}
{आससाद धृष्टबुद्धिः शोकभारेण पीडितः}% ॥३०॥

\twolineshloka
{तद्वस्त्रबिन्दुस्पर्शेन गतपापो हताशुभः}
{कौण्डिन्यस्याग्रतः स्थित्वा प्रत्युवाच कृताञ्जलि}% ॥३१॥

\uvacha{धृष्टबुद्धिरुवाच}

\twolineshloka
{भो भो ब्रह्मन्द्विजश्रेष्ठ दयां कृत्वा ममोपरि}
{येन पुण्यप्रभावेन मुक्तिर्भवति तद्वद}% ॥३२॥

\uvacha{कौण्डिन्य उवाच}

\twolineshloka
{शृणुष्वैकमनाभूत्वा येन पापक्षयस्तव}
{वैशाखस्य सिते पक्षे मोहिनी नाम विश्रुता}% ॥३३॥

\twolineshloka
{एकादशीव्रतं तस्याः कुरु मद्वाक्यनोदितः}
{मेरुतुल्यानि पापानि क्षयं गच्छन्ति देहिनाम्}% ॥३४॥

\twolineshloka
{बहुजन्मार्जितान्येषा मोहिनी समुपोषिता}
{इति वाक्यं मुनेः श्रुत्वा धृष्टबुद्धिः प्रसन्नधीः}% ॥३५॥

\twolineshloka
{व्रतं चकार विधिवत्कौण्डिन्यस्योपदेशतः}
{कृते व्रते नृपश्रेष्ठ गतपापो बभूव सः}% ॥३६॥

\twolineshloka
{दिव्यदेहस्ततो भूत्वा गरुडोपरिसंस्थितः}
{जगाम वैष्णवं लोकं सर्वोपद्रव वर्जितम्}% ॥३७॥

\twolineshloka
{इतीदृशं रामचन्द्र उत्तमं मोहिनी व्रतम्}
{नातः परतरं किञ्चित्त्रैलोक्ये सचराचरे}% ॥३८॥

\twolineshloka
{यज्ञादितीर्थदानानि कलां नार्हन्ति षोडशीम्}
{पठनाच्छ्रवणाद्राजन्गोसहस्र फलं लभेत्}% ॥३९॥

॥इति श्रीपाद्मे महापुराणे पञ्चपञ्चाशत्साहस्र्यां संहितायामुत्तरखण्डे उमापतिनारदसंवादान्तर्गतकृष्णयुधिष्ठिरसंवादे वैशाख-शुक्ल-मोहिनी-एकादशी-माहात्म्यं नाम नामैकपञ्चाशत्तमोऽध्यायः॥५१॥


\hyperref[sec:ekadashi_mahatmyam_padma_puranam]{\closesub}
\clearpage

\sect{ज्येष्ठ-कृष्णापरा-एकादशी-माहात्म्यम्}
\label{sec:padma-jyeshtha-krishnapara}


\uvacha{युधिष्ठिर उवाच}

\twolineshloka
{ज्येष्ठस्य कृष्णपक्षे तु किं नामैकादशी भवेत् 	।}
{श्रोतुमिच्छामि माहात्म्यं तद्वदस्व जनार्दन}% ॥१॥

\uvacha{श्रीकृष्ण उवाच}

\twolineshloka
{साधु पृष्टं त्वया राजन्लोकानां हितकाम्यया}
{बहुपुण्यप्रदा ह्येषा महापातकनाशिनी}% ॥२॥

\twolineshloka
{अपरा नाम राजेन्द्र अपरा पुत्रदायिनी}
{लोके प्रसिद्धितां याति अपरां यस्तु सेवते}% ॥३॥

\twolineshloka
{ब्रह्महत्याभिभूतोऽपि गोत्रहा भ्रूणहा तथा}
{परापवादवादी च परस्त्री रसिकोऽपि च}% ॥४॥

\twolineshloka
{अपरा सेवनाद्राजन्विपाप्मा भवति ध्रुवम्}
{कूटसाक्ष्यं कूटमानं तुलाकूटं करोति यः}% ॥५॥

\twolineshloka
{कूटवेदं पठेद्यस्तु कूटशास्त्रं तथैव च}
{ज्योतिषी गणकः कूटः कूटायुर्वैदिको भिषक्}% ॥६॥

\twolineshloka
{कूटसाक्ष्य समायुक्तो विज्ञेया नरकौकसः}
{अपरा सेवनाद्राजन्पापैर्मुक्ता भवन्ति ते}% ॥७॥

\twolineshloka
{क्षत्त्रियः क्षात्रधर्मं यस्त्यक्त्वा युद्धात्पलायते}
{स याति नरकं घोरं स्वीयधर्मबहिष्कृतः}% ॥८॥

\twolineshloka
{अपरा सेवनात्सोऽपि पापं त्यक्त्वा दिवं व्रजेत्}
{विद्यावान्यः स्वयं शिष्यो गुरुनिन्दां करोति च}% ॥९॥

\twolineshloka
{स महापातकैर्युक्तो निरयं याति दारुणम्}
{अपरा सेवनात्सोऽपि सद्गतिं प्राप्नुयान्नरः}% ॥१०॥

\twolineshloka
{महिमानमपरायाः शृणु राजन्वदाम्यहम्}
{मकरस्थे रवौ माघे प्रयागे यत्फलं नृणाम्}% ॥११॥

\twolineshloka
{काश्यां यत्प्राप्यते पुण्यमुपरागे निमज्जनात्}
{गयायां पिण्डदानेन पितॄणां तृप्तिदो यथा}% ॥१२॥

\twolineshloka
{सिंहस्थिते देवगुरौ गौतम्यां स्नातको नरः}
{कन्यागते गुरौ राजन्कृष्णवेणी निमज्जनात्}% ॥१३॥

\twolineshloka
{यत्फलं समवाप्नोति कुम्भकेदार दर्शनात्}
{बदर्याश्रमयात्रायां तत्तीर्थसेवनादपि}% ॥१४॥

\twolineshloka
{यत्फलं समवाप्नोति कुरुक्षेत्रे रविग्रहे}
{गजाश्व हेमदानेन यज्ञं कृत्वा सदक्षिणम्}% ॥१५॥

\twolineshloka
{तादृशं फलमाप्नोति अपरा व्रतसेवनात्}
{अर्धप्रसूतां गां दत्वा सुवर्णं वसुधां तथा}% ॥१६॥

\twolineshloka
{नरो यत्फलमाप्नोति अपराया व्रतेन तत्}
{पापद्रुमकुठारीयं पापेन्धन दवानलः}% ॥१७॥

\twolineshloka
{पापान्धकारतरणिः पापसारङ्ग केसरी}
{बुद्बुदा इव तोयेषु पुत्तिका इव जन्तुषु}% ॥१८॥

\twolineshloka
{जायन्ते मरणायैव एकादश्या व्रतं विना}
{अपरां समुपोष्यैव पूजयित्वा त्रिविक्रमम्}% ॥१९॥

\threelineshloka
{सर्वपापविनिर्मुक्तो विष्णुलोके महीयते}
{लोकानां च हितार्थाय तवाग्रे कथितं मया}
{पठनाच्छ्रवणाद्राजन्गोसहस्रफलं लभेत्}% ॥२०॥

॥इति श्रीपाद्मे महापुराणे पञ्चपञ्चाशत्साहस्र्यां संहितायामुत्तरखण्डे उमापतिनारदसंवादान्तर्गतकृष्णयुधिष्ठिरसंवादे ज्येष्ठ-कृष्णापरा-एकादशी-माहात्म्यं नाम द्विपञ्चाशत्तमोऽध्यायः॥५२॥


\hyperref[sec:ekadashi_mahatmyam_padma_puranam]{\closesub}
\clearpage

\sect{ज्येष्ठ-शुक्ल-निर्जला-एकादशी-माहात्म्यम्}
\label{sec:padma-jyeshtha-shukla-nirjala}


\uvacha{युधिष्ठिर उवाच}

\twolineshloka
{अपरायाश्च माहात्म्यं श्रुतं सर्वं जनार्दन}
{ज्येष्ठस्य शुक्लपक्षे तु स्याद्या तां वद मानद}% ॥१॥

\uvacha{श्रीकृष्ण उवाच}

\twolineshloka
{एनां वक्ष्यति धर्मात्मा व्यासः सत्यवतीसुतः}
{सर्वशास्त्रार्थतत्त्वज्ञो वेदवेदाङ्गपारगः}% ॥२॥

\uvacha{युधिष्ठिर उवाच}

\twolineshloka
{श्रुता मे मानवा धर्मा वासिष्ठाश्च श्रुता मया}
{द्वैपायन यथावत्त्वं वैष्णवान्वक्तुमर्हसि}% ॥३॥

\uvacha{श्रीवेदव्यास उवाच}

\twolineshloka
{श्रुतास्तु मानवा धर्मा वैदिकाश्च श्रुतास्त्वया}
{कलौ युगे न शक्यन्ते ते वै कर्तुं नराधिप}% ॥४॥

\twolineshloka
{सुखोपायमल्पधनमल्पक्लेशं महाफलम्}
{पुराणानां च सर्वेषां सारभूतं महामते}% ॥५॥

\twolineshloka
{एकादश्यां न भुञ्जीत पक्षयोरुभयोरपि}
{द्वादश्यां तु शुचिर्भूत्वा पुष्पैः सम्पूज्य केशवम्}% ॥६॥

\twolineshloka
{अन्नं भुञ्जीत सत्कृत्य पश्चाद्विप्रपुरःसरम्}
{सूतकेऽपि न भोक्तव्यं नाशौचे च जनाधिप}% ॥७॥

\twolineshloka
{यावज्जीवं व्रतमिदं कर्तव्यं पुरुषर्षभ}
{स्वर्गतिं प्राप्तुमिच्छद्भिरत्र नैवास्ति संशयः}% ॥८॥

\twolineshloka
{आपपापा दुराचाराः पापिष्ठा धर्मवर्जिताः}
{एकादश्यां न भुञ्जाना न ते यान्ति यमान्तिकम्}% ॥९॥

\twolineshloka
{इति तद्वचनं श्रुत्वा कम्पितोऽश्वत्थपत्रवत्}
{भीमसेनो महाबाहुर्नत्वोवाच गुरुं प्रति}% ॥१०॥

\uvacha{भीमसेन उवाच}

\twolineshloka
{पितामह महाबुद्धे शृणु मे परमं वचः}
{युधिष्ठिरश्च कुन्ती च तथा द्रुपदनन्दिनी}% ॥११॥

\twolineshloka
{अर्जुनो नकुलश्चैव सहदेवस्तथैव च}
{एकादश्यां न भुञ्जन्ति कदाचिदपि सुव्रताः}% ॥१२॥

\twolineshloka
{ते मां ब्रुवन्ति वै नित्यं मा भुङ्क्ष्व त्वं वृकोदर}
{अहं तानब्रुवं तात बुभुक्षा दुःसहा मम}% ॥१३॥

\twolineshloka
{दानं दास्यामि विधिवत्पूजयिष्यामि केशवम्}
{भीमसेनवचः श्रुत्वा व्यासो वचनमब्रवीत्}% ॥१४॥

\uvacha{व्यास उवाच}

\twolineshloka
{यदि स्वर्गमभीष्टं ते नरकं दुष्टमेव च}
{एकादश्यां न भोक्तव्यं पक्षयोरुभयोरपि}% ॥१५॥

\uvacha{भीमसेन उवाच}

\twolineshloka
{पितामह महाबुद्धे कथयामि तवाग्रतः}
{एकभक्ते न शक्नोमि उपवासे कुतः प्रभो}% ॥१६॥

\twolineshloka
{वृकोऽपिनाम यो वह्निः स सदा जठरे मम}
{अतिवेलं यदाश्नामि तदा समुपशाम्यति}% ॥१७॥

\threelineshloka
{नैकं शक्नोम्यहं कर्तुमुपवासं महामुने}
{येनैव प्राप्यते स्वर्गस्तत्कर्ताऽस्मि यथातथम्}
{तदेकं वद निश्चित्य येन श्रेयोऽहमाप्नुयाम्}% ॥१८॥

\uvacha{व्यास उवाच}

\twolineshloka
{वृषस्थे मिथुनस्थे वा यदा चैकादशी भवेत्}
{ज्येष्ठमासे प्रयत्नेन सोपोष्योदकवर्जिता}% ॥१९॥

\twolineshloka
{गण्डूषाचमनं वारि वर्जयित्वोदकं बुधः}
{उपभुञ्जीत नैवेह व्रतभङ्गोऽन्यथा भवेत्}% ॥२०॥

\twolineshloka
{उदयादुदयं यावद्वर्जयित्वोदकं नरः}
{श्रूयतां समवाप्नोति द्वादशद्वादशी फलम्}% ॥२१॥

\twolineshloka
{ततः प्रभाते विमले द्वादश्यां स्नानमाचरेत्}
{जलं सुवर्णं दत्त्वा च द्विजातिभ्यो यथाविधि}% ॥२२॥

\twolineshloka
{भुञ्जीत कृतकृत्यस्तु ब्राह्मणैः सहितो वशी}
{एवं कृते च यत्पुण्यं भीमसेन शृणुष्व तत्}% ॥२३॥

\twolineshloka
{संवत्सरे तु याश्चैव एकादश्यो भवन्ति हि}
{तासां फलमवाप्नोति ह्यत्र मे नास्ति संशयः}% ॥२४॥

\twolineshloka
{इति मां केशवः प्राह शङ्खचक्रगदाधरः}
{सर्वान्परित्यज्य पुमान्मामेकं शरणं व्रजेत्}% ॥२५॥

\twolineshloka
{एकादश्यां निराहारस्ततः पापात्प्रमुच्यते}
{द्रव्यशुद्धिः कलौ नास्ति संस्कारः स्मार्त एव च}% ॥२६॥

\twolineshloka
{वैदिकस्तु कुतश्चापि प्राप्ते दुष्टे कलौ युगे}
{किं नु ते बहुनोक्तेन वायुपुत्र पुनः पुनः}% ॥२७॥

\twolineshloka
{एकादश्यां न भुञ्जीत पक्षयोरुभयोरपि}
{एकादश्यां सिते पक्षे ज्येष्ठेमास्युदकं विना}% ॥२८॥

\twolineshloka
{पुण्यं फलमवाप्नोति तच्छृणुष्व वृकोदर}
{संवत्सरे तु या प्रोक्ताः शुक्लः कृष्णा वृकोदर}% ॥२९॥

\twolineshloka
{उपोषिता हि सर्वाः स्युरेकादश्यो न संशयः}
{धनधान्यप्रदा पुण्या पुत्रारोग्यशुभप्रदा}% ॥३०॥

\twolineshloka
{उपोषिता नरव्याघ्र इति सत्यं ब्रवीमि ते}
{यमदूता महाकायाः करालाः कृष्णरूपिणः}% ॥३१॥

\twolineshloka
{दण्डपाशधरा रौद्रा नोपसर्पन्ति तं नरम्}
{पीताम्बरधरा सौम्याश्चक्रहस्ता मनोजवाः}% ॥३२॥

\twolineshloka
{अन्तकालेन यन्त्येते वैष्णवान्वैष्णवीपुरीम्}
{तस्मात्सर्वप्रयत्नेन उपोष्योदकवर्जिता}% ॥३३॥

\twolineshloka
{जलधेनुं तदा दत्त्वा सर्वपापैः प्रमुच्यते}
{ततस्त्वमस्यां कौन्तेय सोपवासोऽर्चनं हरेः}% ॥३४॥

\twolineshloka
{कुरु सर्वप्रयत्नेन सर्वपापप्रशान्तये}
{स्वप्नेन मेऽपराधोस्ति दन्तरागतयापि वा}% ॥३५॥

\twolineshloka
{भोक्ष्येपरेऽह्नि देवेश ह्यशनं वासराद्धरेः}
{इत्युच्चार्य ततो मन्त्र उपवासपरो भवेत्}% ॥३६॥

\twolineshloka
{सर्वपापविनाशाय श्रद्धा दम समन्वितः}
{मेरुमन्दरमात्राघं स्त्रिया पुंसा च यत्कृतम्}% ॥३७॥

\twolineshloka
{सर्वं तद्भस्मतां याति एकादश्याः प्रभावतः}
{न शक्नुवन्ति ये दातुं जलधेनुं नराधिप}% ॥३८॥

\twolineshloka
{सकाञ्चनः प्रदातव्यो घटको वस्त्रसंयुतः}
{तोयस्य नियमं योऽस्या कुरुते वै स पुण्यभाक्}% ॥३९॥

\twolineshloka
{फलं कोटिसुवर्णस्य यामेयामे श्रुतं फलम्}
{स्नानं दानं जपं होमं यदस्यां कुरुते नरः}% ॥४०॥

\twolineshloka
{तत्सर्वं चाक्षयं प्राप्तमेतत्कृष्णप्रभाषितम्}
{किं वापरेण धर्मेण निर्जलैकादशीं विना}% ॥४१॥

\twolineshloka
{उपोष्य सम्यग्विधिवद्वैष्णवं पदमाप्नुयात्}
{सुवर्णमन्नं वासो वा यदस्यां सम्प्रदीयते}% ॥४२॥

\twolineshloka
{तदस्य कुरुशार्दूल सर्वं चाप्यक्षयं भवेत्}
{एकादश्यां दिने योऽन्नं भुङ्क्ते पापं भुनक्ति सः}% ॥४३॥

\twolineshloka
{इहलोके स चाण्डालो मृतः प्राप्नोति दुर्गतिम्}
{ये च दास्यन्ति दानानि द्वादश्यां समुपोषिताः}% ॥४४॥

\twolineshloka
{ज्येष्ठमासे सिते पक्षे प्राप्स्यन्ति परमं पदम्}
{ब्रह्महा मद्यपः स्तेनो गुरुद्वेषी सदानृती}% ॥४५॥

\twolineshloka
{मुच्यन्ते पातकैः सर्वैर्निर्जलायैरुपोषिता}
{विशेषं शृणु कौन्तेय निर्जलैकादशी दिने}% ॥४६॥

\twolineshloka
{यत्कर्तव्यं नरैः स्त्रीभिर्दानं श्रद्धासमन्वितैः}
{जलशायी च सम्पूज्यो देया धेनुस्तथाम्मयी}% ॥४७॥

\twolineshloka
{प्रत्यक्षा वा नृपश्रेष्ठ घृतधेनुरथापि वा}
{दक्षिणाभिः सुपुष्टाभिर्मिष्टान्नैश्च पृथग्विधैः}% ॥४८॥

\twolineshloka
{तोषणीयाः प्रयत्नेन द्विजा धर्मभृतां वर}
{तुष्टा भवन्ति वै विप्रास्तैस्तुष्टैर्मोक्षदो हरिः}% ॥४९॥

\twolineshloka
{आत्मद्रोहः कृतस्तैर्हि यैरेषा न ह्युपोषिता}
{पापात्मानो दुराचारा मुष्टास्ते नात्र संशयः}% ॥५०॥

\twolineshloka
{कुलानां शतमागामि अतीतानां तथा शतम्}
{आत्मना सह तैर्नीतं वासुदेवस्य मन्दिरम्}% ॥५१॥

\twolineshloka
{शान्तैर्दान्तैर्दानपरैरर्चयद्भिस्तथा हरिम्}
{कुर्वद्भिर्जागरं रात्रौ यैरेषा समुपोषिता}% ॥५२॥

\twolineshloka
{अन्नं वस्त्रं तथा गावो जलं शय्यासनं शुभम्}
{कमण्डलुं तथा छत्रं दातव्यं निर्जला दिने}% ॥५३॥

\twolineshloka
{उपानहौ यो ददाति पात्रभूते द्विजोत्तमे}
{स सौवर्णेन यानेन स्वर्गलोके महीयते}% ॥५४॥

\twolineshloka
{यश्चेमां शृणुयाद्भक्त्या यश्चापि परिकीर्तयेत्}
{उभौ तौ स्वर्गमाप्नोति नात्र कार्या विचारणा}% ॥५५॥

\twolineshloka
{यत्फलं सन्निहत्यायां राहुग्रस्ते दिवाकरे}
{कृत्वा श्राद्धं लभेन्मर्त्यस्तदस्याः श्रवणादपि}% ॥५६॥

\twolineshloka
{नियमं च प्रकर्तव्यं दन्तधावनपूर्वकम्}
{एकादश्यां निराहारो वर्जयिष्यामि वै जलम्}% ॥५७॥

\twolineshloka
{केशवप्रीणनार्थाय अन्यदाचमनादृते}
{द्वादश्यां देवदेवेशः पूजनीयस्त्रिविक्रमः}% ॥५८॥

\twolineshloka
{गन्धैर्धूपैस्तथा पुष्पैर्वासोभिः प्रियदर्शनैः}
{पूजयित्वा विधानेन मन्त्रमेतमुदीरयेत्}% ॥५९॥

\twolineshloka
{देवदेव हृषीकेश संसारार्णवतारक}
{उदकुम्भप्रदानेन नय मां परमां गतिम्}% ॥६०॥

\twolineshloka
{ज्येष्ठे मासि तु वै भीम या शुक्लैकादशी शुभा}
{निर्जला समुपोष्यात्र जलकुम्भान्सशर्करान्}% ॥६१॥

\twolineshloka
{प्रदाय विप्रमुख्येभ्यो मोदते विष्णुसन्निधौ}
{ततः कुम्भाः प्रदातव्या ब्राह्मणानां  च भक्तितः}% ॥६२॥

\twolineshloka
{भोजयित्वा ततो विप्रान्स्वयं भुञ्जीत तत्परः}
{एवं यः कुरुते पूर्णां द्वादशीं पापनाशिनीम्}% ॥६३॥

\threelineshloka
{सर्वपापविनिर्मुक्तो पदं गच्छत्यनामयम्}
{ततः प्रभृति भीमेन कृता ह्येकादशी शुभा}
{पाण्डवद्वादशी नाम्ना लोके ख्याता बभूव ह}% ॥६४॥

॥इति श्रीपाद्मे महापुराणे पञ्चपञ्चाशत्साहस्र्यां संहितायामुत्तरखण्डे उमापतिनारदसंवादान्तर्गतकृष्णयुधिष्ठिरसंवादे ज्येष्ठ-शुक्ल-निर्जला-एकादशी-माहात्म्यं नाम त्रिपञ्चाशत्तमोऽध्यायः॥५३॥


\hyperref[sec:ekadashi_mahatmyam_padma_puranam]{\closesub}
\clearpage

\sect{आषाढ-कृष्ण-योगिनी-एकादशी-माहात्म्यम्}
\label{sec:padma-ashadha-krishna-yogini}


\uvacha{युधिष्ठिरउवाच}

\twolineshloka
{आषाढकृष्णपक्षे तु किं वै एकादशी भवेत्}
{कथयस्व प्रसादेन वासुदेव ममाग्रतः}% ॥१॥

\uvacha{श्रीकृष्ण उवाच}

\twolineshloka
{व्रतानामुत्तमं राजन्कथयामि तवाग्रतः}
{सर्वपापक्षयकरं सर्वमुक्तिप्रदायकम्}% ॥२॥

\twolineshloka
{आषाढस्यासिते पक्षे योगिनी नाम नामतः}
{एकादशी नृपश्रेष्ठ महापातकनाशिनी}% ॥३॥

\twolineshloka
{संसारार्णवमग्नानां पोतभूता सनातनी}
{जगत्त्रये सारभूता योगिनी व्रतकारिणाम्}% ॥४॥

\twolineshloka
{कथयामि तवाग्रेऽहं कथां पौराणिकीं शुभाम्}
{अलकायां राजराजः शिवभक्तिपरायणः}% ॥५॥

\twolineshloka
{तस्यासीत्पुष्पबटुको हेममालीति नामतः}
{तस्य पत्नी सुरूपा च विशालाक्षीति नामतः}% ॥६॥

\twolineshloka
{स तस्यां चासक्तमना कामपाशवशं गतः}
{मानसात्पुष्पनिचयमानीय स्वगृहे स्थितः}% ॥७॥

\twolineshloka
{पत्नीप्रेमरसासक्तो न कुबेरालयं गतः}
{कुबेरो देवसदने करोति शिवपूजनम्}% ॥८॥

\twolineshloka
{मध्याह्नसमये राजन्पुष्पागमसमीक्षकः}
{हेममाली स्वभवने रमते कान्तया सह}% ॥९॥

\threelineshloka
{यक्षराट्प्रत्युवाचाथ कालातिक्रमकोपितः}
{कस्मान्नायाति भो यक्षा हेममाली दुरात्मवान्}
{निश्चयः क्रियतामस्य इत्युवाच पुनः पुनः}% ॥१०॥

\uvacha{यक्षा ऊचुः}

\twolineshloka
{वनिताकामुको गेहे रमते स्वेच्छया नृप}
{तेषां वाक्यं समाकर्ण्य कुबेरः कोपपूरितः}% ॥११॥

\twolineshloka
{आह्वयामास तं तूर्णं बटुकं हेममालिनम्}
{ज्ञात्वा कालात्ययं सोऽपि भयव्याकुललोचनः}% ॥१२॥

\twolineshloka
{अस्नात एव आगत्य कुबेरस्याग्रतः स्थितः}
{तं दृष्ट्वा धनदः क्रुद्धः क्रोधसंरक्तलोचनः}% ॥१३॥
\onelineshloka*
{प्रत्युवाच रुषाविष्टः कोपप्रस्फुरिताधरः}

\uvacha{धनद उवाच}
\onelineshloka
{आः पाप दुष्ट दुर्वृत्त कृतवान्देवहेलनम्}% ॥१४॥

\twolineshloka
{अष्टादशकुष्ठवृतो वियुक्तः कान्तया तया}
{अस्मात्स्थानादपध्वस्तो गच्छ स्वप्नमथाधम}% ॥१५॥

\twolineshloka
{इत्युक्ते वचने तस्य तस्मात्स्थानात्पपात सः}
{महादुःखाभिभूतश्च कुष्ठैः पीडितविग्रहः}% ॥१६॥

\twolineshloka
{न सुखं दिवसे तस्य न निद्रां लभते निशि}
{छायायां पीडिततनुर्निदाघेऽत्यन्तपीडितः}% ॥१७॥

\twolineshloka
{शिवपूजाप्रभावेन स्मृतिस्तस्य न लुप्यते}
{पातकेनाभिभूतोऽपि पूर्वं कर्म स्मरत्यसौ}% ॥१८॥

\twolineshloka
{भ्रममाणस्ततो गच्छन्हिमाद्रिं पर्वतोत्तमम्}
{तत्रापश्यन्मुनिवरं मार्कण्डेयं तपोनिधिम्}% ॥१९॥

\twolineshloka
{यस्यायुर्विद्यते राजन्ब्रह्मणो वयसा समम्}
{ववन्दे चरणौ तस्य दूरतः पापकर्मकृत्}% ॥२०॥

\twolineshloka
{मार्कण्डेयो मुनिवरो दृष्ट्वा तं कम्पितं तथा}
{परोपकरणार्थाय समाहूयेदमब्रवीत्}% ॥२१॥

\twolineshloka
{कस्मात्कुष्ठाभिभूतस्त्वं कुतो निन्द्यतरो ह्यसि}
{इत्युक्तः स प्रत्युवाच मार्कण्डेयं महामुनिम्}% ॥२२॥

\uvacha{हेममाल्युवाच}

\twolineshloka
{राजराजस्यानुचरो हेममालीति नामतः}
{मानसात्पद्मनिचयमानीय प्रत्यहं मुने}% ॥२३॥

\twolineshloka
{शिवपूजनवेलायां कुबेराय समर्पये}
{एकस्मिन्दिवसे चैव कालश्चाविदितो मया}% ॥२४॥

\twolineshloka
{पत्नीसौख्यप्रसक्तेन शोकव्याकुलचेतसा}
{ततः क्रुद्धेन शप्तोऽस्मि राजराजेन वै मुने}% ॥२५॥

\twolineshloka
{कुष्ठाभिभूतः सञ्जातो वियुक्तः कान्तया तया}
{अधुना तव सान्निध्यं प्राप्तोऽस्मि शुभकर्मणा}% ॥२६॥

\twolineshloka
{सतां स्वभावतश्चित्तं परोपकरणे क्षमम्}
{इति ज्ञात्वा मुनिश्रेष्ठ मां प्रशाधि कृतागसम्}% ॥२७॥

\uvacha{मार्कण्डेय उवाच}

\twolineshloka
{त्वया सत्यमिह प्रोक्तं नासत्यं भाषितं यतः}
{अतो व्रतोपदेशं ते कथयामि शुभप्रदम्}% ॥२८॥

\twolineshloka
{आषाढे कृष्णपक्षे तु योगिनीव्रतमाचर}
{अस्य व्रतस्य पुण्येन कुष्ठं यास्यति वै ध्रुवम्}% ॥२९॥

\twolineshloka
{इति वाक्यमृषेः श्रुत्वा दण्डवत्पतितो भुवि}
{उत्थापितः स मुनिना बभूवातीव हर्षितः}% ॥३०॥

\twolineshloka
{मार्कण्डेयोपदेशेन व्रतं तेन कृतं यथा}
{अष्टादशेव कुष्ठानि गतानि तस्य सर्वशः}% ॥३१॥

\twolineshloka
{मुनेर्वाचा ततः सम्यग्व्रते चीर्णेऽभवत्सुखी}
{ईदृग्विधं नृपश्रेष्ठ कथितं योगिनीव्रतम्}% ॥३२॥

\twolineshloka
{अष्टाशीतिसहस्राणि द्विजान्भोजयते तु यः}
{तत्समं फलमाप्नोति योगिनीव्रतकृन्नरः}% ॥३३॥

\twolineshloka
{महापापप्रशमनं महापुण्यफलप्रदम्}
{पठनाच्छ्रवणान्मर्त्यः सर्वपापैः प्रमुच्यते}% ॥३४॥

॥इति श्रीपाद्मे महापुराणे पञ्चपञ्चाशत्साहस्र्यां संहितायामुत्तरखण्डे उमापतिनारदसंवादान्तर्गतकृष्णयुधिष्ठिरसंवादे आषाढ-कृष्ण-योगिनी-एकादशी-माहात्म्यं नाम चतुष्पञ्चाशत्तमोऽध्यायः॥५४॥


\hyperref[sec:ekadashi_mahatmyam_padma_puranam]{\closesub}
\clearpage

\sect{आषाढ-शुक्ल-शयनी-एकादशी-माहात्म्यम्}
\label{sec:padma-ashadha-shukla-shayani}


\uvacha{युधिष्ठिर उवाच}

\twolineshloka
{आषाढस्य सिते पक्षे का च एकादशी भवेत्}
{किं नाम को विधिस्तस्या एतद्विस्तरतो वद}% ॥१॥

\uvacha{श्रीकृष्ण उवाच}

\twolineshloka
{कथयामि महापुण्यां स्वर्गमोक्षप्रदायिनीम्}
{शयनीं नामनामेति सर्वपापहरां पराम्}% ॥२॥

\twolineshloka
{यस्याः श्रवणमात्रेण वाजपेयफलं लभेत्}
{सत्यं सत्यं मया प्रोक्तं नातः परतरं नृणाम्}% ॥३॥

\twolineshloka
{पापिनां पापनाशाय सृष्टा धात्रा महोत्तमा}
{अतः परा न राजेन्द्र वर्तते मोक्षदायिनी}% ॥४॥

\twolineshloka
{एतस्मात्कारणाद्राजन्श्रूयतां गतिरुत्तमा}
{भवेन्नराणां श्रोतॄणां कथायाः श्रवणादपि}% ॥५॥

\twolineshloka
{ते सदा वैष्णवा राजन्मम भक्तिपरायणाः}
{आषाढे वामनश्चैव पूज्यते परमेश्वरः}% ॥६॥

\twolineshloka
{वामनः पूजितो येन कमलैः कमलेक्षणः}
{आषाढस्य सिते पक्षे कामिकाया दिने तथा}% ॥७॥

\twolineshloka
{तेनार्चितं जगत्सर्वं त्रयो देवाः सनातनाः}
{कृता चैकादशी येन हरिवासरमुत्तमम्}% ॥८॥

\uvacha{युधिष्ठिर उवाच}

\twolineshloka
{संशयोऽस्ति महान्मेऽत्र श्रूयतां पुरुषोत्तम}
{कथं सुप्तोऽसि देवेश कथं च बलिमाश्रितः}% ॥९॥

\twolineshloka
{कथं च भूमौ संवेशः किं कुर्वन्ति जनाः परे}
{एतद्वद महाप्राज्ञ संशयोऽस्ति महान्मम}% ॥१०॥

\uvacha{श्रीकृष्ण उवाच}

\twolineshloka
{श्रूयतां राजशार्दूल कथां पापहरां पराम्}
{यस्याः श्रवणमात्रेण सर्वपापक्षयो भवेत्}% ॥११॥

\twolineshloka
{बलिनामा पूर्वमासीद्दैत्यस्त्रेतायुगे नृप}
{पूजयंश्चैव मां नित्यं मद्भक्तो मत्परायणः}% ॥१२॥

\twolineshloka
{यज्ञैस्तु विधिवद्दैत्यो यजते मां सनातनम्}
{भक्त्या च परया राजन्यज्ञकृद्व्रतकृत्तथा}% ॥१३॥

\twolineshloka
{परं विचार्य बहुधा मघोना चैव सूक्तिभिः}
{गुरुणा दैवतैः सार्धं बहुधा पूजितोऽप्यहम्}% ॥१४॥

\twolineshloka
{ततो वामनरूपेण अवतारे च पञ्चमे}
{अत्युग्ररूपेण तदा सर्वब्रह्माण्डरूपिणा}% ॥१५॥

\twolineshloka
{वाक्छलेन जिता दैत्याः सत्यमाश्रित्य संस्थितः}
{शुक्रस्तं वारयामास यन्नारायण इत्ययम्}% ॥१६॥

\twolineshloka
{याचिता वसुधा राजन्सार्धत्रयपदी मया}
{सङ्कल्पोदकमात्रे तु करे तेनैव चार्पिते}% ॥१७॥

\twolineshloka
{रूपमीदृग्विधं राजंस्तदा शृणु मया कृतम्}
{भूर्लोके चरणौ न्यस्य भुवर्लोके तु जानुनी}% ॥१८॥

\twolineshloka
{स्वर्लोके च कटिं न्यस्य महर्लोके तथोदरम्}
{जनलोके च हृदयं तपोलोके तु कण्ठकम्}% ॥१९॥

\twolineshloka
{सत्यलोके मुखं स्थाप्य मस्तकं च तदूर्ध्वकम्}
{चन्द्रसूर्यग्रहाश्चैव नक्षत्राणि तथैव च}% ॥२०॥

\twolineshloka
{देवाः सेन्द्राश्च नागाश्च यक्षगन्धर्वकिन्नराः}
{स्तुवन्तो वेदसम्भूतैः सूक्तैश्च विविधैस्तथा}% ॥२१॥

\twolineshloka
{करे गृहीत्वा च बलिं त्रिपदैः पूरिता मही}
{अर्धं च तस्य पृष्ठे च पदं न्यस्तं मया तदा}% ॥२२॥

\twolineshloka
{गतो रसातलं राजन्दानवो मम पूजकः}
{क्षिप्तोऽधो दानवश्चैव किमकुर्वं ततः परम्}% ॥२३॥

\twolineshloka
{विनयेनानतोसौ वै सुप्रसन्नो जनार्दनः}
{आषाढशुक्लपक्षे तु कामिका हरिवासरः}% ॥२४॥

\twolineshloka
{तस्यामेका च मूर्तिर्मे बलिमाश्रित्य तिष्ठति}
{द्वितीया शेषपृष्ठे वै क्षीरसागरमध्यतः}% ॥२५॥

\twolineshloka
{स्वपित्येव महाराज यावदागामि कार्तिकी}
{तावद्भवेत्सुधर्मात्मा सर्वधर्मोत्तमोत्तमः}% ॥२६॥

\twolineshloka
{व्रतं च कुरुते मर्त्यः स याति परमां गतिम्}
{एतस्मात्कारणाद्राजन्कर्तव्या च प्रयत्नतः}% ॥२७॥

\twolineshloka
{नातः परतरा काचित्पवित्रा पापनाशिनी}
{यस्यां स्वपिति देवेशः शङ्खचक्रगदाधरः}% ॥२८॥

\twolineshloka
{तस्यां च पूजयेद्देवं शङ्खचक्रगदाधरम्}
{रात्रौ जागरणं कृत्वा भक्त्या चैव विशेषतः}% ॥२९॥

\twolineshloka
{नास्याः पुण्यस्य सङ्ख्यानं कर्तुं शक्तश्चतुर्मुखः}
{एवं यः कुरुते राजन्नेकादश्या व्रतोत्तमम्}% ॥३०॥

\twolineshloka
{सर्वपापहरं चैव भुक्तिमुक्तिप्रदायकम्}
{स च लोके मम सदा श्वपचोऽपि प्रियङ्करः}% ॥३१॥

\twolineshloka
{दीपदानेन पालाशपत्रे भुक्त्या व्रतेन च}
{चातुर्मास्यं नयन्तीह ते नरा मम वल्लभाः}% ॥३२॥

\twolineshloka
{चातुर्मास्ये हरौ सुप्ते भूमिशायी भवेन्नरः}
{श्रावणे वर्जयेच्छाकं  दधि भाद्रपदे तथा}% ॥३३॥

\twolineshloka
{दुग्धमाश्वयुजि त्याज्यं कार्तिके द्विदलं त्यजेत्}
{अथवा ब्रह्मचर्यस्थः स याति परमां गतिम्}% ॥३४॥

\twolineshloka
{एकादश्या व्रतेनैव पुमान्पापैर्विमुच्यते}
{कर्तव्या सर्वदा राजन्विस्मर्तव्या न कर्हिचित्}% ॥३५॥

\twolineshloka
{शयनी बोधिनी मध्ये या कृष्णैकादशीभवेत्}
{सैवोपोष्या गृहस्थस्य नान्या कृष्णा कदाचन}% ॥३६॥

\twolineshloka
{शृणुयाच्चैव यो राजन्कथां पापहरां पराम्}
{अश्वमेधस्य यज्ञस्य फलं प्राप्नोति मानवः}% ॥३७॥

॥इति श्रीपाद्मे महापुराणे पञ्चपञ्चाशत्साहस्र्यां संहितायामुत्तरखण्डे उमापतिनारदसंवादान्तर्गतकृष्णयुधिष्ठिरसंवादे आषाढ-शुक्ल-शयनी-एकादशी-माहात्म्यं नाम पञ्चपञ्चाशत्तमोऽध्यायः॥५५॥


\hyperref[sec:ekadashi_mahatmyam_padma_puranam]{\closesub}
\clearpage

\sect{श्रावण-कृष्ण-कामिका-एकादशी-माहात्म्यम्}
\label{sec:padma-shravana-krishna-kamika}


\uvacha{युधिष्ठिर उवाच}

\twolineshloka
{श्रावणस्य सिते पक्षे किन्नामैकादशीभवेत्}
{तन्नः कथय गोविन्द वासुदेव नमोऽस्तु ते}% ॥१॥

\uvacha{श्रीकृष्ण उवाच}

\twolineshloka
{शृणु राजन्प्रवक्ष्यामि आख्यानं पापनाशनम्}
{यत्प्रोक्तं ब्रह्मणा पूर्वं पृच्छते नारदाय वै}% ॥२॥

\uvacha{नारद उवाच}

\twolineshloka
{भगवन्श्रोतुमिच्छामि त्वत्तोऽहं कमलासन}
{श्रावणस्यासिते पक्षे किं नामैकादशी भवेत्}% ॥३॥

\twolineshloka
{को देवः को विधिस्तस्याः किं पुण्यं कथय प्रभो}
{इति तस्य वचः श्रुत्वा ब्रह्मा वचनमब्रवीत्}% ॥४॥

\uvacha{ब्रह्मोवाच}

\twolineshloka
{शृणु नारद ते वच्मि लोकानां हितकाम्यया}
{श्रावणैकादशी कृष्णा कामिका नाम नामतः}% ॥५॥

\twolineshloka
{अस्याः श्रवणमात्रेण वाजपेयफलं लभेत्}
{अस्यां यजति देवेशं शङ्खचक्रगदाधरम्}% ॥६॥

\twolineshloka
{श्रीधराख्यं हरिं विष्णुं माधवं मधुसूदनम्}
{पूजयेद्ध्यायते यो वै तस्य पुण्यम्फलं शृणु}% ॥७॥

\twolineshloka
{न गङ्गायां न काश्यां च नैमिषे न च पुष्करे}
{तत्फलं समवाप्नोति यत्फलं कृष्णपूजनात्}% ॥८॥

\twolineshloka
{गोदावर्यां गुरौ सिंहे व्यतीपाते च दण्डके}
{यत्फलं समवाप्नोति तत्फलं कृष्णपूजनात्}% ॥९॥

\twolineshloka
{ससागरवनोपेतां यो ददाति वसुन्धराम्}
{कामिकाव्रतकारी च ह्युभौ समफलौ स्मृतौ}% ॥१०॥

\twolineshloka
{प्रसूयमानां यो धेनुं दद्यात्सोपस्करां नरः}
{तत्फलं समवाप्नोति कामिकाव्रतकारकः}% ॥११॥

\twolineshloka
{श्रावणे श्रीधरं देवं पूजयेद्यो नरोत्तमः}
{तेनैव पूजिता देवा गन्धर्वोरगपन्नगाः}% ॥१२॥

\twolineshloka
{तस्मात्सर्वप्रयत्नेन कामिकादिवसे हरिः}
{पूजनीयो यथाशक्ति मानुषैः पापभीरुभिः}% ॥१३॥

\twolineshloka
{ये संसारार्णवे मग्नाः पापपङ्कसमाकुले}
{तेषामुद्धरणार्थाय कामिकाव्रतमुत्तमम्}% ॥१४॥

\twolineshloka
{नातः परतरा काचित्पवित्रा पापहारिणी}
{एवं नारद जानीहि स्वयमाह परो हरिः}% ॥१५॥

\twolineshloka
{अध्यात्मविद्या निरतैर्यत्फलं प्राप्यते नरैः}
{ततो बहुतरं विद्धि कामिकाव्रतसेविनाम्}% ॥१६॥

\twolineshloka
{रात्रौ जागरणं कृत्वा कामिकाव्रतकृन्नरः}
{न पश्यति यमं रौद्रं नैव गच्छति दुर्गतिम्}% ॥१७॥

\twolineshloka
{न पश्यति कुयोनिं च कामिकाव्रतसेवनात्}
{कामिकाया व्रते चीर्णे कैवल्यं योगिनो गताः}% ॥१८॥

\twolineshloka
{तस्मात्सर्वप्रयत्नेन कर्तव्या नियतात्मभिः}
{तुलसीप्रभवैः पत्रैः यो नरः पूजयेद्धरिम्}% ॥१९॥

\twolineshloka
{न लिप्यते स पापेन पद्मपत्रमिवाम्भसा}
{सुवर्णभारमेकं तु रजतं च चतुर्गुणम्}% ॥२०॥

\twolineshloka
{तत्फलं समवाप्नोति तुलसीदलपूजनात्}
{रत्न मौक्तिक वैडूर्य प्रवालादिभिरर्चितः}% ॥२१॥

\twolineshloka
{न तुष्यति तथा विष्णुस्तुलसीदलतो यथा}
{तुलसीमञ्जरीभिश्च पूजितो येन केशवः}% ॥२२॥

\fourlineindentedshloka
{या दृष्टा निखिलाघसङ्घशमनी स्पृष्टा वपुः पावनी}
{रोगाणामभिवन्दितानि रसिनी सिक्तान्तकत्रासिनी}
{प्रत्यासत्तिविधायिनी भगवतः कृष्णस्य संरोपिता}
{न्यस्ता तच्चरणे विमुक्तिफलदा तस्यै तुलस्यै नमः}% ॥२३॥

\twolineshloka
{दीपं ददाति यो मर्त्यो दिवारात्रं हरेर्दिने}
{तस्य पुण्यस्य सङ्ख्यातुं चित्रगुप्तो न वेत्त्यलम्}% ॥२४॥

\twolineshloka
{कृष्णाग्रे दीपको यस्य ज्वलत्येकादशीदिने}
{पितरस्तस्य तृप्यन्ति अमृतेन दिवि स्थिताः}% ॥२५॥

\twolineshloka
{घृतेन दीपं प्रज्वाल्य तिलतैलेन वा पुनः}
{प्रयाति सूर्यलोकं च दीपकोटिशतार्चितः}% ॥२६॥

\twolineshloka
{अयं तवाग्रे कथितः कामिकामहिमा मया}
{अतो नरैः प्रकर्तव्या सर्वपातकहारिणी}% ॥२७॥

\twolineshloka
{ब्रह्महत्यापहरणी भ्रूणहत्याविनाशिनी}
{वैष्णवस्थानदात्री च महापुण्यफलप्रदा}% ॥२८॥

\twolineshloka
{श्रुत्वा माहात्म्यमेतस्या नरः श्रद्धासमन्वितः}
{विष्णुलोकमवाप्नोति सर्वपापैः प्रमुच्यते}% ॥२९॥

॥इति श्रीपाद्मे महापुराणे पञ्चपञ्चाशत्साहस्र्यां संहितायामुत्तरखण्डे उमापतिनारदसंवादान्तर्गतकृष्णयुधिष्ठिरसंवादे श्रावण-कृष्ण-कामिका-एकादशी-माहात्म्यं नाम षट्पञ्चाशत्तमोऽध्यायः॥५६॥


\hyperref[sec:ekadashi_mahatmyam_padma_puranam]{\closesub}
\clearpage

\sect{श्रावण-शुक्ल-पुत्रदा-एकादशी-माहात्म्यम्}
\label{sec:padma-shravana-shukla-putrada}

युधिष्ठिर उवाच

\twolineshloka
{श्रावणस्य सिते पक्षे किं नामैकादशीभवेत्}
{कथयस्व प्रसादेन ममाग्रे मधुसूदन}% ॥१॥

\uvacha{श्रीकृष्ण उवाच}

\twolineshloka
{शृणुष्वावहितो राजन्कथां पापहरां पराम्}
{यस्याः श्रवणमात्रेण वाजपेयफलं भवेत्}% ॥२॥

\twolineshloka
{द्वापरस्य युगस्यादौ पुरा माहिष्मती पुरे}
{राजा महीजिदाख्यातो राज्यं पालयति स्वकम्}% ॥३॥

\twolineshloka
{पुत्रहीनस्य तस्यैव न तद्राज्यं सुखप्रदम्}
{अपुत्रस्य सुखं नास्ति इहलोके परत्र च}% ॥४॥

\twolineshloka
{चिन्तयास्य सुतस्यैवं कालो बहुतरो गतः}
{न प्राप्तश्च सुतो राज्ञा सर्वसौख्यप्रदो नृणाम्}% ॥५॥

\twolineshloka
{दृष्ट्वात्मानं प्रवयसं राजा चिन्तापरोऽभवत्}
{तदागतः प्रजामध्ये इदं वचनमब्रवीत्}% ॥६॥

\twolineshloka
{इहजन्मनि भो लोका न मया पातकं कृतम्}
{अन्यायोपार्जितं वित्तं क्षिप्तं कोशे मया न हि}% ॥७॥

\threelineshloka
{ब्रह्मस्वं देवद्रविणं न गृहीतं मया क्वचित्}
{न्यासापहारो न कृतः परस्य बहुपापदः}
{पुत्रवत्पालितो लोको धर्मेण विजिता मही}% ॥८॥

\twolineshloka
{दुष्टेषु पातितो दण्डो बन्धुपुत्रोपमेष्वपि}
{शिष्टास्तु पूजिता नित्यं न द्वेष्याश्च मया जनाः}% ॥९॥

\twolineshloka
{इत्येवं ब्रुवतो मार्गं धर्मयुक्तं द्विजोत्तमाः}
{कस्मान्मम गृहे पुत्रो न जातस्तद्विमृश्यताम्}% ॥१०॥

\twolineshloka
{इति वाक्यं द्विजाः श्रुत्वा सप्रजाः सपुरोहिताः}
{मन्त्रयित्वा नृपहितं जग्मुस्ते गहनं वनम्}% ॥११॥

\twolineshloka
{इतस्ततश्च पश्यन्त आश्रमानृषिसेवितान्}
{नृपतेर्हितमिच्छन्तो ददृशुर्मुनिसत्तमम्}% ॥१२॥

\twolineshloka
{तप्यमानं तपो घोरं निरालम्बं निरामयम्}
{निराहारं जितात्मानं जितक्रोधं सनातनम्}% ॥१३॥

\twolineshloka
{लोमशं धर्मतत्वज्ञं सर्वशास्त्रविशारदम्}
{दीर्घायुषं महात्मानं सकेशं ब्रह्मसम्मितम्}% ॥१४॥

\twolineshloka
{कल्पेकल्पे गते तस्य एकं लोम विशीर्यते}
{अतो लोमशनामायं त्रिकालज्ञो महामुनिः}% ॥१५॥

\twolineshloka
{तं दृष्ट्वा हर्षिताः सर्वे आजग्मुस्तस्य सन्निधिम्}
{यथान्यायं यथार्हं ते नमश्चक्रुर्यथोदितम्}% ॥१६॥

\threelineshloka
{विनयावनताः सर्वे ऊचुस्ते च परस्परम्}
{अस्मद्भाग्यवशादेव प्राप्तोऽयं मुनिसत्तमः}
{तास्तथा स प्रजा वीक्ष्य उवाच ऋषिसत्तमः}% ॥१७॥

\uvacha{लोमश उवाच}

\twolineshloka
{किमर्थमिह सम्प्राप्तः कथयध्वं सकारणम्}
{दर्शनाद्धृष्टमनसः स्तुवन्तश्चैव मां किमु}% ॥१८॥

\twolineshloka
{असंशयं करिष्यामि भवतां यद्धितं भवेत्}
{परोपकृतये जन्म मादृशानां न संशयः}% ॥१९॥

\uvacha{जना ऊचुः}

\twolineshloka
{श्रूयतामभिधास्यामो वयं स्वागमकारणम्}
{संशयच्छेदनार्थाय तव सान्निध्यमागताः}% ॥२०॥

\twolineshloka
{पद्मयोनेः परतरस्त्वत्तः श्रेष्ठो न विद्यते}
{अतः कार्यवशात्प्राप्ताः समीपं भवतो वयम्}% ॥२१॥

\twolineshloka
{महीजिन्नाम राजासौ पुत्रहीनोऽस्ति साम्प्रतम्}
{वयं तस्य प्रजा ब्रह्मन्पुत्रवत्तेन पालिताः}% ॥२२॥

\twolineshloka
{तं पुत्ररहितं दृष्ट्वा तस्य दुःखेन दुःखिताः}
{तपः कर्तुमिहायाता मतिं कृत्वा तु नैष्ठिकीम्}% ॥२३॥

\twolineshloka
{तस्य भाग्येन दृष्टोऽसि ह्यस्माभिस्त्वं द्विजोत्तम}
{महतां दर्शनेनैव कार्यसिद्धिर्भवेन्नृणाम्}% ॥२४॥

\threelineshloka
{उपदेशं वद मुने राज्ञः पुत्रो यथा भवेत्}
{इति तेषां वचः श्रुत्वा मुहूर्तं ध्यानमास्थितः}
{प्रत्युवाच मुनिर्ज्ञात्वा तस्य जन्म पुरातनम्}% ॥२५॥

\uvacha{लोमश उवाच}

\twolineshloka
{पुरा जन्मनि वैश्योऽयं धनहीनो नृशंसकृत्}
{वाणिज्यकर्मनिरतो ग्रामाद्ग्रामान्तरं भ्रमन्}% ॥२६॥

\twolineshloka
{ज्येष्ठे मासि सिते पक्षे दशमी दिवसे तथा}
{मध्यगे द्युमणौ प्राप्ते ग्रामसीम्नि जलाशयम्}% ॥२७॥

\twolineshloka
{कूपिकां सजलां दृष्ट्वा जलपाने मनोदधे}
{सद्यस्ततः सवत्सा च धेनुस्तत्र समागता}% ॥२८॥

\twolineshloka
{तृष्णातुरा निदाघार्ता तस्यामम्बुपपौ तु सा}
{पिबन्तीं वारयित्वा तामसौ तोयं पपौ स्वयम्}% ॥२९॥

\twolineshloka
{कर्मणा तेन पापेन पुत्रहीनो नृपो भवेत्}
{कस्यापि जन्मनः पुण्यात्प्राप्तं राज्यमकण्टकम्}% ॥३०॥

\uvacha{लोका ऊचुः}

\threelineshloka
{पुण्यात्पापं क्षयं याति पुराणे श्रूयते मुने}
{पुण्योपदेशं कथय येन पापक्षयो भवेत्}
{यथा भवत्प्रसादेन पुत्रो भवति भूपतेः}% ॥३१॥

\uvacha{लोमश उवाच}

\twolineshloka
{श्रावणे शुक्लपक्षे तु पुत्रदा नाम विश्रुता}
{एकादशी वाञ्च्छितदा कुरुध्वं तद्व्रतं जनाः}% ॥३२॥

\twolineshloka
{इति श्रुत्वा नमस्कृत्य मुनिमेत्य पुरं व्रतम्}
{यथाविधियथान्यायं कृतं तैर्जागरान्वितम्}% ॥३३॥

\twolineshloka
{तस्य पुण्यं सुविमलं दत्तं नृपतये जनैः}
{दत्ते पुण्येऽथ सा राज्ञी गर्भमाधत्त शोभनम्}% ॥३४॥

\twolineshloka
{प्राप्तो प्रसवकाले सा सुषुवे पुत्रमूर्जितम्}
{श्रावणस्य सिते पक्षे कर्कटस्थे दिवाकरे}% ॥३५॥

\twolineshloka
{द्वादश्यां वासुदेवाय पवित्रारोपणं स्मृतम्}
{हेम रौप्य ताम्र क्षौमैः सूत्रैः कौशेयपद्मजैः}% ॥३६॥

\twolineshloka
{कुशैः काशैश्च कार्पासैर्ब्राह्मण्या कर्तितैः शुभैः}
{स्नात्वा त्रिगुणितं सूत्रं त्रिगुणीकृत्य शोधयेत्}% ॥३७॥

\twolineshloka
{गोदोहान्तरिते काले पूर्वेद्युरधिवासनम्}
{ब्राह्मणेभ्यो नमस्कृत्य गुरुपादौ प्रणम्य च}% ॥३८॥

\twolineshloka
{गीतमङ्गलनिर्घोषः कुर्याज्जागरणं ततः}
{ब्राह्मणाः क्षत्रिया वैश्या भिल्लाः शूद्रास्तथैव च}% ॥३९॥

\twolineshloka
{स्वधर्मावस्थिताः सर्वे भक्त्या कुर्युः पवित्रकम्}
{ततः पवित्रं गुरवे दद्याद्वै विधिपूर्वकम्}% ॥४०॥

\twolineshloka
{ब्राह्मणान्वैष्णवांश्चैव गन्धपुष्पादिनार्चयेत्}
{अतो देवेति मन्त्रेण द्विजो विष्णौ निवेदयेत्}% ॥४१॥

\twolineshloka
{शूद्रस्तु मूलमन्त्रेण यथा विष्णौ तथा शिवे}
{वर्षेवर्षे प्रकर्तव्यं पवित्रारोपणं नरैः}% ॥४२॥

\twolineshloka
{भुक्तिं मुक्तिं च इच्छद्भिः संसारे शोकसागरे}
{न करोति विधानेन पवित्रारोपणं तु यः}% ॥४३॥

\threelineshloka
{तस्य सांवत्सरी पूजा निष्फला वैष्णवस्य तु}
{श्रुत्वा माहात्म्यमेतस्या नरः पापात्प्रमुच्यते}
{इह पुत्रसुखं प्राप्य परत्र स्वर्गतिं भवेत्}% ॥४४॥

॥इति श्रीपाद्मे महापुराणे पञ्चपञ्चाशत्साहस्र्यां संहितायामुत्तरखण्डे उमापतिनारदसंवादान्तर्गतकृष्णयुधिष्ठिरसंवादे श्रावण-शुक्ल-पुत्रदा-एकादशी-माहात्म्यं नाम सप्तपञ्चाशत्तमोऽध्यायः॥५७॥


\hyperref[sec:ekadashi_mahatmyam_padma_puranam]{\closesub}
\clearpage

\sect{भाद्रपद-कृष्णाजा-एकादशी-माहात्म्यम्}
\label{sec:padma-bhadrapada-krishnaja}


\uvacha{युधिष्ठिर उवाच}

\twolineshloka
{भाद्रस्य कृष्णपक्षे तु किन्नामैकादशीभवेत्}
{एतदिच्छाम्यहं श्रोतुं कथयस्व जनार्दन}% ॥१॥

\uvacha{श्रीकृष्ण उवाच}

\twolineshloka
{शृणुष्वैकमना राजन्कथयिष्यामि विस्तरात्}
{अजेति नामतः प्रोक्ता सर्वपापप्रणाशिनी}% ॥२॥

\twolineshloka
{पूजयित्वा हृषीकेशं व्रतमस्यां करोति यः}
{पापानि तस्य नश्यन्ति व्रतस्य श्रवणादपि}% ॥३॥

\twolineshloka
{नातः परतरा राजन्लोकद्वयहिताय वै}
{सत्यमुक्तं मया ह्येतन्नासत्यं मम भाषितम्}% ॥४॥

\twolineshloka
{हरिश्चन्द्र इति ख्यातो बभूव नृपतिः पुरा}
{चक्रवर्ती सत्यसन्धः समस्ताया भुवः पतिः}% ॥५॥

\twolineshloka
{कस्यापि कर्मणः प्राप्तौ राज्यभ्रष्टो बभूव सः}
{विक्रीतौ वनितापुत्रौ स चकारात्मविक्रयम्}% ॥६॥

\twolineshloka
{पुल्कसस्य च दासत्वं गतो राजा स पुण्यकृत्}
{सत्यमालम्ब्य राजेन्द्र मृतचैलापहारकः}% ॥७॥

\twolineshloka
{सोऽभवन्नृपतिश्रेष्ठो न सत्याच्चलितस्तथा}
{एवं च तस्य नृपतेर्बहवो वत्सरा गताः}% ॥८॥

\twolineshloka
{ततश्चिन्तापरो राजा स बभूवातिदुःखितः}
{किं करोमि क्व गच्छामि निष्कृतिर्मे कथं भवेत्}% ॥९॥

\twolineshloka
{इति चिन्तयतस्तस्य मग्नस्य वृजिनार्णवे}
{आजगाम मुनिः कश्चिज्ज्ञात्वा राजानमातुरम्}% ॥१०॥

\twolineshloka
{परोपकारणार्थाय निर्मिता ब्रह्मणा द्विजाः}
{स तं दृष्ट्वा द्विजवरं ननाम नृपसत्तमः}% ॥११॥

\twolineshloka
{कृताञ्जलिपुटो भूत्वा गौतमस्याग्रतः स्थितः}
{कथयामास वृत्तान्तमात्मनो दुःखसंयुतम्}% ॥१२॥

\twolineshloka
{श्रुत्वा नृपतिवाक्यानि गौतमो विस्मयान्वितः}
{उपदेशं नृपतये व्रतस्यास्य ददौ मुनिः}% ॥१३॥

\twolineshloka
{मासि भाद्रपदे राजन्कृष्णपक्षेति शोभना}
{एकादशी समायाता  अजा नामेति पुण्यदा}% ॥१४॥

\twolineshloka
{अस्याः कुरु व्रतं राजन्पापस्यान्तो भविष्यति}
{तव भाग्यवशादेषा सप्तमेऽह्नि समागता}% ॥१५॥

\twolineshloka
{उपवासपरो भूत्वा रात्रौ जागरणं कुरु}
{एवमस्या व्रते चीर्णे तव पापक्षयो ध्रुवम्}% ॥१६॥

\twolineshloka
{तव पुण्यप्रभावेण चागतोऽहं नृपोत्तम}
{इत्येवं कथयित्वा च मुनिरन्तरधीयत}% ॥१७॥

\twolineshloka
{मुनिवाक्यं नृपः श्रुत्वा चकार व्रतमुत्तमम्}
{कृते तस्मिन्व्रते राज्ञः पापस्यान्तोऽभवत्क्षणात्}% ॥१८॥

\twolineshloka
{श्रूयतां राजशार्दूल प्रभावोऽस्य व्रतस्य च}
{यद्दुःखम्बहुभिर्वर्षैर्भोक्तव्यन्तत्क्षयोभवेत्}% ॥१९॥

\twolineshloka
{निस्तीर्णदुःखो राजासीद्व्रतस्यास्य प्रभावतः}
{पत्न्या सह समायोगं पुत्रजीवनमाप सः}% ॥२०॥

\twolineshloka
{दिवि दुन्दुभयो नेदुः पुष्पवर्षमभूद्दिवः}
{एकादश्याः प्रभावेन प्राप्यराज्यमकण्टकम्}% ॥२१॥

\twolineshloka
{स्वर्गं लेभे हरिश्चन्द्रः सपुरः सपरिच्छदः}
{ईदृग्विधं व्रतंराजन्ये कुर्वन्ति च मानवाः}% ॥२२॥

\twolineshloka
{सर्वपापविनिर्मुक्तास्त्रिदिवं यान्ति ते नृप}
{पठनाच्छ्रवणाद्वाऽपि अश्वमेधफलं लभेत्}% ॥२३॥

॥इति श्रीपाद्मे महापुराणे पञ्चपञ्चाशत्साहस्र्यां संहितायामुत्तरखण्डे उमापतिनारदसंवादान्तर्गतकृष्णयुधिष्ठिरसंवादे भाद्रपद-कृष्णाजा-एकादशी-माहात्म्यं नाम अष्टपञ्चाशत्तमोऽध्यायः॥५८॥


\hyperref[sec:ekadashi_mahatmyam_padma_puranam]{\closesub}
\clearpage

\sect{भाद्रपद-शुक्ल-पद्मा-एकादशी-माहात्म्यम्}
\label{sec:padma-bhadrapada-shukla-padma}


\uvacha{युधिष्ठिर उवाच}

\twolineshloka
{नभस्यस्य सिते पक्षे किन्नामैकादशी भवेत्}
{को देवः को विधिस्तस्य एतदाख्याहि केशव}% ॥१॥

\uvacha{श्रीकृष्ण उवाच}

\twolineshloka
{कथयामि महीपाल कथामाश्चर्यकारिणीम्}
{कथयामास यां ब्रह्मा नारदाय महात्मने}% ॥२॥

\uvacha{नारद उवाच}

\threelineshloka
{कथयस्व प्रसादेन चतुर्मुख नमोऽस्तु ते}
{नभस्य शुक्लपक्षे तु किं नामैकादशी भवेत्}
{एतदिच्छाम्यहं श्रोतुं विष्णोराराधनाय वै}% ॥३॥

\uvacha{ब्रह्मोवाच}

\twolineshloka
{वैष्णवोऽसि मुनिश्रेष्ठ साधुपृष्टं किल त्वया}
{नातः परतरा लोके पवित्रा हरिवासरात्}% ॥४॥

\twolineshloka
{पद्मा नामेति विख्याता नभस्यैकादशी सिता}
{हृषीकेशः पूज्यतेऽस्यां कर्तव्यं व्रतमुत्तमम्}% ॥५॥

\twolineshloka
{कथयामि तवाग्रेऽहं कथां पौराणिकीं शुभाम्}
{यस्याः श्रवणमात्रेण महापापं प्रणश्यति}% ॥६॥

\twolineshloka
{मान्धाता नाम राजर्षिर्विवस्वद्वंशसम्भवः}
{बभूव चक्रवर्ती स सत्यसन्धः प्रतापवान्}% ॥७॥

\twolineshloka
{धर्मतः पालयामास प्रजाः पुत्रानिवौरसान्}
{न तस्य राज्ये दुर्भिक्षं नाधयो व्याधयस्तथा}% ॥८॥

\twolineshloka
{निरातङ्काः प्रजास्तस्य धनधान्यसमेधिताः}
{न्यायेनोपार्जितं वित्तं तस्य कोशे महीपते}% ॥९॥

\twolineshloka
{स्वस्वधर्मे प्रवर्तन्ते सर्वे वर्णाश्रमास्तथा}
{कामधेनुसमाभूमिस्तस्य राज्ये महीपतेः}% ॥१०॥

\twolineshloka
{तस्यैवं कुर्वतो राज्यं बहुवर्षगणा गताः}
{अथैकस्मिंश्च सम्प्राप्ते विपाकः कर्मणः खलु}% ॥११॥

\twolineshloka
{वर्षत्रयं तद्विषये न ववर्ष बलाहकः}
{तेन भग्नाः प्रजास्तस्य बभूवुः क्षुधयार्दिताः}% ॥१२॥

\threelineshloka
{स्वाहा स्वधा वषट्कार वेदाध्ययनवर्जिताः}
{बभूव विषयस्तस्या भाग्येन दैवपीडितः}
{अथ प्रजाः समागम्य राजानमिदमब्रुवन्}% ॥१३॥

\uvacha{प्रजा ऊचुः}

\twolineshloka
{श्रोतव्यं नृपशार्दूल प्रजानां वचनं त्वया}
{आपो नारा इति प्रोक्ताः पुराणेषु मनीषिभिः}% ॥१४॥

\twolineshloka
{अयनं भगवतस्तस्मान्नारायण इति स्मृतः}
{पर्जन्यरूपो भगवान्विष्णुः सर्वगतः स्थितः}% ॥१५॥

\threelineshloka
{स एवं कुरुते वृष्टिं वृष्टेरन्नं ततः प्रजाः}
{तदभावे नृपश्रेष्ठ क्षयं गच्छन्ति वै प्रजाः}
{तथा कुरु नृपश्रेष्ठ योगः क्षेमो यथा भवेत्}% ॥१६॥

\uvacha{राजोवाच}

\twolineshloka
{सत्यमुक्तं भवद्भिश्च न मिथ्याभिहितं क्वचित्}
{अन्नं ब्रह्म यतः प्रोक्तमन्ने सर्वं प्रतिष्ठितम्}% ॥१७॥

\twolineshloka
{अन्नाद्भवन्ति भूतानि जगदन्नेन वर्तते}
{इत्येवं श्रूयते लोके पुराणे बहुविस्तरे}% ॥१८॥

\twolineshloka
{नृपाणामपचारेण प्रजानां पीडनं भवेत्}
{नाहं पश्याम्यात्मकृतमेवं बुद्ध्या विचारयन्}% ॥१९॥

\twolineshloka
{तथापि प्रयतिष्यामि प्रजानां हितकाम्यया}
{इति कृत्वा मतिं राजा परिमेयपरिच्छदः}% ॥२०॥

\twolineshloka
{नमस्कृत्य विधातारं जगाम गहनं वनम्}
{चचार मुनिमुख्यांश्च आश्रमान्तापसैः श्रितान्}% ॥२१॥

\twolineshloka
{ददर्शाथ ब्रह्मसुतमृषिमाङ्गिरसं नृपः}
{तेजसा द्योतितदिशं द्वितीयमिव पद्मजम्}% ॥२२॥

\twolineshloka
{तं दृष्ट्वा हर्षितो राजा अवतीर्य स्ववाहनात्}
{नमश्चक्रेऽस्य चरणौ कृताञ्जलिपुटो वशी}% ॥२३॥

\twolineshloka
{मुनिस्तमभिनन्द्याथ स्वस्तिवाचनपूर्वकम्}
{पप्रच्छ कुशलं राज्ये सप्तस्वङ्गेषु भूपतेः}% ॥२४॥

\twolineshloka
{निवेदयित्वा कुशलं पप्रच्छानामयं नृपः}
{दत्तासनो गृहीतार्घ्य उपविष्टोऽस्य सन्निधौ}% ॥२५॥

\onelineshloka
{प्रत्युवाच मुनिं राजा पृष्टो ह्यागमकारणम्}% ॥२६॥

\uvacha{राजोवाच}

\twolineshloka
{भगवन्धर्मविधिना मम पालयतो महीम्}
{अनावृष्टिश्च संवृत्ता नाहं वेद्म्यत्र कारणम्}% ॥२७॥

\twolineshloka
{संशयच्छेदनायात्र आगतोऽहं तवान्तिके}
{योगक्षेमविधानेन प्रजानां कुरु निर्वृतिम्}% ॥२८॥

\uvacha{ऋषिरुवाच}

\twolineshloka
{एतत्कृतयुगं राजन्युगानामुत्तमं मतम्}
{अत्र ब्रह्मपरा लोका धर्मश्चात्र चतुष्पदः}% ॥२९॥

\twolineshloka
{अस्मिन्युगे तपोयुक्ता ब्राह्मणा नेतरा जनाः}
{विषये तव राजेन्द्र वृषलोऽयं तपस्यति}% ॥३०॥

\twolineshloka
{एतस्मात्कारणाच्चैव न वर्षति बलाहकः}
{कुरु तस्य वधे यत्नं येन दोषः प्रशाम्यति}% ॥३१॥

\uvacha{राजोवाच}

\twolineshloka
{नाहमेनं वधिष्यामि तपस्यन्तमनागसम्}
{धर्मोपदेशं कथय उपसर्गविनाशनम्}% ॥३२॥

\uvacha{ऋषिरुवाच}

\twolineshloka
{यद्येवं तर्हि नृपते कुरुष्वैकादशीव्रतम्}
{नभस्यस्य सिते पक्षे पद्मा नामेति विश्रुता}% ॥३३॥

\twolineshloka
{तस्या व्रतप्रभावेन सुवृष्टिर्भविता ध्रुवम्}
{सर्वसिद्धिप्रदा ह्येषा सर्वोपद्रवनाशिनी}% ॥३४॥

\twolineshloka
{अस्या व्रतं कुरु नृप सप्रजः सपरिच्छदः}
{इति वाक्यमृषेः श्रुत्वा राजा स्वगृहमागतः}% ॥३५॥

\twolineshloka
{भाद्रमासे सिते पक्षे पद्माव्रतमथाकरोत्}
{प्रजाभिः सह सर्वाभिश्चातुर्वर्ण्यसमन्वितः}% ॥३६॥

\twolineshloka
{एवं व्रते कृते राजन्प्रववर्ष बलाहकः}
{जलेन प्लाविता भूमिरभवत्सस्यशालिनी}% ॥३७॥

\twolineshloka
{ऋषीश्वरप्रभावेन लोकाः सौख्यं प्रपेदिरे}
{एतस्मात्कारणादेवं कर्तव्यं व्रतमुत्तमम्}% ॥३८॥

\twolineshloka
{दध्योदनयुतं तस्यां जलपूर्णं घटं द्विजे}
{वस्त्रसंवेष्टितं दत्त्वा छत्रोपानहमेव च}% ॥३९॥

\twolineshloka
{नमो नमस्ते गोविन्द बुधश्रवणसंज्ञक}
{अघौघसङ्क्षयं कृत्वा सर्वसौख्यप्रदो भव}% ॥४०॥

\twolineshloka
{भुक्तिमुक्तिप्रदश्चैव लोकानां सुखदायकः}
{पठनाच्छ्रवणाद्राजन्सर्वपापैः प्रमुच्यते}% ॥४१॥

॥इति श्रीपाद्मे महापुराणे पञ्चपञ्चाशत्साहस्र्यां संहितायामुत्तरखण्डे उमापतिनारदसंवादान्तर्गतकृष्णयुधिष्ठिरसंवादे भाद्रपद-शुक्ल-पद्मा-एकादशी-माहात्म्यं नाम एकोनषष्टितमोऽध्यायः॥५९॥


\hyperref[sec:ekadashi_mahatmyam_padma_puranam]{\closesub}
\clearpage

\sect{आश्विन-कृष्णेन्दिरा-एकादशी-माहात्म्यम्}
\label{sec:padma-ashvina-krishnendira}


\uvacha{युधिष्ठिर उवाच}

\twolineshloka
{कथयस्व प्रसादेन ममाग्रे मधुसूदन}
{इषस्य कृष्णपक्षे तु किन्नामैकादशीभवेत्}% ॥१॥

\uvacha{श्रीकृष्ण उवाच}

\twolineshloka
{आश्विने कृष्णपक्षे तु इन्दिरा नाम नामतः}
{तस्या व्रतप्रभावेन महापापं प्रणश्यति}% ॥२॥

\twolineshloka
{अधोयोनि गतानां च पितॄणां गतिदायिनी}
{शृणुष्वावहितो राजन्कथां पापहरां पराम्}% ॥३॥

\twolineshloka
{यस्याः श्रवणमात्रेण वाजपेयफलं लभेत्}
{पुरा कृतयुगे राजन्बभूव नृपनन्दनः}% ॥४॥

\twolineshloka
{इन्द्रसेन इति ख्यातः पुरा माहिष्मतीपतिः}
{स राजा पालयामास धर्मेण यशसान्वितः}% ॥५॥

\twolineshloka
{पुत्रपौत्रसमायुक्तो धनधान्यसमन्वितः}
{माहिष्मत्यधिपो राजा विष्णुभक्तिपरायणः}% ॥६॥

\twolineshloka
{जपन्गोविन्दनामानि मुक्तिदानि नराधिपः}
{कालं नयति विधिवदध्यात्मध्यानचिन्तकः}% ॥७॥

\twolineshloka
{एकस्मिन्दिवसे राज्ञि सुखासीने सदो गते}
{अवतीर्यागमत्तत्र ह्यम्बरान्नारदो मुनिः}% ॥८॥

\twolineshloka
{तमागतमभिप्रेक्ष्य प्रत्युत्थाय कृताञ्जलि}
{पूजयित्वाऽथ विधिना चासने सन्न्यवेशयत्}% ॥९॥
\onelineshloka*
{सुखोपविष्टं स मुनिं प्रत्युवाच नृपोत्तमः}

\uvacha{राजोवाच}
\onelineshloka
{त्वत्प्रसादान्मुनिश्रेष्ठ सर्वं च कुशलं मम}% ॥१०॥

\twolineshloka
{अद्य क्रतुक्रियाः सर्वाः सफलास्तव दर्शनात्}
{प्रसादं कुरु देवर्षे ब्रूह्यागमनकारणम्}% ॥११॥

\uvacha{नारद उवाच}

\twolineshloka
{श्रूयतां नृपशार्दूल मद्वचो विस्मयप्रदम्}
{ब्रह्मलोकादहं प्राप्तो यमलोकं नृपोत्तम}% ॥१२॥

\twolineshloka
{शमनेनार्चितो भक्त्या उपविष्टो वरासने}
{धर्मशीलः सत्यवांस्तु भास्करिं समुपासते}% ॥१३॥

\twolineshloka
{बहुपुण्यप्रकर्ता च व्रतवैकल्यदोषतः}
{सभायां श्राद्धदेवस्य मया दृष्टः पिता तव}% ॥१४॥

\twolineshloka
{कथितस्तेन सन्देशस्तं निबोध जनेश्वर}
{इन्द्रसेन इति ख्यातो राजा माहिष्मतीपतिः}% ॥१५॥

\twolineshloka
{तस्याग्रे कथयब्रह्मन्स्थितं मां यमसन्निधौ}
{केनापि चान्तरायेण पूर्वजन्मोद्भवेन च}% ॥१६॥

\twolineshloka
{स्वर्गं प्रेषय मां पुत्र इन्दिरा पुण्य दानतः}
{इत्युक्तोऽहं समायातः समीपं तव पार्थिव}% ॥१७॥

\twolineshloka
{पितुः स्वर्गकृते राजन्निन्दिरा व्रतमाचर}
{तेन व्रतप्रभावेन स्वर्गं यास्यति ते पिता}% ॥१८॥

\uvacha{राजोवाच}

\twolineshloka
{कथयस्व प्रसादेन भगवन्निन्दिरा व्रतम्}
{विधिना केन कर्तव्यं कस्मिन्पक्षे तिथौ तथा}% ॥१९॥

\uvacha{नारदउवाच}

\twolineshloka
{शृणु राजेन्द्र ते वच्मि व्रतस्यास्य विधिं शुभम्}
{आश्विनस्यासिते पक्षे दशमी दिवसे शुभे}% ॥२०॥

\twolineshloka
{प्रातःस्नानं प्रकुर्वीत श्रद्धायुक्तेन चेतसा}
{ततो मध्याह्नसमये स्नानं कृत्वा समाहितः}% ॥२१॥

\threelineshloka
{पितॄणां प्रीतये श्राद्धं कुर्याच्छ्रद्धा समन्वितः}
{एकभक्तं ततः कृत्वा रात्रौ भूमौ शयीत च}
{प्रभाते विमले जाते प्राप्ते चैकादशी दिने}% ॥२२॥

\twolineshloka
{मुख प्रक्षालनं कुर्याद्दन्तधावन वर्जितम्}
{उपवासस्य नियमं गृह्णीयाद्भक्तिभावतः}% ॥२३॥

\twolineshloka
{अद्यस्थित्वा निराहारः सर्वभोग विवर्जितः}
{श्वो भोक्ष्ये पुण्डरीकाक्ष शरणं मे भवाच्युत}% ॥२४॥

\twolineshloka
{इत्येवं नियमं कृत्वा मध्याह्न समये तथा}
{शालग्राम शिलाग्रे तु स्नानं कुर्याद्यथाविधि}% ॥२५॥

\twolineshloka
{पूजयित्वा हृषीकेशं धूप गन्धादिभिस्तथा}
{रात्रौ जागरणं कुर्यात्केशवस्य समीपतः}% ॥२६॥

\twolineshloka
{ततः प्रभातसमये प्राप्ते वै द्वादशी दिने}
{अर्चयित्वा हरिं भक्त्या श्राद्धं कुर्याद्यथाविधि}% ॥२७॥

\twolineshloka
{पितॄणां प्रीतये श्राद्धं कुर्याच्छ्रद्धा समन्वितः}
{गोधूम चूर्णैर्यछ्राद्धं कृतं मेध्यकृतं भवेत्}% ॥२८॥

\twolineshloka
{यवैर्व्रीहितिलैर्माषैर्गोधूमैश्चणकैस्तथा}
{ब्राह्मणान्भोजयेद्राजन् दक्षिणाभिः प्रपूजितान्}% ॥२९॥

\twolineshloka
{बन्धु दौहित्र पुत्राद्यैः स्वयं भुञ्जीत वाग्यतः}
{अनेन विधिना राजन्कुरुव्रतमतन्द्रितः}% ॥३०॥

\twolineshloka
{विष्णुलोकं प्रयास्यन्ति पितरस्तव भूपते}
{इत्युक्त्वा नृपतिं राजन्मुनिरन्तरधीयत}% ॥३१॥

\twolineshloka
{यथोक्त विधिना राजा चकार व्रतमुत्तमम्}
{अन्तःपुरेण सहितः पुत्रभृत्य समन्वितः}% ॥३२॥

\twolineshloka
{कृते व्रते तु कौन्तेय पुष्पवृष्टिरभूद्दिवः}
{तत्पिता गरुडारूढो जगाम हरिमन्दिरम्}% ॥३३॥

\twolineshloka
{इन्द्रसेनोऽपिराजर्षिः कृत्वा राज्यमकण्टकम्}
{राज्ये निवेश्य तनयं जगाम त्रिदिवं स्वयम्}% ॥३४॥

\twolineshloka
{इन्दिरा व्रत माहात्म्यं तवाग्रे कथितं मया}
{पठनाच्छ्रवणाद्राजन् सर्वपापैः प्रमुच्यते}% ॥३५॥

\onelineshloka
{भुक्त्वेह निखिलान्भोगान् विष्णुलोके वसेच्चिरम्}% ॥३६॥

॥इति श्रीपाद्मे महापुराणे पञ्चपञ्चाशत्साहस्र्यां संहितायामुत्तरखण्डे उमापतिनारदसंवादान्तर्गतकृष्णयुधिष्ठिरसंवादे आश्विन-कृष्णेन्दिरा-एकादशी-माहात्म्यं नाम षष्टितमोऽध्यायः॥६०॥



\hyperref[sec:ekadashi_mahatmyam_padma_puranam]{\closesub}
\clearpage

\sect{आश्विन-शुक्ल-पाशाङ्कुशा-एकादशी-माहात्म्यम्}
\label{sec:padma-ashvina-shukla-pashankusha}


\uvacha{युधिष्ठिर उवाच}

\twolineshloka
{कथयस्व प्रसादेन ममाग्रे मधुसूदन}
{इषस्य शुक्लपक्षे तु किं नामैकादशीभवेत्}% ॥१॥

\uvacha{श्रीकृष्णउवाच}

\twolineshloka
{शृणुराजेन्द्र वक्ष्यामि माहात्म्यं पाप नाशनम्}
{शुक्लपक्षे चाश्विनस्य भवेदेकादशी तु या}% ॥२॥

\twolineshloka
{पापाङ्कुशेति विख्याता सर्वपापहरा परा}
{पद्मनाभाभिधानं मां पूजयेत्तत्र मानवः}% ॥३॥

\twolineshloka
{सर्वाभीष्ट फलं प्राप्तौ स्वर्गमोक्षप्रदं नृणाम्}
{तपस्तप्त्वा पुनस्तीव्रं चिरं सुनियतेन्द्रियः}% ॥४॥

\twolineshloka
{यत्फलं समवाप्नोति तन्न त्वा गरुडध्वजम्}
{कृत्वाऽपि बहुशःपापं नरो मोह समन्वितः}% ॥५॥

\twolineshloka
{न याति नरकं नत्वा सर्वपापहरं हरिम्}
{पृथिव्यां यानि तीर्थानि पुण्यान्यायतनानि च}% ॥६॥

\twolineshloka
{तानि सर्वाण्यवाप्नोति विष्णोर्नामानुकीर्तनात्}
{देवं शार्ङ्गधरं विष्णुं ये प्रपन्ना जनार्दनम्}% ॥७॥

\threelineshloka
{न तेषां यमलोकस्य यातना जायते क्वचित्}
{उपोष्यैकादशीमेकां प्रसङ्गेनापि मानवः}
{न याति यातनां यामीं पापं कृत्वाऽपि दारुणम्}% ॥८॥

\twolineshloka
{वैष्णवः पुरुषो भूत्वा शिवनिन्दां करोति यः}
{न विन्देद्वैष्णवं लोकं स याति नरकं ध्रुवम्}% ॥९॥

\twolineshloka
{शैवः पाशुपतो भूत्वा विष्णुनिन्दां करोति चेत्}
{रौरवे पच्यते घोरे यावदिन्द्राश्चतुर्दश}% ॥१०॥

\twolineshloka
{नेदृशं पावनं किञ्चित्त्रिषु लोकेषु विद्यते}
{यादृशं पद्मनाभस्य व्रतं पातकनाशनम्}% ॥११॥

\twolineshloka
{तावत्पापानि देहेऽस्मिन्तिष्ठन्ति मनुजाधिपम्}
{यावन्नोपवसेज्जन्तुः पद्मनाभदिनं शुभम्}% ॥१२॥

\twolineshloka
{अश्वमेधसहस्राणि राजसूयशतानि च}
{एकादश्युपवासस्य कलां नार्हन्ति षोडशीम्}% ॥१३॥

\twolineshloka
{एकादशीसमं किञ्चिद्व्रतं लोके न विद्यते}
{व्याजेनापि कृता यैश्च न ते यान्ति हि भास्करिम्}% ॥१४॥

\twolineshloka
{स्वर्गमोक्षप्रदा ह्येषा शरीरारोग्यदायिनी}
{कलत्रसुतदा ह्येषा धनमित्रप्रदायिनी}% ॥१५॥

\twolineshloka
{न गङ्गा न गया राजन्न च काशी च पुष्करम्}
{न चापि कौरवं क्षेत्रं पुण्यं भूप हरेर्दिनात्}% ॥१६॥

\twolineshloka
{रात्रौ जागरणं कृत्वा समुपोष्य हरेर्दिनम्}
{अनायासेन भूपाल प्राप्यते वैष्णवं पदम्}% ॥१७॥

\twolineshloka
{दशैव मातृके पक्षे राजेन्द्र दश पैतृके}
{प्रियाया दशपक्षे तु पुरुषानुद्धरेन्नरः}% ॥१८॥

\twolineshloka
{चतुर्भुजा दिव्यरूपा नागारिकृतकेतनाः}
{स्रग्विणः पीतवस्त्राश्च प्रयान्ति हरिमन्दिरम्}% ॥१९॥

\twolineshloka
{बालत्वे यौवनत्वे च वृद्धत्वे च नृपोत्तम}
{उपोष्यैकादशीं नूनं नैव प्राप्नोति दुर्गतिम्}% ॥२०॥

\twolineshloka
{पापाङ्कुशामुपोष्यैव आश्विनस्य सिते नरः}
{सर्वपापविनिर्मुक्तो हरिलोकं स गच्छति}% ॥२१॥

\twolineshloka
{दत्त्वा हेमतिलान्भूमिं गामन्नमुदकं तथा}
{उपानच्छत्रवस्त्रादीन्न पश्यति यमं नरः}% ॥२२॥

\twolineshloka
{यस्य पुण्यविहीनानि दिनान्यायान्ति यान्ति च}
{स लोहकारभस्त्रेव श्वसन्नपि न जीवति}% ॥२३॥

\twolineshloka
{अवन्ध्यं दिवसं कुर्याद्दरिद्रोऽपि नृपोत्तम}
{सदाचरन्यथाशक्ति स्नानदानादिकाः क्रियाः}% ॥२४॥

\twolineshloka
{होमस्नानजपध्यानसत्रादिपुण्यकर्मणाम्}
{कर्तारो नैव पश्यन्ति घोरां तां यमयातनाम्}% ॥२५॥

\twolineshloka
{दीर्घायुषो धनाढ्याश्च कुलीना रोगवर्जिताः}
{दृश्यन्ते मानवा लोके पुण्यकर्तार ईदृशाः}% ॥२६॥

\twolineshloka
{किमत्र बहुनोक्तेन यान्त्यधर्मेण दुर्गतिम्}
{आरोहन्ति दिवं धर्मैर्नात्रकार्या विचारणा}% ॥२७॥

\twolineshloka
{इति ते कथितं राजन्यत्पृष्टोऽहं त्वयानघ}
{पापाङ्कुशाया माहात्म्यं किमन्यच्छ्रोतुमिच्छसि}% ॥२८॥

॥इति श्रीपाद्मे महापुराणे पञ्चपञ्चाशत्साहस्र्यां संहितायामुत्तरखण्डे उमापतिनारदसंवादान्तर्गतकृष्णयुधिष्ठिरसंवादे आश्विन-शुक्ल-पाशाङ्कुशा-एकादशी-माहात्म्यं नाम एकषष्टितमोऽध्यायः॥६१॥


\hyperref[sec:ekadashi_mahatmyam_padma_puranam]{\closesub}
\clearpage

\sect{कार्त्तिक-कृष्ण-रमा-एकादशी-माहात्म्यम्}
\label{sec:padma-karttika-krishna-rama}


\uvacha{युधिष्ठिर उवाच}

\twolineshloka
{कथयस्व प्रसादेन मयि स्नेहाज्जनार्दन}
{कार्त्तिकस्यासिते पक्षे किन्नामैकादशी भवेत्}% ॥१॥

\uvacha{श्रीकृष्ण उवाच}

\twolineshloka
{श्रूयतां राजशार्दूल कथयामि तवाग्रतः}
{कार्त्तिके कृष्णपक्षे तु रमा नाम सुशोभना}% ॥२॥

\twolineshloka
{एकादशी समाख्याता महापापहरा परा}
{अस्याः प्रसङ्गतो राजन्माहात्म्यं प्रवदामि ते}% ॥३॥

\twolineshloka
{मुचुकुन्द इति ख्यातो बभूव नृपतिः पुरा}
{देवेन्द्रेण समं तस्य मित्रत्वमभवन्नृप}% ॥४॥

\twolineshloka
{यमेन वरुणेनैव कुबेरेणापि सर्वथा}
{विभीषणेन यस्यैव सखित्वमभवन्नृप}% ॥५॥

\twolineshloka
{विष्णुभक्तः सत्यसन्धो बभूव नृपतिः परः}
{तस्यैवं शासतो राजन्राज्यं निहतकण्टकम्}% ॥६॥

\twolineshloka
{बभूव दुहिता गेहे चन्द्रभागा सरिद्वरा}
{शोभनाय च सा दत्ता चन्द्रसेनसुताय वै}% ॥७॥

\twolineshloka
{स कदाचित्समायातः श्वशुरस्य गृहे नृप}
{एकादशीव्रतदिनं समायातं सुपुण्यदम् }% ॥८॥

\twolineshloka
{समागते व्रतदिने चन्द्रभागा त्वचिन्तयत्}
{किं भविष्यति देवेश मम भर्तातिदुर्बलः}% ॥९॥

\twolineshloka
{क्षुधां न क्षमते सोढुं पिता चैवोग्रशासनः}
{पटहस्ताड्यते यस्य सम्प्राप्ते दशमीदिने}% ॥१०॥

\twolineshloka
{न भोक्तव्यं न भोक्तव्यं न भोक्तव्यं हरेर्दिने}
{श्रुत्वा पटहनिर्घोषं शोभनस्त्वब्रवीत्प्रियाम्}% ॥११॥
\onelineshloka*
{किं कर्तव्यं मया कान्ते देहि शिक्षां वरानने}

\uvacha{चन्द्रभागोवाच}
\onelineshloka
{मत्पितुर्वेश्मनि प्रभो भोक्तव्यं नाद्य केनचित्}% ॥१२॥

\twolineshloka
{गजैरश्वैश्च कलभैरन्यैः पशुभिरेव  च}
{तृणमन्नं तथा वारि न भोक्तव्यं हरेर्दिने}% ॥१३॥

\twolineshloka
{मानवैश्च कुतः कान्त भुज्यते हरिवासरे}
{यदि त्वं भोक्ष्यसे कान्त ततो गर्हां प्रयास्यसि}% ॥१४॥
\onelineshloka*
{एवं विचार्य मनसा सुदृढं मानसं कुरु}

\uvacha{शोभन उवाच}
\onelineshloka
{सत्यमेतत्प्रिये वाक्यं करिष्येऽहमुपोषणम्}% ॥१५॥

\twolineshloka
{दैवेन विहितं यद्धि तत्तथैव भविष्यति}
{इति दृढां मतिं कृत्वा चकार नियमं व्रते}% ॥१६॥

\twolineshloka
{क्षुधया पीडिततनुः स बभूवातिदुःखितः}
{इति चिन्तयतस्तस्य आदित्योऽस्तमगाद्गिरिम्}% ॥१७॥

\twolineshloka
{वैष्णवानां नराणां सा निशा हर्षविवर्धनी}
{हरिपूजारतानां च जागरासक्तचेतसाम्}% ॥१८॥

\twolineshloka
{बभूव नृपशार्दूल शोभनस्यातिदुःखदा}
{रवेरुदयवेलायां शोभनः पञ्चतां गतः}% ॥१९॥

\twolineshloka
{दाहयामास राजा तं राजयोग्यैश्च दारुभिः}
{चन्द्रभागा नात्मदेहं ददाह पतिना सह}% ॥२०॥

\twolineshloka
{कृत्वौर्ध्वदैहिकं तस्य तस्थौ जनकवेश्मनि}
{शोभनश्च नृपश्रेष्ठ रमाव्रतप्रभावतः}% ॥२१॥

\twolineshloka
{प्राप्तो देवपुरं रम्यं मन्दराचलसानुनि}
{अनुत्तममनाधृष्यमसङ्ख्येयगुणान्वितम्}% ॥२२॥

\twolineshloka
{हेमस्तम्भमयैसौधै रत्नवैडूर्यमण्डितैः}
{स्फाटिकैर्विविधाकारैर्विचित्रैरुपशोभितैः}% ॥२३॥

\threelineshloka
{सिंहासनसमारूढः सुश्वेतच्छत्रचामरः}
{किरीटकुण्डलयुतो हारकेयूरभूषितः}
{स्तूयमानश्च गन्धर्वैरप्सरोगणसेवितः }% ॥२४॥

\twolineshloka
{शोभनः शोभते यत्र राजराजोपरो यथा}
{सोमशर्मेति विख्यातो मुचुकुन्दपुरेऽभवत्}% ॥२५॥

\twolineshloka
{तीर्थयात्राप्रसङ्गेन भ्रमन्विप्रो ददर्श तम्}
{नृपजामातरं ज्ञात्वा तत्समीपं जगाम सः}% ॥२६॥

\twolineshloka
{शोभनोऽपि तदा ज्ञात्वा सोमशर्माणमागतम्}
{आसनादुत्थितः शीघ्रं नमश्चक्रे द्विजोत्तमम्}% ॥२७॥

\twolineshloka
{चकार कुशलप्रश्नं श्वशुरस्य नृपस्य च}
{कान्तायाश्चन्द्रभागायास्तथैव नगरस्य च}% ॥२८॥

\uvacha{सोमशर्मोवाच}

\twolineshloka
{कुशलं वर्तते राजन्श्वशुरस्य गृहे तव}
{चन्द्रभागा कुशलिनी सर्वतः कुशलं पुरे}% ॥२९॥

\twolineshloka
{स्ववृत्तं कथ्यतां राजन्नाश्चर्यं विद्यतेऽद्भुतम्}
{पुरं विचित्रं रुचिरं न दृष्टं केनचित्क्वचित्}% ॥३०॥
\onelineshloka*
{एतदाचक्ष्व नृपते कुतः प्राप्तमिदं त्वया}

\uvacha{शोभन उवाच}
\onelineshloka
{कार्तिकस्यासिते पक्षे या नामैकादशी रमा}% ॥३१॥

\twolineshloka
{तामुपोष्य मया प्राप्तं द्विजेन्द्र पुरमध्रुवम्}
{ध्रुवं भवति येनैव तत्कुरुष्व द्विजोत्तमः}% ॥३२॥

\uvacha{द्विज उवाच}

\twolineshloka
{कथमध्रुवमेतद्धि कथं हि भवति ध्रुवम्}
{तत्त्वं कथय राजेन्द्र तत्करिष्यामि नान्यथा}% ॥३३॥

\uvacha{शोभन उवाच}

\twolineshloka
{मयैतद्विहितं विप्र श्रद्धाहीनं व्रतोत्तमम्}
{तेनाहमध्रुवं मन्ये ध्रुवं भवति तच्छृणु}% ॥३४॥

\twolineshloka
{मुचुकुन्दस्य दुहिता चन्द्रभागा सुशोभना}
{तस्यै कथय वृत्तान्तं ध्रुवमेतद्भविष्यति}% ॥३५॥

\uvacha{कृष्ण उवाच}

\twolineshloka
{स श्रुत्वा वचनं तस्य मुचुकुन्दपुरं गतः}
{उवाच सर्वं वृत्तान्त्तं चन्द्रभागाग्रतो द्विजः}% ॥३६॥

\uvacha{सोमशर्मोवाच}

\threelineshloka
{प्रत्यक्षं दयितः कान्तस्तव दृष्टो मया शुभे}
{इन्द्रतुल्यमनाधृष्यं दृष्टं तस्य पुरं मया}
{अध्रुवं तेन तत्प्रोक्तं ध्रुवं भवति तत्कुरु}% ॥३७॥

\uvacha{चन्द्रभागोवाच}

\twolineshloka
{तत्र मां नय विप्रर्षे पतिदर्शनलालसाम्}
{आत्मनो व्रतपुण्येन करिष्यामि पुरं ध्रुवम्}% ॥३८॥

\twolineshloka
{आवयोर्द्विजसंयोगो यथा भवति तत्कुरु}
{प्राप्यते हि महत्पुण्यं कृत्वा योगं वियुक्तयोः}% ॥३९॥

\twolineshloka
{इति श्रुत्वा सह तया सोमशर्मा जगाम ह}
{आश्रमं वामदेवस्य मन्दराचलसन्निधौ}% ॥४०॥

\twolineshloka
{वामदेवोऽशृणोत्सर्वं वृत्तान्तं कथितं तयोः}
{अभ्यषिञ्चच्चन्द्रभागां वेदमन्त्रैरथोज्ज्वलाम्}% ॥४१॥

\twolineshloka
{ऋषिमन्त्रप्रभावेन विष्णुवासरसेवनात्}
{दिव्यदेहा बभूवासौ दिव्यां गतिमवाप ह}% ॥४२॥

\twolineshloka
{पत्युः समीपमगमत्प्रहर्षोत्फुल्ललोचना}
{जहर्ष शोभनोऽतीव दृष्ट्वा कान्तां समागताम्}% ॥४३॥

\twolineshloka
{समाहूय स्वके वामे पार्श्वेतां सन्न्यवेशयत्}
{अथोवाच प्रियं हर्षाच्चन्द्रभागा प्रियं वचः}% ॥४४॥

\twolineshloka
{शृणु कान्त हितं वाक्यं यत्पुण्यं विद्यते मयि}
{अष्टवर्षाधिका जाता यदाहं पितृवेश्मनि}% ॥४५॥

\threelineshloka
{ततः प्रभृति यच्चीर्णं मया चैकादशीव्रतम्}
{यथोक्तविधिसंयुक्तं श्रद्धायुक्तेन चेतसा}
{तेन पुण्यप्रभावेन भविष्यति पुरं ध्रुवम्}% ॥४६॥

\twolineshloka
{सर्वकामसमृद्धं च यावदाभूतसम्प्लवम्}
{एवं सा नृपशार्दूल रमते पतिना सह}% ॥४७॥

\twolineshloka
{दिव्यभोगा दिव्यरूपा दिव्याभरणभूषिता}
{शोभनोऽपि तया सार्धं रमते दिव्यविग्रहः}% ॥४८॥

\twolineshloka
{रमाव्रतप्रभावेन मन्दराचलसानुनि}
{चिन्तामणिसमा ह्येषा कामधेनुसमाथ वा}% ॥४९॥

\twolineshloka
{रमाभिधाना नृपते तवाग्रे कथिता मया}
{तस्या माहात्म्यमनघ श्रुतं सर्वं त्वया नृप}% ॥५०॥

\twolineshloka
{मया तवाग्रे कथितं माहात्म्यं पापनाशनम्}
{एकादशीव्रतानां च पक्षयोरुभयोरपि}% ॥५१॥

\twolineshloka
{यथा कृष्णा तथा शुक्ला विभेदं नैव कारयेत्}
{सेवितैकादशी नॄणां भुक्तिमुक्तिप्रदायिनी}% ॥५२॥

\twolineshloka
{धेनुः श्वेता यथा कृष्णा उभयोः सदृशं पयः}
{तथैव तुल्यफलदं स्मृतमेकादशीद्वयम्}% ॥५३॥

\twolineshloka
{एकादशीव्रतानां यो माहात्म्यं शृणुते नरः}
{सर्वपापविनिर्मुक्तो विष्णुलोके महीयते}% ॥५४॥

॥इति श्रीपाद्मे महापुराणे पञ्चपञ्चाशत्साहस्र्यां संहितायामुत्तरखण्डे उमापतिनारदसंवादान्तर्गतकृष्णयुधिष्ठिरसंवादे कार्त्तिक-कृष्ण-रमा-एकादशी-माहात्म्यं नाम द्विषष्टिमोऽध्यायः॥६२॥


\hyperref[sec:ekadashi_mahatmyam_padma_puranam]{\closesub}
\clearpage

\sect{कार्त्तिक-शुक्ल-प्रबोधिनी-एकादशी-माहात्म्यम्}
\label{sec:padma-karttika-shukla-prabodhini}


\uvacha{युधिष्ठिर उवाच}

\twolineshloka
{श्रुतं रमाया माहात्म्यं त्वत्तः कृष्ण यथातथम्}
{कार्तिके शुक्लपक्षे या तां मे कथय मानद}% ॥१॥

\uvacha{श्रीकृष्ण उवाच}

\twolineshloka
{शृणु राजन्प्रवक्ष्यामि शुक्ले चोर्जदले तु या}
{सा यथा नारदे प्रोक्ता ब्रह्मणा लोककारिणा}% ॥२॥

\uvacha{नारद उवाच}

\twolineshloka
{प्रबोधिन्याश्च माहात्म्यं वद विस्तरतो मम}
{यस्यां जागर्ति गोविन्दो धर्मकर्मप्रवर्तकः}% ॥३॥

\uvacha{ब्रह्मोवाच}

\twolineshloka
{प्रबोधिन्याश्च माहात्म्यं पापघ्नं पुण्यवर्धनम्}
{मुक्तिप्रदं सुबुद्धीनां शृणुष्व मुनिसत्तम}% ॥४॥

\twolineshloka
{तावद्गर्जन्ति तीर्थानि आसमुद्रसरांसि च}
{यावत्प्रबोधिनी विष्णोस्तिथिर्नायाति कार्तिके}% ॥५॥

\twolineshloka
{तावद्गर्जन्ति विप्रेन्द्र गङ्गाभागीरथी क्षितौ}
{यावन्नायाति पापघ्नी कार्तिके हरिबोधिनी}% ॥६॥

\twolineshloka
{अश्वमेधसहस्राणि राजसूयशतानि च}
{एकेनैवोपवासेन प्रबोधिन्या लभेन्नरः}% ॥७॥

\twolineshloka
{यद्दुर्ल्लभं यदप्राप्यं त्रैलोक्यस्य न गोचरम्}
{तदप्यप्रार्थितं पुत्रं ददाति हरिबोधिनी}% ॥८॥

\twolineshloka
{ऐश्वर्यं सम्पदं प्रज्ञां राज्यं च सुखसम्पदः}
{ददात्युपोषिता भक्त्या जनेभ्यो हरिबोधिनी}% ॥९॥

\twolineshloka
{मेरुमन्दरमात्राणि पापान्युक्तानि यानि च}
{एकेनैवोपवासेन दहते पापनाशिनी}% ॥१०॥

\twolineshloka
{पूर्वजन्मसहस्रेषु यत्पापं समुपार्जितम्}
{निशि जागरणं चास्या दहते तूलराशिवत्}% ॥११॥

\twolineshloka
{उपवासं प्रबोधिन्यां यः करोति स्वभावतः}
{विधिवन्मुनिशार्दूल यथोक्तं लभते फलम्}% ॥१२॥

\twolineshloka
{यथोक्तं कुरुते यस्तु विधिवत्सुकृतं नरः}
{स्वल्पं मुनिवरश्रेष्ठ मेरुतुल्यं भवेत्फलम्}% ॥१३॥

\twolineshloka
{विधिहीनं तु यः कुर्यात्सुकृतं मेरुमात्रकम्}
{अणुमात्रं तदाप्नोति फलं धर्मस्य नारद}% ॥१४॥

\twolineshloka
{ये ध्यायन्ति मनोवृत्या ये करिष्यन्ति बोधिनीम्}
{वसन्ति पितरो हृष्टा विष्णुलोके च तस्य वै}% ॥१५॥

\twolineshloka
{विमुक्ता नारकैर्दुःखैर्याति विष्णोः परं पदम्}
{कृत्वा तु पातकं घोरं ब्रह्महत्यादिकं नरः}% ॥१६॥

\twolineshloka
{कृत्वा तु जागरं विष्णोर्धौतपापो भवेन्नरः}
{दुष्प्राप्यं यत्फलं विप्र अश्वमेधादिकैर्मखैः}% ॥१७॥

\twolineshloka
{प्राप्यते तत्सुखेनैव प्रबोधिन्यास्तु जागरे}
{आप्लुत्य सर्वतीर्थेषु प्रदत्त्वा काञ्चनं महीम्}% ॥१८॥

\twolineshloka
{तत्फलं समवाप्नोति यत्कृत्वा जागरं हरेः}
{जातः स एव सुकृती कुलं तेनैव पावितम्}% ॥१९॥

\twolineshloka
{कार्तिके मुनिशार्दूल कृता येन प्रबोधिनी}
{यथा ध्रुवं नृणां मृत्युर्धनं गात्रं तथाध्रुवम्}% ॥२०॥

\twolineshloka
{इति ज्ञात्वा मुनिश्रेष्ठ कर्तव्यं वैष्णवं दिनम्}
{यानि कानि च तीर्थानि त्रैलोक्ये सम्भवन्ति च}% ॥२१॥

\twolineshloka
{तानि तस्य गृहे सम्यग्यः करोति प्रबोधिनीम्}
{किं तस्य बहुभिः पुण्यैः कृता येन प्रबोधिनी}% ॥२२॥

\twolineshloka
{पुत्रपौत्रप्रदा ह्येषा कार्तिके हरिबोधिनी}
{स ज्ञानी च स योगी च स तपस्वी जितेन्द्रियः}% ॥२३॥

\twolineshloka
{भोगो मोक्षश्च तस्यास्ति उपास्ते हरिबोधिनीम्}
{विष्णोः प्रियतरा ह्येषा धर्मसारसहायिनी}% ॥२४॥

\twolineshloka
{यः करोति नरो भक्त्या भुक्तिभाक्स भवेन्नरः}
{प्रबोधिनीमुपोषित्वा गर्भे न विशते नरः}% ॥२५॥

\twolineshloka
{सर्वधर्मान्परित्यज्य तस्मात्कुर्वीत नारद}
{कर्मणा मनसा वाचा पापं यत्समुपार्जितम्}% ॥२६॥

\twolineshloka
{तत्क्षालयति गोविन्दः प्रबोधिन्यां तु जागरे}
{स्नानं दानं जपः पूजां समुद्दिश्य जनार्दनम्}% ॥२७॥

\twolineshloka
{नरो यत्कुरुते वत्स प्रबोधिन्यां तदक्षयम्}
{येर्चयन्ति नरास्तस्यां भक्त्या देवं च माधवम्}% ॥२८॥

\twolineshloka
{समुपोष्य प्रमुच्यन्ते पापैस्तैः शतजन्मजैः}
{महाव्रतमिदं पुत्र महापापौघनाशनम्}% ॥२९॥

\twolineshloka
{प्रबोधवासरं विष्णोर्विधिवत्समुपोषयेत्}
{व्रतेनानेन देवेशं परितोष्य जनार्दनम्}% ॥३०॥

\twolineshloka
{विराजयन्दिशः सर्वाः प्रयाति हरिमन्दिरम्}
{कर्तव्यैषा प्रयत्नेन नरैः कान्तिधनार्थिभिः}% ॥३१॥

\twolineshloka
{बाल्ये यत्सञ्चितं पापं यौवने वार्धके तथा}
{शतजन्मकृतं पापं स्वल्पं वा यदि वा बहु}% ॥३२॥

\twolineshloka
{तत्क्षालयति गोविन्दश्चास्यामभ्यर्चितो नृणाम्}
{धनधान्यवहा पुण्या सर्वपापहरा परा}% ॥३३॥

\twolineshloka
{तामुपोष्य हरेर्भक्त्या दुर्ल्लभं न भवेत्क्वचित्}
{चन्द्रसूर्योपरागे च यत्फलं परिकीर्तितम्}% ॥३४॥

\twolineshloka
{तत्सहस्रगुणं प्रोक्तं प्रबोधिन्यां प्रजागरे}
{स्नानं दानं जपो होमः स्वाध्यायोऽभ्यर्चनं हरेः}% ॥३५॥

\twolineshloka
{तत्सर्वं कोटिगुणितं प्रबोधिन्यां कृतं तु यत्}
{जन्मप्रभृतियत्पुण्यं नरेणोपार्जितं भवेत्}% ॥३६॥

\twolineshloka
{वृथा भवति तत्सर्वमकृत्वा कार्तिके व्रतम्}
{अकृत्वा नियमं विष्णोः कार्तिकं यः क्षिपेन्नरः}% ॥३७॥

\twolineshloka
{न जन्मार्जितपुण्यस्य फलं प्राप्नोति नारद}
{तस्मात्सर्वप्रयत्नेन देवदेवं जनार्दनम्}% ॥३८॥

\twolineshloka
{उपसेवेत विप्रेन्द्र सर्वकामफलप्रदम्}
{परान्नं वर्जयेद्यस्तु कार्तिके विष्णुतत्परः}% ॥३९॥

\twolineshloka
{परान्नवर्जनाद्वत्स चान्द्रायणफलं लभेत्}
{नित्यं शास्त्रविनोदेन कार्तिके मधुसूदनः}% ॥४०॥

\twolineshloka
{स दहेत्सर्वपापानि यज्ञायुतफलं लभेत्}
{न तथा तुष्यते यज्ञैर्न दानैर्वाजपादिभिः}% ॥४१॥

\twolineshloka
{यथा शास्त्रकथालापैः कार्तिके मधुसूदनः}
{ये कुर्वन्ति कथां विष्णोर्ये शृण्वन्ति शुभान्विताः}% ॥४२॥

\twolineshloka
{श्लोकार्धं श्लोकपादं वा कार्तिके गोशतं फलम्}
{सर्वधर्मान्परित्यज्य कार्तिके केशवाग्रतः}% ॥४३॥

\twolineshloka
{शास्त्रावधारणं कार्यं श्रोतव्यं च महामुने}
{श्रेयसा लोभबुद्ध्या च यः करोति हरेः कथाम्}% ॥४४॥

\twolineshloka
{कार्तिके मुनिशार्दूल कुलानां तारयेच्छतम्}
{नियमेन नरो यस्तु शृणुते वैष्णवीं कथाम्}% ॥४५॥

\twolineshloka
{कार्तिके तु विशेषेण गोसहस्रफलं लभेत्}
{प्रबोधवासरे विष्णोः शृणुते यो हरेः कथाम्}% ॥४६॥

\twolineshloka
{सप्तद्वीपवती दाने तत्फलं लभते मुने}
{श्रुत्वा विष्णुकथां दिव्यां येऽर्चयन्ति कथाविदम्}% ॥४७॥

\twolineshloka
{स्वशक्त्या मुनिशार्दूल तेषां लोकोऽक्षयः स्मृतः}
{गीतशास्त्रविनोदेन कार्तिकं यो नयेन्नरः}% ॥४८॥

\twolineshloka
{न तस्य पुनरावृत्तिर्मया दृष्टा कलिप्रिय}
{गीतं नृत्यं च वाद्यं च भव्यां विष्णुकथां मुने}% ॥४९॥

\twolineshloka
{यः करोति स पुण्यात्मा त्रैलोक्योपरि संस्थितः}
{बहुपुष्पैर्बहुफलैः कर्पूरागुरुकुङ्कुमैः}% ॥५०॥

\twolineshloka
{हरेः पूजा विधातव्या कार्तिके बोधवासरे}
{यस्मात्पुण्यमसङ्ख्यातं प्राप्यते मुनिसत्तम}% ॥५१॥

\twolineshloka
{फलैर्नानाविधैर्द्रव्यैः प्रबोधिन्यां तु जागरे}
{शङ्खे तोयं समादाय अर्घो देयो जनार्दने}% ॥५२॥

\twolineshloka
{यत्फलं सर्वतीर्थेषु सर्वदानेषु यत्फलम्}
{तत्फलं कोटिगुणितं दत्त्वार्घं बोधवासरे}% ॥५३॥

\twolineshloka
{गुरुपूजा ततः कार्या भोजनाच्छादनादिभिः}
{दक्षिणाभिश्च देवर्षे तुष्ट्यर्थं चक्रपाणिनः}% ॥५४॥

\twolineshloka
{भागवतं शृणुते यस्तु पुराणं च पठेन्नरः}
{प्रत्यक्षरं भवेत्तस्य कपिलादानजं फलम्}% ॥५५॥

\twolineshloka
{कार्त्तिके मुनिशार्दूल स्वशक्त्या वैष्णवं व्रतम्}
{यः करोति यथोक्तं तु मुक्तिस्तस्य सुनिश्चला}% ॥५६॥

\twolineshloka
{केतक्या एकपत्रेण पूजितो गरुडध्वजः}
{समाः सहस्रं सुप्रीतो भवति मधुसूदनः}% ॥५७॥

\twolineshloka
{अगस्तिकुसुमैर्देवं पूजयेद्यो जनार्दनम्}
{दर्शनात्तस्य देवर्षे नरकाग्निः प्रणश्यति}% ॥५८॥

\twolineshloka
{मुनिपुष्पार्चितो विष्णुः कार्तिके पुरुषोत्तमः}
{ददात्यभिमतान्कामान्शशिसूर्यग्रहे यथा}% ॥५९॥

\twolineshloka
{विहाय सर्वपुष्पाणि मुनिपुष्पेण केशवम्}
{कार्तिके योऽर्चयेद्भक्त्या वाजिमेधफलं लभेत्}% ॥६०॥

\twolineshloka
{तुलसीदलानि पुष्पाणि ये यच्छन्ति जनार्दने}
{कार्तिके सकलं वत्स पापं जन्मायुतं दहेत्}% ॥६१॥

\twolineshloka
{दृष्टा स्पृष्टाथ वा ध्याता कीर्तिता नामतस्तु ता}
{रोपिता सिञ्चिता नित्यं पूजिता तुलसी शुभा}% ॥६२॥

\twolineshloka
{नवधा तुलसीभक्तिं ये कुर्वन्ति दिनेदिने}
{युगकोटिसहस्राणि तन्वन्ति सुकृतं मुने}% ॥६३॥

\twolineshloka
{यावच्छाखाप्रशाखाभिर्बीजपुष्पदलैर्मुने}
{रोपिता तुलसी पुम्भिर्वर्धते वसुधातले}% ॥६४॥

\twolineshloka
{तेषां वंशे तु ये जाता ये भविष्यन्ति ये गताः}
{आकल्पवर्षसाहस्रं तेषां वासो हरेर्गृहे}% ॥६५॥

\twolineshloka
{यत्फलं सर्वपुष्पेषु सर्वपत्रेषु नारद}
{तुलसीदलेन चैकेन कार्तिके प्राप्यते तु तत्}% ॥६६॥

\twolineshloka
{सम्प्राप्तं कार्त्तिकं दृष्ट्वा नियमेन जनार्दनः}
{पूजनीयो महाविष्णुः कोमलैस्तुलसीदलैः}% ॥६७॥

\twolineshloka
{इष्ट्वा क्रतुशतैर्देवान्दत्त्वा दानान्यनेकशः}
{तुलसीदलैस्तु तत्पुण्यं कार्तिके केशवार्चने}% ॥६८॥

॥इति श्रीपाद्मे महापुराणे पञ्चपञ्चाशत्साहस्र्यां संहितायामुत्तरखण्डे उमापतिनारदसंवादान्तर्गतकृष्णयुधिष्ठिरसंवादे कार्त्तिक-शुक्ल-प्रबोधिनी-एकादशी-माहात्म्यं नाम त्रिषष्टितमोऽध्यायः॥६३॥


\hyperref[sec:ekadashi_mahatmyam_padma_puranam]{\closesub}
\clearpage

\sect{पुरुषोत्तम-मासस्य कृष्ण-कमला-एकादशी-माहात्म्यम्}
\label{sec:padma-purushottama-masasya krishna-kamala}


\uvacha{युधिष्ठिर उवाच}

\twolineshloka
{भगवञ्छ्रोतुमिच्छामि व्रतानामुत्तमं व्रतम्}
{सर्वपापहरं विष्णोः फलदं व्रतिनां च यत्}% ॥१॥

\twolineshloka
{पुरुषोत्तममासस्य कथां ब्रूहि जनार्दन}
{को विधिः किं फलं तस्य को देवस्तत्र पूज्यते}% ॥२॥

\twolineshloka
{अधिमासे च सम्प्राप्ते व्रतं ब्रूहि जनार्दन}
{कस्य दानस्य किं पुण्यं किं कर्तव्यं नृभिः प्रभो}% ॥३॥

\twolineshloka
{कथं स्नानं च किं जाप्यं कथं पूजाविधिः स्मृतः}
{किं भोज्यमुत्तमं चान्नं मासेऽस्मिन्पुरुषोत्तमे}% ॥४॥

\uvacha{श्रीकृष्ण उवाच}

\twolineshloka
{कथयिष्यामि राजेन्द्र भवतः स्नेहकारणात्}
{पुरुषोत्तममासस्य माहात्म्यं पापनाशनम्}% ॥५॥

\twolineshloka
{अधिमासे तु सम्प्राप्ते भवेदेकादशी तु या}
{कमला नाम सा नाम्ना तिथीनामुत्तमा तिथिः}% ॥६॥

\twolineshloka
{तस्या व्रतप्रभावेन कमलाभिमुखी भवेत्}
{ब्राह्मे मुहूर्ते चोत्थाय संस्मृत्य पुरुषोत्तमम्}% ॥७॥

\twolineshloka
{स्नात्वा चैव विधानेन नियमं कारयेद्व्रती}
{गृहे त्वेकगुणं जाप्यं नद्यां तु द्विगुणं स्मृतम्}% ॥८॥

\twolineshloka
{गवां गोष्ठे सहस्रोर्ध्वमग्न्यागारे शतान्वितम्}
{शिवक्षेत्रेषु तीर्थेषु देवतानां च सन्निधौ}% ॥९॥

\twolineshloka
{लक्षं तुलस्याः सान्निध्ये ह्यनन्तं विष्णुसन्निधौ}
{अवन्त्यामभवत्कश्चिच्छिवशर्मा द्विजोत्तमः}% ॥१०॥

\twolineshloka
{तस्यात्मजास्तु पञ्चासन्कनिष्ठो दोषवानभूत्}
{तदा पित्रा परित्यक्तस्त्यक्तः स्वजनबान्धवैः}% ॥११॥

\twolineshloka
{कुकर्मणः प्रभावेन गतो दूरतरं वनम्}
{एकदा दैवयोगेन तीर्थराजं समागतः}% ॥१२॥

\twolineshloka
{क्षुत्क्षामो दीनवदनस्त्रिवेण्यां स्नानमाचरत्}
{मुनीनामाश्रमांस्तत्र विचिन्वन्क्षुधयार्दितः}% ॥१३॥

\twolineshloka
{हरिमित्रमुनेस्तत्र ददर्शाश्रममुत्तमम्}
{पुरुषोत्तममासे वै जनानां च समागमे}% ॥१४॥

\twolineshloka
{तत्राश्रमे कथयतां कथां कल्मषनाशिनीम्}
{ब्राह्मणानां मुखात्तेन श्रद्धया कमला श्रुता}% ॥१५॥

\twolineshloka
{एकादशी पुण्यतमा भुक्तिमुक्तिप्रदायिनी}
{जयशर्मा विधानेन श्रुत्वेमां कमला तिथिम्}% ॥१६॥

\twolineshloka
{एकादशीपुण्यतमा भुक्तिमुक्तिप्रदायिनी}
{व्रतं कृतं च तैः सार्धं स्थित्वा शून्यालये तदा}% ॥१७॥

\twolineshloka
{निशीथे समनुप्राप्ते लक्ष्मीस्तत्र समागता}
{वरं ददामि भो विप्र कमलायाः प्रभावतः}% ॥१८॥

\uvacha{जयशर्मोवाच}

\twolineshloka
{का त्वं कस्यासि रम्भोरु प्रसन्ना च कथं मम}
{इन्द्राणी सुरनाथस्य भवानी शङ्करस्य वा}% ॥१९॥

\twolineshloka
{गन्धर्वी किन्नरी वाऽथ वधूर्वा चन्द्रसूर्ययोः}
{त्वत्सदृक्षा न दृष्टा च न श्रुता च शुभानने}% ॥२०॥

\uvacha{लक्ष्मीरुवाच}

\twolineshloka
{प्रसन्ना साम्प्रतं जाता वैकुण्ठादहमागता}
{प्रेरिता देवदेवेन कमलायाः प्रभावतः}% ॥२१॥

\twolineshloka
{पुरुषोत्तममासस्य या पक्षे प्रथमे भवेत्}
{तस्या व्रतं त्वया चीर्णं प्रयागे मुनिसन्निधौ}% ॥२२॥

\twolineshloka
{व्रतस्यास्य प्रभावेन वशगाहं न संशयः}
{तववंशे भविष्यन्ति मानवा द्विजसत्तम}% ॥२३॥
\onelineshloka*
{मत्प्रसादादवाप्स्यन्ति सत्यं ते व्याहृतं मया}

\uvacha{ब्राह्मण उवाच}
\onelineshloka
{प्रसन्ना यदि मे पद्मे व्रतं विस्तरतो वद}% ॥२४॥
\onelineshloka*
{यत्कथासु प्रवर्तन्ते साधवो ये जना द्विजाः}

\uvacha{लक्ष्मीरुवाच}
\onelineshloka
{श्रोतॄणां परमं श्राव्यं पवित्राणामनुत्तमम्}% ॥२५॥

\twolineshloka
{दुःस्वप्ननाशनं पुण्यं श्रोतव्यं यत्नतस्ततः}
{उत्तमः श्रद्धया युक्तः श्लोकं श्लोकार्धमेव वा}% ॥२६॥

\twolineshloka
{पठित्वा मुच्यते सद्यो महापातककोटिभिः}
{मासानां परमो मासः पक्षिणां गरुडो यथा}% ॥२७॥

\twolineshloka
{नदीनां च यथा गङ्गा तिथीनां द्वादशी तिथिः}
{अद्यापि निर्जराः सर्वे भारते जन्मलिप्सवः}% ॥२८॥

\twolineshloka
{तमर्चयन्ति विविधा नारायणमनामयम्}
{ये यजन्ति सदा भक्त्या देवं नारायणं प्रभुम्}% ॥२९॥

\twolineshloka
{तानर्चयन्ति सततं ब्रह्माद्या देवतागणाः}
{येऽपि नामपरा ये च हरिकीर्तनतत्पराः}% ॥३०॥

\twolineshloka
{हरिपूजापरा ये च ते कृतार्थाः कलौ युगेः}
{शुक्ले वा यदि वा कृष्णे भवेदेकादशी द्वयम्}% ॥३१॥

\threelineshloka
{गृहस्थानां भवेत्पूर्वा यतीनामुत्तरा स्मृता}
{एकादशी द्वादशी च रात्रिशेषे त्रयोदशी}
{तत्र क्रतुशतं पुण्यं त्रयोदश्यां तु पारणे}% ॥३२॥

\twolineshloka
{एकादश्यां निराहारः स्थित्वाहमपरेऽहन्नि}
{भोक्ष्यामि पुण्डरीकाक्ष शरणं मे भवाच्युत}% ॥३३॥

\twolineshloka
{अमुं मन्त्रं समुच्चार्य देवदेवस्य चक्रिणः}
{भक्तिभावेन तुष्टात्मा उपवासं समाचरेत्}% ॥३४॥

\twolineshloka
{कुर्याद्देवस्य पुरतो जागरं नियतो व्रती}
{गीतैर्वाद्यैश्च नृत्यैश्च पुराणपठनादिभिः}% ॥३५॥

\twolineshloka
{ततः प्रातः समुत्थाय द्वादशीदिवसे व्रती}
{स्नात्वा विष्णुं समभ्यर्च्य विधिवत्प्रयतेन्द्रियः}% ॥३६॥

\twolineshloka
{पञ्चामृतेन संस्नाप्य एकादश्यां जनार्दनः}
{द्वादश्यां च पयःस्नानाद्धरेः सारूप्यमश्नुते}% ॥३७॥

\twolineshloka
{अज्ञानतिमिरान्धस्य व्रतेनानेन केशव}
{प्रसीद सन्मुखो भूत्वा ज्ञानदृष्टिप्रदो भव}% ॥३८॥

\twolineshloka
{एवं विज्ञाप्य देवेशं देवदेवं गदाधरम्}
{ब्राह्मणान्भोजयेद्भक्त्या तेभ्यो दद्याच्च दक्षिणाम्}% ॥३९॥

\twolineshloka
{ततः स्वबन्धुभिः सार्धं नारायणपरायणः}
{कृत्वा पञ्चमहायज्ञान्स्वयं भुञ्जीत वाग्यतः}% ॥४०॥

\twolineshloka
{एवं यः प्रयतः कुर्यात्पुण्यमेकादशीव्रतम्}
{स याति विष्णुभवनं पुनरावृत्तिदुर्ल्लभम्}% ॥४१॥

\twolineshloka
{इत्युक्त्वा कमला तस्मै वरं दत्त्वा तिरोदधे}
{सोऽपि विप्रो धनी भूत्वा पितुर्गेहं समागतः}% ॥४२॥

\uvacha{श्रीकृष्ण उवाच}

\twolineshloka
{एवं यः कुरुते राजन्कमलाव्रतमुत्तमम्}
{शृणुयाद्वासरे विष्णोः सर्वपापैः प्रमुच्यते}% ॥४३॥

॥इति श्रीपाद्मे महापुराणे पञ्चपञ्चाशत्साहस्र्यां संहितायामुत्तरखण्डे उमापतिनारदसंवादान्तर्गतकृष्णयुधिष्ठिरसंवादे पुरुषोत्तम-मासस्य कृष्ण-कमला-एकादशी-माहात्म्यं नाम चतुःषष्टितमोऽध्यायः॥६४॥


\hyperref[sec:ekadashi_mahatmyam_padma_puranam]{\closesub}
\clearpage

\sect{पुरुषोत्तम-मासस्य शुक्ल-कामदा-एकादशी-माहात्म्यम्}
\label{sec:padma-purushottama-masasya shukla-kamada}


\uvacha{युधिष्ठिर उवाच}

\twolineshloka
{श्रुतानि बहुधर्माणि व्रतानि च जगत्प्रभो}
{एकादशीसमं किञ्जिच्छ्रुतं नैव जनार्दन}% ॥१॥

\twolineshloka
{पुनस्त्वेकादशीं ब्रूहि पापघ्नीं पुण्यदायिनीम्}
{यां कृत्वा मनुजो लोके प्राप्नुयात्परमं पदम्}% ॥२॥

\uvacha{श्रीकृष्ण उवाच}

\twolineshloka
{शुक्ले वा यदि वा कृष्णे यदा चैकादशी भवेत्}
{न त्याज्या जगतीपाल मोक्षसौख्यविवर्धनी}% ॥३॥

\twolineshloka
{एकादशी कलौ राजन्भवबन्धविमोचनी}
{कामदा सर्वकामानां पापानां पापहा भुवि}% ॥४॥

\twolineshloka
{रविवारेऽथ माङ्गल्ये सङ्क्रमे वा नृपोत्तम}
{एकादशी सदोपोष्या पुत्रपौत्रविवर्धनी}% ॥५॥

\twolineshloka
{एकादशीव्रतं क्वापि न त्याज्यं विष्णुवल्लभैः}
{आयुः कीर्तिप्रदं नित्यं सन्तानारोग्यवित्तदम्}% ॥६॥

\twolineshloka
{मोक्षदं रूपदं राज्यं नित्यमेकादशीव्रतम्}
{ये कुर्वन्ति महीपाल श्रद्धया परया युताः}% ॥७॥

\twolineshloka
{यथोक्तविधिना लोके ते नरा विष्णुरूपिणः }
{जीवन्मुक्तास्तु भूपाल दृश्यन्ते नात्र संशयः}% ॥८॥

\uvacha{युधिष्ठिर उवाच}

\twolineshloka
{जीवन्मुक्ताः कथं कृष्ण विष्णुरूपाः कथं पुनः}
{पापरूपाश्च दृश्यन्ते परं कौतूहलं हि मे}% ॥९॥

\uvacha{श्रीकृष्ण उवाच}

\twolineshloka
{ये च राजन्कलौ भक्त्या निर्जलं व्रतमुत्तमम्}
{एकादश्याः प्रकुर्वन्ति विधिदृष्टेन कर्मणा}% ॥१०॥

\twolineshloka
{न कथं विष्णुरूपास्ते जीवन्मुक्ताः कथं नहि}
{सर्वपापहरं पुण्यं व्रतमेकादशीसमम्}% ॥११॥

\twolineshloka
{न किञ्चिद्विद्यते राजन्सर्वकामप्रदं नृणाम्}
{एकाशनं दशम्यां च नन्दायां निर्जलं व्रतम्}% ॥१२॥

\twolineshloka
{पारणं चैव भद्रायां कृत्वा विष्णुसमा नराः}
{श्रद्धावान्यस्तु कुरुते कामदाया व्रतं शुभम्}% ॥१३॥

\twolineshloka
{वाञ्छितं लभते सोऽपि इहलोके परत्र च}
{पवित्रा पावनी ह्येषा महापातकनाशिनी}% ॥१४॥

\twolineshloka
{भुक्तिमुक्तिप्रदा चैव कर्तॄणां नृपसत्तम}
{कामदायां विधानेन पूजयेत्पुरुषोत्तमम्}% ॥१५॥

\twolineshloka
{पुष्पधूपादिभिश्चैव नैवेद्यैर्विविधैस्तथा}
{कांस्य मांसमसूरांश्च चणकान्कोद्रवांस्तथा}% ॥१६॥

\twolineshloka
{शाकं मधु परान्नं च पुनर्भोजन मैथुनम्}
{वैष्णवो व्रतकर्ता च दशम्यां दश वर्जयेत्}% ॥१७॥

\twolineshloka
{द्यूतं क्रीडां तथा निद्रा ताम्बूलं दन्तधावनम्}
{परापवादं पैशुन्यं स्तेयं हिंसां तथा रतिम्}% ॥१८॥

\twolineshloka
{क्रोधं च वितथं वाक्यमेकादश्यां विवर्जयेत्}
{कांस्यं मांसं मसूरांश्च तैलं वितथभाषणम्}% ॥१९॥

\twolineshloka
{व्यायामं च प्रवासं च पुनर्भोजनमैथुनम्}
{वृषपृष्ठं परान्नं च शाकं च द्वादशीदिने}% ॥२०॥

\twolineshloka
{अनेन विधिना राजन्विहिता यैश्च कामदा}
{रात्रौ जागरणं कृत्वा पूजितः पुरुषोत्तमः}% ॥२१॥

\twolineshloka
{सर्वपापविनिर्मुक्तास्ते यान्ति परमां गतिम्}
{पठनाच्छ्रवणाद्राजन्गोसहस्रफलं लभेत्}% ॥२२॥

॥इति श्रीपाद्मे महापुराणे पञ्चपञ्चाशत्साहस्र्यां संहितायामुत्तरखण्डे उमापतिनारदसंवादान्तर्गतकृष्णयुधिष्ठिरसंवादे पुरुषोत्तम-मासस्य शुक्ल-कामदा-एकादशी-माहात्म्यं नाम पञ्चषष्टितमोऽध्यायः॥६५॥

\hyperref[sec:ekadashi_mahatmyam_padma_puranam]{\closesub}