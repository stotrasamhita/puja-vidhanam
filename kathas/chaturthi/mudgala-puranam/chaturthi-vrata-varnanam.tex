% ॥ मुद्गलपुराणं खण्डः ४॥

% ॥ अथ श्रीमुद्गलपुराणे चतुर्थः खण्डः प्रारभ्यते ॥

\sect{४.१ --- प्रथमोऽध्यायः --- चतुर्थीतपोवर्णनम्}

\centerline{॥ श्रीगणेशाय नमः ॥}

\uvacha{शौनक उवाच}

\twolineshloka
{त्वया सूत महाप्राज्ञ महोदरचरित्रकम्}
{कथितं शान्तिदं मोहनाशकं संश्रुतं मया} % १

\twolineshloka
{तथा तृप्तिं न याम्येव पायं पायं सुधामिव}
{नानेन सदृशं किञ्चिच्छान्तिदं विद्यते द्विज} % २

\twolineshloka
{अतो गजाननस्यापि चरित्रं वद विस्तरात्}
{सर्वज्ञस्त्वं महाभाग मतोऽस्माभिर्न संशयः} % ३

\twolineshloka
{मुद्गलस्य च दक्षस्य संवादं वद मानद}
{किं पृष्टं ब्रह्मपुत्रेण दक्षेणाग्रे सुबुद्धिना} % ४

\uvacha{सूत उवाच}

\twolineshloka
{श‍ृणु भार्गव विप्रेन्द्र महाज्ञानकरं प्रभो}
{गजाननस्य माहात्म्यं कथयामि यथा श्रुतम्} % ५

\twolineshloka
{श्रुत्वा महोदराख्यानं नानाख्यानसमन्वितम्}
{मोहनाशकरं दक्षो हर्षितो ह्यभवन् मुने} % ६

\twolineshloka
{मुद्गलं पुनरप्याह योगीन्द्रं वेदपारगम्}
{विनयेन समायुक्तो गणेशज्ञानलालसः} % ७

\uvacha{दक्ष उवाच}

\twolineshloka
{महोदरस्य माहात्म्यं श्रुतं भक्तियुतेन भोः}
{तेनाऽऽनन्दसमायुक्तः कृतोऽहं योगिना त्वया} % ८

\twolineshloka
{अधुना वद विप्रेश गजाननचरित्रकम्}
{कीदृशोऽवतरद्देवः कीदृशं ब्रह्म तत्र च} % ९

\twolineshloka
{किमर्थं देहधारी स बभूव मुनिसत्तम}
{किं कर्मा किं गुणज्ञश्च वद सर्वं महामते} % १०

\twolineshloka
{पूर्वपुण्यप्रभावेण सङ्गतिस्ते प्रजायते}
{धन्योऽहं सर्वभावेन श‍ृणोमि च कथां शुभाम्} % ११

\uvacha{सूत उवाच}

\twolineshloka
{दक्षेणैवं महायोगी स पृष्टो बुद्धिशालिना}
{तं प्रत्युवाच भावज्ञो गाणपत्यपरायणः} % १२

\uvacha{मुद्गल उवाच}

\twolineshloka
{श‍ृणु दक्ष महाभाग धन्योऽस्यत्र न संशयः}
{कथां वर्धयसि प्राज्ञ गाणेशीं योगदां पराम्} % १३

\twolineshloka
{तव भावेन सन्तुष्टो वदामि सकलं प्रभो}
{गजाननस्य माहात्म्यं योगशान्तिपदप्रदम्} % १४

\twolineshloka
{साङ्ख्यं ब्रह्म विदेहाख्यं कथितं योगिभिः परम्}
{तदेव गजवक्त्रं वै जानीहि त्वं प्रजापते} % १५

\twolineshloka
{बिन्दुब्रह्मात्मको देहः सोऽहं वक्त्रं प्रकीर्तितम्}
{तयोरभेदको बोधो बभावेव स देहवान्} % १६

\twolineshloka
{त्रिविधेषु स्थितो देवो तथापि तद्विवर्जितः}
{विदेहो गजवक्त्रश्च शोभत साङ्ख्यधारकः} % १७

\twolineshloka
{बोधत्यागे महाभाग कः सङ्ख्यां कुरुते वद}
{ब्रह्मणां तेन साङ्ख्यं तद्ब्रह्म वेदे प्रकाशितम्} % १८

\twolineshloka
{लोभासुरविनाशाय प्रकटोऽभूद्गजाननः}
{देवैर्विप्रैः प्रजानाथ प्रार्थितो भक्तिलालसः} % १९

\twolineshloka
{चतुर्थ्यां मध्यगे भानौ देहधारी समागतः}
{सा तिथिः परमा तस्य प्रीतिदा सम्बभूव वै} % २०

\uvacha{सूत उवाच}

\twolineshloka
{इति श्रुत्वा वचो रम्यं दक्षो हृष्टमना मुने}
{जगाद मुद्गलं विप्रं चतुर्थीज्ञानसिद्धये} % २१

\uvacha{दक्ष उवाच}

\twolineshloka
{का चतुर्थी तिथिः प्रोक्ता शुक्ला कृष्णा वद प्रभो}
{गणनाथप्रियाऽत्यन्तं बभूव कथमेव सा} % २२

\twolineshloka
{पुनः पुनस्त्वया ब्रह्मन् कथितं तद् व्रतं शुभम्}
{अतो ब्रूहि चतुर्थ्यास्त्वं माहात्म्यं सकलं परम्} % २३

\twolineshloka
{दक्षस्य वचनं श्रुत्वा हर्षयुक्तो महामुनिः}
{मुद्गलस्तमथोवाच चतुर्थीसम्भवां कथाम्} % २४

\uvacha{मुद्गल उवाच}

\twolineshloka
{सङ्क्षेपेण प्रवक्ष्यामि चतुर्थ्याश्च चरित्रकम्}
{गणनाथप्रियं पूर्णं ज्ञास्यसि त्वं प्रजापते} % २५

\twolineshloka
{पुरा सृष्टिं स सृष्ट्वा वै ब्रह्मा लोकपितामहः}
{स्थितानां तत्र जन्तूनां कालार्थं स दधे मनः} % २६

\twolineshloka
{नानाकार्यप्रसिद्ध्यर्थं सञ्चिन्त्य गणपं हृदि}
{अभवद्ध्यानमास्थाय संस्थितश्चिन्तयान्वितः} % २७

\twolineshloka
{ततस्तस्य शरीराद्वै निःसृता प्रकृतिः परा}
{महामाया तिथीनां सा जननी कामरूपिणी} % २८

\twolineshloka
{चतुष्पदां तथा दक्ष चतुर्हस्तां सुशोभनाम्}
{चतुर्मुखयुतां वीक्ष्य हर्षितोऽभूत् प्रजापतिः} % २९

\twolineshloka
{ततः सा तं नमस्कृत्य तुष्टाव जगदीश्वरम्}
{नानास्तोत्रैः प्रसाद्यैनमुवाच घननिःस्वना} % ३०

\uvacha{प्रकृतिरुवाच}

\twolineshloka
{तवाङ्गनिःसृतां मां त्वं विद्धि ब्रह्माण्डनायक}
{आज्ञां कुरु पितर्मां च कुर्वेऽहं भावयन्त्रिता} % ३१

\twolineshloka
{स्थानं देहि तथा भक्ष्यं नानाभोगादिकं प्रभो}
{मह्यं देव दयासिन्धो नमस्ते परमेश्वर} % ३२

\twolineshloka
{तस्यास्तद्वचनं श्रुत्वा तां जगाद प्रजापतिः}
{सृष्टिं गणेशं सञ्चिन्त्य विचित्रां कुरु मानदे} % ३३

\twolineshloka
{स तस्यै गणनाथस्य ददौ मन्त्रं षडक्षरम्}
{सविधिं सा नमस्कृत्य ययौ तं वनमादरात्} % ३४

\twolineshloka
{तताप तप उग्रं सा नासाग्रनयना सती}
{गणेशं हृदि सन्ध्याय जजाप मन्त्रमुत्तमम्} % ३५

\twolineshloka
{प्रजापते गते वर्षसहस्रे तु गजाननः}
{आययौ तां महाभागामगदद्भक्तवत्सलः} % ३६

\uvacha{गणेश उवाच}

\twolineshloka
{वरान् वृणु महाभागे बहून् मनसि वाञ्छितान्}
{निराहारेण सन्तुष्टो ददामि तपसा च ते} % ३७

\twolineshloka
{तस्य तद्वचनं श्रुत्वा हर्षिता प्रणनाम तम्}
{गजाननं प्रसम्पूज्य तुष्टाव च कृताञ्जलिः} % ३८

\uvacha{प्रकृतिरुवाच}

\twolineshloka
{नमस्ते विघ्ननाथाय गणेशाय परात्मने}
{अनाथाय विशेषेण सर्वनाथाय ते नमः} % ३९

\twolineshloka
{नमो मूषकवाहाय मूषकध्वजिने नमः}
{स्वानन्दपतये तुभ्यं गणानां पतये नमः} % ४०

\twolineshloka
{सिद्धिबुद्धिप्रदात्रे च सिद्धिबुद्धिविहारिणे}
{योगेशाय सदा शान्तिप्रदात्रे योगिने नमः} % ४१

\twolineshloka
{सर्वादये सदा सर्वपूज्याय भक्तपालका}
{ज्येष्ठराजाय ज्येष्ठानां पतये ते नमो नमः} % ४२

\twolineshloka
{ब्रह्मणे ब्रह्मदात्रे वै ब्रह्मणां पतये नमः}
{सिद्धेश्वराय देवानां दैत्यानां पतये नमः} % ४३

\twolineshloka
{चतुर्भुजाय हेरम्ब परशोर्धारकाय ते}
{अङ्कुशन्धारिणे तुभ्यं निरङ्कुश नमो नमः} % ४४

\twolineshloka
{रजसा सृष्टिकर्त्रे ते पालने सात्त्विकाय ते}
{तामसाय महाहन्त्रे गुणेशाय नमो नमः} % ४५

\twolineshloka
{स्थावराय चरायैव चराचरमयाय ते}
{चराचरविहीनाय बोधाय च नमो नमः} % ४६

\twolineshloka
{चतुर्विधस्वरूपाय चतुर्विधसुखप्रद}
{चतुःसुखस्वरूपाय स्वसंवेद्याय ते नमः} % ४७

\twolineshloka
{विनायकाय सर्वेषां नायकाय नमो नमः}
{गजाननाय देवाय देवदेवेश ते नमः} % ४८

\twolineshloka
{किं स्तौमि त्वां गणाध्यक्ष ब्रह्मणस्पतिरूपिणम्}
{यं स्तोतुं न समर्थाश्च योगिनो वेदमुख्यकाः} % ४९

\twolineshloka
{तवदर्शनबोधेन तथापि संस्तुतो मया}
{तेन तुष्टो भव स्वामिन् भक्तिं देहि दृढां त्वयि} % ५०

\twolineshloka
{इति स्तुत्वा गणेशानं दण्डवत् प्रणनाम तम्}
{तामुत्थाप्य गणाधीश उवाच व्रतमातरम्} % ५१

\uvacha{गणेश उवाच}

\twolineshloka
{वरं वृणु महाभागे यं यमिच्छसि शोभने}
{तं तं दास्यामि सुप्रीतो भक्त्या स्तोत्रेण तोषितः} % ५२

\twolineshloka
{त्वया कृतमिदं स्तोत्रं सर्वसिद्धिप्रदं भवेत्}
{यः पठेच्छृणुयाद्देवि ईप्सितं लभते नरः} % ५३

\twolineshloka
{एककालं द्विकालं वा त्रिकालं लभते सदा}
{अव्रती व्रतसाफल्यं पठनान्नात्र संशयः} % ५४

\twolineshloka
{ब्रुवन्तं गणनाथं सोवाच गम्भीरनिःस्वना}
{देवी प्रणम्य भावेन साश्रुनेत्रा प्रजापते} % ५५

\uvacha{देव्युवाच}

\twolineshloka
{वरदोऽसि गणाधीश तदा देहि त्वयि स्थिराम्}
{भक्तिं मे च तथा कार्ये सामर्थ्यं करुणानिधे} % ५६

\twolineshloka
{सृष्टिसर्जनसामर्थ्यं देहि नाथ नमोऽस्तु ते}
{सदा तव प्रियत्वं मे वियोगो न च ते भवेत्} % ५७

\twolineshloka
{ओमित्युक्त्वा गणाधीशो जगाद वदतां वरः}
{चतुर्विधप्रदा देवि प्रिया मम भविष्यसि} % ५८

\twolineshloka
{तिथीनां मातृभावेन चतुर्थी संज्ञिता भव}
{वामभागे सदा कृष्णा शुक्ला दक्षिणभागके} % ५९

\twolineshloka
{मम जन्मतिथिस्त्वं वै भविष्यसि न संशयः}
{सदा तव व्रते संस्थान् पालयामि विशेषतः} % ६०

\twolineshloka
{मदीयव्रतजा पुण्या तिथिस्त्वं च भविष्यसि}
{व्रतेषु त्वत्समं नैव दायकं कुत्र वर्तते} % ६१

\twolineshloka
{एवमुक्त्वा गणेशानोऽन्तर्धानं प्रचकार ह}
{व्रतमाता प्रजानाथाऽभवत्तत्रैव संस्थिता} % ६२

॥ॐ तत्सदिति श्रीमदान्त्ये पुराणोपनिषदि श्रीमन्मौद्गले महापुराणे चतुर्थे खण्डे गजाननचरिते चतुर्थीतपोवर्णनं नाम प्रथमोऽध्यायः॥४.१॥


\sect{४.२ --- द्वितीयोऽध्यायः --- शुक्लकृष्णचतुर्थीवरदानवर्णनम्}

\centerline{॥ श्रीगणेशाय नमः ॥}

\uvacha{मुद्गल उवाच}

\twolineshloka
{ततो देवी गणाध्यक्षं स्मृत्वा स्रष्टुं मनो दधे}
{ततोऽकस्माद् द्विधा साऽभूद्वामदक्षिणभागतः} % १

\twolineshloka
{वामभागं कृष्णवर्णं दक्षिणाङ्गं तथा बभौ}
{शुक्लवर्णं महाभाग ततः सा विस्मिताऽभवत्} % २

\twolineshloka
{पुनर्गणपतिं ध्यात्वा स्रष्टुं तत्रोपचक्रमे}
{ततस्तस्या मुखाम्भोजात् प्रतिपन्निःसृता तिथिः} % ३

\twolineshloka
{नासिकायां द्वितीया वै वक्षसश्च तृतीयिका}
{अङ्गुलीभ्यस्तथा दक्ष पञ्चमी चाभवत्तिथिः} % ४

\twolineshloka
{उत्पन्ना हृदयात् षष्ठी चक्षुषोः सप्तमी बभौ}
{अष्टमी बाहुदेशाच्च समुत्पन्ना महातिथिः} % ५

\twolineshloka
{नवमी तिथिरुत्पन्ना उदराच्च प्रजापते}
{दशमी कर्णदेशाद्वै कण्ठाच्चैकादशी मता} % ६

\twolineshloka
{द्वादशी पादयोस्तस्याः समुत्पन्ना तथा विभो}
{स्तनात् त्रयोदशी तस्या अहङ्काराच्चतुर्दशी} % ७

\twolineshloka
{मनसः पूर्णिमा जाता अमावास्या तथाऽभवत्}
{जिह्वायाः सर्वभावेन तिथयो भिन्नतां दधुः} % ८

\twolineshloka
{कृष्णा चतुर्थी प्रोक्ता या तस्याः कृष्णाः प्रकीर्तिताः}
{तिथयोः दशपञ्चाख्या ज्ञातव्या विबुधैः किल} % ९

\twolineshloka
{शुक्लाः शुक्लचतुर्थी देहाच्चतुर्दश वै मताः}
{तिथयो विधिवत् सर्वा बभुः स्वस्वगुणान्विताः} % १०

\twolineshloka
{दिनांशाः पुरुषा जाता रात्रयः स्त्रीस्वरूपकाः}
{स्वस्वभावविहारज्ञा बभूवुश्च हिते रताः} % ११

\twolineshloka
{तिथिभिः सहिता देवी चतुर्थी तपसि स्थिता}
{गणेशमभजन्नित्यं मन्त्रध्यानपरायणा} % १२

\twolineshloka
{वर्षेणैकेन विघ्नेशस्तां ययौ भक्तवत्सलः}
{शुक्लां मध्याह्नसमये कृष्णां चन्द्रोदये तथा} % १३

\twolineshloka
{वरं ब्रूहि गणेशानस्तामुवाच विशेषतः}
{प्रणम्य विघ्नपं पूजयित्वा स्तुत्वा जगाद सा} % १४

\uvacha{चतुर्थ्युवाच}

\twolineshloka
{त्वदेकनिलयां देव मां कुरुष्व गजानन}
{नान्यं याचे वरं स्वामिन् भक्तिमिच्छामि शाश्वतीम्} % १५

\twolineshloka
{तस्यास्तद्वचनं श्रुत्वा तां जगाद गजाननः}
{प्राप्तं चतुर्थि मध्याह्ने मदीयं दर्शनं त्वया} % १६

\twolineshloka
{अतो मध्याह्नकालेऽन्ये मां भजन्ते शिवादयः}
{चतुर्थ्यां शुक्लपक्षस्य स्थितायां सर्वदा जनाः} % १७

\twolineshloka
{मदीयव्रतमुख्यत्वे भव त्वं सर्वभावतः}
{निराहारेण मां तत्र त्वद्युक्तं पर्युपासते} % १८

\twolineshloka
{चतुर्विधं प्रदास्यामि नानाभावनियन्त्रितम्}
{सञ्चितं नास्ति चेद्देवि तथाप्यत्र न संशयः} % १९

\twolineshloka
{सञ्चितं चास्ति चेद्वापि ददासि त्वं महामते}
{अतस्ते नाम विख्यातं वरदेति भविष्यति} % २०

\twolineshloka
{त्वां पूजयन्ति ये नैव मद्युक्तां व्रतभावतः}
{तेषां व्रतानि सर्वाणि निष्फलानि भवन्तु वै} % २१

\twolineshloka
{एवमुक्त्वाऽन्तर्दधेऽसौ गणेशो देवनायकः}
{तदादि सा तिथिः ख्याता वरदा च प्रजापते} % २२

\twolineshloka
{गणेशस्य प्रियाऽत्यन्तमिति जन्मतिथिः स्मृता}
{तस्यामुपोषणं कार्यं नरैरात्महितेप्सुभिः} % २३

\twolineshloka
{पञ्चम्यां पारणं दक्ष कर्तव्यं द्विजसाक्षिकम्}
{चतुर्विधं फलं तैश्च सम्प्राप्तं नात्र संशयः} % २४

\twolineshloka
{यद्यदिच्छति तत्तत् स लभते व्रतकारकः}
{अन्ते स्वानन्दगो भूत्वा सायुज्यं ब्रह्मणस्तथा} % २५

\twolineshloka
{तत्र व्रते नराः पापा अन्नं भक्षन्ति चेत् प्रियम्}
{नारकास्ते भविष्यन्ति हीना अन्नैः पुनर्जनौ} % २६

\twolineshloka
{एवं वरं चतुर्थ्यै स ददौ विघ्नविदारणः}
{सा तिथिः सर्वमान्या च बभूव व्रतमूलगा} % २७

\twolineshloka
{ततः कृष्णां चतुर्थीं स जगाद गणनायकः}
{वरं वृणु महाभागे दास्यामि मनसीप्सितम्} % २८

\twolineshloka
{श्रुत्वा सा तं प्रणम्याऽऽदौ पूजयित्वा गजाननम्}
{स्तुत्वा जगाद वाक्यं वै हर्षेण महता युता} % २९

\uvacha{कृष्णचतुर्थ्युवाच}

\twolineshloka
{यदि प्रसन्नभावेन वरदोऽसि गजानन}
{तदा भक्तिं दृढां देहि त्वदीयां मे महोदर} % ३०

\twolineshloka
{त्वत्प्रियत्वं सदा नाथ वियोगो न भवेच्च मे}
{सर्वमान्यां कुरुष्व त्वदेकनिष्ठस्वभावतः} % ३१

\twolineshloka
{तस्यास्तद्वचनं श्रुत्वा तां जगाद गजाननः}
{सदा मम प्रियाऽत्यन्तं भविष्यसि महातिथे} % ३२

\twolineshloka
{चन्द्रोदये त्वयाऽहं वै प्राप्तस्तेन चतुर्थिके}
{तत् काले च व्रतं मुख्यं त्वद्युक्तं भवतु प्रियम्} % ३३

\twolineshloka
{न वियोगो भवेद्भक्ताः प्रियेऽन्नजलवर्जिताः}
{उपासते च तेषां त्वं सङ्कष्टहरणं कुरु} % ३४

\twolineshloka
{प्रहरान् पञ्च मद्भक्ताः प्रियेऽन्नजलवर्जिताः}
{उपासते च तेषां त्वं सङ्कष्टहरणं कुरु} % ३५

\twolineshloka
{चतुर्विधं स्वसङ्कष्टं जन्म मृत्युश्च कर्मजम्}
{व्रतिनां तत्फलं कर्म विद्यते न कदाचन} % ३६

\twolineshloka
{इत्यादि विविधं देवि विशेषेण चतुर्विधम्}
{सङ्कष्टदं हर त्वं वै मत्प्रसादाच्चतुर्थिके} % ३७

\twolineshloka
{न सङ्कष्टं कदाचिद्वै व्रतकारि नरस्य च}
{भवेन् मे भक्तियुक्तस्य पुनस्त्वद्युक्तसेविनः} % ३८

\twolineshloka
{इह भुक्त्वाऽखिलान् भोगानन्ते स्वानन्दमाप्नुयात्}
{सङ्कष्टहरणी नाम भवेत्तव चतुर्थिके} % ३९

\twolineshloka
{न करोति चतुर्थीं चेन्नरो मद्भक्तिकारकः}
{निष्फलं सकलं तस्य भजनं वै भविष्यति} % ४०

\twolineshloka
{त्वां त्यक्त्वा ये नराः पापा व्रतमन्यच्च कुर्वते}
{व्रतादिकं निष्फलकं तेषां सर्वं भविष्यति} % ४१

\twolineshloka
{वर्णाश्रमस्थको भूत्वा चतुर्थीं न करोति चेत्}
{तस्य सर्वं स्वधर्मस्थं कर्म यन्निष्फलं भवेत्} % ४२

\twolineshloka
{यती रात्रौ निराहारयुक्तः सन् व्रतमाचरेत्}
{सङ्कष्टहरणं देवि स्वानन्दार्थं न संशयः} % ४३

\twolineshloka
{अन्यैश्च द्विजसंयुक्तैर्भोजनं द्विजसाक्षिकम्}
{रात्रौ कर्तव्यमेतस्मिन् मां प्रपूज्य वरानने} % ४४

\twolineshloka
{श्रावणे लड्डुकान् भक्षेत् भाद्रके दधिभोजनम्}
{निर्जलं त्वाश्विने प्रोक्तं दुग्धपानं च कार्तिके} % ४५

\twolineshloka
{मार्गशीर्षे जलाहारः पौषे गोमूत्रभक्षणम्}
{माघे मासि तिलान् शुक्लान् फाल्गुने घृतशर्कराम्} % ४६

\twolineshloka
{पञ्चगव्यं मधौ मासे वैशाखे पद्मबीजकम्}
{ज्येष्ठे घृतं गवां भक्ष्यमाषाढे मधुभोजनम्} % ४७

\twolineshloka
{यतीनां सर्वदा ह्येतत् व्रतं युक्तं प्रकीर्तितम्}
{अन्येषां भोजने नैवानेन वा व्रतकं स्मृतम्} % ४८

\twolineshloka
{याममात्राऽवशिष्टायां रात्रावुत्थाय सत्वरम्}
{प्रातःकृत्यं सुसङ्क्षेपात् कर्तव्यं तन्नरेण वै} % ४९

\twolineshloka
{तथा माध्याह्निकं कर्म कर्तव्यं प्रातरेव च}
{चतुर्घटिकरात्रौ सूर्योदयः स प्रकीर्तितः} % ५०

\twolineshloka
{एवं कर्मादिकं कृत्वा मां सम्पूज्याल्पमन्त्रतः}
{प्रत्यक्षोदयकाले वै सूर्योपस्थानमाचरेत्} % ५१

\twolineshloka
{चन्द्रोदयस्य पर्यन्तं शमीमूले जपं जपेत्}
{मदीयं मौनिभावेन पश्चात् स्नानं समाचरेत्} % ५२

\twolineshloka
{सायाह्नकालजं कर्म तत्र कुर्यान्नरोत्तमः}
{विधिना पूज्य मां पश्चादर्घ्यदानं समाचरेत्} % ५३

\twolineshloka
{आदौ तेऽर्ध्यप्रदानं च ततो मेऽर्घ्यनिवेदनम्}
{ततः सप्तार्घ्यदानं व्रती चन्द्राय समाचरेत्} % ५४

\twolineshloka
{ब्राह्मणैः पूजितैर्देवि नरः कुर्याच्च भोजनम्}
{मोदकापूपलड्डूकपायसादिभिरादरात्} % ५५

\twolineshloka
{रात्रौ जागरणं कुर्यान् मदीयकथयान्वितः}
{पञ्चम्यामुपचारैः सम्पूजयन् मां प्रयत्नतः} % ५६

\twolineshloka
{एवं व्रतं नरः कुर्यात् स वै सर्वार्थसिद्धिभाक्}
{भविष्यत्यमरैर्मान्योऽन्ते ब्रह्मणि लयं व्रजेत्} % ५७

\twolineshloka
{यथाऽहं देवतादीनां पूज्यः सर्वाद्यभावतः}
{तथा त्वं व्रतजातीनामादिपूज्या भविष्यसि} % ५८

\twolineshloka
{शुक्लां कृष्णां चतुर्थीं ये करिष्यन्त्यमरादयः}
{सर्वव्रतादि भावेषु तेषां सिद्धिर्भविष्यति} % ५९

\twolineshloka
{एवमुक्त्वान्तर्दधेऽसौ गणेशो ब्रह्मनायकः}
{चतुर्थी भक्तिसंयुक्ताऽभवत्तत्रैव संस्थिता} % ६०

॥ॐ तत्सदिति श्रीमदान्त्ये पुराणोपनिषदि श्रीमन्मौद्गले महापुराणे चतुर्थे खण्डे गजाननचरिते शुक्लकृष्णचतुर्थीवरदानवर्णनं नाम द्वितीयोऽध्यायः॥४.२॥


\sect{४.३ --- तृतीयोऽध्यायः --- प्रतिपदादिव्रतवर्णनम्}

\centerline{॥ श्रीगणेशाय नमः ॥}

\uvacha{मुद्गल उवाच}

\twolineshloka
{ततो गणपतिर्दक्ष ययौ सर्वतिथीः प्रति}
{उवाच वृणुत प्राज्ञा वरं यं मनसीप्सितम्} % १

\twolineshloka
{श्रुत्वा गणपतेर्वाक्यं प्रणम्य गणनायकम्}
{पूजयित्वा च तं स्तुत्वा जगदुस्तिथयः पराः} % २

\uvacha{प्रतिपदाद्या उचुः}

\twolineshloka
{यदि विघ्नेश्वर स्वामिन् वरदोऽसि गजानन}
{किं कर्तव्यं दयासिन्धो अस्माभिर्वद साम्प्रतम्} % ३

\twolineshloka
{स्वस्वव्यापारजं देवं सामर्थ्यं देहि चाद्भुतम्}
{भक्तिं त्वदीयपादाब्जे सर्वं मान्यं त्वया प्रभो} % ४

\twolineshloka
{तासां वचनमाकर्ण्य जगाद गणनायकः}
{ताः श‍ृणु त्वं प्रजानाथ सर्वेषां हितकारकम्} % ५

\uvacha{गणेश उवाच}

\twolineshloka
{मदीया भक्तिरत्यन्तं भविष्यति न संशयः}
{भवतीनां महाभागास्तिथयः पुण्यभागिकाः} % ६

\twolineshloka
{मत्कलांशसमुद्भूता देवास्तान् प्रीणयन्तु वै}
{भवतीभिर्युता देवा नराणां प्रीतिदाऽस्तु च} % ७

\twolineshloka
{सदा वह्निं प्रतिपदि पूजयिष्यन्ति ये नराः}
{हुतद्रव्यैश्च हुत्वा तं दुग्धाहारा व्रते रताः} % ८

\twolineshloka
{उपोषणं करिष्यन्ति तैश्च देवाः सहेश्वराः}
{तोषितास्ते सदा तेभ्यो दास्यन्ति मनसीप्सितम्} % ९

\twolineshloka
{ते मृताः सङ्गमिष्यन्त्यग्निलोके भोगकारकाः}
{अग्नितुल्यप्रकाशेन विमानेन विहायसा} % १०

\twolineshloka
{चरिष्यन्ति न सन्देहः प्रतिपद्व्रतकारकाः}
{ब्रह्मार्पणस्वभावेन गमिष्यन्ति मदात्मनि} % ११

\twolineshloka
{द्वितीयायां नरो यस्तु पूजयेदश्विनौ शुभौ}
{पत्राहारसमायुक्तो रूपवान् जायते किल} % १२

\twolineshloka
{इह भुक्त्वाऽखिलान् भोगानन्ते तल्लोकमाप्नुयात्}
{द्वितीयाव्रतकारी स देवमान्यो भविष्यति} % १३

\twolineshloka
{ब्रह्मार्पणस्वभावेन द्वितीयाव्रतमाचरेत्}
{स स्वानन्दे मदीये वै लीनश्चैव भविष्यति} % १४

\twolineshloka
{तृतीयायां महाशक्तिं पूजयेद्भक्तिसंयुतः}
{सर्वैः सौभाग्यदैर्द्रव्यैर्लवणाहारवर्जितः} % १५

\twolineshloka
{स शक्तिलोकगो भूत्वा भोगयुक्तो भविष्यति}
{ब्रह्मार्पणतया सोऽपि मल्लोके चागमिष्यति} % १६

\twolineshloka
{पञ्चम्यां नागमुख्यांश्च दुग्धेन स्नापयेन्नरः}
{पूजयेत्तान् प्रयत्नेन निरम्लाहारकारकः} % १७

\twolineshloka
{स नागलोकगो भूत्वा भोगयुक्तश्चरिष्यति}
{निष्कामेन मदीये वै लोकेऽन्ते आगमिष्यति} % १८

\twolineshloka
{षष्ठ्यां स्कन्दं फलाहारः पूजयेद्भक्तिसंयुतः}
{स्कन्दलोके चरेत् सोऽपि महाभोगपरायणः} % १९

\twolineshloka
{निष्कामभावयुक्तश्चेत् स्वानन्देऽन्ते गमिष्यति}
{शुक्लगत्याः क्रमेणैव सर्वभावनियन्त्रितः} % २०

\twolineshloka
{सप्तम्यामर्चयेत् सूर्यमुपोषणपरायणः}
{स सूर्यलोकमाश्रित्य प्रचरेद्भोगसंयुतः} % २१

\twolineshloka
{निष्कामव्रतकारी चेन् महालयसमुत्थिते}
{स्वानन्दे मे सदेहो वै ब्रह्मभूतो भविष्यति} % २२

\twolineshloka
{अष्टम्यां मातृकानां यः पूजको भावसंयुतः}
{बिल्वाहारसमायुक्तो मातृकालोकगो भवेत्} % २३

\twolineshloka
{मदर्पणस्वभावेन सदा व्रतपरायणः}
{अष्टम्यां सोऽपि मल्लोके गच्छेत् क्रमविनिश्चिते} % २४

\twolineshloka
{नवम्यामेव दुर्गायाः पूजनं यः समाचरेत्}
{पिष्टाशी भोगसंयुक्तोऽन्ते तल्लोकमवाप्नुयात्} % २५

\twolineshloka
{ब्रह्मार्पणतया सोऽपि निजलोके गमिष्यति}
{शुक्लगत्या महाभागो ब्रह्मभूतो न संशयः} % २६

\twolineshloka
{दशम्यां व्रतसंस्थो यो दधिभक्षणसंयुतः}
{दिशां दिगीशकानां वै पूजकस्तत्प्रियो भवेत्} % २७

\twolineshloka
{इह भुक्त्वाऽखिलात् भोगानन्ते तल्लोकमाप्नुयात्}
{निष्कामव्रतभावेन ब्रह्मभूतो भविष्यति} % २८

\twolineshloka
{एकादश्यां नरो भक्षेद्वह्निपक्वविवर्जितम्}
{धनपं पूजयेच्चैव भक्तियुक्तेन चेतसा} % २९

\twolineshloka
{स वै तस्य वसेल्लोके नानाभोगकरः सदा}
{ब्रह्मार्पणतया तद्वद्व्रतं कृत्वा सुखी भवेत्} % ३०

\twolineshloka
{विष्णुं सम्पूजयेद्यो वै द्वादश्यां घृतभोजनः}
{स विकुण्ठे वसेन्नित्यं नानाभोगपरायणः} % ३१

\twolineshloka
{ब्रह्मार्पणविधानेन व्रतं कुर्याद्यदा नरः}
{महालये मदीये स लोके ब्रह्ममयो भवेत्} % ३२

\twolineshloka
{धर्मं त्रयोदशीसंस्थं पूजयेत् क्षीरभोजनः}
{स धर्मलोकगो भूत्वा भुञ्जीत विविधं सुखम्} % ३३

\twolineshloka
{ब्रह्मार्पणतया येन साधिता चेत् त्रयोदशी}
{स स्वानन्दे समागम्य ब्रह्मभूतो भविष्यति} % ३४

\twolineshloka
{चतुर्दश्यां शिवं यश्च पूजयेद्भक्तिसंयुतः}
{उपोषणसमायुक्तो गोधूमान्नेन पारणम्} % ३५

\twolineshloka
{करिष्यति च कैलासे वासस्तस्य भविष्यति}
{निष्कामश्चेत्तदन्ते स मल्लोके मन्मयो भवेत्} % ३६

\twolineshloka
{पूर्णिमायां देवगणान् देवांश्चद्रमसं तथा}
{पूजयेद्भक्तिभावेन चन्द्रलोकं स आप्नुयात्} % ३७

\twolineshloka
{इह भुक्त्वाऽखिलान् भोगानन्ते देवान् गमिष्यति}
{निष्कामनो मदीये स लोके ब्रह्ममयो भवेत्} % ३८

\twolineshloka
{उपोषणसमायुक्तोऽर्घ्यदानं यः करिष्यति}
{चन्द्राय निशिभुग् वाऽपि सर्वदेवपरायणः} % ३९

\twolineshloka
{सर्वदेवमयी रम्या पूर्णिमा परिकीर्तिता}
{चान्द्री चैव तथा ज्ञेया तिथिः पुण्यप्रदा मता} % ४०

\twolineshloka
{अमायां तर्पयेद्यो वै पितॄन् पितृपरायणः}
{उपोषणं वा कुर्वीत स सर्वार्थमवाप्नुयात्} % ४१

\twolineshloka
{अन्ते पितृमये लोके वसतिस्तस्य सम्भवेत्}
{आगमिष्यति निष्कामश्चेत् स्वानन्दे स तल्लये} % ४२

\twolineshloka
{कृष्णाः शुक्लाः सदा पूज्यास्तिथयो व्रतकारिणा}
{एवं भावयुतेनैव स सर्वार्थमवाप्नुयात्} % ४३

\twolineshloka
{मत्प्रियास्तिथयः सर्वा भविष्यथ न संशयः}
{मत्कलांशयुतैर्देवैः संयुक्ताः प्रभवेत वै} % ४४

\twolineshloka
{चतुर्थीं ये न कुर्वन्ति तेषां सर्वं निरर्थकम्}
{भवदीयव्रते संस्थं भविष्यति न संशयः} % ४५

\twolineshloka
{अतश्चतुर्थीसंयुक्ताः सदा भवत मानदाः}
{तिथयः कालमानेन सर्वपूज्या भविष्यथ} % ४६

\twolineshloka
{एवमुक्त्वाऽन्तर्दधेऽसौ गणेशो गणवल्लभः}
{तथा जातं प्रजानाथ तिथियुक्तव्रतं सदा} % ४७

\twolineshloka
{चतुर्थीजं च माहात्म्यं तिथीनां यद्विशेषतः}
{श‍ृणुयाद्यः पठेद्वाऽपि स सर्वार्थमवाप्नुयात्} % ४८

॥ॐ तत्सदिति श्रीमदान्त्ये पुराणोपनिषदि श्रीमन्मौद्गले महापुराणे चतुर्थे खण्डे गजाननचरिते प्रतिपदादिव्रतवर्णनं नाम तृतीयोऽध्यायः॥४.३॥


\sect{४.४ --- चतुर्थोऽध्यायः --- दशरथव्रतोपदेशः}

\centerline{॥ श्रीगणेशाय नमः ॥}

\uvacha{मुद्गल उवाच}

\twolineshloka
{अधुना श‍ृणु दक्ष त्वं संवादं तं पुरातनम्}
{दशरथस्य राजर्षेर्वसिष्ठस्य महात्मनः} % १

\twolineshloka
{येन त्वं तिथिमुख्यायाश्चतुर्थ्या व्रतजे परे}
{महिम्नि निपुणोऽत्यन्तं सम्भवेर्ब्रह्मपुत्रक} % २

\twolineshloka
{महायशा दशरथो वन्ध्यदोषभयार्दितः}
{वसिष्ठं शरणं गत्वा नत्वा पप्रच्छ तं मुनिम्} % ३

\uvacha{दशरथ उवाच}

\twolineshloka
{पुत्रप्राप्त्यर्थमेवं भो वदोपायं महामुने}
{शिष्यं मां शाधि भावज्ञ तारयस्व भवार्णवात्} % ४

\twolineshloka
{ततस्तं मुनिशार्दूल उवाच नृपमुत्तमम्}
{कुलदेवं गणेशं ते भजस्व व्रतसंयुतः} % ५

\twolineshloka
{वसिष्ठस्य वचः श्रुत्वा पुनस्तं प्रणनाम ह}
{पप्रच्छ विनयेनैव संयुतः स नृपोत्तमः} % ६

\twolineshloka
{कीदृशं तद्व्रतं विप्र गणेशस्य महात्मनः}
{कृतं केन व्रतं तत्र भवेत् सिद्धिश्च कीदृशी} % ७

\twolineshloka
{कस्मिन् काले व्रतं कार्यं पूजनं कीदृशं मुने}
{वद सर्वं विशेषेण करिष्यामि त्वदाज्ञया} % ८

\uvacha{वसिष्ठ उवाच}

\twolineshloka
{चतुर्थी गणनाथस्य प्रियाऽत्यन्तं नृपोत्तम}
{शुक्ला कृष्णा समाख्याता सर्वसिद्धिप्रदा भवेत्} % ९

\twolineshloka
{चन्द्रोदये यदा प्राप्ता कृष्णा सैव चतुर्थिका}
{सा सङ्कष्टहरा प्रोक्ता चतुर्विधफलप्रदा} % १०

\twolineshloka
{चतुर्विधं जगत् सर्वं जानीहि त्वं नृपात्मज}
{सङ्कष्टं तत् समाख्यातं बन्धनेन विशेषतः} % ११

\twolineshloka
{इह भुक्त्वाऽखिलान् भोगान् सङ्कष्टीकरणान्नृप}
{सर्वसङ्कटहीनः स स्वानन्दे प्रगमिष्यति} % १२

\twolineshloka
{यदा चन्द्रोदये भूप चतुर्थी नैव लभ्यते}
{प्रदोषव्यापिनी ग्राह्या सर्वसङ्कटमुक्तये} % १३

\twolineshloka
{उभये चन्द्रसंयुक्ता प्राप्ता सा व्रतधारिणी}
{तदा स्वेच्छाविहारेण कर्तव्या चन्द्रव्यापिनी} % १४

\twolineshloka
{तृतीयायां प्रदोषाच्च चन्द्रस्योदयगा भवेत्}
{तदा सैव सदा ग्राह्या परां त्यक्त्वा विशेषतः} % १५

\twolineshloka
{शुक्ला सा वरदा प्रोक्ता चतुर्णां नात्र संशयः}
{सञ्चिताऽसञ्चितानां वै व्रतकारिजनाय च} % १६

\twolineshloka
{धर्मार्थकाममोक्षाद्याश्चत्वारो ये पदार्थकाः}
{तान् ददाति गणेशस्य प्रीतिदा सा चतुर्थिका} % १७

\twolineshloka
{मध्याह्नव्यापिनी ग्राह्या सदा शुक्लचतुर्थिका}
{उभयत्र यदा नाप्ता पूर्वविद्धा मता तिथिः} % १८

\twolineshloka
{उभयत्र यदा प्राप्ता भवेत् प्रहरभानुगा}
{तृतीया चेत्तदा साऽपि पूर्वविद्धा मता तिथिः} % १९

\twolineshloka
{प्रहरादूर्ध्वगा भूप तृतीया सूर्यसंयुता}
{तदा पूर्वा परा वाऽपि कार्या सैव चतुर्थिका} % २०

\twolineshloka
{चतुर्थीव्रतहीनस्य फलहीनानि भूमिप}
{व्रतानि सर्वकालेष्वाद्या कार्या सा ततः प्रभो} % २१

\twolineshloka
{एतद्व्रतं महाभाग ब्रह्मणाऽऽचरितं पुरा}
{व्रतस्यैव प्रभावेण निर्ममे सकलं जगत्} % २२

\twolineshloka
{विष्णुना शङ्करेणैव कृतं व्रतमनुत्तमम्}
{पालने हरणे देवौ समर्थौ तौ बभूवतुः} % २३

\twolineshloka
{योगिभिः शुकमुख्यैश्च कश्यपादिमुनीश्वरैः}
{मुनिभिश्च कृतं पूर्वं स्वस्वकार्यार्थसिद्धये} % २४

\twolineshloka
{इह भुक्त्वाऽखिलान् भोगानन्ते स्वानन्दगा बभुः}
{नानेन सदृशं विद्धि चतुर्णां साधकं नृप} % २५

\twolineshloka
{अन्यैर्नानाविधैरेतत् कृतं वर्णाश्रमस्थितैः}
{ते सर्वे ब्रह्मभूता वै ययुः स्वानन्दके पुरे} % २६

\twolineshloka
{एतद् व्रतं महीपाल कुरु त्वं भावसंयुतः}
{भविताऽसि सपुत्रश्च तथा स्वानन्दगो भवेः} % २७

॥ॐ तत्सदिति श्रीमदान्त्ये पुराणोपनिषदि श्रीमन्मौद्गले महापुराणे चतुर्थे खण्डे गजाननचरिते दशरथव्रतोपदेशो नाम चतुर्थोऽध्यायः॥४.४॥


\sect{४.५ --- पञ्चमोऽध्यायः --- चतुर्थीविवेकवर्णनम्}

\centerline{॥ श्रीगणेशाय नमः ॥}

\uvacha{दशरथ उवाच}

\twolineshloka
{ब्रह्मादिभिः कथं विप्र चतुर्थीगं व्रतं कृतम्}
{गणनाथप्रियं पूर्णं वद सर्वं प्रविस्तरात्} % १

\uvacha{दक्ष उवाच}

\twolineshloka
{वद मुद्गल माहात्म्यं चतुर्थ्यास्त्वं प्रविस्तरात्}
{श्रुत्वा कथां महारम्यां न तृप्यामि मुनीश्वर} % २

\twolineshloka
{दक्षस्य भक्तिसंयुक्तं वाक्यं श्रुत्वा महामुनिः}
{जगाद तं विशेषज्ञो मुद्गलः श‍ृणु शौनक} % ३

\twolineshloka
{वचनं नृपवर्यस्य श्रुत्वा दशरथस्य च}
{जगाद तं मुनिश्रेष्ठो वसिष्ठः सर्ववित् प्रभुः} % ४

\uvacha{वसिष्ठ उवाच}

\twolineshloka
{कदाचित् प्रलये वृत्ते नष्टं स्थावरजङ्गमम्}
{पञ्चभूतमयं सर्वं व्यवस्थासंयुतं पुरा} % ५

\twolineshloka
{शून्यवत् सर्वभावैस्तद्धीनं ब्रह्ममयं बभौ}
{योगनिद्रा गणेशस्य सैव वेदे प्रकथ्यते} % ६

\twolineshloka
{पुनः स कस्मिंश्चित्काले वेदवाग्भिः प्रबोधितः}
{गणेशो निर्ममे सर्वं तत्त्वैस्तत्त्वप्रधारकः} % ७

\twolineshloka
{त्रिगुणेभ्यः समुत्पन्ना देवाः पञ्च नृपेश्वर}
{ब्रह्मा विष्णुः शिवः शक्तिः सूर्यः सर्वधरा बभुः} % ८

\twolineshloka
{तैस्तपश्चरितं घोरमेकाक्षरविधानतः}
{तपसा गणनाथस्तु प्रसन्नः प्रबभूव ह} % ९

\twolineshloka
{हृदये दर्शयामास स्वात्मानं सर्वदं परम्}
{तेन ज्ञानयुता जाता देवाः पञ्च नरेश्वर} % १०

\twolineshloka
{तैश्च सम्प्रार्थितो योगी भृशुण्डी सर्वपारगः}
{तेन सम्प्रेषिता देवा आययुर्नगरे निजे} % ११

\twolineshloka
{कर्मभूमौ गणेशस्य क्षेत्रं स्वानन्दवाचकम्}
{वेदैः सन्दर्शितं स्थापयामासुस्तत्र विघ्नपम्} % १२

\twolineshloka
{यस्मिन् काले च देवेशैः स्थापिता मूर्तिरादरात्}
{तस्मिन् काले चतुर्थी सा भाद्री शुक्ला बभूव ह} % १३

\twolineshloka
{मध्याह्नसमये संस्था सा मुख्या प्रबभूव ह}
{तत आराधितस्तैः स गणराजो विशेषतः} % १४

\twolineshloka
{माघमासे सिते पक्षे चतुर्थ्यां राजसत्तम}
{प्रकटोऽभूत् पुरस्तेषां वरदाता गजाननः} % १५

\twolineshloka
{वरप्रसादेन तस्य निर्मितं सकलं जगत्}
{देवेशैः सा तिथिर्मुख्या बभूव वरदायिनी} % १६

\twolineshloka
{ब्रह्मादिभिश्च सर्वेभ्य उपदिष्टा चतुर्थिका}
{आदौ काले तथा तत्र ज्येष्ठी मुख्या बभूव सा} % १७

\twolineshloka
{ज्येष्ठशुद्धचतुर्थ्यां तद्व्रतदानं कृतं पुरा}
{तस्मिन् काले धृता सर्वैः सद्यः सा वरदायिनी} % १८

\twolineshloka
{अतो मुख्यत्वमापन्ना ज्येष्ठी माघी च भाद्रिका}
{चतुर्थी नृपशार्दूल शुक्लपक्षेषु सर्वदा} % १९

\twolineshloka
{तत्र भाद्रपदे मासे चतुर्थी या महामते}
{शुक्ला साऽत्यन्तभावेन मुख्या जाताऽद्य भावतः} % २०

\twolineshloka
{चतुर्थी समभावेन न किञ्चिद्वर्तते व्रतम्}
{अतः किं वर्णनीयं वै तदौपम्यं तदेव च} % २१

\twolineshloka
{भाद्रे चतुर्थी या शुक्ला साधिता सा नरेण चेत्}
{स द्वादशचतुर्थीसम्भवं पुण्यं समाप्नुयात्} % २२

\twolineshloka
{अङ्गारकयुता भूप सा नरेण कृता भवेत्}
{चतुर्थी षोडशभवं पुण्यं प्राप्तं न संशयः} % २३

\twolineshloka
{नरेण धीमता सा चेद्यदि चन्द्रयुता कृता}
{चतुर्थी विंशतिभवं फलं प्राप्तं महामते} % २४

\twolineshloka
{यदि स्वातियुता भूप चन्द्रवारेण चेत्कृता}
{प्राप्तं चतुर्विंशतिचतुर्थीजं किल पुण्यकम्} % २५

\twolineshloka
{माघे शुक्लचतुर्थी सा कृता चेन्नात्र संशयः}
{अष्टानां च चतुर्थीनां फलं प्राप्तं नरेण तत्} % २६

\twolineshloka
{अङ्गारकयुता चेद्वै साधिता सा नरेण च}
{द्वादशानां चतुर्थीनां पुण्यं प्राप्तं व्रतोद्भवम्} % २७

\twolineshloka
{ज्येष्ठे शुक्लचतुर्थीं यः करोति व्रतधारकः}
{चतुश्चतुर्थीजं पुण्यं लभते स नराधिप} % २८

\twolineshloka
{अङ्गारकयुता चेद् द्वादशभिः समतां व्रजेत्}
{नरेण सा प्रयत्नेन कर्तव्या सर्वदायिका} % २९

\twolineshloka
{अन्येषु श्रावणाद्येषु चतुर्थी वरदा कृता}
{धर्मार्थकाममोक्षाणां दात्री शास्त्रेषु सम्मतम्} % ३०

\twolineshloka
{तत्र श्रावणमासे या चतुर्थी वरदा मता}
{द्विचतुर्थीभवं पुण्यं ददाति व्रतकारिणे} % ३१

\twolineshloka
{चतुर्थ्या महिमाऽनन्तोऽशक्यो वर्णयितुं भवेत्}
{चतुःपदप्रदा प्रोक्ता सा कथं वर्ण्यते मया} % ३२

\twolineshloka
{अधुना श‍ृणु राजेन्द्र कृष्णपक्षे समागताम्}
{चतुर्थीं सर्वदां पुण्यां सङ्कष्टहरिणीं शुभाम्} % ३३

\twolineshloka
{चतुर्विधं च सङ्कष्टं प्राप्तं सर्वैर्नराधिप}
{जन्ममृत्युव्यवस्थाख्यं चतुर्थं कर्मजं फलम्} % ३४

\twolineshloka
{इत्यादिबहवो भेदाः कथनं नैव शक्यते}
{चतुर्विधं जगत् सर्वं पञ्चमं ब्रह्म उच्यते} % ३५

\twolineshloka
{चतुर्विधं सुसङ्कष्टं व्रतमात्रेण हन्ति या}
{सङ्कष्टहरिणी प्रोक्ता चतुर्थी कृष्णपक्षगा} % ३६

\twolineshloka
{तस्यापि महिमानं कः क्षमो वर्णयितुं भवेत्}
{सर्वसिद्धिप्रदायास्तु व्रताचरणमात्रतः} % ३७

\twolineshloka
{माघमासे कृताऽऽदौ च ब्रह्मणा विश्वयोनिना}
{अतः सा मुख्यतां प्राप्ताऽष्टचतुर्थीव्रतैः समा} % ३८

\twolineshloka
{अङ्गारकयुता तत्र विंशत्या समतां व्रजेत्}
{साधिता व्रतभावेन चेन्नरेण नराधिप} % ३९

\twolineshloka
{श्रावणे या समाख्याता चतुर्थी कृष्णपक्षगा}
{चतुश्चतुर्थीतुल्या सा भवत्यत्र न संशयः} % ४०

\twolineshloka
{ब्रह्मणा चोपदिष्टा वै श्रावणे सा ततोऽधिका}
{आदौ देवमुनीन्द्रेभ्यः सर्वसङ्कष्टहारिणी} % ४१

\twolineshloka
{भाद्रमासे नराधीश चतुर्थी कृष्णपक्षगा}
{चतुश्चतुर्थीतुल्या सा श‍ृणु कारणमत्र वै} % ४२

\twolineshloka
{देवानां गणनाथश्च वरदः सम्बभूव ह}
{व्रतमात्रेण तस्मात् सा मुख्या जाता न संशयः} % ४३

\twolineshloka
{अङ्गारकयुता कुत्र सम्प्राप्ता चेत् सुसिद्धिदा}
{द्वादशव्रतजं पुण्यं ददाति व्रतकारिणे} % ४४

\twolineshloka
{शुक्ला कृष्णा सदा कार्या चतुःपदसुसिद्धये}
{अन्यथा सिद्धिहीनानि प्रभवन्ति व्रतानि च} % ४५

॥ॐ तत्सदिति श्रीमदान्त्ये पुराणोपनिषदि श्रीमन्मौद्गले महापुराणे चतुर्थे खण्डे गजाननचरिते चतुर्थीविवेकवर्णनं नाम पञ्चमोऽध्यायः॥४.५॥


\sect{४.६ --- षष्ठोऽध्यायः --- चन्द्रदर्शनदोषहरणचरितवर्णनम्}

\centerline{॥ श्रीगणेशाय नमः ॥}

\uvacha{दशरथ उवाच}

\twolineshloka
{अङ्गारकयुता स्वामिन् कथं श्रेष्ठा बभूव सा}
{तन्मे ब्रूहि विशेषेण संशयोऽत्र महांश्च मे} % १

\uvacha{वसिष्ठ उवाच}

\twolineshloka
{भरद्वाजात् समुत्पन्नः पृथ्व्यां भौमो महामतिः}
{गणानां त्वेतिमन्त्रेण पूजयामास विघ्नपम्} % २

\twolineshloka
{तताप स तपो रम्यं ध्यात्वा देवं गजाननम्}
{पारिनेराच्च नगरात् पश्चिमे वनसंस्थितः} % ३

\twolineshloka
{शतवर्षैर्गणाध्यक्षः प्रसन्नस्तं ययौ नृप}
{वरं ब्रूहि महाभाग भूमिपुत्रेति सोऽब्रवीत्} % ४

\twolineshloka
{ततो भौमो गणाधीशं पूजयन् भक्तिभावतः}
{स्तुत्वा तं स जगादेति वरं सर्वसुखप्रदम्} % ५

\twolineshloka
{नरदेहेन युक्तोऽहं तथापि गणनायक}
{इच्छामि चामृतं पातुं ग्रहं मां कुरु मानद} % ६

\twolineshloka
{तव भक्त्येकनिलयं नाम्ना मङ्गलसंज्ञकम्}
{ऋणहर्तारमेवं मां प्रभो कुरु धनप्रदम्} % ७

\twolineshloka
{माघकृष्णचतुर्थ्यां यद्दर्शनं ते मया कृतम्}
{मया युक्ता ततो देवी महापुण्यप्रदाऽस्तु सा} % ८

\twolineshloka
{तथेति गणनाथेन कृतं सर्वं नराधिप}
{अतोऽन्गारयुता पुण्या चतुर्थी मुख्यतां गता} % ९

\twolineshloka
{अधुना चेतिहासं ते कथयामि विशेषतः}
{भाद्रशुक्लचतुर्थ्यां वै न द्रष्टव्यो निशाकरः} % १०

\twolineshloka
{तत्र ते कारणं सर्वं कथयामि सविस्तरम्}
{येन दोषविहीनस्त्वं गणेशं भजसे नृप} % ११

\twolineshloka
{एकदा देवसंयुक्तैर्मुनिभिश्चैव निश्चितम्}
{गणेशांशसमुद्भूताः पञ्च देवा न संशयः} % १२

\twolineshloka
{ते गणेशस्वरूपाद्वै कदा भिन्ना भवन्ति न}
{तेषां स्मरणमात्रेण सन्तुष्टो गणनायकः} % १३

\twolineshloka
{भविष्यति न सन्देहस्तस्मादादौ मुनीश्वराः}
{शिवादीन् पूजयत च तेन सिद्धिमवाप्स्यथ} % १४

\twolineshloka
{एवं ज्ञानबलेनाज्ञैः कृतं देवैः सहर्षिभिः}
{पञ्चभिर्विष्णुमुख्यैस्तत्तथेति प्रतिपादितम्} % १५

\twolineshloka
{ततः सर्वे मुनिगणाश्चक्रुस्ते तादृशीं क्रियाम्}
{तया सिद्धिविहीनास्ते भ्रान्ताः सर्वे बभूविरे} % १६

\twolineshloka
{समागता देवगणैः ब्रह्माणं तुष्टुवुर्नृप}
{वृत्तान्तं कथयामासुः सोऽपि चिन्तातुरोऽभवत्} % १७

\twolineshloka
{सस्मार शिवमुख्यांश्च ब्रह्माणं ते समाययुः}
{विष्णुः शिवोऽर्यमा शक्तिस्तेऽपि श्रुत्वा सुविस्मिताः} % १८

\twolineshloka
{एतस्मिन्नन्तरे तत्र शब्दः सर्वभयङ्करः}
{बभूव तीव्रघोषेण युक्तः प्रलयसूचकः} % १९

\twolineshloka
{शब्दं श्रुत्वा भयोद्विग्ना बभूवुः शम्भुमुख्यकाः}
{ततोऽकस्मान् महाभीमः पुरुषः प्रकटोऽभवत्} % २०

\twolineshloka
{विकरालं च तं दृष्ट्वा मुमूर्च्छुः शङ्करादयः}
{मुनयो देवताः केऽपि पलायन्त च मूर्च्छिताः} % २१

\twolineshloka
{मुहूर्तमात्रकाले वै गते ब्रह्मादयः सुराः}
{सावधाना बभूवुस्तं ददृशुः पुरतः स्थितम्} % २२

\twolineshloka
{ततोऽतिभयसंयुक्तास्तुष्टुवुर्गणनायकम्}
{ध्यात्वा हृदि महद्रूपं गजवक्त्रादिचिह्नितम्} % २३

\uvacha{ब्रह्माद्या ऊचुः}

\twolineshloka
{नमस्ते गणनाथाय विघ्नानां पतये नमः}
{अनाथानां सुनाथाय नमो विघ्ननिवारण} % २४

\twolineshloka
{भक्तेभ्यः सर्वदात्रे ते निराकाराय साक्षिणे}
{अमेयायाप्रतर्क्याय हेरम्बाय नमो नमः} % २५

\twolineshloka
{गजाननाय देवाय शूर्पकर्णाय ते नमः}
{महोदराय सर्वेषामादिपूज्याय वै नमः} % २६

\twolineshloka
{सर्वादये महादात्रे सर्वपूज्याय वै नमः}
{सर्वभावस्थितायैव ढुण्ढिराजाय ते नमः} % २७

\twolineshloka
{स्वानन्दवासिने तुभ्यं योगशान्तिमयाय च}
{योगिभ्यो योगदात्रे वै योगानां पतये नमः} % २८

\twolineshloka
{सृष्टिकर्त्रे च पात्रे ते सृष्टिहर्त्रे नमो नमः}
{गणेशाय गुणानां वै चालकाय नमो नमः} % २९

\twolineshloka
{आत्मनेऽनात्मने चैव कारणानां प्रकाशिने}
{ब्रह्मेशाय सदानाथ देवपालकरूपिणे} % ३०

\twolineshloka
{निवारय महाविघ्नं सहसा समुपस्थितम्}
{ग्रसिष्यति न चेदण्डमस्मान् वै नात्र संशयः} % ३१

\twolineshloka
{त्वदीयपादपद्मस्य वयं दासा गजानन}
{मरिष्यामो यदा नाथ यशस्तव गतं तदा} % ३२

\twolineshloka
{अधुना रक्ष देवेश न वयं द्रष्टुमुग्रकम्}
{क्षमा गच्छन्ति नः प्राणा रक्ष विघ्नेश रक्ष भोः} % ३३

\twolineshloka
{एवं संस्तुवतां तेषां पुरः सोऽपि महापुमान्}
{गणेशाकाररूपेण बभूव नृपसत्तम} % ३४

\twolineshloka
{ततस्ते तं प्रणेमुश्च पूजयामासुरादरात्}
{मानसीं स्वसुतां तस्मै ददुः सर्वे शिवादयः} % ३५

\twolineshloka
{सरस्वतीं ददौ ब्रह्मा विष्णुः पुष्टिं च मानसीम्}
{योगिनीं शङ्करश्चैव मोहिनीं जगदम्बिका} % ३६

\twolineshloka
{सञ्जीवनीं ददौ तस्मै भानुः सर्वप्रियङ्करः}
{एवं सम्पूज्य विघ्नेशं प्रणतास्ते महीपते} % ३७

\twolineshloka
{उवाच तान् स उत्थाप्य विघ्नेशो भक्तिभावतः}
{विष्णुमुख्यान महेशांश्च मेघगम्भीरनिःस्वनः} % ३८

\uvacha{श्रीगणेश उवाच}

\twolineshloka
{शम्भुमुख्याः श‍ृणुत च देवेन्द्रा मुनिमुख्यकाः}
{मदीयांशा भवन्तोऽपि गर्वं मा कुरुत प्रियाः} % ३९

\twolineshloka
{चराचरं मदीयांशं मद्रहितं न किञ्चन}
{तेभ्यः सिद्धिप्रदानार्थं देहधारी भवाम्यहम्} % ४०

\twolineshloka
{यदि ज्ञानेन मां त्यक्त्वा पूजयेत् तृणरूपकम्}
{तेनाऽहं किं महादेवाः पूजितो वदत प्रियाः} % ४१

\twolineshloka
{तत्रांशो वर्तते किञ्चिन् मदीयस्तत् समा च मे}
{तृप्तिर्जाता न सन्देहः पूर्णस्तृप्तो भवामि न} % ४२

\twolineshloka
{भवन्तोऽपि तथा विप्रैः कलांशाः पूजिता यदि}
{कलांशरूपा मे तृप्तिस्तेन जाता गुणात्मिका} % ४३

\twolineshloka
{ब्रह्माण्डे पूजिते पूर्णे न तृप्तोऽहं भवामि च}
{अनन्तानि मदीयेषु ब्रह्माण्डानि स्वलोमसु} % ४४

\twolineshloka
{गजवक्त्रादिचिह्नेन चिह्नितोऽहं महेश्वरः}
{जगद्ब्रह्ममयस्तत्र तिष्ठामि योगशान्तिदः} % ४५

\twolineshloka
{तस्य पूजनमात्रेण जगद्ब्रह्म सुपूजितम्}
{अहं तयोश्च योगे वै पूर्णः सम्पूजितो मतः} % ४६

\twolineshloka
{मां त्यक्त्वा ज्ञानमोहेन मोहिता मुनिदेवताः}
{भवतां पूजनं चक्रुः सर्वादौ विघ्नसंयुताः} % ४७

\twolineshloka
{तत्रापि मदसंयुक्ता भवन्तस्तादृशात्मकाः}
{सिद्धिहीनास्ततो जाताः पुनर्मा कुरुत त्विदम्} % ४८

\twolineshloka
{पूर्वस्मिंश्च भवद्भिर्वै सेवितो भक्तिसंयुतैः}
{अतःसहामि देवेशा दत्तं वो दर्शनं मया} % ४९

\twolineshloka
{कलांशमात्रेण यदि सन्तुष्टोऽहं भवामि चेत्}
{तदा देहधरः कस्मात् स्थास्यामि वदत प्रियाः} % ५०

\twolineshloka
{एवमुक्त्वा गणाधीशो विरराम महीपते}
{तेऽपि प्रणम्य विघ्नेशं स्वदोषं प्रक्षमापयन्} % ५१

\twolineshloka
{एतस्मिन्नन्तरे तत्र चन्द्रः शङ्करभालगः}
{जहास विवृतैर्दन्तैर्गणेशं रूपगर्वितः} % ५२

\twolineshloka
{उवाच मोहितोऽत्यन्तं प्रणम्य गणनायकम्}
{गजवक्त्रयुतं रूपं किं धृतं विकटं प्रभो} % ५३

\twolineshloka
{पुनः पुनरुवाचेदं वाक्यं हास्यपरायणः}
{ततस्तं कुपितो देवः शशाप शशलाञ्छनम्} % ५४

\twolineshloka
{ये त्वां विलोकयन्ते वै चन्द्र ते पापिनः सदा}
{भवन्तु विघ्नसंयुक्ता मद्वाक्यान्नात्र संशयः} % ५५

\twolineshloka
{एवमुक्त्वान्तर्दधेऽसौ गणेशो ब्रह्मनायकः}
{भार्याभिः सहितो भूप चन्द्रश्च मलिनोऽभवत्} % ५६

\twolineshloka
{अहङ्कारविहीनास्ते जाताः शम्भ्वादयः सुराः}
{तदारभ्य महाभाग गाणपत्या विशेषतः} % ५७

\twolineshloka
{वेदज्ञाऽहङ्कृतिः पूर्णा ब्रह्मणो हृदि संस्थिता}
{तया मदयुतो ब्रह्मा भवत्येवं पुनः पुनः} % ५८

\twolineshloka
{सरस्वती सा दत्ता वै गणेशाय महात्मने}
{तया युक्तः सदा रेमे ब्रह्मणो हृदि विघ्नपः} % ५९

\twolineshloka
{हृदि स्थिते गणेशे स कथं मोहयुतो भवेत्}
{सदा तं भजते भक्त्या गाणपत्यस्वभावतः} % ६०

\twolineshloka
{सर्वपोषणता तद्वद्विष्णोर्हृदि समास्थिता}
{तत् सामर्थ्यविमोहेन गर्वं गच्छति केशवः} % ६१

\twolineshloka
{तां ददौ विघ्नराजाय पुष्टिं विश्वस्य पोषिणीम्}
{तया युक्तो गणेशानो रेमे विष्णोः सदा हृदि} % ६२

\twolineshloka
{सोऽपि तं भजते नित्यं गाणपत्यविचारवित्}
{हृदिस्थे स कथं मोहं प्राप्नोति नृप विघ्नपे} % ६३

\twolineshloka
{योगेन मोहहीनत्वं सदा शम्भोर्हृदि प्रभो}
{तेनाहङ्कारभावेन मन्यते किं न शङ्करः} % ६४

\twolineshloka
{तामेव योगिनीं देवीं हृदिस्थां प्रददौ शिवः}
{गणेशस्तु तया युक्तो रेमे शम्भोर्हृदि स्थितः} % ६५

\twolineshloka
{गणेशे हृदि संस्थे स मोहं नैव प्रपद्यते}
{सदा गाणेशको भूत्वा भजते गणनायकम्} % ६६

\twolineshloka
{वृष्ट्या दानप्रदाने वै करोति स दिवाकरः}
{सर्वेषां जीवनं तेन धृतं कर्ममयं नृप} % ६७

\twolineshloka
{तदेव हृदये तस्य वर्ततेऽहंस्वभावतः}
{तां सञ्जीवनिकां सूर्यो ददौ विघ्नेश्वराय च} % ६८

\twolineshloka
{तया युक्तो गणाधीशः सदा हृदि स खेलति}
{भानोस्तेन रविस्तं स भजतेऽनन्यचेतसा} % ६९

\twolineshloka
{शक्तिर्हृदयगां देवीं मोहिनीं सर्वमोहिनीम्}
{ददौ विघ्नेश्वरायैव शक्तिभावसमन्विताम्} % ७०

\twolineshloka
{तया युक्तो गणेशानस्तस्या हृदि विशेषतः}
{तेन खेलति सा देवी भजते गणनायकम्} % ७१

\twolineshloka
{इत्यहङ्कारनिर्मुक्ता देवेन्द्रास्ते बभूविरे}
{गाणपत्यप्रिया जाता भजन्ते विघ्ननायकम्} % ७२

\twolineshloka
{चित्ते चिन्तामणिः स्थित्वा प्रकृत्या खेलति प्रभुः}
{मिथ्याहङ्कारभावेन बन्धनं प्रतिपद्यते} % ७३

\twolineshloka
{कर्ता कारयिता देवो गणेशो नात्र संशयः}
{हृदिस्थं तं समर्प्यैव स्वात्मानं सुखिनोऽभवन्} % ७४

\uvacha{मुद्गल उवाच}

\twolineshloka
{श्रुत्वा दशरथो राजा जगाद मुनिसत्तमम्}
{चन्द्रस्य का गतिर्जाता तां मे वद महामुने} % ७५

\uvacha{वसिष्ठ उवाच}

\twolineshloka
{गणेशशापसंयुक्तश्चन्द्रः स स्वगृहे गतः}
{एकान्ते दुःखसंयुक्तो विचारमकरोद्धृदि} % ७६

\twolineshloka
{सम्प्रज्ञातस्वरूपश्च देहो गणपतेः स्मृतः}
{असम्प्रज्ञातरूपं वै शिरस्तस्य महात्मनः} % ७७

\twolineshloka
{तयोर्योगे गणेशोऽयं भक्तानुग्रहकारणात्}
{देहधारी स्वयं साक्षाद्बभूवात्र न संशयः} % ७८

\twolineshloka
{वेदेषु कथितं ते यद्रूपं परमयोगदम्}
{तन्न ज्ञातं मया देव मोहेनैव गजानन} % ७९

\twolineshloka
{अपराधं दयासिन्धो क्षम त्वं चित्तगं प्रभो}
{एवं क्षमाप्य विघ्नेशं ध्यात्वा लीनो बभूव ह} % ८०

\twolineshloka
{गङ्गातीरे समागम्य तताप तप उत्तमम्}
{अष्टादशाक्षरेणैव तोषयामास विघ्नपम्} % ८१

\twolineshloka
{चन्द्रे लीने जगत् सर्वं दुःखयुक्तं बभूव ह}
{सूर्यस्य तेजसा दग्धं रसहीनं चराचरम्} % ८२

\twolineshloka
{ततो देवगणैः सर्वैर्भयभीतैः समन्ततः}
{शम्भुविष्णुमुखास्तत्र तपस्तेपुः सहेश्वराः} % ८३

\twolineshloka
{षडक्षरविधानेन ध्यात्वा हृदि गजाननम्}
{भक्तियुक्ता नृपश्रेष्ठ तोषयामासुरादरात्} % ८४

\twolineshloka
{वर्षाणां शतके तत्र गते चन्द्रं गणाधिपः}
{आययौ वरदानार्थं तपो बद्धो नृपोत्तम} % ८५

\twolineshloka
{वर्षेणैकेन देवेशान् ययौ विघ्नाधिपो विभुः}
{समकाले च देवेशाश्चन्द्रस्तं तुष्टुवुर्विभुम्} % ८६

\twolineshloka
{अथर्वशिरसा ढुण्ढिं तोषयामासुरादरात्}
{सन्तुष्टस्तानुवाचेदं गणेशो ब्रह्मनायकः} % ८७

\uvacha{श्रीगणेश उवाच}

\twolineshloka
{वरं वृणुत देवेशाः सन्तुष्टोऽहं ददामि तम्}
{यं यमिच्छथ तं तं वै सफलं वः करोम्यहम्} % ८८

\twolineshloka
{गणेशवचनं श्रुत्वा देवास्तं प्रणिपत्य ते}
{कृत्वा करपुटं सर्वे जगुः परमहर्षिताः} % ८९

\uvacha{देवा ऊचुः}

\twolineshloka
{यदि प्रसन्नभावेन वरदोऽसि गजानन}
{तदा चन्द्रं सुनिर्दोषं कुरु नाथ नमोऽस्तु ते} % ९०

\twolineshloka
{स तानुवाच सर्वात्मा भक्त्या सन्तोषितः प्रभुः}
{भाद्रशुक्लचतुर्थ्यां वै न द्रष्टव्यः सुधाकरः} % ९१

\twolineshloka
{हास्यं कृतं मदीयं वै तद् दिने तेन देवपाः}
{दोषयुक्तो विधुस्तस्मान्नान्यथा मे वचो भवेत्} % ९२

\twolineshloka
{ततोऽतिहर्षिता देवास्तथेति प्रतिपाद्य ते}
{ययुश्चन्द्रं तु सन्द्रष्टुं गणेशेऽन्तर्हिते नृप} % ९३

\twolineshloka
{चन्द्रो गणपतिं दृष्ट्वा ननाम दण्डवत् क्षितौ}
{पूजयामास तं देवं तुष्टाव स कृताञ्जलिः} % ९४

\uvacha{चन्द्र उवाच}

\twolineshloka
{नमस्ते विघ्नपालाय गणेशाय परात्मने}
{ब्रह्मेशाय स्वभक्तेभ्यो ब्रह्मभूयप्रदाय ते} % ९५

\twolineshloka
{अनामयाय सर्वादिपूज्याय तु नमो नमः}
{शिवात्मजाय देवाय विष्णुपुत्राय ते नमः} % ९६

\twolineshloka
{ब्रह्मपुत्राय सूर्यस्य पुत्र ते विघ्नहारिणे}
{शक्तिपुत्राय शेषस्य पुत्राय च नमो नमः} % ९७

\twolineshloka
{सर्वपुत्राय सर्वेषां मात्रे पित्रे नमो नमः}
{सर्वेशाय परेशाय परात्परतराय ते} % ९८

\twolineshloka
{सिद्धिबुद्धिपते तुभ्यं सिद्धिबुद्धिप्रचालक}
{हेरम्बाय महेशानां महेशाय नमो नमः} % ९९

\twolineshloka
{स्रष्टे पात्रे च संहर्त्रे परमात्मस्वरूपिणे}
{सर्वेभ्यो वरदात्रे तेऽनादिसिद्धाय भो नमः} % १००

\twolineshloka
{क्षमापराधं देवेश त्वन्मायामोहधारिणः}
{शरणं ते प्रसन्नस्य रक्ष मां महतो भयात्} % १०१

\twolineshloka
{मां दृष्ट्वा हर्षसम्पन्ना अभवन् देवतादयः}
{अधुना मां निरीक्ष्यैव दोषयुक्ता भयातुराः} % १०२

\twolineshloka
{अतो मां नैव पश्यन्ति पापिनां पापरूपिणम्}
{निर्दोषं कुरु विघ्नेश करुणालय ते नमः} % १०३

\twolineshloka
{सर्वदा दर्शनं ते वै ममास्तां विघ्नवारण}
{तेनाऽहं कृतकृत्यश्च भवेयं योगिसम्मतः} % १०४

\twolineshloka
{ततोऽतिभक्तिसंयुक्तं नृत्यन्तं देवसन्निधौ}
{रोमाञ्चाश्रुसमायुक्तं दृष्ट्वा ढुण्ढिर्जगाद तम्} % १०५

\uvacha{श्रीगणेश उवाच}

\twolineshloka
{त्वया कृतमिदं स्तोत्रं मम प्रीतिविवर्धनम्}
{अपराधाननन्तांश्च सहे तेऽनेन निश्चितम्} % १०६

\twolineshloka
{सर्वसिद्धिप्रदं पूर्णं भविष्यति विशेषतः}
{पठतां श‍ृण्वतां चैव नानासुखकारकं भवेत्} % १०७

\twolineshloka
{यथापूर्वं स्थितश्चन्द्र तादृशो भव नित्यदा}
{यस्मिन् दिने कृतं हास्यं तत्र दोषयुतो भवेः} % १०८

\twolineshloka
{ललाटे भूषणं मे त्वं भव सेवापरायणः}
{चतुर्थ्यां कृष्णपक्षस्य व्रते ते पूजनं भवेत्} % १०९

\twolineshloka
{मां पूजयित्वा तारेश अर्घ्यं यस्ते प्रदास्यति}
{तस्य सिद्धिर्भवेत् पूर्णा व्रतजा नान्यथा क्वचित्} % ११०

\twolineshloka
{द्वितीयायां च सायाह्ने त्वया प्राप्तोऽहमादरात्}
{अतः शुक्लद्वितीयायां नमस्यो मानवैर्भव} % १११

\twolineshloka
{द्वितीयायां प्रयत्नेन त्वां नमस्यन्ति मानवाः}
{मासगं दुःखमुत्सृज्य क्वचित् तिष्ठति नान्यथा} % ११२

\twolineshloka
{एवमुक्त्वाऽन्तर्दधेऽसौ गणेशश्चन्द्रपूजितः}
{तस्मिन् काले च देवेशा देवास्तत्र समाययुः} % ११३

\twolineshloka
{मानयामास सम्पूज्य चन्द्रस्तांस्तैः समन्वितः}
{गणेशं ब्राह्मणैः सार्धं संस्थाप्यापूजयत् प्रभुम्} % ११४

\twolineshloka
{गणेशभालसंस्थानं प्राप्तं चन्द्रेण यत्र वै}
{भालचन्द्रं गणेशानं वदन्ति तत्र संस्थितम्} % ११५

\twolineshloka
{देवैः सम्पूजितस्तत्र ब्राह्मणैर्गणनायकः}
{प्रगृह्य चन्द्रं निर्दोषं ततः स्वर्गं ययुः सुराः} % ११६

\twolineshloka
{चन्द्रः स्वांशस्वरूपेण स्वर्गे तिष्ठति भूमिप}
{पूर्णरूपेण विघ्नेशं सेवते नित्यमादरात्} % ११७

\twolineshloka
{गङ्गातीरे निवासं स चकारात्रिसमुद्भवः}
{विधुश्च भालचन्द्रं तं सेवते भक्तितत्परः} % ११८

\twolineshloka
{अथ कस्मिंश्च समये देवैः सम्प्रार्थितो हरिः}
{यादवेषु समुत्पन्नो वासुदेवो बभूव ह} % ११९

\twolineshloka
{अज्ञानेन विधोस्तेन दर्शनं प्रकृतं नृप}
{भाद्रशुक्लचतुर्थ्यां तु सदोषः स बभूव ह} % १२०

\twolineshloka
{अन्तर्ज्ञानं गतं तस्य नरतुल्यो बभूव सः}
{तथापि विघ्नसंयुक्तो हरिस्तच्छृणु भूमिप} % १२१

\twolineshloka
{सत्राजिन्नाम राजर्षिर्यादवेषु बभूव वै}
{तेन सूर्यतपस्तप्तं दारुणं शतवत्सरम्} % १२२

\twolineshloka
{सूर्येण स्वगले संस्थो मणिर्दत्तः स्यमन्तकः}
{सत्राजिते स तद्रत्नं गले भक्त्या बबन्ध ह} % १२३

\twolineshloka
{नित्यं सुवर्णभाराणामष्टौ स्रवति सन्मणिः}
{सूर्यतेजःसमानश्चाऽशोभयद्यदुनन्दनम्} % १२४

\twolineshloka
{द्वारकायां स्थितं तं स सूर्यभक्तिपरायणम्}
{शुचिभावेन रत्नं सोऽपूजयत्तान्नरन्तरम्} % १२५

\twolineshloka
{ततो बहुगते काले कृष्णो दोषयुतोऽभवत्}
{तस्य बुद्धिः समुत्पन्ना मणेः सङ्ग्रहणे नृप} % १२६

\twolineshloka
{सभायां संस्थितः कृष्णः स सत्राजितमाह्वयत्}
{आगतं तमुवाचाऽसौ विनयेन समन्वितः} % १२७

\uvacha{श्रीकृष्ण उवाच}

\twolineshloka
{उग्रसेनश्च सर्वेषां यादवानां महामते}
{राजाऽस्माभिः कृतस्तं ते यादवाः पूजयन्ति भोः} % १२८

\twolineshloka
{अस्माभिर्दिग्जये प्राप्तमुग्रसेनाय यादव}
{दत्तं श्रेष्ठं तथा त्वं तं मणिं देहि स्यमन्तकम्} % १२९

\twolineshloka
{सुवर्णं मणिसम्भूतं गृहाण त्वं तु नित्यदा}
{शोभार्थं कण्ठगं तात मणिं कुरु नृपस्य तम्} % १३०

\twolineshloka
{एवमुक्तः स सत्राजित्तमुवाच समन्युना}
{प्रसेनाय मया दत्तः स मे नैव ददाति तम्} % १३१

\twolineshloka
{एवमुक्त्वा गृहं गत्वा स्वानुजाय ददौ मणिम्}
{प्रसेनाय स्ववृत्तान्तं कथयामास विस्तरात्} % १३२

\twolineshloka
{ततः कदाचित् कृष्णेन सहिता यदुनन्दनाः}
{मृगयार्थं वने राजन् जग्मुर्हर्षसमन्विताः} % १३३

\twolineshloka
{तत्राशुचिस्वभावेन प्रसेनस्तं मणिं परम्}
{बबन्ध स्वगले सोऽपि तैः सह प्रययौ वनम्} % १३४

\twolineshloka
{अशुचित्वस्य दोषेण सिंहकस्तं प्रसेनकम्}
{साश्वं हत्वाऽऽगृह्य मणिं ययौ स्थानं स्वकं महान्} % १३५

\twolineshloka
{ततस्तं जाम्बवान् हत्वा गृह्य सिंहं मणिं ययौ}
{स्थानं पुत्र्या मणिं हस्ते तं ददौ स महाबलः} % १३६

\twolineshloka
{तव भर्ता सुते यो वै भविष्यति मणिं परम्}
{तस्मै दास्यामि भावेन त्वदीयोऽयं मणिः स्मृतः} % १३७

\twolineshloka
{सायाह्ने यादवाः सर्वे आगताः स्वस्वमन्दिरे}
{प्रसेनेन विहीनास्ते सत्राजित् क्रोधमादधे} % १३८

\twolineshloka
{अहो कृष्णेन मे भ्राता संहतो नात्र संशयः}
{मणिलोभवशेनाऽपि पापिना पापचेतसा} % १३९

\twolineshloka
{ततो लोकाः पुरे संस्थाः सत्यमामानयंस्तु ते}
{उग्रसेनादयः सर्वे कृष्णं न्यभर्त्सयन्नृप} % १४०

\twolineshloka
{ततः कृष्णश्च दुःखार्तः स ध्यानेनावलोकयन्}
{नरवन्न बुबोधाऽपि स दोषो वसुदेवजः} % १४१

\twolineshloka
{ततोऽतिदुःखितो भूत्वा विचारं देवकीसुतः}
{चकारान्तरजं ज्ञानं क्व गतं मम साम्प्रतम्} % १४२

\twolineshloka
{नरतुल्यः कथं जातः कारणं नात्र वेद्मि वै}
{देहत्यागं करिष्यामि वृथा लाञ्छनकारणात्} % १४३

\twolineshloka
{ततः सभां समागम्याऽऽगृह्य यादवमुख्यकान्}
{वनं ययौ प्रसेनस्य निश्चयार्थं महाद्युतिः} % १४४

\twolineshloka
{पादमार्गं प्रसेनस्य ते सङ्गृह्य महावने}
{गतास्तत्र मृतं साश्वं प्रसेनं ददृशुर्नराः} % १४५

\twolineshloka
{सिंहेन संहतं दृष्ट्वा मणिहीनं तु यादवाः}
{पुरतस्ते ययुः सर्वे सिंहमार्गप्रधारिणः} % १४६

\twolineshloka
{ततो महावने सिंहमृक्षेण प्रहतं पुनः}
{ददृशुर्न मणिं तत्र लेभिरे भयसङ्कुलाः} % १४७

\twolineshloka
{ऋक्षपादं तथाऽऽगृह्य ययुर्यादवमुख्यकाः}
{महावने बिले घोरे गतमृक्षमलोकयन्} % १४८

\twolineshloka
{ततस्तांस्तत्र संस्थाप्य कृष्णः परपुरञ्जयः}
{बिले सम्प्रविवेशाऽथ लाञ्छनस्यापनोदनात्} % १४९

\twolineshloka
{बिले योजनमात्रं स गत्वा तत्र ददर्श ह}
{प्रकाशयुक्तं संस्थानं शोभमानं सुविस्तरम्} % १५०

\twolineshloka
{दोलायां संस्थितं वीक्ष्य मणिं कृष्णो जहर्ष च}
{तत्र सुप्तं शिशुं दृष्ट्वा शनैरागात् स यादवः} % १५१

\twolineshloka
{ततोऽकस्माज्जाम्बवतो बहिः पुत्री समाययौ}
{अपूर्वं पुरुषं दृष्ट्वा चुक्रोश भयसङ्कुला} % १५२

\twolineshloka
{ततो जाम्बवता राजन् श्रुतं पुत्र्याः प्ररोदनम्}
{आययौ च स तं दृष्ट्वा युयुधे विष्णुना स्वयम्} % १५३

\twolineshloka
{गतास्तयोरेकविंशतिघस्रास्तु प्रयुध्यतोः}
{यादवा जग्मिरे सर्वे बहिःस्था अष्टमे दिने} % १५४

\twolineshloka
{द्वारकायां स्ववृत्तान्तं कथयन्ति स्म विह्वलाः}
{दिवसाः सप्त तत्रैव किलास्माकं गता बहिः} % १५५

\twolineshloka
{तथापि विवरात् कृष्णो न बहिश्च समागतः}
{मृतो वा जीवितो वा स ज्ञायते किं न यादवः} % १५६

\twolineshloka
{वसुदेवादयः श्रुत्वा सस्त्रीका नृप दुःखिताः}
{उग्रसेनादयः सर्वे रुरुदुर्दुःखकर्शिताः} % १५७

\twolineshloka
{परेऽस्माकं विशेषेण शत्रवः संस्थिता अहो}
{जरासन्धादयश्चान्ये किं भविष्यति यादवाः} % १५८

\twolineshloka
{हीनाः कृष्णेन सञ्जाता अधुना नात्र संशयः}
{द्विषद्भिः संहताः सर्वे मरिष्यामः सपक्षकाः} % १५९

\twolineshloka
{ततो रामेण ते सर्वे सान्त्विता बहुलोक्तिभिः}
{निःश्वस्य संस्थिता भूप दुःखयुक्ता भयाऽऽतुराः} % १६०

\twolineshloka
{सत्राजितं निनिन्दुस्ते व्यर्थलाञ्छनदं तदा}
{सोऽपि दुःखसमायुक्तो बभूव भयसङ्कुलः} % १६१

\twolineshloka
{अथ तं जाम्बवान् कृष्णमुवाच घननिःस्वनः}
{कस्त्वं वीर महातेजा योधयन्नसि मां वद} % १६२

\twolineshloka
{ततस्तं प्रत्युवाचेदं कृष्णो वचनमुत्तमम्}
{यदुवंशभवं कृष्णं मां जानीहि महामते} % १६३

\twolineshloka
{श्रुत्वा खेदसमायुक्तो जाम्बवांस्तं ननाम वै}
{रामेणोक्तं वचो रम्यं स त्वं विष्णुः समागतः} % १६४

\twolineshloka
{अहं दासस्त्वदीयो वै क्षमस्व करुणानिधे}
{अज्ञानाच्छक्तिगर्वेणापराधं युद्धजं च मे} % १६५

\twolineshloka
{स्तुतः सम्पूजितस्तेन कृष्णो वृत्तान्तमादरात्}
{अगदत्तं ततः श्रुत्वा कन्यायुक्तं मणिं ददौ} % १६६

\twolineshloka
{गृहीत्वा सस्त्रियं कृष्णो द्वारकायां समागतः}
{आगतं कृष्णमालोक्य हर्षयुक्ता बभूविरे} % १६७

\twolineshloka
{यादवैश्च ततः कृष्णः समानाय्य महामणिम्}
{सत्राजिते ददौ तत्र सर्वेषां पश्यतां नृप} % १६८

\twolineshloka
{सत्राजिन् मनसाऽत्यन्तं निन्दयामास तां तनुम्}
{भयभीतः स कृष्णस्य विरोधेन महायशाः} % १६९

\twolineshloka
{सत्यभामां ददौ तस्मै मणियुक्तां महात्मने}
{कृष्णाय स उवाचेदं श्वशुरो विनयान्वितः} % १७०

\twolineshloka
{सूर्यभक्तियुतस्त्वं वै मणिं रक्ष महामते}
{दौहित्रास्ते भविष्यन्ति ग्रहीष्यन्ति मणिं च ते} % १७१

\twolineshloka
{तथेति यादवेनैव कृतं सत्राजिता नृप}
{एतस्मिन्नन्तरे तत्र वृत्तान्तो यादवैः श्रुतः} % १७२

\twolineshloka
{लाक्षागृहेषु सन्दग्धाः कुन्त्या युक्ताश्च पाण्डवाः}
{पुरोचनेन तज् ज्ञात्वा रुरुदुर्यादवा बहु} % १७३

\twolineshloka
{रामेण हस्तिनापुर्यां कृष्णः सद्यो जगाम ह}
{धृतराष्ट्रादिभिः सोऽपि रुरोद शोकसङ्कुलः} % १७४

\twolineshloka
{शतधन्वा ततोऽक्रूरः कृतवर्मा च यादवः}
{मिलिताः क्रोधसंयुक्ता विचारं चक्रुरादरात्} % १७५

\twolineshloka
{सत्राजिता वयं सर्वे कन्यार्थं संवृताः पुरा}
{अस्मान् सन्त्यज्य कृष्णाय ददौ कन्यां सुरूपिणीम्} % १७६

\twolineshloka
{कृष्णः पाण्डवशोकार्तोऽभवत् तत्रैव संस्थितः}
{स रामो निश्यतः कार्यं कर्तव्यं तत् त्रिभिः किल} % १७७

\twolineshloka
{हत्वा सत्राजितं सुप्तं मणिर्ग्राह्यो न संशयः}
{एवं निश्चित्य रात्रौ तं शतधन्वा जगाम ह} % १७८

\twolineshloka
{सत्राजितं स हत्वा वै मणिं जग्राह दारुणः}
{संस्थाप्य तैलद्रोण्यां तं कृष्णं सत्याययो ततः} % १७९

\twolineshloka
{श्वशुरं स हतं श्रुत्वा रामेण सहसा गतम्}
{कृष्णं ज्ञात्वा रोषितं शतधन्वाऽक्रूरमाययौ} % १८०

\twolineshloka
{अक्रूरेण समात्यक्तः कृतवर्माणमाययौ}
{तेनापि न धृतः पक्षः पुनः सोऽक्रूरमाययौ} % १८१

\twolineshloka
{मणिं तत्रैव निक्षिप्य पपाल भयसङ्कुलः}
{शतयोजनगाऽश्विन्यां समारुह्य नृपात्मज} % १८२

\twolineshloka
{पलायमानं विज्ञाय कृष्णः सङ्कर्षणान्वितः}
{रथमारुह्य वेगेन तमनु प्रययौ तदा} % १८३

\twolineshloka
{योजनानां शतं गत्वा ममार ह पुरोऽश्विनी}
{शतधन्वा भयैर्युक्तः पलायत पदा ततः} % १८४

\twolineshloka
{पदातिनं च तं ज्ञात्वा कृष्णोऽधावत् सुसत्वरः}
{पदातिर्बलभद्रे स रथं त्यक्त्वा महाद्युतिः} % १८५

\twolineshloka
{ततश्चक्रेण कृष्णेन हतस्तत्र पपात ह}
{शतधन्वा तेन तस्मिन् न दृष्टो मणिरुत्तमः} % १८६

\twolineshloka
{ततः शोकसमायुक्तो बलभद्रमुवाच ह}
{शतधन्वा हतो हीनो मणिना तत्र मानद} % १८७

\twolineshloka
{उवाच कोपयुक्तस्तं स ततो रोहिणीसुतः}
{मणिस्त्वया सुसङ्गुप्तो मिथ्या वाक् त्वं महाखल} % १८८

\twolineshloka
{अग्रजाय प्रदातव्यस्तदर्थं मणिरादरात्}
{कृतं त्वया महालोभिन् कर्म साधु जुगुप्सितम्} % १८९

\twolineshloka
{ततस्तं विनयेनैव सान्त्वयामास केशवः}
{द्विजदेवगवां ते मे शपथो न कृतं मया} % १९०

\twolineshloka
{तं तिरस्कृत्य वेगेन त्यक्त्वा रामो जगाम ह}
{विदर्भ राजनीतिज्ञं स्वमित्रं नृपतिं तदा} % १९१

\twolineshloka
{ततोऽतिदुःखसंयुक्तः कृष्णः परमदारुणम्}
{रुरोद बलभद्रं स स्मृत्वा स्मृत्वा विशेषतः} % १९२

\twolineshloka
{ज्ञानदृष्ट्या स कृष्णेन मणिर्नाप्तो महीपते}
{ततोऽतिदुःखितो भूत्वा रुरोद च पुनः पुनः} % १९३

\twolineshloka
{अहो ज्ञानं मदीयं यद्गतं कुत्र विशेषतः}
{नरतुल्यः कृतः केन धिङ् मे जीवितकं किल} % १९४

\twolineshloka
{रथसंस्थः समायातो द्वारकायां महायशाः}
{लोकवृत्तान्तमुग्रं स कथयामास विस्तरात्} % १९५

\twolineshloka
{ततो यादवमुख्यैः स भर्त्सितोऽतिविशेषतः}
{कृष्णः सत्राजितः कर्म चकारोत्तरकं नृप} % १९६

\twolineshloka
{देशे देशे नृपाद्याश्च जनाः कृष्णं विशेषतः}
{निनिन्दुः कर्म कृष्णेन किं कृतं दुष्टबुद्धिना} % १९७

\twolineshloka
{समर्थं बलभद्रं योऽत्यजज्ज्येष्ठं सुदुर्मतिः}
{मणिलोभी महापापी विश्वास्यो नैव केनचित्} % १९८

\twolineshloka
{द्वारकायां जनाः सर्वे तं कृष्णं न प्रमेनिरे}
{त्यक्तः सर्वैस्तथा कृष्णः शुशोच परमव्यथः} % १९९

\twolineshloka
{अस्थित्वचासमायुक्तं सदा स्वगृहसंस्थितम्}
{नारदस्तं समालोच्याययौ दृष्ट्वा सुविस्मितः} % २००

\twolineshloka
{तं प्रणम्य महात्मानं कृष्णः परमदुःखितः}
{उवाच भावभक्त्या वै कृताञ्जलिपुटः शनैः} % २०१

\uvacha{श्रीकृष्ण उवाच}

\twolineshloka
{सर्वैः सन्त्याजितो दुःखी गृहे नित्यं वसाम्यहम्}
{तत्र ते दर्शनं स्वामिन्नभवत् भाग्ययोगतः} % २०२

\twolineshloka
{ततस्तं नारदः प्राह किमर्थं त्यज्यसे वद}
{एवं पृष्टः पुनः प्राह वृत्तान्तं सर्वमञ्जसा} % २०३

\twolineshloka
{श्रुत्वा वृत्तान्तमुग्रं वै ध्यानेनालोक्य नारदः}
{देहत्यागे समुद्योगं कुर्वन्तं तमुवाच ह} % २०४

\uvacha{नारद उवाच}

\twolineshloka
{भाद्रशुक्लचतुर्थ्यां च त्वया चन्द्रो विलोकितः}
{शप्तोऽसौ विघ्नराजेन तेन सन्दुःखितो भवान्} % २०५

\twolineshloka
{देहत्यागं महाबुद्धे मा कुरुष्व विशेषतः}
{गणेशं भज भावेन सङ्कष्टीव्रतसंयुतम्} % २०६

\twolineshloka
{मिथ्याऽपवादशून्यस्त्वं भविष्यसि न चान्यथा}
{सज्ञानश्च समर्थेशस्तस्मात् तं शरणं व्रज} % २०७

\twolineshloka
{एवमुक्त्वा महायोगी नारदः प्रययौ ततः}
{गणेशगानसंयुक्तो वीणावादनलालसः} % २०८

\twolineshloka
{ततः कृष्णः प्रहर्षेण युक्तो विघ्नेशमादरात्}
{अभजद् ध्यानमार्गज्ञो ध्यानेनोग्रेण भूमिप} % २०९

\twolineshloka
{एकाक्षरेण मन्त्रेणापूजयन्नित्यमेव तम्}
{व्रतं चकार स ततः सङ्कष्टीसंज्ञमुत्तमम्} % २१०

\twolineshloka
{रात्रावेकान्तसंस्थं तं ध्याननिष्ठं जनार्दनम्}
{निश्चलं गणराजः स ययौ भक्तं सुखप्रदः} % २११

\twolineshloka
{बोधयामास तं कृष्णं गणेशो ब्रह्मनायकः}
{तमुत्थाय ननामाऽसौ पूजयामास चादरात्} % २१२

\twolineshloka
{पुनः प्रणम्य तुष्टाव विघ्नेशं सामगेन सः}
{स्तोत्रेणाष्टकसंज्ञेन नाम्ना तं प्रननर्त ह} % २१३

\twolineshloka
{सरोमाञ्चं गणेशानः कृष्णं दृष्ट्वा जगाद ह}
{वरं वरय मत्तस्त्वं दास्यामि हृदि वाञ्छितम्} % २१४

\twolineshloka
{गणेशवचनं श्रुत्वा तं जगाद महायशाः}
{कृष्णो भक्त्या समायुक्तो वचनं खेदसंयुतः} % २१५

\uvacha{श्रीकृष्ण उवाच}

\twolineshloka
{यदि तुष्टोऽसि विघ्नेश वरदोऽसि विशेषतः}
{तदा ते भक्तिमुग्रां मे देहि ह्यव्यभिचारिणीम्} % २१६

\twolineshloka
{भाद्रशुक्लचतुर्थ्यां ये चन्द्रं पश्यति विघ्नप}
{ते शापदुःखहीना वै भवन्तु कृपया च ते} % २१७

\twolineshloka
{अहं चासन्नमरणजनवज्ज्ञानवर्जितः}
{प्रकृतोऽन्यस्य का वार्ता प्रार्थयामि ततः प्रभो} % २१८

\twolineshloka
{जगाद गणनाथः स ततस्तं भक्तिमोहितः}
{मम भक्तिर्दृढा कृष्ण भविष्यति च तेऽनघ} % २१९

\twolineshloka
{स्यमन्तकस्य माहात्म्यं त्वदीयं लाञ्छनप्रदम्}
{वरदानं मदीयं च यः श‍ृणोति जनार्दन} % २२०

\twolineshloka
{पठेत्तु भावयुक्तश्चेन् मां धृत्वा हृदये हरे}
{स एव दोषसंहीनो भवत्वत्र न संशयः} % २२१

\twolineshloka
{एवमुक्त्वा गणेशानः कृष्णं भक्तिपरायणम्}
{अन्तर्धानं चकाराऽसौ कृष्णस्तत्र समास्थितः} % २२२

\twolineshloka
{सम्पूर्णं जागरं कृत्वा पञ्चम्यां नियतः प्रभुः}
{पूजयित्वा गणेशानं ददौ दानं विशेषतः} % २२३

\twolineshloka
{गणेशकृपया तस्याऽन्तर्ज्ञानं सम्बभूव ह}
{ग्रहणं च मणेस्तेन ज्ञातमक्रूरकारितम्} % २२४

\twolineshloka
{सभायां स समागम्याक्रूरं तत्राजुहाव च}
{तं प्रणम्य महाभागमुवाच हास्यसंयुतः} % २२५

\twolineshloka
{मणिश्च तात सन्त्यक्तो गेहे ते शतधन्वना}
{बलभद्रार्थमद्य त्वं मणिं दर्शय मानद} % २२६

\twolineshloka
{ततोऽक्रूरेण चानीतो मणिस्तत्र सुसंसदि}
{दृष्ट्वोग्रसेनमुख्यास्ते प्रशशंसुर्जनार्दनम्} % २२७

\twolineshloka
{अक्रूरश्च विदर्भं तं गत्वा हलधरं प्रभुम्}
{मणिं दत्त्वा समानीय सभायां संस्थितोऽभवत्} % २२८

\twolineshloka
{समागतं महावीरमनमत् केशवः पुनः}
{सर्वे हर्षयुता जाता द्वारकावासिनस्ततः} % २२९

\twolineshloka
{निर्दोषं वासुदेवं ते मानयामासुरादरात्}
{राजानः सुहृदः सर्वे नानादेशनिवासिनः} % २३०

\twolineshloka
{सर्वेषां हृदि संस्थोऽयं गणेशो बुद्धिनायकः}
{तस्यैव कृपया राजन् भवेत् किं किं सुदुर्लभम्} % २३१

\twolineshloka
{अतो भाद्रचतुर्थ्यां वै न द्रष्टव्यः कदाचन}
{चन्द्रः कदाचित् दृष्टश्चेदिदं श्रोतव्यमादरात्} % २३२

\twolineshloka
{तेन दोषविहीनश्च जायते नात्र संशयः}
{अन्यथा भ्रष्टभावेन नारकी स नरो भवेत्} % २३३

\twolineshloka
{स्यमन्तकमणेश्चित्रं चरितं कथितं परम्}
{चन्द्रदर्शनजं दोषं हरति श्रवणेन च} % २३४

॥ॐ तत्सदिति श्रीमदान्त्ये पुराणोपनिषदि श्रीमन्मौद्गले महापुराणे चतुर्थे खण्डे गजाननचरिते भाद्रशुक्लचतुर्थी चन्द्रदर्शनदोषहरणचरितवर्णनं नाम षष्ठोऽध्यायः॥४.६॥


\sect{४.७ --- सप्तमोऽध्यायः --- भाद्रपदशुक्लचतुर्थीमाहात्म्यवर्णनम्}

\centerline{॥ श्रीगणेशाय नमः ॥}

\uvacha{दशरथ उवाच}

\twolineshloka
{श्रुतं मया महाख्यानं त्वत्तो वेदविदांवर}
{चतुर्थ्योरुभयोः सारं माहात्म्येन समन्वितम्} % १

\twolineshloka
{तथापि वद मे ब्रह्मन् पृथक् व्रतसमुद्भवम्}
{चरित्रं विस्तरेणैव केन केन कृतं पुरा} % २

\twolineshloka
{कीदृशी चाऽभवत् सिद्धिः केन प्राप्तो गजाननः}
{श्रुत्वा व्रतस्य माहात्म्यं यामि नो तृप्तिमादरात्} % ३

\uvacha{मुद्गल उवाच}

\twolineshloka
{एवं पृष्टो महायोगी वसिष्ठः परमार्थवित्}
{जगाद तं प्रजानाथ श‍ृणु तत् सुखदं परम्} % ४

\uvacha{वसिष्ठ उवाच}

\twolineshloka
{आदौ शम्भ्वादिभिश्चैतत् कृतं व्रतमनुत्तमम्}
{कृष्णशुक्लचतुर्थ्यां यदुपोषणपरायणैः} % ५

\twolineshloka
{ततः स्वायम्भुवाद्यैश्च कृतं सर्वार्थसिद्धिदम्}
{ततो मुनिगणैः सर्वैरन्यैर्वर्णस्थजन्तुभिः} % ६

\twolineshloka
{अधुनाऽहं पृथक्त्वेन कथयामि व्रतोद्भवम्}
{माहात्म्यं सर्वदं पुण्यं सङ्क्षेपेण श‍ृणुष्व तत्} % ७

\twolineshloka
{पुरा दौष्यन्तिको नाम भरतः सम्बभूव ह}
{तेजस्वी वीरमुख्येशः शस्त्रास्त्रज्ञः प्रतापवान्} % ८

\twolineshloka
{जित्वा भूमण्डलं सर्वं सार्वभौमो बभूव ह}
{सप्तद्वीपवतीं पृथ्वीमपालद्धर्मसंयुतः} % ९

\twolineshloka
{तद्देशे दैवयोगेन ह्यनावृष्टिः सुदारुणा}
{बभूव सर्वलोकास्ते भयभीता बभूविरे} % १०

\twolineshloka
{चराचरं व्यथायुक्तं दृष्ट्वा राजा महायशाः}
{कण्वं मुनिवरं शान्तं ययौ शरणमादरात्} % ११

\twolineshloka
{तं प्रणम्य महात्मानं पूजयामास भक्तितः}
{कृत्वा करपुटं तस्याऽग्रे स्थितो भरतोऽभवत्} % १२

\twolineshloka
{तमुवाच महायोगी गाणपत्यप्रियः सदा}
{कण्वो वेदार्थसन्निष्ठो भरतं राजसत्तमम्} % १३

\uvacha{कण्व उवाच}

\twolineshloka
{किमर्थमागतो राजन् कुशलं ते महामते}
{अस्ति देशादिभावेषु तिष्ठ त्वं चासनोत्तमे} % १४

\twolineshloka
{कण्वस्य वचनं श्रुत्वा हर्षयुक्तो महीपतिः}
{मुनिदत्ते विवेशासावासने स कृताञ्जलिः} % १५

\twolineshloka
{उवाच तं महात्मानं कण्वं वेदविदां वरम्}
{भरतो राजनीतिज्ञो भक्तियुक्तो विशेषतः} % १६

\uvacha{भरत उवाच}

\twolineshloka
{त्वत्प्रसादेन योगीन्द्र कुशलं मे प्रवर्तते}
{दैवयोगेन तदपि दुःखितोऽहं समागतः} % १७

\twolineshloka
{नित्यं स्वधर्मयुक्तेन मया सम्पालिता मही}
{देवव्रतातिथिप्राज्ञप्रीत्यर्थं योगिसत्तम} % १८

\twolineshloka
{वर्णाश्रमाचारयुता जनाः सर्वे भवन्ति च}
{तथापि पापजं दुःखं सम्प्राप्तं तन्न वेद्म्यहम्} % १९

\twolineshloka
{अनावृष्टिश्च सम्भूता प्रभो सर्वत्र दुःखदा}
{चराचरं भयोद्विग्नं बभूव रसहीनतः} % २०

\twolineshloka
{तदर्थं त्वां महायोगिन्नागतोऽहं विशेषतः}
{त्वद्दर्शनजपुण्येन सफलो मे भवोऽभवत्} % २१

\twolineshloka
{अधुना ब्रूहि पापस्य स्वरूपं तन्निहन्म्यहम्}
{येन वृष्टिर्भवेत् पूर्णा तथा कुरुष्व मानद} % २२

\twolineshloka
{ततस्तं नृपशार्दूलमुवाच मुनिसत्तमः}
{कण्वः श्रुत्वा च वृत्तान्तं दयायुक्तः स्वभावतः} % २३

\uvacha{कण्व उवाच}

\twolineshloka
{श‍ृणु राजन् महत् पापं तव राज्ये प्रवर्तितम्}
{तेनाऽवृष्टिभवं दुःखं प्रादुर्भूतं न संशयः} % २४

\twolineshloka
{चतुःपदविहीनं ते राज्यं भवति निश्चितम्}
{जनैः सर्वैः सुशीलैश्च पापं घोरं कृतं महत्} % २५

\twolineshloka
{त्वं पापशीलभावेन वर्तसे जनवत्सल}
{धर्मार्थकाममोक्षैश्च हीनोऽसि पुरुषाधम} % २६

\twolineshloka
{चतुर्थीसम्भवं तात व्रतं सर्वार्थसिद्धिदम्}
{कार्ष्णं शौक्लं सुविख्यातं भ्रष्टं राज्येऽभवन्नृप} % २७

\twolineshloka
{चतुर्विधं जगत्सर्वं स्थूलसूक्ष्मादिभेदतः}
{सङ्कष्टं यत्तदेव त्वं जानीहि नृपनायक} % २८

\twolineshloka
{चतुर्विधं च सङ्कष्टं हरति व्रतकारिणः}
{सा सङ्कष्टचतुर्थी वै कृष्णा ते कथिता मया} % २९

\twolineshloka
{चतुर्विधं या ददाति सा शुक्ला वरदा मता}
{सञ्चितं नास्ति चेद्राजन् व्रतकारिजनाय वै} % ३०

\twolineshloka
{चतुर्थीजं स माहात्म्यं कथयामास विस्तरात्}
{कण्वस्तद्भरतायैव स्वशिष्याय विशेषतः} % ३१

\twolineshloka
{व्रतानि वै निष्फलानि चतुर्थीहीनकानि चेत्}
{ज्ञात्वा प्रणम्य तं राजोवाच हर्षसमन्वितः} % ३२

\uvacha{भरत उवाच}

\twolineshloka
{स्वामिन् वद गणेशस्य माहात्म्यं सर्वसिद्धिदम्}
{एतादृशं व्रतं यस्य तं भजिष्यामि नित्यदा} % ३३

\uvacha{कण्व उवाच}

\twolineshloka
{श‍ृणु राजन् गणेशस्य माहात्म्यं सर्वदं परम्}
{क्रतुना कथितं मे यद्ब्रह्मपुत्रेण धीमता} % ३४

\twolineshloka
{एकदाऽहं तपोयुक्तस्तिष्ठामि स्वाश्रमे पुरा}
{वायुमात्राशनो भूत्वा तपामि स्म तपो महत्} % ३५

\twolineshloka
{मदीयतपसा सर्वं व्याप्तमासीन्नराधिप}
{तथापि योगमास्थाय संस्थितोऽहं विशेषतः} % ३६

\twolineshloka
{ततः प्रजापतिः साक्षादाश्रमे मे समागतः}
{क्रतुर्योगीन्द्रवन्द्यो यो गाणपत्यो महायशाः} % ३७

\twolineshloka
{तं प्रणम्य महात्मानं पूजयित्वा पुरः स्थितम्}
{कृताञ्जलिं मामुवाच स तथा भक्तवत्सलः} % ३८

\uvacha{क्रतुरुवाच}

\twolineshloka
{वत्स तिष्ठस्व मे दत्त आसने किं महामते}
{इच्छा ते वद मां तात करिष्यामि हि तं च ते} % ३९

\twolineshloka
{तस्य तद्वचनं श्रुत्वा हर्षितोऽहं नराधिप}
{आसने समुपावेश्यावदत्तं विनयान्वितः} % ४०

\uvacha{कण्व उवाच}

\twolineshloka
{तव दर्शनमात्रेण कृतकृत्योऽस्मि साम्प्रतम्}
{तथापि शान्तिदं योगं वद पूर्णं दयानिधे} % ४१

\uvacha{क्रतुरुवाच}

\twolineshloka
{सम्यक् पृष्टं त्वया वत्स श्रेष्ठाच्छ्रेष्ठतमं महत्}
{कथयामि महाभाग ब्रह्मणः संश्रुतं मया} % ४२

\twolineshloka
{तपस्त्यक्त्वा पुरा तात योगमार्गपरायणः}
{न शान्तिं प्रलभे तत्र योगभूमिं प्रसाधयन्} % ४३

\twolineshloka
{ततोऽहं पितरं गत्वा ब्रह्माणं सर्ववेदिनम्}
{वन्द्यं तं योगशान्त्यर्थमपृच्छं विनयान्वितः} % ४४

\twolineshloka
{स्वामिन् शान्तिप्रदं ब्रह्म कीदृशं वद मे प्रभो}
{केन योगेन लभ्यं तत् प्रभवेत् कृपयान्वितः} % ४५

\uvacha{ब्रह्मोवाच}

\twolineshloka
{योगशान्तिप्रदं ब्रह्म गाणेशं विद्धि पुत्रक}
{मनोवाणीमयं सर्वं त्यज योगस्य सेवया} % ४६

\twolineshloka
{मनोवाणीविहीनं यदेव तत्तादृशं मतम्}
{गणेशोऽहं न भिन्नश्च ब्रह्मणां नायकः स्मृतः} % ४७

\twolineshloka
{मनोवाणीमयः प्रोक्तो गकारो वेदवादतः}
{मनोवाणीविहीनश्च णकारः सर्वसम्मतः} % ४८

\twolineshloka
{तयोः स्वामी गणेशानो नाम्ना गणपतेर्यदा}
{गकारस्य णकारस्य योगो वेदप्रमाणतः} % ४९

\twolineshloka
{समाधिना लभ्यते यच्चित्तेन च महामते}
{गकाराक्षरगं ज्ञानं ज्ञातव्यं वेदवादतः} % ५०

\twolineshloka
{चित्तेन यन्न लभ्येत णकारं विद्धि मानद}
{ज्ञानाज्ञानमयं चित्तं त्यक्त्वा शान्तिमवाप्स्यसि} % ५१

\twolineshloka
{एतदेव परं गुह्यं शान्तिदं कथितं मया}
{तदर्थं गणराजं त्वं भजस्व भावसंयुतः} % ५२

\twolineshloka
{एकाक्षरं महामन्त्रं गृहाण त्वं महामते}
{ध्यानयोगेन विघ्नेशं प्राप्स्यसि त्वं न संशयः} % ५३

\twolineshloka
{एवमुक्त्वा महामन्त्रं ददौ मह्यं विधानतः}
{तं प्रणम्य वने तात गतोऽहं योगकारणात्} % ५४

\twolineshloka
{क्रमेण चित्तभूमीनां योगं त्यक्त्वा मया परम्}
{चित्तं चिन्तामणौ पुत्र क्षिप्तं तद्रूपभावतः} % ५५

\twolineshloka
{ततो योगीन्द्रवन्द्योऽहं जातस्तदपि नित्यदा}
{गणेशध्यानसंयुक्तोऽभवं तद्भक्तिकाम्यया} % ५६

\twolineshloka
{ततो विघ्नपतिः साक्षाद्दर्शनं मे ददौ मया}
{स्तुतः सम्पूजितः सोऽपि ददौ स्वभक्तिमुत्तमाम्} % ५७

\twolineshloka
{तदादिगाणपत्योऽहं जातः सर्वैश्च वन्दितः}
{भजस्व गणराजं तमतस्त्वं योगकाम्यया} % ५८

\uvacha{कण्व उवाच}

\twolineshloka
{एवमुक्त्वा महायोगी मह्यं मन्त्रं ददौ ततः}
{एकाक्षरं यथान्यायमन्तर्धानं चकार ह} % ५९

\twolineshloka
{अहं तथा गणेशानं साधयित्वा विशेषतः}
{शान्तिं प्राप्तस्तं तथापि भजाम्यनन्यचेतसा} % ६०

\twolineshloka
{अतस्त्वमपि राजेन्द्र भज विघ्नपतिं सदा}
{चतुर्थीव्रतसंयुक्तो ब्रह्मभूतो भविष्यसि} % ६१

\twolineshloka
{तत एकाक्षरं मन्त्रं ददौ तस्मै महामुनिः}
{कण्वं प्रणम्य राजर्षिर्ययौ स्वनगरे तदा} % ६२

\twolineshloka
{तस्मिन् काले चतुर्थी सा शुक्ला भाद्री समागता}
{कृता तेन सुभक्त्या वै नगरस्थजनैः सह} % ६३

\twolineshloka
{ततः सर्वत्र सङ्घोषः कृतस्तेन महीभृता}
{शुक्लां कृष्णां चतुर्थीं ये नाऽऽचरिष्यन्ति नित्यदा} % ६४

\twolineshloka
{ते दण्डैः पीडनीया वै ततः सर्वेऽभवन् जनाः}
{व्रतकारिण एतस्मात् पुण्यात् वृष्टिर्बभूव ह} % ६५

\twolineshloka
{हृष्टपुष्टजनाः सर्वे तया जाता नृपात्मज}
{रोगादिदोषहीनास्ते भजंस्तं गणनायकम्} % ६६

\twolineshloka
{भरतः स्म महाराजोऽभजत्सोऽनन्य चेतसा}
{गणेशं भक्तिसंयुक्तो गाणपत्यप्रियोऽभवत्} % ६७

\twolineshloka
{स पुत्रे राज्यमुग्रं तन्निक्षिप्य वनगोऽभवत्}
{अन्ते स्वानन्दगो भूत्वा ब्रह्मभूतो बभूव ह} % ६८

\twolineshloka
{क्रमेण भूमिसंस्था येऽभवन् स्वानन्दगा जनाः}
{चतुर्थी पुण्ययोगेन यथा योगपरायणाः} % ६९

\twolineshloka
{स्पर्शेन भरतस्यैव जनाः कीटादिका नृप}
{पुण्यरूपा बभूवुस्तेऽथान्ते स्वानन्दगामिनः} % ७०

\twolineshloka
{एतादृशेन भूपेन पुण्यशालिजनाः कृताः}
{यज्ञैः सर्वा धरा येन चित्रिता पुण्यकारिणा} % ७१

\twolineshloka
{भरतेन समो राजन् न कश्चित् प्रबभूव ह}
{ज्ञानेन स्वबलेनाऽपि यशसा धर्मशालिना} % ७२

\twolineshloka
{भाद्रशुक्लचतुर्थ्यास्तु महिमा कथितो मया}
{चतुर्वर्गफलैर्युक्तो ब्रह्मभूयपदप्रदः} % ७३

\twolineshloka
{भाद्रशुक्लचतुर्थ्यां तु माहात्म्यं यः श‍ृणोति चेत्}
{पठेद्वा तस्य राजेन्द्र सर्वदं प्रभवेद् ध्रुवम्} % ७४

\twolineshloka
{भाद्रशुक्लचतुर्थ्यां वै शङ्करस्य हृदि प्रभुः}
{प्रादुर्बभूव मध्याह्ने ध्यानजः स सुतोऽभवत्} % ७५

\twolineshloka
{तदादि सा तिथिर्मुख्या बभूव जन्मधारिणी}
{गणेशस्य न सन्देहो ब्रह्मभूयपदप्रदा} % ७६

\twolineshloka
{सृष्ट्यादौ पञ्च देवेशैः स्थापिता मूर्तिरुत्तमा}
{मयूरे गणनाथस्य मध्याह्ने भाद्रगे तिथौ} % ७७

\twolineshloka
{मयूरेशावतारो भाद्रपदेयो बभूव सः}
{मध्याह्ने शङ्करगृहे चतुर्थ्यां शुक्लपक्षके} % ७८

\twolineshloka
{तस्यां ये मृन्मयीं मूर्तिं पूजयन्ति नरादयः}
{देवाः शङ्करमुख्याश्च महोत्सवपरायणाः} % ७९

\twolineshloka
{ते सर्वे विघ्नहीनाश्च भवन्ति सुखभोगिनः}
{अन्ते स्वानन्दगा भूप ब्रह्मभूता भवन्ति च} % ८०

\twolineshloka
{मध्याह्ने पूजनं प्रोक्तं गणेशस्य विशेषतः}
{उपोषणसमायुक्तैश्चतुर्थ्यां व्रतकारिभिः} % ८१

\twolineshloka
{पञ्चम्यां पारणं कृत्वा द्विजैः सह महामते}
{मूर्तिं तां मृन्मयीं पूज्यां विसृज्य निनयेज्जलम्} % ८२

\twolineshloka
{चतुर्थ्यां मृन्मयीं मूर्तिं भाद्रे ये नार्चयन्ति चेत्}
{तेषां निष्फलरूपं वै कर्म सर्वं भविष्यति} % ८३

\twolineshloka
{न तेषां दर्शनं कार्यं नरैरात्महितेप्सुभिः}
{पतितास्ते मताः शास्त्रे नारकाश्च भवन्त्यतः} % ८४

\twolineshloka
{इह विघ्नसमायुक्ता नानारोगप्रपीडिताः}
{दारिद्र्यादिसमायुक्ता महापापा मता नृप} % ८५

\twolineshloka
{चतुर्थ्यां सर्ववर्णस्थैर्भाद्रे पूज्यो गजाननः}
{मृन्मयो विघ्नहीनास्ते भवन्ति सफलक्रियाः} % ८६

\twolineshloka
{इयं भाद्रपदे मासि चतुर्थी शुक्लरूपिणी}
{तस्याश्चरितमाद्यं ते कथितं स्वल्पभावतः} % ८७

\twolineshloka
{अत्र ते वर्णयिष्येऽहं इतिहासं पुरातनम्}
{तच्छृणुष्व महाभाग चतुर्थीव्रतजं महत्} % ८८

\twolineshloka
{द्राविडे नगरे राजंश्चाण्डालः कोऽपि पापकृत्}
{कुष्ठरोगयुतः पूर्णः परस्त्रीलम्पटोऽभवत्} % ८९

\twolineshloka
{चतुर्थ्यां भाद्रमासे स ज्वरयुक्तो बभूव ह}
{ज्वरस्य पीडयाऽत्यन्तं पीडितो राजसत्तम} % ९०

\twolineshloka
{अन्नेन स जलेनाऽपि हीनोऽभूद्दैवयोगतः}
{पञ्चम्यां मरणे प्राप्ते विमानेन जगाम ह} % ९१

\twolineshloka
{तस्याङ्गस्पर्शतो वायुर्यमलोके जगाम ह}
{तेन स्पृष्टा नरास्तत्र नरकस्थाः समन्ततः} % ९२

\twolineshloka
{ते सर्वे यानगा भूत्वा गताः स्वानन्दके पुरे}
{दृष्ट्वा गणपतिं तैः स ब्रह्मभूतो बभूव ह} % ९३

\twolineshloka
{व्रतमज्ञानतश्चैवं फलप्रदमिदं मतम्}
{यथाविधि कृतं येन तत्र चित्रं किमप्यहो} % ९४

\twolineshloka
{चतुर्थ्या महिमाऽयं कथयितुं न प्रशक्यते}
{पुरुषार्थाश्च चत्वारः प्राप्यन्ते व्रतमात्रतः} % ९५

\uvacha{मुद्गल उवाच}

\twolineshloka
{वसिष्ठवचनं श्रुत्वा दशरथस्तमब्रवीत्}
{श‍ृणु दक्ष महाभाग तां कथां पावनीं प्रभो} % ९६

\uvacha{दशरथ उवाच}

\twolineshloka
{चाण्डालो गणनाथस्य पुपूज स न मृन्मयीम्}
{मूर्तिं दोषी कथं स्वामिन् ब्रह्मभूतो बभूव ह} % ९७

\uvacha{वसिष्ठ उवाच}

\twolineshloka
{सम्यक् पृष्टं त्वया राजन् श‍ृणु संशयनाशनम्}
{ज्ञानं ते कथयिष्यामि भवेल्लोकोपकारदम्} % ९८

\twolineshloka
{चाण्डालस्य चतुर्थ्यास्तु ज्ञानं नाऽभूद् दुरात्मनः}
{पूजनं च तथा तस्योल्लङ्घनं न ततोऽभवत्} % ९९

\twolineshloka
{अतोऽयं दोषहीनश्च स्वानन्दस्थो बभूव ह}
{पूजनोल्लङ्घनाभ्यां स वर्जितो ज्ञानभावतः} % १००

\twolineshloka
{एवं नानाजनाश्चेह भुक्त्वा तु विविधं सुखम्}
{अन्ते स्वानन्दगा राजन् बभूवुर्व्रतमात्रतः} % १०१

\twolineshloka
{तत्रैकं कथितं प्रोक्तुं नालं वर्षायुतैरपि}
{पूर्णं भवति माहात्म्यं सङ्क्षेपेण निरूपितम्} % १०२

॥ॐ तत्सदिति श्रीमदान्त्ये पुराणोपनिषदि श्रीमन्मौद्गले महापुराणे चतुर्थे खण्डे गजाननचरिते भाद्रपदशुक्लचतुर्थीव्रतवर्णनं नाम सप्तमोऽध्यायः॥४.७॥


\sect{४.८ --- नामाष्टमोऽध्यायः --- आश्विनशुक्लचतुर्थीमाहात्म्यवर्णनम्}

\centerline{॥ श्रीगणेशाय नमः ॥}

\uvacha{वसिष्ठ उवाच}

\twolineshloka
{आश्विने वरदात्री या चतुर्थी शुक्लपक्षगा}
{तां श‍ृणुष्व महाभाग सव्रतां सर्वदायिनीम्} % १

\twolineshloka
{इतिहासं प्रवक्ष्यामि पुरातनभवं नृप}
{व्रतसंयुक्तमाहात्म्यं भवेत् सर्वार्थसाधकम्} % २

\twolineshloka
{रैवतान्तरगो राजा कीर्तिमांश्च बभूव ह}
{मतो नाम्ना धर्मधरः पूर्णशस्त्रास्त्रपारगः} % ३

\twolineshloka
{देवविप्रातिथिप्रेप्सुः पञ्चयज्ञपरायणः}
{नीतिज्ञः पुत्रवल्लोकान् पालयन् स्वहिते रतः} % ४

\twolineshloka
{भार्या तस्याऽभवत् साऽपि पातिव्रत्यगुणान्विता}
{सर्वलक्षणसंयुक्ता विप्रदेवातिथिप्रिया} % ५

\twolineshloka
{गजानां च हयानां वै पदातीनां महीपतेः}
{रथानां नैव सङ्ख्याऽस्ति धानुष्काणां विशेषतः} % ६

\twolineshloka
{सप्तद्वीपवतीं पृथ्वीं पालयन् स नराधिपः}
{देवादीनां च सङ्ग्रामे ह्यजेयः परवीरहा} % ७

\twolineshloka
{तस्य वन्ध्यत्वदोषेण नृप पुत्रो बभूव नो}
{नानायत्नपरो राजा पुत्रार्थे प्रबभूव ह} % ८

\twolineshloka
{तीर्थयात्रादिकं सर्वं चकार विधिवन्नृपः}
{अनुष्ठानव्रतादीनि देवानां पूजनं तथा} % ९

\twolineshloka
{एवं नानाविधैः पुण्यैर्न बभूव सुतस्ततः}
{राज्यं त्यक्त्वा वने राजा सस्त्रीकः स जगाम ह} % १०

\twolineshloka
{तत्र भ्रमणयुक्तः स ददर्श ह महावनम्}
{सिंहव्यालादिसंयुक्तं भयदं सर्वजन्मिनाम्} % ११

\twolineshloka
{दुःखयुक्तः स राजर्षिः प्रवेशं स चकार ह}
{वने तत्र मुनिश्रेष्ठं सौभरिं सन्ददर्श च} % १२

\twolineshloka
{तं प्रणम्य महाभागः सस्त्रीकः पुरतो मुनेः}
{कृताञ्जलिपुटो भूत्वा तस्थौ स नृपसम्मुखः} % १३

\twolineshloka
{ततः सौभरिणा सोऽपि सत्कृतो वचनेन च}
{निषसादासने तत्र मुनिना दर्शिते नृपः} % १४

\twolineshloka
{तमुवाच महाभागं राजानं मुनिसत्तमः}
{कोऽसि त्वं वन उग्रे मेऽत्र किमर्थं समागतः} % १५

\twolineshloka
{इति पृष्टो महीपालस्तमुवाच सुहर्षितः}
{कीर्तिमान् सर्वधर्मज्ञः कृत्वा करपुटं वचः} % १६

\uvacha{कीर्तिमानुवाच}

\twolineshloka
{द्राविडे वसतिर्मेऽस्ति नगरे सुरसत्तमे}
{राज्यं करोमि तत्राऽहं सार्वभौमो महामुने} % १७

\twolineshloka
{अपुत्रो दैवयोगेन जातोऽहं मुनिसत्तम}
{पुत्रार्थे व्रततीर्थादीन् नानाधर्मान् करोमि वै} % १८

\twolineshloka
{राज्यं त्यक्त्वा वने योगिन्नागतः पुत्रकाम्यया}
{तत्र ते दर्शनं प्राप्तं सर्वसिद्धिप्रदं प्रभो} % १९

\twolineshloka
{तव दर्शनमात्रेण सफलो मे भवो भवेत्}
{मातृपित्रादिकं सर्वं धन्यं जातं न संशयः} % २०

\twolineshloka
{अधुना ब्रूहि मे नाथ पुत्रप्राप्त्यर्थमादरात्}
{उपायं तं चरिष्यामि त्वदाज्ञावशगो मुने} % २१

\twolineshloka
{इह जन्मनि भो विप्र न कृतं पापमुल्बणम्}
{मया राज्यं कृतं भूमेर्भययुक्तेन चेतसा} % २२

\twolineshloka
{तथापि वन्ध्यजो दोषो मया प्राप्तो महामुने}
{पूर्वजन्मकृतं पापं ज्ञायते नैव चेतसा} % २३

\twolineshloka
{कथयस्व महोग्रं मे पापं सर्वविदां वर}
{योगीन्द्रोऽसि महातेजाः साक्षाद्ब्रह्मतनोर्धरः} % २४

\uvacha{वसिष्ठ उवाच}

\twolineshloka
{एवं विनययुक्तेन राज्ञा पृष्टो महामुनिः}
{सौभरिस्तं जगादेदं वचनं गणपप्रियः} % २५

\uvacha{सौभरिरुवाच}

\twolineshloka
{कृतं त्वया महत्पापं महाराज विशेषतः}
{पूर्वजन्मकृतं नैव पापं ते विद्यतेऽधम} % २६

\twolineshloka
{तव राज्ये महामूर्ख चतुर्थीव्रतमुत्तमम्}
{लयं प्राप्तं विशेषेण व्रतादौ फलदं मतम्} % २७

\twolineshloka
{चतुर्थीव्रतमाद्यं यन्मानवेन नराधम}
{न कृतं चेद्व्रतानीह निष्फलानि भवन्ति च} % २८

\twolineshloka
{विशेषतस्त्वया कर्म नानापुण्यादिकं कृतम्}
{चतुर्थीहीनभावेन निष्फलं तद्बभूव ह} % २९

\twolineshloka
{चतुर्विधपदार्थानां दात्री सा वरदा मता}
{चतुर्विधं तु सङ्कष्टं हरन्ती सङ्कटी मता} % ३०

\twolineshloka
{भुनक्ति राजा पापं राष्ट्रकृतं शास्त्रसम्मतम्}
{जनानां व्रतहीनानां पापभागी भवान् मतः} % ३१

\twolineshloka
{अतः पापमयी मूर्तिस्त्वमेवात्र न संशयः}
{तेन वन्ध्यत्वमापन्नो नराधम न बुद्ध्यसे} % ३२

\twolineshloka
{सौभरेर्वचनं श्रुत्वा कीर्तिमांस्तं जगाद ह}
{विनयेन समायुक्तो भयभीतश्च पार्थिवः} % ३३

\uvacha{कीर्तिमानुवाच}

\twolineshloka
{अज्ञानेन कृतं कर्म मया स्वामिन् सुपापिना}
{कीदृशं तद्व्रतं विप्र मह्यं वद विधानतः} % ३४

\twolineshloka
{पुत्रप्राप्त्यर्थमेवं मे वदोपायं महामते}
{येन पापविहीनोऽहं भवामि पुत्रवान् सुखी} % ३५

\uvacha{सौभरिरुवाच}

\twolineshloka
{चतुर्थीव्रतमाद्यं त्वं कुरुष्व नृप नित्यदा}
{जनैः सर्वैस्तदा सर्वपापहीनो भविष्यसि} % ३६

\twolineshloka
{अज्ञानेन करोषि स्म पापं ज्ञात्वाऽनुतापवान्}
{व्रताचरणमात्रेण निष्पापः पुण्यभाग् भवेः} % ३७

\twolineshloka
{इत्युक्त्वा व्रतमाहात्म्यं कथयामास विस्तरात्}
{ततः सोऽपि महाबुद्धिः पप्रच्छ विनयान्वितः} % ३८

\uvacha{कीर्तिमानुवाच}

\twolineshloka
{कीदृशोऽयं गणाधीशो व्रतं यस्य चतुःपदम्}
{ब्रह्मभूयकरं प्रोक्तं भजिष्यामि विशेषतः} % ३९

\twolineshloka
{ततस्तं मुनिशार्दूलः सौभरिः पुनरब्रवीत्}
{माहात्म्यं गणनाथस्य शान्तियोगपदप्रदम्} % ४०

\uvacha{सौभरिरुवाच}

\twolineshloka
{पुराऽहं तपसा युक्तो नानाछन्दपरायणः}
{अभवं तत्र देवा वै भयभीता बभूविरे} % ४१

\twolineshloka
{अहो तपःप्रभावेण जित्वा सर्वं द्विजोत्तमः}
{किमिच्छति पदं श्रेष्ठं ज्ञायतेऽस्माभिरेव न} % ४२

\twolineshloka
{प्रेषयामास सस्त्रीकं ततः कामं सुराधिपः}
{तपोभङ्गार्थमेवं मे कामस्तत्र समागतः} % ४३

\twolineshloka
{उर्वशीसहिताभिश्चाप्सरोभिर्मधुना तथा}
{आत्तबाणः स्वयं कामः पीडयामास मां शरैः} % ४४

\twolineshloka
{अहं तपःप्रभावेण जित्वा कामं सह स्त्रिया}
{मोहहीनस्तपस्तत्राऽतपं सुदृढनिश्चयः} % ४५

\twolineshloka
{ततो मे तपसोग्रेण दाहयुक्तो बभूव ह}
{कामः पलाय्य सर्वैस्तं मघवन्तं जगाद सः} % ४६

\twolineshloka
{ततोऽहं योगमार्गेणाऽन्तर्निष्ठश्चाभवन्नृप}
{जडोन्मत्तादिमार्गेषु संस्थितो योगकारणात्} % ४७

\twolineshloka
{ततः शुको महायोगी गाणपत्यः समागतः}
{ममाश्रमे स मां दृष्ट्वा जगादेच्छसि किं मुने} % ४८

\twolineshloka
{ततस्तं प्रणतो भूत्वा कृताञ्जलिः पुरः प्रभोः}
{स्थित्वाऽवदं सुवाक्यं तच्छृणु राजन् सुसिद्धिदम्} % ४९

\twolineshloka
{मम श्रेष्ठेन भाग्येन त्वं प्राप्तोऽसि महायशाः}
{शान्तिं वद महायोगिन् यया शान्तो भवाम्यहम्} % ५०

\uvacha{श्रीशुक उवाच}

\twolineshloka
{चित्तं पञ्चविधं त्यक्त्वा चित्तं कृत्वा च तन्मयम्}
{निरोधेनैव भूमीनां शान्तिं प्राप्स्यसि निश्चितम्} % ५१

\twolineshloka
{चिन्तामणिं भजस्व त्वं मन्त्रेणैकाक्षरेण च}
{तेन चिन्तामणौ विप्र सञ्चित्तः सुभविष्यसि} % ५२

\twolineshloka
{त्यक्त्वा जडादिकं मार्गं शमदमपरायणः}
{गणनाथं महाभाग भज यत्नेन नित्यदा} % ५३

\twolineshloka
{एवमुक्त्वा शुको योगी ययौ स्वेच्छापरायणः}
{गणेशनाम सङ्कीर्त्य जपंश्चैव विशेषतः} % ५४

\twolineshloka
{अहं गणपतिं भक्त्याऽभजं सम्भक्तिसंयुतः}
{एकाक्षरविधानेनाऽऽस्थाप्य मूर्तिं पुरो नृप} % ५५

\twolineshloka
{ततः स्वल्पेन कालेन शान्तिं प्राप्तोऽहमादरात्}
{तथापि पूजने सक्तोऽभजं तं गणनायकम्} % ५६

\twolineshloka
{दशवर्षे गते काले विघ्नेशो मां समागतः}
{मया सम्पूजितो राजन् स्तुतश्च विविधैः स्तवैः} % ५७

\twolineshloka
{गाणपत्यपदं दत्त्वा गतः स्वानन्दके पुरे}
{तदादि गाणपत्योऽहं भजामि ब्रह्मनायकम्} % ५८

\twolineshloka
{एवमुक्त्वा स राजानं ददौ मन्त्रं विधानतः}
{षडक्षरं स राजर्षिस्तं प्रणम्य ययौ पुरम्} % ५९

\twolineshloka
{स आश्विन्यां द्वितीयायां शुक्लायां तु गृहे गतः}
{तस्मिन् मासे चतुर्थ्यां च शुक्लायां व्रतमारभत्} % ६०

\twolineshloka
{जनैः सर्वैः समायुक्त उपोषणपरायणः}
{मध्याह्ने गणपं तत्र प्रपूज्य विधिवन्नृपः} % ६१

\twolineshloka
{रात्रौ जागरणं चक्रे बालवृद्धसमन्वितः}
{नरैः स्त्रीभिस्तद्व्रतं च कृतं सर्वैर्यथातथम्} % ६२

\twolineshloka
{शुक्लां कृष्णां चतुर्थीं ये न कुर्वन्ति नराधमाः}
{ताडनीयाः प्रयत्नेन पृथिव्यां यत्र तत्र सः} % ६३

\twolineshloka
{घोषेण घोषयामास ततः सर्वे तथाऽभवन्}
{व्रतं ततो वै बभूव प्रशस्तं भूमिमण्डले} % ६४

\twolineshloka
{एवं भूमण्डले राज्यं कृत्वा पुत्रे निवेद्य सः}
{वने गत्वा गणेशानं सस्त्रीको नृप आभजन्} % ६५

\twolineshloka
{अन्ते स्वानन्दगो भूत्वा ब्रह्मभूतो बभूव ह}
{तस्य राज्ये स्थिता लोकाः सर्वे स्वानन्दगा बभुः} % ६६

\twolineshloka
{एवमन्यं दशरथ श‍ृणुष्व त्वं व्रतोद्भवम्}
{इतिहासं प्रवक्ष्यामि चाश्विन्यां परमाद्भुतम्} % ६७

\twolineshloka
{भीमो नाम महाव्याधः पापकर्मपरायणः}
{मार्गे जनान् निहत्वाऽगृह्य धनं स तुतोष ह} % ६८

\twolineshloka
{एकदा वनमध्ये स ब्राह्मणं हन्तुमुद्यतः}
{पलायत द्विजस्तत्र वने भयसमाकुलः} % ६९

\twolineshloka
{एतस्मिन्नन्तरे तत्राश्वगः शस्त्रधरः पुमान्}
{धावंश्च ब्राह्मणं दृष्ट्वा भीमं धृत्वा गतः पुरे} % ७०

\twolineshloka
{ततो द्विजः सुखेनैव स्वाश्रमं प्रजगाम ह}
{पुरुषो भीमव्याधं तं राज्ञे दुष्टं न्यवेदयत्} % ७१

\twolineshloka
{तत्र सोऽपि क्षुधाविष्टो व्याधः संस्थापितोऽभवत्}
{राज्ञाऽऽश्विन्यां चतुर्थ्यां वै शुक्लायां दैवयोगतः} % ७२

\twolineshloka
{पञ्चम्यां तं जघानैव ततो व्याधं गजाननः}
{चतुर्थ्यां क्षुधितत्वात् स ब्रह्मभूतं चकार ह} % ७३

\twolineshloka
{एवं नानाजना राजन् चतुर्थीव्रतयोगतः}
{स्वानन्दस्था भवन्तीह मया वक्तुं न शक्यते} % ७४

\twolineshloka
{चतुर्थीजमिदं चित्रं चरितं कथितं मया}
{शुक्लाऽऽश्विन्यां समुद्भूतं श्रवणात् सर्वसिद्धिदम्} % ७५

\twolineshloka
{श‍ृणुयाद्यः पठेद्वाऽपि भुक्तिं मुक्तिं लभेन्नरः}
{पुत्रपौत्रादिसंयुक्तः सुहृद्भिर्नृपसत्तम} % ७६

॥ॐ तत्सदिति श्रीमदान्त्ये पुराणोपनिषदि श्रीमन्मौद्गले महापुराणे चतुर्थे खण्डे गजाननचरिते शुक्लाऽऽश्विनी चतुर्थी व्रतवर्णनं नामाष्टमोऽध्यायः॥४.८॥


\sect{४.९ --- नवमोऽध्यायः --- कार्तिकशुक्लचतुर्थीमाहात्म्यवर्णनम्}

\centerline{॥ श्रीगणेशाय नमः ॥}

\uvacha{वसिष्ठ उवाच}

\twolineshloka
{कार्तिके मासि शुक्ला या चतुर्थी सर्वसिद्धिदा}
{तां श‍ृणुष्व महाभाग इतिहाससमन्विताम्} % १

\twolineshloka
{सूर्यवंशोद्भवो राजा सुधन्वा नीतिसंयुतः}
{शस्त्रास्त्रबलसंयुक्तो बभूव परमद्युतिः} % २

\twolineshloka
{धर्मशीलो वदान्यश्च सत्यवाक् साधुसम्मतः}
{देवविप्रातिथिप्राज्ञपञ्चयज्ञपरायणः} % ३

\twolineshloka
{भार्या कलावती तस्य बभूवे रूपशालिनी}
{पतिव्रता महोदारा धर्मशीला विशेषतः} % ४

\twolineshloka
{जित्वा भूमण्डलं सर्वं राजा तेजस्विनां वरः}
{पालयामास पृथ्वीं स नित्यं धर्मपरायणः} % ५

\twolineshloka
{सामन्ता वशगा यस्य सैन्यं स्म गणनातिगम्}
{सम्पच्च धनदेनैव तुल्या सर्वत्र सम्बभौ} % ६

\twolineshloka
{अर्धायुषा समायुक्तो बभूव नृपसत्तमः}
{अकस्मात् कुष्ठसंयुक्तः कीटैः सम्पीडितोऽभवत्} % ७

\twolineshloka
{पूयशोणितघर्मौघैर्व्याप्तो दुर्गन्धिसंयुतः}
{न चाऽलभत् सुखं किञ्चिच्छूलप्रोतो यथा नरः} % ८

\twolineshloka
{औषधानि विशेषेण सिषेवे यत्नसंयुतः}
{नानामन्त्रप्रयोगादि कारयामास मानवैः} % ९

\twolineshloka
{अनुष्ठानं द्विजैः सोऽपि वेदमन्त्रैः सुखप्रदैः}
{अकारयत्तथा तेभ्यो न फलं चाऽभवत् कदा} % १०

\twolineshloka
{ततस्तीर्थानि बभ्राम स्नानदानपरायणः}
{तथापि रोगसंयुक्तोऽधिकं राजा बभूव ह} % ११

\twolineshloka
{ततो निवृत्तिमापन्नो जगाद सचिवान्नृपः}
{राज्यं मे परिपाल्यं वै यावदागमनं पुनः} % १२

\twolineshloka
{सान्त्वयित्वा स सस्त्रीकः सुहृदः सर्वनागरान्}
{वनं ययौ नृपश्रेष्ठो बभ्राम यत्र तत्र च} % १३

\twolineshloka
{ततो गणपतिं राजा सस्मार दुःखसंयुतः}
{विघ्नहीनार्थमेवं स तत्र चित्रं बभूव ह} % १४

\twolineshloka
{अकस्मान् मुनिशार्दूलः पुलस्त्यस्तत्र चाययौ}
{तं दृष्ट्वा हर्षसंयुक्तो ननाम प्रियया सह} % १५

\twolineshloka
{कृत्वा करपुटं राजोवाच तं मुनिनायकम्}
{किं पुण्यं मे पुरा चीर्णं येन दृष्टो भवान् मुने} % १६

\twolineshloka
{धन्यं जन्म तथा ज्ञानं जनको जननी च मे}
{तपो धर्मादिकं सर्वं त्वदङ्घ्रियुगदर्शनात्} % १७

\twolineshloka
{एवं विवदमानं तं जगाद मुनिसत्तमः}
{किमर्थं राजनीतिज्ञ वने त्वं च समागतः} % १८

\twolineshloka
{एवं पृष्ट्वा स राजानं वृक्षच्छायासमाश्रितः}
{पुलस्त्य उपविश्याथ तमुपावेश्य सम्बभौ} % १९

\twolineshloka
{ततो राज्ञा स्वकीयो वै वृत्तान्तः कथितोऽभवत्}
{जगाद प्रणनामैवं पुनस्तं हर्षसंयुतः} % २०

\uvacha{सुधन्वोवाच}

\twolineshloka
{दयाकराश्च योगीन्द्राः पुराणेषु वदन्ति यत्}
{तदेव सत्यमभवत् त्वां दृष्ट्वा दयया युतम्} % २१

\twolineshloka
{दुःखितं मां विदित्वा त्वं संस्थितो मुनिसत्तम}
{साक्षात् प्रजापतिः प्रोक्तः पुलस्त्यो ब्रह्मणः सुतः} % २२

\twolineshloka
{सर्वज्ञस्त्वं महायोगिन् न्यायं मे वद मानद}
{धर्मयुक्ततया राज्यं करोमि स्म निरन्तरम्} % २३

\twolineshloka
{पूर्वजन्मकृतं मे किं महापापं समागतम्}
{येनाऽहं कुष्ठसंयुक्तोऽभवं पश्य दयायुतः} % २४

\uvacha{वसिष्ठ उवाच}

\twolineshloka
{सुधन्वनो वचः श्रुत्वा तमुवाच महामुनिः}
{अत्यन्तं पीडितं दृष्ट्वा करुणायुतचेतसा} % २५

\uvacha{पुलस्त्य उवाच}

\twolineshloka
{इहजन्मकृतं पापं बुद्ध्यसे न नराधम}
{तेन कुष्ठयुतो जातः श‍ृणु तत्ते वदाम्यहम्} % २६

\twolineshloka
{तव राज्ये नृपश्रेष्ठ व्रतं गाणेश्वरं महत्}
{नष्टं चतुर्थीसंज्ञं यत् सर्वसिद्धिप्रदं परम्} % २७

\twolineshloka
{चतुर्णां पुरुषार्थानां साधनं सर्वसम्मतम्}
{तेन प्रोक्ता चतुर्थी सा वरदा सङ्कटा मता} % २८

\twolineshloka
{सर्वादौ न कृतं चेद्वै भवेत् सर्वं सुनिष्फलम्}
{कृतं कर्म नरेणाऽपि चतुर्वर्गविहीनकम्} % २९

\twolineshloka
{वर्णैः सर्वैः कृतं पापं राजानमुपतिष्ठति}
{तेन त्वं कुष्ठसंयुक्तोऽधुना जातो नराधम} % ३०

\twolineshloka
{मरिष्यसि यदा राजंस्तदा ते नरके गतिः}
{भविष्यति न सन्देहश्चतुर्वर्गविहीनता} % ३१

\twolineshloka
{पुलस्त्यवचनं श्रुत्वा दुःखयुक्तो महीपतिः}
{उवाच तं महाप्राज्ञं कृताञ्जलिपुटोऽभवत्} % ३२

\uvacha{सुधन्वोवाच}

\twolineshloka
{भगवन् सर्वतत्त्वज्ञ त्वया यत् कथितं वचः}
{तदेव सत्यरूपं वै मया ज्ञातं न संशयः} % ३३

\twolineshloka
{अधुना तद्व्रतं ब्रूहि कीदृशं कस्य पूजनम्}
{कस्मिन् काले प्रकर्तव्यं सर्वसिद्धिप्रदायकम्} % ३४

\twolineshloka
{कुष्ठनाशार्थमेवं मे वदोपायं महाप्रभो}
{प्रायश्चित्तं करिष्यामि व्रतलोपप्रदोषहृत्} % ३५

\twolineshloka
{एवं पृष्टो महायोगी पुलस्त्यो हर्षसंयुतः}
{तं जगाद गणेशाय नम इत्युपसंस्मरन्} % ३६

\uvacha{पुलस्त्य उवाच}

\twolineshloka
{अज्ञानेन कृतं दोषं प्रायश्चित्तेन हन्ति तम्}
{नरस्तस्मात्त्वमेवाशु व्रतं कुरु जनैः सह} % ३७

\twolineshloka
{तेन कुष्ठविहीनस्त्वं सुरूपः प्रभविष्यसि}
{अनुतापाच्च ते राजन् पापं नष्टं न संशयः} % ३८

\twolineshloka
{इत्युक्त्वा तं ततो योगी जगाद व्रतसम्भवाम्}
{कथां सर्वां स संश्रुत्य हर्षयुक्तो नृपोऽभवत्} % ३९

\twolineshloka
{उवाच तं मुनिश्रेष्ठं प्रणम्य च पुनः पुनः}
{धन्यं मे जन्म भो नाथ श्रुतं येन महद्व्रतम्} % ४०

\twolineshloka
{नानेन सदृशं किञ्चिन् मया ज्ञातं महामते}
{त्वत्तो वद महाप्राज्ञ गणेशस्य स्वरूपकम्} % ४१

\twolineshloka
{तज् ज्ञात्वा सर्वभावेन भजिष्यामि महामुने}
{नित्यं भक्तिसमायुक्तो देवदेवेशमादरात्} % ४२

\twolineshloka
{एवं पृष्टः स राजानं वचनं प्रजगाद ह}
{पुलस्त्यः सर्वभावज्ञो गाणपत्यो महायशाः} % ४३

\uvacha{पुलस्त्य उवाच}

\twolineshloka
{सुधन्वञ्छृणु मे वाक्यं गणेशज्ञानकारकम्}
{ब्रह्मभूयमयं पूर्णं योगाकारं विशेषतः} % ४४

\twolineshloka
{पुराऽहं योगशान्त्यर्थं नानायोगपरायणः}
{असाधयञ्छमेनैव दमेन मनसो जयात्} % ४५

\twolineshloka
{तथापि शान्तिहीनोऽहं शरणं शङ्करं गतः}
{तं प्रणम्य महात्मानमपृच्छं योगमुत्तमम्} % ४६

\twolineshloka
{ततस्तेन समाख्यातं तच्छृणुष्व नराधिप}
{येन त्वं गाणपत्यश्च साधनेन भविष्यसि} % ४७

\uvacha{श्रीशिव उवाच}

\twolineshloka
{योगशान्तिमयं विद्धि गणेशं भज भावतः}
{मनोवाणीविहीनं तं मनोवाणीमयं न च} % ४८

\twolineshloka
{मनोवाणीमयं सर्वं सम्प्रज्ञातसमुद्भवम्}
{गकाराक्षरगं विद्धि पश्य वेदे महामते} % ४९

\twolineshloka
{मनोवाणीविहीनं यदसम्प्रज्ञातगं मतम्}
{णकाराक्षरसम्भूतं नाम्नो गणपतेर्यदि} % ५०

\twolineshloka
{तयोः स्वामी गणेशानः शान्त्या योगेन लभ्यते}
{चित्तभूमिनिरोधेन तं भजस्व विनायकम्} % ५१

\twolineshloka
{एवमुक्त्वा महादेवो विरराम विशेषवित्}
{तं प्रणम्य वनं गत्वाऽसाधयं तं सुयत्नतः} % ५२

\twolineshloka
{अष्टाक्षरेण मन्त्रेण ध्यात्वा गणपतिं नृप}
{अतोषयं विशेषेण चित्तनिग्रहभावतः} % ५३

\twolineshloka
{ततः स्वल्पेन कालेन शान्तिं प्राप्तोऽहमात्मनि}
{तथापि मन्त्रराजं तमजपं पूजने रतः} % ५४

\twolineshloka
{एकविंशतिवर्षेषु गतेषु स विनायकः}
{आययौ मे वरं दातुं भक्तानुग्रहकारकः} % ५५

\twolineshloka
{तं दृष्ट्वा प्रणतो भूत्वाऽपूजयं तु यथाविधि}
{स्तौमि नामाष्टकेन स्म कौथुमेन महाप्रभुम्} % ५६

\twolineshloka
{गाणपत्यं स मां कृत्वा ययौ स्वानन्दके पुरे}
{तदारभ्याहमत्यन्तं भजामि गणनायकम्} % ५७

\twolineshloka
{एवमुक्त्वा महीपालं तथा दशरथ स्वयम्}
{मन्त्रमष्टाक्षरं तस्मै ददौ विधिसमन्वितम्} % ५८

\twolineshloka
{तेन स्तुतो महायोगी पुलस्त्योऽन्तर्दधे प्रभुः}
{राजा स्वनगरे गत्वा कार्तिके हर्षितोऽभवत्} % ५९

\twolineshloka
{जनैः सर्वैर्महाभागश्चकार व्रतमुत्तमम्}
{कार्तिके शुक्लपक्षस्य चतुर्थ्यां गणपं स्मरन्} % ६०

\twolineshloka
{पञ्चम्यां पारणं चक्रे राजाऽसौ जनसंयुतः}
{ब्राह्मणेभ्यो ददौ दानं सर्वान् अन्नैस्त्वतोषयत्} % ६१

\twolineshloka
{ततः कुष्ठविहीनश्च बभूव स जनाधिपः}
{सुरूपः कामदेवेन समः शोभाधरो बभौ} % ६२

\twolineshloka
{लोका वन्ध्यत्वदोषेण रोगादिभिः प्रपीडिताः}
{ते सर्वे दुःखहीनाश्च बभूवुर्व्रतसेवनात्} % ६३

\twolineshloka
{ततस्तेन नृपेणाऽथ सर्वत्र भूमिमण्डले}
{प्रकाशितं प्रयत्नेन व्रतं गाणेश्वरं नृप} % ६४

\twolineshloka
{ततः शुक्लां तथा कृष्णां चतुर्थीं चक्रिरे जनाः}
{तेनाऽऽनन्दसमायुक्ता बुभुजुर्विविधं सुखम्} % ६५

\twolineshloka
{ततः सुधन्वा स्थाप्य स्वं पुत्रं राज्ये महामतिः}
{एकान्ते संस्थितो भूत्वाऽभजत्तं गणपं सदा} % ६६

\twolineshloka
{अन्ते स्वानन्दगो भूत्वा ब्रह्मभूतो बभूव ह}
{तथा जनाश्च सर्वे ते स्वानन्दस्था बभूविरे} % ६७

\twolineshloka
{एवं ते कथितं राजन्नथो श‍ृणु महामते}
{महिमानं व्रतस्यैव सर्वसिद्धिकरस्य ह} % ६८

\twolineshloka
{माहिष्मत्यां च चाण्डालो वसन् कः पापकारकः}
{प्राप्य कार्तिकगां शुक्लां चतुर्थीं स वने गतः} % ६९

\twolineshloka
{तत्र व्याघ्रेण सन्दृष्टः पलायन् वृक्षमारुहत्}
{व्याघ्रो वृक्षतले तत्र संस्थितस्तं प्रतीक्षयन्} % ७०

\twolineshloka
{तत्र रात्रिर्गता तस्य चाण्डालस्य प्रजागरः}
{सम्पूर्णश्चाभवद्भूप पुनश्चित्रं बभूव ह} % ७१

\twolineshloka
{समागतो महासर्पो वनस्थो वृक्षमारुहत्}
{पपात भयभीतः स तं दृष्ट्वा कम्पवेगतः} % ७२

\twolineshloka
{व्याघ्रेण सङ्गृहीतः स पञ्चम्यां भक्षितोऽभवत्}
{स विमानं समारुह्य ययौ स्वानन्दकं पुरम्} % ७३

\twolineshloka
{अज्ञातव्रतजेनैव पुण्येन गणपं गतः}
{दृष्ट्वा योगपरो भूत्वा ब्रह्मभूतो बभूव सः} % ७४

\twolineshloka
{एवं नृप अपारा वै चतुर्वर्गफलैर्युताः}
{ब्रह्मभूताश्च सञ्जाताश्चतुर्थीव्रतकारकाः} % ७५

\twolineshloka
{कार्तिके शुक्लपक्षस्य चतुर्थ्या महिमा मया}
{कथितो लेशतो भूप श्रोतुमिच्छसि किं पुनः} % ७६

\twolineshloka
{श‍ृणुयाद्यः पठेद्वाऽपि स वै सर्वार्थसिद्धिभाक्}
{पुत्रपौत्रादिसंयुक्तः प्रभवेद्गणपप्रियः} % ७७

॥ॐ तत्सदिति श्रीमदान्त्ये पुराणोपनिषदि श्रीमन्मौद्गले महापुराणे चतुर्थे खण्डे गजाननचरिते कार्तिकशुक्लचतुर्थीवर्णनं नाम नवमोऽध्यायः॥४.९॥


\sect{४.१० --- दशमोऽध्यायः --- मार्गशीर्षशुक्लचतुर्थीमाहात्म्यवर्णनम्}

\centerline{॥ श्रीगणेशाय नमः ॥}

\uvacha{दशरथ उवाच}

\twolineshloka
{गणेशस्य कथां श्रुत्वा हर्षश्चेतसि जायते}
{मार्गशीर्षे च या शुक्ला तां मे वद मुने यतः} % १

\uvacha{वसिष्ठ उवाच}

\twolineshloka
{अत्र ते वर्णयिष्येऽहमितिहासं पुरातनम्}
{मार्गशीर्षे चतुर्थ्यां यः शुक्लायामभवन्नृप} % २

\twolineshloka
{काशीपतिः पुरा राजा पुण्यकीर्तिर्बभूव ह}
{अजातशत्रुको नाम सर्वशास्त्रविशारदः} % ३

\twolineshloka
{देवद्विजातिथिप्रेप्सुर्नानाधर्मपरायणः}
{प्रजानां पालने सक्तो यथाशास्त्रेण मानदः} % ४

\twolineshloka
{तत्रैव नारदोऽकस्मान्नृपं द्रष्टुं समागतः}
{तं प्रणम्य महात्मानं पूजयामास भक्तितः} % ५

\twolineshloka
{स्वयं पादस्य संवाहं चकार नृपसत्तमः}
{उवाच तं प्रहर्षेण नारदं सर्वगं परम्} % ६

\uvacha{अजातशत्रुरुवाच}

\twolineshloka
{धन्यं मे जन्म पुत्रादि राज्यं निहतकण्टकम्}
{धन्यौ च जनकौ ज्ञानं त्वदङ्घ्रियुगदर्शनात्} % ७

\twolineshloka
{सर्वसारं वदस्व त्वं योगशान्तिप्रदायकम्}
{येन संसारदुःखेभ्यो मुच्यते मानवः क्षणात्} % ८

\twolineshloka
{ततस्तं नारदो योगी गाणपत्याग्रणीर्नृप}
{जगाद हर्षसंयुक्तो वाक्यं सारमयं हसन्} % ९

\uvacha{नारद उवाच}

\twolineshloka
{सम्यक् पृष्टं त्वया राजन् सर्वेभ्यो ब्रह्मदं परम्}
{श‍ृणु ते कथयिष्यामि योगं शान्तिप्रदायकम्} % १०

\twolineshloka
{ब्रह्म नानाविधं वेदे वर्णितं पात्रभेदतः}
{न मुख्यं ब्रह्मभूतत्वं तदेव भवति प्रभो} % ११

\twolineshloka
{ब्रह्मणस्पतिनामानं गणेशं भज भावतः}
{चित्तवृत्तिनिरोधेन चिन्तामणिर्भविष्यसि} % १२

\twolineshloka
{गणेशोऽहं न सन्देहो मम तत्र कथं भवेत्}
{संयोगायोगकं राजन् तेन शान्तिमवाप्स्यसि} % १३

\twolineshloka
{तस्य व्रतं महाभाग चतुर्थीसंज्ञकं महत्}
{राज्ये नष्टे च तेन त्वं नारकी प्रभविष्यसि} % १४

\twolineshloka
{धर्मार्थकाममोक्षाणां प्रदं पूर्णं प्रकीर्तितम्}
{चतुर्थीसंज्ञकं राजन् वरदं सङ्कटं मतम्} % १५

\twolineshloka
{नानाकर्माणि कुर्वन्ति नराः सर्वार्थसिद्धये}
{चतुर्थीव्रतहीनाश्चेन्निष्फलाः प्रभवन्ति ते} % १६

\twolineshloka
{इत्युक्त्वा तं महायोगी नारदः करुणायुतः}
{माहात्म्यं कथयामास चतुर्थीसम्भवं नृप} % १७

\twolineshloka
{ततस्तं नृपवर्यः स प्रणम्य भावसंयुतः}
{पप्रच्छ सर्वमार्गज्ञं गणेशोपासनं पुनः} % १८

\uvacha{अजातशत्रुरुवाच}

\twolineshloka
{ब्रह्मणस्पतिमाहात्म्यं श्रुतं मया महामते}
{तस्योपासनमार्गं मे वद सर्वज्ञ ते नमः} % १९

\uvacha{नारद उवाच}

\twolineshloka
{एकाक्षरविधानेन भज ढुण्ढिं विनायकम्}
{तेन साध्यो गणेशस्ते प्रत्यक्षश्च भविष्यति} % २०

\twolineshloka
{तस्मै स विधिना सर्वं गणेशोपासनं ददौ}
{मन्त्रमेकाक्षरं साङ्गं ततश्चान्तर्हितोऽभवत्} % २१

\twolineshloka
{तत्राऽऽदौ मार्गशीर्षस्था सम्प्राप्ता शुक्लगा नृप}
{चतुर्थी सा कृता तेनोपोषणेन यथाविधि} % २२

\twolineshloka
{जनैः सर्वैस्तथा राजन् कृता सर्वप्रदायिनी}
{ततस्तेन च सर्वत्र प्रशस्ता सुकृताऽभवत्} % २३

\twolineshloka
{शुक्लां कृष्णां चतुर्थीं ये न कुर्वन्ति नराधमाः}
{स्त्रियश्चेत्ताडनीयास्ते महापापिन एव च} % २४

\twolineshloka
{ढुण्ढिराजं नृपाध्यक्षो नित्यं सम्पूज्य यत्नतः}
{भक्त्याऽभजत् सम्बभूव योगीन्द्रो योगिसम्मतः} % २५

\twolineshloka
{ततो बहुगते काले प्रत्यक्षः स गजाननः}
{बभूव तस्य भूपस्य वरं ब्रूहि तमब्रवीत्} % २६

\twolineshloka
{स्तुतः सम्पूजितस्तेन ढुण्ढिराजस्तुतोष ह}
{गाणपत्यं चकाराऽसौ नृपं चाजातशत्रुकम्} % २७

\twolineshloka
{ईप्सितं तं वरं दत्त्वा तत्रैवान्तरधीयत}
{ढुण्ढिराजं तमभजत् राजेन्द्रोऽनन्यचेतसा} % २८

\twolineshloka
{अन्ते नागरसंयुक्तो जगाम गणपं नृपः}
{सर्वैर्बभूव च ब्रह्मभूतो वै योगिसम्मतः} % २९

\twolineshloka
{एवं शुक्लचतुर्थ्यां ते मार्गशीर्षेऽभवन् महत्}
{माहात्म्यं कथितं राजन् सङ्क्षेपेण न संशयः} % ३०

\twolineshloka
{अन्यत् कथान्तरं भूप श‍ृणु सर्वभयापहम्}
{वेश्याया व्रतसंयोगाद्ब्रह्मभूयकरं महत्} % ३१

\twolineshloka
{मिथिलायां कदाचित् का वेश्या नरविमोहिनी}
{आगता तां निरीक्ष्यैव मोहिताः सकला नराः} % ३२

\twolineshloka
{राज्ञा सम्मानिताऽत्यन्तं तत्र वासमरोचयत्}
{कदाचित्तीर्थगा सा वै सन्दृष्टा रक्षसा पुरः} % ३३

\twolineshloka
{तां सङ्गृह्य ययौ रक्षः स्ववासं हर्षसंयुतः}
{सा तं दृष्ट्वा भयोद्विग्ना विललाप भृशातुरा} % ३४

\twolineshloka
{रक्षसा सान्त्विता तत्र न शोकं साऽत्यजत् कदा}
{तस्मिन् काले चतुर्थी वै शुक्ला मार्गे समागता} % ३५

\twolineshloka
{अतिशोकतया राजन्न बभक्ष जलादिकम्}
{पञ्चम्यां सा मृता तत्र भययुक्तेन चेतसा} % ३६

\twolineshloka
{ततो गणेशदूतेन नीता स्वानन्दके पुरे}
{ब्रह्मभूता च सा जाता व्रतपुण्यप्रभावतः} % ३७

\twolineshloka
{अज्ञानेन कृतं चैतत् वरदाख्यं व्रतं महत्}
{ब्रह्मसायुज्यदं प्रोक्तं किं पुनर्ज्ञानभावतः} % ३८

\twolineshloka
{मार्गशीर्षगतायाश्च शुक्लायाः कुरुते नरः}
{चतुर्थ्या लभते सोऽपि सर्वार्थं संश‍ृणोति यः} % ३९

॥ॐ तत्सदिति श्रीमदान्त्ये पुराणोपनिषदि श्रीमन्मौद्गले महापुराणे चतुर्थे खण्डे गजाननचरिते मार्गशीर्षशुक्लचतुर्थीवर्णनं नाम दशमोऽध्यायः॥४.१०॥


\sect{४.११ --- नामैकादशोऽध्यायः --- पौषशुक्लचतुर्थीमाहात्म्यवर्णनम्}

\centerline{॥ श्रीगणेशाय नमः ॥}

\uvacha{दशरथ उवाच}

\twolineshloka
{श्रुतं यन् मार्गशुक्लस्थचतुर्थी संज्ञितं मया}
{व्रतं तेन महाभाग सन्तृप्तो गुरुसत्तम} % १

\twolineshloka
{अधुना पौषमासे या चतुर्थी वरदायिनी}
{तस्या माहात्म्यमेवं मे ब्रूहि त्वं मुनिसत्तम} % २

\uvacha{वसिष्ठ उवाच}

\twolineshloka
{श‍ृणु राजंश्च माहात्म्यं सङ्क्षेपेण वदाम्यहम्}
{विस्तरेण तु को वक्तुं समर्थः प्रभवेद्भुवि} % ३

\twolineshloka
{अवन्तीनगरीमध्येऽवसत्तु ब्राह्मणोत्तमः}
{सुदन्त इति विख्यातः सर्वशास्त्रविशारदः} % ४

\twolineshloka
{राज्ञः पुरोहितः श्रेष्ठः सदा नीतिपरायणः}
{धर्मशास्त्रानुसारेण राजानमप्यबोधयत्} % ५

\twolineshloka
{राजा बृहद्रथो नाम तदाज्ञावशगोऽभवत्}
{पालयामास भूमिं स नानाधर्मकरः परः} % ६

\twolineshloka
{सुदन्तस्याऽभवद्भार्या नाम्ना ख्याता विलासिनी}
{बभूव कर्मदोषेण वन्ध्या सा वै पतिव्रता} % ७

\twolineshloka
{अपत्यं सुषुवे सा यज्जातमात्रं मृतं भवेत्}
{तदर्थं व्रतदानादिधर्मं चक्रे च स द्विजः} % ८

\twolineshloka
{नाऽभवत्तस्य तदपि पुत्रः परमसौख्यदः}
{विप्रोऽतिदुःखितः सस्त्रीको जगाम वनं ततः} % ९

\twolineshloka
{बभ्राम भ्रान्तचित्तोऽसौ यत्र तत्र महामतिः}
{मरणे निश्चयं कृत्वा तपोयुक्तो बभूव ह} % १०

\twolineshloka
{तत्राऽऽजगाम योगीन्द्रो वामदेवः प्रतापवान्}
{यदृच्छाविचरंस्तेन दृष्टः सन्नमितोऽभवत्} % ११

\twolineshloka
{पूजयित्वा महात्मानं वामदेवं कृताञ्जलिः}
{जगाद खेदसंयुक्तः सुदन्तो योगिसत्तमम्} % १२

\uvacha{सुदन्त उवाच}

\twolineshloka
{वामदेव च मे धन्यं दुःखितस्य तपो वयः}
{ज्ञानादिकं विशेषेण त्वत्पादपद्मदर्शनात्} % १३

\twolineshloka
{वन्ध्योऽहं मुनिशार्दूल का गतिर्मे भविष्यति}
{मृतस्य स्वर्गहीनस्य वद योगीन्द्रसत्तम} % १४

\uvacha{वामदेव उवाच}

\twolineshloka
{शुणु द्विज महाभाग त्वं सदा धर्मसंयुतः}
{तथापि पापचारी त्वं येन जातः श‍ृणुष्व तत्} % १५

\twolineshloka
{अवन्तीपुरपालस्य आदरात्त्वं पुरोहितः}
{राज्ञा कृतं महत्पापं यत्तदेव त्वया कृतम्} % १६

\twolineshloka
{चतुर्थीव्रतलोपश्च बभूवे भूमिमण्डले}
{चतुर्वर्गफलैर्हीना जाता भूवासिनो जनाः} % १७

\twolineshloka
{धर्मस्याऽऽचरणं पूर्णं कृतं राज्ञा तथा त्वया}
{निष्फलं व्रतलोपेन तेन त्वं दुःखितोऽधुना} % १८

\twolineshloka
{एवं श्रुत्वा वचो रम्यं वामदेवस्य धीमतः}
{विस्मितस्तं सुदन्तोऽसौ जगाद विनयान्वितः} % १९

\uvacha{सुदन्त उवाच}

\twolineshloka
{कीदृशं तद्व्रतं तात वद मे हितकारकम्}
{चतुर्णां पुरुषार्थानां साधकं कथमाभवत्} % २०

\twolineshloka
{तेन हीनो नरो यस्तु स कथं फलहीनकः}
{कर्मणस्तत् समाचक्ष्व दयासागर मानद} % २१

\uvacha{वसिष्ठ उवाच}

\twolineshloka
{एवं पृष्टो महायोगी तज्जगाद सविस्तरम्}
{माहात्म्यं सकलं तस्मै स श्रुत्वा विस्मितोऽभवत्} % २२

\twolineshloka
{पुनः पप्रच्छ तं विप्रो हर्षयुक्तेन चेतसा}
{गणेशज्ञानबोधार्थं गाणपत्यं महामुनिम्} % २३

\uvacha{सुदन्त उवाच}

\twolineshloka
{कीदृशोऽयं गणाधीशो वद तस्य स्वरूपकम्}
{ज्ञात्वा तं प्रभजिष्यामि नित्यं भक्तिसमन्वितः} % २४

\twolineshloka
{एवं पृष्टो महायोगी वामदेवस्तमब्रवीत्}
{गणेशबोधदाने स कुशलः सर्वपारगः} % २५

\uvacha{वामदेव उवाच}

\twolineshloka
{सुदन्त श‍ृणु विप्रर्षे गाणेशं ज्ञानमुत्तमम्}
{गाणपत्यो येन भक्तिभावितो भविताऽसि भोः} % २६

\twolineshloka
{अहं पुरा तपोनिष्ठस्त्वभवं यत्नसंयुतः}
{तपसा मे महाभाग व्याप्तं सर्वं चराचरम्} % २७

\twolineshloka
{ततो मया तपस्त्यक्तं योगार्थं ब्राह्मणोत्तम}
{शमे दमे परेणान्तर्निष्ठेन मनसो जयात्} % २८

\twolineshloka
{योगभूमिक्रमेणाऽहं कालेन महता द्विज}
{सहजे संस्थितो भूत्वा यत्र तत्राऽचरं तु च} % २९

\twolineshloka
{सहजं मोहहीनं यत् स्वाधीनत्वसमायुतम्}
{दृष्ट्वा शान्त्यर्थमत्यन्तं तत् त्यक्तं च मया ततः} % ३०

\twolineshloka
{सन्धृतं मनसि ब्रह्म मनोवाणीविवर्जितम्}
{कथं स्वाधीनता तत्र निर्मोहश्च प्रवर्तते} % ३१

\twolineshloka
{अधुना किं मया कार्यं विचार्य शरणं गतः}
{शङ्करं योगिवन्द्यं तं शैवोऽहं प्रणतोऽभवम्} % ३२

\twolineshloka
{शैवमार्गे रतं नित्यं दृष्ट्वा हर्षसमन्वितः}
{शम्भुः प्रोवाच मां विप्र स्थीयतां मुनिसत्तम} % ३३

\twolineshloka
{किमर्थमागतस्तात वामदेव महामते}
{वद मे सकलं वृत्तं करिष्यामि प्रियं च ते} % ३४

\twolineshloka
{शिवस्य वचनं श्रुत्वा संस्थितोऽहं कृताञ्जलिः}
{अवदं तं महादेवं भक्तवाञ्छासुरद्रुमम्} % ३५

\twolineshloka
{सहजं यत्परं ब्रह्म शैवं स्वेच्छामयं प्रभो}
{तस्मात् परं न विद्येत तथापि नुद संशयम्} % ३६

\twolineshloka
{ब्रह्मणि ब्रह्मभूतस्य स्वेच्छा तत्र कुतो भवेत्}
{स्वाधीनता दोषयुक्तं सहजं न परं मतम्} % ३७

\twolineshloka
{अतः शान्त्यर्थमेव त्वामहं प्रष्टुं समागतः}
{योगशान्तिप्रदं पूर्णं वद मां करुणानिधे} % ३८

\uvacha{वसिष्ठ उवाच}

\twolineshloka
{वामदेववचः श्रुत्वा हर्षितः शङ्करोऽब्रवीत्}
{तं पुनः सर्वयोगज्ञो गाणपत्यस्वभाववान्} % ३९

\uvacha{श्रीशिव उवाच}

\twolineshloka
{योगशान्तिप्रदं पूर्णं गणेशं विद्धि भो मुने}
{जानीहि न परं ब्रह्म सहजं योगसेवया} % ४०

\twolineshloka
{स्वानन्दाद्यत् समुत्पन्नमसत्यं सत्यरूपकम्}
{समं च सहजं विद्धि चतुर्धाऽसौ विभज्यते} % ४१

\twolineshloka
{चतुर्णामत्र संयोगे स्वानन्दः परिकीर्तितः}
{अयोगे नैव संयोगः केषाञ्चिद्ब्रह्मणां भवेत्} % ४२

\twolineshloka
{गकाराक्षरगं ज्ञानं विद्धि तन्निजबोधतः}
{णकाराक्षरगं ज्ञानं निवृत्या लभ्यते जनैः} % ४३

\twolineshloka
{तयोः स्वामी गणेशानो योगरूपः प्रकीर्तितः}
{शान्त्याऽसौ लभ्यते विप्र शान्तिभ्यः शान्तिदायकः} % ४४

\twolineshloka
{चित्तं पञ्चविधं विद्धि बुद्धिरूपं न संशयः}
{चित्ते मोहात्मिका सिद्धिर्माये ते परिकीर्तिते} % ४५

\twolineshloka
{तयोर्बिबं गणेशश्च बिम्बिभावं त्यज प्रभो}
{अधुना गणनाथस्त्वं भविष्यसि न संशयः} % ४६

\twolineshloka
{एवमुक्त्वा महादेवस्तस्मै मन्त्रं ददौ पुनः}
{एकाक्षरं गणेशस्य सविधिं करुणायुतः} % ४७

\twolineshloka
{तं प्रणम्य महेशानं वने यातोऽहमादरात्}
{तत्रैव गणनाथं तमभजं भक्तिसंयुतः} % ४८

\twolineshloka
{एकाक्षरविधानेन सन्तुष्टो गणनायकः}
{योगशान्तिं ददौ पूर्णां भक्तवात्सल्यकारणात्} % ४९

\twolineshloka
{ततोऽहं योगिवन्द्यश्च जातः सर्वत्र सम्मतः}
{तथापि विघ्नदहनमभजं नित्यमादरात्} % ५०

\twolineshloka
{गतेषु दशवर्षेषु गणाधीशः समाययौ}
{ममाश्रमं वरं दातुमुवाच घननिःस्वनः} % ५१

\uvacha{गणेश उवाच}

\twolineshloka
{वामदेव महाभाग वरं वृणु हृदीप्सितम्}
{तव दास्यामि भक्त्याऽहं सन्तुष्टो योगिसत्तम} % ५२

\twolineshloka
{गणेशस्य वचः श्रुत्वा त्यक्त्वा ध्यानं समुत्थितः}
{प्रणम्य तं गणेशं सम्पूज्य स्तोतुं प्रचक्रमे} % ५३

\uvacha{वामदेव उवाच}

\twolineshloka
{गणेशाय नमस्तुभ्यं सदा स्वानन्दवासिने}
{सिद्धिबुद्धिपते तुभ्यं विघ्नेशाय नमो नमः} % ५४

\twolineshloka
{मूषकारूढ हेरम्ब भक्तवाञ्छाप्रपूरक}
{ढुण्ढिराजाय ते देव रक्ष मां ते नमो नमः} % ५५

\twolineshloka
{आदिमध्यान्तहीनाय लम्बोदर नमोऽस्तु ते}
{आदिमध्यान्तरूपाय शङ्करप्रियसूनवे} % ५६

\twolineshloka
{नानामायाधरायैव मायिभ्यो मोहदायिने}
{मायामायिकभेदैस्त्वं क्रीडसे ते नमो नमः} % ५७

\twolineshloka
{विष्णुपुत्राय शेषस्य पुत्राय ब्राह्मणाय ते}
{ब्रह्मपुत्राय सर्वेश सर्वपुत्राय ते नमः} % ५८

\twolineshloka
{सर्वेषां चैव पित्रे ते मात्रे सर्वात्मकाय ते}
{महोदराय देवेन्द्रपाय ज्येष्ठाय वै नमः} % ५९

\twolineshloka
{महोग्राय महेशाय विष्णवे प्रभविष्णवे}
{अमृताय तु सूर्याय नानाशक्तिस्वरूपिणे} % ६०

\twolineshloka
{पुरुषाय प्रकृतये गुणेशाय गुणात्मने}
{एकानेकात्मकायैव विघ्नकर्त्रे नमो नमः} % ६१

\twolineshloka
{भक्तेभ्यः सर्वदात्रे ते ब्रह्मणां पतये नमः}
{योगाय योगनाथाय योगिनां पतये नमः} % ६२

\twolineshloka
{स्तौमि किं त्वां गणेशान मनोवाणीविहीनकम्}
{मनोवाणीमयं नैवातस्ते देव नमोऽस्तु ते} % ६३

\twolineshloka
{सहसैवं संस्तुवतस्तस्य भक्तिरसेन च}
{रोमोद्गमः प्रादुरासीत् कण्ठरोधो बभूव ह} % ६४

\twolineshloka
{उवाच वामदेवं स नृत्यन्तं विघ्ननायकः}
{वरं वृणु महाभाग यत्ते चित्ते स्थितं परम्} % ६५

\twolineshloka
{त्वया कृतमिदं स्तोत्रं सर्वसिद्धिकरं परम्}
{श‍ृणोति यः पठति चेत्तस्मै योगप्रदं तथा} % ६६

\twolineshloka
{भक्तिदं भक्तियुक्तेभ्यः पुत्रपौत्रादिकप्रदम्}
{धनधान्यप्रदं प्रोक्तं मयि प्रीतिविवर्धनम्} % ६७

\twolineshloka
{गणेशवचनं श्रुत्वा तं जगाद महामुनिः}
{वामदेवः प्रसन्नात्मा ब्रह्मेशं ब्रह्मभावितः} % ६८

\twolineshloka
{भक्तिं देहि गणाधीश गाणपत्यां विशेषतः}
{नान्यं याचे वरं देव यदि तुष्टोऽसि विघ्नप} % ६९

\twolineshloka
{तथेति तमुवाचैव गणेशोऽन्तर्दधे स्वयम्}
{वामदेवः प्रसन्नात्मा गाणपत्यो बभूव ह} % ७०

\twolineshloka
{तदादि शान्तिमापन्नस्त्वहं विप्र महामते}
{अतस्त्वं गणराजं तं भज शान्तिमवाप्स्यसि} % ७१

\twolineshloka
{एकाक्षरं महामन्त्रं सुदन्ताय ददौ ततः}
{सविधिं वामदेवः सोऽन्तर्धानमकरोन् मुनिः} % ७२

\twolineshloka
{सुदन्तो विस्मितो भूत्वा ययौ स्वस्थानमुत्तमम्}
{राज्ञा सम्मानितः सोप्यभजत्तं गणनायकम्} % ७३

\twolineshloka
{बृहद्रथाय वृत्तान्तं कथयामास भो नृप}
{तदाज्ञया नृपः सद्यो घोषयामास तद्व्रतम्} % ७४

\twolineshloka
{तत्रादौ पौषमासे या चतुर्थी शुक्लगाऽऽगता}
{तां चकार द्विजः सद्यो जनै राज्ञा पुरः स्थितैः} % ७५

\twolineshloka
{व्रताचरणमात्रेण गर्भयुक्ता बभूव ह}
{मुनिपत्नी सुतं लेभे ज्ञानयुक्तं चिरायुषम्} % ७६

\twolineshloka
{ततो भूमितले सर्वे चक्रुर्व्रतमनुत्तमम्}
{शौक्लं कार्ष्णं विशेषेण चतुर्थीसंज्ञितं नृप} % ७७

\twolineshloka
{सर्वे रोगादिभिर्हीना जाताः पुत्रादिसंयुताः}
{धनधान्यादिभिर्युक्ता अन्ते स्वानन्दगा बभुः} % ७८

\twolineshloka
{सुदन्तो योगिवन्द्यश्च बभूवे योगसेवया}
{राजाऽपि ज्ञानसंयुक्तो गाणपत्यो बभूव ह} % ७९

\twolineshloka
{पौषशुक्लचतुर्थीजमेतत्ते कथितं व्रतम्}
{राजन् सर्वार्थदं पूर्णं पुनस्त्वं श‍ृणु मानद} % ८०

\twolineshloka
{वैश्यो मार्गे स्थितः सोऽपि श्रमयुक्तो धनप्रियः}
{तत्र चोरैः समायातैर्लुण्ठितं तैर्धनं महत्} % ८१

\twolineshloka
{तेषां शस्त्राभिघातेन पपात धरणीतले}
{वैश्यो वने दुःखितश्च विललाप भृशातुरः} % ८२

\twolineshloka
{दैवयोगेन सा देवी चतुर्थी शुक्लगा गता}
{पौषी तस्यां जलाद्यैश्च हीनस्तत्र बभूव सः} % ८३

\twolineshloka
{रात्रौ जागरणं तस्य सञ्जातं पीडया तदा}
{पञ्चम्यां स मृतः पापी धनलुब्धो महामते} % ८४

\twolineshloka
{अज्ञातव्रतजेनैव महिम्ना सोऽपि भूपते}
{स्वानन्दे गणपं दृष्ट्वा ब्रह्मभूतो बभूव ह} % ८५

\twolineshloka
{एवं जना व्रतेनैव मुक्ताः संसारसागरात्}
{इह भुक्त्वाऽखिलान् भोगान् वर्णितुं तन्न शक्यते} % ८६

\twolineshloka
{इदं पौषचतुर्थ्यास्तु माहात्म्यं यः श‍ृणोति चेत्}
{पठेद्वै तस्य भो राजन् सर्वदं प्रभविष्यति} % ८७

॥ॐ तत्सदिति श्रीमदान्त्ये पुराणोपनिषदि श्रीमन्मौद्गले महापुराणे चतुर्थे खण्डे गजाननचरिते पौषशुक्लचतुर्थीमाहात्म्यवर्णनं नामैकादशोऽध्यायः॥४.११॥


\sect{४.१२ --- द्वादशोऽध्यायः --- माघशुक्लचतुर्थीमाहात्म्यवर्णनम्}

\centerline{॥ श्रीगणेशाय नमः ॥}

\uvacha{दशरथ उवाच}

\twolineshloka
{चतुर्थ्या महिमानं च श्रुत्वा हर्षः प्रवर्धते}
{न तृप्यामि महायोगिन्नमृतादधिकं मतम्} % १

\twolineshloka
{माघे शुक्लचतुर्थी या तस्या माहात्म्यमुत्तमम्}
{वद येन जनाः सर्वे भवन्ति सुखभोगिनः} % २

\uvacha{वसिष्ठ उवाच}

\twolineshloka
{माघी शुक्ला चतुर्थी या तस्यां जातो विनायकः}
{कश्यपस्य गृहे साक्षादङ्गारकयुता नृप} % ३

\uvacha{मुद्गल उवाच}

\twolineshloka
{एवमुक्त्वा वसिष्ठस्तं विनायकचरित्रकम्}
{कथयामास भो दक्ष देवान्तकवधाश्रितम्} % ४

\twolineshloka
{श्रुत्वा सोऽपि मुदा युक्तो बभूवाजस्य नन्दनः}
{पुनस्तं प्रेरयामास व्रतार्थं मुनिमादरात} % ५

\twolineshloka
{तस्यादरं स विज्ञाय वसिष्ठस्तमुवाच ह}
{हर्षेण महता युक्तो गाणपत्येन्द्रसत्तमः} % ६

\uvacha{वसिष्ठ उवाच}

\twolineshloka
{अत्र ते वर्णयिष्येऽहमितिहासं पुरातनम्}
{श्रुतश्चेत् सर्वदः पूर्णो भविष्यति नरोत्तम} % ७

\twolineshloka
{कर्णाटे भानुपुर्यां च राजा परमधार्मिकः}
{देवविप्रातिथिप्रेप्सुर्नीत्या राज्यं चकार सः} % ८

\twolineshloka
{शस्त्रास्त्रज्ञो विशेषेण नाम्ना सोमश्च वीर्यवान्}
{जित्वा भूमण्डलं सर्वं चक्रे सौराज्यमुत्तमम्} % ९

\twolineshloka
{भार्या यशोवती तस्य नाम्ना पूर्णपतिव्रता}
{धर्मशीला रता दाने बभूवे रूपशालिनी} % १०

\twolineshloka
{तस्याऽपि दीर्घकालेन कुर्वतो राज्यमुत्तमम्}
{अनावृष्टिभवं दुःखं प्राप्तं परमदारुणम्} % ११

\twolineshloka
{स तु शौनकनामानं मुनिं सर्वार्थकोविदम्}
{ययौ दुःखनिवृत्त्यर्थं गाणपत्यं वने पुरात्} % १२

\twolineshloka
{महावनं समासाद्य तं ननाम महामुनिम्}
{साष्टाङ्गं च पुरस्तस्य कृताञ्जलिपुटोऽभवत्} % १३

\twolineshloka
{उवाच तं मुनिश्रेष्ठं सोमो विनयसंयुतः}
{धन्यं मे जन्म कर्माद्यं येन दृष्टो भवान् मुने} % १४

\twolineshloka
{वदन्तमेवं राजानमुवाच शौनको मुनिः}
{सर्वशास्त्रार्थतत्त्वज्ञो न्यस्तस्वपरविभ्रमः} % १५

\uvacha{शौनक उवाच}

\twolineshloka
{पुरा राजन् महाभाग वद मां सकलं तव}
{चेष्टितं यद्वने कस्मादागतो दुर्गमे च मे} % १६

\twolineshloka
{एवं पृष्टो मुनीन्द्रेण सोमस्तं प्रत्युवाच ह}
{हर्षयुक्तेन चित्तेन मुनिं वेदज्ञमुत्तमम्} % १७

\uvacha{सोम उवाच}

\twolineshloka
{कर्णाटे भानुपुर्यां मे वसतिर्धर्ममिच्छतः}
{अनावृष्टिभवं दुःखं प्राप्तं तत्र महामुने} % १८

\twolineshloka
{धर्मेण नीतियुक्तेन मया राज्यं प्रपालितम्}
{तथापि पापयोगेन दुःखं प्राप्तं जनैः सह} % १९

\twolineshloka
{तत्रोपायं वदस्व त्वं साक्षाद्योगीश्वरो भवान्}
{राज्यं त्यक्त्वा वने तेऽद्य शरणं चागतो नृपः} % २०

\twolineshloka
{श्रुत्वा तस्य वचो रम्यं तं जगाद महीपतिम्}
{शौनको मुनिशार्दूलो विनयेन समन्वितम्} % २१

\uvacha{शौनक उवाच}

\twolineshloka
{श‍ृणु राजन् महत् पापं तव राज्ये बभूव ह}
{तेन रोगयुता लोका वन्ध्यतादोषसंयुताः} % २२

\twolineshloka
{तत्रापि यन् महद् दुःखमनावृष्टिसमुद्भवम्}
{सम्प्राप्तं कारणं तत्र वदामि त्वां नृपाधम} % २३

\twolineshloka
{चतुर्थीसंज्ञकं राजन् व्रतं नष्टं विशेषतः}
{शौक्लं कार्ष्णं तदर्थं त्वं यत्नयुक्तो भवाधुना} % २४

\twolineshloka
{नो चेद्वर्षसहस्रैस्त्वं सुखं न लभसे कदा}
{अतः सर्वत्र विख्यातं तद्व्रतं कुरु मानद} % २५

\twolineshloka
{इत्युक्त्वा व्रतमाहात्म्यं कथयामास विस्तरात्}
{नृपाय सोऽपि संश्रुत्य विस्मितो मानसेऽभवत्} % २६

\twolineshloka
{पुनः पप्रच्छ भावेन शौनकं मुनिसत्तमम्}
{वद ब्रह्मन् गणेशस्य स्वरूपं शान्तिदं परम्} % २७

\uvacha{शौनक उवाच}

\twolineshloka
{पुराऽहं तपसा युक्तो नृपाऽतिष्ठं स्वाश्रमे}
{अन्तर्निष्ठस्वभावेन ब्रह्मचिन्तनतत्परः} % २८

\twolineshloka
{ततोऽकस्मान् महायोगी भृगुरस्माकमेव सः}
{बीजभूतः समायात आश्रमेऽनुग्रहाय मे} % २९

\twolineshloka
{तं दृष्ट्वा सहसोत्थाय प्रणतोऽहं महामुनिम्}
{आसने समुपावेश्याऽपूजयं च स्वभक्तितः} % ३०

\twolineshloka
{कृताञ्जलिपुटं नम्रं पुरस्तस्य महात्मनः}
{संस्थितं मां ततो दृष्ट्वा सुविनीतमुवाच सः} % ३१

\uvacha{भृगुरुवाच}

\twolineshloka
{तात किं योगनिष्ठस्त्वं शान्तिं प्राप्तो वदस्व माम्}
{किमिच्छसि महाभाग वद तेऽहं ददामि तत्} % ३२

\twolineshloka
{अस्मत्कुले महाभाग भवान् साधुगुणान्वितः}
{तेनाऽहं तृप्तिमायातः पूरयिष्यामि वाञ्छितम्} % ३३

\twolineshloka
{एवं ब्रुवन्तमाद्यं तं मुनिमानम्य शौनकः}
{जगाद हर्षसंयुक्तो भृगुं योगतपोनिधिम्} % ३४

\uvacha{शौनक उवाच}

\twolineshloka
{योगशान्तिप्रदं पूर्णं ब्रूहि योगं महामुने}
{तेनाऽहं तृप्तिमत्यन्तं यास्यामि त्वदनुग्रहात्} % ३५

\uvacha{भृगुरुवाच}

\twolineshloka
{योगशान्तिमयं विद्धि गणेशं ब्रह्मनायकम्}
{तं भजस्व विधानेन तेन शान्तो भविष्यसि} % ३६

\twolineshloka
{चित्तं पञ्चविधं तात तत्र चिन्तामणिः स्थितः}
{तं ज्ञात्वा ब्रह्मभूताश्चाऽभवन् ब्रह्मादयः सुत} % ३७

\twolineshloka
{चित्तेन ज्ञायते यद्वै तत्र मोहः प्रवर्तते}
{मोहं चित्तं परित्यज्य भव चिन्तामणिः स्वयम्} % ३८

\twolineshloka
{मनोवाणीमयं चित्तं मनोवाणीविवर्जितम्}
{चित्तं जानीहि पुत्र त्वं तत्त्यक्त्वा च सुखी भव} % ३९

\twolineshloka
{एवमुक्त्वा गणेशस्य गणानां त्वा मनुं ददौ}
{शौनकाय महायोगी भृगुश्चान्तर्हितोऽभवत्} % ४०

\uvacha{शौनक उवाच}

\twolineshloka
{भृगुणैवमहं राजन् उपदिष्टोऽभजं परम्}
{गणेशं योगभावेन चित्तनिग्रहणे रतः} % ४१

\twolineshloka
{क्रमेण योगिवन्द्योऽहं जातस्तस्य महात्मनः}
{कृपया गणराजस्य तथापि स्म भजामि तम्} % ४२

\twolineshloka
{ततो मे वरदो जातो गणेशो भक्तवत्सलः}
{गाणपत्यं चकाराऽसौ तदहात् गणपोऽभवम्} % ४३

\twolineshloka
{एवमुक्त्वा स राजेन्द्रं गणानां त्वा मनुं ददौ}
{शौनको मुनिशार्दूलो विधियुक्तं विधानवित्} % ४४

\twolineshloka
{शौनकं प्रणनामाऽऽज्ञां गृहीत्वा राजसत्तमः}
{आययौ स्वगृहे तत्र प्रधानैरभिनन्दितः} % ४५

\twolineshloka
{नागरैः सह सोमश्च माघे शुक्लचतुर्थिका}
{सम्प्राप्ता साऽऽदिकाले स चकार व्रतमुत्तमम्} % ४६

\twolineshloka
{ततस्तेन प्रविख्यातं कृतं भूमितले नृप}
{शुक्लकृष्णचतुर्थीजं व्रतं चक्रुर्जना भुवि} % ४७

\twolineshloka
{व्रताचरणपुण्येन वृष्टिस्तत्र बभूव च}
{सर्वे हर्षयुता लोका गणेशभजने रताः} % ४८

\twolineshloka
{गणेशभजनं सोमश्चकार प्रेमसंयुतः}
{गणेशप्रीतये तत्राऽकरोद्देवालयं तथा} % ४९

\twolineshloka
{महामौल्यं नृपाध्यक्षः स्थापयामास विघ्नपम्}
{गणेशं वरदाख्यं सोऽपूजयन्नित्यमादरात्} % ५०

\twolineshloka
{पुत्रपौत्रयुतो राजा स चक्रे राज्यमुत्तमम्}
{लोकाः सर्वे सुखे मग्ना रोगवन्ध्यत्ववर्जिताः} % ५१

\twolineshloka
{स्वधर्मनिरता नित्यमभजन् गणनायकम्}
{न गणेशसमं किञ्चिद्धारयामासुरादरात्} % ५२

\twolineshloka
{सोमश्चान्ते गणेशानं ययौ लोकसमन्वितः}
{स्वानन्दे तं प्रणम्याऽसौ ब्रह्मभूतो बभूव ह} % ५३

\twolineshloka
{एतत्ते कथितं भूप तथाऽन्यच्छृणु सुन्दरम्}
{चरित्रं व्रतसम्भूतं सर्वपापप्रणाशनम्} % ५४

\twolineshloka
{कोऽप्यन्त्यजो विदर्भे वै कौण्डिन्ये नगरेऽवसत्}
{भानुनामा महापापी दुष्टकर्मपरायणः} % ५५

\twolineshloka
{परस्त्रीलम्पटोऽत्यन्तं मद्यद्यूतरतः सदा}
{मार्गे जनान् जघानाऽसौ द्रव्यलोभी विशेषतः} % ५६

\twolineshloka
{जीवं दृष्ट्वा दुष्टकर्मा जघान स च भूमिप}
{कार्यहीनतया पापी ब्राह्मणानां वधे रतः} % ५७

\twolineshloka
{इत्यादिदोषबाहुल्ययुक्तः परमदारुणः}
{कदाचिद्दैवयोगेन वने बभ्राम सोऽन्त्यजः} % ५८

\twolineshloka
{तस्मिन् दिने समायाता माघी शौक्ली चतुर्थिका}
{तत्रान्नजलहीनोऽयं पर्वतान्तरगोऽभवत्} % ५९

\twolineshloka
{न प्राप्तवान् वने किञ्चिद् दुष्टस्तेन सुदुःखितः}
{क्षुधार्तस्तृष्णया युक्तस्तत्रैव निशि संस्थितः} % ६०

\twolineshloka
{प्रभाते विमले सद्यः समुत्थाय गृहागतः}
{तृषितः स जलं तत्र पपावत्यन्तमादरात्} % ६१

\twolineshloka
{ततोऽकस्मात् सुदुःखेन वमित्वा तज्जलं नृप}
{ममार तं गृहीत्वा ते गाणेशा गणपं ययुः} % ६२

\twolineshloka
{गणेशदर्शनेनैव निष्पापो ज्ञानसंयुतः}
{सायुज्यं गणनाथस्य लेभे वै तत्प्रभावतः} % ६३

\twolineshloka
{अज्ञानव्रतजेनैव पुण्येनान्त्यजजातिजः}
{ब्रह्मभूतो महापापी बभूवे सूर्यवंशज} % ६४

\twolineshloka
{एतादृशी महापुण्या चतुर्थी शुक्लगा मता}
{माघी तस्याश्च माहात्म्यं कथितुं न क्षमो भवेत्} % ६५

\twolineshloka
{एवं नाना जना भूप ब्रह्मभूता बभूविरे}
{कथितुं न प्रशक्यं यच्चरित्रं तत्समुद्भवम्} % ६६

\twolineshloka
{माघे शुक्लचतुर्थ्यां यो महिमानं श‍ृणोति चेत्}
{पठेद्वा तस्य भो राजन् सिद्धिदं सुभविष्यति} % ६७

॥ॐ तत्सदिति श्रीमदान्त्ये पुराणोपनिषदि श्रीमन्मौद्गले महापुराणे चतुर्थे खण्डे गजाननचरिते माघशुक्लचतुर्थीमाहात्म्यवर्णनं नाम द्वादशोऽध्यायः॥४.१२॥


\sect{४.१३ --- त्रयोदशोऽध्यायः --- फाल्गुनशुक्लचतुर्थीमाहात्म्यवर्णनम्}

\centerline{॥ श्रीगणेशाय नमः ॥}

\uvacha{दशरथ उवाच}

\twolineshloka
{अधुना ब्रूहि मे विप्र माहात्म्यं फाल्गुनोद्भवम्}
{पुण्यं शुक्लचतुर्थीजं न तृप्यामि कथामृतात्} % १

\uvacha{वसिष्ठ उवाच}

\twolineshloka
{फाल्गुने शुक्लपक्षे या चतुर्थी वरदा भवेत्}
{तस्या माहात्म्यमाद्यं ते सङ्क्षेपेण ब्रवीम्यहम्} % २

\twolineshloka
{मालवे नगरं ख्यातं सर्वशोभासमन्वितम्}
{भारकं सर्वधर्मज्ञैर्लोकैस्तत् सङ्कुलं बभौ} % ३

\twolineshloka
{तत्र हेमाङ्गदो राजा चकार राज्यमुत्तमम्}
{जित्वा भूमण्डलं सर्वं शस्त्रास्त्रज्ञो विचक्षणः} % ४

\twolineshloka
{प्रधानै राजभिः सर्वैर्मान्योऽभूत् सर्वसम्मतः}
{पुत्रैस्तेजोयुतैर्युक्तो द्विजदेवातिथिप्रियः} % ५

\twolineshloka
{ततोऽकस्मात् स राजर्षिः शूलयुक्तो बभूव ह}
{उदरे शस्त्रसम्पातैरिव युक्तो रुरोद ह} % ६

\twolineshloka
{नानायत्नाः कृताः सर्वैः शूलनाशार्थमादरात्}
{मन्त्रैश्चौषधिभिः शूलं पिपीडातितरां नृपम्} % ७

\twolineshloka
{तीर्थदेवादिकं सोऽपि ब्राह्मणैर्वेदपारगैः}
{असेवत ततस्तस्य शूलमुग्रं बभूव वै} % ८

\twolineshloka
{ततः शूलेन राजाऽतिपीडितो दारुणेन वै}
{विषादिभिर्मतिं चक्रे देहत्यागे विशेषतः} % ९

\twolineshloka
{ततोऽकस्मान् महायोगी पर्वतः सहसाऽऽगतः}
{तस्य गेहे तं ननाम स राजा दुःखसंयुतः} % १०

\twolineshloka
{ब्राह्मणैः पूजयामास तं सर्वज्ञं विशेषतः}
{उवाच दुःखसंयुक्तः शूलपीडां नियम्य सः} % ११

\uvacha{हेमाङ्गद उवाच}

\twolineshloka
{धन्यं मे जन्म दानादि पिता माता कुलादिकम्}
{स्वधर्मपुत्रदारादि त्वदङ्घ्रियुगदर्शनात्} % १२

\twolineshloka
{ततोऽतिपीडया युक्तः पपात धरणीतले}
{रुरोद दारुणं राजा तत् दृष्ट्वा विस्मितो द्विजः} % १३

\twolineshloka
{पप्रच्छ सचिवान् विप्रो मुनीनां पर्वतो वरः}
{किमिदं दुःखमुग्रं वै राज्ञा प्राप्तं प्रकथ्यताम्} % १४

\uvacha{सचिवा ऊचुः}

\twolineshloka
{शूलमुग्रं बभूवास्यानिवार्यं त्वौषधादिभिः}
{तीर्थमन्त्रप्रयोगैश्च नानायत्नैर्महामुने} % १५

\twolineshloka
{श्रुत्वा तेषां वचः क्रूरं पर्वतो ध्यानमास्थितः}
{ज्ञात्वा पापं महायोगी राज्ञः सर्वानुवाच ह} % १६

\uvacha{पर्वत उवाच}

\twolineshloka
{श‍ृणु राजंस्त्वदीये यद्राज्ये पापं सुदारुणम्}
{वर्तते त्वं ततः पूर्णशूलयुक्तोऽसि साम्प्रतम्} % १७

\twolineshloka
{चतुर्थ्यास्ते व्रतं मुख्यं राज्ये नष्टं विशेषतः}
{तेन दोषसमूहेन सजनस्त्वं सुदुःखितः} % १८

\twolineshloka
{पर्वतस्य वचः श्रुत्वा राजा तं हर्षितोऽवदत्}
{कृत्वा करपुटं भूप विनयावनतो वचः} % १९

\uvacha{हेमाङ्गद उवाच}

\twolineshloka
{कीदृशं तद्व्रतं स्वामिन् केन सञ्चरितं पुरा}
{कस्यैव पूजनं तत्र वद मे सर्वमञ्जसा} % २०

\uvacha{पर्वत उवाच}

\twolineshloka
{सर्वसिद्धिकरं राजन् व्रतं गाणेश्वरं परम्}
{चतुर्णां पुरुषार्थानां साधनं सर्वसम्मतम्} % २१

\twolineshloka
{ततश्चतुर्थीमाहात्म्यं कथयामास विस्तरात्}
{तेन संहर्षितो राजा श्रुत्वा तं पुनरब्रवीत्} % २२

\twolineshloka
{कीदृशोऽयं गणाधीशो वद तस्य स्वरूपकम्}
{ज्ञात्वा तं देवदेवेशं भजिष्यामि विशेषतः} % २३

\uvacha{पर्वत उवाच}

\twolineshloka
{पुरा वृत्तं गणेशस्य चरित्रं यन्मया श्रुतम्}
{तदेव कथयिष्यामि निदाघात् सर्वदं परम्} % २४

\twolineshloka
{अहं तपसि सन्तिष्ठन्नानायत्नपरायणः}
{संव्याप्तं तपसा सर्वं मदीयेन महामते} % २५

\twolineshloka
{इन्द्रः प्रक्षुभितोऽत्यन्तं काममप्सरसा युतम्}
{प्रेषयामास विघ्नार्थं तपसो मे मदान्वितः} % २६

\twolineshloka
{ततः सोऽपि भयोद्विग्नः कामो दाहसमन्वितः}
{तपसस्तेजसा सत्यं पलायत स सैनिकः} % २७

\twolineshloka
{ततोऽहं तपसा युक्तोऽन्तर्निष्ठश्चाऽभवं नृप}
{तपस्त्यक्त्वा महाबाहो शमदमपरायणः} % २८

\twolineshloka
{नानाभूमिपरो जातस्ततोऽतिभाग्य गौरवात्}
{निदाघः सहसा तत्राऽऽगतोऽवधूतमार्गवित्} % २९

\twolineshloka
{दृष्ट्वा तं योगिनं पूर्णं प्रणतोऽहं विशेषतः}
{अपूजयं विधानेन ततस्तं स्म वदामि वै} % ३०

\twolineshloka
{धन्यो मे जनको माता तपो धन्यं जनुस्तथा}
{ब्रह्मभूयप्रदस्याऽपि पादपद्मस्य दर्शनात्} % ३१

\twolineshloka
{अधुना वद मे नाथ योगं शान्तिप्रदायकम्}
{येनाऽहं योगिवन्द्यश्च भवेयं साधनान् मुने} % ३२

\uvacha{वसिष्ठ उवाच}

\twolineshloka
{इति पृष्टो महातेजा निदाघस्तमुवाच ह}
{पर्वतं तपसा शुद्धं प्रहृष्टेनान्तरात्मना} % ३३

\uvacha{निदाघ उवाच}

\twolineshloka
{श‍ृणु वृत्तं मदीयं ते कथयामि पुरातनम्}
{येन योगीन्द्र वन्द्यश्च जातोऽहं योगसेवया} % ३४

\twolineshloka
{अहं योगार्थमत्यन्तं योगभूमिपरायणः}
{असाधयं महायोगं ब्रह्मभूयकरं परम्} % ३५

\twolineshloka
{ततोऽन्तेऽहं सदानन्दे समः स्वानन्दकेऽभवम्}
{संस्थितस्तत्र भो राजन् दृष्टमानन्दरूपकम्} % ३६

\twolineshloka
{द्वन्द्वैः सर्वत्र संव्याप्तं द्वन्द्वे तद् योगरूपि च}
{तेन शान्तिर्गता मेऽभूत्ततो विष्णुं गतोऽभवम्} % ३७

\twolineshloka
{तं प्रणम्य महात्मानमवदं वद शान्तिदम्}
{योगं विष्णो महायोगिन् येन शान्तो भवाम्यहम्} % ३८

\uvacha{विष्णुरुवाच}

\twolineshloka
{योगशान्तिप्रदं तात गणेशं विद्धि मानद}
{तं ज्ञात्वा शान्तिमापन्ना वयं सर्वे च योगिनः} % ३९

\twolineshloka
{मनोवाणीमयं विद्धि गकारं च तयोः परम्}
{मनोवाणीविहीनं तं णकारं योगिसम्मतम्} % ४०

\twolineshloka
{तयोः स्वामी गणाधीशस्तं भजस्व विधानतः}
{ततः शान्तिं महाभाग लभसे नाऽत्र संशयः} % ४१

\twolineshloka
{एवमुक्त्वा महाविष्णुर्विरराम स्वयं ततः}
{तं प्रणम्य वने गत्वा योगाभ्यासपरोऽभवम्} % ४२

\twolineshloka
{एकाक्षरस्य मन्त्रस्य सन्तुष्टोऽभूद्गजाननः}
{गतेषु दश वर्षेषु जपेन सहसाऽऽगतः} % ४३

\twolineshloka
{ध्यानस्थं मां समालोक्य जगाद वृणु वाञ्छितम्}
{ततोऽहं तं प्रणम्यैवापूज्य स्तोतुं प्रचक्रमे} % ४४

\twolineshloka
{स्तोत्रं यत् सामवेदोक्तं तच्छ्रुत्वा तुष्टिमागतः}
{योगशान्तिं गणेशानो दत्त्वा मे प्रजगाम ह} % ४५

\twolineshloka
{तदादि गाणपत्योऽहं जातस्तं प्रभजामि वै}
{भजस्व गणराजं त्वमतस्तं शान्तिदं परम्} % ४६

\twolineshloka
{एवमुक्त्वा निदाघः सं ददौ तस्मै महामनुम्}
{एकाक्षरं विधानेन ततः सोऽन्तर्हितोऽभवत्} % ४७

\uvacha{पर्वत उवाच}

\twolineshloka
{भजामि गणनाथं स्म ततोऽहं नित्यमादरात्}
{तेनोक्तविधिना राजन् शान्तिं प्राप्तो विशेषतः} % ४८

\twolineshloka
{अपूजयं हि गणपं ततो नित्यं तपोन्वितः}
{ततो मां दर्शयामास रूपं योगमयं प्रभुः} % ४९

\twolineshloka
{स्तुतः सम्पूजितो मे स भक्तिं दत्त्वा महामते}
{स्वानन्दे स गतो राजन् भक्तानन्दविवर्धनः} % ५०

\twolineshloka
{एवमुक्त्वा पर्वतस्तं ददौ मन्त्रं दशाक्षरम्}
{गणेशस्य महाराज ततश्चान्तर्हितोऽभवत्} % ५१

\twolineshloka
{राजा सर्वजनैरादौ व्रतं चक्रे स हर्षितः}
{फाल्गुने भूप शुक्लस्थचतुर्थीसंज्ञकं परम्} % ५२

\twolineshloka
{ततो राज्ञाऽपि सर्वत्र कृतो घोषो जनैर्नृप}
{शौक्लं कार्ष्णं व्रतं भूमौ क्रियतां भावसंयुतैः} % ५३

\twolineshloka
{ततः शुलव्यथाहीनो बभूवे राजसत्तमः}
{दुःखहीना जनाः सर्वे जाता व्रतप्रभावतः} % ५४

\twolineshloka
{पुत्रपौत्रादिसंयुक्ता रोगाद्यभिविवर्जिताः}
{धनधान्यादिभिः सर्वे मुमुदुर्भूमिमण्डले} % ५५

\twolineshloka
{ततो हेमाङ्गदो राजा भक्त्या विघ्नेशमुत्तमैः}
{उपचारैः प्रपूज्याऽपि सिषेवे नित्यमादरात्} % ५६

\twolineshloka
{पुत्रं राज्ये निधायाऽसौ सस्त्रीकः शान्तिसंयुतः}
{गणेशं सर्वभावेनान्ते च तन्मयतामयात्} % ५७

\twolineshloka
{क्रमेण भूमिसंस्था ये नराः स्वानन्दगा बभुः}
{भुक्त्वा भोगान् विशेषेण व्रतपुण्यप्रभावतः} % ५८

\twolineshloka
{अधुना श‍ृणु चान्यत्त्वं माहात्म्यं व्रतसंश्रितम्}
{भाकरे नगरे विप्रो वसद्वै जातिदूषणः} % ५९

\twolineshloka
{बाल्यात्प्रभृति तेनैव कृतं कर्म सुदुःखदम्}
{हिंसादिसंयुतं राजन् चौर्यं क्लेशविवर्धनम्} % ६०

\twolineshloka
{परस्त्रीलम्पटो नित्यं यौवने सोऽतिदारुणः}
{पतिव्रता हठेनैव व्रतभ्रष्टाश्चकार ह} % ६१

\twolineshloka
{शस्त्रधारी वने गत्वा जन्तून् जघ्ने स नित्यशः}
{द्विजादीन् द्रव्यलुब्धश्च मानवान् मनुजप्रिय} % ६२

\twolineshloka
{एवं पापसमाचारो वने कस्मिन् दिने स्थितः}
{वैश्यं दृष्ट्वा च तं हन्तुमधावत् स वधप्रियः} % ६३

\twolineshloka
{पलता वैश्यपुत्रेण नादस्तत्र कृतो महान्}
{तं श्रुत्वा मार्गसंस्थाश्च चत्वारः पुरुषाऽऽययुः} % ६४

\twolineshloka
{ते धृत्वा राजदूतास्तं ताडयामासुरुल्बणम्}
{राजानं दर्शयामासुश्चौरं बद्धं जनास्ततः} % ६५

\twolineshloka
{राज्ञाऽऽज्ञप्ताश्च तं तत्राताडयंस्ते द्विजाधमम्}
{नाविदन् ब्राह्मणं राजंश्चिक्षिपुर्निगडे द्विजम्} % ६६

\twolineshloka
{ततः फाल्गुनसंस्था या चतुर्थी शुक्लगाऽऽगता}
{बद्धस्तत्र निराहारो बभूवे स नराधमः} % ६७

\twolineshloka
{पञ्चम्यां तं मृतं ज्ञात्वा ततो गाणेशकाऽऽययुः}
{दूता नेतुं द्विजं तत्र व्रतपुण्यप्रभावतः} % ६८

\twolineshloka
{स स्वानन्दपुरे नीतो गाणपत्यैर्नृपात्मज}
{तत्र विघ्नेश्वरं दृष्ट्वा ब्रह्मभूतो बभूव ह} % ६९

\twolineshloka
{एवं नाना जनाश्चैव ब्रह्मभूता बभूविरे}
{तेषां चरित्रकं सर्वं वक्तुं ते न प्रशक्यते} % ७०

\twolineshloka
{फाल्गुने वरदायास्तु चरित्रं यः श‍ृणोति चेत्}
{पठेद्वा तस्य विघ्नेशः सर्वान् कामान् ददाति हि} % ७१

॥ॐ तत्सदिति श्रीमदान्त्ये पुराणोपनिषदि श्रीमन्मौद्गले महापुराणे चतुर्थे खण्डे गजाननचरिते फाल्गुनशुक्लचतुर्थीचरित्रवर्णनं नाम त्रयोदशोऽध्यायः॥४.१३॥


\sect{४.१४ --- चतुर्दशोऽध्यायः --- चैत्रशुक्लचतुर्थीमाहात्म्यवर्णनम्}

\centerline{॥ श्रीगणेशाय नमः ॥}

\uvacha{दशरथ उवाच}

\twolineshloka
{श्रुतं चरित्रं मुख्यं यच्चतुर्थ्याः फाल्गुने मया}
{शुक्लायाः सर्वदं योगिन् न तृप्तोऽहं भवामि तु} % १

\twolineshloka
{अतो मां चैत्रगाया यद्वरदायाश्चरित्रकम्}
{गुरो सुदयया ब्रूहि पावनं सर्वजन्मिनाम्} % २

\uvacha{वसिष्ठ उवाच}

\twolineshloka
{बङ्गालं नगरं ख्यातं शोभाभद्राख्यमुत्तमम्}
{तत्र चन्द्रप्रियो राजा प्रचक्रे राज्यमुत्तमम्} % ३

\twolineshloka
{शस्त्रास्त्रैः शत्रुसङ्घातान् जित्वा वीरश्रिया युतः}
{सागरान्तां धरित्रीं स चकार वशवर्तिनीम्} % ४

\twolineshloka
{सामन्ता वशगाः सर्वे चक्रुः सेवादिकं नृपाः}
{तस्य राज्ये प्रजाः सर्वाः स्वस्वधर्मपरायणाः} % ५

\twolineshloka
{स्वयं धर्मपरो राजा नीत्या वर्णाश्रमादिषु}
{संस्थितान् रक्षयंश्चैव सदा हीनानदण्डयत्} % ६

\twolineshloka
{चौराणां न भयं तत्र स्वस्वव्यापारकारिणाम्}
{तथापि रोगसंयुक्ता नरा नार्योऽभवन् हि वै} % ७

\twolineshloka
{वन्ध्यादिदोषसंयुक्ता जनाः सर्वे समाययुः}
{राजानं सर्वनीतिज्ञं पप्रच्छुश्चेष्टितं महत्} % ८

\uvacha{नागरादय ऊचुः}

\twolineshloka
{वयं राजन् स्वधर्मस्था त्वदाज्ञावशवर्तिनः}
{अपि त्वं धर्मसंयुक्तः प्रधानादिभिरादरात्} % ९

\twolineshloka
{तथापि रोगदोषैश्च वन्ध्यदोषैर्विशेषतः}
{धनधान्यविहीनाश्च पीडिता वयमुत्कटम्} % १०

\twolineshloka
{पृथिवी रसहीना वै फलहीना महीरुहाः}
{गावो दुग्धविहीनाश्च तव राज्ये महामते} % ११

\twolineshloka
{धर्मशीले नृपे राजन् सर्वे सुखयुता जनाः}
{विपरीतं महाराज त्वयि राज्यं प्रशासति} % १२

\twolineshloka
{अतोऽस्मान् रक्ष भूपाल रोगादिभिः प्रपीडितान्}
{अस्माकं बलमत्यन्तं त्वमेव परमं मतम्} % १३

\uvacha{वसिष्ठ उवाच}

\twolineshloka
{चन्द्रप्रियो वचः श्रुत्वा तेषामेवं सुदुःखितः}
{प्रधानेषु समाक्षिप्य राज्यं निहतकण्टकम्} % १४

\twolineshloka
{वने गन्तुं मनश्चक्रे ततस्तत्राऽऽजगाम ह}
{अष्टावक्रो महायोगी तं दृष्ट्वा प्रणनाम सः} % १५

\twolineshloka
{सम्पूज्य भोजयामास पादसंवाहने रतः}
{स्वयं स्थितो महाविप्रमूचेऽसौ दुःखसंयुतः} % १६

\uvacha{चन्द्रप्रिय उवाच}

\twolineshloka
{धन्यं मे जन्म कर्मादि पिता माता तपो यशः}
{दानं ज्ञानं तथा योगिन् त्वत्पादयुगदर्शनात्} % १७

\twolineshloka
{ततः सर्वं स्ववृत्तान्तं कथयामास यत्नतः}
{अष्टावक्रश्च संश्रुत्य तं जगाद नराधिपम्} % १८

\uvacha{अष्टावक्र उवाच}

\twolineshloka
{राजन् राज्ये त्वदीये यत् महत्पापं प्रवर्तते}
{तेन दुःखयुता सर्वा प्रजा जाता न संशयः} % १९

\twolineshloka
{चतुर्णां पुरुषार्थानां प्रापकं यन्महाव्रतम्}
{चतुर्थीसंज्ञकं नष्टं तव राज्ये नृपाधम} % २०

\twolineshloka
{पुरुषार्थैर्नरास्तेन हीनाः सर्वेऽधुना परम्}
{अन्ते नरकगा राजन् भविष्यन्ति त्वया सह} % २१

\twolineshloka
{अष्टावक्रवचः श्रुत्वा तं प्रणम्य महीपतिः}
{उवाच दुःखसंयुक्तो वचनं स्वहितप्रदम्} % २२

\uvacha{चन्द्रप्रिय उवाच}

\twolineshloka
{ब्रह्मन् दयानिधे स्वामिन् वद मे व्रतमुत्तमम्}
{कीदृशं कस्य देवस्य प्रियमित्यादिकं प्रभो} % २३

\uvacha{अष्टावक्र उवाच}

\twolineshloka
{कृष्णशुक्लचतुर्थीजं व्रतं गाणेश्वरं नृप}
{सर्वादौ तत् प्रकर्तव्यं धर्मकामार्थमुक्तये} % २४

\twolineshloka
{इत्युक्त्वा कथयामास चरित्रं सकलं द्विजः}
{चतुर्थीसम्भवं तात श्रुत्वा राजा सुविस्मितः} % २५

\twolineshloka
{पुनः पप्रच्छ तं विप्रं विनयेन समन्वितः}
{कीदृशो गणराजोऽयं व्रतं यस्य महाद्भुतम्} % २६

\twolineshloka
{वद तस्य स्वरूपं मे ज्ञात्वा तं सर्वभावतः}
{व्रतयुक्तो भजिष्यामि गणेशं सर्वसिद्धिदम्} % २७

\uvacha{अष्टावक्र उवाच}

\twolineshloka
{गणेशस्य स्वरूपं तु मया वक्तुं न शक्यते}
{तथापि श‍ृणु भूपाल येन त्वं ज्ञास्यसि प्रभुम्} % २८

\twolineshloka
{पुराऽहं तपसि प्राज्ञ संस्थितो यत्नधारकः}
{सुरूपका अप्सरसो याताश्चलयितुं ततः} % २९

\twolineshloka
{तासामत्याग्रहं दृष्ट्वा कुपितः स्म शपामि ताः}
{मृत्युलोके पतध्वं वै चौरग्रस्ता भवेत हि} % ३०

\twolineshloka
{इति मद्गिरमाकर्ण्य भयभीताः समन्ततः}
{ता मां प्रसादयामासुस्ततोऽहं दयया युतः} % ३१

\twolineshloka
{अवदं तत्र विष्णुर्वै यादवेषु भविष्यति}
{तस्य पत्न्यो भविष्यन्त्योऽन्ते चौराणां भविष्यथ} % ३२

\twolineshloka
{गता अप्सरसो राजंस्ततोऽहं तप आचरम्}
{प्रभावेणातितपसोऽन्तर्ज्ञानं मेऽभवत् परम्} % ३३

\twolineshloka
{ततोऽहं तप उत्सृज्य शमी दमपरोऽभवम्}
{नानायोगविचारेण समाधिं साधयन्नृप} % ३४

\twolineshloka
{एवमन्ते समानन्दे संस्थितोऽहं विशेषतः}
{तत्र मोहं समालोक्य समरूपे सुविस्मितः} % ३५

\twolineshloka
{ततः शान्त्यर्थमत्यन्तं क्लिश्यामि स्म महामते}
{ततोऽकस्मान् महायोगी ऋभुस्तत्र समागतः} % ३६

\twolineshloka
{वर्णाश्रमविहीनं तं दृष्ट्वा हर्षसमन्वितः}
{प्रणम्याऽहं विशेषेणापूज्योवाचं महामुनिम्} % ३७

\twolineshloka
{धन्यं मे तप उग्रं यज्जन्म ज्ञानादिकं तथा}
{तव पादस्य योगेन कृतकृत्योऽस्मि साम्प्रतम्} % ३८

\twolineshloka
{एवं मदीयं वाक्यं स श्रुत्वोवाच महामुनिः}
{किं वाञ्छसि महाप्राज्ञ वदस्वाऽहं करोमि तत्} % ३९

\twolineshloka
{ततस्तं प्रणतो भूत्वा कृताञ्जलिपुटः पुनः}
{अवदं योगशान्तिं मे वद योगीन्द्रसत्तम} % ४०

\uvacha{ऋभुरुवाच}

\twolineshloka
{संयोगः पञ्चधा तात सदसत्समनेतितः}
{स्वसंवेद्यमयेनैव मुने योगेन लभ्यते} % ४१

\twolineshloka
{अयोगः पञ्चभिर्हीनो निवृत्या लभ्यते जनैः}
{तयोर्योगे भवेद्योगो लभ्यते शान्तिमार्गतः} % ४२

\twolineshloka
{तमेव गणराजं त्वं ज्ञात्वा भज महामते}
{तेन शान्तिभवं सौख्यं लभसे नान्यथा क्वचित्} % ४३

\twolineshloka
{संयोगो हि गकारश्च णकारोऽयोगवाचकः}
{तयोः स्वामी गणेशानः पश्य वेदे महामते} % ४४

\twolineshloka
{अहं पुरा युतो भ्रान्त्या नानायोगपरायणः}
{अभवं तत्र भो विप्र सहजेन समाश्रितः} % ४५

\twolineshloka
{स्वाधीनत्वमहो दृष्ट्वा तत्र तेन सुविस्मितः}
{शरणं शङ्करस्यैव गतोऽहं योगकाम्यया} % ४६

\twolineshloka
{शङ्करेणोपदिष्टः स्म भजामि गणनायकम्}
{एकाक्षरविधानेन तेन शान्तिं गतोऽभवम्} % ४७

\twolineshloka
{एवमुक्त्वा गणेशस्य मह्यं मन्त्रं ददौ ऋभुः}
{एकाक्षरस्य मन्त्रस्य जपैः शान्तिं गतोऽभवम्} % ४८

\twolineshloka
{अष्टावक्रः प्रसन्नात्मा योगिवन्द्यो महायशाः}
{तस्मै नृपाय मन्त्रं स ददौ द्वादशवर्णकम्} % ४९

\twolineshloka
{सविधिं मन्त्रराजं तं दत्त्वा चान्तर्हितोऽभवत्}
{राजा विघ्नेश्वरस्यैव भजने तत्परोऽभवत्} % ५०

\twolineshloka
{तत्रादौ चैत्रमासे या नृप शुक्ला समागता}
{चतुर्थी सा कृता तेन नागरैः स्वजनैः पुरा} % ५१

\twolineshloka
{ततः स घोषयामास व्रतं जनपदेषु तत्}
{शौक्लं कार्ष्णं विशेषेण चक्रुः सर्वे जनास्तथा} % ५२

\twolineshloka
{ततो रोगादिभिर्हीना बभूवुः पुत्रसंयुताः}
{नानादोषविहीनास्ते विहारं चक्रुरादरात्} % ५३

\twolineshloka
{चन्द्रप्रियो गणेशानमभजन्नान्यचेतसा}
{पुत्रे राज्यं निवेद्यैव सस्त्रीको वनगोऽभवत्} % ५४

\twolineshloka
{ततः स्वल्पेन कालेन सस्त्रीको नृपसत्तमः}
{स्वानन्दे गणपं गत्वा ब्रह्मभूतो बभूव ह} % ५५

\twolineshloka
{क्रमेण भूमिसंस्था ये व्रतपुण्यप्रभावतः}
{ब्रह्मभूता बभूवुस्ते दृष्ट्वा गणपतिं नृप} % ५६

\twolineshloka
{अन्यत्त्वं श‍ृणु चैत्रे वै शुक्लपक्षे चतुर्थिका}
{तस्याश्चरित्रं पापघ्नं सर्वसिद्धिप्रदायकम्} % ५७

\twolineshloka
{मालवे शूद्रजः पापी कोऽवसन् वनगोचरान्}
{जघान पथिकान् दुष्टो द्रव्यलोभी नराधमः} % ५८

\twolineshloka
{तस्य पापस्य सङ्ख्यानं कर्तुं नैव भवाम्यहम्}
{समर्थो गणना त्यक्ता ग्रन्थबाहुल्यदोषतः} % ५९

\twolineshloka
{तं कदाचिद्वने संस्थं वृक्षाग्रे शस्त्रधारकम्}
{वञ्चयित्वा जनान् घ्नन्तं तत्र सर्पो ददंश ह} % ६०

\twolineshloka
{ततः सोऽपि भयोद्विग्नः स्वगृहं प्रत्यपद्यत}
{ततो विषेण राजेन्द्र पीडितो मूर्च्छितोऽभवत्} % ६१

\twolineshloka
{दैवयोगेन तस्यापि चतुर्थी शुक्लगा नृप}
{चैत्रे प्राप्ता च शूद्रोऽभूत् पीडयाऽन्नविवर्जितः} % ६२

\twolineshloka
{पञ्चम्यां स मृतः पापी गतः स्वानन्दके पुरे}
{व्रतपुण्यप्रभावेणाऽज्ञानेनाऽपि नृपात्मज} % ६३

\twolineshloka
{एवं नानाविधा राजन् ब्रह्मभूता बभूविरे}
{तेषां चरित्रकं वक्तुं शक्यते न मया कदा} % ६४

\twolineshloka
{इदं चैत्रचतुर्थ्यास्तु शुक्लायाः संश‍ृणोति चेत्}
{माहात्म्यं स लभेत् कामानीप्सितान् पठते यदा} % ६५

॥ॐ तत्सदिति श्रीमदान्त्ये पुराणोपनिषदि श्रीमन्मौद्गले महापुराणे चतुर्थे खण्डे गजाननचरिते चैत्रशुक्लचतुर्थीवर्णनं नाम चतुर्दशोऽध्यायः॥४.१४॥


\sect{४.१५ --- पञ्चदशोऽध्यायः --- वैशाखशुक्लचतुर्थीमाहात्म्यवर्णनम्}

\centerline{॥ श्रीगणेशाय नमः ॥}

\uvacha{दशरथ उवाच}

\twolineshloka
{अधुना शुक्लगायास्तु वैशाखे चरितं वद}
{चतुर्थ्याः शुभदं पूर्णं न तृप्यामि कथामृतात्} % १

\uvacha{वसिष्ठ उवाच}

\twolineshloka
{गुर्जरे सर्वशोभाढ्यं नगरं भद्रकं परम्}
{तत्र राज्यं चकारैव राजा ब्रह्मप्रियो महान्} % २

\twolineshloka
{सर्वशास्त्रार्थसम्पन्नो यज्वा दानप्रियः सदा}
{देवद्विजातिथिप्रेप्सुः शस्त्रास्त्रे पारगोऽभवत्} % ३

\twolineshloka
{सर्वान् राज्ञो वशे कृत्वा पृथिवीमण्डलाधिपः}
{बभूव बलसम्पन्नः सर्वमान्यो महामतिः} % ४

\twolineshloka
{तस्य पत्नी मृता सद्यो रजो दर्शनमात्रतः}
{द्वितीया च कृता पत्नी समशीला नृपेण ह} % ५

\twolineshloka
{साऽपि तद्वन् मृता राजन्नेवं पञ्च मृतास्ततः}
{पत्न्योऽभवन् सुदुःखार्तो राजा ब्रह्मप्रियोऽभवत्} % ६

\twolineshloka
{राज्यं निक्षिप्य राजर्षिः प्रधानेषु ययौ वनम्}
{स्वगुरुं श्वेतकेतुं स प्रणम्य पुरतः स्थितः} % ७

\twolineshloka
{श्वेतकेतुः स्वशिष्यं तमासनादिषु मानदम्}
{संस्थाप्य कुशलत्वेन मानयामास हर्षितः} % ८

\twolineshloka
{भोजयित्वा स राजानं महत् पप्रच्छ कारणम्}
{किमर्थं राजशार्दूल कृतमागमनं त्वया} % ९

\twolineshloka
{वद तत् कारणं मुख्यं तत् करिष्येऽहमादरात्}
{ममाश्रमे त्वमेकाकी राज्यं त्यक्त्वा समागतः} % १०

\twolineshloka
{एवं पृष्टः स राजर्षिस्तं प्रणम्य कृताञ्जलिः}
{जगाद सर्वं वृत्तान्तं पत्नीनाशात्मकं स्वयम्} % ११

\twolineshloka
{तच्छ्रुत्वा गाणपत्यश्च श्वेतकेतुर्महातपाः}
{ध्यानेनालोक्य तं भूपं जगाद क्रोधसंयुतः} % १२

\uvacha{श्वेतकेतुरुवाच}

\twolineshloka
{नृपाधम महापापिन् श‍ृणु मे वचनं हितम्}
{व्रतं मुख्यं चतुर्थ्यास्ते राज्ये नष्टं विशेषतः} % १३

\twolineshloka
{चतुर्थीव्रतहीनस्य कर्म सर्वं सुनिष्फलम्}
{नारकी च भवत्यन्ते स नरो नात्र संशयः} % १४

\twolineshloka
{चतुर्भिः पुरुषार्थैस्त्वं वर्जितो नितरां नृप}
{चतुःपदार्थदातृत्वाच्चतुर्थी कथिता बुधैः} % १५

\twolineshloka
{श्वेतकेतुवचः श्रुत्वा कोपयुक्तं महीपतिः}
{प्रणम्य तमुवाचाऽथ लज्जितः खेदसंयुतः} % १६

\uvacha{ब्रह्मप्रिय उवाच}

\twolineshloka
{अज्ञानेन कृतं विप्र सर्वदस्य व्रतस्य यत्}
{अनाचरणकं तेन क्षमस्व करुणानिधे} % १७

\twolineshloka
{वद मां कीदृशं स्वामिन् व्रतं सर्वार्थदं परम्}
{पूजनं कस्य वा कार्यं विधियुक्तं च तत्र वै} % १८

\twolineshloka
{एवं पृष्टो महातेजाः श्वेतकेतुस्तमब्रवीत्}
{माहात्म्यं व्रतमुख्यस्य श्रुत्वा सोऽपि तमब्रवीत्} % १९

\uvacha{ब्रह्मप्रिय उवाच}

\twolineshloka
{कीदृशोऽयं गणाधीशो व्रतं यस्य महाद्भुतम्}
{स्वरूपं वद मे तस्य भजिष्यामि तमादरात्} % २०

\uvacha{श्वेतकेतुरुवाच}

\twolineshloka
{श‍ृणु राजन् पुरावृत्तं कथयामि समासतः}
{मदीयं चेष्टितं तेन गणेशं ज्ञास्यसे परम्} % २१

\twolineshloka
{अहं पुरा तपोनिष्ठो बभूवातितरां नृप}
{उद्दालकं प्रणम्यैव पितरं साधने रतः} % २२

\twolineshloka
{ततोऽतितपसा युक्तं दृष्ट्वा मां जनको वचः}
{जगाद स्नेहसंयुक्तः शान्तिदाता महायशाः} % २३

\uvacha{उद्दालक उवाच}

\twolineshloka
{पुत्र शान्त्यर्थमेव त्वं तपस्त्यक्त्वा महामते}
{कुरु श्रमं विशेषेण ब्रह्माहमिति धारयन्} % २४

\twolineshloka
{कोऽसि त्वं कुत आयातो कुत्र गच्छसि मां वद}
{चित्ते चिन्तामणिं तात पश्य पश्य विशेषतः} % २५

\twolineshloka
{पितुर्वचनमाकर्ण्य तमहं पुनरब्रवम्}
{वद तात स्वपुत्राय ज्ञानं शान्तिप्रदं महत्} % २६

\twolineshloka
{चित्तं च कीदृशं तात तत्र चिन्तामणिः कथम्}
{सोऽपि तिष्ठति विप्रेश कथं ज्ञेयो महात्मभिः} % २७

\uvacha{वसिष्ठ उवाच}

\twolineshloka
{एवं पृष्टो महायोगी तमारुणिरुवाच ह}
{हर्षेण महता युक्तो गाणपत्यस्वभाववान्} % २८

\uvacha{उद्दालक उवाच}

\twolineshloka
{श‍ृणु पुत्र महाभाग योगशान्तिप्रदायकम्}
{येन त्वं सर्वयोगज्ञो ब्रह्मभूतो भविष्यसि} % २९

\twolineshloka
{चित्तं पञ्चविधं प्रोक्तं क्षिप्तं मूढं महामते}
{विक्षिप्तं च तथैकाग्रं निरोधं भूमिसंज्ञितम्} % ३०

\twolineshloka
{तत्र प्रकाशकर्ताऽसौ हृदि चिन्तामणिः स्थितः}
{साक्षाद्योगेन योगज्ञैर्लभ्यते भूमिनाशनात्} % ३१

\twolineshloka
{चित्तरूपा स्वयं बुद्धिश्चित्तभ्रान्तिकरी मता}
{सिद्धिर्माये गणेशस्य मायाखेलक उच्यते} % ३२

\twolineshloka
{अतो गणेशमन्त्रेण गणेशं भज पुत्रक}
{तेन त्वं ब्रह्मभूतत्वं शान्त्या योगेन यास्यसि} % ३३

\twolineshloka
{इत्युक्त्वा गणराजस्य ददौ मन्त्रं तथाऽऽरुणिः}
{एकाक्षरं स्वपुत्राय नृप ध्यानादिसंयुतम्} % ३४

\twolineshloka
{तेनाऽहं साधयामि स्म गणेशं सर्वसिद्धिदम्}
{क्रमेण शान्तिमापन्नो योगिवन्द्योऽभवं ततः} % ३५

\twolineshloka
{अतस्त्वमपि राजेन्द्र भजस्व गणनायकम्}
{तेनेहपरलोकस्थसुखं प्राप्स्यसि शाश्वतम्} % ३६

\twolineshloka
{ब्रह्मणि ब्रह्मभूतस्त्वं भविष्यसि न संशयः}
{एवमुक्त्वा स द्वात्रिंशदक्षरस्थं ददौ मनुम्} % ३७

\twolineshloka
{विधियुक्तं मनुं राजाऽऽगृह्य स्वगृहमाययो}
{श्वेतकेतुं प्रणम्यैव हर्षयुक्तेन चेतसा} % ३८

\twolineshloka
{तत्राऽऽदौ शुक्लपक्षस्था वैशाखी च समागता}
{चतुर्थी तां चकाराऽसौ नागरैर्हर्षसंयुतः} % ३९

\twolineshloka
{तेन सर्वत्र घोषश्च व्रतस्य प्रकृतो महान्}
{शौक्लं कार्ष्णं व्रतं चक्रुर्भूमिसंस्था जना नृप} % ४०

\twolineshloka
{राजा स्वस्त्रीसमायुक्तोऽभजत्तं गणनायकम्}
{पुत्रं राज्ये समास्थाप्य निवृत्त्या संयुतोऽभवत्} % ४१

\twolineshloka
{मन्त्रं जजाप विघ्नेशं ध्यात्वा पूजापरायणः}
{व्रतयुक्तस्तथाऽन्ते स ब्रह्मभूतो बभूव ह} % ४२

\twolineshloka
{ब्रह्मप्रियेति नामाऽभूत् तदेवं सार्थकं कृतम्}
{तेन राज्ञा महाभाग गाणपत्यस्वभावतः} % ४३

\twolineshloka
{तस्य राज्ये जनाः सर्वे व्रतपुण्येन भूमिप}
{क्रमेण ब्रह्मभूतास्तेऽभवन् स्वानन्दके पुरे} % ४४

\twolineshloka
{उद्दालकश्वेतकेतू गाणपत्यस्वभावतः}
{अवधूतौ तु विख्यातौ बभूवाते महाप्रभू} % ४५

\twolineshloka
{श्वेतकेतुद्विजेनाऽपि मर्यादा प्रकृता बलात्}
{सर्वेभ्यः सुखदाऽत्यन्तं तां श‍ृणुष्व विशेषतः} % ४६

\twolineshloka
{कदाचिदारुणिस्तत्र सस्त्रीकः स्वाश्रमे स्थितः}
{आजगाम द्विजः कश्चित् सर्वशास्त्रविशारदः} % ४७

\twolineshloka
{जगाद स्त्रियमेवं स आरुणेः कामविह्वलः}
{मैथुनाय समागच्छ मया सह सुरूपिणि} % ४८

\twolineshloka
{तच्छ्रुत्वा श्वेतकेतुं तं क्रुद्धमारुणिरब्रवीत्}
{मा क्रोधं कुरु पुत्र त्वमवृतास्तु स्त्रियो मताः} % ४९

\twolineshloka
{तच्छ्रुत्वा पितरं प्राह श्वेतकेतू रुषा युतः}
{एवं चेत् पतिमेकं वृणोति सा कामिनी कथम्} % ५०

\twolineshloka
{अतो वै वेधसा स्वामिन्न कृतं कर्म शाश्वतम्}
{असमञ्जसकं मत्वा मर्यादां कारयाम्यहम्} % ५१

\twolineshloka
{अद्यप्रभृति रागेण स्पृशेद्यश्च परस्त्रियम्}
{स्त्रीहत्यां लभतां तत्र पुरुषः स न संशयः} % ५२

\twolineshloka
{अथवा पुरुषं कञ्चित् पतिं त्यक्त्वा च कामिनी}
{गच्छेन् मैथुनभावार्थं पतिहत्यां तु सा लभेत्} % ५३

\twolineshloka
{यद्यहं गणराजस्य भक्तश्चेद्विघ्ननायक}
{तदा मे वचनं सत्यं भवेत्तत्ते नमो नमः} % ५४

\twolineshloka
{तदादि संवृता स्त्री सा सेवते भावतः पतिम्}
{विनायकप्रसादेन तेनेयं स्थापिताऽभवत्} % ५५

\twolineshloka
{प्रसङ्गात्ते मया ख्यातमुद्दालकविचेष्टितम्}
{माहात्म्यमन्यदधुना चतुर्थीसम्भवं श‍ृणु} % ५६

\twolineshloka
{महाराष्ट्रे द्विजः कश्चित् परस्त्रीलम्पटोऽभवत्}
{तदर्थं तेन मद्यस्य कृतं पानं विशेषतः} % ५७

\twolineshloka
{चौर्येण धनमादाय हिंसाकर्मपरायणः}
{नानापापरतो भूत्वा परस्त्रियमसेवत} % ५८

\twolineshloka
{जगाम शूद्रगेहे स एकदा कामविह्वलः}
{धनं गृहीत्वा स्वं लब्धं धनं दत्त्वा रराम ह} % ५९

\twolineshloka
{एवं किञ्चिद्गते काले शूद्रः स्वगृहमागतः}
{तेन शस्त्राभिघातेन हतोऽभूत्स द्विजाधमः} % ६०

\twolineshloka
{तस्मिन् दिने समायाता चतुर्थी दैवयोगतः}
{तत्र शस्त्राभिघातस्य पीडया संयुतोऽभवत्} % ६१

\twolineshloka
{अन्नादिभिर्विहीनश्च पञ्चम्यां स ममार ह}
{ततः स्वानन्दके लोके जगाम भृशपूजितः} % ६२

\twolineshloka
{वैशाखशुक्लपक्षस्य चतुर्थीव्रतयोगतः}
{ब्रह्मभूतः स वै जातो महापापपरायणः} % ६३

\twolineshloka
{अज्ञानव्रतजेनैव पुण्येनाऽसौ द्विजाऽधमः}
{ब्रह्मभूतश्च सञ्जातो ज्ञानिनां तत्र का कथा} % ६४

\twolineshloka
{नाना जना व्रतं कृत्वा ज्ञानतोऽज्ञानतो नृप}
{बभूवुर्ब्रह्मभूताश्च मया वक्तुं न शक्यते} % ६५

\twolineshloka
{एतद् वैशाखशुक्लायाश्चतुर्थ्याः श‍ृणुयात्तु यः}
{माहात्म्यं प्रपठेद्वाऽपि स लभेदीप्सितं फलम्} % ६६

॥ॐ तत्सदिति श्रीमदान्त्ये पुराणोपनिषदि श्रीमन्मौद्गले महापुराणे चतुर्थे खण्डे गजाननचरिते वैशाखशुक्लचतुर्थीमाहात्म्यवर्णनं नाम पञ्चदशोऽध्यायः॥४.१५॥


\sect{४.१६ --- षोडशोऽध्यायः --- ज्येष्ठशुक्लचतुर्थीमाहात्म्यवर्णनम्}

\centerline{॥ श्रीगणेशाय नमः ॥}

\uvacha{दशरथ उवाच}

\twolineshloka
{श्रुतं वैशाखमासे वै शुक्लायाः फलमुत्तमम्}
{चतुर्थ्या अधुना योगिन् ज्येष्ठशुक्लां वद प्रभो} % १

\uvacha{वसिष्ठ उवाच}

\twolineshloka
{आन्ध्रे शेषपुरे राजन् राजाऽभूत् कर्दमाभिधः}
{शस्त्रास्त्रनिपुणोऽत्यन्तं नानाधर्मपरायणः} % २

\twolineshloka
{वशं भूमण्डलं यस्य समुद्रान्तं बभूव ह}
{तेजसा धर्मनीत्या वै ह्यतुलो यशसाऽभवत्} % ३

\twolineshloka
{दैवयोगेन तस्याऽभूत् प्रमेहश्चाऽतिदारुणः}
{अग्निवद् दाहको देहे पापेन प्रेरितः परः} % ४

\twolineshloka
{तेनातिपीडितो राजा मूत्रितुं न शशाक सः}
{रुरोद कर्दमो भूत्वा सदा दाहयुतो भृशम्} % ५

\twolineshloka
{नानोपायाः कृतास्तेन शान्तो रोगो बभूव न}
{ततो राज्यं परित्यज्य सस्त्रीकः स ययौ वनम्} % ६

\twolineshloka
{वनाद्वनान्तरं गत्वा महोग्रं भयवर्धनम्}
{सिंहव्याघ्रादियुक्तं च मरणाय रुरोद ह} % ७

\twolineshloka
{तत्राऽऽजगाम विप्रेन्द्रो भरद्वाजो महायशाः}
{तं दृष्ट्वा सहसोत्थाय सस्त्रीकः प्रणनाम सः} % ८

\twolineshloka
{स हि बध्वा करपुटं तत्पुरः संस्थितोऽभवत्}
{ततोऽतिदाहसंयुक्तः पपात च रुरोद सः} % ९

\twolineshloka
{तादृशं नृपनाथं स दृष्ट्वा योगीन्द्रसत्तमः}
{ध्यानेनालोक्य राजानमुवाच दयया युतम्} % १०

\uvacha{भरद्वाज उवाच}

\twolineshloka
{श‍ृणु कर्दम राजेन्द्र वचनं मे हितावहम्}
{येन दुःखविहीनस्त्वं भविष्यसि महामते} % ११

\twolineshloka
{चतुर्थीव्रतमत्यन्तं नष्टं राज्ये त्वदीयके}
{तेन पापसमायुक्तस्त्वं जातोऽसि नृपाधमः} % १२

\twolineshloka
{अतस्त्वं जनसंयुक्तो व्रतं कुरु महामते}
{पुण्येन तेन राजेन्द्र दुःखहीनो भविष्यसि} % १३

\twolineshloka
{भरद्वाजवचः श्रुत्वा तं पप्रच्छ प्रणम्य सः}
{कर्दमो हर्षसंयुक्तो व्रतस्याऽऽचरणाय च} % १४

\uvacha{कर्दम उवाच}

\twolineshloka
{भगवन् सर्वतत्त्वज्ञ त्वयाऽहमनुकम्पितः}
{व्रतस्य वद माहात्म्यमधुना तत् करोम्यहम्} % १५

\uvacha{भरद्वाज उवाच}

\twolineshloka
{सङ्कष्टं वरदं कृष्णे चतुर्थीजं च शुक्लके}
{व्रतं सर्वार्थदं पूर्णं चतुर्णां साधकं मतम्} % १६

\twolineshloka
{सर्वादौ तन्नरः कुर्यात्तदा सर्वं नराधिप}
{चतुर्णां पुरुषार्थानां दातृकर्मफलं लभेत्} % १७

\twolineshloka
{अतो हीनश्चतुर्भिस्त्वं स जनो नरके नृप}
{पतिष्यसि न सन्देहस्तदर्थं यत्नमाचर} % १८

\twolineshloka
{एवमुक्त्वा स माहात्म्यं चतुर्थीसम्भवं नृप}
{कथयामास भूपाय श्रुत्वा राजा ननन्द ह} % १९

\twolineshloka
{पुनः पप्रच्छ तं विप्रं भरद्वाजं स कर्दमः}
{गणेशस्य स्वरूपं मे वद सर्वज्ञ ते नमः} % २०

\uvacha{भरद्वाज उवाच}

\twolineshloka
{गणेशस्य स्वरूपं तु वक्तुं वेदादिका नृप}
{न समर्थास्तथाऽपि त्वं श‍ृणु सारं सुखप्रदम्} % २१

\twolineshloka
{पुराऽहं तपसा युक्तोऽतपं च तप उत्तमम्}
{सर्वं चराचरं राजन् मदधीनं बभूव ह} % २२

\twolineshloka
{तथाऽपि तपसोग्रेणाऽसाधयं तु तपः पुनः}
{ततोऽन्तर्ज्ञानभावे मे मतिर्जाता सुपुण्यतः} % २३

\twolineshloka
{तपस्त्यक्त्वा ततोऽहं तु शमी दमपरोऽभवम्}
{जडोन्मत्तादिजे मार्गे संस्थितो योगकारणात्} % २४

\twolineshloka
{ततः क्रमेण भो राजन्नसत्स्वानन्दगोऽभवम्}
{तत्र सम्पूर्णयोगेन शान्तिं प्राप्तो विशेषतः} % २५

\twolineshloka
{तस्माद्भेदादिकं सर्वमुत्पन्नं तु विशेषतः}
{तद् दृष्ट्वा क्षुभितोऽत्यन्तं शान्तिहीनो यथाऽभवम्} % २६

\twolineshloka
{ततोऽकस्माद्रैवतो यो महायोगी समागतः}
{आश्रमे मे च तं दृष्ट्वा प्रणतोऽहं सुपूजयन्} % २७

\twolineshloka
{स्वासने सुखमासीनमब्रवं तं महामुनिम्}
{अद्याऽहं कृतकृत्यश्च जातस्ते दर्शनेन वै} % २८

\twolineshloka
{वद मे योगशान्तिं त्वं योगीन्द्राणां गुरुर्भवान्}
{तिष्ठामि शान्तिगो भूत्वा ययाऽहं योगधारकः} % २९

\uvacha{रैवत उवाच}

\twolineshloka
{स्वानन्दः पञ्चधा प्रोक्तः सदसत्समनेतितः}
{चतुर्णां चैव संयोगे स्वस्वरूपः प्रकीर्तितः} % ३०

\twolineshloka
{अयोगः स्वस्वरूपेण हीनः सर्वत्र सम्मतः}
{तयोर्योगो भरद्वाज योगशान्तिप्रदायकः} % ३१

\twolineshloka
{स्वसंवेद्यो गकारश्च णकारो योग उच्यते}
{तयोः स्वामी गणेशोऽयं ब्रह्मणस्पतिवाचकः} % ३२

\twolineshloka
{तं भजस्व महाभाग ततः शान्तिमवाप्स्यसि}
{नान्यथा शतवर्षैस्त्वं भ्रमयुक्तो भविष्यसि} % ३३

\twolineshloka
{एतद्विष्णुमुखाद्ब्रह्मन् श्रुत्वा ज्ञानं महत् पुरा}
{तेनाऽहं शान्तिमापन्नो गाणपत्यश्चरामि वै} % ३४

\twolineshloka
{एवमुक्त्वा ददौ तस्मै स्वमन्त्रं सिद्धिदायकम्}
{एकाक्षरं गणेशस्य सविधिं न्याससंयुतम्} % ३५

\twolineshloka
{पूजितो रैवतस्तेन जगाम स्वेच्छया चरन्}
{भरद्वाजो गणेशानमभजन्नान्यचेतसा} % ३६

\twolineshloka
{ततः शान्तिं समापन्नो तथापि भजने रतः}
{गतेषु दशवर्षेषु विघ्नेशस्तं समाययौ} % ३७

\twolineshloka
{तं दृष्ट्वा पूजयामास भरद्वाजः प्रतापवान्}
{स साम्नामष्टनामार्थस्तोत्रेण प्रणनाम ह} % ३८

\twolineshloka
{ततो मां गाणपत्यं स कृत्वा स्वानन्दगोऽभवत्}
{गणेशस्तं विशेषेणाऽहं भजामि सुभक्तितः} % ३९

\twolineshloka
{एवमुक्त्वा ददौ तस्मै कर्दमाय महामनुम्}
{अष्टाक्षरं गणेशस्य विधियुक्तं विधानवित्} % ४०

\twolineshloka
{ततश्चान्तर्हितो राजन् भरद्वाजो महामुनिः}
{कर्दमः स्वगृहे गत्वा हर्षयुक्तः पुपूज तम्} % ४१

\twolineshloka
{तत्राऽऽदौ ज्येष्ठमासे या नृप शुक्ला समागता}
{चतुर्थी सा कृता तेन नागरैर्हर्षसंयुतः} % ४२

\twolineshloka
{वरदं सङ्कटं शुक्ले चतुर्थीजं च कृष्णके}
{न करोति नरो यः स सन्ताड्यो नगराद्बहिः} % ४३

\twolineshloka
{प्रकाशितं ततस्तेन सर्वत्र व्रतमुत्तमम्}
{कृतं भूमितले राजन् चक्रुः सर्वे जना व्रतम्} % ४४

\twolineshloka
{प्रमेहदुःखनिर्मुक्तो बभूवे राजसत्तमः}
{कर्दमः स स्वकीयं वै राज्यं पुत्राय सन्ददौ} % ४५

\twolineshloka
{वाटिकायां स्थितो राजा सस्त्रीको ह्यभजत् सदा}
{गणेशं सर्वभावेन सोऽन्ते तन्मयतामयात्} % ४६

\twolineshloka
{तस्य राज्ये जनाः सर्वे ते क्रमेण महामते}
{अभवन् ब्रह्मभूताश्च व्रतपुण्यप्रभावतः} % ४७

\twolineshloka
{कथां रम्यां दशरथ अन्यां श‍ृणु सुसिद्धिदाम्}
{गौडदेशेऽन्त्यजः कश्चिद्वभूवे पापकारकः} % ४८

\twolineshloka
{वने गत्वा जघानाऽसौ द्रव्यलोभी जनान् सदा}
{योनिलम्पटभावेन दूषितामकरोत् स्त्रियम्} % ४९

\twolineshloka
{वन एकाकिनीं दृष्ट्वा ब्राह्मणीं क्षत्रियां तथा}
{धृत्वाऽयभत् स वेगेन शूद्रीं वैश्यस्त्रियं खलः} % ५०

\twolineshloka
{एवं नानास्वभावेन पापं चक्रे स नित्यशः}
{एकदा ब्राह्मणं कञ्चिद् दृष्ट्वा हन्तुं तमाययौ} % ५१

\twolineshloka
{पलायत भयोद्विग्नो हाहाकारपरायणः}
{तस्य नादं समाकर्ण्य पुरुषाः पञ्च आययुः} % ५२

\twolineshloka
{तैर्हतः शस्त्रघातेन पपात धरणीतले}
{तस्मिन् दिने समायाता ज्येष्ठी शुक्ला चतुर्थिका} % ५३

\twolineshloka
{स उपोषणयुक्तश्च बभूवे दैवयोगतः}
{ममार पीडया युक्तः पञ्चम्यां पापरूपकः} % ५४

\twolineshloka
{अज्ञानव्रतपुण्येन ब्रह्मभूतो बभूव ह}
{चाण्डालः किं पुनर्भूप ज्ञानिनां चित्रमेव च} % ५५

\twolineshloka
{एवं नानाविधा राजन् व्रतपुण्यप्रभावतः}
{इह भुक्त्वाऽखिलान् भोगानन्ते स्वानन्दगा बभुः} % ५६

\twolineshloka
{तत्रैवं कति भूपाल ब्रूयां वक्तुं न शक्यते}
{अपारमहिमा तस्मादानन्त्यं कथितं द्विजैः} % ५७

\twolineshloka
{इदं ज्येष्ठचतुर्थ्या यन् माहात्म्यं संश‍ृणोति सः}
{पठेद्वा यो लभेत् सर्वं शुक्लाया मनसीप्सितम्} % ५८

॥ॐ तत्सदिति श्रीमदान्त्ये पुराणोपनिषदि श्रीमन्मौद्गले महापुराणे चतुर्थे खण्डे गजाननचरिते ज्येष्ठशुक्लचतुर्थीमाहात्म्यवर्णनं नाम षोडशोऽध्यायः॥४.१६॥


\sect{४.१७ --- सप्तदशोऽध्यायः --- आषाढशुक्लचतुर्थीमाहात्म्यवर्णनम्}

\centerline{॥ श्रीगणेशाय नमः ॥}

\uvacha{दशरथ उवाच}

\twolineshloka
{वद ब्रह्मंश्च शुक्लाया आषाढे चरितं महत्}
{चतुर्थ्या नैव तृप्यामि कथां श्रुत्वा सुसिद्धिदाम्} % १

\uvacha{वसिष्ठ उवाच}

\twolineshloka
{मैथिले विषये राजन्नगरं गण्डकी महत्}
{तत्र राज्यं भद्रसेनश्चकारामिततेजसा} % २

\twolineshloka
{शस्त्रास्त्रनिपुणोऽत्यन्तं परराष्ट्रावमर्दनः}
{जित्वा भूमण्डलं सर्वं समुद्रवलयाङ्कितम्} % ३

\twolineshloka
{राज्यं चकार धर्मेण नीत्या दण्डेन भूमिप}
{यज्वा विनीतको मानी द्विजदेवातिथिप्रियः} % ४

\twolineshloka
{वशगाः सर्वराजानः सेवन्ते स्म नराधिपम्}
{अपारसेनया युक्तं करदा इतरेऽभवन्} % ५

\twolineshloka
{तस्य राज्ये शुका राजन् शलभा मूषकास्तथा}
{अपारा भक्षयन्ति स्म धान्यं वस्त्रादिकं बलात्} % ६

\twolineshloka
{तेषां नाशार्थमत्यन्तं भद्रसेनः प्रतापवान्}
{अस्त्रैर्यत्नपरो भूत्वा मारयामास तान् बहून्} % ७

\twolineshloka
{अग्न्यस्त्रेण समन्तात्तान् दग्धानपि च तादृशान्}
{ददर्श पुनरुत्पन्नांस्ततो राजाऽतिविस्मितः} % ८

\twolineshloka
{यत्र तत्र गृहान्तेषु चेरुस्ते शलभादयः}
{अशक्तः स वने गत्वा बभूव ह सुदुःखितः} % ९

\twolineshloka
{उपोषणपरो राजा तताप तप उत्तमम्}
{शिवं स्मृत्वा स तुष्टावापूज्य रौद्रेण भावतः} % १०

\twolineshloka
{गते वर्षे महायोगी बकदाल्भ्यः समाययौ}
{तं देशं दैवयोगेन ददर्शाऽसौ भ्रमन्नृपम्} % ११

\twolineshloka
{राजा तं पूजयामास सम्भोज्य नयसंयुतः}
{पप्रच्छ योगिनं तत्र दुःखयुक्तेन चेतसा} % १२

\uvacha{भद्रसेन स्वाच}

\twolineshloka
{स्वामिन् राज्ये मदीये वै मूषकाः शलभाः शुकाः}
{अत्यन्तं पीडयन्ति स्म जनान्मां भक्षणे रताः} % १३

\twolineshloka
{तत्रोपायाः कृता विप्राऽभवन् सर्वे सुनिष्फलाः}
{राज्यं त्यक्त्वा वनेऽहं च संस्थितो दुःखकारणात्} % १४

\twolineshloka
{शङ्करं सम्भजंस्तत्र दर्शनं ते महामते}
{प्राप्तस्तपः प्रभावेण तत्रोपायं वद प्रभो} % १५

\uvacha{वसिष्ठ उवाच}

\twolineshloka
{भद्रसेनवचः श्रुत्वा गाणपत्यो महायशाः}
{जगाद बकदाल्भ्यः स तं भूपं हर्षयन्निव} % १६

\uvacha{बकदाल्भ्य उवाच}

\twolineshloka
{शिवेन प्रेषितोऽत्राऽहं त्वदर्थं राजसत्तम}
{श‍ृणु मे परमं वाक्यं दुःखनाशकरं महत्} % १७

\twolineshloka
{राज्ये ते भद्रसेनाऽद्य व्रतं नष्टं बभूव ह}
{चतुर्थी संज्ञकं तेन विघ्नयुक्तोऽसि साम्प्रतम्} % १८

\twolineshloka
{प्रजाः सर्वा भयोद्विग्ना जाता दुष्टे नराधिपे}
{राज्यकर्तरि रे पापिन् नरके गच्छसि ह्यतः} % १९

\twolineshloka
{सर्वादौ तद्व्रतं सर्वैः कर्तव्यं नित्यवत्प्रभो}
{चतुर्णां पुरुषार्थानां दायकं विघ्नहारकम्} % २०

\twolineshloka
{न कृतं चेत्तदा सर्वं कृतं कर्म निरर्थकम्}
{चतुःपदार्थहीनत्वाद्विचारय महामते} % २१

\twolineshloka
{एवमुक्त्वा चतुर्थ्या यन् माहात्म्यं बकदाल्भ्यकः}
{कथयामास राज्ञे वै श्रुत्वा तं सोऽब्रवीद्वचः} % २२

\uvacha{भद्रसेन उवाच}

\twolineshloka
{कीदृशोऽयं गणाधीशो वद तस्य स्वरूपकम्}
{ज्ञात्वा तं सर्वदेवेशं भजिष्यामि विशेषतः} % २३

\uvacha{बकदाल्भ्य उवाच}

\twolineshloka
{पुरावृत्तं मदीयं यच्चेष्टितं श‍ृणु भूमिप}
{तेन त्वं गणराजस्य ज्ञाने सुनिपुणो भवेः} % २४

\twolineshloka
{पुराऽहं तपसा युक्तो वायुमात्राशनोऽभवम्}
{चराचरं ततो मत्तेजसा व्याप्तं भयातुरम्} % २५

\twolineshloka
{तथापि तपसा राजन् युक्तोऽहं तत्र मेऽद्भुतम्}
{बभूव विश्वरूपस्य दर्शनं सर्वगं परम्} % २६

\twolineshloka
{तेनाऽहं ज्ञानभावेन संस्थितो योगकारणात्}
{तपस्त्यक्त्वा विशेषेण जडोन्मत्तादिके रतः} % २७

\twolineshloka
{शमी दमपरो भूत्वा योगं पूर्णसुखप्रदम्}
{असाधयं सदाऽत्यन्तं मनोनिग्रहतत्परः} % २८

\twolineshloka
{एवं क्रमेण भूपाल सहजे ब्रह्मणि ह्यहम्}
{ब्रह्मभूतस्वभावेन संस्थितो हर्षवांस्तदा} % २९

\twolineshloka
{ततो मयाऽतिमोहेन शून्यं स्वाधीनरूपकम्}
{दृष्टं तेनाऽभवं भ्रान्तो ब्रह्मणि त्वीदृशं कथम्} % ३०

\twolineshloka
{ततोऽहं शरणं शम्भुं गत्वा तं स्तुतवान् स्तवैः}
{प्रसन्नं शङ्करं दृष्ट्वाऽवदं योगं वद प्रभो} % ३१

\uvacha{शिव उवाच}

\twolineshloka
{सत्यासत्यसमानानि सहजेन युतानि वै}
{निजात्मबोधतो विद्धि स्वानन्दादुद्भवानि च} % ३२

\twolineshloka
{संयोगे स्वस्वरूपत्वं ज्ञातव्यं वेदवादतः}
{अयोगे सर्वसंयोगो नश्यत्यत्र न संशयः} % ३३

\twolineshloka
{संयोगश्च तथा योगस्तयोर्योगो महामते}
{योगशान्तिप्रदः प्रोक्तो योगिभिर्योगसेवया} % ३४

\twolineshloka
{योगशान्तिमयं विद्धि गणेशं ब्रह्मनायकम्}
{तं भजस्व विधानेन ब्रह्मभूतो भविष्यसि} % ३५

\twolineshloka
{संयोगश्च गकाराख्यो णकारो योगवाचकः}
{तयोः स्वामी गणाधीशः पश्य वेदे विशेषतः} % ३६

\twolineshloka
{एवमुक्त्वा महादेवो ददौ मन्त्रं द्विजाय मे}
{एकाक्षरं गणेशस्य विधियुक्तं सुयोगदम्} % ३७

\twolineshloka
{प्रणम्य शङ्करं भूप गतोऽहं वनमेव च}
{अभजं गणराजं तं ध्यात्वा जपपरायणः} % ३८

\twolineshloka
{ततस्तत्कृपया शान्तिर्मया प्राप्ता विशेषतः}
{तथाऽपि पूजने सक्तस्तं ध्यायामि स्म चेतसा} % ३९

\twolineshloka
{ततो मां दर्शयामास रूपं शुण्डाविराजितम्}
{दृष्ट्वा तं प्रणमामि स्म स्तौमि हर्षसमन्वितः} % ४०

\twolineshloka
{भक्तिं दत्त्वा गणाधीशो ह्यन्तर्धानं चकार मे}
{तदारभ्य महाभाग गाणपत्योऽहमादरात्} % ४१

\twolineshloka
{एवमुक्त्वा ददौ तस्मै मन्त्रं पञ्चाक्षरं मुनिः}
{अन्तर्धानं गणेशस्य स चकार द्विजोत्तमः} % ४२

\twolineshloka
{बकदाल्भ्यं गतं दृष्ट्वा राजा हर्षसमन्वितः}
{चकार स्वपुरे राज्यं पुनरागत्य धर्मतः} % ४३

\twolineshloka
{ततः स नागरैः सार्धं चकार व्रतमुत्तमम्}
{आषाढे शुक्लपक्षे वै चतुर्थीजं सुभक्तितः} % ४४

\twolineshloka
{ततः सर्वत्र तेनैव प्रशस्तं तद्व्रतं कृतम्}
{शुक्लकृष्णचतुर्थीजं व्रतं चक्रुर्जना नृप} % ४५

\twolineshloka
{शुकाश्च मूषकास्तत्र शलभा नाशमाययुः}
{धनधान्ययुता लोकाः पुष्टिं लेभुः सुहर्षतः} % ४६

\twolineshloka
{नानादुःखं परित्यज्यानन्देनैव समावृताः}
{अन्ते स्वानन्दगाः सर्वे ब्रह्मभूता बभूविरे} % ४७

\twolineshloka
{राज्येऽभिषिच्य पुत्रं स्वं सपत्नीको ययौ वनम्}
{तत्र तं गणराजं सोऽभजतानन्यभावतः} % ४८

\twolineshloka
{अन्ते स्वानन्दगो भूत्वा सस्त्रीकोऽजसमुद्भव}
{ब्रह्मभूतो व्रतस्यास्य प्रसादेन बभूव ह} % ४९

\twolineshloka
{अन्यच्छृणु चरित्रं तद्व्रतजं पुण्यवर्धनम्}
{आषाढे वरदायाश्च भुक्तिमुक्तिफलप्रदम्} % ५०

\twolineshloka
{बङ्गदेशे समुत्पन्नो वाणिजः पापकारकः}
{स्वधर्मं स परित्यज्य दृष्टकर्मरतोऽभवत्} % ५१

\twolineshloka
{द्यूतमद्यादिकं नित्यं हिंसयाऽसेवतान्वितः}
{बलात् धृत्वाऽभुङ्क्त परस्त्रियं वशमनागताम्} % ५२

\twolineshloka
{तस्य कर्म दुराचारं ज्ञात्वा पित्रा तिरस्कृतः}
{असकृत्तेन स क्षुब्धो विषं पित्रे ददौ खलः} % ५३

\twolineshloka
{विषबाधासमायुक्तः स ममार ददाह तम्}
{ततो मातरमागम्य धनं जग्राह वै बलात्} % ५४

\twolineshloka
{लोकाः सर्वे ततो ज्ञात्वा चेष्टितं दुःखदं परम्}
{तस्य दुष्टस्य ते तत्र व्यथिताः सम्बभूविरे} % ५५

\twolineshloka
{जनाः श्रेष्ठा नृपं गत्वाऽकथयंस्तद्विशेषतः}
{चेष्टितं क्षुभितो राजा श्रुत्वा तं तत्र चानयत्} % ५६

\twolineshloka
{शूले नृप नरैः प्रोतो दुर्मतिः स नृपाज्ञया}
{तद्दिने दैवयोगेनाषाढी शुक्ला बभूव ह} % ५७

\twolineshloka
{तत्राऽयं जलहीनश्च निराहारः स्थितोऽभवत्}
{पञ्चम्यां पीडया युक्तो ममार नृप दुष्टधीः} % ५८

\twolineshloka
{अज्ञानव्रतपुण्येन स स्वानन्दगतोऽभवत्}
{महापापी व्रतस्यैव प्रभावेण महीपते} % ५९

\twolineshloka
{स्वानन्दे गणपं दृष्ट्वा ब्रह्मभूतो बभूव ह}
{एवं व्रतप्रभावेण जना ब्रह्म प्रलेभिरे} % ६०

\twolineshloka
{तत्रैवं कति सङ्ख्यातुं शक्यते न कदाचन}
{वर्षायुतैर्महाराजाऽखिलं केनाऽपि योगिना} % ६१

\twolineshloka
{यदि ज्ञानेन सा देवी चतुर्थी साधिता भवेत्}
{वरदा तत्र किं चित्रं तस्मै ब्रह्मप्रदा भवेत्} % ६२

\twolineshloka
{इदमाषाढगायाश्च चतुर्थ्याश्चरितं पठेत्}
{श‍ृणुयाद्वरदायाश्चेत् स लभेदीप्सितं फलम्} % ६३

॥ॐ तत्सदिति श्रीमदान्त्ये पुराणोपनिषदि श्रीमन्मौद्गले महापुराणे चतुर्थे खण्डे गजाननचरिते आषाढशुक्ल चतुर्थीचरितवर्णनं नाम सप्तदशोऽध्यायः॥४.१७॥


\sect{४.१८ --- नामाष्टादशोऽध्यायः --- श्रावणशुक्लचतुर्थीमाहात्म्यवर्णनम्}

\centerline{॥ श्रीगणेशाय नमः ॥}

\uvacha{दशरथ उवाच}

\twolineshloka
{श्रावणे वरदायाश्च माहात्म्यं वद विस्तरात्}
{न तृप्यामि कथां श्रुत्वा ब्रह्मभूयप्रदां प्रभो} % १

\uvacha{वसिष्ठ उवाच}

\twolineshloka
{अङ्गदेशे पुरं श्रीमच्छततारं बभूव ह}
{राजा राज्यं चकाराश्वसेनसंज्ञोऽपि तत्र हि} % २

\twolineshloka
{धर्मेण नीतियुक्तेन नानादानपरायणः}
{देवविप्रातिथिप्रेप्सुर्बभूवातिपराक्रमी} % ३

\twolineshloka
{नानाव्रतपरो राजा यज्वा तेजस्विनां वरः}
{शस्त्रास्त्रैः पृथिवीं सर्वां जिग्ये बलसमन्वितः} % ४

\twolineshloka
{सार्वभौमः स विख्यातस्त्रिलोकीकीर्तिकारकः}
{राज्ञः सर्वान् वशे चक्रे सेवकान् करदायिनः} % ५

\twolineshloka
{एवं राज्यं वशं कृत्वा भूतलं स चकार ह}
{ततोऽकस्माज्ज्वरस्तस्य समुत्पन्नोऽतिदाहकः} % ६

\twolineshloka
{ज्वरेण पीडितोऽत्यन्तं निशि निद्रां न चाऽलभत्}
{नानोपायांश्चकाराऽसौ ज्वरहीनो बभूव न} % ७

\twolineshloka
{एवं वर्षे गते पूर्णे सोऽस्थिचर्मावशेषितः}
{अत्यन्तदुःखितो राजा विललाप भृशातुरः} % ८

\twolineshloka
{विषेण देहपातार्थमुद्यतोऽभून् महीपतिः}
{जगाम सहसा तत्र देवलो योगिसत्तमः} % ९

\twolineshloka
{तं दृष्ट्वा प्रणनामाथ पूजयामास बान्धवैः}
{भुक्त्वा तृप्तं जगादेदं वचनं साधुमार्गवित्} % १०

\uvacha{अश्वसेन उवाच}

\twolineshloka
{धन्यं मे जन्म कर्माऽपि पिता माता व्रतादिकम्}
{अधुना सफलं दानं येन ते दर्शनं मुने} % ११

\twolineshloka
{ज्वरेण पीडितोऽत्यन्तं विषपाने महामुने}
{तत्परोऽहं प्रपश्याम्यागतं त्वां योगिसत्तम} % १२

\twolineshloka
{मृतोऽहं विषपानेनात्महत्यां न लभे कदा}
{तव दर्शनमात्रेण मुक्तो यास्यामि धाम तत्} % १३

\twolineshloka
{एवमुक्त्वा रुरोदाऽपि ततस्तं देवलो मुनिः}
{उवाच भावगम्भीरः सर्वशास्त्रार्थतत्त्ववित्} % १४

\uvacha{देवल उवाच}

\twolineshloka
{राज्ये नष्टं त्वदीये यच्चतुर्थीसंज्ञितं व्रतम्}
{तेन त्वं रोगयुक्तोऽसि मृतो गच्छसि नारके} % १५

\twolineshloka
{चतुःपदार्थदं पूर्णं सर्वादौ सम्मतं परम्}
{न कृतं चेत् फलैर्हीनं कर्म सर्वं भवेन्नृप} % १६

\twolineshloka
{त्वया यच्च कृतं कर्म नानापुण्यात्मकं महत्}
{चतुःपदार्थहीनं तच्चतुर्थीव्रतहीनकम्} % १७

\twolineshloka
{एवमुक्त्वा महायोगी चतुर्थीव्रतसंश्रिताम्}
{कथां संश्रावयामास राजा श्रुत्वा तमब्रवीत्} % १८

\uvacha{अश्वसेन उवाच}

\twolineshloka
{अहो व्रतस्य माहात्म्यं संश्रुतं परमाद्भुतम्}
{अधुना वद माहात्म्यं गणेशस्य महामुने} % १९

\uvacha{देवल उवाच}

\twolineshloka
{गणेश्वरस्य माहात्म्यं मया वक्तुं न शक्यते}
{उपाधिना युतं राजन् वदिष्यामि स्वरूपकम्} % २०

\twolineshloka
{असितान् मे पितुर्वक्त्राच्छ्रुतं तत्ते वदाम्यहम्}
{सर्वसिद्धिकरं पूर्णं ब्रह्मयोगप्रकाशकम्} % २१

\twolineshloka
{पुराऽहं तपसा युक्तोऽभवंस्तत्र समागतः}
{जैगीषव्यो महायोगी शैवः सहजगो बभौ} % २२

\twolineshloka
{मया सुसत्कृतो योगी संस्थितः पूजितः स्तुतः}
{ममाश्रमे न किञ्चिन् माऽवदत् सोऽपि महायशाः} % २३

\twolineshloka
{तस्य चिह्नं समालोक्य मनसा धारयाम्यहम्}
{अकर्मकारकश्च स्म देहरक्षणतत्परः} % २४

\twolineshloka
{ध्यानादिशून्यभावेन तिष्ठति भ्रष्ट एव च}
{शान्तियोगेन हीनः स्म तथा योगिस्वरूपधृक्} % २५

\twolineshloka
{ततोऽहं तं गृहे त्यक्त्वा गतः सामुद्रके जले}
{मदीयचित्तगं ज्ञात्वा स तत्रैव समाययौ} % २६

\twolineshloka
{अहो केनैव मार्गेणाऽऽगतोऽयं योगिसत्तमः}
{तापसेन मयाऽऽकाशे गच्छता नावलोकितः} % २७

\twolineshloka
{त्यक्त्वा ततस्तं तत्राहमगमं स्वर्गमण्डलम्}
{तत्रापि जैगीषव्यश्च मया दृष्टो महामते} % २८

\twolineshloka
{एवं नानाविधैर्मार्गैः स्वर्गेषु गतवानहम्}
{मया तेषु महायोगी दृष्टः सम्पूजितोऽमरैः} % २९

\twolineshloka
{ततोऽकस्मान् महायोगी सोऽन्तर्धानं चकार ह}
{तद् दृष्ट्वा सिद्धसङ्घांश्चापृच्छं तैः कथितं नृप} % ३०

\twolineshloka
{ब्रह्मलोके गतः सोऽपि तत्र तेन गतिर्भवेत्}
{ततोऽहं खेदसंयुक्तः स्वाश्रमं पुनरागमम्} % ३१

\twolineshloka
{तत्रापि गृहमध्येऽसौ मया दृष्टो महायशाः}
{सम्पूज्य प्रणतो भूत्वाऽवदं योगपरायणः} % ३२

\twolineshloka
{तारयस्व महायोगिन् संसारान् मां कुशिष्यकम्}
{छलनायां समायुक्तं दयया त्वं दयानिधे} % ३३

\uvacha{जैगीषव्य उवाच}

\twolineshloka
{हिंसात्मकं कुरुष्व त्वं कर्म मा यत्नधारकः}
{ब्रह्माहमिति बोधेन शमी दमपरो भव} % ३४

\twolineshloka
{यदाज्ञावशगं सर्वं वर्तते तं भजस्व च}
{तेन योगी त्वमेवेह भविष्यसि न संशयः} % ३५

\twolineshloka
{जैगीषव्यश्चैवमुक्त्वा तत्रैवान्तरधीयत}
{अहं तत्र समासीनो हिंसां त्यक्त्वा तथाऽभवम्} % ३६

\twolineshloka
{ततः स्म पितरः सर्वे महत् कुर्वन्ति रोदनम्}
{अहो कर्म परित्यज्य देवलो न उपेक्षते} % ३७

\twolineshloka
{अस्मान् कः पुष्यति प्राज्ञो निराधारा वयं कृताः}
{देवलेन न सन्देहः क्षुधाविष्टास्तृषा युताः} % ३८

\twolineshloka
{तान् दृष्ट्वा दुःखितोऽहं वै कर्म कर्तुं समुद्यतः}
{मया दृष्टा ओषधयो महत् कुर्वन्ति रोदनम्} % ३९

\twolineshloka
{अस्मानहोऽभयं दत्वा देवलः क्रौर्यमाश्रितः}
{निर्लज्जः पुनरेवाऽसौ छेदयिष्यति दुर्मतिः} % ४०

\twolineshloka
{ततोऽहं नितरां खिन्नो ह्युभयत्रैव सङ्कटात्}
{विचार्य स्वहिते सक्तः शमी दमपरोऽभवम्} % ४१

\twolineshloka
{सर्वाभयस्वभावेन कर्मणा योगसेवया}
{क्रमेण सहजे योग्यऽहं ब्रह्मणि परोऽभवम्} % ४२

\twolineshloka
{तत्र स्वाधीनतां दृष्ट्वा शान्तिहीनोऽभवं पुनः}
{अगमं पितरं तत्र योगीन्द्रैर्वन्दितं स्तुतम्} % ४३

\twolineshloka
{तत्राऽसितं प्रणम्यैवाऽपृच्छं ब्रह्म सनातनम्}
{योगशान्तिप्रदं पूर्णं ततो मां सोऽब्रवीद्वचः} % ४४

\uvacha{असित उवाच}

\twolineshloka
{संयोगाद्विद्धि पुत्र त्वं स्वसंवेद्यात्मकात् किल}
{चतुर्विधं समुत्पन्नं तत्त्यजस्व समाधिदम्} % ४५

\twolineshloka
{अयोगेन च संयोगो दृश्यते योगिभिः कदा}
{संयोगायोगयोर्योगो योगशान्तिप्रदो मतः} % ४६

\twolineshloka
{भजस्व गणराजं त्वं तदर्थं नित्यमादरात्}
{सदा योगस्वरूपं तं शान्तिदं योगिनोऽब्रुवन्} % ४७

\twolineshloka
{संयोगेऽयं गकाराख्यो णकाराख्यस्त्वयोगके}
{तयोर्योगे गणाधीशो ज्ञातव्यो विबुधैः सदा} % ४८

\twolineshloka
{एवमुक्त्वा ददौ मन्त्रं मह्यमेकाक्षरं परम्}
{गाणपत्यः प्रसन्नात्मा पिता मे गणपं भजन्} % ४९

\twolineshloka
{ततस्तं प्रणिपत्यैव गतोऽहं स्वाश्रमे नृप}
{गणेशं पूजयामि स्म ध्यात्वा जपपरोऽभवम्} % ५०

\twolineshloka
{ततः स्वल्पेन कालेन शान्तिं प्राप्तोऽहमेव च}
{योगिवन्द्योऽभवं राजंस्तथापि गणपं भजे} % ५१

\twolineshloka
{गते वर्षे समायातो गणेशो भक्तवत्सलः}
{स्तुतः सम्पूजितस्तत्र मया हृष्टेन चेतसा} % ५२

\twolineshloka
{ततो मां गाणपत्यं स पितृतुल्यं महोदरः}
{कृत्वा स्वानन्दके धाम्नि गतो ब्रह्मपतिः प्रभुः} % ५३

\twolineshloka
{उक्त्वा दशाक्षरं मन्त्रं ददावेवं नृपाय च}
{अन्तर्धाय स्वमात्मानं देवलः स्वाश्रमेऽगमत्} % ५४

\twolineshloka
{ततः स राजशार्दूलो गणेशभजने रतः}
{तत्राऽऽदौ श्रावणी प्राप्ता वरदा या चतुर्थिका} % ५५

\twolineshloka
{तां चकार जनैः सर्वैः पुरवासिभिरादरात्}
{उपोषणसमायुक्तः पञ्चम्यां पारणे रतः} % ५६

\twolineshloka
{ततो दाहविनिर्मुक्तो ज्वरहीनो बभूव ह}
{जनाः सर्वे च रोगाद्यैर्हीना जातास्तदद्भुतम्} % ५७

\twolineshloka
{ततस्तेन महीपृष्ठे विख्यातं तद्व्रतं कृतम्}
{शुक्लकृष्णभवं चक्रुर्व्रतं सर्वे धरातले} % ५८

\twolineshloka
{आरोग्यादिसमायुक्ता हृष्टपुष्टा जनाः बभुः}
{व्रतस्य पुण्ययोगेन पुत्रपौत्रधनान्विताः} % ५९

\twolineshloka
{पुत्रे राज्यं परित्यज्य राजा शान्तिपरायणः}
{सस्त्रीक एकान्ते सोऽभूदभजद्गणनायकम्} % ६०

\twolineshloka
{अन्ते स्वानन्दगो राजा ब्रह्मभूतो बभूव ह}
{क्रमेण सर्वे लोका वै ब्रह्मभूता बभूविरे} % ६१

\twolineshloka
{एवं व्रतस्य माहात्म्यं सर्वसिद्धिप्रदायकम्}
{कथितं ते महाराज पुनः श‍ृणु कथानकम्} % ६२

\twolineshloka
{गुर्जरे पापकर्माकः क्षत्रियो दुर्बलोऽभवत्}
{बाल्यात् प्रारभ्य दुष्टात्मा पापकर्मपरायणः} % ६३

\twolineshloka
{स्त्रीमांसमदिरासक्तो जनानां च पशोस्तथा}
{वधे रतो गुरुं चैव मारयामास मन्दधीः} % ६४

\twolineshloka
{गुरोर्द्रव्यं गृहीत्वा स वनगश्च बभूव ह}
{एवं नानाविधं पापं चकार नित्यमादरात्} % ६५

\twolineshloka
{एकदा ज्वरयुक्तः स बभूव क्षत्रियाधमः}
{अत्यन्तं व्याकुलो जातो देहशुद्धिं न चास्मरत्} % ६६

\twolineshloka
{दैवयोगेन शुक्ला सा श्रावणे सुसमागता}
{चतुर्थी तत्र तेनैवान्नं जलं भक्षितं न च} % ६७

\twolineshloka
{पञ्चम्यां तेन किञ्चिद्वै भक्षितं त्वन्नमादरात्}
{रोगहीनो बभूवाऽपि पुनः पापं समाचरत्} % ६८

\twolineshloka
{कालेन निधनं प्राप्तं तं नेतुं च समागताः}
{दुष्टं सङ्गृह्य गाणेशा ब्रह्मभूतं प्रचक्रिरे} % ६९

\twolineshloka
{अज्ञानव्रतजेनैव पुण्येन क्षत्रियो ययौ}
{स्वानन्दं किं पुनर्ज्ञानिनां कथा का नृपात्मज} % ७०

\twolineshloka
{एवं नाना जनाद्याश्च चतुर्थीव्रतपुण्यतः}
{ब्रह्मभूता मया वक्तुं न शक्यास्ते बभूविरे} % ७१

\twolineshloka
{इदं श्रावणगायाश्च वरदायाः पठेन्नरः}
{माहात्म्यं श‍ृणुयाच्चेद्वा स सर्वं संल्लभेत् फलम्} % ७२

॥ॐ तत्सदिति श्रीमदान्त्ये पुराणोपनिषदि श्रीमन्मौद्गले महापुराणे चतुर्थे खण्डे गजाननचरिते श्रावणशुक्लचतुर्थीव्रतकथनं नामाष्टादशोऽध्यायः॥४.१८॥


\sect{४.१९ --- एकोनविंशोऽध्यायः --- मलमासशुक्लचतुर्थीमाहात्म्यवर्णनम्}

\centerline{॥ श्रीगणेशाय नमः ॥}

\uvacha{दशरथ उवाच}

\twolineshloka
{वद ब्रह्मन् मले मासे चतुर्थी या समागता}
{वरदा चरितं तस्याः सर्वसिद्धिप्रदायकम्} % १

\uvacha{वसिष्ठ उवाच}

\twolineshloka
{अत्र ते कथयिष्यामि चेतिहासं पुरातनम्}
{चतुर्थी महिमायुक्तं श्रवणात् सर्वदं परम्} % २

\twolineshloka
{आन्ध्रे राजा सुषेणश्च चकार राज्यमुत्तमम्}
{नगरे चम्पके भूप शस्त्रास्त्रे पारगो मतः} % ३

\twolineshloka
{नीत्या धर्मेण कीर्त्या च दानेन यश आर्जयत्}
{देवविप्रातिथिप्रेप्सुर्यज्वा व्रतपरायणः} % ४

\twolineshloka
{जित्वा भूमण्डलं सर्वं राज्यं चकार धर्मतः}
{राजानः करदाः सर्वे सेवां चक्रुर्महीपतेः} % ५

\twolineshloka
{तस्य राज्ये महा सर्पा बभक्षुस्तज्जनान् सदा}
{यत्र तत्राऽचरन् क्षुब्धाः सर्पाः परमदारुणाः} % ६

\twolineshloka
{जनैः कुत्र महीपाल प्रगन्तुं नैव शक्यते}
{तथैव दुःखितैः स्थातुं वनेषु च गृहेषु च} % ७

\twolineshloka
{राजा तथैव दुःखार्तो महाभयपरायणः}
{नागानां शान्तये सर्वैः पञ्चमी साधिताऽभवत्} % ८

\twolineshloka
{तथा स्म न शमं यान्ति नागाः परमदारुणाः}
{नानापुण्यानि तीर्थानि जनाश्चक्रुर्विशेषतः} % ९

\twolineshloka
{तथाऽधिकं ययुः क्रूरा नागाः सर्वत्र भूमिषु}
{राजेन्द्रः खगुरुं तत्र जैमिनिं शरणं ययौ} % १०

\twolineshloka
{तं प्रणम्य पुरस्तस्य संस्थितः स कृताञ्जलिः}
{तं मुनिर्मानयामासाऽऽसनदानाद्विशेषतः} % ११

\twolineshloka
{पप्रच्छ विनयेनैव युक्तं किमर्थमागतः}
{सुषेणः स तथोवाच निःश्वस्य मुनिपुङ्गवम्} % १२

\uvacha{सुषेण उवाच}

\twolineshloka
{स्वामिन् सर्पा अपाराश्च धर्षन्तश्च सदा जनान्}
{न स्थातुं शक्यते तत्र भयसङ्कुलितैर्मुने} % १३

\twolineshloka
{नानोपायाः कृतास्तत्र न शान्तिं लेभिरे ततः}
{तदर्थं त्वां महाभाग प्रष्टुमत्र समागतः} % १४

\twolineshloka
{वद सर्पगणानां च शान्तये किं करोम्यहम्}
{विप्र नो चेत् त्वत्समीपे देहत्यागं करोमि वै} % १५

\uvacha{वसिष्ठ उवाच}

\twolineshloka
{एवं पृष्टो महातेजा जैमिनिस्तमुवाच ह}
{स्वशिष्यं निन्दयन् राजंस्तच्छृणुष्व समाहितः} % १६

\uvacha{जैमिनिरुवाच}

\twolineshloka
{चतुर्थीजं महापापिन्नष्टं राज्ये तव व्रतम्}
{तेन त्वं च मृतैः सर्वैर्नरके प्रपतिष्यसि} % १७

\twolineshloka
{चतुर्णां पुरुषार्थानां दातृत्वाद्वरदा मता}
{तथा सङ्कटनाशित्वात् सङ्कष्टी सा प्रकीर्तिता} % १८

\twolineshloka
{यदि व्रतं च सर्वादौ न कृतं चेन्निरर्थकम्}
{कर्महीनं चतुर्भिस्तन्नित्यं कार्या ततः प्रभो} % १९

\twolineshloka
{एवमुक्त्वा चतुर्थ्याश्च माहात्म्यं मुनिसत्तमः}
{कथयामास तच्छ्रुत्वा सुषेणस्तं तथाऽब्रवीत्} % २०

\uvacha{सुषेण उवाच}

\twolineshloka
{कीदृशोऽयं गणेशानो व्रतं यस्य महाद्भुतम्}
{वद तस्य स्वरूपं मे भजिष्यामि महाप्रभुम्} % २१

\uvacha{जैमिनिरुवाच}

\twolineshloka
{न वक्तुं शक्यते राजन् केनाप्येतत् स्वरूपकम्}
{उपाधिना युतं दुण्ढिं वदामि श‍ृणु तत्त्वतः} % २२

\twolineshloka
{अहं पुरा सुशान्त्यर्थं व्यासं च शरणं गतः}
{मह्यं सङ्कथितं तेन साक्षान्नारायणेन च} % २३

\twolineshloka
{तदेव त्वां वदिष्यामि स्वशिष्यं च नृपाधमम्}
{यदि तं भजसि ह्यद्य सर्वसिद्धिप्रदायकम्} % २४

\twolineshloka
{देहदेहिमयं सर्वं गकाराक्षरवाचकम्}
{णकारवाचकं ब्रह्म संयोगायोगरूपकम्} % २५

\twolineshloka
{तयोः स्वामी गणेशस्तु पश्य वेदे महामते}
{चित्ते निवासकत्वाच्चिन्तामणिर्वै स कथ्यते} % २६

\twolineshloka
{चित्तरूपा स्वयं बुद्धिर्भ्रान्तिरूपा महीपते}
{सिद्धिस्तत्र तयोर्योगे प्रलभ्येत तयोः पतिः} % २७

\twolineshloka
{एवमुक्त्वा गणेशस्य ददौ मन्त्रं षडक्षरम्}
{सुषेणाय यथान्यायं विधियुक्तं विशेषतः} % २८

\twolineshloka
{ततस्तेनाभ्यनुज्ञातः स्वपुरं स जगाम ह}
{तत्राऽऽदौ मलमासश्च सम्प्राप्तस्तेन भूमिप} % २९

\twolineshloka
{मलमासे गणेशस्य वरदाख्यं समागतम्}
{सोऽपि शुक्ले चतुर्थीजं व्रतं चक्रे सुभक्तितः} % ३०

\twolineshloka
{नागरैर्विविधैर्लोकैः समीपग्रामसंस्थितैः}
{तत्कृतं विधिना राजन् सर्वसिद्धिप्रदायकम्} % ३१

\twolineshloka
{ततस्तेन सुषेणेन प्रशस्तं भूमिमण्डले}
{कृतं ततो जनाः सर्वे चक्रुर्हर्षसमन्विताः} % ३२

\twolineshloka
{अभवन् व्रतपुण्येन सर्पा अन्तर्हिता बभुः}
{लोका रोगादिभिः सर्वे वर्जिता हर्षसंयुताः} % ३३

\twolineshloka
{पुत्रपौत्रादिसंयुक्ता धनधान्यसमन्विताः}
{नानासुखेषु संसक्ता अभवंस्तत्र ते ततः} % ३४

\twolineshloka
{राजा गणपतिं नित्यमभजन्नान्यचेतसा}
{षडक्षरेण मन्त्रेण विधियुक्तेन पूजयन्} % ३५

\twolineshloka
{पुत्रे राज्यं विनिक्षिप्य सस्त्रीकः स वनं ययौ}
{तत्र विघ्नेशमेवं सोऽपूजयञ्जपतत्परः} % ३६

\twolineshloka
{अन्ते स्वानन्दगो भूत्वा ब्रह्मभूतो बभूव ह}
{जनाः सर्वे व्रते संस्थाः शुक्लकृष्णभवे परे} % ३७

\twolineshloka
{स्वान्ते क्रमेण ते सर्वे ब्रह्मभूता बभूविरे}
{व्रतस्यैव प्रभावेण न किञ्चिद् दुर्लभं भवेत्} % ३८

\twolineshloka
{अन्यच्छृणु महीपाल चरितं मलमासगम्}
{पुण्यं शुक्लचतुर्थीजं श्रवणात् सर्वदं भवेत्} % ३९

\twolineshloka
{हस्तिनापुरवासी च क्षत्रियः कः सुदारुणः}
{बाल्यात् प्रभृति वै सोऽपि पापाचारो बभूव ह} % ४०

\twolineshloka
{द्रव्यलोभार्थमेवं स्वजनकं स जघान ह}
{परस्त्रीलालसः पूर्णश्चौरकर्मपरोऽभवत्} % ४१

\twolineshloka
{एवं नानाविधं पापं चक्रेऽसौ दुर्मतिः सदा}
{वने स्थितः कदाचिद्वै जनान् जघ्ने महाखलः} % ४२

\twolineshloka
{एतस्मिन्नन्तरे तत्र सर्पेणैव महीपते}
{दष्टः सोऽपि भयोद्विग्नो गृहे गन्तुं मनो दधे} % ४३

\twolineshloka
{ततो विषेण संव्याप्तो देहस्तेन पपात ह}
{दैवयोगेन तत्राऽसौ जलान्नरहितोऽभवत्} % ४४

\twolineshloka
{शुक्लपक्षे मले मासे चतुर्थी सा तिथिस्तदा}
{तेन प्राप्ता तत्र भूप अज्ञानेन व्रतं कृतम्} % ४५

\twolineshloka
{पञ्चम्यां स मृतस्तत्र वने घोरे महाखलः}
{स्वानन्दे गणपं दृष्ट्वा ब्रह्मभूतो बभूव ह} % ४६

\twolineshloka
{नाना जना व्रतेनैव ब्रह्मभूता बभूविरे}
{इह भुक्त्वाऽखिलान् भोगानन्ते वक्तुं न शक्यते} % ४७

\twolineshloka
{इदं चतुर्थीमाहात्म्यं मलमासे श‍ृणोति चेत्}
{शुक्लपक्षे लभेत् सोऽपि पठेद्वा सर्वमञ्जसा} % ४८

॥ॐ तत्सदिति श्रीमदान्त्ये पुराणोपनिषदि श्रीमन्मौद्गले महापुराणे चतुर्थे खण्डे गजाननचरिते मलमासशुक्लचतुर्थीचरितवर्णनं नाम एकोनविंशोऽध्यायः॥४.१९॥


\sect{४.२० --- विंशोऽध्यायः --- माघकृष्णचतुर्थीमाहात्म्यवर्णनम्}

\centerline{॥ श्रीगणेशाय नमः ॥}

\uvacha{दशरथ उवाच}

\twolineshloka
{संश्रुत्य वरदाया वै माहात्म्यं सर्वदं परम्}
{अत्यन्तं कृतकृत्योऽहं जातो नास्त्यत्र संशयः} % १

\twolineshloka
{नानेन सदृशं किञ्चिद्व्रतं सर्वार्थदं परम्}
{धन्यास्ते पुरुषा विप्र चतुर्थीव्रतकारकाः} % २

\twolineshloka
{अधुना सङ्कटायाश्च माहात्म्यं वद विस्तरात्}
{न तृप्यामि सुधारूपां कथां श्रुत्वा समासतः} % ३

\uvacha{मुद्गल उवाच}

\twolineshloka
{एवं पृष्टो महायोगी वसिष्ठस्तमुवाच ह}
{तच्छृणुष्व प्रजानाथ श्रवणानन्ददायकम्} % ४

\uvacha{वसिष्ठ उवाच}

\twolineshloka
{तव भावं विदित्वाऽहं सन्तुष्टो नृपसत्तम}
{पुण्यवानसि येन त्वं गणेशे भक्तिमानसि} % ५

\twolineshloka
{न ह्यल्पपुण्ययोगेन गणेशस्य महात्मनः}
{कथायां जायते प्रीतिर्धन्योऽस्यत्र न संशयः} % ६

\twolineshloka
{कथां श‍ृणु महाभाग चतुर्थीसंश्रितां पराम्}
{कृष्णपक्षे महापुण्यां सर्वसिद्धिप्रदायिनीम्} % ७

\twolineshloka
{अत्र ते कथयिष्यामि चेतिहासं पुरातनम्}
{महापुण्यप्रदः पूर्णश्रवणात् पठनाद्भवेत्} % ८

\twolineshloka
{सहस्राख्ये पुरे राजन् शूरसेनो महीपतिः}
{सभायां गानसंयुक्तो बभूवे भूपसेवितः} % ९

\twolineshloka
{तत्राऽकस्माद्विमानं वै पतितं ज्वलनप्रभम्}
{तद् दृष्ट्वा परमाश्चर्यं ज्ञातुं दूतानचोदयत्} % १०

\twolineshloka
{दूतैः संज्ञापितो राजा विमानं द्रष्टुमुत्सुकः}
{स ययौ नगरप्रान्ते पुरवासिभिरावृतः} % ११

\twolineshloka
{तत्रेन्द्रं देवयुक्तं स दृष्ट्वा विस्मितमानसः}
{प्रणनाम पदा गत्वा दण्डवत् पृथिवीतले} % १२

\twolineshloka
{पुनरुत्थाय राजाऽसौ कृत्वा करपुटं शनैः}
{हर्षयुक्त उवाचेदं वचनं भक्तिसंयुतः} % १३

\uvacha{शूरसेन उवाच}

\twolineshloka
{धन्यं जन्म वयो दानं पितरौ नगरादिकम्}
{कृतकृत्योऽस्मि देवेन्द्र दर्शनात्ते सहानुगः} % १४

\twolineshloka
{वद कुत्र विभो गन्तुं कुतो व्रजसि वा स्थलात्}
{लालसस्ते विमानं च कथं निपतितं भुवि} % १५

\twolineshloka
{न जाने केन पुण्येन सर्वेषां दर्शनं च ते}
{जातं देवसमूहेन वदाज्ञां किं करोम्यहम्} % १६

\twolineshloka
{विमानचलने यत्नं कुरु त्वं देवनायक}
{केन पापेन ते यानं पतितं तद्वदस्व मे} % १७

\twolineshloka
{एवं पृष्टो महातेजा महेन्द्रस्तमुवाच ह}
{विनीतं भक्तिसंयुक्तं राजेन्द्र हर्षसंयुतः} % १८

\uvacha{इन्द्र उवाच}

\twolineshloka
{कैवर्तको महापापी नामा विघ्नेशमाभजत्}
{मुद्गलेनोपदिष्टश्च नाममन्त्रपरायणः} % १९

\twolineshloka
{सहस्रे तद्वरेणाऽसौ वर्षेषु ब्राह्मणोत्तमः}
{भ्रूमध्यान्निःसृता शुण्डा भृशुण्डी चाऽभवन् मुनिः} % २०

\twolineshloka
{तस्य दर्शनमात्रेण पुनर्जन्म न विद्यते}
{दर्शनार्थं गतोऽहं तु सम्पूज्येह समागमम्} % २१

\twolineshloka
{अत्रापि पापरूपस्य दृशा वैश्यस्य कुष्ठिनः}
{विमानं पतितं भूमौ तव दूतस्य मानद} % २२

\twolineshloka
{यदि सङ्कष्टिका राजंश्चतुर्थी कृष्णपक्षगा}
{साधिता व्रत भावन तस्याः पुण्यं प्रदीयताम्} % २३

\twolineshloka
{तदा विमानमेवेदं चलेदत्र न संशयः}
{नान्यथा पुरुषार्थश्च विमानचलनं भवेत्} % २४

\twolineshloka
{एतस्मिन्नन्तरे तत्र विमानं च द्वितीयकम्}
{समागतं गणेशस्य दूतयुक्तं महाद्भतम्} % २५

\twolineshloka
{तत्रैव नगरे भूप चाण्डाली पापकारिणी}
{अन्धा कुष्ठयुताऽत्यन्तं कृमिभारसमाकुला} % २६

\twolineshloka
{पूयशोणितदिग्धाङ्गा दुर्गधेन समावृता}
{बभूव भिक्षया सा वैरता जठरपोषणे} % २७

\twolineshloka
{पौषकृष्णतृतीयायां निशा किञ्चावशेषिता}
{मद्यपानं तया तत्र कृतं सुष्वाप निर्भया} % २८

\twolineshloka
{मद्यपानबलेनैव निद्रां लेभे तथाऽन्त्यजा}
{चतुर्थ्यामुदिते चन्द्रे जाते जागरिताऽभवत्} % २९

\twolineshloka
{ततोऽपि क्षुधयाऽऽविष्टा रात्री भिक्षार्थमेव च}
{बभ्राम तत्र केनाऽपि दत्तान्नं बुभुजे च सा} % ३०

\twolineshloka
{पञ्चम्यां सा मृता तत्र नृप सङ्गृह्य गाणपाः}
{विमानेन ययुस्ते वै स्वानन्दे व्रतकारिणीम्} % ३१

\twolineshloka
{गच्छन्ती वायुना स्पृष्टा विमानेन्त्यजजा नृप}
{स वायुः सहसा गत्या स्पृशदिन्द्रविमानकम्} % ३२

\twolineshloka
{व्रतकारिशरीरस्य स्पृष्टो वायुः सुपुण्यवान्}
{विमानं चालयामास देवेन्द्रस्य महाद्भुतम्} % ३३

\twolineshloka
{तद् दृष्ट्वा विस्मिताः सर्वे देवा मुनिगणादयः}
{ज्ञानदृष्टया च तज्ज्ञात्वा शशंसू राजसत्तमम्} % ३४

\twolineshloka
{इन्द्रे स्वर्ग गते देवै राजा स्वनगरं ययौ}
{विस्मितस्तं प्रणम्यैव वसिष्ठं मां समानयत्} % ३५

\twolineshloka
{मया विधियुतं तस्योपदिष्टं मुख्यकं व्रतम्}
{चकार स यथान्यायमङ्गारकयुतं पुरा} % ३६

\twolineshloka
{माघे कृष्णचतुथ्यों स कृत्वा व्रतमनुत्तमम्}
{पुनः सम्पूज्य पञ्चम्यां गणेशं संस्थितोऽभवत्} % ३७

\twolineshloka
{एतस्मिन्नन्तरे तत्र विमानं सहसाऽऽगतम्}
{गणेशदूतसंयुक्तं शूरसेनपुरे महत्} % ३८

\twolineshloka
{विमानाद्गण नाथस्य दूता उत्तीर्य तं नृपम्}
{जगुर्वतप्रभावेण चल राजन् गणेश्वरम्} % ३९

\twolineshloka
{दूतानां वचनं श्रुत्वा राजा विस्मितमानसः}
{प्रणिपत्य प्रपूज्यैव तानुवाच सुहर्षितः} % ४०

\uvacha{शूरसेन उवाच}

\twolineshloka
{आरभ्य जन्मतो देवा अद्यावधि मया हठात्}
{यत्किचिन्नागरैः सार्धं सदा भुक्तं शुभाशुभम्} % ४१

\twolineshloka
{अधुना तान् परित्यज्य कथं यामि गजाननम्}
{ततस्तं गणराजस्य दूता ऊचुः श‍ृणुष्व तत्} % ४२

\uvacha{गाणपत्या ऊचुः}

\twolineshloka
{नागरैः सह राजेन्द्र चल त्वं विघ्ननायकम्}
{यथाविधि व्रतस्यैव प्रभावेण महामते} % ४३

\twolineshloka
{यथाशास्त्रं व्रतस्यास्य महिमा केन गम्यते}
{अतो विश्वस्य चोद्धारे समर्थस्तु सुपुण्यतः} % ४४

\twolineshloka
{ततोऽतिहर्षितो राजा चतुर्वर्णजनैः सह}
{नगरस्थैर्विमानं स समारुह्य स्थितोऽभवत्} % ४५

\twolineshloka
{ततो गाणेशकैस्तत्र विमानं चालितं नृप}
{न चचाल जडीभूतं ध्यानयुक्ता बभूविरे} % ४६

\twolineshloka
{ज्ञात्वा तं कुष्ठिनं त्यक्तुं गाणेशाश्च समागताः}
{तान् प्रणम्य जगादाऽसौ राजेन्द्रो रोदनाकुलः} % ४७

\twolineshloka
{महापापी च वैश्योऽयं यस्य दृष्टया पपात ह}
{विमानं सुरनाथस्य तथापि श‍ृणुत प्रियाः} % ४८

\twolineshloka
{किमनेन कृतं पापं मां ततो वदत ह्यहम्}
{प्रायश्चित्तं करिष्याम्यनेन गच्छामि संयुतः} % ४९

\twolineshloka
{ततस्तं गणनाथस्य दूता ऊचुः प्रहर्षिताः}
{श‍ृणु राजन्नयं वैश्यः पूर्वजन्मनि वाडवः} % ५०

\twolineshloka
{बुधनामा महापापी बाल्यात् प्रारभ्य सर्वदा}
{माता पतिव्रता चास्य शाकिनी परिकीर्तिता} % ५१

\twolineshloka
{दूर्वो नाम पिता चास्य तपस्वी वेदपारगः}
{पत्नी पतिव्रता प्रोक्ता सावित्री कुलजाऽभवत्} % ५२

\twolineshloka
{यौवनस्थः स्त्रियं त्यक्त्वा परस्त्रीलालसोऽभवत्}
{कदाचिद्गौडपुर्यां का वेश्या रूपवती ययौ} % २३

\twolineshloka
{तां दृष्ट्वा विस्मितोऽत्यन्तं तया रेमे निरन्तरम्}
{हृत्वा गृहस्थं द्रव्यं स ददौ तस्यै विशेषतः} % ५४

\twolineshloka
{पात्रभूषण कायं वै हृत्वा परगृहस्थितम्}
{चौर्यमद्यादिसंयुक्तस्तया रेमे सुदुर्मतिः} % ५५

\twolineshloka
{कदाचित् स निशायां वै नागतः स्वगृहे नृप}
{तदर्थं मुनिमुख्योऽसौ दूर्वो बभ्राम तत्पुरे} % ५६

\twolineshloka
{जनाः केऽपि न विप्रं त्वकथयस्तं विशेषतः}
{पुत्रस्य दुर्विनीतस्य चरित्रं क्रोधभीतितः} % ५७

\twolineshloka
{गृहे गृहे स्वपुत्रं स नालभच्च ततो ययौ}
{निशीथे स्वगृहे प्राप्ते पप्रच्छ वनितां द्विजः} % ५८

\twolineshloka
{सोवाच नागतः स्वामिन् पुत्रो मे स्वगृहे प्रभो}
{एकपुत्रलेहवशात् पुनस्तं सो गवेषयत्} % ५९

\twolineshloka
{निशीथे निद्रित लोके जनान् पप्रच्छ मार्गगान्}
{वुधः कुत्र प्रदृष्टो मामेकं वदत पुत्रकम्} % ६०

\twolineshloka
{एवं नाना जनास्तेन पृष्टास्तेन जगुः सुतम्}
{ततो भीमोऽत्यजो वृद्धो मिलितस्तं जगाद सः} % ६१

\twolineshloka
{तेनैव कथितं स्पष्टं पुत्रो वेश्यागृहे स्थितः}
{मद्यपः किं पृच्छसि त्वं तपस्विन् जातिदूषणम्} % ६२

\twolineshloka
{तस्य तद्वचनं श्रुत्वा विस्मितो मुनिपुङ्गवः}
{वेश्यागृहे स्वपुत्रं तं दृष्ट्वोवाच मदान्वितम्} % ६३

\twolineshloka
{किं त्वं मद्यप वेश्याया गृहे तिष्ठसि दूषक}
{अधुना देहमुत्सृज्य व्रजस्व यममन्दिरे} % ६४

\twolineshloka
{पुन्नामनरकात् त्राता पुत्रस्तेन प्रकीर्तितः}
{नरकप्रद एवं मे किमर्थं पुत्र आगतः} % ६५

\twolineshloka
{एवं पितुर्वचः श्रुत्वा बुधः क्रोधसमन्वितः}
{लत्तया ताडयामास पितरं च पुनः पुनः} % ६६

\twolineshloka
{वृद्धः क्षुधातुरः सोऽपि मर्मस्थाने प्रहारतः}
{ममार तं प्रगृह्याऽसौ बुधश्चिक्षेप वै बहिः} % ६७

\twolineshloka
{प्रभाते स्वगृहे पुत्रमागतं सा बुधं ततः}
{जननी लेहसंयुक्ता पप्रच्छ क स्थितं त्वया} % ६८

\twolineshloka
{गवेषितुं च ते पुत्र पिता त्वां प्रजगाम ह}
{तमानय महाभाग ततः स्नानं समाचर} % ६९

\twolineshloka
{मातुर्वचनमेवं स बुधः श्रुत्वा पुनः पुनः}
{क्रोधयुक्तः स्वयष्ट्या तां ताडयामास मस्तके} % ७०

\twolineshloka
{मस्तकः स्फुटितस्तस्याः सा ममार तपस्विनी}
{बहिः प्रक्षिप्य तां वेश्यागृहमागात् स हर्षितः} % ७१

\twolineshloka
{अदहंस्तौ जनाः सन्तो ब्राह्मणी ब्राह्मणं तथा}
{पितृगौरवभावेन राजा तं न शशास ह} % ७२

\twolineshloka
{पुनर्गृहागतं सा तं सावित्री दुःखसंयुता}
{उवाच किं कृतं स्वामिन् पितुर्मातुर्वधात्मकम्} % ७३

\twolineshloka
{वेश्यां त्वं मां परित्यज्य किमर्थं गच्छसि प्रभो}
{सर्वावयवसम्पूर्णा सुन्दरी धर्मपालिनीम्} % ७४

\twolineshloka
{चतुरा करजेष्वेव नरदेहः सुदुर्लभः}
{ज्ञानकर्मादिसंयुक्तस्तत्र ब्राह्मण्यकं कुतः} % ७२

\twolineshloka
{ब्राह्मणत्वं समासाद्य पापमिच्छति दुर्मतिः}
{निधिं त्यक्त्वा स्वहस्तस्थं विष्ठाभक्षणलालसः} % ७६

\twolineshloka
{तथा कृतं त्वया नाथ तेनाऽहं दुःखिता भृशम्}
{कां गतिं च त्वया सार्धं गमिष्यामि महामते} % ७७

\twolineshloka
{एवमाकर्ण्य दुष्टात्मा तां जघान स्वयष्टितः}
{काष्ठैः सा मर्मभेदेन ममार स्वर्गगाऽभवत्} % ७८

\twolineshloka
{एवं स्वल्पे गते काले बुधोऽगात् स्वगुरोहे}
{अयभद्गुरुपत्नीमेकाकिनी तां सुदुर्मतिः} % ७९

\twolineshloka
{दारुणानि च पापानि एवं नानाविधानि सः}
{चकारान्ते गृहे याम्ये यातनां बुभुजे ततः} % ८०

\twolineshloka
{भुक्तभोगः स वैश्योऽयं जातः कुष्ठसमन्वितः}
{पापाचरणमत्रापि कुरुते दारुणं महत्} % ८१

\twolineshloka
{द्रव्यलोभार्थमेवायं वने गत्वा द्विजादिकान्}
{हन्ति नित्यं सतीनां च स वने दूषकोऽभवत्} % ८२

\twolineshloka
{शूरसेन अतस्त्यक्त्वैनं याहि त्वं गजाननम्}
{अस्य स्पर्शन राजेन्द्र सचैलं स्नानमापतेत्} % ८३

\twolineshloka
{दूतानां वचनं श्रुत्वा कम्पितो राजसत्तमः}
{जगाद तान् प्रणम्यैव हा हा कृत्वा सुदारुणम्} % ८४

\uvacha{शूरसेन उवाच}

\twolineshloka
{पापानां गणना नास्ति प्रायश्चित्तं न विद्यते}
{अधुना तादृशं चात्र मया किं क्रियते गणाः} % ८५

\twolineshloka
{कृपया सर्वपापानां प्रायश्चित्तं महाद्भतम्}
{कथ्यतां तत् करिष्यामि देहपातावधिं किल} % ८६

\twolineshloka
{ततस्तं गाणपत्यास्तेऽब्रुवन् हर्षसमन्विताः}
{गजानन इति प्राज्ञ श्रावयस्व जनाधमम्} % ८७

\twolineshloka
{चतुर्वेदसमुद्भूतं सारं ब्रह्ममुखोद्गतम्}
{चतुर्मुखैश्च सङ्गीतं सुयोगन गजाननम्} % ८८

\twolineshloka
{तेषां वचनमाकर्ण्य शूरसेनः प्रहर्षितः}
{गजाननेति वैश्यस्याऽजपत् कर्णे स सादरः} % ८९

\twolineshloka
{श्रुत्वा पापविनिर्मुक्तो वैश्यः स्वर्णतनुर्यथा}
{कुष्ठहीनः स शोभाभियुक्तो वै हर्षितोऽभवत्} % ९०

\twolineshloka
{ततः सर्वैः समायुक्तो नगरस्थैविशेषतः}
{चतुराकरजै राजा ययौ विघ्नेश मादरात्} % ९१

\twolineshloka
{स्वानन्दे गणपं दृष्ट्वा ब्रह्मभूताश्च जन्तवः}
{बभूवुः स यथा राजा गाणपत्यो बभूव ह} % ९२

\twolineshloka
{ब्रह्मभूतः स राजेन्द्रो व्रतपुण्यप्रभावतः}
{जन्तुभिः सहितश्चातः कोऽओ वर्णयितुं व्रतम्} % ९३

\twolineshloka
{चतुर्विधं जगत् सर्व सङ्कष्टं सम्मतं नृप}
{तत्त्यक्त्वा व्रतपुण्येन ब्रह्मभूतो भवेन्नरः} % ९४

\twolineshloka
{इदं माघचतुर्थ्यास्तु सङ्कष्टयाः संश‍ृणोति यः}
{माहात्म्यं वापि पठति सर्वार्थ लभते परम्} % ९५

॥ॐ तत्सदिति श्रीमदान्त्ये पुराणोपनिषदि श्रीमन्मौद्गले महापुराणे चतुर्थे खण्डे गजाननचरिते माघकृष्णचतुर्थीमाहात्म्यवर्णनं नाम विंशोऽध्यायः॥४.२०॥


\sect{४.२१ --- नामैकविंशतितमोऽध्यायः --- फाल्गुनकृष्णचतुर्थीमाहात्म्यवर्णनम्}

\centerline{॥ श्रीगणेशाय नमः ॥}

\uvacha{दशरथ उवाच}

\twolineshloka
{ब्रह्मन् श्रुत्वा च वै माघी सङ्कष्टीजं सुविस्मितः}
{माहात्म्यं सर्वदं पुण्यं सर्वसङ्कटहारकम्} % १

\twolineshloka
{फाल्गुने कृष्णपक्षे या चतुर्थी सङ्कटी मता}
{तस्याश्चरित्रमेवं मे कथयस्व कृपानिधे} % २

\uvacha{वसिष्ठ उवाच}

\twolineshloka
{पौलस्त्यो रावणः प्रोक्तस्तपस्तप्त्वा सुदारुणम्}
{वरं लब्ध्वा त्रिलोकस्य राज्यं चक्रे महाबलः} % ३

\twolineshloka
{प्रधानेषु समाक्षिप्य राज्यं राक्षससत्तमः}
{एकान्ते निर्जने गत्वा ध्यानसंस्थो बभूव ह} % ४

\twolineshloka
{वेदोपनिषदि प्रोक्तं ब्रह्म वेदान्तपारगम्}
{ध्यायति स्म विचारज्ञो ज्ञानयोगपरायणः} % ५

\twolineshloka
{एवं बहौ गते काले न लेभे ज्ञानमुत्तमम्}
{तदा खेदसमायुक्तः सोऽतिष्ठच्छङ्करं स्मरन्} % ६

\twolineshloka
{ततः शिवेन तत्रैव प्रेषितो नारदो मुनिः}
{तं दृष्ट्वा नारदं रक्षः प्रणनाम कृताऽन्जलिः} % ७

\twolineshloka
{पप्रच्छ खेदसंयुक्तो नारदं योगिनां वरम्}
{स्वामिन् ज्ञानप्रदं किञ्चिद्वदस्व करुणायुतः} % ८

\twolineshloka
{ज्ञानार्थं ध्याननिष्ठोऽहं नित्यं तिष्ठामि चादरात्}
{न लेभे तन् महत् ज्ञानं किं करोमि महामुने} % ९

\uvacha{नारद उवाच}

\twolineshloka
{शिवेन प्रेषितोऽहं वै त्वदर्थं राक्षसोत्तम}
{श‍ृणु ज्ञानप्रदं पूर्णं वाक्यं तत्कुरु सादरः} % १०

\twolineshloka
{चतुर्विधं जगद्ब्रह्म सङ्कष्टं सम्मतं बुधैः}
{तन्नाशार्थं व्रतं मुख्यं सङ्कष्टीसंज्ञकं कुरु} % ११

\twolineshloka
{एवमुक्त्वा चतुर्थ्याः स माहात्म्यं नारदोऽब्रवीत्}
{तच्छ्रुत्वा रावणस्तं चाब्रवीद्धर्षसमन्वितः} % १२

\uvacha{रावण उवाच}

\twolineshloka
{कीदृशोऽयं गणाधीशस्तस्य ज्ञानं वद प्रभो}
{यस्य व्रतं चतुर्णां च पदार्थानां प्रदायकम्} % १३

\twolineshloka
{न कृतं चेच्चतुर्णां तन्नाशकं नाऽत्र संशयः}
{सर्वादौ सम्मतं विप्र सर्वसिद्धिप्रदायकम्} % १४

\twolineshloka
{एवं पृष्टो महायोगी नारदस्तमुवाच ह}
{हर्षेण महता युक्तो रावणं लोकरावणम्} % १५

\uvacha{नारद उवाच}

\twolineshloka
{महज्ज्ञानं कथयितुं गणेशस्य न शक्यते}
{उपाधिना वदिष्यामि राक्षसाधिप तच्छृणु} % १६

\twolineshloka
{गणः समूहरूपश्च समूहा ब्रह्मवाचकाः}
{बाह्यान्तरादियोगेऽयं समूहो जायते यतः} % १७

\twolineshloka
{देहिदेहमयं ब्रह्म गकाराक्षरवाचकम्}
{संयोगायोगरूपं यण्णकाराक्षरगं मतम्} % १८

\twolineshloka
{तयोर्योगे गणेशश्च स्वामी सर्वत्र सम्मतः}
{तं भजस्व विधानेन तदा शान्तिमवाप्स्यसि} % १९

\twolineshloka
{एवमुक्त्वा गणेशस्य ददौ मन्त्रं दशाक्षरम्}
{रावणाय महायोगी विधियुक्तं तमब्रवीत्} % २०

\twolineshloka
{यदि त्यजसि दैत्येन्द्र तदा भ्रंशमवाप्स्यसि}
{अतो गणेशमन्त्रो न त्याज्यो रक्षः कदाचन} % २१

\twolineshloka
{एवमुक्त्वा महातेजा नारदोऽन्तर्दधे स्वयम्}
{रावणस्य तदा प्राप्ता सङ्कष्टी फाल्गुनी नृप} % २२

\twolineshloka
{सा कृता तेन हर्षेण विधियुक्तेन कर्मणा}
{व्रतपुण्यप्रभावेण स्फूर्तिः प्राप्ता च तत्क्षणात्} % २३

\twolineshloka
{चकार स ततः शुक्लां कृष्णां वै राक्षसाधिपः}
{जनान् विशेषेणाबोध्य सह तैर्गणपे रतः} % २४

\twolineshloka
{ततो ज्ञानं समालब्धं रावणेन महात्मना}
{ययौ स्वनगरे दैत्यै राज्यं चक्रे मदान्वितः} % २५

\twolineshloka
{दुष्टसङ्गतियोगेन क्रमेण ज्ञानमुत्तमम्}
{नष्टं तस्य सुदुष्टस्य स्त्रीमांसादिपरोऽभवत्} % २६

\twolineshloka
{अहं गणेशरूपश्च नाऽन्यो जगति वर्तते}
{न पापपुण्यभोक्तृत्वं कस्य पूजनमाचरे} % २७

\twolineshloka
{ततो ज्ञानमदेनैव त्यक्तो मन्त्रः सुखप्रदः}
{व्रतं त्यक्तमिदं पुण्यं पूजा त्यक्ता विशेषतः} % २८

\twolineshloka
{नष्टं ज्ञानं स्थितिर्नष्टा राक्षसो राक्षसोऽभवत्}
{धर्मलोपे रतोऽत्यन्तं चकार कर्मखण्डनम्} % २९

\twolineshloka
{तेनैव दोषयुक्तोऽभूद्धतो रामेण तत्क्षणात्}
{राक्षसैः स्वजनैः सार्धं तव पुत्रेण धीमता} % ३०

\uvacha{दशरथ उवाच}

\twolineshloka
{अहं वन्ध्यश्च विप्रेश रावणश्च प्रवर्तते}
{किमिदं भाषसे स्वामिन् कूटरूपं भ्रमप्रदम्} % ३१

\uvacha{वसिष्ठ उवाच}

\twolineshloka
{कल्पे कल्पे स रामो वै तव पुत्रोऽभवत्प्रभुः}
{जघान रावणं वीरो गाणपत्यबलान्वितः} % ३२

\twolineshloka
{अन्यच्छृणु चरित्रं त्वं फाल्गुने सङ्कटीभवम्}
{सर्वपापहरं पूर्णं भुक्तिमुक्तिप्रदायकम्} % ३३

\twolineshloka
{महाराष्ट्रे द्विजः कश्चित् पापकर्मा बभूव ह}
{ब्राह्मणत्वं परित्यज्य चाण्डाल्यां निरतोऽभवत्} % ३४

\twolineshloka
{चाण्डालैर्योनिसम्बन्धं पुत्रपुत्रीसमुद्भवम्}
{चकार मन्दधीः सोऽपि मद्यमांसपरायणः} % ३५

\twolineshloka
{स कदाचिद्वने संस्थो जनान् हन्तुं समुद्यतः}
{द्रव्यलोभी महापापी परस्त्रीलालसोऽभवत्} % ३६

\twolineshloka
{तत्र फाल्गुनमासे सा चतुर्थी कृष्णगाऽऽगता}
{संस्थितः पर्वतद्रोण्यां बभूव ब्राह्मणोऽधमः} % ३७

\twolineshloka
{तत्र कश्चिन्नृपः सैन्यैश्चतुरङ्गैः समागतः}
{तद्भयात् सोऽपि तत्रैव संस्थितोऽन्नजलैर्विना} % ३८

\twolineshloka
{महत् सैन्यं नृपस्यैव मार्गे गमनकारकम्}
{किञ्चिद्दिवसशेषे तत् सम्पूर्णगतमाभवत्} % ३९

\twolineshloka
{ततः सोऽपि बहिर्वीक्ष्य निःसृतो भयवर्जितः}
{जगाम स्वगृहं चन्द्रोदये दुष्टः क्षुधातुरः} % ४०

\twolineshloka
{बभक्षान्नं स्वपुत्रैः स रात्रौ सुप्तो निजालये}
{तत्र सर्पेण दष्टश्च ममार नृप दुर्मतिः} % ४१

\twolineshloka
{ततो गणेशदूतैः स नीतः स्वानन्दके पुरे}
{दृष्ट्वा विघ्नेश्वरं तत्र ब्रह्मभूतो बभूव ह} % ४२

\twolineshloka
{अज्ञानव्रतपुण्येन विधिहीनेन भो नृप}
{मुक्तश्चतुर्भिरेवं स किं पुनर्ज्ञानिनां परम्} % ४३

\twolineshloka
{एतादृशा महाभागा विधियुक्ता विधिं विना}
{व्रतपुण्यप्रभावेण ब्रह्मभूता बभूविरे} % ४४

\twolineshloka
{तत्रैवं कति ते ब्रूयां नालं वर्षायुतैरपि}
{भवामि नृप माहात्म्यं सङ्क्षेपेण निरूपितम्} % ४५

\twolineshloka
{इदं फाल्गुनमासे या चतुर्थी कृष्णगा मता}
{तस्याः श‍ृणोति माहात्म्यं पठेद्वा सर्वमालभेत्} % ४६

॥ॐ तत्सदिति श्रीमदान्त्ये पुराणोपनिषदि श्रीमन्मौद्गले महापुराणे चतुर्थे खण्डे गजाननचरिते फाल्गुनकृष्णचतुर्थीवर्णनं नामैकविंशतितमोऽध्यायः॥४.२१॥


\sect{४.२२ --- द्वाविंशोऽध्यायः --- चैत्रकृष्णचतुर्थीमाहात्म्यवर्णनम्}

\centerline{॥ श्रीगणेशाय नमः ॥}

\uvacha{दशरथ उवाच}

\twolineshloka
{चैत्रकृष्णचतुर्थी या तां मे वद महामुने}
{न तृप्यामि गणेशस्य कथां श्रुत्वा सुखप्रदाम्} % १

\uvacha{वसिष्ठ उवाच}

\twolineshloka
{अत्र ते कथयिष्यामि चेतिहासं पुरातनम्}
{चतुर्थीमहिमायुक्तं सर्वसिद्धिप्रदायकम्} % २

\twolineshloka
{कलिङ्गे नृपवर्यश्चोग्रसेनो धर्मतत्परः}
{यज्वा दानपरो नित्यं धर्मात्मा शंसितव्रतः} % ३

\twolineshloka
{शस्त्रास्त्रकुशलो धीमान् सत्यवाक् नीतिसंयुतः}
{जित्वा राजगणान् सर्वांश्चकार राज्यमुत्तमम्} % ४

\twolineshloka
{तस्य राज्ये नृपश्रेष्ठ बभक्षुः सर्वमानवान्}
{अपारास्तत्पुरे व्याघ्राः समागत्य महाबलाः} % ६

\twolineshloka
{उग्रसेनश्च शस्त्रैस्तान् जघान नृपसंयुतः}
{तथा स्म न शमं यान्ति व्याघ्रा दैवोद्भवा यथा} % ६

\twolineshloka
{लोकाः सम्पीडिता व्याघ्रैर्निनिन्दुस्तं नृपं तदा}
{पापकर्मा नृपोऽयं वै प्रजाभ्यो दुःखदायकः} % ७

\twolineshloka
{धर्मयुक्तो यदा राजा प्रजास्तत्र सुखे रताः}
{किं कर्तव्यं प्रजाभिश्च दुष्टे राजनि नित्यदा} % ८

\twolineshloka
{स राजर्षिर्दूतमुखाच्छ्रुत्वा दुःखसमन्वितः}
{ययौ वनं प्रधानेषु राज्यं त्यक्त्वा सुदारुणम्} % ९

\twolineshloka
{तत्र गत्वा स एकान्ते तताप तप उत्तमम्}
{सूर्यं ध्यात्वा च सौरैस्तं तोषयामास नित्यदा} % १०

\twolineshloka
{निराहारपरो राजा गते वर्षे बभूव ह}
{अस्थिचर्मावशेषः स तथाऽपि तप आचरत्} % ११

\twolineshloka
{ततो ब्राह्मणरूपेण रविस्तं प्रजगाम ह}
{राजा प्रणम्य तं विप्रं फलैरापूजयत्ततः} % १२

\twolineshloka
{फलानि भक्षयित्वा स उवाच राजसत्तमम्}
{तपः किमर्थं राजन् ते देहं शोषयसे वद} % १३

\twolineshloka
{उग्रसेनस्ततो विप्रं तमुवाच कृताञ्जलिः}
{दुःखेन महता युक्तो निःश्वस्य स पुनः पुनः} % १४

\uvacha{उग्रसेन उवाच}

\twolineshloka
{करोमि नीतियुक्तेन राज्यं धर्मेण नित्यदा}
{तत्र व्याघ्रगणाः क्रूरा नरान् सम्भक्षयन्ति भोः} % १५

\twolineshloka
{शान्तयेऽत्र मया तेषां नानोपायाः कृता मुने}
{अपरास्ते भवन्तीह तदर्थं तप आचरम्} % १६

\twolineshloka
{राज्ञो वचनमाकर्ण्य विप्रस्तं पुनरब्रवीत्}
{निन्दयन् सर्वभावेन साक्षाद्भानुः प्रतापवान्} % १७

\uvacha{द्विज उवाच}

\twolineshloka
{महापापी त्वमेवासि कर्मदूषणतो नृप}
{चतुर्थीजं व्रतं मुख्यं नष्टं राज्ये सुदुर्मते} % १८

\twolineshloka
{सर्वादौ तत् प्रकर्तव्यं सर्वसिद्धिप्रदायकम्}
{तदा कर्म कृतं भूप फलयुक्तं भवेत् सदा} % १९

\twolineshloka
{चतुःपदार्थदं पूर्णं व्रतानामुत्तमं व्रतम्}
{व्याघ्रास्तद्दोषभावेन पीडयन्ति च मानवान्} % २०

\twolineshloka
{एवमुक्त्वा चतुर्थीजं माहात्म्यं द्विजसत्तमः}
{श्रावयामास तस्मै स श्रुत्वा सोऽभूत् सुविस्मितः} % २१

\twolineshloka
{जगाद हर्षसंयुक्तस्तं विप्रं ज्ञानदं परम्}
{कृताञ्जलिः प्रणम्याऽसौ वचनं स्वहितप्रदम्} % २२

\uvacha{उग्रसेन उवाच}

\twolineshloka
{किं त्वं साक्षात् स्वयं भानुस्तपसा तुष्टचेतसा}
{आगतोऽनुग्रहार्थं मे धन्योऽहं ते प्रदर्शनात्} % २३

\twolineshloka
{अधुना वद मे ब्रह्मन् गणेशस्य स्वरूपकम्}
{ज्ञात्वा तं देवदेवशं भजिष्यामि विशेषतः} % २४

\uvacha{द्विज उवाच}

\twolineshloka
{श‍ृणु राजन् गणेशस्य स्वरूपं योगदं परम्}
{भुक्तिमुक्तिप्रदं पूर्णं धारितं चेन्नरेण वै} % २५

\twolineshloka
{चित्ते चिन्तामणिः साक्षात् पञ्चचित्तप्रचालकः}
{पञ्चवृत्तिनिरोधेन प्राप्यते योगसेवया} % २६

\twolineshloka
{असम्प्रज्ञातसंस्थश्च गजशब्दो महामते}
{तदेव मस्तकं यस्य देहः सर्वात्मकोऽभवत्} % २७

\twolineshloka
{भ्रान्तिरूपा महामाया सिद्धिर्वामाङ्गसंश्रिता}
{भ्रान्तिधारकरूपा च बुद्धिः सा दक्षिणाङ्गके} % २८

\twolineshloka
{तयोः स्वामी गणेशश्च मायाभ्यां खेलते सदा}
{तं भजस्व विधानेन तदा शं लभसे नृप} % २९

\twolineshloka
{एवमुक्त्वा गणेशस्य ददौ मन्त्रं दशाक्षरम्}
{विधियुक्तं ततः सूर्योऽन्तर्धानं प्रचकार ह} % ३०

\twolineshloka
{राजा स्वनगरे गत्वा प्रधानैरनुमोदितः}
{कथयामास वृत्तान्तं सर्वेभ्यः सुखदायकम्} % ३१

\twolineshloka
{तत्रादौ चैत्रगा कृष्णा चतुर्थी सहसाऽऽगता}
{तां चकार विधानेन गणेशे भक्तिसंयुतः} % ३२

\twolineshloka
{समीपे नागराः संस्था नरा ग्रामान्तरे स्थिताः}
{चक्रुस्ते हर्षसंयुक्ताः सङ्कष्टीं कष्टहारिणीम्} % ३३

\twolineshloka
{तत् सर्वत्र व्रतं मुख्यं प्रशस्तं स चकार ह}
{भयाद्धठेन चक्रुस्ते जना भूमिस्थिताः परे} % ३४

\twolineshloka
{ततो व्याघ्रा महोग्राः स्म सर्वेऽन्तर्धानमाययुः}
{रोगादिभिर्विनिर्मुक्ताश्चिक्रीडुर्हर्षिता जनाः} % ३५

\twolineshloka
{राजा गणपतिं नित्यमभजन्नान्यचेतसा}
{गुरुरूपेण भानुं स पूजयामास नित्यदा} % ३६

\twolineshloka
{गते काले ततः पुत्रं स संस्थाप्य महामतिः}
{राज्ये निवृत्तिमास्थाय गणेशमभजत् परम्} % ३७

\twolineshloka
{अन्ते स्वानन्दगो भूत्वा ब्रह्मभूतो बभूव ह}
{तस्य राज्ये जनाः सर्वे क्रमात्ते मुक्तिमाप्नुवन्} % ३८

\twolineshloka
{एवं व्रतस्य माहात्म्यं लेशतः कथितं मया}
{अन्यच्छृणु महाभाग पापनाशकरं परम्} % ३९

\twolineshloka
{द्राविडे भिल्लजातिस्थः क्षत्रियः पापकारकः}
{भिल्लैः संस्कारहीनश्च सम्बन्धं स चकार ह} % ४०

\twolineshloka
{एकदा वनसंस्थश्च द्रव्यलोभी दुरात्मवान्}
{कञ्चित् दृष्ट्वा नरं तत्राऽधावच्छस्त्रप्रधारकः} % ४१

\twolineshloka
{पपाल सोऽपि दूरं वै नरो भयसमन्वितः}
{ऋक्षस्तत्र समायातो वने कश्चिन् महाबलः} % ४२

\twolineshloka
{तेनैव क्षत्रियः पापी धृतो वेगेन भूमिप}
{ऋक्षं स तं स शस्त्रेण पातयामास भूतले} % ४३

\twolineshloka
{भिल्लस्तत्र पपाताऽसौ भृशमृक्षेण पीडितः}
{निर्जने वनमध्ये स विललापाऽतिदारुणम्} % ४४

\twolineshloka
{दैवयोगेन सा देवी चैत्री सङ्कष्टहारिणी}
{तद्दिने तेन सम्प्राप्ता चतुर्थी कृष्णपक्षगा} % ४५

\twolineshloka
{जलान्नसंविहीनोऽयं बभूवे पापकारकः}
{चन्द्रोदये फलं तत्र पपात नृप वृक्षतः} % ४६

\twolineshloka
{दुष्टेन क्षत्रियेणैव भक्षितं विकलेन तत्}
{पञ्चम्यां स मृतस्तत्र ब्रह्मभूतो बभूव ह} % ४७

\twolineshloka
{एवं नाना जना राजन् व्रतस्यैव प्रभावतः}
{इह भुक्त्वाऽखिलान् भोगानन्ते ब्रह्म प्रलेभिरे} % ४८

\twolineshloka
{तत्र ते कति शक्यं न वक्तुं वर्षायुतैरपि}
{नानेन सदृशं किञ्चिद्व्रतं सर्वार्थदायकम्} % ४९

\twolineshloka
{अज्ञानेन कृतं दुष्टैर्व्रतं गाणेश्वरं महत्}
{ब्रह्मभूतकरं प्रोक्तं ज्ञानिनां तत्र का कथा} % ५०

\twolineshloka
{इदं चैत्रचतुर्थ्या यो माहात्म्यं प्रपठेन्नरः}
{श‍ृणोति चेच्च कृष्णायाः स सर्वं प्रलभेद् ध्रुवम्} % ५१

॥ॐ तत्सदिति श्रीमदान्त्ये पुराणोपनिषदि श्रीमन्मौद्गले महापुराणे चतुर्थे खण्डे गजाननचरिते चैत्रकृष्णचतुर्थीचरितवर्णनं नाम द्वाविंशोऽध्यायः॥४.२२॥


\sect{४.२३ --- त्रयोविंशोऽध्यायः --- वैशाखकृष्णचतुर्थीमाहात्म्यवर्णनम्}

\centerline{॥ श्रीगणेशाय नमः ॥}

\uvacha{दशरथ उवाच}

\twolineshloka
{वैशाखे कृष्णगायास्त्वं चतुर्थ्या वद साम्प्रतम्}
{माहात्म्यं मुनिशार्दूल न तृप्यामि समासतः} % १

\uvacha{वसिष्ठ उवाच}

\twolineshloka
{अगस्त्यो मुनिमुख्यश्च समुद्रशोषणे रतः}
{न शशाक महातेजा ब्रह्माणं शरणं ययौ} % २

\twolineshloka
{ब्रह्मणा नोदितः सोऽपि चकार व्रतमुत्तमम्}
{वैशाखे कृष्णपक्षे स चतुर्थ्यां विधिपूर्वकम्} % ३

\twolineshloka
{ज्ञात्वा माहात्म्यमुग्रं स नित्यं मन्त्रपरायणः}
{शौक्लं कार्ष्णं व्रतं सर्वैश्चकार मुनिसत्तमैः} % ४

\uvacha{दशरथ उवाच}

\twolineshloka
{तपस्तेजोयुतः साक्षादगस्त्यः सर्वशास्त्रवित्}
{व्रतहीनः कथं सोऽपि कुण्ठितश्च महाद्युतिः} % ५

\uvacha{वसिष्ठ उवाच}

\twolineshloka
{श‍ृणु राजन् महाभाग सम्यक् पृष्टं विचक्षण}
{अधुना तस्य माहात्म्यं कथयामि समासतः} % ६

\twolineshloka
{अगस्त्यस्तपसा युक्तो बभूवाऽतीव दारुणः}
{न समस्तेन राजेन्द्र ब्राह्मणेषु तपस्विषु} % ७

\twolineshloka
{न व्रतं तेन तदपि कृतमज्ञानभावतः}
{ब्रवीमि कारणं तत्र श‍ृणु संशयनाशनम्} % ८

\twolineshloka
{वेदशास्त्रपुराणेषु कर्म नानाविधं नृप}
{कथितं तच्च सर्वं तु कर्तुं केन प्रशक्यते} % ९

\twolineshloka
{स्वेच्छया कर्मणः कर्ता स्वयं भवति मानवः}
{तत्र मार्गं प्रवक्ष्यामि श‍ृणुष्व सुसमाहितः} % १०

\twolineshloka
{नित्यं नैमित्तिकं कर्म द्विविधं शास्त्रमार्गतः}
{तत्रागस्त्यश्चकाराऽसौ नित्यं कर्म महीपते} % ११

\twolineshloka
{अन्यव्रतादिकं सर्वं त्यक्त्वा विधिसमन्वितः}
{तपस्सु तत्परो नान्यदजानात् स तपोधनः} % १२

\twolineshloka
{अतस्तेन व्रतं मुख्यं न धृतं राजसत्तम}
{न समर्थेन शास्त्रेऽन्यच्छृणुष्व कथयामि ते} % १३

\twolineshloka
{गणेशभजनं मुख्यं सर्वेषां नात्र संशयः}
{सर्वसिद्धिकरं प्रोक्तं वेदादिषु विशेषतः} % १४

\twolineshloka
{राजशार्दूल तदपि नराः संस्कारहीनकाः}
{नाभजंस्तं गणेशानं ब्रह्मणां नायकं परम्} % १५

\twolineshloka
{तपसा दग्धपापश्च पुण्यराशिः प्रजायते}
{तदा रुचिर्भवेत्तत्र गणेशे भक्तिसंयुता} % १६

\twolineshloka
{अतोऽयं तपसा युक्तोऽगस्त्यस्तेजस्विनां वरः}
{जातः पात्रं विजानीहि ज्ञाने गाणेशसंज्ञिते} % १७

\twolineshloka
{अगस्त्यः क्रोधसंयुक्तो नृप वातापिरक्षकम्}
{समुद्रं तं शोषयितुं क्षोभयामास चोद्यतः} % १८

\twolineshloka
{तत्राऽतिकुण्ठितो जातस्तदर्थं विस्मितः स्वयम्}
{गत्वा ब्रह्माणमानम्य वचनं स जगाद ह} % १९

\uvacha{अगस्त्य उवाच}

\twolineshloka
{तपसा जलधिं ब्रह्मन् शोषयामि न संशयः}
{तत्र मे तपसः शक्तिः कुण्ठिताऽभूत् कथं वद} % २०

\uvacha{ब्रह्मोवाच}

\twolineshloka
{श‍ृणु पुत्र महाभाग कारणं कथयामि ते}
{नित्यकर्मपरस्त्वं वै सदा तपसि संस्थितः} % २१

\twolineshloka
{गणेशस्मरणं वत्स पूजनं नित्यमादरात्}
{करोषि तेन भवति कर्म ते सफलं मुने} % २२

\twolineshloka
{तथापि नित्यवत्तात चतुर्थी व्रतमुत्तमम्}
{कर्तव्यं यत्त्वया त्यक्तं तन्नैमित्तिकमानतः} % २३

\twolineshloka
{चतुर्णां पुरुषार्थानां दायकं शास्त्रसम्मतम्}
{तेन हीनस्त्वमद्यैव कुण्ठितो नात्र संशयः} % २४

\twolineshloka
{अधुना तद्व्रतं मुख्यं कुरु भावसमन्वितः}
{नित्यवत्तन् महाभाग समाचर विशेषतः} % २५

\twolineshloka
{तदा सङ्कटहीनस्त्वं चतुष्पदसमन्वितः}
{पूर्णयोगी भवसि वै गाणपत्यमवाप्स्यसि} % २६

\twolineshloka
{ततस्तेन कृतं राजन् व्रतं गाणेश्वरं महत्}
{पुरुषार्थसमायुक्तस्तेनैव स बभूव ह} % २७

\twolineshloka
{समुद्रं शोषयामास व्रतपुण्यप्रभावतः}
{अचलां गणनाथस्य भक्तिं चकार नित्यदा} % २८

\twolineshloka
{ततस्तेन मुनीन्द्रेषु विख्यातं तत् कृतं व्रतम्}
{नित्यवत्ते व्रतं मुख्यं चक्रुः सर्वार्थसिद्धये} % २९

\twolineshloka
{अन्यच्च श‍ृणु माहात्म्यं पुण्यदं श्रवणान्नृणाम्}
{अवन्तीनगरे वैश्यो बभूवे पापकारकः} % ३०

\twolineshloka
{बाल्यात् प्रारभ्य तेनैव कृतं पापं महाद्भुतम्}
{मरणावधि राजेन्द्र वक्तुं तन्न प्रशक्यते} % ३१

\twolineshloka
{मातृहा द्रव्यलोभार्थं पितृहा सम्बभूव ह}
{गुरुद्रोहं चकारैव ब्रह्महत्यां समाचरत्} % ३२

\twolineshloka
{ततो लोकैः समाज्ञप्तो राजा तं निरवासयत्}
{ग्रामाद्वनं समागत्य गुहायां संस्थितोऽभवत्} % ३३

\twolineshloka
{जघान मार्गगान् लोकान् जीवान्नानाविधान् सदा}
{तत्र प्रयोजनेनैव हीनो वा युक्त एव वा} % ३४

\twolineshloka
{ततो बहुधनो जातश्चौराः सम्मिलिताः परे}
{सह तैस्तत्र तेनैव कृतं दुर्गं सुदुर्गमम्} % ३५

\twolineshloka
{पर्वतद्रोणगं तत्र गृहं चक्रे महाखलः}
{चौरास्तत्र निवासार्थमाययुः सर्वतः स्थिताः} % ३६

\twolineshloka
{तेषां राजा बभूवाऽसौ भार्या तत्र महीपते}
{समानीताऽतिदुष्टेन पुत्रपुत्रीभिरावृता} % ३७

\twolineshloka
{यथेष्टं रमते तत्र चौरैश्चौर्यपरायणः}
{चकार लेशमात्रं स न पुण्यं दुर्मतिः कदा} % ३८

\twolineshloka
{मार्गे परस्त्रियं धृत्वा स सतीदूषकोऽयभत्}
{काश्चित्तत्र मृता राजन् स्त्रियः सत्यो महाभयात्} % ३९

\twolineshloka
{सोऽवन्तीपालको लोकै राजेन्द्रो बोधितस्ततः}
{मार्गरोधभयाच्चौरं तं हन्तुं प्रोद्यतोऽभवत्} % ४०

\twolineshloka
{सर्वत्र दशदिक्ष्वेव सैनिकाः प्रेषितास्ततः}
{राज्ञा चक्रुः प्रयत्नेन रुद्धं सर्वत्र तं खलम्} % ४१

\twolineshloka
{ततस्ते सन्धृतं चौरं पुत्रश्चौरैः समन्वितम्}
{राज्ञे निवेदयामासुस्तं दुष्टं वैश्ययोनिजम्} % ४२

\twolineshloka
{राजा चौरान् जघानापि जनैः शस्त्रधरैर्नृप}
{तं वैश्यं बन्धने क्षिप्त्वा ताडयन्नित्यदा खलम्} % ४३

\twolineshloka
{ततोऽकस्माच्चतुर्थी सा वैशाखी सहसाऽऽगता}
{कृष्णा तत्र च वैश्येन न प्राप्तं त्वन्नभक्षणम्} % ४४

\twolineshloka
{नित्यताडनभावेन दुःखितोऽतितरां खलः}
{क्षुधया पीडितोऽत्यन्तं पापी स निशि चैककः} % ४५

\twolineshloka
{ततो दयायुतैस्तत्र राजदूतैश्च भक्षणम्}
{किञ्चिद्दत्तं विधोरेवोदये वैश्यो बभक्ष सः} % ४६

\twolineshloka
{पञ्चम्यां स मृतस्तत्र बन्धने ताडितो भृशम्}
{गाणेशैर्ब्रह्मभूतः स कृतः स्वानन्दके पुरे} % ४७

\twolineshloka
{एतादृशः सुदुष्टात्मा व्रतपुण्यप्रभावतः}
{ब्रह्मभूतो बभूवाऽपि किं पुनर्व्रतकारिणः} % ४८

\twolineshloka
{एवं व्रतप्रभावेण जना ब्रह्म प्रलेभिरे}
{तेऽत्र वक्तुं न शक्या वै भवन्ति नृपसत्तम} % ४९

\twolineshloka
{इदं वैशाखमासे वै चतुर्थ्याः संश‍ृणोति चेत्}
{कृष्णाया वाऽपि पठति माहात्म्यं लभते परम्} % ५०

॥ॐ तत्सदिति श्रीमदान्त्ये पुराणोपनिषदि श्रीमन्मौद्गले महापुराणे चतुर्थे खण्डे गजाननचरिते वैशाखकृष्णचतुर्थीचरितं नाम त्रयोविंशोऽध्यायः॥४.२३॥


\sect{४.२४ --- चतुर्विंशोऽध्यायः --- ज्येष्ठकृष्णचतुर्थीमाहात्म्यवर्णनम्}

\centerline{॥ श्रीगणेशाय नमः ॥}

\uvacha{दशरथ उवाच}

\twolineshloka
{माहात्म्यं तच्छ्रुतं मुख्यं मया वैशाखके परम्}
{सङ्कष्टीसम्भवं मुख्यं न तृप्तस्तदपि प्रभो} % १

\twolineshloka
{अतो ज्येष्ठे चतुर्थी या सङ्कष्टी मुनिसत्तम}
{तस्याश्चरितमेवं मे वद पूर्णं समासतः} % २

\uvacha{वसिष्ठ उवाच}

\twolineshloka
{निषधेषु महाभागो नलो नामाऽभवन्नृपः}
{तेजस्वी शस्त्रसम्पन्नः शास्त्रज्ञश्च बभूव ह} % ३

\twolineshloka
{रूपवान् धनसम्पन्नश्चतुरङ्गबलान्वितः}
{शस्त्रास्त्रैर्भूमिपालांश्च बभौ जित्वा महाबलान्} % ४

\twolineshloka
{सामन्ता वशगा यस्य करदा इतरे नृपाः}
{सार्वभौमो महाराजः शशास पृथिवीमिमाम्} % ५

\twolineshloka
{धर्मेण दानशीलेन नीत्या त्यागेन भूमिपः}
{यशसा पूरयामास त्रैलोक्यं स चराचरम्} % ६

\twolineshloka
{इन्द्रलोके विधेर्लोके कैलासे वैष्णवे पदे}
{यस्य वार्तां प्रचक्रुस्ते धर्मशीलस्य देवपाः} % ७

\twolineshloka
{त्रिलोकीगमने सक्तः साधुदर्शनतत्परः}
{देवविप्रातिथिप्रेप्सुर्दीनान्धादीनपालयत्} % ८

\twolineshloka
{दमयन्ती महाभागा रूपेणाऽप्रतिमा भुवि}
{भार्या यस्य विशालाक्षी जगन्मोहकरी बभौ} % ९

\twolineshloka
{गुणा वर्णयितुं नैव शक्यास्तस्य महात्मनः}
{पुण्यश्लोकः स वै राजा कलिं जिग्ये महायशाः} % १०

\twolineshloka
{कदाचिद्वनगेनैव भ्रात्रा कलिवशेन सः}
{जितो द्यूतेन राज्यं त्यक्त्वा वनं स जगाम ह} % ११

\twolineshloka
{तत्राऽपि कलिना मत्स्यमिषेणैव महासती}
{प्रेरितेन नलेनाऽसौ त्यक्ता नृप वनान्तरे} % १२

\twolineshloka
{अर्धवस्त्रधरा साऽपि दमयन्ती पितुर्गृहे}
{नानादुःखसमायुक्ता जगाम नलवर्जिता} % १३

\twolineshloka
{गृप्तरूपेण वर्षाणि त्रीणि क्रमणलालसः}
{नलो बभ्राम तेजस्वी वने वस्त्रार्धधारकः} % १४

\twolineshloka
{धर्मशीलं स राजानं कलिना पीडितं परम्}
{दृष्ट्वा मुनिवरस्तत्र नारदः प्रजगाम ह} % १५

\twolineshloka
{तं दृष्ट्वा लज्जितो राजा प्रणनाम कृताञ्जलिः}
{लज्जितं तं समाज्ञाय नारदः स उवाच ह} % १६

\uvacha{नारद उवाच}

\twolineshloka
{मा लज्जां कुरु राजेन्द्र कर्मणां गतिरीदृशी}
{सावधानमना भूत्वा श‍ृणु मे परमं वचः} % १७

\twolineshloka
{तव राज्ये महाभाग व्रतं नष्टं महाद्भुतम्}
{चतुर्थीसंज्ञितं तेन राज्यभ्रष्टोऽसि साम्प्रतम्} % १८

\twolineshloka
{सर्वादौ तत् प्रकर्तव्यं चतुर्वर्गफलप्रदम्}
{न कृतं चेन् महाराज कर्म सर्वं निरर्थकम्} % १९

\twolineshloka
{चतुःफलविहीनस्त्वं भवस्यत्र न संशयः}
{अतोऽवश्यकभावेन कुरु त्वं सर्वसिद्धिदम्} % २०

\twolineshloka
{एवमुक्त्वा महायोगी नारदः श्रावयन्नृपम्}
{माहात्म्यं व्रतमुख्यस्य चतुर्थ्याः शान्तिकारकम्} % २१

\twolineshloka
{तच्छ्रुत्वा विस्मितो राजा नलस्तं प्रत्युवाच ह}
{गणेश्वरस्य माहात्म्यं भजिष्यामि वदस्व मे} % २२

\uvacha{नारद उवाच}

\twolineshloka
{गणेश्वरस्य माहात्म्यं शान्तियोगफलप्रदम्}
{वक्तुं न शक्यते राजंस्तथापि श‍ृणु मे वचः} % २३

\twolineshloka
{ब्रह्मा विष्णुश्च शम्भुश्च शक्तिः सूर्योऽमरास्तथा}
{शेषादिनागमुख्यास्तं भजन्ति कुलदैवतम्} % २४

\twolineshloka
{अन्नप्राणादिकान्येव ब्रह्माणि नृपसत्तम}
{भजन्ति तं विशेषेण कुलदेवं सनातनम्} % २५

\twolineshloka
{महावाक्यादिभिर्वेदास्तं भजन्ति सुयोगदम्}
{महावाक्यानि राजेन्द्र भजन्ति ब्रह्मनायकम्} % २६

\twolineshloka
{सर्वसिद्धिप्रदा यस्य वामाङ्गे प्रकृतिः परा}
{पञ्चचित्तमयी बुद्धिर्दक्षिणाङ्गे व्यवस्थिता} % २७

\twolineshloka
{स्वानन्दे वसतिर्यस्य सर्वपूज्योऽयमुच्यते}
{सर्वादिर्गणनाथश्च विप्रैः सोऽन्तेषु तिष्ठति} % २८

\twolineshloka
{गणाः समूहरूपाश्चान्तरबाह्यादियोगतः}
{तेषां स्वामी गणेशोऽयं तं भजस्व विधानतः} % २९

\twolineshloka
{एवमुक्त्वा ददौ तस्मै नृप मन्त्रं षडक्षरम्}
{विधियुक्तं ततः सोऽन्तर्दधे गाणेशको मुनिः} % ३०

\twolineshloka
{ततः स नृपशार्दूलो ध्यात्वा हृदि गजाननम्}
{मन्त्रं जजाप भावेन स्वात्मानं निन्दयन् भृशम्} % ३१

\twolineshloka
{ततः कर्कोटकेनैव दष्टस्तेन बभूव ह}
{विरूपस्तं ददौ वस्त्रं नागः पुनः स्वरूपदम्} % ३२

\twolineshloka
{तेन संहर्षितो भूप ऋतुपर्णं जगाम ह}
{तेनैव मानितोऽत्यन्तं विरूपो गुणसंयुतः} % ३३

\twolineshloka
{पितुर्गृहे गता नारी दमयन्ती महासती}
{तद्व्रतं कारयामास गत्वा तामपि नारदः} % ३४

\twolineshloka
{नलेन दमयन्त्या तज्ज्येष्ठमासे समागतम्}
{प्रथमं तद्व्रतं मुख्यं सङ्कष्टीसंज्ञकं कृतम्} % ३५

\twolineshloka
{त्रीणि वर्षाणि पूर्णानि गुप्तरूपेण संस्थितः}
{नलश्चतुर्थीजेनैव पुण्येन ज्ञानवानभूत्} % ३६

\twolineshloka
{ततः श्वशुरमागम्य ऋतुपर्णेन संयुतः}
{स्वात्मानं नागवस्त्रेण पूर्वरूपं चकार ह} % ३७

\twolineshloka
{दमयन्त्या युतः सोऽपि श्वशुरेण च सेनया}
{ऋतुपर्णेन राजेन्द्रो गतः स्वनगरे ततः} % ३८

\twolineshloka
{भ्रात्रा सम्मानितोऽत्यन्तं चकार राज्यमुत्तमम्}
{विघ्नहीनः स्वभावेन हृष्टपुष्टजनैर्वृतः} % ३९

\twolineshloka
{ततो बहौ गते काले पुत्रं राज्येऽभिषिच्य सः}
{सपत्नीको वने गत्वाऽभजत्तं गणनायकम्} % ४०

\twolineshloka
{अन्ते जगाम राजाऽसौ स्वानन्दे गणपं नृप}
{तत्रैव ब्रह्मभूतः स बभूव नलनामकः} % ४१

\twolineshloka
{नलेन सम्प्रदिष्टा ये जना भूमितले ततः}
{शुक्लकृष्णचतुर्थीजं मुख्यं चक्रुर्व्रतं नृप} % ४२

\twolineshloka
{ते सर्वे सुखसंयुक्ता बुभुजुः सुखमुत्तमम्}
{पुत्रपौत्रादिभिर्युक्ता रोगैः संवर्जिता नृप} % ४३

\twolineshloka
{अन्ते क्रमेण सर्वे ते ब्रह्मभूता बभूविरे}
{एवं कृष्णे ज्येष्ठमासे माहात्म्यं सङ्कटीभवम्} % ४४

\twolineshloka
{कथितं ते नृपश्रेष्ठ पुनस्त्वं श‍ृणु सिद्धिदम्}
{ज्येष्ठकृष्णचतुर्थीजं सर्वसङ्कटहारकम्} % ४५

\twolineshloka
{मालवेन्त्यजजः कश्चित् पापकर्मा बभूव ह}
{वनं गत्वा द्विजादीन् स जघान द्रव्यलोलुपः} % ४६

\twolineshloka
{एकाकिनीं स्त्रियं दृष्ट्वा बलेन बुभुजे खलः}
{सती परनरस्पर्शहीना तत्याज सा तनुम्} % ४७

\twolineshloka
{एवं नानाविधं पापं चकार जातिदूषणः}
{न तच्छक्यं कथयितुं पापभोगभयान् मया} % ४८

\twolineshloka
{एकदा वनसंस्थश्च चाण्डालः शस्त्रधारकः}
{कञ्चित् पुरुषकं दृष्ट्वाऽधावत् स हननाय तम्} % ४९

\twolineshloka
{पलायनपरः सोऽपि जगाम भयसङ्कुलः}
{हाहा कृत्वा पुरोदेशे स ततो निष्फलोऽभवत्} % ५०

\twolineshloka
{एतस्मिन्नन्तरे तत्र पुरुषाः शस्त्रधारकाः}
{राज्ञः समागतास्तत्र श्रुत्वा कोलाहलं तयोः} % ५१

\twolineshloka
{ते धृत्वा ताडयामासुः पुनस्तं राजसेवकाः}
{बद्ध्वा राज्ञे ददुर्दुष्टं राजा चिक्षेप बन्धने} % ५२

\twolineshloka
{तत्रैव दैवयोगेन प्राप्ता ज्येष्ठचतुर्थिका}
{कृता तेनान्नहीनोऽयं बभूवे बन्धनाकुलः} % ५३

\twolineshloka
{चन्द्रोदये दयायुक्तै रक्षकैस्तैर्महामते}
{अन्नं दत्तं तथा भुक्तं तेन दुष्टेन तत्क्षणात्} % ५४

\twolineshloka
{राजाज्ञया च तं दुष्टं पञ्चम्यां जघ्नुरादरात्}
{स मृतोन्त्यजजस्तत्र ब्रह्मभूतो बभूव ह} % ५५

\twolineshloka
{एवं नाना जना राजन् व्रतपुण्यप्रभावतः}
{भुक्त्वा भोगांश्च ते सर्वे ब्रह्मभूता बभूविरे} % ५६

\twolineshloka
{अपारमाहात्म्ययुतं व्रतं वक्तुं न शक्यते}
{तथाऽपि कथितं भूप माहात्म्यं सङ्कटीभवम्} % ५७

\twolineshloka
{इदं ज्येष्ठचतुर्थ्यास्तु कृष्णायाः संश‍ृणोति यः}
{माहात्म्यं वा पठति स लभते स्वेप्सितं फलम्} % ५८

॥ॐ तत्सदिति श्रीमदान्त्ये पुराणोपनिषदि श्रीमन्मौद्गले महापुराणे चतुर्थे खण्डे गजाननचरिते ज्येष्ठकृष्णचतुर्थीवर्णनं नाम चतुर्विंशोऽध्यायः॥४.२४॥


\sect{४.२५ --- पञ्चविंशोऽध्यायः --- आषाढकृष्णचतुर्थीमाहात्म्यवर्णनम्}

\centerline{॥ श्रीगणेशाय नमः ॥}

\uvacha{दशरथ उवाच}

\twolineshloka
{आषाढे सङ्कटी प्रोक्ता चतुर्थी चरितं शुभम्}
{तस्या वद महायोगिन् सर्वदं पापनाशनम्} % १

\uvacha{वसिष्ठ उवाच}

\twolineshloka
{अत्र ते कथयिष्यामि चेतिहासं पुरातनम्}
{सर्वसिद्धिकरं पूर्णं सेवितं चेन्नृपात्मज} % २

\twolineshloka
{वृत्रेण पीडितोऽत्यन्तं महेन्द्रश्चिन्तयान्वितः}
{वने वसत्तु सन्त्रस्तस्त्यक्त्वा राज्यादिकं पुरा} % ३

\twolineshloka
{मुनयः श्रुतिभिर्हीना भ्रष्टाचाराः समन्ततः}
{अतिष्ठंस्ते भयोद्विग्ना वर्णाः सर्वे तथा नृप} % ४

\twolineshloka
{कर्मखण्डनभावेन देवाः परमविह्वलाः}
{उपोषणपरा भूत्वाऽतिष्ठंस्ते मरणोन्मुखाः} % ५

\twolineshloka
{गिरेर्गुहासु संस्थास्ते ददृशुर्मुनिपुङ्गवम्}
{विशेषज्ञं गृत्समदं योगिनां गुरुमागतम्} % ६

\twolineshloka
{तं दृष्ट्वा दण्डवत् सर्वे प्रणेमुश्चेन्द्रमुख्यकाः}
{पप्रच्छुः पूजयित्वा तं स्वासने संस्थितं मुनिम्} % ७

\uvacha{इन्द्र उवाच}

\twolineshloka
{दृष्टिर्धन्या जन्म धन्यं पिता माता व्रतादिकम्}
{यज्ञो ज्ञानादिकं मे वै त्वदङ्घ्रेर्दर्शनात् प्रभो} % ८

\twolineshloka
{तव दर्शनमात्रेण कल्याणं नो भविष्यति}
{साक्षाद्योगीश्वरस्यैव सर्वसिद्धिप्रदस्य च} % ९

\twolineshloka
{वृत्रासुरेण दुष्टेन निर्जिता वयमेव च}
{राज्यं त्यक्त्वा वने विप्र तिष्ठामः पशवो यथा} % १०

\twolineshloka
{अत्रातिकर्मनाशेनोपोषणेन समन्विताः}
{मरिष्यामो न सन्देहस्तत्र किं दर्शनं भवेत्} % ११

\twolineshloka
{तथाऽपि पुण्ययोगेन प्राप्तं ते दर्शनं परम्}
{वद ब्रह्मन् दयां कृत्वा वृत्रनाशकरं महत्} % १२

\twolineshloka
{जगत् सर्वं महायोगिन् भ्रष्टाचारं कृतं सदा}
{किं पश्यसि सुसंहारे प्राप्ते योगीन्द्रसत्तम} % १३

\uvacha{वसिष्ठ उवाच}

\twolineshloka
{एवं पृष्टो गृत्समदस्तमुवाच दयान्वितः}
{विश्वरक्षणभावार्थं दुष्टनाशकरं वचः} % १४

\uvacha{गृत्समद उवाच}

\twolineshloka
{राज्यं प्राप्य महाभाग सुर त्यक्ता चतुर्थिका}
{व्रतभ्रष्टोऽसि देवेन्द्र अधुना तद्व्रतं कुरु} % १५

\twolineshloka
{ज्ञानमदेन विघ्नेशमन्त्रस्त्यक्तस्तथा त्वया}
{किं चित्रं राज्यहीनत्वे दुर्मते मदलालस} % १६

\twolineshloka
{एवमुक्त्वा चतुर्थ्या यन् माहात्म्यं मुनिसत्तमः}
{कथयामास तस्मै तत् पुनर्मन्त्रं ददौ स्वयम्} % १७

\twolineshloka
{यदृच्छया गते राजन् मुनाविन्द्रेण तत्ततः}
{व्रतं कृतं द्विजैर्देवैर्गणेशं सम्प्रपूज्य च} % १८

\twolineshloka
{तत्रादौ देवराजेनाषाढी कृष्णा महातिथिः}
{प्राप्ता चतुर्थिका भूप सा कृता भूतिदायिका} % १९

\twolineshloka
{गणेशं मनसा ध्यात्वा दधीचेरस्थिजं महत्}
{वज्रं धृत्वा ययौ देवैर्वृत्रं युद्धपरायणः} % २०

\twolineshloka
{चतुर्थीव्रतजेनाऽसौ प्रभावेण महासुरम्}
{कृत्वा युद्धं महाघोरं वज्रेणैव जघान तम्} % २१

\twolineshloka
{हत्वा वृत्रं महावीर्यं देवैः सह शतक्रतुः}
{स्वस्थोऽभूदमरावत्या मुमुदे स्वजनैर्नृप} % २२

\twolineshloka
{ततो देवगणाः सर्वे स्वस्वस्थानेषु नित्यशः}
{शुक्लकृष्णभवं चक्रुश्चतुर्थीसंज्ञितं व्रतम्} % २३

\twolineshloka
{पृथिव्यां सर्वलोकास्तद्व्रतं चक्रुविशेषतः}
{इन्द्रेण बोधितं भक्त्या सर्वसिद्धिकरं महत्} % २४

\twolineshloka
{मन्वन्तरे गते देवैरिन्द्रः स्वानन्दगोऽभवत्}
{दृष्ट्वा विघ्नेश्वरं तत्र ब्रह्मभूतो बभूव ह} % २५

\twolineshloka
{भूमिसंस्था नराः सर्वे क्रमेण प्रययुर्नृप}
{गणेशं ब्रह्मभूतास्ते व्रतपुण्यप्रभावतः} % २६

\twolineshloka
{एतत्ते कथितं स्वल्पं माहात्म्यं व्रतसम्भवम्}
{पुनरन्यच्छृणुष्व त्वं पापकञ्चुकनाशनम्} % २७

\twolineshloka
{गुर्जरे ब्राह्मणः कश्चिद् बाल्यात् पापपरायणः}
{परयोनिषु संसक्तो जीवघातं चकार ह} % २८

\twolineshloka
{एकदा भगिनी यब्धा तेन दुष्टेन चाऽभवत्}
{मद्यमांसपरेणाऽपि चरता वनगह्वरे} % २९

\twolineshloka
{द्रव्यलोभी जघानाऽसो द्विजादींश्च विशेषतः}
{पशुपक्षिगणान् राजन् भ्रष्टाचारः सुदुर्मतिः} % ३०

\twolineshloka
{नित्यं चकार स खलो यवनैः सह भोजनम्}
{न शक्यं तन्मया तस्य पापं वर्णयितुं परम्} % ३१

\twolineshloka
{एकदा वनमध्यस्थो बभूवे जातिदूषकः}
{आषाढे कृष्णपक्षे स चतुर्थ्यां पापकारकः} % ३२

\twolineshloka
{दैवयोगेन तेनैव न प्राप्तं वनगह्वरे}
{किञ्चित्तेनाऽतिदुःखार्तस्तदा बभ्राम पर्वते} % ३३

\twolineshloka
{तथाऽपि नालभत् किञ्चित् ततो दुःखपरायणः}
{अपराह्णे पुनः सोऽपि स्वगृहं गन्तुमुत्सुकः} % ३४

\twolineshloka
{न ददर्श स तत्राऽपि जलान्नं दैवयोगतः}
{आजगाम क्षुधार्तः स स्वगृहे शोकसंयुतः} % ३५

\twolineshloka
{यवनजां समालिङ्ग्य निशि चन्द्रोदये नृप}
{बभक्षान्नं सुतैः सार्धं मोहितो मायया भृशम्} % ३६

\twolineshloka
{ततो ज्वरयुतोऽकस्माद्बभूवे पापनिश्चयः}
{राजन् यवनजातेन न स्पृष्टा पापकारिणी} % ३७

\twolineshloka
{पञ्चम्यां स मृतस्तत्र भृशं दुःखेन पीडितः}
{गाणेशास्तं प्रगृह्यैव ब्रह्मभूतं प्रचक्रिरे} % ३८

\twolineshloka
{एतादृशा महापापा उद्धरन्ति व्रतेन वै}
{अज्ञानकृतपुण्येन तत्र किं वर्णयाम्यहम्} % ३९

\twolineshloka
{विधियुक्तं व्रतं भूप ये कुर्वन्ति नरोत्तमाः}
{दर्शनेन जनानां ते तारकाः सम्भवन्ति हि} % ४०

\twolineshloka
{एवं व्रतप्रभावेण ब्रह्मभूता बभूविरे}
{अनन्ताश्चरितं तेषां वक्तुं नैव प्रशक्यते} % ४१

\twolineshloka
{इदमाषाढसङ्कष्ट्याश्चरितं यः श‍ृणोति चेत्}
{पठेद्वा लभते सोऽपि वाञ्छितं नात्र संशयः} % ४२

॥ॐ तत्सदिति श्रीमदान्त्ये पुराणोपनिषदि श्रीमन्मौद्गले महापुराणे चतुर्थे खण्डे गजाननचरिते आषाढकृष्णचतुर्थीवर्णनं नाम पञ्चविंशोऽध्यायः॥४.२५॥


\sect{४.२६ --- षड्विंशोऽध्यायः --- श्रावणकृष्णचतुर्थीमाहात्म्यवर्णनम्}

\centerline{॥ श्रीगणेशाय नमः ॥}

\uvacha{दशरथ उवाच}

\twolineshloka
{श्रावणे कृष्णपक्षे या चतुर्थी सङ्कटी भवेत्}
{माहात्म्यं वद तस्या मे सर्वेभ्यः सुखदायकम्} % १

\uvacha{वसिष्ठ उवाच}

\twolineshloka
{इतिहासं प्रवक्ष्यामि श्रावणे सङ्कटीभवम्}
{श्रवणात् सुखदः पूर्णो भवते पठनात्तथा} % २

\twolineshloka
{चन्द्रवंशे समुत्पन्नो युधिष्ठिरः प्रतापवान्}
{धर्मशीलो वदान्यश्च मान्यान् मानयिता भृशम्} % ३

\twolineshloka
{साक्षाद्धर्मावतारश्च सत्यवाक् करुणायुतः}
{पाण्डुपुत्रो महातेजाः प्रजापालनतत्परः} % ४

\twolineshloka
{द्रौपद्या भ्रातृभिश्चैव सहितः सर्वसम्मतः}
{कुन्त्या मात्रा महाभागो विख्यातः सर्वमण्डले} % ५

\twolineshloka
{कृष्णो यस्य सहायोऽभूत् सदा वृष्णिकुलोद्भवः}
{साक्षान्नारायणः श्रीमान् भूभारहरणे रतः} % ६

\twolineshloka
{एकदा स्वगृहे राजा संस्थितो राजभिर्वृतः}
{राजसूयं महायज्ञं कृत्वा कृष्णविवर्जितः} % ७

\twolineshloka
{धृतराष्ट्रेण तत्रैव प्रेषितो विदुरः स्वयम्}
{आजगाम स तं दृष्ट्वा मानयामास कुन्तिजः} % ८

\twolineshloka
{भोजितं परमान्नेन विदुरं सदसि स्थितम्}
{तमुवाच महाभागो हर्षितः स युधिष्ठिरः} % ९

\twolineshloka
{धन्यं मे जन्म कर्मादि त्वदङ्घ्रियुगदर्शनात्}
{वयं पाल्याः सदा तात त्वया धर्मभृता प्रभो} % १०

\twolineshloka
{कुशलं स्वजनानां ते किमर्थं त्वं समागतः}
{वद मे सकलं पूर्णं वृत्तान्तं धृतराष्ट्रगम्} % ११

\uvacha{विदुर उवाच}

\twolineshloka
{सर्वत्र कुशलं वत्स स्वजनेषु महामते}
{धृतराष्ट्रस्य दुर्बुद्धेरादराद्वचनं श‍ृणु} % १२

\twolineshloka
{पुत्रवत्सलभावेन दुर्योधनसमाश्रितः}
{तेनाऽहं प्रेषितो भूप त्वत्समीपे विचक्षणः} % १३

\twolineshloka
{इच्छति त्वां विजेतुं स प्रभो द्यूतेन मन्दधीः}
{दुर्योधनो महापापी कुलक्लेशविवर्धनः} % १४

\twolineshloka
{कर्णेन प्रेरितोऽत्यन्तं शकुनिना विशेषतः}
{सक्तस्ते राज्यहरणे दुःशासनसमन्वितः} % १५

\twolineshloka
{धृतराष्ट्रस्तथा राजन् मन्यते मनसि ध्रुवम्}
{स त्वामाह्वयते लोभी मया चैव युधिष्ठिर} % १६

\twolineshloka
{मा गच्छ दुर्मतिं तात पुत्रपक्षसमाश्रितम्}
{धृतराष्ट्रं च कुन्त्या त्वं सस्त्रीको भ्रातृभिः कदा} % १७

\uvacha{वसिष्ठ उवाच}

\twolineshloka
{विदुरस्य वचः श्रुत्वा युधिष्ठिरो महायशाः}
{उवाच तं महाभागं निःश्वस्य वचनं हितम्} % १८

\uvacha{युधिष्ठिर उवाच}

\twolineshloka
{श‍ृणु तात मदीयं त्वं वाक्यं धर्मयुतं महत्}
{राज्ये सुसंस्थितो राजा धृतराष्ट्रो विशेषतः} % १९

\twolineshloka
{तदाज्ञावशगाः सर्वे वयं भीष्मादयः किल}
{अतो यामि त्वया सार्धं भावि यत्तद्भविष्यति} % २०

\twolineshloka
{धर्मयुक्तं वचः श्रुत्वा विदुरः खिन्नमानसः}
{न किञ्चिदुक्तवांस्तत्र तं निःश्वस्य महायशाः} % २१

\twolineshloka
{विदुरेण युताः सर्वे सस्त्रीकाः पाण्डवा नृप}
{समागताश्च द्रौपद्या धृतराष्ट्रं सभास्थितम्} % २२

\twolineshloka
{तेनाज्ञप्ताश्च ते सर्वे भ्रातरः कलहे रताः}
{चिक्रीडुर्द्यूतमुग्रं वै दुर्योधनमुखैः पुरा} % २३

\twolineshloka
{शकुनिना महोग्रेण तपसा साधितं पुरा}
{द्यूतं तदेव राजेन्द्र तदधीनं बभूव ह} % २४

\twolineshloka
{क्रमेण राज्यमुग्रं तत् सर्वं जित्वा सुयोधनः}
{वनवासे पाण्डवान् स स्थापयामास दुर्मतिः} % २५

\twolineshloka
{द्रौपदी तेन दुष्टेन सभायां सर्वसन्निधौ}
{समानीता स तां प्राह पत्नी मे भव सुन्दरि} % २६

\twolineshloka
{पतयः पञ्च कुत्राऽपि न दृश्यन्ते स्त्रियाः क्वचित्}
{अस्माकं कुलजं सर्वं यशो नष्टं त्वया खले} % २७

\twolineshloka
{भ्राता तां हठसंयुक्तो नग्नां चक्रे महाखलः}
{भीष्मादींश्च तिरस्कृत्य दुःशासनसमन्वितः} % २८

\twolineshloka
{पाण्डवा धर्मरक्षार्थमसहंस्तस्य कर्म तत्}
{द्रौपद्या संस्मृतः कृष्णो गुप्तरूपधरो ययौ} % २९

\twolineshloka
{मायया वस्त्रसंयुक्तां द्रौपदीं स पुनः पुनः}
{चकार च ततः सर्वे क्षुभिताः स्वगृहं ययुः} % ३०

\twolineshloka
{द्रौपदी पाण्डवैः सार्धं ययौ मुक्ता वने तदा}
{तत्र कृष्णः समायातो बलेन क्रोधसंयुतः} % ३१

\onelineshloka*
{उवाच राजशार्दूलं युधिष्ठिरं प्रतापवान्}

\uvacha{कृष्ण उवाच}

\onelineshloka
{हत्वा दुर्योधनं तात सपक्षं राज्यकामुकम्} % ३२

\twolineshloka
{त्वां राज्ये स्थापयिष्यामि मा गच्छ वनमेव च}
{ततस्तं धर्मराजस्तु जगाद वचनं हितम्} % ३३

\twolineshloka
{क्रोधं त्यज महाबाहो धर्मं रक्ष जनार्दन}
{धर्मयुक्तं च राज्यं मे देहि वृष्णिकुलोद्भव} % ३४

\twolineshloka
{एवमुक्तो महातेजाः कृष्णस्तं प्रत्युवाच ह}
{कुरु त्वं भ्रातृभिः सार्धं गणेशोपासनं स्त्रिया} % ३५

\twolineshloka
{सत्यासत्यसमाना व्यक्तैश्चतुर्धा बभौ प्रभुः}
{स्वानन्दवासकर्ताऽसावस्माकं कुलदैवतम्} % ३६

\twolineshloka
{एवमुक्त्वा ददौ तस्मै प्रभुर्मन्त्रं दशाक्षरम्}
{स विधिं हर्षसंयुक्तः सर्वसिद्धिकरं परम्} % ३७

\twolineshloka
{माहात्म्यं व्रतजं तत्र श्रावयामास केशवः}
{युधिष्ठिरस्य सर्वं वै पुण्यदं भ्रान्तिनाशनम्} % ३८

\twolineshloka
{चतुर्विधप्रदं पूर्णं सर्वादौ सम्मतं व्रतम्}
{कृतं चेत् कर्म तत् सर्वं सफलं भवति प्रभो} % ३९

\twolineshloka
{तव राज्ये व्रतं मुख्यं नष्टं तेन महामते}
{राज्यभ्रष्टश्च सञ्जातश्चतुःपदविवर्जितः} % ४०

\twolineshloka
{श्रुत्वा युधिष्ठिरो राजा विस्मितः सुमहायशाः}
{गणेशं मनसि ध्यात्वा चकार विधिसंयुतः} % ४१

\twolineshloka
{तत्रादौ श्रावणे मासे प्राप्ता सङ्कटनाशिनी}
{चतुर्थी कृष्णपक्षे तां साधयामास सानुजैः} % ४२

\twolineshloka
{शुक्लकृष्णचतुर्थीजं व्रतं चक्रुश्च पाण्डवाः}
{सस्त्रीकास्तेन ते सर्वे वनेषु सुखिनोऽभवन्} % ४३

\twolineshloka
{पुनः स्वराज्यकर्तारो बभूवुश्च सुयोधनम्}
{हत्वा सर्वनृपैर्युक्तं कृष्णेन हर्षसंयुतः} % ४४

\twolineshloka
{पुनस्तेन पृथिव्यां तद्विख्यातं प्रकृतं व्रतम्}
{शुक्लकृष्णचतुर्थीजं चक्रुः सर्वे जनास्ततः} % ४५

\twolineshloka
{इह भुक्त्वाऽखिलान् भोगान् पौत्रं राज्येऽभिषिच्य ते}
{पाण्डवा ज्ञानसंयुक्ता बभुः स्वांशेषु सङ्गताः} % ४६

\twolineshloka
{अन्ये जनाः क्रमेणैव गता गाणेश्वरं पदम्}
{दृष्ट्वा गणेश्वरं तत्र ब्रह्मभूता बभूविरे} % ४७

\twolineshloka
{एवं व्रतस्य माहात्म्यं कथितं ते नृपोत्तम}
{अन्यच्छृणु महाभाग महापुण्यप्रदं परम्} % ४८

\twolineshloka
{कर्णाटे कारुकः कश्चिन् महापापपरोऽभवत्}
{वनेषु शस्त्रधारी जनान् जघान स नित्यदा} % ४९

\twolineshloka
{परस्त्रियं तत्र दृष्ट्वैकाकिनीं स तथा दरात्}
{अयभद्धठसंयुक्तो बलेन स महाखलः} % ५०

\twolineshloka
{गोवधादिकमेवं स ब्राह्मणानां च हिंसनम्}
{चकार दुष्टभावेन द्रव्यार्थं मांसकारणात्} % ५१

\twolineshloka
{एवं नित्यं वने गत्वा पापं चक्रे स दुर्मतिः}
{चोरो ग्रामे जनान् जघ्ने गृहीत्वाऽर्थं विशेषतः} % ५२

\twolineshloka
{कदाचित् स वने गत्वा गुहायां संस्थितोऽभवत्}
{जनान् हन्तुं महाचोरस्ततो वृष्टिर्बभूव ह} % ५३

\twolineshloka
{अतिवृष्टिप्रभावेन जलैर्नद्यश्च पूरिताः}
{यत्र तत्र जलं पूर्णं गन्तुं मार्गो न विद्यते} % ५४

\twolineshloka
{तत्र वृष्टिभयेनैव संस्थितः क्षुधितो नृप}
{दैवयोगेन सा देवी चतुर्थी तिथिराबभौ} % ५५

\twolineshloka
{श्रावणे कृष्णपक्षे सा प्राप्ता तस्या व्रतं शुभम्}
{कृतं तेन सुदुष्टेन ह्यन्नहीनप्रभावतः} % ५६

\twolineshloka
{चन्द्रोदये समुत्पन्ने जलेनागतमादरात्}
{फलं च नारिकेलं यत्तद्बभञ्ज बभक्ष सः} % ५७

\twolineshloka
{निशाशेषे बहिः पापी निःसृतस्तत्र चाययौ}
{व्याघ्रः सङ्गृह्य तं हत्वा बभक्ष क्षुधितो भृशम्} % ५८

\twolineshloka
{गाणपास्तं सम्प्रगृह्य ययुः स्वानन्दके पुरे}
{चक्रुर्व्रतप्रभावेण तत्र ब्रह्ममयं ततः} % ५९

\twolineshloka
{महापापी व्रतेनैवाज्ञानजेन नृपात्मज}
{ब्रह्मभूतो बभूवाऽथ वर्णये किं महाद्भुतम्} % ६०

\twolineshloka
{नानाजना व्रतेनैव भुक्त्वा भोगान् हृदीप्सितान्}
{ययुः स्वानन्दके वर्णयितुं तत्र न शक्यते} % ६१

\twolineshloka
{इदं श्रावणकृष्णायाश्चतुर्थ्या वर्णनं पठेत्}
{माहात्म्यं श‍ृणुयाद्यो वै भुक्तिं मुक्तिं स विन्दति} % ६२

॥ॐ तत्सदिति श्रीमदान्त्ये पुराणोपनिषदि श्रीमन्मौद्गले महापुराणे चतुर्थे खण्डे गजाननचरिते श्रावणकृष्णचतुर्थीमाहात्म्यवर्णनं नाम षड्विंशोऽध्यायः॥४.२६॥


\sect{४.२७ --- सप्तविंशोऽध्यायः --- भाद्रपदकृष्णचतुर्थीमाहात्म्यवर्णनम्}

\centerline{॥ श्रीगणेशाय नमः ॥}

\uvacha{दशरथ उवाच}

\twolineshloka
{भाद्रकृष्णचतुर्थ्यास्त्वं माहात्म्यं वद भो मुने}
{न तृप्यामि कथां श‍ृण्वन् सर्वसिद्धिप्रदायिनीम्} % १

\uvacha{वसिष्ठ उवाच}

\twolineshloka
{द्राविडे राजशार्दूलो बभूव परमद्युतिः}
{वीरसेनो यशोयुक्तः सार्वभौमो महाबलः} % २

\twolineshloka
{नीतिज्ञो नीतियुक्तश्च चक्रे सौराज्यमुत्तमम्}
{सर्वान् राज्ञो वशे कृत्वा करभारेण संयुतान्} % ३

\twolineshloka
{धर्मशीलः सदानन्दं चाददाद्भक्तिसंयुतः}
{क्षुधितं न नरं नारीमसहत् सुखकारकः} % ४

\twolineshloka
{पुत्रानिव प्रजाः सर्वाः पालयामास यत्नतः}
{अपुत्रो दैवयोगेन बभूव परवीरहा} % ६

\twolineshloka
{पुत्रार्थं यत्नमत्यन्तं जपहोमादिभिः सदा}
{चकार तीर्थक्षेत्रादि विधाने तादृशोऽभवत्} % ६

\twolineshloka
{एवं बहौ गते काले पुत्रो नासीन् महीपते}
{ततोऽतिदुःखितो राजा राज्यं त्यक्त्वा ययौ वनम्} % ७

\twolineshloka
{सस्त्रीकः स वने राजा बभ्राम यत्र तत्र ह}
{ततो महावनं गत्वा पपात क्षुधयान्वितः} % ८

\twolineshloka
{सुमन्तुस्तत्र विप्रर्षिः समिदर्थं समाययौ}
{नृपं दृष्ट्वा दयायुक्त आययौ तं निरीक्षितुम्} % ९

\twolineshloka
{वेदवेदाङ्गवित् साक्षात् सर्वशास्त्रप्रवर्तकः}
{पुराणज्ञो महायोगी व्यासशिष्यः प्रतापवान्} % १०

\twolineshloka
{तं दृष्ट्वा सहसोत्थाय प्रणनाम महामुनिम्}
{वीरसेनः प्रहर्षेण सस्त्रीकः संयुतो नृपः} % ११

\twolineshloka
{करसम्पुटमुत्थाय पुरः कृत्वा महामुनेः}
{तं जगाद विशेषज्ञं राजा वचनमुत्तमम्} % १२

\uvacha{वीरसेन उवाच}

\twolineshloka
{धन्यं मे कर्म जन्मादि पिता माता कुलं यशः}
{विद्या व्रतादिकं सर्वं त्वदङ्घ्रियुगदर्शनात्} % १३

\twolineshloka
{राज्यं त्यक्त्वा वने संस्थः पुत्रार्थं पापवानहम्}
{तत्र ते दर्शनं प्राप्तं पूर्वपुण्यफलोदयात्} % १४

\twolineshloka
{नानायत्नेन विप्रेश पुत्रो मे नाऽभवत्किल}
{त्वत्कृपोपायमेकं च पश्यामि सुखदायकम्} % १५

\uvacha{वसिष्ठ उवाच}

\twolineshloka
{एवं तस्य नृपस्याऽसौ श्रुत्वा वचनमुत्तमम्}
{सन्तुष्टस्तं जगादेदं सुमन्तुर्मुनिसत्तमः} % १६

\uvacha{सुमन्तुरुवाच}

\twolineshloka
{मा चिन्तां कुरु राजेन्द्र पुत्रस्ते कुलतारकः}
{भविष्यति न सन्देहः श‍ृणु मे वचनं हितम्} % १७

\twolineshloka
{राज्ये नष्टं त्वदीये तच्चतुर्थीजं व्रतं महत्}
{तेन त्वं पुत्रहीनोऽसि नृपाधम न बुध्यसे} % १८

\twolineshloka
{सर्वादौ तद्व्रतं मुख्यं कर्तव्यं सर्वसिद्धिदम्}
{चतुःपदार्थदं सर्वैर्नोचेत् सर्वं सुनिष्फलम्} % १९

\twolineshloka
{त्वया कर्म कृतं नानापुण्यदं सर्वसम्मतम्}
{चतुर्भिः पुरुषार्थैर्हीनं चतुर्थीविवर्जितम्} % २०

\twolineshloka
{एवमुक्त्वा नृपायाऽथ माहात्म्यं व्रतजं महत्}
{श्रावयामास तं सोऽपि श्रुत्वा पप्रच्छ भाविकः} % २१

\uvacha{वीरसेन उवाच}

\twolineshloka
{वद ब्रह्मन् गणेशस्य ज्ञानं यस्य चतुःप्रदम्}
{व्रतं सर्वादिसम्मान्यं तं भजिष्यामि नित्यदा} % २२

\uvacha{सुमन्तुरुवाच}

\twolineshloka
{गणेशस्य स्वरूपं यद्वक्तुं शक्तो भवेत्तु कः}
{उपाधिसंयुतं ज्ञानं श‍ृणु राजेन्द्र सर्वदम्} % २३

\twolineshloka
{अहं तपःप्रभावेण पुरा जातो विशेषतः}
{शापानुग्रहणे राजन् समर्थः सर्वमण्डले} % २४

\twolineshloka
{ततोऽतितपसा युक्तोऽभवं तेन महामते}
{अन्तर्ज्ञानं समुत्पन्नं स दृष्ट्वा विस्मितो ह्यहम्} % २५

\twolineshloka
{ततो जडादिका नाना भूमिकाः साधयंस्ततः}
{नानाब्रह्मविभेदेषु संस्थितो ब्रह्मधारकः} % २६

\twolineshloka
{एवं क्रमेण राजेन्द्र आसं स्वानन्दगोऽभवम्}
{तत्र शान्तिसुखे सक्तो ब्रह्मभूतस्वभावतः} % २७

\twolineshloka
{तस्माद्भेदमयं द्वन्द्वं दृष्ट्वा शान्तो महामुनिम्}
{व्यासं गत्वा प्रणम्यैवाऽऽपृच्छं तं स्वहितप्रदम्} % २८

\twolineshloka
{ब्रह्मभूतस्वरूपां मे वद शान्तिं महामुने}
{शिष्योऽहं ते महाभाग तारयस्व भवार्णवात्} % २९

\twolineshloka
{इति पृष्टो महायोगी मामुवाच प्रहर्षितः}
{तत्तेऽहं कथयिष्यामि गणेशज्ञानकारकम्} % ३०

\uvacha{व्यास उवाच}

\twolineshloka
{श‍ृणु पुत्र प्रवक्ष्यामि ब्रह्मभूतस्वरूपकम्}
{शान्तीनां शान्तिरूपं तं ज्ञातव्यं योगसेवया} % ३१

\twolineshloka
{स्वत उत्थानकं ब्रह्म उत्थानं परतस्तथा}
{तयोरभेदभावे च ह्यसत् स्वानन्द उच्यते} % ३२

\twolineshloka
{तत्रामृतमयं यत्तत् सद्रूपं स्वस्वरूपकम्}
{तयोरभेदभावे चानन्दः स्वानन्द उच्यते} % ३३

\twolineshloka
{त्रिभिर्हीनं त्रिभिर्युक्तमव्यक्तं ब्रह्म कथ्यते}
{चतुर्णामेव संयोगे स्वानन्दो ब्रह्म उच्यते} % ३४

\twolineshloka
{पञ्चभिर्गतिहीनं यन्मतं चायोगवाचकम्}
{न तत्र कस्यचिद्योगस्ततः कुत्राऽपि पुत्रक} % ३५

\twolineshloka
{संयोगायोगयोर्योगे योगः शान्तिप्रदायकः}
{स एव गणराजश्च ज्ञातव्यो विबुधैः सदा} % ३६

\twolineshloka
{गणाः समूहरूपाश्चान्तरबाह्यादियोगतः}
{अन्नादिब्रह्मरूपास्ते ज्ञातव्या योगसेवया} % ३७

\twolineshloka
{तेषां स्वामी गणेशानस्तं भजस्व विधानतः}
{तदा त्वं ब्रह्मभूतश्च भविष्यसि न संशयः} % ३८

\twolineshloka
{चित्तं पञ्चविधं प्रोक्तं तदेव बुद्धिवाचकम्}
{चित्तं मोहस्वरूपं यत् सिद्धिरूपं वदन्ति च} % ३९

\twolineshloka
{तयोः स्वामी गणाधीशो मायाभ्यां क्रीडति प्रभुः}
{पञ्चचित्तनिरोधेन लभ्यते चित्तधारकः} % ४०

\twolineshloka
{अतश्चिन्तामणिः प्रोक्तो पश्य वेदेषु पुत्रक}
{चित्तं त्यक्त्वा समोहं त्वं भव चिन्तामणिस्ततः} % ४१

\twolineshloka
{एवमुक्त्वा ददौ मह्यं मन्त्रमेकाक्षरं प्रभुः}
{सविधिं तं प्रणम्यैवाऽऽगतोऽहं स्वाश्रमे पुनः} % ४२

\twolineshloka
{ध्यात्वा गणपतिं राजन् मन्त्रे जपपरोऽभवम्}
{ततः स्वल्पेन कालेन शान्तिं प्राप्तोऽहमेव च} % ४३

\twolineshloka
{तथाऽपि गणराजं त्वपूजयन्तं विशेषतः}
{ततो मे दर्शयामास स्वात्मानं विघ्नपः प्रभुः} % ४४

\twolineshloka
{समागतं गणेशानं दृष्ट्वा विस्मितमानसः}
{अपूजयं प्रहर्षेण तं प्रणम्य पुनः पुनः} % ४५

\twolineshloka
{अथर्वोपनिषद्भिः स संस्तुतो गणनायकः}
{भक्तिं दत्त्वा स्वकीयां मेऽन्तर्धानं प्रचकार ह} % ४६

\twolineshloka
{तदादि गाणपत्योऽहं प्रभजे नित्यमादरात्}
{तं भजस्व महाराज ततः सर्वं शुभं भवेत्} % ४७

\twolineshloka
{एवमुक्त्वा ददौ तस्मै स मन्त्रं षोडशाक्षरम्}
{सविधिं मुनिमुख्यः सोऽन्तर्धानमगमत्ततः} % ४८

\twolineshloka
{राजाऽतिविस्मितो भूत्वा सपत्नीकः प्रणम्य तम्}
{स्वपुरे स समागत्याऽभजत्तं गणनायकम्} % ४९

\twolineshloka
{तत्राऽऽदौ भाद्रमासे सा कृष्णा प्राप्ता चतुर्थिका}
{कृता तेन जनैः सर्वैः पुरवासिभिरादरात्} % ५०

\twolineshloka
{विशेषतस्ततस्तेन प्रशस्तं तद्व्रतं कृतम्}
{शुक्लकृष्णचतुर्थीजं चक्रुः सर्वे धरातले} % ५१

\twolineshloka
{व्रतपुण्यप्रभावेण राजा पुत्रसमन्वितः}
{बभूवू रोगहीनास्ते भूमिपैवं जनास्तथा} % ५२

\twolineshloka
{पुत्रे राज्यं परित्यज्य राजा निर्वृत्तिसंयुतः}
{सपत्नीको गणेशानमभजन्नित्यमादरात्} % ५३

\twolineshloka
{अन्ते गणेशलोके स गत्वा ब्रह्ममयोऽभवत्}
{क्रमेण सर्वलोकाश्च ब्रह्मभूता बभूविरे} % ५४

\twolineshloka
{एतत्ते कथितं भूप माहात्म्यं सङ्कटीभवम्}
{अन्यच्छृणु चतुर्थ्यास्त्वं पापघ्नं सर्वदं परम्} % ५५

\twolineshloka
{आन्ध्रे च धीवरः कश्चिद् बभूवे सर्वहिंसकः}
{बाल्यात् प्रभृति सोऽत्यन्तं पापकर्मपरायणः} % ५६

\twolineshloka
{वृथा साक्ष्यपरो भूत्वा कलहं सोऽकरोन् मिथः}
{लोकानां लोभशीलश्च चौर्यद्रव्यं समाहरत्} % ५७

\twolineshloka
{वने गत्वा महापापी मार्गस्थान् प्रजघान वै}
{ब्राह्मणादीन् विशेषेण नानापापपरायणः} % ५८

\twolineshloka
{न तस्य कर्म वक्तुं वै शक्यते पापभीतितः}
{ग्रन्थबाहुल्यतो भूप परस्त्रीलालसोऽभवत्} % ५९

\twolineshloka
{भाद्रकृष्णचतुर्थीजे वनमध्यस्थ एकदा}
{व्रते प्राप्ते महापापी निराहारो बभूव ह} % ६०

\twolineshloka
{अकस्मात् तत्र सम्प्राप्तो जनसङ्घो महान्नृप}
{भयात्तस्य प्रलीनोऽभूद्गुहायां धीवरः स तु} % ६१

\twolineshloka
{किञ्चिद्दिनावशेषे स जनसङ्घो गतोऽभवत्}
{तदा बहिर्विनिःसृत्य गृहे गन्तुं मनो दधे} % ६२

\twolineshloka
{न प्राप्तं तेन दुष्टेन किञ्चित्तदपि दारुणः}
{स्वात्मन्यास्थाय धैर्यं स मार्गसंस्थो बभूव ह} % ६३

\twolineshloka
{रात्रौ समागतो दुष्टः स्वगृहे राजसत्तम}
{चन्द्रोदये सुतैः सार्धं भोजनं स चकार ह} % ६४

\twolineshloka
{पञ्चम्यां सर्पदंशेन मृतः पापेन निश्चयः}
{गणराजस्य तदपि दूता नेतुं समागमन्} % ६५

\twolineshloka
{गृहीत्वा धीवरं तेऽयुर्गणाः स्वानन्दके पुरे}
{गणेशानं स तं दृष्ट्वा ब्रह्मभूतो बभूव ह} % ६६

\twolineshloka
{व्रतेनाऽज्ञानजेनाऽयं विधिहीनेन भूमिप}
{ब्रह्मभूतो धीवरश्च सम्बभूव वदामि किम्} % ६७

\twolineshloka
{एवं नाना जना राजन् ब्रह्मभूता बभूविरे}
{चतुर्थ्या व्रतपुण्येन वक्तुं नैव प्रशक्यते} % ६८

\twolineshloka
{इदं भाद्रपदे कृष्णचतुर्थीजं श‍ृणोति यः}
{माहात्म्यं वाऽपि पठति स ईप्सितमवाप्नुयात्} % ६९

॥ॐ तत्सदिति श्रीमदान्त्ये पुराणोपनिषदि श्रीमन्मौद्गले महापुराणे चतुर्थे खण्डे गजाननचरिते भाद्रपदकृष्णचतुर्थीमाहात्म्यवर्णनं नाम सप्तविंशोऽध्यायः॥४.२७॥


\sect{४.२८ --- अष्टाविंशोऽध्यायः --- आश्विनकृष्णचतुर्थीमाहात्म्यवर्णनम्}

\centerline{॥ श्रीगणेशाय नमः ॥}

\uvacha{दशरथ उवाच}

\twolineshloka
{माहात्म्यं यद्वद ब्रह्मन्नाश्विने कृष्णके महत्}
{चतुर्थीजं विशेषेण न तृप्तोऽहं भवामि वै} % १

\uvacha{वसिष्ठ उवाच}

\twolineshloka
{इतिहासं प्रवक्ष्यामि सर्वसिद्धिप्रदायकम्}
{अत्र तेन महाराज श‍ृणु तृप्तो भविष्यसि} % २

\twolineshloka
{पार्वतीशङ्कराभ्यां च तपसाऽऽराधितोऽभवत्}
{दिव्यवर्षसहस्रेण गणेशः पुत्रतां गतः} % ३

\twolineshloka
{कार्तिकेयस्तथा जातः स्ववीर्यात् स्नेहकारकः}
{न तथा प्रीतिरुत्पन्ना गणेशे च तयोरभूत्} % ४

\twolineshloka
{कार्तिकेयो गणेशानं स्पर्धते नित्यमादरात्}
{विवाहो मे पुरा पश्चात्तवापि स्म भविष्यति} % ५

\twolineshloka
{एवं स्पर्धयतोर्वीक्ष्य पुत्रयोः पार्वतीशिवौ}
{विचारं चक्रतुस्तत्र विवाहार्थं महामते} % ६

\twolineshloka
{अयं विघ्नेश्वरः साक्षाद्ब्रह्मरूपो न संशयः}
{क्षोभितश्चेत् पदभ्रष्टं करिष्यति विशेषतः} % ७

\twolineshloka
{साक्षात् पुत्रं समुत्पन्नं स्कन्दं सन्त्यज्य विघ्नपम्}
{ज्येष्ठं विवाहयेच्चेद्वा सेनानीः कोपितो भवेत्} % ८

\twolineshloka
{त्यक्त्वा स्वपुत्रं सेनान्यं विवाहो नैव शोभते}
{अतः कपटरूपेण कर्तव्यं कार्यमुत्तमम्} % ९

\twolineshloka
{ततः शिवो गणेशानं स्कन्दं चैवाऽब्रवीद्वचः}
{धराप्रदक्षिणां कृत्वा य आदौ सुसमागतः} % १०

\twolineshloka
{पूर्वं विवाहकं तस्य करिष्यामि न संशयः}
{ततस्तं गणनाथस्तु जगाद वचनं हितम्} % ११

\twolineshloka
{ज्येष्ठं मां किं परित्यज्य प्रभो वदसि शङ्कर}
{वेदहीनं तथा मेऽस्तु भवदाज्ञां करोम्यहम्} % १२

\twolineshloka
{स्थूलदेहधरोऽहं मे वाहनं मूषको मतः}
{पृथ्वीप्रदक्षिणां कर्तुं न शक्नोमि सदाशिव} % १३

\twolineshloka
{एवमुक्त्वा गणाधीशः स्वगृहे संस्थितोऽभवत्}
{जगाम स्कन्द आरुह्य मयूरं हर्षसंयुतः} % १४

\twolineshloka
{पृथ्वीप्रदक्षिणायां स रतः स्कन्दः स वेगवान्}
{चकार मायया तत्र गणेशः कौतुकं महत्} % १५

\twolineshloka
{सम्पूज्य पार्वतीं शङ्करं चकार प्रदक्षिणाम्}
{उवाच शङ्करं तत्र विवाहो मे विधीयताम्} % १६

\twolineshloka
{प्रदक्षिणा कृता येन पितुर्मातुः सदाशिव}
{पृथ्वीप्रदक्षिणा तेन कृता शास्त्रे न संशयः} % १७

\twolineshloka
{श्रुत्वा गणपतेर्वाक्यं कम्पितः शङ्करोऽब्रवीत्}
{करोमि ते विवाहं वै क्रोधं मा कुरु पुत्रक} % १८

\twolineshloka
{तत एकान्तगं शम्भुं पार्वती प्रेमविह्वला}
{उवाच कार्तिकेयं किं त्यक्त्वा ढुण्ढिः स मानितः} % १९

\twolineshloka
{ततस्तां शङ्करो वाक्यं जगाद श‍ृणु पार्वति}
{अयं क्रुद्धश्च सर्वस्वं हरत्यत्र न संशयः} % २०

\twolineshloka
{अतः स्नेहं परित्यज्य रक्ष धर्मं सनातनम्}
{नाऽयं पुत्रो महादेवि ब्रह्मभूतः समागतः} % २१

\twolineshloka
{एवमुक्त्वा सतीं शम्भुविवाहमकरोत्ततः}
{गणेशस्य विधानेन हृदयेनाविदूयता} % २२

\twolineshloka
{सिद्धे विवाहके पूर्णे कार्तिकेयः समागतः}
{तत्र विघ्नकरो विघ्नं चकार परमाद्भुतम्} % २३

\twolineshloka
{नारदः सहसाऽऽगत्य कार्तिकेयमुवाच ह}
{विवाहकृत्यमेतस्य पित्रा मात्रा कृतं पुरा} % २४

\twolineshloka
{त्वां वञ्चयित्वा सेनानीः कृतं कर्म जुगुप्सितम्}
{नावलोक्यं मुखं मातुः पितुश्चापि त्वया कदा} % २५

\twolineshloka
{एवमुक्त्वा गतो विप्रो नारदः कलहप्रियः}
{स्कन्दो निःश्वस्य तस्मात् स ययौ श्रीशैलपर्वतम्} % २६

\twolineshloka
{ज्ञात्वा पुत्रस्य वृत्तान्तं पार्वतीशङ्कराविमम्}
{शोधार्थं तत्र शैले तौ गतौ ज्ञात्वा पपाल सः} % २७

\twolineshloka
{क्रौञ्चे स्कन्दं स्थितं ज्ञात्वा पार्वतीशङ्करौ नृप}
{शोकसंविग्नचित्तौ तौ विलापं चक्रतुः पुरा} % २८

\twolineshloka
{तत्राऽऽजगाम विप्रेन्द्रो मुद्गलः सर्वमुद्गलः}
{तं दृष्ट्वा सहसोत्थाय पूजयामासतुर्मुनिम्} % २९

\twolineshloka
{सुखासीनं महाभाग उवाच शोकसङ्कुलः}
{भगवन् सर्वतत्त्वज्ञ वद मां शोकनाशनम्} % ३०

\twolineshloka
{स्कन्दसंयोगदं पूर्णमुपायं तं करोम्यहम्}
{रुरोद पार्वती तत्र ततस्तौ मुनिरब्रवीत्} % ३१

\uvacha{मुद्गल उवाच}

\twolineshloka
{शम्भो श‍ृणु महाभाग त्वं साक्षात् सर्ववित् प्रभुः}
{विद्यारूपेयमाद्या सा शक्तिः सर्वार्थदायिनी} % ३२

\twolineshloka
{तथापि कथयामि त्वां पुत्रौ शास्त्रे प्रकीर्तितौ}
{द्विविधौ तौ त्वया प्राप्तौ त्वत्समो नैव दृश्यते} % ३३

\twolineshloka
{योगाभ्यासेन विघ्नेशो ब्रह्मणां नायकस्तव}
{पुत्रः समाधिजः प्रोक्तस्ततश्चिन्तामणिः स्मृतः} % ३४

\twolineshloka
{देहः शक्तिमयस्तत्रात्मा त्वं पुरुष उच्यसे}
{ताभ्यां योगबलेनैव क्रियते ब्रह्मपुत्रकः} % ३५

\twolineshloka
{अर्धनारीश्वरस्त्वं च तस्माज्जातो गजाननः}
{ब्रह्मभूतपदस्याऽसौ पालको नात्र संशयः} % ३६

\twolineshloka
{भिन्ना शक्तिरियं जाता भिन्नस्त्वं देहधारकः}
{भिन्नभावात् समुत्पन्नः पुत्रस्ते स्कन्द एव च} % ३७

\twolineshloka
{देहसौख्यकरः प्रोक्तः स्कन्दः सर्वत्र सम्मतः}
{शान्तिसौख्यकरश्चैव गणेशः पुत्रभावतः} % ३८

\twolineshloka
{तत्र त्वं भिन्नभावेन मोहितोऽसि न संशयः}
{स्कन्दं स्नेहेन शम्भो त्वमधिकं मन्यसे सदा} % ३९

\twolineshloka
{न तथा ब्रह्मणि श्रद्धाऽधिका विश्वेश्वरेऽधुना}
{तेन विघ्नसमायुक्तो जातोऽसि च भजस्व तम्} % ४०

\twolineshloka
{ब्रह्मणि स्नेहभावेन विषयान्निन्दयन् स्थितः}
{तदा विषयतापास्ते स्वयं वश्या भवन्ति वै} % ४१

\twolineshloka
{विघ्नेश्वरमनादृत्य स्कन्दं विवाहसंयुतम्}
{आदौ पश्यामि शक्त्या वै सन्धृतं मानसे त्वया} % ४२

\twolineshloka
{तदैव निष्फलं सर्वं कृतं विघ्नकरेण ते}
{वियोगश्च समुत्पन्नस्तस्मात्तं शरणं व्रज} % ४३

\twolineshloka
{एवमुक्त्वा मुद्गलस्तं जगाम स्वेच्छया चरन्}
{शिवः शक्त्या युतः स स्म मन्यते सत्यमेव तत्} % ४४

\twolineshloka
{त्यक्त्वा स्कन्दभवं स्नेहं विघ्नेशमभजत् सदा}
{तत्रादौ सङ्कटी प्राप्ताऽऽश्विनी तां तौ प्रचक्रतुः} % ४५

\twolineshloka
{मायां भ्रान्तिकरीं सर्वां भिन्नभावप्रमोहतः}
{त्यक्त्वा शिवश्च शक्तिश्च शान्तिस्थौ तौ बभूवतुः} % ४६

\twolineshloka
{कस्य पुत्रश्च का माता पिता को भ्रमदायकम्}
{कृतं विघ्नेश्वरेणैव तमावां शरणं गतौ} % ४७

\twolineshloka
{ततो विघ्नेश्वरेणैव कृतं कौतुकमादरात्}
{बुद्धिभेदः कृतस्तत्र स्कन्दस्य नृप तच् श‍ृणु} % ४८

\twolineshloka
{त्यक्तौ मयाऽतिमूढेन पार्वतीशङ्करावहो}
{ज्येष्ठं गणपतिं नित्यं स्पर्धयामि महाप्रभुम्} % ४९

\twolineshloka
{शक्त्या शिवेन विघ्नेशस्तपसाऽऽराधितोऽभवत्}
{सोऽयं पुत्रस्वरूपेण बभूव वरदानतः} % ५०

\twolineshloka
{स्वानन्दवासकारी स सिद्धिबुद्धिपतिः स्वयम्}
{योगशान्तिस्वरूपोऽयं न जानामि सुमूर्खवत्} % ५१

\twolineshloka
{एवं विचार्य स्कन्दः स ययौ कैलासमादरात्}
{आदौ प्रणम्य विघ्नेशं पूजयामास भक्तितः} % ५२

\twolineshloka
{अथर्वशिरसा तं स ततः स्तुत्वा सदाशिवम्}
{उमां सम्पूज्य तुष्टाव प्रणनामाऽथ दण्डवत्} % ५३

\twolineshloka
{सम्मानितः स तैः स्कन्दोऽतिष्ठत्तद्भक्तिसंयुतः}
{तदाज्ञावशगो भूत्वा शान्तियोगपरोऽभवत्} % ५४

\twolineshloka
{विवाहार्थं शिवेनाऽसौ प्रेरितस्तमुवाच ह}
{स्त्रियं बन्धकरीं नाथ नेच्छामि गणपे रतः} % ५५

\twolineshloka
{अत्याग्रहयुतः स्कन्दोऽभजत्तं गणनायकम्}
{मायासुखं परित्यज्य तन्निष्ठः स बभूव ह} % ५६

\twolineshloka
{एवं सङ्कटहारिण्या चतुर्थ्या महिमाद्भुतः}
{कथितः शान्तिदः सर्वसम्मेलनकरोऽभवत्} % ५७

\twolineshloka
{अन्यच्छृणु महाभाग चरित्रं पापनाशनम्}
{वङ्गदेशे द्विजः कश्चित् पापकर्मा बभूव ह} % ५८

\twolineshloka
{त्यक्त्वा ब्राह्मणमार्गं स मद्यपानरतोऽभवत्}
{यवनीं स समादाय यवनोऽभून् महाबलः} % ५९

\twolineshloka
{वक्तुं न शक्यते तस्य कर्म पापमयं महत्}
{कथने परदोषाणां दोषी भवति मानवः} % ६०

\twolineshloka
{कदाचिज्ज्वरसंयुक्तो बभूवे जातिदूषणः}
{तत्राश्विनभवा प्राप्ता नृप कृष्णा चतुर्थिका} % ६१

\twolineshloka
{तत्र तेन जलान्नं च भक्षितं न भयान्नृप}
{सुपक्वानां तन्दुलानां पीतं चन्द्रोदये जलम्} % ६२

\twolineshloka
{ततो बहौ गते काले ममार यवनो द्विजः}
{सङ्गृह्य ब्रह्मभूतं तं गाणेशाश्चक्रिरे नृप} % ६३

\twolineshloka
{एतादृशा महापापा व्रतपुण्यप्रभावतः}
{ब्रह्मभूता बभूवुश्च का कथा विधिकारिणाम्} % ६४

\twolineshloka
{एवं नाना जना राजन् भुक्त्वा भोगान् यथेप्सितान्}
{ब्रह्मण्यं ते गतास्तत्तु मया वक्तुं न शक्यते} % ६५

\twolineshloka
{इदमाश्विनजायास्तु माहात्म्यं यः श‍ृणोति वा}
{पठेत् स कृष्णसङ्कष्ट्याः सर्वार्थं लभते नरः} % ६६

॥ॐ तत्सदिति श्रीमदान्त्ये पुराणोपनिषदि श्रीमन्मौद्गले महापुराणे चतुर्थे खण्डे गजाननचरिते आश्विनकृष्णचतुर्थीमाहात्म्यवर्णनं नाम अष्टाविंशोऽध्यायः॥४.२८॥


\sect{४.२९ --- एकोनत्रिंशोऽध्यायः --- कार्तिककृष्णचतुर्थीमाहात्म्यवर्णनम्}

\centerline{॥ श्रीगणेशाय नमः ॥}

\uvacha{दशरथ उवाच}

\twolineshloka
{कार्तिके कृष्णपक्षे या चतुर्थी सङ्कटी मता}
{माहात्म्यं वद तस्यास्त्वं सर्वपापहरं परम्} % १

\uvacha{वसिष्ठ उवाच}

\twolineshloka
{अत्र ते कथयिष्यामि चेतिहासं पुरातनम्}
{महाराष्ट्रे महातेजा राजाऽभूत् कर्दमाभिधः} % २

\twolineshloka
{वेदवेदार्थतत्त्वज्ञः स्वधर्मनिरतोऽभवत्}
{लोकान् सङ्गृह्य षष्ठांशं पालयन् पुत्रकान् यथा} % ३

\twolineshloka
{अपारसेनया युक्तश्चतुरङ्गप्रमोदया}
{पूर्णकोशः कुबेराभो बभौ शस्त्रास्त्रपारगः} % ४

\twolineshloka
{स्वबलेन नृपान् सर्वान् वशान् कृत्वा महाबलः}
{चकार वशगां पृथ्वीं समुद्रान्तां विशेषतः} % ५

\twolineshloka
{करदा इतरे सर्वे राजानः सेवका इव}
{सामन्ताश्च महाराज तस्याज्ञावशवर्तिनः} % ६

\twolineshloka
{पुण्यवान् धर्मशीलश्च नानादानपरायणः}
{देवविप्राऽतिथिप्रेप्सुः स्वदारनिरतोऽभवत्} % ७

\twolineshloka
{यज्वा विनीतकस्तीर्थकारकः परवीरहा}
{नानागुणयुतः सोऽपि मया वक्तुं न शक्यते} % ८

\twolineshloka
{स एव यक्ष्मणाऽत्यन्तं पीडितः पापरूपिणा}
{शोकाकुलोऽभवद्राजा अस्थिशेषो बभूव ह} % ९

\twolineshloka
{नानोपायाः कृतास्तस्य शान्तये न च सोऽभवत्}
{शान्तस्ततो महाराजस्तीर्थयात्रापरोऽभवत्} % १०

\twolineshloka
{नानातीर्थानि राजाऽसौ चकार तदपि ह्यहो}
{पीडया नैव मुक्तोऽभूत्ततो दुःखितमानसः} % ११

\twolineshloka
{देवार्चनरता विप्राः कृतास्तेन महात्मना}
{तथापि रोगसंयुक्तोऽधिको राजा बभूव ह} % १२

\twolineshloka
{ततः शान्तिधरो राजा वैराग्ये निदधे मनः}
{राज्यं त्यक्त्वा प्रधानेषु सस्त्रीकः स वनं ययौ} % १३

\twolineshloka
{बभ्राम वनगो भूत्वा यत्र तत्र महीपतिः}
{ततो मङ्कणकं विप्रमदर्शद्योगिसत्तमम्} % १४

\twolineshloka
{तद् दृष्ट्वा निपपातोर्व्यां रुरोद भृशमेव सः}
{ततस्तं मुनिमुख्योऽसौ जगाद दयया युतः} % १५

\uvacha{मङ्कणक उवाच}

\twolineshloka
{श‍ृणु राजन् महाभाग येन त्वं रोगपीडितः}
{अभवस्तदघं सर्वं कथयामि नृपाधम} % १६

\twolineshloka
{मुख्यं नष्टं च राज्ये ते चतुर्थीसंज्ञितं व्रतम्}
{तेन त्वं पापसंयुक्तो नरके गच्छसि ध्रुवम्} % १७

\twolineshloka
{सर्वादौ तद्व्रतं कार्यं नित्यवद्राजसत्तम}
{सर्वसिद्धिकरं पूर्णं चतुर्विधफलप्रदम्} % १८

\twolineshloka
{तद्व्रतेन विहीनस्त्वं नानाधर्मपरायणः}
{चतुःपदार्थहीनत्वान्निष्फलोऽसि नृपाधम} % १९

\twolineshloka
{एवमुक्त्वा च माहात्म्यं चतुर्थीसम्भवं मुनिः}
{श्रावयामास तस्मै तच्छ्रुत्वा सोऽपि तमब्रवीत्} % २०

\uvacha{कर्दम उवाच}

\twolineshloka
{धन्यं मे जन्म कर्मादि त्वदङ्घ्रियुगदर्शनात्}
{व्रतस्य श्रवणाद्धन्यो कृतकृत्यो न संशयः} % २१

\twolineshloka
{दयानिधे गणेशस्य स्वरूपं वद विस्तरात्}
{सर्वदेवाधिदेवं तं भजिष्यामि निरन्तरम्} % २२

\uvacha{मङ्कणक उवाच}

\twolineshloka
{गणेशस्य स्वरूपं कथयितुं शक्यते न तत्}
{तथाप्युपाधिसंयुक्तं वदामि श‍ृणु कर्दम} % २३

\twolineshloka
{पुराऽहं पत्रभक्षः संस्तपसे संस्थितोऽभवम्}
{बहुकाले गते हस्तो विद्धो मे कण्टकेन च} % २४

\twolineshloka
{तदा पत्ररसस्तस्मादस्रवत् किञ्चिदप्यहो}
{दृष्ट्वा विस्मितचित्तोऽहं नृत्यं चाकरवं तदा} % २५

\twolineshloka
{अहो देहाद्गतं कुत्र रुधिरं मे स्वभावजम्}
{पत्रभक्षणभावेन रसः स्रवति हस्ततः} % २६

\twolineshloka
{ततो नृत्यस्य वेगेन कम्पितं स चराचरम्}
{मदीयतपसा युक्तं देवाः शङ्करमाययुः} % २७

\twolineshloka
{शङ्करः सर्वदेवैश्चागत्य मामब्रवीद्वचः}
{किं नृत्यसि महाभाग पश्य मे कौतुकं महत्} % २८

\twolineshloka
{इत्युक्त्वा त्रुटिका तेन कृता तत्र रसस्य मे}
{भस्मरूपं कृतं भूपाऽभवं तेनातिविस्मितः} % २९

\twolineshloka
{त्यक्त्वा नृत्यं महेशानं प्रणम्य स्तुतवान् क्षणात्}
{अन्तर्धाय स्वमात्मानं शङ्करः स गतोऽभवत्} % ३०

\twolineshloka
{मया मनसि राजेन्द्र सन्धृतं शङ्करात् परम्}
{विद्यते न तथा कर्तुमन्यथा कर्तुमीश्वरः} % ३१

\twolineshloka
{दर्शयित्वा तपोरूपमतो भस्म कृतं महत्}
{रसस्य सर्वगः सोऽपि ब्रह्माकारः प्रदृश्यते} % ३२

\twolineshloka
{त्यक्त्वा विष्णुं महेशस्य भजने निरतोऽभवम्}
{क्रमेण योगयुक्तोऽहं तपस्त्यक्त्वा शमे रतः} % ३३

\twolineshloka
{जडादिका मया योगभूमयः क्रमितास्ततः}
{क्रमेण सहजे संस्थोऽभवं वै मोहवर्जितः} % ३४

\twolineshloka
{यत्र तत्र महीदेशे भ्रान्तोऽहं योगधारकः}
{सहजं ब्रह्म मोहेन हीनं दृष्टं मया ततः} % ३५

\twolineshloka
{स्वाधीनं सहजं ब्रह्म न योग्यं शान्तिदे परे}
{तेनाऽहं ब्रह्मणि भ्रान्तः सूर्यं च शरणं गतः} % ३६

\twolineshloka
{प्रणम्य सौरमार्गेण स्तोत्रेण प्रस्तुतो मया}
{प्रसन्नो माऽवदत्तत्र वरं वरय चेप्सितम्} % ३७

\twolineshloka
{ततोऽहमवदं सूर्यं वद शान्तिप्रदं प्रभो}
{सहजं मोहहीनत्वात् स्वाधीनं नैव शान्तिदम्} % ३८

\twolineshloka
{ततो मामब्रवीत् सूर्यः प्रसन्नो भक्तियन्त्रितः}
{गणेशं भज भावेन तदा शान्तिमवाप्स्यसि} % ३९

\twolineshloka
{असच्छक्तिश्च सत् सूर्यः समो विष्णुर्महामुने}
{सहजः शङ्करः प्रोक्तस्तेषां योगे गणेश्वरः} % ४०

\twolineshloka
{चतुर्णां चैव संयोगे स्वानन्दः परिकीर्तितः}
{अयोगः पञ्चभिर्हीनस्तत्र निर्मायिको बभौ} % ४१

\twolineshloka
{संयोगे मायया युक्तो योगे मायाविवर्जितः}
{तयोर्योगे शान्तिदः स गणेशो ब्रह्मनायकः} % ४२

\twolineshloka
{चित्तरूपा महाबुद्धिः पञ्चधा परिकीर्तिता}
{चित्तमोहप्रदा सिद्धिस्ताभ्यां क्रीडति विघ्नपः} % ४३

\twolineshloka
{चित्तं मोहयुतं विप्र त्यक्त्वा योगेन शान्तिदम्}
{चिन्तामणिं भजस्व त्वं योगिवन्द्यो भविष्यसि} % ४४

\twolineshloka
{एवमुक्त्वा ददौ मन्त्रं प्रभुर्मह्यं षडक्षरम्}
{स विधिं तं नमस्कृत्य स्वाश्रमेऽहं समागमम्} % ४५

\twolineshloka
{तत्र विघ्नेश्वरं भक्त्या भजं तं नित्यमादरात्}
{ततः स्वल्पेन शान्तिस्थोऽभवं योगस्य सेवया} % ४६

\twolineshloka
{तथापि गणराजं सम्पूज्य मन्त्रपरायणः}
{अभजं तेन सन्तुष्टः प्रत्यक्षं प्रययौ स्वयम्} % ४७

\twolineshloka
{मया सम्पूजितो भक्त्या स्तुतो नानाविधैः स्तवैः}
{मह्यं दत्त्वा दृढां भक्तिमन्तर्धानं चकार ह} % ४८

\twolineshloka
{तदारभ्य महीपाल गाणपत्योऽहमादरात्}
{भजामि तं गणेशानं स्वच्छन्देन महामते} % ४९

\twolineshloka
{एवमुक्त्वा ददौ तस्मै कर्दमाय षडक्षरम्}
{मन्त्रं विधियुतं विप्रो गणेशकृपया युतः} % ५०

\twolineshloka
{अन्तर्धानं चकारापि महान् मङ्कणको मुनिः}
{राजा हर्षसमायुक्तः स्वगृहं प्रत्यपद्यत} % ५१

\twolineshloka
{तत्राऽऽदौ कार्तिके मास्यागता कृष्णा चतुर्थिका}
{तां चकार पुरस्थैश्च जनैर्हर्षयुतो नृप} % ५२

\twolineshloka
{ततो भूपेन विख्यातं कृतं व्रतमनुत्तमम्}
{शुक्लकृष्णचतुर्थीजं व्रतं चक्रुर्जना भुवि} % ५३

\twolineshloka
{रोगहीनः स राजर्षिरभजद्गणनायकम्}
{जना वन्ध्यत्वहीनाश्चाऽभजंस्ते भावसंयुताः} % ५४

\twolineshloka
{पुत्रे राज्यं परित्यज्य सस्त्रीकः स वनं ययौ}
{तत्र विघ्नेश्वरं भक्त्याऽभजच्चानन्यचेतसा} % ५५

\twolineshloka
{अन्ते स्वानन्दगो भूत्वा ब्रह्मभूतो बभूव ह}
{क्रमेण भूमिसंस्थास्ते ब्रह्मभूता बभूविरे} % ५६

\twolineshloka
{एवं व्रतस्य माहात्म्यं कथितं ते नृपोत्तम}
{अन्यत्त्वं श‍ृणु भावेन परं कार्तिकसम्भवम्} % ५७

\twolineshloka
{चाण्डालः कोऽपि पापात्मा हिंसायां तत्परोऽभवत्}
{जघान ब्राह्मणानन्यान् जन्तून् वनसमाश्रितान्} % ५८

\twolineshloka
{परस्त्रियं वने दृष्ट्वाऽयभत्तां हठसंयुतः}
{एवं नानाविधं पापं चकार दुर्मतिः सदा} % ५९

\twolineshloka
{कदाचित् कार्तिके मासि चतुर्थ्यां वनमाश्रितः}
{कृष्णायां तत्र सर्पेण दष्टः पापपरायणः} % ६०

\twolineshloka
{ततोऽतिभयसंयुक्तः स्वगृहं प्रत्यपद्यत}
{तत्रोपायाः कृता नाना स्वजनैर्विषहारकाः} % ६१

\twolineshloka
{तथापि विषबाधा च न जहौ तं सुदुर्मतिम्}
{ततश्चन्द्रोदये किञ्चित् सावधानो बभूव ह} % ६२

\twolineshloka
{तत्रान्नं भक्षयित्वा स पुनस्तद्गरलेन च}
{व्याकुलः पतितस्तत्र गतचेष्टोऽन्त्यजोऽभवत्} % ६३

\twolineshloka
{पञ्चम्यां स मृतः पापी ब्रह्मभूतो बभूव ह}
{व्रतपुण्यप्रभावेन तदद्भुतमिवाऽभवत्} % ६४

\twolineshloka
{एवं नानाविधा राजन् व्रतपुण्यप्रभावतः}
{बभूविरे ब्रह्मभूता मया वक्तुं न शक्यते} % ६५

\twolineshloka
{एतत्ते कथितं सर्वं माहात्म्यं कार्तिके नृप}
{सङ्कष्ट्या यस्तु पठति श‍ृणोतीप्सितमालभेत्} % ६६

॥ॐ तत्सदिति श्रीमदान्त्ये पुराणोपनिषदि श्रीमन्मौद्गले महापुराणे चतुर्थे खण्डे गजाननचरिते कार्तिककृष्णचतुर्थीमाहात्म्यवर्णनं नाम एकोनत्रिंशोऽध्यायः॥४.२९॥


\sect{४.३० --- त्रिंशोऽध्यायः --- मार्गशीर्षकृष्णचतुर्थीमाहात्म्यवर्णनम्}

\centerline{॥ श्रीगणेशाय नमः ॥}

\uvacha{दशरथ उवाच}

\twolineshloka
{मार्गशीर्षे च कृष्णायाश्चतुर्थ्या वद मानद}
{माहात्म्यं सर्वदं स्वामिन् येन तृप्तो भवाम्यहम्} % १

\uvacha{वसिष्ठ उवाच}

\twolineshloka
{इतिहासं पुरावृत्तं कथयामि सविस्तरम्}
{राजा बभूव तेजस्वी हरिश्चन्द्रः प्रतापवान्} % २

\twolineshloka
{अयोध्यायां स्थितः सोऽपि नानाधर्मपरायणः}
{सदा सद्गुणसंयुक्तो यज्वा कर्मपरायणः} % ३

\twolineshloka
{यद्यच्च याचते विप्रस्तद्ददाति स हर्षतः}
{तस्य पुण्येन तुल्यं तु न बभूव धरातले} % ४

\twolineshloka
{ततश्छलयितुं योगी विश्वामित्रः समाययौ}
{सर्वराज्यं नृपस्यैव जगृहे दानमार्गतः} % ५

\twolineshloka
{राज्यभ्रष्टं नृपं कृत्वा परप्रान्ते गतं पुनः}
{नानामिषेण तं तत्राच्छलयन् मुनिसत्तमः} % ६

\twolineshloka
{ततः सोऽपि नृपोऽत्यन्तं धर्मयुक्तो बभूव ह}
{न चचालातिदुःखेन संयुक्तः परमार्थवित्} % ७

\twolineshloka
{तत्रैकदा चतुर्थी सा मार्गशीर्षे समागता}
{कृष्णा नृपः क्षुधायुक्तो बभूव तृषया युतः} % ८

\twolineshloka
{छलितो मुनिनाऽत्यन्तं ततश्चन्द्रोदयोऽभवत्}
{तत्रान्नं भक्षयामास राजेन्द्रो दैवयोगतः} % ९

\twolineshloka
{पञ्चम्यां बुद्धिसम्भेदो विश्वामित्रस्य चाऽभवत्}
{तज्ज्ञात्वा विस्मितो योगी विचचार स्वचेतसि} % १०

\twolineshloka
{ततस्तेन व्रतं तत्र ज्ञातं राज्ञा कृतं महत्}
{प्रत्यक्षं वरदो भूत्वा हरिश्चन्द्रमुवाच ह} % ११

\twolineshloka
{कुरु राज्यं महाबुद्धे श‍ृणु मे परमं वचः}
{तव राज्ये चतुर्थीजं व्रतं नष्टं बभूव ह} % १२

\twolineshloka
{तेन त्वं पीडया युक्तः प्रजातोऽसि नृपाधम}
{अहं सङ्क्षुभितो वाऽपि तदर्थं नात्र संशयः} % १३

\twolineshloka
{सर्वादौ तद्व्रतं कार्यं चतुःपादप्रदं नृप}
{नित्यवन्नात्र सन्देहः सर्वसिद्धिप्रदायकम्} % १४

\twolineshloka
{त्वया नानाविधं पुण्यं कृतं विधियुतं महत्}
{तत्सर्वं निष्फलं जातं व्रतहीनप्रभावतः} % १५

\twolineshloka
{जनैः सर्वैर्महाराज नरके त्वं पतिष्यसि}
{त्वया कृतं व्रतं मुख्यमधुनाऽज्ञानतो महत्} % १६

\twolineshloka
{तेनाऽहं प्रीतिसंयुक्तः कृतो वै राजसत्तम}
{तत् कुरुष्व महीपाल जनैः सौख्यमवाप्स्यसि} % १७

\twolineshloka
{एवमुक्त्वा ददौ तस्मै मन्त्रमष्टाक्षरं मुनिः}
{सविधिं गणनाथस्य सोऽपि तं पुनरब्रवीत्} % १८

\uvacha{हरिश्चन्द्र उवाच}

\twolineshloka
{गणेशस्य स्वरूपं मे वद सर्वज्ञ ते नमः}
{व्रतस्यैव प्रभावं त्वं विधियुक्तं महामुने} % १९

\uvacha{विश्वामित्र उवाच}

\twolineshloka
{गणेश्वरस्य माहात्म्यं श‍ृणु सङ्क्षेपतो नृप}
{एक एव गणेशोऽयं सर्वे तस्य गणाः स्मृताः} % २०

\twolineshloka
{मनोवाणीमयं विश्वं तदेव गणवाचकम्}
{नानाभावसमायुक्तं मया वक्तुं न शक्यते} % २१

\twolineshloka
{मनोवाणीविहीनं यन्नानाब्रह्मप्रवाचकम्}
{गणरूपं तु विज्ञेयं वेदे पश्य महीपते} % २२

\twolineshloka
{सिद्धिर्भ्रान्तिकरी प्रोक्ता बुद्धिर्भ्रान्तिधरा नृप}
{ताभ्यां क्रीडति विघ्नेशश्चित्ते चिन्तामणिः स्थितः} % २३

\twolineshloka
{तं भजस्व विधानेन तदा शान्तिमवाप्स्यसि}
{राज्यकर्ताऽपि वन्द्यस्त्वं योगिनां प्रभविष्यसि} % २४

\twolineshloka
{चित्तं पञ्चविधं प्रोक्तं तत्र मोहश्च पञ्चधा}
{समोऽहं चित्तमुत्सृज्य चिन्तामणिर्भविष्यसि} % २५

\twolineshloka
{एवमुक्त्वा व्रतस्यैव माहात्म्यं मुनिपुङ्गवः}
{श्रावयामास तस्मै तत्ततश्चान्तर्हितोऽभवत्} % २६

\twolineshloka
{विश्वामित्रं गतं ज्ञात्वा राजा हर्षसमन्वितः}
{अयोध्यायां ययौ वेगात् प्रधानैरनुमोदितः} % २७

\twolineshloka
{नागरैः सह राजाऽपि चकार व्रतमुत्तमम्}
{शुक्लकृष्णचतुर्थीजं गणेशभजने रतः} % २८

\twolineshloka
{सर्वत्रैवं प्रशस्तं तद् व्रतं चकार हर्षतः}
{भूमिसंस्था जनाः सर्वे व्रतं चक्रुर्विशेषतः} % २९

\twolineshloka
{नानेन सदृशं किञ्चित् सर्वदं व्रतमुत्तमम्}
{ततो रोगादिभिर्हीनो बभूवे हर्षसंयुतः} % ३०

\twolineshloka
{राजा गणपतिं नित्यमभजन्नान्यचेतसा}
{पुत्रे राज्यं विनिक्षिप्य सस्त्रीकश्च ययौ वनम्} % ३१

\twolineshloka
{तत्रैव गणराजं सोऽभजत् सुतपसा युतः}
{शान्तिं प्राप्तो विशेषेणाऽभवद्योगी स पार्थिवः} % ३२

\twolineshloka
{अन्ते स्वानन्दगो भूत्वा गणेशे लीनतां गतः}
{एवं व्रतस्य माहात्म्यं कथितं तेऽजनन्दन} % ३३

\twolineshloka
{जनाः सर्वे क्रमेणैव ब्रह्मभूता बभूविरे}
{चतुर्थीव्रतजेनैव पुण्येन परमेण ते} % ३४

\twolineshloka
{अन्यच्छृणु महीपाल मालवे नगरं महत्}
{कर्णा नाम च तत्राऽऽसीद्वेश्या नरविमोहिनी} % ३५

\twolineshloka
{लोकान् सा हावभावेन मोहयामास तत्क्षणात्}
{नाना जनाः स्वधर्मं त्यक्त्वाऽभवन् वै तदात्मकाः} % ३६

\twolineshloka
{स्वस्वस्त्रियं परित्यज्य तन्निष्ठास्ते बभूविरे}
{मद्यमांसरताः सर्वे नराधिप कृतास्तया} % ३७

\twolineshloka
{ब्राह्मणादय एवं ते भ्रष्टा जाता विशेषतः}
{चौर्यादिकर्मसंयुक्ताः किं वदामि महामते} % ३८

\twolineshloka
{कथने परदोषाणां दोषी भवति मानवः}
{एतादृशी तु सा वेश्या पापरूपाऽतिदारुणा} % ३९

\twolineshloka
{नानाजनैर्युता राजन्नरके कृतनिश्चया}
{न विरामं चकाराऽपि नराणां मोहने कदा} % ४०

\twolineshloka
{एकदा ज्वरसंयुक्ता मार्गशीर्षे बभूव च}
{चतुर्थ्यां कृष्णपक्षे साऽभवद्दाहेन पीडिता} % ४१

\twolineshloka
{पञ्चम्यां सा मृता तत्र ब्रह्मभूता बभूव ह}
{व्रतस्यैव प्रभावेण पतितानां सहायिनी} % ४२

\twolineshloka
{एवं नाना जना राजन् ब्रह्मभूता बभूविरे}
{तत्र ते कति वै ब्रूयां नालं वर्षायुतैरपि} % ४३

\twolineshloka
{मार्गशीर्षचतुर्थ्या यन् माहात्म्यं कथितं परम्}
{पठनात् श्रवणात् सर्वसिद्धिदं भवति ह्यहो} % ४४

॥ॐ तत्सदिति श्रीमदान्त्ये पुराणोपनिषदि श्रीमन्मौद्गले महापुराणे चतुर्थे खण्डे गजाननचरिते मार्गशीर्षकृष्णचतुर्थीमाहात्म्यवर्णनं नाम त्रिंशोऽध्यायः॥४.३०॥


\sect{४.३१ --- नामैकत्रिंशोऽध्यायः --- पौषकृष्णचतुर्थीमाहात्म्यवर्णनम्}

\centerline{॥ श्रीगणेशाय नमः ॥}

\uvacha{दशरथ उवाच}

\twolineshloka
{पौषकृष्णचतुर्थ्यास्तु माहात्म्यं वद भो गुरो}
{न तृप्यामि कथां श‍ृण्वन् सर्वसिद्धिकरीं पराम्} % १

\uvacha{वसिष्ठ उवाच}

\twolineshloka
{इतिहासं प्रवक्ष्यामि पुरातनभवं महान्}
{राजा वीरसहो नाम बभूवे हस्तिनापुरे} % २

\twolineshloka
{बलवान्नीतियुक्तश्चाभवच्छस्त्रास्त्रपारगः}
{धर्मशीलो वदान्यः स मान्यान् मानयिता भृशम्} % ३

\twolineshloka
{नानादानरतश्चापि वान्ध्यदोषयुतोऽभवत्}
{तदर्थं यत्नमास्थाय देवसेवापरोऽभवत्} % ४

\twolineshloka
{ब्राह्मणैः पुत्रकामार्थं पुत्रीयामिष्टिमाहरत्}
{नैव लेभे स तदपि पुत्रं वंशविवर्धनम्} % ५

\twolineshloka
{तीर्थक्षेत्रादिकस्थाने नानाधर्मानकारयत्}
{स्वयं पूजापरो भूत्वा न लेभे सन्ततिं ततः} % ६

\twolineshloka
{ततोऽतिदुःखितो राजा त्यक्त्वा राज्यं वने ययौ}
{सस्त्रीको भ्रममाणः स शीतोष्णैः पीडितोऽभवत्} % ७

\twolineshloka
{महावनान्तरं गत्वा दुःखशोकसमन्वितः}
{सिंहव्याघ्रादिकान् दृष्ट्वा भयहीनश्चचार ह} % ८

\twolineshloka
{तत्राऽऽजगाम योगीन्द्रस्त्रितो नाम महामुनिः}
{तं दृष्ट्वा प्रणनामाऽऽदौ कृताञ्जलिपुटः स्थितः} % ९

\twolineshloka
{तेन पृष्ट उवाचाऽसौ निजं सर्वं महायशाः}
{रुरोद तं महायोगी जगाद करुणायुतः} % १०

\uvacha{त्रित उवाच}

\twolineshloka
{राज्ये ते नृपशार्दूल सर्वसिद्धिकरं व्रतम्}
{नष्टं चतुर्थीजं पूर्णं तेन वन्ध्योऽसि निश्चितम्} % ११

\twolineshloka
{सर्वादौ तत् प्रकर्तव्यं चतुर्वर्गफलप्रदम्}
{तदा कर्म कृतं सर्वं चतुःपादप्रदं भवेत्} % १२

\twolineshloka
{नानाधर्मपरस्त्वं वै जनास्तथैव मानद}
{फलहीनप्रभावेण पतिष्यन्ति च रौरवे} % १३

\twolineshloka
{एवमुक्त्वा चतुर्थ्यास्तु माहात्म्यं मुनिसत्तमः}
{श्रावयामास सम्पूर्णं ततस्तं त्वब्रवीन्नृपः} % १४

\uvacha{वीरसह उवाच}

\twolineshloka
{गणेश्वरस्य माहात्म्यं वद मे करुणानिधे}
{एतादृशं व्रतं यस्य तं भजिष्यामि नित्यदा} % १५

\uvacha{त्रित उवाच}

\twolineshloka
{गणेशस्य स्वरूपं यद्वक्तुं शक्यं कथं भवेत्}
{उपाधियुक्तं भावेन कथयामि श‍ृणुष्व तत्} % १६

\twolineshloka
{अहं पुरा तपोयुक्तः शापानुग्रहणे क्षमः}
{एको द्वितश्च मे राजन् भ्रातरौ विद्यया युतौ} % १७

\twolineshloka
{ताभ्यां यज्ञार्थमेवाहं गतो राज्ञो महात्मनः}
{अभवं कारयित्वा तं स्वगृहे गन्तुमुत्सुकः} % १८

\twolineshloka
{दक्षिणापशुभिर्युक्तं विपुलैर्मार्गसंस्थितम्}
{भ्रातरौ लोकसंयुक्तौ कूपे क्षिप्त्वा प्रजग्मतुः} % १९

\twolineshloka
{जलहीने हि कूपेऽहं पतितः खेदसंयुतः}
{तत्र मानसिकं यज्ञं प्रचकार हितावहम्} % २०

\twolineshloka
{आजग्मुस्तत्र देवाद्या हविर्भागार्थमादरात्}
{मया सम्पूजिताः सर्वे तृप्ताः संहर्षिता बभुः} % २१

\twolineshloka
{अमरा ईप्सितं सर्वे वरं मह्यं ददुर्नृप}
{ततः कूपाच्च मां सर्वे बहिः क्षिप्त्वा प्रजग्मिरे} % २२

\twolineshloka
{ततोऽहं याचयित्वाऽन्यं नृपं गोभिः समन्वितः}
{हर्षेण स्वाश्रमं गत्वा संस्थितस्तप आचरन्} % २३

\twolineshloka
{ततोऽतितपसा ज्ञानं सर्वगं ह्यभवत्ततः}
{तपस्त्यक्त्वा च योगेन प्राभवं भूमिसाधकः} % २४

\twolineshloka
{जडोन्मत्तपिशाचाद्या भूमयः साधिता मया}
{अन्ते ब्रह्मगतोऽतिष्ठं समात्माऽहं स्वशान्तिदे} % २५

\twolineshloka
{स्वाश्रमे हर्षयुक्तोऽहं ततो मुद्गल आगतः}
{धर्मशीलो वदान्यश्च मान्यान् मानयिता भृशम्} % २६

\threelineshloka
{योगीन्द्रैर्वन्दितो नित्यं तं दृष्ट्वा प्रणतोऽभवम्}
{पूजयित्वा सुविश्रान्तं तमपृच्छं कृताञ्जलिः}
{शान्तियोगं वद स्वामिन् शान्तिभ्यः शान्तिदायकम्} % २७

\uvacha{मुद्गल उवाच}

\twolineshloka
{असच्छक्तिश्च सत्सूर्यः समो विष्णुर्महामुने}
{अव्यक्तः शङ्करस्तेषां संयोगे गणपोऽभवत्} % २८

\twolineshloka
{संयोगे मायया युक्तो गणेशो ब्रह्मनायकः}
{अयोगे मायया हीनो भव ते मुनिसत्तम} % २९

\twolineshloka
{संयोगायोगयोर्योगे योगो गाणेशसंज्ञकः}
{शान्तिभ्यः शान्तिदः प्रोक्तो भज तं भक्तिसंयुतः} % ३०

\twolineshloka
{एवमुक्त्वा गणेशस्य ददौ मन्त्रं स मुद्गलः}
{एकाक्षरं विधियुतं ततः सोऽन्तर्हितोऽभवत्} % ३१

\twolineshloka
{ततोऽहं गणराजं तमभजं सर्वभावतः}
{तेन शान्तिसमायुक्तश्चरामि ह्यकुतोभयः} % ३२

\twolineshloka
{न गणेशात् परं ब्रह्म न गणेशात् परं तपः}
{न गणेशात् परं कर्म ज्ञानं न गणपात् परम्} % ३३

\twolineshloka
{न गणेशात् परो योगो भक्तिर्न गणपात् परा}
{तस्माच्च सर्वपूज्योऽयं सर्वादौ सिद्धिदायकः} % ३४

\twolineshloka
{गणेशानं परित्यज्य कर्म ज्ञानादिकं चरेत्}
{तत्सर्वं निष्फलं याति भस्मनि प्रहुतं यथा} % ३५

\twolineshloka
{सर्वांस्त्यक्त्वा गणेशानं भजतेऽनन्यचेतसा}
{सर्वसिद्धिं लभेत् सद्यो ब्रह्मभूतः स कथ्यते} % ३६

\twolineshloka
{एवमुक्त्वा त्रितस्तस्मै ददौ मन्त्रं दशाक्षरम्}
{विधियुक्तं ततः साक्षादन्तर्धानं चकार ह} % ३७

\twolineshloka
{राजा वीरसहो राजन् हर्षतः स्वपुरं ययौ}
{प्रधानैर्नागरैः सर्वैर्मानितः संस्थितोऽभवत्} % ३८

\twolineshloka
{तत्रादौ पौषगां कृष्णां चतुर्थीं तां समागताम्}
{चकार स्वपुरस्थैः स विधियुक्तां सुहर्षितः} % ३९

\twolineshloka
{प्रशस्तं तद् व्रतं तेन कृतं सर्वत्र भूमिप}
{शुक्लकृष्णचतुर्थीजं व्रतं चक्रुर्जना भुवि} % ४०

\twolineshloka
{ततः पुत्रं महाशूरं नीतिज्ञं धर्मसंयुतम्}
{लेभे नानागुणैर्युक्तं जनाः संहर्षिता बभुः} % ४१

\twolineshloka
{रोगादिभिर्विनिर्मुक्ता हृष्टाः पुष्टाश्च तेऽभवन्}
{यत्र तत्र चतुर्थीजं माहात्म्यमवदन् जनाः} % ४२

\twolineshloka
{पुत्रं राज्ये समास्थाप्य राजा निर्वृत्तिधारकः}
{सस्त्रीको गणनाथं तमभजद्भावसंयुतः} % ४३

\twolineshloka
{अन्ते ब्रह्ममयः सोऽपि बभूव व्रतपुण्यतः}
{क्रमेण भूमिसंस्थास्ते ब्रह्मभूता बभूविरे} % ४४

\twolineshloka
{एवं व्रतस्य माहात्म्यं कथितं ते नृपोत्तम}
{अन्यच्छृणु महाभाग पापनाशकरं महत्} % ४५

\twolineshloka
{गौडे कश्चिद् वणिक् पापी लोभयुक्तो बभूव ह}
{द्रव्यार्थं पितरं सोऽपि जघान विषदोषतः} % ४६

\twolineshloka
{मातरं स तथा दुष्टो मार्गे नानाजनांस्तथा}
{वणिजो विषयोगेन मारयामास नित्यदा} % ४७

\twolineshloka
{एवं बहुधनो जातः परस्त्रीलालसोऽभवत्}
{मद्यमांसपरो दुष्टो सतीसङ्गेषु लालसः} % ४८

\twolineshloka
{नानामिषेण दुष्टः स्म सतीर्यभति वेगतः}
{परस्त्रियस्ततो राज्ञा धृतोऽसौ ताडितोऽभवत्} % ४९

\twolineshloka
{निगडैर्बद्ध एवाऽसौ संस्थितो राजवेश्मनि}
{दैवयोगेन पौषी सा प्राप्ता कृष्णा चतुर्थिका} % ५०

\twolineshloka
{तस्यामन्नादिभिर्हीनः स्थितो दुःखसमन्वितः}
{चन्द्रोदये रक्षकैर्यद्दत्तमन्नमभक्षयत्} % ५१

\twolineshloka
{पञ्चम्यां राजदूतैः स शूलप्रोतो ममार ह}
{ब्रह्मभूतः स वै जातो व्रतपुण्यप्रभावतः} % ५२

\twolineshloka
{एवमज्ञानसंयुक्ता ज्ञानयुक्ता बभूविरे}
{ब्रह्मभूता व्रतस्यैव प्रभावेण महामते} % ५३

\twolineshloka
{तत्रैव कति ते राजन् कथयामि विशेषतः}
{न वक्तुं शक्यते तस्मादुक्तं सङ्क्षेपतो मया} % ५४

\twolineshloka
{इदं पौषस्य सङ्कष्ट्या माहात्म्यं संश‍ृणोति यः}
{पठेद्वा तस्य शुभदं भविष्यति निरन्तरम्} % ५५

॥ॐ तत्सदिति श्रीमदान्त्ये पुराणोपनिषदि श्रीमन्मौद्गले महापुराणे चतुर्थे खण्डे गजाननचरिते पौषकृष्णचतुर्थीमाहात्म्यवर्णनं नामैकत्रिंशोऽध्यायः॥४.३१॥


\sect{४.३२ --- द्वात्रिंशोऽध्यायः --- ?? माहात्म्यवर्णनम्}

\centerline{॥ श्रीगणेशाय नमः ॥}

\uvacha{दशरथ उवाच}

\twolineshloka
{मलमासे च सङ्कष्ट्याश्चतुर्थ्या वद मानद}
{माहात्म्यं प्रीतिदं पूर्णं न तृप्तोऽहं कृपानिधे} % १

\uvacha{वसिष्ठ उवाच}

\twolineshloka
{आन्ध्रदेशे महाराजो बभूवे रैवतान्तरे}
{धर्मदत्त इति ख्यातो नानाधर्मपरायणः} % २

\twolineshloka
{यज्वा विनीतकोऽत्यन्तं नीतिज्ञोऽखिलतोषकृत्}
{षष्ठांशग्राहकः सोऽपि प्रजापालनतत्परः} % ३

\twolineshloka
{जलोदरयुतोऽकस्मादभवन् मानदायकः}
{सार्वभौमोऽतिपापेन गृहीतः सर्वशास्त्रवित्} % ४

\twolineshloka
{नानोपायाः कृतास्तस्य शान्तये नृपसत्तम}
{तथाऽपि रोगनिर्मुक्तो न बभूव महीपतिः} % ५

\twolineshloka
{तुलादानादिकान्येव महादानानि पीडितः}
{चकार रोगयुक्तस्तदप्यत्यन्तमजायत} % ६

\twolineshloka
{परिभ्रमन् स तीर्थानि चकार विधिसंयुतः}
{देवताऽऽराधनं राजा नाऽभवद्रोगवर्जितः} % ७

\twolineshloka
{ततोऽतिदुःखितो राजा जलेनैवात्मघातनम्}
{विषेण कर्तुमुद्युक्तो नानाजनैः सुविह्वलैः} % ८

\twolineshloka
{ततोऽकस्मान् महायोगी चाष्टावक्रः समाययौ}
{तद्गृहे तं नमस्कृत्य पूजयामास बान्धवैः} % ९

\twolineshloka
{भोजितं तं प्रणम्यैव कृत्वा करपुटं नृपः}
{धर्मदत्त उवाचाऽसौ विनयावनतोऽभवत्} % १०

\uvacha{धर्मदत्त उवाच}

\twolineshloka
{धन्योऽस्म्यनुगृहीतोऽस्मि सफलो मे भवोऽभवत्}
{तव दर्शनतो विप्र साक्षाद्योगीश्वरस्य तु} % ११

\twolineshloka
{आत्मघातेऽहमुद्युक्तस्तत्र ते दर्शनं महत्}
{प्राप्तं पुण्यसमूहेन नरको न भविष्यति} % १२

\uvacha{वसिष्ठ उवाच}

\twolineshloka
{एवं वदन्तमानन्दयुक्तो योगीन्द्रसत्तमः}
{उवाच तं महाराजं दयया करुणानिधिः} % १३

\uvacha{अष्टावक्र उवाच}

\twolineshloka
{किमर्थं देहघाते त्वमुद्यतोऽसि महामते}
{रोगदुःखविनिर्मुक्तो भविष्यसि न संशयः} % १४

\twolineshloka
{तव राज्ये महीपाल चतुर्थीमुख्यकव्रतम्}
{नष्टं तेन महोग्रेण पापेन संयुतो भवान्} % १५

\twolineshloka
{राजा राष्ट्रकृतं पापं त्वया प्राप्तं तथैव तत्}
{त्वमपि व्रतसंहीनः पतितोऽसि नृपाधम} % १६

\twolineshloka
{सर्वादौ तद्व्रतं कार्यं चतुर्वर्गफलप्रदम्}
{तदा कर्म कृतं राजन् फलयुक्तं भवेत् किल} % १७

\twolineshloka
{त्वया धर्मयुतं कर्म कृतं नानाविधं तथा}
{चतुःपदार्थहीनं तद्बभूव ह जनैः कृतम्} % १८

\twolineshloka
{सर्वैः सह जनै राजन्नरके त्वं पतिष्यसि}
{तदर्थं व्रतराजं तत् कुरु सर्वसमन्वितः} % १९

\twolineshloka
{एवमुक्त्वा चतुर्थ्या यन् माहात्म्यं मुनिसत्तमः}
{श्रावयामास तस्मै स करुणायुतचेतसा} % २०

\twolineshloka
{श्रुत्वा मुख्यं स माहात्म्यं धर्मदत्तः सुविस्मितः}
{उवाच तं हि योगीन्द्रं हर्षयुक्तेन चेतसा} % २१

\uvacha{धर्मदत्त उवाच}

\twolineshloka
{पूर्वपुण्यं महन् मेऽस्ति कथनाय न शक्यते}
{मयेदृशं व्रतं विप्र श्रुतं तेन महाद्भुतम्} % २२

\twolineshloka
{अधुना विघ्नराजस्य स्वरूपं मे दयानिधे}
{कथयस्व सदा देवं भजिष्यामि महाप्रभुम्} % २३

\twolineshloka
{न सामान्यं चरित्रं तत्तस्य देवाधिपस्य यत्}
{यस्येदृशं व्रतं मुख्यं सर्वसिद्धिप्रदं भवेत्} % २४

\uvacha{अष्टावक्र उवाच}

\twolineshloka
{धर्मदत्त महाभाग श‍ृणु त्वं गणपस्य यत्}
{यस्येदृशं व्रतं मुख्यं सर्वसिद्धिप्रदं भवेत्} % २५

\twolineshloka
{निर्गुणं गजवाच्यं यद्ब्रह्म मस्तकमेव च}
{तस्य देहः सगुणकं ब्रह्म योगे गजाननः} % २६

\twolineshloka
{सिद्धिर्भ्रान्तिस्वरूपा सा बुद्धिर्भ्रान्तिधराऽभवत्}
{ताभ्यां क्रीडति विघ्नेशो मायाभ्यां परमार्थतः} % २७

\twolineshloka
{स्वसंवेद्येन योगेन लभ्यते दर्शनं महत्}
{तेन स्वानन्दवासी स कथ्यते योगिभिः प्रभुः} % २८

\twolineshloka
{संयोगेऽयं गकारः सो योगेऽयं च णकारकः}
{तयोः स्वामी गणेशानः शान्तियोगेन लभ्यते} % २९

\twolineshloka
{पञ्चधा चित्तवृत्तिर्या तस्यां मोहस्तु पञ्चधा}
{समोहं चित्तमुत्सृज्य लभ्यते चित्तचालकः} % ३०

\twolineshloka
{तेनाऽयं गणनाथस्तु चिन्तामणिरिति स्मृतः}
{तं भजस्व महाभाग तदा शान्तो भविष्यसि} % ३१

\twolineshloka
{एवमुक्त्वा ददौ तस्मै मन्त्रं मालात्मकं प्रभुः}
{सविधिं धर्मदत्ताय ततः सोऽन्तर्हितोऽभवत्} % ३२

\twolineshloka
{राजाऽपि हर्षसंयुक्तो बभूवातितरां ततः}
{धन्यमात्मानमेवं स्म मन्यते भाग्यगौरवात्} % ३३

\twolineshloka
{तत्रादौ मलमासे या चतुर्थी कृष्णपक्षगा}
{समागता जनै राजा तां चकार विधानतः} % ३४

\twolineshloka
{ततस्तेन व्रतं मुख्यं तत् सर्वत्र प्रकाशितम्}
{ततो जना व्रतं चक्रुर्भूमिस्था यत्र तत्र ते} % ३५

\twolineshloka
{ततो रोगविनिर्मुक्तो धर्मदत्तो बभूव ह}
{जनाः सर्वे तथा हृष्टा बभूवू रोगवर्जिताः} % ३६

\twolineshloka
{नित्यं गणपतिं राजाऽभजच्चानन्यमानसः}
{गणेशप्रीतये नानादानानि स ददौ सदा} % ३७

\twolineshloka
{ततो राज्ये स्वपुत्रं स संस्थाप्यैकान्तसंस्थितः}
{सस्त्रीको गणनाथं तमभजद्ध्यानमार्गतः} % ३८

\twolineshloka
{अन्ते जगाम विघ्नेशं तल्लीनः स बभूव ह}
{जनाः सर्वे क्रमेणैते ब्रह्मभूता बभूविरे} % ३९

\twolineshloka
{एवं व्रतस्य माहात्म्यं सर्वसिद्धिप्रदायकम्}
{कथितं ते महीपाल श‍ृण्वन्यत्त्वं कथानकम्} % ४०

\twolineshloka
{गान्धारदेशवासीयः क्षत्रियः पापकारकः}
{बभूव स परद्रव्यहारकश्चौर्यमार्गतः} % ४१

\twolineshloka
{द्यूतमद्यरतो नित्यं परस्त्रीलम्पटोऽभवत्}
{गोमांसभक्षणे सक्तो मिथ्यावाक् साक्ष्यपूरकः} % ४२

\twolineshloka
{नानाभेदैः सुदुर्बुद्धिः कलहं यत्र तत्र वै}
{चकार सर्वदा राजन् हिंसाकर्मपरायणः} % ४३

\twolineshloka
{ततो राज्ञा महापापी वने निष्कासितो बलात्}
{जघान मार्गगांस्तत्र जन्तून्नानाविधान् खलः} % ४४

\twolineshloka
{द्विजादीनां वधे सक्तः स्त्रीबालवधकारकः}
{एवं नानाविधं पापं नित्यमेव चकार सः} % ४५

\twolineshloka
{ततो द्रव्ययुतोऽत्यन्तं बुभुजे भोगमुत्तमम्}
{हृदीप्सितं स दुष्टात्मा नानाचौरसमन्वितः} % ४६

\twolineshloka
{कदाचित् पर्वतद्रोणीं मलमासे स आश्रितः}
{चतुर्थ्यां कृष्णपक्षस्य हिंसार्थं लोभसंयुतः} % ४७

\twolineshloka
{ततो मानवमेकं तु मार्गस्थं स ददर्श ह}
{निःसृत्याधावदत्यन्तं शस्त्रहस्तो वधाय वै} % ४८

\twolineshloka
{पपाल क्रूरकर्माणं दृष्ट्वा तं सोऽपि भीतितः}
{क्रोशमात्रं ततो दुष्टः पपात पदव्युत्क्रमात्} % ४९

\twolineshloka
{पतितं तं समाज्ञाय नरः क्रोधसमन्वितः}
{पाषाणेन महादुष्टं मारयामास मस्तके} % ५०

\twolineshloka
{मस्तकः स्फुटितस्तत्र क्षत्रियस्य नराधिप}
{तेन स व्याकुलोऽत्यन्तं गतचेष्टो बभूव ह} % ५१

\twolineshloka
{रात्रौ चन्द्रोदये किञ्चित् सावधानोऽभवत् कुधीः}
{पायसं भक्षयामास स्वल्पं दुःखसमन्वितः} % ५२

\twolineshloka
{ततो ममार पञ्चम्यां गतः स्वानन्दके पुरे}
{तत्र दृष्ट्वा गणेशानं ब्रह्मभूतो बभूव ह} % ५३

\twolineshloka
{एवं व्रतप्रभावेण पापकर्मा स क्षत्रियः}
{ब्रह्मभूतो बभूवाऽपि किं पुनर्ज्ञानतः कृतम्} % ५४

\twolineshloka
{नानाजना व्रतेनैव भुक्त्वा भोगान् हृदीप्सितान्}
{ब्रह्मभूता बभूवुश्च कथितुं तन्न शक्यते} % ५५

\twolineshloka
{इदं कृष्णचतुर्थ्या यन् माहात्म्यं मलमासके}
{श‍ृणोति यस्तु पठति स ईप्सितमवाप्नुयात्} % ५६

॥ॐ तत्सदिति श्रीमदान्त्ये पुराणोपनिषदि श्रीमन्मौद्गले महापुराणे चतुर्थे खण्डे गजाननचरिते मलमासकृष्णपक्षचतुर्थीमाहात्म्यवर्णनं नाम द्वात्रिंशोऽध्यायः॥४.३२॥


\sect{४.३३ --- त्रयस्त्रिंशोऽध्यायः --- शमीमूलचतुर्थीव्रताचरणवर्णनम्}

\centerline{॥ श्रीगणेशाय नमः ॥}

\uvacha{दशरथ उवाच}

\twolineshloka
{शमीमूले च यस्तिष्ठेच्चतुर्थ्यां मुनिसत्तम}
{सङ्कष्ट्यां तस्य माहात्म्यं वद मे करुणानिधे} % १

\uvacha{वसिष्ठ उवाच}

\twolineshloka
{इतिहासं प्रवक्ष्यामि पुरातनभवं नृप}
{सर्वसिद्धिकरं पूर्णं श्रवणात् पठनान्नृणाम्} % २

\twolineshloka
{पाञ्चाले ब्राह्मणः कश्चिदत्रिगोत्रसमुद्भवः}
{पूर्वकर्मविपाकेन कुष्ठयुक्तो बभूव ह} % ३

\twolineshloka
{नानाक्षतसमाकीर्णः प्रस्रवत् पूयशोणितः}
{कीटकैः सर्वतो व्याप्तो ह्यन्धः कुब्जोऽभवन्नृप} % ४

\twolineshloka
{तं दृष्ट्वा तादृशं पुत्रं पिता परमदुःखितः}
{अत्रिं जगाम शरणं तं पप्रच्छ प्रणम्य सः} % ५

\twolineshloka
{तेनापि कथितं राजन् शमीमूले स्थितो भवेत्}
{चतुर्थीं सङ्कटीं कुर्यात्तदा पापात् प्रमुच्यते} % ६

\twolineshloka
{ततः सोऽपि स्वपुत्रं तं शमीमूले महीपते}
{समानाय्य बबन्धैव दाम्ना सर्वार्थसाधकः} % ७

\twolineshloka
{सूर्योदयात् समारभ्य यावच्चन्द्रोदयोऽभवत्}
{तावत्तत्र महारोगी सोऽतिष्ठद्बन्धनाकुलः} % ८

\twolineshloka
{ततः पित्रा समानीतः स्वगृहे सोऽपि तत्क्षणात्}
{नेत्रयुक्तो बभूवाऽथ रोगादिभिर्विवर्जितः} % ९

\twolineshloka
{कुब्जताहीनरूपं तं संस्थितं विगतज्वरम्}
{तद् दृष्ट्वा परमाश्चर्यं विप्राः सर्वे विसिस्मिरे} % १०

\twolineshloka
{ततः सोऽपि गणेशानमभजन्नित्यमादरात्}
{चतुर्थ्यां कृष्णपक्षे वै शमीमूले स्म तिष्ठति} % ११

\twolineshloka
{अन्ते शान्तिसमायुक्तो ब्रह्मभूतो बभूव ह}
{एवं तिष्ठेच्छमीमूले सोऽसाध्यं कुरुते वशम्} % १२

\twolineshloka
{अन्यच्छृणु महीपाल माहात्म्यं सर्वसिद्धिदम्}
{चोरः कश्चिन् महान् पापी नगरे स गतोऽभवत्} % १३

\twolineshloka
{स्फोटयित्वा गृहं किञ्चित्तेन तत्र प्रविश्य सः}
{धनं गृहीत्वा निर्यातो नरैर्दृष्टः पपाल सः} % १४

\twolineshloka
{जनैः कोलाहलस्तत्र कृतः परमदारुणः}
{श्रुत्वा राज्ञो ययुर्दूताः शस्त्रहस्ता महाबलाः} % १५

\twolineshloka
{ज्ञात्वा वृत्तान्तमुग्रं ते धाविताः क्रोधसंयुताः}
{वेगेन तं महाचौरं जघ्नुः शस्त्रप्रहारतः} % १६

\twolineshloka
{छिन्नाङ्गं पतितं चौरं दूताः सङ्गृह्य दारुणम्}
{तं बबन्धुः शमीवृक्षे ततो राजानमाययुः} % १७

\twolineshloka
{समागता चतुर्थी सा तत्राऽकस्मान् महीपते}
{कृष्णा सोऽपि महाचौरो शमीमूलस्थितोऽभवत्} % १८

\twolineshloka
{पञ्चम्यां क्षतहीनाङ्गं चौरं सर्वत्र शोभनम्}
{सञ्जातं राजदूतास्ते ज्ञात्वा राज्ञे न्यवेदयन्} % १९

\twolineshloka
{राजा तत्र समागम्य तं प्रणम्य पुरःस्थितः}
{एतस्मिन्नन्तरे तत्र विमानं सहसाऽऽगतम्} % २०

\twolineshloka
{तं तत्रागृह्य निक्षिप्य गणान् शुण्डाविराजितान्}
{तान् प्रणम्य स पप्रच्छ राजर्षिर्गन्तुमुत्सुकान्} % २१

\uvacha{महासेन उवाच}

\twolineshloka
{भो भो गणपतेर्दूताः किमनेन कृतं महत्}
{वदेत करुणायुक्तास्तद्वयं करवामहै} % २२

\uvacha{गाणेशा ऊचुः}

\twolineshloka
{कृष्णपक्षे चतुर्थ्यां यः शमीमूले जपेन्मनुम्}
{तिष्ठेत् सूर्योदयाद्यावत्तथा चन्द्रोदयो भवेत्} % २३

\twolineshloka
{निराहारः स विघ्नेशं पूजयेत् व्रतकारकः}
{पश्चाद्विप्रांश्च सम्भोज्य स्वयं भुञ्जीत वाग्यतः} % २४

\twolineshloka
{रात्रौ जागरणं कुर्यात् पञ्चम्यां पूजनं पुनः}
{प्रकुर्यात् गणराजस्य स ईप्सितमवाप्नुयात्} % २५

\twolineshloka
{फलयुक्तः सद्य एव भवत्यत्र न संशयः}
{अन्ते स्वानन्दगो भूत्वा ब्रह्मभूतः स जायते} % २६

\twolineshloka
{एवमुक्त्वा गता दूताश्चौरेण सहिता नृप}
{चौरो विघ्नेश्वरं दृष्ट्वा ब्रह्मभूतो बभूव ह} % २७

\twolineshloka
{ततो राजा महाभागो गणेशभजने रतः}
{स चतुर्थ्यां शमीमूले तिष्ठन्नियमसंयुतः} % २८

\twolineshloka
{ततो राजा गतिं लेभे त्रिलोक्यां व्रतपुण्यतः}
{ब्रह्मादीनां सभायां स गत्वाऽतिष्ठत् स्वयेच्छया} % २९

\twolineshloka
{पुनः स्वनगरे राजा चम्पावत्यां जगाम ह}
{एवं सर्वत्रगो भूत्वा रराजे राजमण्डले} % ३०

\twolineshloka
{अन्ते सर्वजनै राजा नगरस्थैर्महामते}
{जगाम विघ्नराजं तं ब्रह्मभूतो बभूव ह} % ३१

\twolineshloka
{इदं ते मुख्यमाहात्म्यं शमीमूलस्थितेर्महत्}
{चतुर्थ्यां कथितं राजन् सङ्कष्ट्यां सर्वसिद्धिदम्} % ३२

\twolineshloka
{श‍ृणुयाद्यः पठेद्वा स सद्य एव फलं लभेत्}
{ईप्सितं गणराजस्य प्रसादान्नात्र संशयः} % ३३

\twolineshloka
{एवं नाना जना राजन्नान्ध्यदोषादिवर्जिताः}
{बभूवुश्चरितं तेषां मया वक्तुं न शक्यते} % ३४

॥ॐ तत्सदिति श्रीमदान्त्ये पुराणोपनिषदि श्रीमन्मौद्गले महापुराणे चतुर्थे खण्डे गजाननचरिते शमीमूलचतुर्थीव्रताचरणवर्णनं नाम त्रयस्त्रिंशोऽध्यायः॥४.३३॥


\sect{४.३४ --- चतुस्त्रिंशोऽध्यायः --- वसिष्ठदशरथसंवादसमाप्तिवर्णनम्}

\centerline{॥ श्रीगणेशाय नमः ॥}

\uvacha{दशरथ उवाच}

\twolineshloka
{श्रावणादिषु मासेषु लड्वादीन् यस्तु भक्षति}
{तस्य विधियुतं विप्र माहात्म्यं वद सर्वदम्} % १

\uvacha{वसिष्ठ उवाच}

\twolineshloka
{मनोर्वैवस्वतस्यैव पुत्र इक्ष्वाकुसंज्ञकः}
{राज्यलोभी कनिष्ठः सन् व्रतं चकार सिद्धिदम्} % २

\twolineshloka
{बभक्ष श्रावणे कृष्णसङ्कष्ट्यां पञ्चखाद्यजान्}
{लड्डुकान् सप्त भावेन व्रतसाधनतत्परः} % ३

\twolineshloka
{दधिभक्षणमात्रं स भाद्रे कृष्णे चकार ह}
{आश्विने निर्जलं तद्वच्चकार व्रतमद्भुतम्} % ४

\twolineshloka
{कार्तिके दुग्धभुक् राजाऽतिष्ठत् स्वनियमे रतः}
{मार्गशीर्षे जलं पीत्वाऽभजत्तं गणनायकम्} % ५

\twolineshloka
{पौषे गोमूत्रमात्रं तु व्रतं पीत्वा समाचरत्}
{माघे शुक्लतिलान् राजा बभक्ष व्रतकारणात्} % ६

\twolineshloka
{फाल्गुने शर्करायुक्तं मानवोऽभक्षयद् घृतम्}
{चैत्रे तद्वत् पञ्चगव्यं पूजयित्वा विनायकम्} % ७

\twolineshloka
{वैशाखे पद्मबीजं स बभक्ष व्रतधारकः}
{ज्येष्ठे घृतं गवां भक्षंश्चकार व्रतमुत्तमम्} % ८

\twolineshloka
{आषाढे मधुमात्रं स बभक्ष नियमे रतः}
{चन्द्रोदये गणेशानं पूजयित्वा विधानतः} % ९

\twolineshloka
{ततस्तस्य व्रतस्यैव प्रभावेण महामते}
{सुद्युम्नो मनुपुत्रो य इलो ज्येष्ठोऽभवन्नृप} % १०

\twolineshloka
{ततो मुनिभिराद्यं तं राजानं स चकार ह}
{मनुरिक्ष्वाकुमेवं तु सार्वभौमं महाद्युतिम्} % ११

\twolineshloka
{एवं त्वसाध्यकं राजन्निच्छेद्वै यदि मानवः}
{धर्ममार्गेण तत् सोऽपि लभते नात्र संशयः} % १२

\twolineshloka
{एवं नाना जना राजन् व्रतपुण्यप्रभावतः}
{लब्ध्वाऽन्ते ब्रह्मभूताश्च बभूवुर्भूमिमण्डले} % १३

\twolineshloka
{अतस्त्वं राजशार्दूल पापरूपोऽसि साम्प्रतम्}
{तव राज्ये व्रतं नष्टं नृपाधम न बुद्ध्यसे} % १४

\twolineshloka
{जनैः सह व्रतं तस्मात् कुरुष्व त्वं भविष्यसि}
{पुत्रयुक्तश्च विघ्नेशमन्ते गच्छसि निश्चितम्} % १५

\uvacha{मुद्गल उवाच}

\twolineshloka
{प्रणम्य तं दशरथो महाराजो जगाम ह}
{स्वपुरे जनसंयुक्तश्चकार व्रतमुत्तमम्} % १६

\twolineshloka
{ततो भूपेन सर्वत्र प्रशस्तं तद्व्रतं कृतम्}
{चक्रुः सर्वजना भूमौ व्रतं गाणेश्वरं महत्} % १७

\twolineshloka
{व्रतपुण्यप्रभावेण विष्णुः साक्षाद्बभूव ह}
{चतुर्धाऽऽत्मानमाभज्य पुत्रस्तस्य महात्मनः} % १८

\twolineshloka
{रामश्च भरतो राजा लक्ष्मणः शत्रुसूदनः}
{पुत्राश्चत्वार एवं ते बभूवुर्बलसंयुताः} % १९

\twolineshloka
{अन्ते जगाम विघ्नेशं दशरथः प्रतापवान्}
{लोका दुःखविहीनाश्च ब्रह्मभूता बभूविरे} % २०

\twolineshloka
{वसिष्ठर्षेर्दशरथस्येमं विप्रस्य धीमतः}
{संवादं व्रतहेतुं तु यः श‍ृणोति पठेन्नरः} % २१

\twolineshloka
{तस्मै विघ्नेश्वरः साक्षाद्ददाति फलमुत्तमम्}
{पुत्रपौत्रादिसंयुक्तं धनधान्यसमन्वितम्} % २२

\twolineshloka
{नानारोगविहीनत्वमैश्वर्यादियुतं महत्}
{अन्ते गणपतौ लीनं करोत्यत्र न संशयः} % २३

\twolineshloka
{चतुर्थ्याश्चरितं पूर्णं कथितं ते प्रजापते}
{श्रोतुमिच्छा पुनः किं ते वर्तते वद साम्प्रतम्} % २४

॥ॐ तत्सदिति श्रीमदान्त्ये पुराणोपनिषदि श्रीमन्मौद्गले महापुराणे चतुर्थे खण्डे गजाननचरिते वसिष्ठदशरथसंवादसमाप्तिवर्णनं नाम चतुस्त्रिंशोऽध्यायः॥४.३४॥


\sect{४.३५ --- पञ्चत्रिंशोऽध्यायः --- चतुर्थ्युद्यापननिरूपणम्}

\centerline{॥ श्रीगणेशाय नमः ॥}

\uvacha{दक्ष उवाच}

\twolineshloka
{उद्यापनं व्रतस्यास्य वद त्वं मुनिसत्तम}
{येन साङ्गं फलं सर्वं लभते व्रतकारकः} % १

\uvacha{मुद्गल उवाच}

\twolineshloka
{उद्यापनं प्रजानाथ व्रतस्यादौ मतं बुधैः}
{अथवा वर्षमध्ये तत् कर्तव्यं व्रतकारिणा} % २

\twolineshloka
{अथवा भाद्रमासे वै शुक्लायां च प्रजापते}
{उद्यापनं प्रकर्तव्यं कृष्णायां माघमासके} % ३

\twolineshloka
{महामण्डपकं कुर्यात् कदलीस्तम्भमण्डितम्}
{नानाशोभासमायुक्तं विभवे सति मानद} % ४

\twolineshloka
{वित्तशाठ्यं न कुर्यात् स यथाविभवमाचरेत्}
{छत्राऽऽदर्शादिभिः शोभां दीपमालाविराजितम्} % ५

\twolineshloka
{मणिमुक्तादिकैः पूर्णं भूषणैश्च विशेषतः}
{यत्र तत्र यथायोग्यं प्रकुर्याद्व्रतकारकः} % ६

\twolineshloka
{मध्याह्नसमये तस्याः शुक्लायाः कथितं प्रभो}
{चन्द्रोदये च कृष्णायाश्चतुर्थ्याः पूजनं चरेत्} % ७

\twolineshloka
{गोमयेन प्रलिप्याऽथ धान्यराशिं प्रकल्पयेत्}
{तन्मध्येऽष्टदलं पद्मं कुर्यात्तद्गुरुणा स्वयम्} % ८

\twolineshloka
{तत्र संस्थाप्य सौवर्णं कलशं ताम्रजं तथा}
{रौप्यं वा मृन्मयं दक्ष वस्त्रयुग्मेन वेष्टयेत्} % ९

\twolineshloka
{तस्योपरि सुवर्णस्य पात्रं स्थाप्य महामतिः}
{वस्त्रं ततो गणेशस्य यन्त्रं कुर्याद्विचक्षणः} % १०

\twolineshloka
{तत्र गणपतेश्चैव मूर्तिर्धातुविनिर्मिता}
{सर्वावयवसंयुक्ता भूषणै राजिता भवेत्} % ११

\twolineshloka
{द्विजोत्तमश्च संस्थाप्य पूजयेत्तां यथाविधि}
{ब्राह्मणान् वरयेच्छुद्धानेकविंशतिसङ्ख्यकान्} % १२

\twolineshloka
{गणानां त्वेति मन्त्रस्य जपं कुर्याद्विशेषतः}
{ततो होमं प्रकुर्वीत सहस्रं वा तदर्धकम्} % १३

\twolineshloka
{अष्टोत्तरशतं वाऽपि तथा जपदशांशकम्}
{गीतवाद्यपुराणानि वेदाध्ययनमेव च} % १४

\twolineshloka
{नानाशास्त्रप्रवादांश्च कुर्वन्तु ब्राह्मणास्ततः}
{यत्र तत्र गणेशस्य कथाः सङ्कथयन्तु ते} % १५

\twolineshloka
{ततः पूर्णाहुतिं कुर्याद्वसोर्धारां स पातयेत्}
{बलिदानं ततः कुर्यात्ततो वायनकं चरेत्} % १६

\twolineshloka
{एकविंशतिपक्वान्नैरेकविंशतिसङ्ख्यकैः}
{नैवेद्यैर्ब्राह्मणान् सम्भोज्यार्थयेत्तान् विशेषतः} % १७

\twolineshloka
{दक्षिणां विपुलां दत्त्वा सपत्नीकान् प्रतोषयेत्}
{योषिद्भ्यः कञ्चुकीर्दद्याद्भूषणानि महामतिः} % १८

\twolineshloka
{कृष्णायां च चतुर्थ्यां स चन्द्रायार्घ्यं प्रदापयेत्}
{शुक्लायां तत्तथा दक्ष चरेच्चन्द्रार्घ्यवर्जितम्} % १९

\twolineshloka
{आदौ स तिथये दद्यात् पश्चाद्विघ्नेश्वराय वै}
{ततश्चन्द्राय तन् मन्त्रैरर्घ्यं मन्त्रसमन्वितम्} % २०

\twolineshloka
{तिथीनां मातृरूपे त्वं देवि सर्वार्थदायिनि}
{गृहाणार्घ्यं मया दत्तं चतुर्थ्यै ते नमो नमः} % २१

\twolineshloka
{गजानन नमस्तुभ्यं नानासिद्धिप्रदायिने}
{गृहाणार्घ्यं बुद्धिपते मया दत्तं शुभप्रद} % २२

\twolineshloka
{अत्रिगोत्रसमुद्भूत गणेशप्रीतिवर्धन}
{गृहाणार्घ्यं मया दत्तं रोहिण्यामृतधारक} % २३

\twolineshloka
{कृष्णपक्षे सदा रात्रौ भोजनं स समाचरेत्}
{शुक्लपक्षे च पञ्चम्यां ब्राह्मणादींस्तु भोजयेत्} % २४

\twolineshloka
{चतुर्थ्यां जागरं कुर्यात् गणेशकथयान्वितः}
{पञ्चम्यां पूर्ववत् पूज्य नैवेद्याद्यैर्महामतिः} % २५

\twolineshloka
{दीनान्धकृपणेभ्यश्च दद्यादन्नादिकं तथा}
{ब्राह्मणेभ्यो विशेषेण दक्षिणां दापयेन्नरः} % २६

\twolineshloka
{सर्वान् सम्भोज्य दानेन तोषयेद्देवसन्निधौ}
{गणेशदृढभक्तिं स याचयेत्तेभ्य आदरात्} % २७

\twolineshloka
{ततः सोपस्करां मूर्तिं दद्यात् स्वगुरवे स्वयम्}
{एकविंशतिविप्रेभ्यः कलशान् दापयेत्ततः} % २८

\twolineshloka
{सन्तोष्योदारधीः सर्वान् स भावेन क्षमापयेत्}
{ततो गणपतिं नित्यं भजेतानन्यचेतसा} % २९

\twolineshloka
{एवमुद्यापनं कुर्याद्व्रतस्यास्य महामतिः}
{स सम्पूर्णफलं भुक्त्वा ब्रह्मभूतो भविष्यति} % ३०

\twolineshloka
{वक्तुं न शक्यते दक्ष माहात्म्यं च व्रतोद्भवम्}
{सोद्यापनं विशेषेण कथितं ते यथामति} % ३१

\twolineshloka
{इदं चतुर्थीजं दक्ष चरितं संश‍ृणोति चेत्}
{पठेद्वा तस्य विघ्नेशो मानसेप्सितदो भवेत्} % ३२

॥ॐ तत्सदिति श्रीमदान्त्ये पुराणोपनिषदि श्रीमन्मौद्गले महापुराणे चतुर्थे खण्डे गजाननचरिते चतुर्थ्युद्यापननिरूपणं नाम पञ्चत्रिंशोऽध्यायः॥४.३५॥
