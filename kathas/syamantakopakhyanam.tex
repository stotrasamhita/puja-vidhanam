\sect{स्यमन्तकोपाख्यानम्}

\dnsub{सारभूतः श्लोकः}
“सिंहः प्रसेनम् अवधीत् सिंहो जाम्बवता हतः।\\
सुकुमारक मा रोदीः तव ह्येष स्यमन्तकः॥”\\

\dnsub{स्यमन्तकोपाख्यानपठने प्रमाणवचनानि}
कन्यादित्ये चतुर्थ्यां च शुक्ले चन्द्रस्य दर्शनम्।\\
मिथ्याभिदूषणं कुर्यात् तस्मात् पश्येन्न तं तदा॥\\
तद्दोषशान्तये जाप्यं विष्णुनोक्तं स्यमन्तकम्॥\\
(व्रतचूडामण्यादौ व्रतग्रन्थेषु)\\

ये शृण्वन्ति तवाख्यानं स्यमन्तकमणीयकम्॥ \\
चन्द्रस्य चरितं सर्वं तेषां दोषो न जायते॥१३२॥\\
(अत्रैव अग्रे उपाख्याने)\\

\uvacha{नन्दिकेश्वर उवाच}

\twolineshloka
{शृणुष्वैकाग्रचित्तः सन् व्रतं गाणेश्वरं महत्}
{चतुर्थ्यां शुक्लपक्षे तु सदा कार्यं प्रयत्नतः}%॥१॥

\twolineshloka
{सनत्कुमार योगीन्द्र यदीच्छेच्छुभमात्मनः}
{नारी वा पुरुषो वाऽपि यः कुर्याद् विधिवद् व्रतम्}%॥२॥

\twolineshloka
{मोचयत्याशु विप्रेन्द्र सङ्कष्टाद् व्रतिनं हि तत्}
{अपवादहरं चैव सर्वविघ्नप्रणाशनम्}%॥३॥

\twolineshloka
{कान्तारे विषमे वाऽपि रणे राजकुलेऽथवा}
{सर्वसिद्धिकरं विद्धि व्रतानामुत्तमं व्रतम्}%॥४॥

\twolineshloka
{गजाननप्रियं चाथ त्रिषु लोकेषु विश्रुतम्}
{अतो न विद्यते ब्रह्मन् सर्वसङ्कष्टनाशनम्}%॥५॥

\uvacha{सनत्कुमार उवाच}

\twolineshloka
{केन चादौ पुरा चीर्णं मर्त्यलोकं कथं गतम्}
{एतत्समस्तं विस्तार्य ब्रूहि गाणेश्वरं व्रतम्}%॥६॥

\uvacha{नन्दिकेश्वर उवाच}
\twolineshloka
{चक्रे व्रतं जगन्नाथो वासुदेवः प्रतापवान्}
{आदिष्टं नारदेनैव वृथालाञ्छनमुक्तये}%॥७॥

\uvacha{सनत्कुमार उवाच}
\twolineshloka
{षड्गुणैश्वर्य-सम्पन्नः सृष्टिसंहार-कारकः} 
{वासुदेवो जगद्व्यापी प्राप्तवान् लाञ्छनं कथम्}%॥८॥

एतदाश्चर्यमाख्यानं ब्रूहि त्वं नन्दिकेश्वर।

\uvacha{नन्दिकेश्वर उवाच}
\onelineshloka{भूमिभारनिवृत्त्यर्थं वसुदेवसुतावुभौ}%॥९॥

\twolineshloka
{रामकृष्णौ समुत्पन्नौ पद्मनाभ-फणीश्वरौ}
{जरासन्धभयात् कृष्णो द्वारकां समकल्पयत्}%॥१०॥

\twolineshloka
{विश्वकर्माणमाहूय पुरीं हाटकनिर्मिताम्}
{तत्र षोडशसाहस्रं स्त्रीणां चैव शताधिकम्}%॥११॥

\twolineshloka
{भवनानि मनोज्ञानि तेषां मध्ये व्यकल्पयत्}
{पारिजाततरुं मध्ये तासां भोगाय कल्पयत्}%॥१२॥

\twolineshloka
{यादवानां गृहास्तत्र षट् पञ्चाशच्च कोटयः}
{अन्येऽपि बहवो लोका वसन्ति विगतज्वराः}%॥१३॥

\twolineshloka
{यत् किञ्चित् त्रिषु लोकेषु सुन्दरं तत्र दृश्यते}
{सत्राजितप्रसेनाख्यौ पुत्रावुग्रस्य विश्रुतौ}%॥१४॥

\twolineshloka
{अम्भोधितीरमासाद्य तन्मनस्कतया च सः}
{सत्राजितस्तपस्तेपे सूर्यमुद्दिश्य बुद्धिमान्}%॥१५॥

\twolineshloka
{व्रतं निरशनं गृह्य सूर्यसम्बद्धलोचनः}
{ततः प्रसन्नो भगवान् सत्राजितपुरः स्थितः}%॥१६॥

\twolineshloka
{सत्राजितोऽपि तुष्टाव दृष्ट्वा देवं दिवाकरम्}
{तेजोराशे नमस्तेऽस्तु नमस्ते सर्वतोमुख}%॥१७॥

\twolineshloka
{विश्वव्यापिन् नमस्तेऽस्तु नमस्ते विश्वरूपिणे}
{काश्यपेय नमस्तेऽस्तु हरिदश्व नमोऽस्तु ते}%॥१८॥

\twolineshloka
{ग्रहराज नमस्तेऽस्तु नमस्ते चण्डरोचिषे}
{वेदत्रय नमस्तेऽस्तु सर्वदेव नमोऽस्तु ते}%॥१९॥

\twolineshloka
{प्रसीद पाहि देवेश सुदृष्ट्या मां दिवाकर}
{इत्थं संस्तूयमानोऽसौ देवदेवो दिवाकरः}%॥२०॥

{स्निग्धगम्भीरमधुरं सत्राजितमुवाच ह।}

\uvacha{सूर्य उवाच}

\onelineshloka{वरं ब्रूहि प्रदास्यामि यत् ते मनसि वर्तते}%॥२१॥

{सत्राजित महाभाग तुष्टोऽहं तव निश्चयात्।}

\uvacha{सत्राजित उवाच}

\onelineshloka
{स्यमन्तकमणिं देहि यदि तुष्टोऽसि भास्कर}%॥२२॥

{ददौ तस्य च तद् रत्नं स्वकण्ठादवतार्य सः।}

\uvacha{भास्कर उवाच}
\onelineshloka
{भाराष्टकं शातकुम्भं स्रवतेऽसौ महामणिः}%॥२३॥

\threelineshloka
{शुचिष्मता सदा धार्यं रत्नमेतन्महोत्तमम्}
{सत्राजित क्षणेनैतदशुचिं हन्ति मानवम्}
{इत्युक्त्वाऽन्तर्दधे देवस्तेजोराशिर्दिवाकरः}%॥२४॥

\fourlineindentedshloka
{तत्कण्ठरत्नज्वलमानरूपी}
{पुरीं स कृष्णस्य विवेश सत्वरम्} 
{दृष्ट्वा तु लोका मनसा दिवाकरं}
{सञ्चिन्तयन्तो हि विमुष्टदृष्टयः}%॥२५॥

\fourlineindentedshloka
{समागतोऽयं हरिदश्वदीधिति-}
{र्जनार्दनं द्रष्टुमसंशयेन} 
{नायं सहस्रांशुरितीह लोकाः}
{सत्राजितोऽयं मणिकण्ठभास्वान्}%॥२६॥

\twolineshloka
{स्यमन्तकं महारत्नं दृष्ट्वा तत्कण्ठमण्डले}
{स्पृहां चक्रे जगन्नाथो न जहार मणिं तु सः}%॥२७॥

\twolineshloka
{सत्राजितो जातभयो याचयिष्यति मां हरिः}
{प्रसेनाय ददौ भ्रात्रे धार्योऽयं शुचिना त्वया}%॥२८॥

\twolineshloka
{एकदा कण्ठदेशेऽसौ क्षिप्त्वा तं मणिमुत्तमम्}
{मृगया क्रीडनार्थाय ययौ कृष्णेन संयुतः}%॥२९॥

\twolineshloka
{अश्वारूढोऽशुचिश्चासौ हतः सिंहेन तत्क्षणात्}
{रत्नमादाय सिंहोऽपि गच्छन् जाम्बवता हतः}%॥३०॥

\twolineshloka
{नीत्वा स विवरे रत्नं ददौ पुत्राय जाम्बवान्}
{पुरीं विवेश कृष्णोऽपि स्वकैः सर्वैः समावृतः}%॥३१॥

\twolineshloka
{प्रसेनोऽद्यापि नाऽऽयाति हतः कृष्णेन निश्चितम्}
{मणिलोभेन हा कष्टं बान्धवः पापिना हतः}%॥३२॥

\twolineshloka
{द्वारकावासिनः सर्वे जना ऊचुः परस्परम्}
{वृथापवादसन्तप्तः कृष्णोऽपि निरगाच्छनैः}%॥३३॥

\twolineshloka
{सहैव तैर्गतोऽरण्यं दृष्ट्वा सिंहेन पातितम्}
{प्रसेनं वाहनयुतं तत्पदानुचरः शनैः}%॥३४॥

\twolineshloka
{ऋक्षेण निहतं दृष्ट्वा कृष्णश्चर्क्षबिलं गतः}
{विवेश योजनशतमन्धकारं स्वतेजसा}%॥३५॥

\twolineshloka
{निवारयन् ददर्शाग्रे प्रासादं बद्धभूमिकम्}
{तं कुमारं जाम्बवतो दोलायाममितद्युतिम्}%॥३६॥

\twolineshloka
{माणिक्यं लम्बमानं च ददर्श भगवान् हरिः}
{रूपयौवनसम्पन्नां कन्यां जाम्बवतीं पुनः}%॥३७॥

\threelineshloka
{दोलां दोलयमानां च ददर्श कमलेक्षणः}
{महान्तं विस्मयं चक्रे दृष्ट्वा तां चारुहासिनीम्}
{दोलां दोलयमाना सा जगौ गीतमिदं मुहुः}%॥३८॥

\begingroup
\bfseries
\twolineshloka
{सिंहः प्रसेनमवधीत् सिंहो जाम्बवता हतः}
{सुकुमारक मा रोदीस्तव ह्येष स्यमन्तकः}%॥३९॥
\endgroup

\twolineshloka
{मदनज्वरदाहार्ता दृष्ट्वा तं कमलेक्षणम्}
{उवाच ललितं बाला गम्यतां गम्यतामिति}%॥४०॥

\twolineshloka
{रत्नं गृहीत्वा वेगेन यावच्छेते तु जाम्बवान्}
{इत्याकर्ण्य वचः शौरिः शङ्खं दध्मौ प्रतापवान्}%॥४१॥

\twolineshloka
{आकर्ण्य सहसोत्थाय युयुधे ऋक्षराट् ततः}
{तयोर्युद्धमभूद्घोरं हरिजाम्बवतोस्तदा}%॥४२॥

\twolineshloka
{द्वारकावासिनः सर्वे गतास्ते सप्तमे दिने}
{मृतः कृष्णो भक्षितो वा निःसन्दिग्धं विचार्य च}%॥४३॥

\twolineshloka
{परलोकक्रियां चक्रुः परेतस्य तु ते तदा}
{एकविंशद्दिनं यावद् बाहुप्रहरणो विभुः}%॥४४॥

\twolineshloka
{युयुधे तेन ऋक्षेण युद्धकर्मणि तोषितः}
{जाम्बवान् प्राक्तनं स्मृत्वा दृष्ट्वा देवबलं महत्}%॥४५॥

\uvacha{जाम्बवानुवाच}

\twolineshloka
{अजेयोऽहं सुरैः सर्वैर्यक्षराक्षसदानवैः}
{त्वया जितोऽहं देवेश देवस्त्वमसि निश्चितम्}%॥४६॥

\twolineshloka
{जाने त्वां वैष्णवं तेजो नान्यथा बलमीदृशम्}
{इति प्रसाद्य देवेशं ददौ माणिक्यमुत्तमम्}%॥४७॥

\twolineshloka
{सुतां जाम्बवतीं नाम भार्यार्थं वरवर्णिनीम्}
{पाणिं वै ग्राहयामास देवदेवं च जाम्बवान्}%॥४८॥

\twolineshloka
{मणिमादाय देवोऽपि जाम्बवत्याऽपि संयुतः}
{तद्वृत्तान्तं समाचष्टे द्वारकावासिनां स्वयम्}%॥४९॥

\twolineshloka
{सत्राजितस्य माणिक्यं दत्तवान् संसदि स्थितः}
{मिथ्यापवादसंशुद्धिं प्राप्तवान् मधुसूदनः}%॥५०॥

\twolineshloka
{सत्राजितोऽपि सन्त्रस्तः कृष्णाय प्रददौ सुताम्}
{सत्यभामां महाबुद्धिस्तदा सर्वगुणान्विताम्}%॥५१॥

\twolineshloka
{शतधन्वाक्रूरमुखा यादवा दुष्टमानसाः}
{सत्राजितेन ते वैरं चक्रू रत्नाभिलाषिणः}%॥५२॥

\twolineshloka
{दुरात्मा शतधन्वाऽपि गते कृष्णे च कुत्रचित्}
{सत्राजितं निहत्याशु मणिं जग्राह पापधीः}%॥५३॥

\twolineshloka
{कृष्णस्य पुरतः सत्या समाचष्टे विचेष्टितम्}
{अन्तर्हृष्टो बहिःकोपी कृष्णः कपटनायकः}%॥५४॥

\twolineshloka
{बलदेवपुरो वाक्यमुवाच धरणीधरः}
{हत्वा सत्राजितं दुष्टो मणिमादाय गच्छति}%॥५५॥

\twolineshloka
{निहत्य शतधन्वानं गृह्णीमो रत्नमावयोः}
{मम भोग्यं च तद् रत्नं भविष्यति सुनिश्चितम्}%॥५६॥

\twolineshloka
{एतच्छ्रुत्वा  भयत्रस्तः शतधन्वाऽपि यादवः}
{आहूयाक्रूरनामानं माणिक्यं प्रददौ च सः}%॥५७॥

\twolineshloka
{आरुह्य वडवां वेगान्निर्गतो दक्षिणां दिशम्}
{रथस्थावनुगच्छेतां तदा रामजनार्दनौ}%॥५८॥

\twolineshloka
{शतयोजनमात्रेण ममार वडवा तदा}
{पलायमानो निहतः पदातिस्तु पदातिना}%॥५९॥

\twolineshloka
{रथस्थे बलदेवे तु हरिणा रत्नलोभतः}
{न दृष्टं तत्र तद्रत्नं बलदेवपुरोऽवदत्}%॥६०॥

\twolineshloka
{तदाकर्ण्य महारोषादुवाच वचनं बली}
{कपटी त्वं सदा कृष्ण लोभी पापी सुनिश्चितम्}%॥६१॥

\twolineshloka
{अर्थाय स्वजनं हंसि कस्त्वां बन्धुः समाश्रयेत्}
{अनेकशपथैः कृष्णो बलदेवं प्रसादयत्}%॥६२॥

\twolineshloka
{सोऽपि धिक् कष्टमित्युक्त्वा ययौ वैदर्भमण्डलम्}
{कृष्णोऽपि रथमारुह्य द्वारकां प्रययौ पुनः}%॥६३॥

\twolineshloka
{तथैवोचुर्जनाः सर्वे न साधीयानयं हरिः}
{निष्कासितो रत्नलोभाज्ज्येष्ठो भ्राता बलो बली}%॥६४॥

\twolineshloka
{तच्छ्रुत्वा दीनवदनः पापीयानिव संस्थितः}
{वृथाभिशापात् सन्तप्तो बभूव स जगत्पतिः}%॥६५॥

\twolineshloka
{अक्रूरोऽपि विनिष्क्रम्य तीर्थयात्रानिमित्ततः}
{काशीं गत्वा सुखेनासौ यजन् यज्ञपतिं प्रभुम्}%॥६६॥

\twolineshloka
{तोषमुत्पादयामास तेन द्रव्येण बुद्धिमान्}
{सुरालयगृहैश्चित्रैर्नगरं समकल्पयत्}%॥६७॥

\twolineshloka
{न दुर्भिक्षं न वै रोग ईतयो न च विड्वरम्}
{शुचिना धार्यते यत्र मणिः सूर्यस्य निश्चितम्}%॥६८॥

\twolineshloka
{जानन्नपि हि तत् सर्वं मानुषं भावमाश्रितः}
{लोकाचारं तथा मायामज्ञानं च समाश्रितः}%॥६९॥

\twolineshloka
{बन्धुवैरं समुत्पन्नं लाञ्छनं समुपस्थितम्}
{वृथापवादबहुलं जायमानं कथं सहे}%॥७०॥

\twolineshloka
{इति चिन्तातुरं कृष्णं नारदः समुपस्थितः}
{गृहीत्वा तत्कृतां पूजां सुखासीनस्ततोऽब्रवीत्}%॥७१॥

\uvacha{नारद उवाच}

\twolineshloka
{किमर्थं खिद्यसे देव किं वा ते शोककारणम्}
{यथावृत्तं समाचष्टे नारदाय च केशवः}%॥७२॥

\uvacha{नारद उवाच}

\twolineshloka
{जानामि कारणं देव यदर्थं लाञ्छनं तव}
{त्वया भाद्रपदे शुक्लचतुर्थ्यां चन्द्रदर्शनम्}%॥७३॥

{कृतं तेन समुत्पन्नं लाञ्छनं तु वृथैव हि।}

\uvacha{श्रीकृष्ण उवाच}

\onelineshloka
{वद नारद मे शीघ्रं को दोषश्चन्द्रदर्शने}%॥७४}

{किमर्थं तु द्वितीयायां तस्य कुर्वन्ति दर्शनम्।}

\uvacha{नारद उवाच}

\onelineshloka
{गणनाथेन संशप्तश्चन्द्रमा रूपगर्वतः}%॥७५॥

{त्वद्दर्शने नराणां हि वृथानिन्दा भविष्यति।}

\uvacha{श्रीकृष्ण उवाच}
\onelineshloka
{किमर्थं गणनाथेन शप्तश्चन्द्रः सुधामयः}%॥७६॥

{इदमाख्यानकं श्रेष्ठं यथावद् वक्तुमर्हसि।} 

\uvacha{नारद उवाच}
\onelineshloka
{गणानामाधिपत्ये च रुद्रेण विहितः पुरा}%॥७७॥

\twolineshloka
{अणिमा महिमा चैव लघिमा गरिमा तथा}
{प्राप्तिः प्राकाम्यमीशित्वं वशित्वं चाष्टसिद्धयः}%॥७८॥

\twolineshloka
{भार्यार्थं प्रददौ देवो गणेशस्य प्रजापतिः}
{पूजयित्वा गणाध्यक्षं स्तुतिं कर्तुं प्रचक्रमे}%॥७९॥

\uvacha{ब्रह्मोवाच}

\twolineshloka
{गजवक्त्र गणाध्यक्ष लम्बोदर वरप्रद}
{विघ्नाधीश्वर देवेश सृष्टिसंहारकारक}%॥८०॥

\twolineshloka
{यः पूजयेद् गणाध्यक्षं मोदकाद्यैः प्रयत्नतः}
{तस्य प्रजायते सिद्धिर्निविघ्नेन न संशयः}%॥८१॥

\twolineshloka
{असम्पूज्य गणाध्यक्षं ये वाञ्छन्ति सुरासुराः}
{न तेषां जायते सिद्धिः कल्पकोटिशतैरपि}%॥८२॥

\twolineshloka
{त्वद्भक्त्या तु गणाध्यक्ष विष्णुः पालयते सदा}
{रुद्रोऽपि संहरत्याशु त्वद्भक्त्यैव करोम्यहम्}%॥८३॥

\twolineshloka
{इत्थं संस्तूयमानोऽसौ देवदेवो गजाननः}
{उवाच परमप्रीतो ब्रह्माणं जगतां पतिम्}%॥८४॥

\uvacha{श्रीगणेश उवाच}

{वरं ब्रूहि प्रदास्यामि यत् ते मनसि वर्तते।}

\uvacha{ब्रह्मोवाच}
\onelineshloka
{क्रियमाणस्य मे सृष्टिर्निविघ्नं जायतां प्रभो}%॥८५॥

\twolineshloka
{एवमस्त्विति देवोऽसौ गृहीत्वा मोदकान् करे}
{सत्यलोकात् समागच्छन् स्वेच्छया गगने शनैः}%॥८६॥

\twolineshloka
{चन्द्रलोकं समासाद्य चलितो गणनायकः}
{उपहासं तदा चक्रे सोमो रूपमदान्वितः}%॥८७॥

\twolineshloka
{तं दृष्ट्वा कोपताम्राक्षो गणनाथः शशाप ह}
{दर्शनीयः सुरूपोऽहं सुन्दरश्चाहमित्यथ}%॥८८॥

\twolineshloka
{गर्वितोऽसि शशाङ्क त्वं फलं प्राप्स्यसि सत्वरम्}
{अद्यप्रभृति लोकास्त्वां न हि पश्यन्ति पापिनम्}%॥८९॥

\twolineshloka
{ये पश्यन्ति प्रमादेन त्वां नरा मृगलाञ्छनम्}
{मिथ्याभिशापसंयुक्ता भविष्यन्तीह ते ध्रुवम्}%॥९०॥

\twolineshloka
{हाहाकारो महाञ्जातः श्रुत्वा शापं च भीषणम्}
{अत्यन्तं म्लानवदनश्चन्द्रो जलमथाविशत्}%॥९१॥

\twolineshloka
{कुमुदं कौमुदीनाथः स्थितस्तत्र कृतालयः}
{ततो देवर्षिगन्धर्वा निराशा दीनमानसाः}%॥९२॥

\twolineshloka
{तुरासाहं पुरोधाय जग्मुस्ते तं पितामहम्}
{देवं शशंसुश्चन्द्रस्य गणेशस्य च चेष्टितम्}%॥९३॥

\twolineshloka
{दत्तः शापो गणेशेन कथयामासुरादरात्}
{विचार्य भगवान् ब्रह्मा तान् सुरानिदमब्रवीत्}%॥९४॥

\twolineshloka
{गणेशशापो देवेन्द्र शक्यते केन वाऽन्यथा}
{कर्तुं रुद्रेण न मया विष्णुना चापि निश्चितम्}%॥९५॥

\twolineshloka
{तमेव देवदेवेशं व्रजध्वं शरणं सुराः}
{स एव शापमोक्षं च करिष्यति न संशयः}%॥९६॥

\uvacha{देवा ऊचुः}

\twolineshloka
{केनोपायेन वरदो गजवक्त्रो गणेश्वरः}
{पितामह महाप्राज्ञ तदस्माकं वद प्रभो}%॥९७॥

\uvacha{पितामह उवाच}

\twolineshloka
{चतुर्थ्यां देवदेवोऽसौ पूजनीयः प्रयत्नतः}
{कृष्णपक्षे विशेषेण नक्तं कुर्याच्च तद् व्रतम्}%॥९८॥

\twolineshloka
{अपूपैर्घृतसंयुक्तैर्मोदकैः परितोषयेत्}
{मधुरान्नं हविष्यं च स्वयं भुञ्जीत वाग्यतः}%॥९९॥

\twolineshloka
{स्वर्णरूपं गणेशस्य दातव्यं द्विजसत्तम}
{शक्त्या च दक्षिणां दद्याद् वित्तशाठ्यं न कारयेत्}%॥१००॥

\twolineshloka
{एवं श्रुत्वा च तैः सर्वैर्गीष्पतिः प्रेषितस्तदा}
{स गत्वा कथयामास चन्द्राय ब्रह्मणोदितम्}%॥१०१॥

\twolineshloka
{व्रतं चक्रे ततश्चन्द्रो यथोक्तं ब्रह्मणा पुरा}
{आविर्बभूव भगवान् गणेशो व्रततोषितः}%॥१०२॥

\fourlineindentedshloka
{तं क्रीडमानं गणनायकं च}
{तुष्टाव दृष्ट्वा तु कलानिधानः}
{त्वं कारणं कारणकारणानां}
{वेत्तासि वेद्यं च विभो प्रसीद}%॥१०३॥

\fourlineindentedshloka
{प्रसीद देवेश जगन्निवास}
{गणेश लम्बोदर वक्रतुण्ड}
{विरिञ्चिनारायणपूज्यमान}
{क्षमस्व मे गर्वकृतं च हास्यम्}%॥१०४॥

\fourlineindentedshloka
{ये त्वामसम्पूज्य गणेश नूनं}
{वाच्छन्ति मूढाः स्वकृतार्थसिद्धिम्}
{ते दैवनष्टा निभृतं च लोके}
{ज्ञातो मया ते सकलः प्रभावः}%॥१०५॥

\fourlineindentedshloka
{ये चाप्युदासीनतरास्तु पापाः}
{ते यान्ति वासं नरके सदैव}
{हेरम्ब लम्बोदर मे क्षमस्व}
{दुश्चेष्टितं तत् करुणासमुद्र}%॥१०६॥

\twolineshloka
{एवं संस्तूयमानोऽसौ चन्द्रेणाह गजाननः}
{तुष्टोऽहं तव दास्यामि वरं ब्रूहि निशाकर}%॥१०७॥

\uvacha{चन्द्र उवाच}

\twolineshloka
{लोकानां दर्शनीयोऽहं भवामि पुनरेव हि}
{विशापोऽहं भविष्यामि त्वत्प्रसादाद् गणेश्वर}%॥१०८॥

\uvacha{गणेश उवाच}

\twolineshloka
{वरमन्यं प्रदास्यामि नैतद् देयं मया तव}
{ततो ब्रह्मादयः सर्वे समाजग्मुर्भयार्दिताः}%॥१०९॥

\twolineshloka
{विशापं कुरु देवेश प्रार्थयामो वयं तव}
{विशापमकरोच्चन्द्रं कमलासनगौरवात्}%॥११०॥

\twolineshloka
{भाद्रशुक्लचतुर्थ्यां तु ये पश्यन्ति सदैव हि}
{मिथ्यापवादमावर्षं प्राप्स्यन्तीह न संशयः}%॥१११॥

\twolineshloka
{मासादौ पूर्वमेव त्वां ये पश्यन्ति सदा जनाः}
{भद्रायां शुक्लपक्षस्य तेषां दोषो न जायते}%॥११२॥

\twolineshloka
{तदाप्रभृति लोकोऽयं द्वितीयायां कृतादरः}
{पुनरेव तु पप्रच्छ कलावान् गणनायकम्}%॥११३॥

{केनोपायेन देवेश तुष्टो भवसि तद्वद।}

\uvacha{गणेश उवाच}
\onelineshloka{यश्च कृष्णचतुर्थ्यां तु मोदकाद्यैः प्रपूज्य माम्}%॥११४॥

\twolineshloka
{रोहिण्या सहितं त्वां च समभ्यर्च्यार्घ्यदानतः}
{यथाशक्त्या च मद्रूपं स्वर्णेन परिकल्पितम्}%॥११५॥

\twolineshloka
{दत्त्वा द्विजाय भुञ्जीत कथां श्रुत्वा विधानतः}
{सदा तस्य करिष्यामि सङ्कष्टस्य निवारणम्}%॥११६॥

\twolineshloka
{भाद्रशुक्लचतुर्थ्यां तु मृन्मयी प्रतिमा शुभा}
{हेमाभावे तु कर्तव्या नानापुष्पैः प्रपूज्य माम्}%॥११७॥

\twolineshloka
{ब्राह्मणान् भोजयेत् पश्चाज्जागरं च विशेषतः}
{स्थापयेदव्रणं कुम्भं धान्यस्योपरि शोभितम्}%॥११८॥

\twolineshloka
{यथाशक्त्या च मद्रूपं शातकुम्भेन निर्मितम्}
{वस्त्रद्वयसमाच्छन्नं मोदकाद्यैः प्रपूज्य माम्}%॥११९॥

\twolineshloka
{रक्ताम्बरधरो मर्त्यो ब्रह्मचर्यव्रतः शुचिः}
{रोहिणीसहितं त्वां च पूजयेत् स्थाप्य मत्पुरः}%॥१२०॥

\twolineshloka
{रजतस्य तु रूपं ते कृत्वा शक्त्या विनिर्मितम्}
{वस्त्रं शिवप्रियायेति उपवस्त्रं गणाधिपे}%॥१२१॥

\twolineshloka
{गन्धं लम्बोदरायेति पुष्पं सिद्धिप्रदायके}
{धूपं गजमुखायेति दीपं मूषकवाहने}%॥१२२॥

\twolineshloka
{विघ्ननाथाय नैवेद्यं फलं सर्वार्थसिद्धिदे}
{ताम्बूलं कामरूपाय दक्षिणां धनदाय च}%॥१२३॥

\twolineshloka
{इक्षुदण्डैर्मोदकैश्च होमं कुर्याच्च नामभिः}
{विसर्जनं ततः कुर्यात् सर्वसिद्धिप्रदायकम्}%॥१२४॥

\twolineshloka
{एवं सम्पूज्य विघ्नेशं कथां श्रुत्वा विधानतः}
{मन्त्रेणानेन तत् सर्वं ब्राह्मणाय निवेदयेत्}%॥१२५॥

\twolineshloka
{दानेनानेन देवेश प्रीतो भव गणेश्वर}
{सर्वत्र सर्वदा देव निर्विघ्नं कुरु सर्वदा}%॥१२६॥

\twolineshloka
{मानोन्नतिं च राज्यं च पुत्रपौत्रान् प्रदेहि मे}
{गाश्च धान्यं च वासांसि दद्यात् सर्वं स्वशक्तितः}%॥१२७॥

\twolineshloka
{दत्त्वा तु ब्राह्मणे सर्वं स्वयं भुञ्जीत वाग्यतः}
{मोदकापूपमधुरं लवणक्षारवर्जितम्}%॥१२८॥

\twolineshloka
{एवं करोति यश्चन्द्र तस्याहं सर्वदा जयम्}
{सिद्धिं च धनधान्ये च दादामि विपुलां प्रजाम्}%॥१२९॥

\twolineshloka
{इत्युक्त्वान्तर्दधे देवो विघ्नराजो विनायकः}
{तद् व्रतं कुरु कृष्ण त्वं ततः सिद्धिमवाप्स्यसि}%॥१३०॥

\twolineshloka
{नारदेनैवमुक्तस्तु व्रतं चक्रे हरिः स्वयम्}
{मिथ्यापवादं निर्मृज्य ततः कृष्णोऽभवच्छुचिः}%॥१३१॥

\twolineshloka
{ये शृण्वन्ति तवाख्यानं स्यमन्तकमणीयकम्}
{चन्द्रस्य चरितं सर्वं तेषां दोषो न जायते}%॥१३२॥

\twolineshloka
{भाद्रशुक्लचतुर्थ्यां तु क्वचिच्चन्द्रस्य दर्शनम्}
{जातं तत्परिहारार्थं श्रोतव्यं सर्वमेव हि}%॥१३३॥

\threelineshloka
{यदा यदा मनःकष्टं सन्देह उपजायते} 
{तदा तदा च श्रोतव्यमाख्यानं कष्टनाशनम्}
{एवमुक्त्वा गतो देवो गणेशः कृष्णतोषितः}%॥१३४॥

\fourlineindentedshloka
{यदा यदा पश्यति कार्यमुत्थितं}
{नारी नरश्चाथ करोति तद् व्रतम्}
{सिध्यन्ति कार्याणि मनेप्सितानि}
{किं दुर्लभं विघ्नहरे प्रसन्ने}%॥१३५॥

॥इति श्रीस्कन्दपुराणे नन्दिकेश्वरसनत्कुमारसंवादे स्यमन्तकोपाख्यानं सम्पूर्णम्॥

