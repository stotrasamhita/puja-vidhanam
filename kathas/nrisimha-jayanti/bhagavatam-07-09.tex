\sect{श्रीमद्भागवते महापुराणे सप्तमस्कन्धे नवमोऽध्यायः}

\uvacha{श्रीनारद उवाच}

\twolineshloka
{एवं सुरादयः सर्वे ब्रह्मरुद्रपुरः सराः}
{नोपैतुमशकन्मन्यु संरम्भं सुदुरासदम्} %1

\twolineshloka
{साक्षात्श्रीः प्रेषिता देवैर्दृष्ट्वा तं महदद्भुतम्}
{अदृष्टाश्रुतपूर्वत्वात्सा नोपेयाय शङ्किता} %2

\twolineshloka
{प्रह्रादं प्रेषयामास ब्रह्मावस्थितमन्तिके}
{तात प्रशमयोपेहि स्वपित्रे कुपितं प्रभुम्} %3

\twolineshloka
{तथेति शनकै राजन्महाभागवतोऽर्भकः}
{उपेत्य भुवि कायेन ननाम विधृताञ्जलिः} %4

\twolineshloka
{स्वपादमूले पतितं तमर्भकं विलोक्य देवः कृपया परिप्लुतः}
{उत्थाप्य तच्छीर्ष्ण्यदधात्कराम्बुजं कालाहिवित्रस्तधियां कृताभयम्} %5

\twolineshloka
{स तत्करस्पर्शधुताखिलाशुभः सपद्यभिव्यक्तपरात्मदर्शनः}
{तत्पादपद्मं हृदि निर्वृतो दधौ हृष्यत्तनुः क्लिन्नहृदश्रुलोचनः} %6

\twolineshloka
{अस्तौषीद्धरिमेकाग्र मनसा सुसमाहितः}
{प्रेमगद्गदया वाचा तन्न्यस्तहृदयेक्षणः} %7

\uvacha{श्रीप्रह्राद उवाच}


\fourlineindentedshloka
{ब्रह्मादयः सुरगणा मुनयोऽथ सिद्धाः}
{सत्त्वैकतानगतयो वचसां प्रवाहैः}
{नाराधितुं पुरुगुणैरधुनापि पिप्रुः}
{किं तोष्टुमर्हति स मे हरिरुग्रजातेः} %8

\fourlineindentedshloka
{मन्ये धनाभिजनरूपतपःश्रुतौजस्-}
{तेजःप्रभावबलपौरुषबुद्धियोगाः}
{नाराधनाय हि भवन्ति परस्य पुंसो}
{भक्त्या तुतोष भगवान्गजयूथपाय} %9

\fourlineindentedshloka
{विप्राद्द्विषड्गुणयुतादरविन्दनाभ}
{पादारविन्दविमुखात्श्वपचं वरिष्ठम्}
{मन्ये तदर्पितमनोवचनेहितार्थ}
{प्राणं पुनाति स कुलं न तु भूरिमानः} %10

\fourlineindentedshloka
{नैवात्मनः प्रभुरयं निजलाभपूर्णो}
{मानं जनादविदुषः करुणो वृणीते}
{यद्यज्जनो भगवते विदधीत मानं}
{तच्चात्मने प्रतिमुखस्य यथा मुखश्रीः} %11

\fourlineindentedshloka
{तस्मादहं विगतविक्लव ईश्वरस्य}
{सर्वात्मना महि गृणामि यथा मनीषम्}
{नीचोऽजया गुणविसर्गमनुप्रविष्टः}
{पूयेत येन हि पुमाननुवर्णितेन} %12

\fourlineindentedshloka
{सर्वे ह्यमी विधिकरास्तव सत्त्वधाम्नो}
{ब्रह्मादयो वयमिवेश न चोद्विजन्तः}
{क्षेमाय भूतय उतात्मसुखाय चास्य}
{विक्रीडितं भगवतो रुचिरावतारैः} %13

\fourlineindentedshloka
{तद्यच्छ मन्युमसुरश्च हतस्त्वयाद्य}
{मोदेत साधुरपि वृश्चिकसर्पहत्या}
{लोकाश्च निर्वृतिमिताः प्रतियन्ति सर्वे}
{रूपं नृसिंह विभयाय जनाः स्मरन्ति} %14

\fourlineindentedshloka
{नाहं बिभेम्यजित तेऽतिभयानकास्य}
{जिह्वार्कनेत्रभ्रुकुटीरभसोग्रदंष्ट्रात्}
{आन्त्रस्रजःक्षतजकेशरशङ्कुकर्णान्}
{निर्ह्रादभीतदिगिभादरिभिन्नखाग्रात्} %15

\fourlineindentedshloka
{त्रस्तोऽस्म्यहं कृपणवत्सल दुःसहोग्र}
{संसारचक्रकदनाद्ग्रसतां प्रणीतः}
{बद्धः स्वकर्मभिरुशत्तम तेऽङ्घ्रिमूलं}
{प्रीतोऽपवर्गशरणं ह्वयसे कदा नु} %16

\fourlineindentedshloka
{यस्मात्प्रियाप्रियवियोगसंयोगजन्म}
{शोकाग्निना सकलयोनिषु दह्यमानः}
{दुःखौषधं तदपि दुःखमतद्धियाहं}
{भूमन्भ्रमामि वद मे तव दास्ययोगम्} %17

\fourlineindentedshloka
{सोऽहं प्रियस्य सुहृदः परदेवताया}
{लीलाकथास्तव नृसिंह विरिञ्चगीताः}
{अञ्जस्तितर्म्यनुगृणन्गुणविप्रमुक्तो}
{दुर्गाणि ते पदयुगालयहंससङ्गः} %18

\fourlineindentedshloka
{बालस्य नेह शरणं पितरौ नृसिंह}
{नार्तस्य चागदमुदन्वति मज्जतो नौः}
{तप्तस्य तत्प्रतिविधिर्य इहाञ्जसेष्टस्-}
{तावद्विभो तनुभृतां त्वदुपेक्षितानाम्} %19

\fourlineindentedshloka
{यस्मिन्यतो यर्हि येन च यस्य यस्माद्}
{यस्मै यथा यदुत यस्त्वपरः परो वा}
{भावः करोति विकरोति पृथक्स्वभावः}
{सञ्चोदितस्तदखिलं भवतः स्वरूपम्} %20

\fourlineindentedshloka
{माया मनः सृजति कर्ममयं बलीयः}
{कालेन चोदितगुणानुमतेन पुंसः}
{छन्दोमयं यदजयार्पितषोडशारं}
{संसारचक्रमज कोऽतितरेत्त्वदन्यः} %21

\fourlineindentedshloka
{स त्वं हि नित्यविजितात्मगुणः स्वधाम्ना}
{कालो वशीकृतविसृज्यविसर्गशक्तिः}
{चक्रे विसृष्टमजयेश्वर षोडशारे}
{निष्पीड्यमानमुपकर्ष विभो प्रपन्नम्} %22

\fourlineindentedshloka
{दृष्टा मया दिवि विभोऽखिलधिष्ण्यपानाम्}
{आयुः श्रियो विभव इच्छति यान्जनोऽयम्}
{येऽस्मत्पितुः कुपितहासविजृम्भितभ्रू}
{विस्फूर्जितेन लुलिताः स तु ते निरस्तः} %23

\fourlineindentedshloka
{तस्मादमूस्तनुभृतामहमाशिषोऽज्ञ}
{आयुः श्रियं विभवमैन्द्रियमाविरिञ्च्यात्}
{नेच्छामि ते विलुलितानुरुविक्रमेण}
{कालात्मनोपनय मां निजभृत्यपार्श्वम्} %24

\fourlineindentedshloka
{कुत्राशिषः श्रुतिसुखा मृगतृष्णिरूपाः}
{क्वेदं कलेवरमशेषरुजां विरोहः}
{निर्विद्यते न तु जनो यदपीति विद्वान्}
{कामानलं मधुलवैः शमयन्दुरापैः} %25

\fourlineindentedshloka
{क्वाहं रजःप्रभव ईश तमोऽधिकेऽस्मिन्}
{जातः सुरेतरकुले क्व तवानुकम्पा}
{न ब्रह्मणो न तु भवस्य न वै रमाया}
{यन्मेऽर्पितः शिरसि पद्मकरः प्रसादः} %26

\fourlineindentedshloka
{नैषा परावरमतिर्भवतो ननु स्याज्}
{जन्तोर्यथात्मसुहृदो जगतस्तथापि}
{संसेवया सुरतरोरिव ते प्रसादः}
{सेवानुरूपमुदयो न परावरत्वम्} %27

\fourlineindentedshloka
{एवं जनं निपतितं प्रभवाहिकूपे}
{कामाभिकाममनु यः प्रपतन्प्रसङ्गात्}
{कृत्वात्मसात्सुरर्षिणा भगवन्गृहीतः}
{सोऽहं कथं नु विसृजे तव भृत्यसेवाम्} %28

\fourlineindentedshloka
{मत्प्राणरक्षणमनन्त पितुर्वधश्च}
{मन्ये स्वभृत्यऋषिवाक्यमृतं विधातुम्}
{खड्गं प्रगृह्य यदवोचदसद्विधित्सुस्-}
{त्वामीश्वरो मदपरोऽवतु कं हरामि} %29

\fourlineindentedshloka
{एकस्त्वमेव जगदेतममुष्य यत्त्वम्}
{आद्यन्तयोः पृथगवस्यसि मध्यतश्च}
{सृष्ट्वा गुणव्यतिकरं निजमाययेदं}
{नानेव तैरवसितस्तदनुप्रविष्टः} %30

\fourlineindentedshloka
{त्वम्वा इदं सदसदीश भवांस्ततोऽन्यो}
{माया यदात्मपरबुद्धिरियं ह्यपार्था}
{यद्यस्य जन्म निधनं स्थितिरीक्षणं च}
{तद्वैतदेव वसुकालवदष्टितर्वोः} %31

\fourlineindentedshloka
{न्यस्येदमात्मनि जगद्विलयाम्बुमध्ये}
{शेषेऽऽत्मना निजसुखानुभवो निरीहः}
{योगेन मीलितदृगात्मनिपीतनिद्रस्-}
{तुर्ये स्थितो न तु तमो न गुणांश्च युङ्क्षे} %32

\fourlineindentedshloka
{तस्यैव ते वपुरिदं निजकालशक्त्या}
{सञ्चोदितप्रकृतिधर्मण आत्मगूढम्}
{अम्भस्यनन्तशयनाद्विरमत्समाधेर्-}
{नाभेरभूत्स्वकणिकावटवन्महाब्जम्} %33

\fourlineindentedshloka
{तत्सम्भवः कविरतोऽन्यदपश्यमानस्-}
{त्वां बीजमात्मनि ततं स बहिर्विचिन्त्य}
{नाविन्ददब्दशतमप्सु निमज्जमानो}
{जातेऽङ्कुरे कथमुहोपलभेत बीजम्} %34

\fourlineindentedshloka
{स त्वात्मयोनिरतिविस्मित आश्रितोऽब्जं}
{कालेन तीव्रतपसा परिशुद्धभावः}
{त्वामात्मनीश भुवि गन्धमिवातिसूक्ष्मं}
{भूतेन्द्रियाशयमये विततं ददर्श} %35

\fourlineindentedshloka
{एवं सहस्रवदनाङ्घ्रिशिरःकरोरु}
{नासाद्यकर्णनयनाभरणायुधाढ्यम्}
{मायामयं सदुपलक्षितसन्निवेशं}
{दृष्ट्वा महापुरुषमाप मुदं विरिञ्चः} %36

\fourlineindentedshloka
{तस्मै भवान्हयशिरस्तनुवं हि बिभ्रद्}
{वेदद्रुहावतिबलौ मधुकैटभाख्यौ}
{हत्वानयच्छ्रुतिगणांश्च रजस्तमश्च}
{सत्त्वं तव प्रियतमां तनुमामनन्ति} %37

\fourlineindentedshloka
{इत्थं नृतिर्यगृषिदेवझषावतारैर्}
{लोकान्विभावयसि हंसि जगत्प्रतीपान्}
{धर्मं महापुरुष पासि युगानुवृत्तं}
{छन्नः कलौ यदभवस्त्रियुगोऽथ स त्वम्} %38

\fourlineindentedshloka
{नैतन्मनस्तव कथासु विकुण्ठनाथ}
{सम्प्रीयते दुरितदुष्टमसाधु तीव्रम्}
{कामातुरं हर्षशोकभयैषणार्तं}
{तस्मिन्कथं तव गतिं विमृशामि दीनः} %39

\fourlineindentedshloka
{जिह्वैकतोऽच्युत विकर्षति मावितृप्ता}
{शिश्नोऽन्यतस्त्वगुदरं श्रवणं कुतश्चित्}
{घ्राणोऽन्यतश्चपलदृक्क्व च कर्मशक्तिर्}
{बह्व्यः सपत्न्य इव गेहपतिं लुनन्ति} %40

\fourlineindentedshloka
{एवं स्वकर्मपतितं भववैतरण्याम्}
{अन्योन्यजन्ममरणाशनभीतभीतम्}
{पश्यन्जनं स्वपरविग्रहवैरमैत्रं}
{हन्तेति पारचर पीपृहि मूढमद्य} %41

\fourlineindentedshloka
{को न्वत्र तेऽखिलगुरो भगवन्प्रयास}
{उत्तारणेऽस्य भवसम्भवलोपहेतोः}
{मूढेषु वै महदनुग्रह आर्तबन्धो}
{किं तेन ते प्रियजनाननुसेवतां नः} %42

\fourlineindentedshloka
{नैवोद्विजे पर दुरत्ययवैतरण्यास्-}
{त्वद्वीर्यगायनमहामृतमग्नचित्तः}
{शोचे ततो विमुखचेतस इन्द्रियार्थ}
{मायासुखाय भरमुद्वहतो विमूढान्} %43

\fourlineindentedshloka
{प्रायेण देव मुनयः स्वविमुक्तिकामा}
{मौनं चरन्ति विजने न परार्थनिष्ठाः}
{नैतान्विहाय कृपणान्विमुमुक्ष एको}
{नान्यं त्वदस्य शरणं भ्रमतोऽनुपश्ये} %44

\fourlineindentedshloka
{यन्मैथुनादिगृहमेधिसुखं हि तुच्छं}
{कण्डूयनेन करयोरिव दुःखदुःखम्}
{तृप्यन्ति नेह कृपणा बहुदुःखभाजः}
{कण्डूतिवन्मनसिजं विषहेत धीरः} %45

\fourlineindentedshloka
{मौनव्रतश्रुततपोऽध्ययनस्वधर्म}
{व्याख्यारहोजपसमाधय आपवर्ग्याः}
{प्रायः परं पुरुष ते त्वजितेन्द्रियाणां}
{वार्ता भवन्त्युत न वात्र तु दाम्भिकानाम्} %46

\fourlineindentedshloka
{रूपे इमे सदसती तव वेदसृष्टे}
{बीजाङ्कुराविव न चान्यदरूपकस्य}
{युक्ताः समक्षमुभयत्र विचक्षन्ते त्वां}
{योगेन वह्निमिव दारुषु नान्यतः स्यात्} %47

\fourlineindentedshloka
{त्वं वायुरग्निरवनिर्वियदम्बु मात्राः}
{प्राणेन्द्रियाणि हृदयं चिदनुग्रहश्च}
{सर्वं त्वमेव सगुणो विगुणश्च भूमन्}
{नान्यत्त्वदस्त्यपि मनोवचसा निरुक्तम्} %48

\fourlineindentedshloka
{नैते गुणा न गुणिनो महदादयो ये}
{सर्वे मनः प्रभृतयः सहदेवमर्त्याः}
{आद्यन्तवन्त उरुगाय विदन्ति हि त्वाम्}
{एवं विमृश्य सुधियो विरमन्ति शब्दात्} %49

\fourlineindentedshloka
{तत्तेऽर्हत्तम नमः स्तुतिकर्मपूजाः}
{कर्म स्मृतिश्चरणयोः श्रवणं कथायाम्}
{संसेवया त्वयि विनेति षडङ्गया किं}
{भक्तिं जनः परमहंसगतौ लभेत} %50

\uvacha{श्रीनारद उवाच}


\twolineshloka
{एतावद्वर्णितगुणो भक्त्या भक्तेन निर्गुणः}
{प्रह्रादं प्रणतं प्रीतो यतमन्युरभाषत} %51

\uvacha{श्रीभगवानुवाच}


\twolineshloka
{प्रह्राद भद्र भद्रं ते प्रीतोऽहं तेऽसुरोत्तम}
{वरं वृणीष्वाभिमतं कामपूरोऽस्म्यहं नृणाम्} %52

\twolineshloka
{मामप्रीणत आयुष्मन्दर्शनं दुर्लभं हि मे}
{दृष्ट्वा मां न पुनर्जन्तुरात्मानं तप्तुमर्हति} %53

\twolineshloka
{प्रीणन्ति ह्यथ मां धीराः सर्वभावेन साधवः}
{श्रेयस्कामा महाभाग सर्वासामाशिषां पतिम्} %54

\uvacha{श्रीनारद उवाच}

\twolineshloka
{एवं प्रलोभ्यमानोऽपि वरैर्लोकप्रलोभनैः}
{एकान्तित्वाद्भगवति नैच्छत्तानसुरोत्तमः} %॥५५॥\\

॥इति श्रीमद्भागवते महापुराणे पारमहंस्यां संहितायां सप्तमस्कन्धे नवमोऽध्यायः॥

