\sect{कथा}

\uvacha{सूत उवाच}

\twolineshloka
{कैलासशिखरासीनं देवदेवं जगद्गुरुम्}
{पञ्चवक्त्रं दशभुजं त्रिनेत्रं शूलपाणिनम्}%॥१॥

\twolineshloka
{पिनाकशोभितकरं खड्गखेटकधारिणम्}
{कपालखट्वाङ्गधरं नीलकण्ठसुशोभितम्}%॥२॥

\twolineshloka
{भस्माङ्गं व्यालशोभाढ्यमस्थिमालाविभूषितम्}
{नीलजीमूतसङ्काशं सूर्यकोटिसमप्रभम्}%॥३॥

\twolineshloka
{क्रीडन्तं च शिवं तत्र गणैश्च परिवारितम्}
{विसृज्य देवताः सर्वास्तिष्ठन्तं परमेश्वरम्}%॥४॥

\twolineshloka
{तं दृष्ट्वा देवदेवेशं प्रहस्योत्फुल्ललोचनम्}
{पार्वती परिपप्रच्छ विनयावनता स्थिता}%॥५॥

\uvacha{पार्वत्युवाच}

\twolineshloka
{कथयस्व प्रसादेन यद्गोप्यं व्रतमुत्तमम्}
{श्रुतास्त्वयोक्ता देवेश व्रतानां निर्णयाः शुभाः}%॥६॥

\twolineshloka
{तथा वै दानधर्माश्च तीर्थधर्मास्त्वयोदिताः}
{नास्ति मे निश्चयो देव भ्रान्ताऽहं च पुनः पुनः}%॥७॥

\twolineshloka
{तस्माद्वदस्व मे देव ह्येकं निःसंशयं व्रतम्}
{व्रतानामुत्तमं देव भुक्तिमुक्तिप्रदायकम्}%॥८॥

\onelineshloka*
{तदहं श्रोतुमिच्छामि कथयस्व मम प्रभो}


\uvacha{ईश्वर उवाच}
\onelineshloka
{शृणु देवि प्रवक्ष्यामि\footnotemark{} व्रतानामुत्तमं व्रतम्}%॥९॥
\footnotetext{परं गुह्यम् इत्यपि पाठः}

\twolineshloka
{यन्न कस्यचिदाख्यातं रहस्यं मुक्तिदायकम्}
{येनैव कथ्यमानेन यमोऽपि विलयं व्रजेत्}%॥१०॥

\twolineshloka
{तदहं कथयिष्यामि शृणुष्वैकमनाः प्रिये}
{माघमासे कृष्णपक्षे अमायुक्ता चतुर्दशी\footnotemark{}}%॥११॥
\footnotetext{माघान्ते कृष्णपक्षे तु अविद्धा या चतुर्दशी। इत्यपि पाठः}

\twolineshloka
{शिवरात्रिस्तु सा ज्ञेया सर्वयज्ञोत्तमोत्तमा}
{दानैर्यज्ञैस्तपोभिश्च व्रतैश्च विविधैरपि}%॥१२॥

\twolineshloka
{न तीर्थैस्तद्भवेत्पुण्यं यत्पुण्यं शिवरात्रितः}
{शिवरात्रिसमं नास्ति व्रतानामुत्तमं व्रतम्}%॥१३॥

\twolineshloka
{ज्ञानतोऽज्ञानतो वाऽपि कृत्वा मोक्षमवाप्नुयात्}
{मृतास्ते निरयं यान्ति यैरेषा न कृता क्वचित्}%॥१४॥

\twolineshloka
{कृता यैर्निरयं त्यक्त्वा गतास्ते शिवसन्निधौ}
{सर्वमङ्गलशीला च सर्वामङ्गलनाशिनी}%॥१५॥

\onelineshloka*
{भुक्तिमुक्तिप्रदा चैषा सत्यं सत्यं वरानने}


\uvacha{देव्युवाच}
\onelineshloka
{कथं यमपुरं त्यक्त्वा शिवलोके व्रजेन्नरः}%॥१६॥

\onelineshloka*
{एतन्मे महदाश्चर्यं प्रत्यक्षं कुरु शङ्कर}


\uvacha{शङ्कर उवाच}
\onelineshloka
{शृणु देवि यथावृत्तां कथां पौराणिकीं शुभाम्}%॥१७॥

\twolineshloka
{यमशासनहन्त्रीं च शिवस्थानप्रदायिनीम्}
{कश्चिदासीत्पुरा देवि निषादो जीवघातकः\footnotemark{}}%॥१८॥
\footnotetext{निषादस्त्वामिषप्रियः। इत्यपि पाठः}

\twolineshloka
{प्रत्यन्तदेशवासी च भूधरासन्नकेतनः}
{सीमान्ते स सदा तिष्ठन्कुटुम्बपरिपालकः}%॥१९॥

\twolineshloka
{तन्वा पीनो धनुर्धारी श्यामाङ्गः कृष्णकञ्चुकः}
{बद्धगोधाङ्गुलित्राणः सदैव मृगयारतः}%॥२०॥

\twolineshloka
{एवंविधो निषादोऽसौ चतुर्दश्या दिने शुभे}
{व्यवहारिकैश्च द्रव्यार्थं देवागारे प्ररोधितः}%॥२१॥

\twolineshloka
{तेनापि देवता दृष्टा जनानां वचनं श्रुतम्}
{उपवासव्रतीनां च जल्पतां शिवशिवेति च}%॥२२॥

\twolineshloka
{दिनान्ते तैस्तदा मुक्तः प्रातर्द्रव्यं प्रदीयताम्}
{ततोऽसौ धनुरादाय दक्षिणेन गतः स्वयम्}%॥२३॥

\twolineshloka
{आगच्छन्स वनोद्देशे जनहासं चकार सः}
{शिवशिव किमेतद्वै कुर्वन्ति नगरे जनाः}%॥२४॥

\twolineshloka
{वनेचरान्निरीक्षंस्तु चतुर्दिक्षु इतस्ततः}
{पदं च पदमार्गं च अन्विष्यन्सूकरान्मृगान्}%॥२५॥

\twolineshloka
{इतश्चेतश्च धावन्वै आमिषे लुब्धमानसः}
{वनं च पर्वतान्सर्वान्भ्रमित्वा गिरिकन्दराः}%॥२६॥

\twolineshloka
{सम्प्राप्तं तेन नो किञ्चिन्मृगसूकरचित्तलम्\footnotemark{}}
{निराशो लुब्धको यावत्तावदस्तं गतो रविः}%॥२७॥
\footnotetext{तित्तिरमित्यपि पाठः}

\twolineshloka
{चिन्तयित्वा जलोपान्ते जागरं\footnotemark{} जीवघातनम्}
{संविधास्याम्यहं रात्रौ निश्चितं मम जीवनम्}%॥२८॥
\footnotetext{जागर चिन्तायित्वेन्वयः}

\twolineshloka
{तडागसन्निधौ गत्वा तत्तीरे जालिमध्यतः}
{आश्रमं कर्तुमारेभे आत्मनो गुप्तिकारणात्}%॥२९॥

\twolineshloka
{जालिमध्ये महालिङ्गं स्थितं स्वायम्भुवं शुभम्}
{बिल्ववृक्षो महान्दिव्यो जालिमध्ये च संस्थितः}%॥३०॥

\twolineshloka
{गृहीत्वा तस्य पर्णानि मार्गशुद्ध्यर्थमक्षिपत्}
{क्षिप्तानि दक्षिणे भागे निपेतुर्लिङ्गमूर्धनि}%॥३१॥

\twolineshloka
{तस्य गन्धं समासाद्य लुब्धकस्य वरानने}
{न तिष्ठन्ति मृगाः सर्वे शरघातभयात्तदा}%॥३२॥

\twolineshloka
{न दिवा भोजनं जातं संरोधस्य प्रभावतः}
{मृगान्निरीक्षतो रात्रौ निद्रानाशोऽप्यजायत}%॥३३॥

\twolineshloka
{जालिमध्ये गतस्यास्य प्रथमः प्रहरो गतः}
{ततो जलार्थमायाता हरिणी गर्भसंयुता}%॥३४॥

\twolineshloka
{यौवनस्था सुरूपा च स्तनपीना सुशोभना}
{निरीक्षन्ती दिशः सर्वा भृशमुत्फुल्ललोचना}%॥३५॥

\twolineshloka
{लुब्धकेनापि सा दृष्टा बाणगोचरमागता}
{कृतं च बाणसन्धानं तेनैकाग्रेण चेतसा}%॥३६॥

\twolineshloka
{त्रोटयित्वाऽथ पत्राणि प्रक्षिप्तानि शिवोपरि}
{शिवेति संस्मरन्वादं शीतेन परिपीडितः}%॥३७॥

\twolineshloka
{एतस्मिन्नन्तरे दृष्टो हरिण्या लुब्धकस्तदा}
{लुब्धकस्तु स्वरूपेण कृतान्त इव तिष्ठति}%॥३८॥

\twolineshloka
{दृष्ट्वा तु तस्य सन्धानं यमदंष्ट्रासमप्रभम्}
{मृगी सा दिव्यया वाचा लुब्धकं वाक्यमब्रवीत्}%॥३९॥

\uvacha{मृग्युवाच}

\twolineshloka
{स्थिरो भव महाव्याध सर्वजीवनिकृन्तन}
{कथयस्व महाबाहो किमर्थं मां हनिष्यसि}%॥४०॥

\uvacha{शिव उवाच}

\onelineshloka*
{तस्यास्तद्वचनं श्रुत्वा लुब्धकः प्राह तां मृगीम्}

\uvacha{लुब्धक उवाच}
\onelineshloka
{समातृकं कुटुम्बं मे क्षुधया पीड्यते भृशम्}%॥४१॥

\onelineshloka*
{धनं वै मद्गृहे नास्ति तेन त्वां हन्मि शोभने}


\uvacha{सूत उवाच}
\onelineshloka
{याम\footnotemark{}पूजाप्रभावेण जागरोपोषणेन च}%॥४२॥
\footnotetext{जातेत्यपि पाठः}

\twolineshloka
{चतुर्थांशेन पापानां विमुक्तो लुब्धकस्तदा}
{लुब्धकस्तु ततो दृष्ट्वा मृगी मानुषभाषिणीम्}%॥४३॥

\onelineshloka*
{उवाच वचनं तां वै धर्मयुक्तमसंशयम्}

\uvacha{लुब्धक उवाच}
\onelineshloka
{मया हि घातिता जीवा उत्तमाधममध्यमाः}%॥४४॥

\twolineshloka
{न श्रुता ईदृशी वाणी श्वापदानां कथञ्चन}
{कस्मिन् देशे त्वमुत्पन्ना कस्मात्स्थानादिहागता}%॥४५॥

\onelineshloka*
{कथयस्व प्रयत्नेन परं कौतहलं हि मे}

\uvacha{मृग्युवाच}
\onelineshloka
{शृणु त्वं लुब्धकश्रेष्ठ कथयामि तवाखिलम्}%॥४६॥

\twolineshloka
{आसं पूर्वमहं रम्भा स्वर्गे शक्रस्य चाप्सराः}
{अनन्तरूपलावण्या सौभाग्येन च गर्विता}%॥४७॥

\twolineshloka
{सौभाग्यमदपुष्टाङ्गो दानवो बलगर्वितः}
{मयैव च वृतो भर्ता हिरण्याक्षो महासुरः}%॥४८॥

\twolineshloka
{तेन सार्धं मया भुक्तं चिरकालं यथेप्सितम्}
{एवं कालो गतो व्याध क्रीडन्त्या मेऽसुरेण च}%॥४९॥

\twolineshloka
{एकदा प्रेक्षितुं नृत्यं शङ्करस्य गताग्रतः}
{यावद्गच्छाम्यहं तत्र तावन्मां शङ्करोऽब्रवीत्}%॥५०॥

\twolineshloka
{क्व गता त्वं वरारोहे केन वा सङ्गता शुभे}
{किं वा सौभाग्यगर्वेण नाऽऽयाता मम मन्दिरम्}%॥५१॥

\twolineshloka
{सत्यं कथय शीघ्रं त्वं नो वा शापं ददामि ते}
{शापभीत्या मया तत्र सत्यमुक्तं शिवाग्रतः}%॥५२॥

\twolineshloka
{शृणु देव प्रवक्ष्यामि शापानुग्रहकारक}
{ममास्ति भर्ता विश्वेश दानवेन्द्रो महाबलः}%॥५३॥

\twolineshloka
{तेन सार्धं मया देव क्रीडितं निजमन्दिरे}
{तेनाहं नाऽऽगमं शीघ्रं सृष्टिसंहारकारक}%॥५४॥

\twolineshloka
{रुद्रस्तद्वचनं श्रुत्वा सकोपो वाक्यमब्रवीत्}
{मृगः कामातुरो नित्यं हिरण्याक्षो भविष्यति}%॥५५॥

\twolineshloka
{त्वं मृगी तस्य भार्या वै भविष्यसि न संशयः}
{त्यक्त्वा\footnotemark{} स्वर्गं तथा देवान्दानवं भोक्तुमिच्छसि}%॥५६॥
\footnotetext{यत इति शेषः}

\twolineshloka
{तस्मात्त्वं निर्जले देशे तृणाहारा भविष्यसि}
{द्वादशाब्दानि भो भद्रे भविता शाप एष ते}%॥५७॥

\twolineshloka
{परस्परस्य शोकेन शापान्तोऽपि भविष्यति}
{अनुग्रहः पुनस्त्वेष शङ्करेण कृतः स्वयम्}%॥५८॥

\twolineshloka
{कदाचिद्धि व्याधवरो मम सान्निध्यमाश्रितः}
{बाणाग्रे तस्य सम्प्राप्ता पूर्वजन्म स्मरिष्यसि}%॥५९॥

\twolineshloka
{शङ्करस्य तदा रूपं दृष्ट्वा मोक्षमवाप्स्यसि}
{शङ्करो न मया दृष्टो वसन्त्यस्मिन्महावने}%॥६०॥

\twolineshloka
{तेन दुःखमनुप्राप्ता मांसमेदोविवर्जिता}
{गर्भाक्रान्ता विशेषेण न वध्या चेति निश्चितम्}%॥६१॥

\twolineshloka
{सकुटुम्बस्य ते नूनं भोजनं न भविष्यति}
{आयास्यति मृगी त्वन्या मार्गेणानेन लुब्धक}%॥६२॥

\twolineshloka
{पीना यौवनसम्पन्ना बहुमांसा मदोद्धता}
{भोजनं सकुटुम्बस्य तया सद्यो भविष्यति}%॥६३॥

\twolineshloka
{अथवाऽन्यो मृगो व्याध पानार्थं तु जलाशये\footnotemark{}}
{आगमिष्यति प्रत्यूषे क्षुधार्तस्य न संशयः}%॥६४॥
\footnotetext{तव बाणस्य गोचरे इत्यपि पाठः}

\twolineshloka
{गर्भं त्यक्त्वा पुनः प्रातर्बालान्सन्दिश्य बन्धुषु}
{शपथैरागमिष्यामि सन्दिश्य च सखीजनम्}%॥६५॥

\twolineshloka
{तस्यास्तद्वचनं श्रुत्वा व्याधो विस्मयमागतः}
{क्षणमेकं तथा स्थित्वा व्याधो वचनमब्रवीत्}%॥६६॥

\twolineshloka
{नाऽऽगमिष्यति चेदन्यो जीव\footnotemark{}स्त्वमपि गच्छसि}
{क्षुधया पीडितोऽहं वै कुटुम्बं च विशेषतः}%॥६७॥
\footnotetext{मृग इत्यपि पाठः}

\twolineshloka
{प्रातस्त्वया मम गृहमागन्तव्यं यथायथम्}
{शपथैश्च व्रज त्वं हि यथा मे प्रत्ययो भवेत्}%॥६८॥

\twolineshloka
{पृथिवी वायुरादित्यः सत्ये तिष्ठन्ति देवताः}
{पालनीयं ततः सत्यं लोकद्वयमभीप्सुभिः}%॥६९॥

\twolineshloka
{तस्मात्सत्येन गन्तव्यं भवत्या स्वगृहं प्रति}
{तस्य तद्वचनं श्रुत्वा गर्भार्ता सा मृगी तदा}%॥७०॥

\onelineshloka*
{चक्रे सत्यप्रतिज्ञां वै व्याधस्याग्रे पुनः पुनः}


\uvacha{मृग्युवाच}
\onelineshloka
{द्विजो भूत्वा तु यो व्याध वेदभ्रष्टोऽभिजायते}%॥७१॥

\twolineshloka
{स्वाध्यायसन्ध्यारहितः सत्यशौचविवर्जितः}
{अविक्रेयाणां विक्रेता अयाज्यानां च याजकः}%॥७२॥

\twolineshloka
{\footnotemark{}तस्य पापेन लिप्यामि यद्यहं नाऽऽगमं पुनः}
{दुष्टबुद्धौ तु यत्पापं धूर्ते वा ग्रामकण्टके}%॥७३॥
\footnotetext{तस्य यत्पापमिति शेषः। एवमेवाग्रेऽपि।}

\twolineshloka
{नास्तिके च विशीले च परदाररते तथा}
{वेदविक्रयणे चैव शवसूतकभोजने}%॥७४॥

\twolineshloka
{तेन पापेन लिप्यामि यद्यहं नाऽऽगमं पुनः}
{मृतशय्याप्रतिग्राहे माता पित्रोरपालके}%॥७५॥

\twolineshloka
{तेन पापेन लिप्यामि यदि नाऽऽयामि तेऽन्तिकम्}
{दानं दातुं प्रवृत्तस्य योऽन्तरायकरो नरः}%॥७६॥

\twolineshloka
{तेन पापेन लिप्यामि यदि नाऽऽयामि ते गृहम्}
{देवद्रव्यं गुरुद्रव्यं ब्रह्मद्रव्यं हरेत्तु यः}%॥७७॥

\twolineshloka
{तेन पापेन लिप्यामि यदि नाऽऽयामि ते गृहम्}
{दीपं दीपेन यः कुर्यात्पादं पादेन धावयेत्}%॥७८॥

\twolineshloka
{तेन पापेन लिप्यामि यदि नाऽऽयामि ते गृहम्}
{भर्तारं स्वामिनं मित्रमात्मानं बालमेव च}%॥७९॥

\twolineshloka
{गां विप्रं च गुरुं नारीं यो मारयति दुर्मतिः}
{तेन पापेन लिप्यामि यदि नाऽऽयामि ते गृहम्}%॥८०॥

\twolineshloka
{अवैष्णवेचयत्पापं यत्पापं दाम्भिके जने}
{अजितेन्द्रियेषु यत्पापं परदोषानुकीर्तने}%॥८१॥

\twolineshloka
{कृतघ्ने च कदर्ये च परदाररते तथा}
{सदाचारविहीने च परपीडाप्रदायके}%॥८२॥

\twolineshloka
{परपैशुन्ययुक्ते च कन्याविक्रयकारके}
{हैतुके बकवृत्तौ च कूटसाक्ष्यप्रदे तथा}%॥८३॥

\twolineshloka
{एतेषां पातकं मह्यं यदि नाऽऽयामि ते गृहम्}
{यत्पापं ब्रह्महत्यायां पितृमातृवधे तथा}%॥८४॥

\twolineshloka
{तेन पापेन लिप्यामि यदि नाऽऽयामि ते गृहम्}
{यत्पापं लुब्धकानां च यत्पापं गरदायिनाम्}%॥८५॥

\twolineshloka
{तेन पापेन लिप्यामि यदि नाऽऽयामि ते गृहम्}
{द्विभार्यः पुरुषो यस्तु समदृष्ट्या न पश्यति}%॥८६॥

\twolineshloka
{तस्य पापेन लिप्यामि यदि नाऽऽयामि ते गृहम्}
{सकृद्दत्त्वा तु यः कन्यां द्वितीयाय प्रयच्छति}%॥८७॥

\twolineshloka
{तेन पापेन लिप्यामि यदि नाऽऽयामि ते गृहम्}
{कथायां कथ्यमानायामन्तरं कुरुते नरः}%॥८८॥

\twolineshloka
{तस्य पापेन लिप्यामि यदि नाऽऽयामि ते गृहम्}
{पतिनिन्दापरो नित्यं वेदनिन्दापरो हि यः}%॥८९॥

\twolineshloka
{तस्य पापेन लिप्यामि यदि नाऽऽयामि ते गृहम्}
{यस्य सङ्ग्रहणी भार्या ब्राह्मणी च विशेषतः}%॥९०॥

\twolineshloka
{तस्य पापेन लिप्यामि यदि नाऽऽयामि ते गृहम्}
{प्रेतश्राद्धे तु यो भुङ्क्ते पतिते बहुयाजके}%॥९१॥

\twolineshloka
{असच्छास्त्रार्थनिपुणज्ञपुराणार्थविवर्जिते}
{मूर्खे पाखण्डनिरते क्रयविक्रयिके द्वये}%॥९२॥

\twolineshloka
{एतेषां पातकं मह्यं यदि नाऽऽयामि ते गृहम्}
{एकाकी मिष्टमश्नाति भार्यापुत्रविवर्जितः}%॥९३॥

\twolineshloka
{आत्मजां गुणसम्पन्नां समाने सदृशे वरे}
{न प्रयच्छति यः कन्यां नरो वै ज्ञानदुर्बलः}%॥९४॥

\twolineshloka
{तेन पापेन लिप्यामि यदि नाऽऽयामि ते गृहम्}
{मृगीवाक्यं ततः श्रुत्वा लुब्धको हृष्टमानसः}%॥९५॥

\twolineshloka
{संहृत्य बाणं सन्धानान्मुमोच हरिणीं तदा}
{तस्या मुक्तिप्रभावेण लिङ्गस्यापि प्रपूजनात्}%॥९६॥

\twolineshloka
{मुक्तोऽसौ पातकैः सर्वैस्तत्क्षणान्नात्र संशयः}
{द्वितीये प्रहरे प्राप्ते मध्यरात्रे वरानने}%॥९७॥

\twolineshloka
{तस्मिन्नेव क्षणे प्राप्ता कामार्ता मृगसुन्दरी}
{सन्त्रस्ता भयसंविग्ना पतिमन्वेष्यतीमुहुः}%॥९८॥

\twolineshloka
{जालिमध्ये स्थितेनाथ दृष्टा सा लुब्धकेन तु}
{पुनर्वृक्षस्य पत्राणि त्रोटयित्वा करेण तु}%॥९९॥

\twolineshloka
{क्षिप्तानि दक्षिणे भागे लिङ्गोपरिदिदृक्षया}
{तस्या वधार्थं तेनाथ बाणो धनुषि सन्धितः}%॥१००॥

\twolineshloka
{तिष्ठस्तत्रैकचित्तेन कुटुम्बार्थं जिघांसया}
{निरीक्ष्य लुब्धको यावद्बाणं तस्यां विमुञ्चति}%॥१॥

\twolineshloka
{तावन्मृग्या स सन्दृष्टो दृष्ट्वा तं विह्वलाऽभवत्}
{अद्यैव भगिनी मे हि लुब्धकेन विनाशिता}%॥२॥

\twolineshloka
{मम किं जीवितव्येन तस्या दुःखेन पीडिता}
{वरो मृत्युर्न शोको वै दृष्ट्वा व्याधं विशेषतः}%॥३॥

\onelineshloka*
{एवं सञ्चिन्त्य हरिणी लुब्धकं वाक्यमब्रवीत्}


\uvacha{हरिण्युवाच}
\onelineshloka
{धनुर्धरवर व्याध सर्वजीवनिकृन्तन}%॥४॥

\twolineshloka
{देहि मे वचनं चैकं पश्चात्त्वं मां निपातय}
{आयाता हरिणी चैका मार्गेणानेन लुब्धक}%॥५॥

\twolineshloka
{समायाताऽथ वा नैव सत्यं कथय सुव्रत}
{तच्छ्रुत्वा लुब्धकस्तत्र विस्मितः क्षणमैक्षत}%॥६॥

\twolineshloka
{तस्यास्तु यादृशी वाणी अस्याश्चैव तु तादृशी}
{सैवेयमागता नूनं प्रतिज्ञापालनाय च}%॥७॥

\twolineshloka
{अथवाऽन्या समायाता या तया कथिता पुरा}
{एवं सञ्चिन्त्य मनसा लुब्धको वाक्यमब्रवीत्}%॥८॥

\uvacha{लुब्धक उवाच}

\twolineshloka
{शृणु त्वं मृगि मे वाक्यं गता सा निजमन्दिरम्}
{त्वां दत्त्वा मम नूनं हि सा भवेत्सत्यवागपि}%॥९॥

\twolineshloka
{अहोरात्रं कृतं कष्टं कुटुम्बार्थे मया मृगि}
{अधुना त्वां हनिष्यामि देवतास्मरणं कुरु}%॥११०॥

\twolineshloka
{व्याधोक्तं वचनं श्रुत्वा हरिणी दुःखिता भृशम्}
{व्याधं प्राह रुदित्वा वै मा मां व्याध निपातय}%॥११॥

\twolineshloka
{तेजो बलं तथा सर्वं निर्दग्धं विरहाग्निना}
{अहं च दुर्बला नूनं मेदो मांसविवर्जिता}%॥१२॥

\twolineshloka
{केवलं पापभाक् त्वं हि मम प्राणविमोचकः}
{अहं प्राणैर्वियुज्यामि भोजनं ते न जायते}%॥१३॥

\twolineshloka
{बलवांश्च महातेजा मेदोमांससमन्वितः}
{अन्यश्च पीनगौराङ्गो मृगो ह्यत्रागमिष्यति}%॥१४॥

\twolineshloka
{तं हत्वा ते कुटुम्बस्य तृप्तिर्नूनं भविष्यति}
{अथवा त्वद्गृहं प्रातरागमिष्यामि लुब्धक}

\twolineshloka
{तयोक्तं लुब्धकः श्रुत्वा किं करोमीत्यचिन्तयत्}
{सञ्चिन्त्य लुब्धकः प्राह मृगीं शोकातुरां कृशाम्}%॥१५॥

\twolineshloka
{सत्यं वद महाभागे प्रत्ययो मे यथा भवेत्}
{तस्य तद्वचनं श्रुत्वा हरिणी दुःखकर्शिता}%॥१६॥

\onelineshloka*
{चक्रे सत्यप्रतिज्ञां तु व्याधस्याग्रे पुनः पुनः}


\uvacha{मृग्युवाच}
\onelineshloka
{क्षत्रियस्तु रणं दृष्ट्वा तस्माद्यो विनिवर्तते}%॥१७॥

\twolineshloka
{तस्य पापेन लिप्यामि यदि नाऽऽयामि ते गृहम्}
{भेदयन्ति तडागानि वापीश्चाथ गवामपि}%॥१८॥

\twolineshloka
{मार्गं स्थानं च ये घ्नन्ति सर्वसत्त्वभयङ्कराः}
{तेषां वै पातकं मह्यं यदि नाऽऽयामि ते गृहम्}%॥१९॥

\twolineshloka
{एतच्छ्रुत्वा तु व्याधेन साऽपि मुक्ता मृगी तदा}
{जलं पीत्वा तु बहुशो गता सद्यो यथागतम्}%॥१२०॥

\twolineshloka
{जालिमध्ये स्थितस्यास्य द्वितीयः प्रहरो गतः}
{त्रोटित्वा बिल्वपत्राणि पुनर्देवे न्ययोजयत्}%॥२१॥

\twolineshloka
{पीडितोऽतीव शीतेन क्षुधया गृहचिन्तया}
{शिवशिवेति जल्पन्वै न निद्रामुपलब्धवान्}%॥२२॥

\twolineshloka
{कृतं शिवार्चनं तेन तृतीये प्रहरेऽपि च}
{वीक्षते स्म दिशः सर्वा जीवनार्थं वरानने}%॥२३॥

\twolineshloka
{लुब्धकेनाथ दृष्टोऽसौ हरिणश्चञ्चलेक्षणः}
{विलोकयन्दिशः सर्वा मार्गमाणो मृगीपदम्}%॥२४॥

\twolineshloka
{सौभाग्यबलदर्पाढ्यो मदनोन्मत्तपीवरः}
{तं दृष्ट्वा बाणमाकृष्य ह्याकर्णं तुष्टमानसः}%॥२५॥

\twolineshloka
{बाणं मुञ्चति यावद्वै तावद् दृष्टो मृगेण तु}
{कालरूपं तु तं दृष्ट्वा मृगश्चिन्तितवान् भृशम्}%॥२६॥

\twolineshloka
{निश्चितं भविता मृत्युर्गोचरेऽस्य गतो यतः}
{भार्या प्राणसमा मेऽद्य व्याधेनेह निपातिता}%॥२७॥

\twolineshloka
{तया विरहितस्याद्य नूनं मृत्युर्भविष्यति}
{हा हा कालकृतं पापं यद्भार्यादुःखमागता}%॥२८॥

\twolineshloka
{भार्यया न समं सौख्यं गृहेऽपि च वनेऽपि च}
{तया विना न धर्मोऽस्ति नार्थकामौ विशेषतः}%॥२९॥

\twolineshloka
{वृक्षमूलेऽपि दयिता यत्र तिष्ठति तद्गृहम्}
{प्रासादोऽपि तया हीनः कान्तारादतिरिच्यते}%॥१३०॥

\twolineshloka
{धर्मकामार्थकार्येषु भार्या पुंसः सहायिनी}
{विदेशे च गतस्यापि सैव विश्वासकारिणी}%॥३१॥

\twolineshloka
{नास्ति भार्यासमो बन्धुर्नास्ति भार्यासमं सुखम्}
{नास्ति भार्यासमं लोके नरस्याऽऽर्तस्य भेषजम्}%॥३२॥

\twolineshloka
{यस्य भार्या गृहे नास्ति साध्वी च प्रियवादिनी}
{अरण्यं तेन गन्तव्यं यथाऽरण्यं तथा गृहम्}%॥३३॥

\twolineshloka
{एका प्राणसमा मेऽभूद् द्वितीया प्राणदा मम}
{भार्याविरहितस्याद्य जीवितं मम निष्फलम्}%॥३४॥

\onelineshloka*
{इत्येवं चिन्तयित्वा तु लुब्धकं वाक्यमब्रवीत्}


\uvacha{मृग उवाच}
\onelineshloka
{शृणु व्याध नरश्रेष्ठ ह्यामिषाहारभोजन}%॥३५॥

\twolineshloka
{यत्ते पृच्छाम्यहं वीर तत्सत्यं वद मे प्रभो}
{आगतं हरिणीयुग्मं केन मार्गेण तद्गतम्}%॥३६॥

\twolineshloka
{त्वया विनाशितं वाऽथ सत्यं कथय मेऽधुना}
{तस्य तद्वचनं श्रुत्वा लुब्धको विस्मयं गतः}%॥३७॥

\twolineshloka
{असावपि न सामान्यो देवता काऽप्यनुत्तम}
{उवाच लुब्धकः सद्यस्तस्याग्रे वाक्यमुत्तमम्}%॥३८॥

\uvacha{लुब्धक उवाच}

\twolineshloka
{ते गतेनेन मार्गेण सत्यं कृत्वा ममाग्रतः}
{ताभ्यां दत्तोऽसि भुक्त्यर्थं मम त्वमधुनाऽनघ}%॥३९॥

\twolineshloka
{सम्प्रति त्वं हनिष्यामि नैव मोक्ष्यामि कर्हिचित्}
{व्याधोक्तं हि वचः श्रुत्वा हरिणः प्राह सत्वरम्}%॥१४०॥

\uvacha{मृग उवाच}

\twolineshloka
{तत्सत्यं कीदृशं ताभ्यां वाक्यमुक्तं तवाग्रतः}
{येन ते प्रत्ययो जातो मुक्तं तद्धरिणीद्वयम्}%॥४१॥

\onelineshloka*
{ते गते केन मार्गेण ये मुक्ते व्याध तेऽधुना}


\uvacha{व्याध उवाच}
\onelineshloka
{ते गतेऽनेन मार्गेण स्वमाश्रमपदं प्रति}%॥४२॥

\twolineshloka
{व्याधेन कथितास्ताभ्यां शपथा ये कृतास्तदा}
{तच्छ्रुत्वा वचनं तस्य हरिणो हृष्टमानसः}%॥४३॥

\onelineshloka*
{व्याधं प्राह ततः शीघ्रं वचनं धर्मसंहितम्}


\uvacha{मृग उवाच}
\onelineshloka
{ताभ्यां व्याध यदुक्तं च तत्करोमि न चान्यथा}%॥४४॥

\twolineshloka
{प्रभाते त्वद्गृहं नूनमागमिष्यामि निश्चतम्}
{भार्या ऋतुमती मेऽद्य कामार्ताऽप्यधुना भृशम्}%॥४५॥

\twolineshloka
{गत्वा गृहेऽथ भुक्त्वा तामापृच्छ्य च सुहृज्जनान्}
{शपथैरागमिष्यामि गृहं ते नात्र संशयः}%॥४६॥

\twolineshloka
{न मद्देहेऽस्त्यसुङ्मांसं यत्त्वं भोक्तुमभीप्ससि}
{तद्वृथा मरणं मेऽस्माद्यदि मां त्वं हनिष्यसि}%॥४७॥

\onelineshloka*
{तन्मृगस्य वचः श्रुत्वा व्याधो वचनमब्रवीत्}

\uvacha{लुब्धक उवाच}
\onelineshloka
{असत्यं भाषसे धूर्त प्रतारयसि मां वृथा}%॥४८॥

\twolineshloka
{ज्ञातो मृत्युः स्फुटं यत्र तत्र गच्छति कोऽल्पधीः}
{व्याधस्य वचनं श्रुत्वा हरिणो वाक्यमब्रवीत्}%॥४९॥

\onelineshloka*
{शपथैरागमिष्यामि यथा ते प्रत्ययो भवेत्}


\uvacha{व्याध उवाच}
\onelineshloka
{मृग त्वं शपथान्ब्रूहि विश्वासो मे भवेद्यथा}%॥१५॥

\onelineshloka*
{यथा हि प्रेषयामि त्वां स्वगृहं प्रति कामुक}



\uvacha{मृग उवाच}
\onelineshloka
{भर्तारं वञ्चयेद्या स्त्री स्वामिनं वञ्चयेन्नरः}%॥५१॥

\twolineshloka
{मित्रं च वञ्चयेद्यस्तु गुरुद्रोहं करोति यः}
{विषमं तु\footnotemark{} रसं दद्यात्प्रेमभेदं करोति यः}%॥५२॥
\footnotetext{एकपङ्क्तौ भोजने इत्यर्थः।}

\twolineshloka
{भेदयेद्यस्तडागानि प्रासादं पातयेत्तथा}
{प्रवासशीलो यो विप्रः क्रयविक्रयकारकः}%॥५३॥

\twolineshloka
{सन्ध्यास्नानविहीनश्च वेदशास्त्रविवर्जितः}
{मद्यपाः स्त्रीषु रक्ता ये परनिन्दारताश्च ये}%॥५४॥

\twolineshloka
{परस्त्री सेवका विप्राः परपैशून्यसूचकाः}
{शूद्रान्नभोजिनो ये च भार्यापुत्रांस्त्यजन्ति ये}%॥५५॥

\twolineshloka
{वेदनिन्दापरा ये च वेदशास्त्रार्थनिन्दकाः}
{तेषां वै पातकं मह्यं यदि नाऽऽयामि ते गृहम्}%॥५६॥

\twolineshloka
{भार्या सङ्ग्रहणी यस्य व्रतशौचविर्वाजता}
{सर्वाशी सर्वविक्रेता द्विजानामपि निन्दकः}%॥५७॥

\twolineshloka
{त्रिषु वर्णेषु शुश्रूषां यः शूद्रो न करोति वै}
{विप्रवाक्यं परित्यज्य पाखण्डाभिरतः सदा}%॥५८॥

\twolineshloka
{ब्रह्मचर्यरताः शूद्रा ये च पाखण्डसंश्रिताः}
{तेषां वै पातकं मह्यं यदि नाऽऽयामि ते गृहम्}%॥५९॥

\twolineshloka
{तिलांस्तैलं घृतं क्षौद्रं लवणं सगुडं तथा}
{लोहं लाक्षादिकं सर्वं रङ्गान्नानाविधानपि}%॥१६०॥

\twolineshloka
{मद्यं मांसं विषं दुग्धं नीलं च वृषभं तथा}
{मीनं क्षीरं सर्पकूटं चित्रातकफलानि च}%॥६१॥

\twolineshloka
{विक्रीणीते द्विजो यस्तु तस्य पापं भवेन्मम}
{आदित्यं विष्णुमीशानं गणाध्यक्षं तु पार्वतीम्}%॥६२॥

\twolineshloka
{एतांस्त्यक्त्वा गृहे मूढो योऽन्यं पूजयते नरः}
{तस्य पापेन लिप्यामि यदि नाऽऽयामि ते गृहम्}%॥६३॥

\twolineshloka
{यो गां स्पृशति पादेन ह्युदितेऽर्के च सुप्यति}
{एकाकी मिष्टमश्नाति तस्य पापस्य भागहम्}%॥६४॥

\twolineshloka
{मातापित्रोरपोष्टा च क्रियामुद्दिश्य\footnotemark{} पाचकः}
{कन्याशुल्कोपजीवी च देवब्राह्मणनिन्दकः}%॥६५॥
\footnotetext{आत्मोद्देशेनैव भुजिक्रियामित्यर्थः।}

\twolineshloka
{गोग्रासं हन्तकारं च अतिथीनां च पूजनम्}
{ये न कुर्वन्ति गृहिणस्तेषां पापं भवेन्मम}%॥६६॥

\twolineshloka
{वृन्ताकं च पटोलं च कलिङ्ग तुम्बिकाफलम्}
{मूलकं लशुनं कन्दं कुसुम्भं कालशाककम्}%॥६७॥

\twolineshloka
{एतानि भक्षयेद्यस्तु नरो वै ज्ञानदुर्बलः}
{न यस्य जायते शुद्धिश्चान्द्रायणशतैरपि}%॥६८॥

\twolineshloka
{एतस्य पातकं मह्यं यदि नाऽऽयामि ते गृहम्}
{यः पठेत्स्वरहीनं च लक्षणेन विवर्जितम्}%॥६९॥

\twolineshloka
{रथ्यां पर्यटमानस्तु वेदानुद्गिरयेत्तु यः}
{विप्रस्य पठतो यस्य शृणोति यदि चान्त्यजः}%॥१७०॥

\twolineshloka
{वेदोपजीवको विप्रोऽतिलोभाच्छूद्रभोजनः}
{तस्य पापेन लिप्यामि यदि नाऽऽयामि ते गृहम्}%॥७१॥

\twolineshloka
{शूद्रान्नेषु च ये सक्ताः शूद्रसम्पर्कदूषिताः}
{तेषां पापेन लिप्यामि यदि नाऽऽयामि ते गृहम्}%॥७२॥

\twolineshloka
{लेखकश्चित्रकर्ता च वैद्यो नक्षत्रसूचकः}
{कूटकर्ता द्विजो यश्च तस्य पापस्य भागहम्}%॥७३॥

\twolineshloka
{कूटसाक्षी मृषावादी परद्रव्यस्य तस्करः}
{परदाराभिगामी च तथा विश्वासघातकः}%॥७४॥

\twolineshloka
{द्रव्ये द्रव्यं विनिक्षिप्य पानकूटं समाश्रितः}
{वेश्यारताः सदा ये च दानदातुर्निवारकाः}%॥७५॥

\twolineshloka
{भर्तारमर्थहीनं च कुरूपं व्याधिपीडितम्}
{या न पूजयते नारी रूपयौवनगर्विता}%॥७६॥

\twolineshloka
{एकादशीं तथा माघे कृष्णे शिवचतुर्दशीम्}
{पूर्वविद्धां प्रकुर्वन्ति तेषां पापस्य भागहम्}%॥७७॥

\twolineshloka
{अथ किं बहुनोक्तेन भो लुब्धक तवाग्रतः}
{यदि नाऽऽयामि ते गेहं ममासत्यं भवेत्तदा}%॥७८॥

\twolineshloka
{तेन वाक्येन सन्तुष्टो व्याधो वै वीतकल्मषः}
{संहृत्य धनुषो बाणं मृगो मुक्तो गृहं प्रति}%॥७९॥

\twolineshloka
{जलं पीत्वा तु हरिणः प्रविष्टो गहनं प्रति}
{गतोऽसौ तेन मार्गेण गतं येन मृगीद्वयम्}%॥१८०॥

\twolineshloka
{लुब्धकेन तदा तत्र जालिमध्ये स्थितेन हि}
{प्रत्यूषे बिल्वपत्राणि त्रोटयत्वोज्झितानि वै}%॥८१॥

\twolineshloka
{शिवशिवेति जल्पन्वै ह्याशु यातो निजाश्रमम्}
{अथोदिते सूर्यबिम्बे अकामाज्जागरे कृते}%॥८२॥

\twolineshloka
{पापान्मुक्तोऽप्यसौ सद्यः शिवपूजाप्रभावतः}
{यावद्दिशो निरीक्षेत निराशो भोजनं प्रति}%॥८३॥

\twolineshloka
{तावच्छशुवृता चान्या मृगी तत्र समागता}
{दृष्ट्वा मृगी तदा व्याधो बाणं धनुषि योजयन्}%॥८४॥

\twolineshloka
{यावन्मुञ्चत्यसौ बाणं तावत्प्रोवाच तं मृगी}
{मा बाणान्मुञ्च धर्मात्मन्धर्मं मा मुञ्च सुव्रत}%॥८५॥

\twolineshloka
{अहं न वध्या सर्वेषामिति शास्त्रविनिश्चयः}
{शयानो मैथुनासक्तः स्तनपो व्याधिपीडितः}%॥८६॥

\twolineshloka
{न हन्तव्यो मृगो राज्ञा मृगी च शिशुना वृता}
{अथवा धर्ममुत्सृज्य मां हनिष्यसि मानद}%॥८७॥

\twolineshloka
{बालकं स्वगृहे मुक्त्वा सखीनां च निवेद्य वै}
{शपथैरागमिष्यामि शृणु व्याध वचो मम}%॥८८॥

\twolineshloka
{या स्वभर्तारमुत्सृज्य परे पुंसि रता सदा}
{तस्याः पापेन लिप्यामि यदि नाऽऽयामि ते गृहम्}%॥८९॥

\twolineshloka
{मद्यं मांसं विषं दुग्धं नीलीं कुम्भफलानि च}
{एतानि विक्रयेद्यस्तु नरो मोहसमन्वितः}%॥१९०॥

\twolineshloka
{तेषां पापेन लिप्यामि यदि नाऽऽयामि ते गृहम्}
{ये कृताः शपथाः पूर्वं तवाग्रे व्याधसत्तम}%॥९१॥

\twolineshloka
{ते सर्वे मम सन्त्यत्र यदि नायाम्यहं पुनः}
{तस्यास्तद्वचनं श्रुत्वा व्याधो विस्मयमागमत्}%॥९२॥

\twolineshloka
{ततो व्याधेन सा युक्ता गता वै निजमन्दिरम्}
{व्याधोऽपि तत्स्थलं त्यक्त्वा जगाम स्वगृहं प्रति}%॥९३॥
\footnotetext{मुक्त इत्यस्य तेनेति गृहं प्रतीत्यस्य गमनायेति च शेषः।}

\twolineshloka
{सर्वेषां वचनं ध्यायन्मृगाणां सत्यवादिनाम्}
{एतेषां घातको नित्यमहं यास्यामि कां गतिम्}%॥९४॥

\twolineshloka
{एवं चिन्तयता गेहे दृष्टाः क्षुधितबालकाः}
{नान्नं मांसं गृहे तस्य भोजनं येन जायते}%॥९५॥

\twolineshloka
{निरामिषं तु तं दृष्ट्वा निराशास्तेऽभवंस्तदा}
{व्याधोपि च तदा तत्र तेषां वाक्यानि संस्मरन्}%॥९६॥

\twolineshloka
{न भोजनं न निद्रां च लभते विस्मयान्वितः}
{आगमिष्यन्ति ते नूनं शपथैरतियन्त्रिताः}%॥९७॥

\twolineshloka
{न तानहं वधिष्यामि सतां व्रतमनुस्मरन्}
{लुब्धकेन तदा मुक्तो हरिणः शपथैः कृतैः}%॥९८॥

\twolineshloka
{स्वमाश्रमं तु सम्प्राप्तो यत्र तद्धरिणीद्वयम्}
{सद्यः प्रसूता सा चैका द्वितीया रतिलालसा}%॥९९॥

\twolineshloka
{तृतीयाऽपि समायाता बालकैर्बहुभिर्वृता}
{सर्वाः समेता एकत्र मरणे कृतनिश्चयाः}%॥२००॥

\twolineshloka
{परस्परं प्रजल्पन्त्यो लुब्धकस्य विचेष्टितम्}
{सार्तवां हरिणीं भुक्त्वा रूपाढ्यां रतिलालसाम्}%॥१॥

\twolineshloka
{कृतकृत्योऽभवत्ताभिस्ततो वाक्यमथाब्रवीत्}
{युष्माभिरिह संस्थेयं कर्तव्यं प्राणरक्षणम्}%॥२॥

\twolineshloka
{व्याघ्राद्द्विपाल्लुब्धकेभ्यो बालकानां प्रयत्नतः}
{अहमत्र समायातः शपथैरतियन्त्रितः}%॥३॥

\twolineshloka
{अस्या ऋतुप्रदानाय पुनः सन्तानहेतवे}
{ऋतुमतीं तु यो भार्या न भुङ्क्ते मोहसंवृतः}%॥४॥

\twolineshloka
{भ्रूणहा सन्तु विज्ञेयस्तस्य जन्म निरकर्थम्}
{सन्तानात् स्वर्गमाप्नोति इह कीर्तिं च शाश्वतीम्}%॥५॥

\twolineshloka
{सन्ततिर्यत्नतः पाल्या स्वर्गसौख्यप्रदायिका}
{अपुत्रस्य गतिर्नास्ति इह लोके परत्र च}%॥६॥

\twolineshloka
{येन केनाप्युपायेन पुत्रमुत्पादयेत्पुमान्}
{मया च तत्र गन्तव्यं यत्र व्याधस्य मन्दिरम्}%॥७॥

\twolineshloka
{सत्यं तु पालनीयं स्यात्सत्ये धर्मः प्रतिष्ठितः}
{एतच्छ्र ुत्वा तु ता नार्यो वाक्यमूचुः सुदुःखिताः}%॥८॥

\twolineshloka
{वयमप्यागमिष्यामस्त्वया सार्धं मृगोत्तम}
{तथा ते विप्रियं कान्त न स्मरामः कदाचन}%॥९॥

\twolineshloka
{पुष्पितेषु वनान्तेषु नदीनां सङ्गमेषु च}
{कन्दरेषु च शैलानां भवता रमिता वयम्}%॥२१०॥

\twolineshloka
{न कार्यमप्यतः कान्त जीवितेन त्वया विना}
{नारीणां पतिहीनानां जीवितैः किं प्रयोजनम्}%॥११॥

\twolineshloka
{मितं ददाति हि पिता मितं भ्राता मितं सुतः}
{अमितस्य हि दातारं भर्तारं का न पूजयेत्}%॥१२॥

\twolineshloka
{अपि द्रव्ययुता नारी बहुपुत्रसुहृद्वृता}
{सा शोच्या बन्धुवर्गस्य पतिहीनकुलाङ्गना}%॥१३॥

\twolineshloka
{वैधव्यसदृशं दुःखं स्त्रीणामन्यन्न विद्यते}
{धन्यास्ता योषितो यास्तु म्रियन्ते भर्तुरग्रतः}%॥१४॥

\twolineshloka
{नातन्त्री वाद्यते वीणा नाचक्रो भ्रमते रथः}
{नापतिः सुखमाप्नोति नारी पुत्रशतैर्वृता}%॥१५॥

\twolineshloka
{नास्ति भर्तृसमो धर्मो नास्ति धर्मसमः सुहृत्}
{नास्ति भर्तृसमो नाथः स्त्रीणां भर्ता परा गतिः}%॥१६॥

\twolineshloka
{एवं विलिप्य ताः सर्वा मरणे कृतनिश्चयाः}
{बालकैस्तैः समायुक्ता भतृशोकेन दुःखिताः}%॥१७॥

\twolineshloka
{मृगस्तासां वचः श्रुत्वा हदि चिन्तापरोऽभवत्}
{गन्तव्यं किं न गन्तव्यं मया व्याधस्य मन्दिरम्}%॥१८॥

\twolineshloka
{एकतस्तु कृतं रक्षन्कुटुम्बस्य\footnotemark{} क्षयो भवेत्}
{तदन्तिकं न चेद्यामि मम सत्यं क्षयं व्रजेत्}%॥१९॥
\footnotetext{गमिष्यामि चेदिति शेषः।}

\twolineshloka
{वरं पुत्रस्य मरणं भार्याया आत्मनस्तथा}
{सत्ये त्यक्ते नरो नित्यमाकल्पं रौरवं व्रजेत्}%॥२२०॥

\twolineshloka
{तस्मात्सत्यं पालनीयं नरैः श्रेयोर्थिभिः सदा}
{सत्येन धार्यते पृथ्वी सत्येन तपते रविः}%॥२१॥

\twolineshloka
{सत्येन वायवो वान्ति सत्येन वर्धते परम्}
{एवं सञ्चिन्त्य हरिणी धर्मान् हृदि मनोरमान्}%॥२२॥

\twolineshloka
{ताभिः सहैव शनकैः क्षणात्तस्याश्रमं ययौ}
{तस्मिन्सरसि स स्नात्वा कर्मन्यासं चकार ह}%॥२३॥

\twolineshloka
{तल्लिङ्गं प्रणिपत्याशु हृदि ध्यायन्सदाशिवम्}
{भक्ष्यं पानं परित्यज्य मैथुनं भोगमेव च}%॥२४॥

\twolineshloka
{कामं क्रोधं तथा लोभं मायां मोक्षविनाशिनीम्}
{वन्दयित्वा\footnotemark{} तु तं देवं लुब्धकाभिमुखं ययौ}%॥२५॥
\footnotetext{खाद्यपेयादिकं चैवेत्यपि पाठः।}

\twolineshloka
{तस्य भार्याश्च पुत्राश्च मरणे कृतनिश्चयाः}
{अनशनं व्रतं गृह्य पृष्ठलग्नाः समाययुः}%॥२६॥

\twolineshloka
{भार्यापुत्रैः परिवृतो मृगस्तं देशमागमत्}
{क्षुधितैर्बालकैर्युक्तो लुब्धको यत्र तिष्ठति}%॥२७॥

\twolineshloka
{मृगस्तं देशमागत्य कुटुम्बेन समन्वितः}
{पालयन्सर्ववाक्यानि\footnotemark{} लुब्धकं वाक्यमब्रवीत्}%॥२८॥
\footnotetext{पूर्वोक्तानीत्यर्थः}

\uvacha{मृग उवाच}

\twolineshloka
{हन्या मां प्रथमं व्याध पश्चाद्भार्याः क्रमेण तु}
{बालकानि ततः पश्चाद्धन्यतां मा विलम्बय}%॥२९॥

\threelineshloka
{लुब्धकैस्तु मृगा भक्ष्या नास्ति दोषः कदाचन}
{वयं यास्याम स्वर्लोकं सत्यपूता न संशयः}%॥२३०॥
{तवापि सकुटुम्बस्य प्राणपुष्टिर्भविष्यति}

\twolineshloka
{एतच्छ्रुत्वा तु वचनं मृगोक्तं लुब्धकस्तदा}
{आत्मानं निन्दयित्वा तु हरिणं वाक्यमब्रवीत्}%॥३१॥

\uvacha{व्याध उवाच}

\twolineshloka
{अहो मृग महासत्त्व गच्छ गच्छ स्वमाश्रमम्}
{आमिषेण न मे कार्यं यद्भाव्यं तद्भविष्यति}%॥३२॥

\twolineshloka
{जीवानां घातने पापं बन्धने तर्जने तथा}
{नैव पापं करिष्यामि कुटुम्बार्थे कदाचन}%॥३३॥

\twolineshloka
{त्वं गुरुर्मम धर्माणामुपदेष्टा मृगोत्तम}
{गच्छ गच्छ मृगश्रेष्ठ कुटुम्बेन समन्वितः}%॥३४॥

\twolineshloka
{मया त्यक्तानि शस्त्राणि सत्यधर्मः समाश्रितः}
{तद् व्याधवचनं श्रुत्वा हरिणः प्राह तं पुनः}%॥३५॥

\uvacha{मृग उवाच}

\twolineshloka
{कर्मन्यासमहं कृत्वा त्वत्सकाशमिहागतः}
{हन्यतां हन्यतां शीघ्र न ते पापं भविष्यति}%॥३६॥

\twolineshloka
{मया दत्ता पुरा वाक्यं तया बद्धो न याम्यहम्}
{मया मम कुटुम्बेन त्यक्तो लोभं स्वजीवने}%॥३७॥

\onelineshloka*
{एतच्छ्रुत्वा तु वचनं लुब्धको वाक्यमब्रवीत्}

\uvacha{लुब्धक उवाच}
\onelineshloka
{त्वं बन्धुस्त्वं गुरुस्त्राता त्वं मे माता पिता सुहृत्}%॥३८॥

\twolineshloka
{मया त्यक्तानि शस्त्राणि त्यक्तं मायादिकं बलम्}
{कस्य भार्या सुताः कस्य कुटुम्बं कस्य तन्मृग}%॥३९॥

\twolineshloka
{तैः स्वकर्म च भोक्तव्यं मृग गच्छ यथासुखम्}
{इत्युक्त्वा स तदा तूर्णं बभञ्ज सशरं धनुः}%॥२४०॥

\twolineshloka
{मृगं प्रदक्षिणीकृत्य नमस्कृत्य क्षमापयत्}
{एतस्मिन्नन्तरे नेदुर्देवदुन्दुभयो दिवि}%॥४१॥

\twolineshloka
{आकाशात्पुष्पवृष्टिस्तु पपात सुमनोहरा}
{तदा दूतः समायातो विमानं गृह्य शोभनम्}%॥४२॥

\uvacha{देवदूत उवाच}

\twolineshloka
{अहो व्याध महासत्त्व सर्वसत्त्वक्षयङ्कर}
{विमानमिदमारुह्य सदेहः स्वर्गमाविश}%॥४३॥

\twolineshloka
{शिवरात्रिप्रभावेण पातकं ते क्षयं गतम्}
{उपवासस्तु सञ्जातो निशि जागरणं कृतम्}%॥४४॥

\twolineshloka
{यामे यामे कृता पूजा अज्ञानेन शिवस्य च}
{सर्वपापविनिर्मुक्तो गच्छ त्वं रुद्रमन्दिरम्}%॥४५॥

\twolineshloka
{विमानं च समारुह्य सद्यः शिवपदं व्रज}
{मृगराज महासत्व भार्यापुत्रसमन्वितः}%॥४६॥

\twolineshloka
{भार्यात्रितयसंयुक्तो नक्षत्रपदमाप्नुहि}
{तव नाम्ना तु तद् वृक्षं लोके ख्यातं भविष्यति}%॥४७॥

\twolineshloka
{एतच्छ्रुत्वा तु वचनं लुब्धकोऽथ मृगस्तथा}
{विमानानि समारुह्य नाक्षत्रं पदमागताः}%॥४८॥

\twolineshloka
{हरिणीद्वयमन्वेनं पृष्ठतो मृगमेव च}
{तारात्रितयसंयुक्तं मृगशीर्षं तदुच्यते}%॥४९॥

\twolineshloka
{बालकद्वितयं चाग्रे तृतीया पृष्ठतो मृगी}
{पृष्ठतस्तत्र सम्प्राप्ता मृगशीर्षस्य सन्निधौ}%॥२५०॥

\threelineshloka
{मृगराड् दृश्यतेऽद्यापि ऋक्षं व्योमगमुत्तमम्}
{उपवासं करिष्यन्ति जागरेण समन्वितम्}%॥५१॥
{यथोक्तशास्त्रमार्गेण तेषां मोक्षो न संशयः}

\twolineshloka
{शिवरात्रिसमं नास्ति व्रतं पापक्षयावहम्}
{यत्कृत्वा सर्वपापेभ्यो मुच्यते नात्र संशयः}%॥५२॥

\twolineshloka
{अश्वमेधसहस्राणि वाजपेयशतानि च}
{प्राप्नोति तत्फलं सर्वं नात्र कार्या विचारणा}%॥२५२॥

॥इति श्रीलिङ्गपुराणे उमामहेश्वरसंवादे शिवरात्रिव्रतकथा सम्पूर्णा॥