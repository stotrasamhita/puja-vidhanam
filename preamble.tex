% !TeX program = XeLaTeX
% !TeX root = pujavidhanam.tex
\renewcommand{\baselinestretch}{0.9}
\newcommand{\resetSankalpa}{}
\usepackage{shloka}
\usepackage{multirow}
\newcounter{totalshlokas}
\usepackage{wasysym}
\usepackage{longtable}
\usepackage{supertabular}

%%% HEADERS and FOOTERS %%%
\usepackage{fancyhdr}
\pagestyle{fancyplain}
\setlength{\headheight}{28pt}
\lhead[\fancyplain{}{\pagenumfont\large\thepage}]
   {\fancyplain{}{\leftmark}}
\rhead[\fancyplain{}{\leftmark}]
   {\fancyplain{}{\pagenumfont\large\thepage}}
\cfoot{}

\fancypagestyle{fancyplain}{ %
\fancyhf{} % remove everything
\renewcommand{\headrulewidth}{0pt} % remove lines as well
\renewcommand{\footrulewidth}{0pt}
\cfoot{\pagenumfont\large\thepage}}

%%% SECTIONS and CHAPTERS %%%
\makeatletter
\renewcommand\section{\resetShloka\@startsection {section}{1}{\z@}%
%{2.3ex \@plus.2ex}%
%{-3.5ex \@plus -1ex \@minus -.2ex}%
%{2.3ex \@plus.2ex}%
{10pt}
{2pt}
{\normalfont\LARGE\bfseries\centering}}

\renewcommand\chapter{\resetShloka\@startsection {chapter}{1}{\z@}%
{10pt}
{2pt}
{\normalfont\LARGE\bfseries\centering}}
\makeatother

%% ADD TCOLORBOX FOR DHUPAM ETC!

\setcounter{secnumdepth}{-1}
%for weird reasons this does not bookmark the section start, but the start of text in the section!!!
%\renewcommand\thesection{}
\renewcommand{\sectionmark}[1]{%
\markboth{\large #1}{}}
\renewcommand{\subsectionmark}[1]{%
\markboth{\large #1}{}}
\renewcommand{\chaptermark}[1]{%
\markboth{\large #1}{}}

\addtolength{\parskip}{4pt}
\setlength{\parindent}{0pt}
%\addtolength{\headsep}{10pt}
\setlength{\columnseprule}{0pt}
\setlength{\columnsep}{20pt}

%%% HYPERLINKS %%%
\usepackage[a5paper,bookmarks=true,bookmarksopen=true,xetex,colorlinks=true,
linkcolor=black,                    % colour of internal links
citecolor=cyan,                 % colour of links to bibliography
filecolor=magenta,          % colour of file links
urlcolor=blue                  % colour of external links
]{hyperref}
\hypersetup{bookmarksdepth=0}
%%% MISCELLANEOUS %%%
\hbadness=10000
\vbadness=10000
\hfuzz=6pt
%\listfiles


\usepackage{enumitem}
\usepackage{graphicx}

\setlist{nosep}
\newcommand{\see}[1]{\footnote{पृष्टं~\pageref{#1} पश्यताम्}}
\newcommand{\devaName}{देव}
\newcommand{\devAya}{}
\newcommand{\devaH}{}

\newcommand{\achamanam}{आचमनम्।}
\newcommand{\achamya}{(आचम्य)}
\newcommand{\shuklambaradharam}{
\twolineshloka*
{शुक्लाम्बरधरं विष्णुं शशिवर्णं चतुर्भुजम्}
{प्रसन्नवदनं ध्यायेत् सर्वविघ्नोपशान्तये}
}

\newcommand{\blank}{(\mbox{~~~})}

\newcommand{\pranayama}{प्राणान् आयम्य।}
\newcommand{\samvatsara}{\blank\see{app:samvatsara_names}}
\newcommand{\ayane}{\blank}
\newcommand{\rtu}{\blank}
\newcommand{\masa}{\blank}
\newcommand{\paksha}{(शुक्ल / कृष्ण)}
\newcommand{\tithau}{\blank}
\newcommand{\vasara}{(इन्दु / भौम / सौम्य / गुरु / भृगु / स्थिर / भानु)}
\newcommand{\nakshatra}{\blank\see{app:nakshatra_names}}
\newcommand{\yoga}{\blank\see{app:yoga_names}}
\newcommand{\karana}{\blank\see{app:karanam_names}}
\newcommand{\regularSankalpa}{अस्माकं सहकुटुम्बानां क्षेमस्थैर्य-धैर्य-वीर्य-विजय-आयुरारोग्य-ऐश्वर्याभिवृद्ध्यर्थं
धर्मार्थकाममोक्ष\-चतुर्विधफलपुरुषार्थसिद्ध्यर्थं पुत्रपौत्राभि\-वृद्ध्यर्थम् इष्टकाम्यार्थसिद्ध्यर्थं
मम इहजन्मनि पूर्वजन्मनि जन्मान्तरे च सम्पादितानां ज्ञानाज्ञानकृतमहा\-पातकचतुष्टय-व्यतिरिक्तानां
रहस्यकृतानां प्रकाशकृतानां सर्वेषां पापानां सद्य अपनोदनद्वारा सकल-पापक्षयार्थं}
\newcommand{\additionalSankalpa}{}
\newcommand{\kaale}{}
\newcommand{\prakaarena}{}
\newcommand{\prityartham}{} 
\newcommand{\pujaam}{}

\renewcommand{\resetSankalpa}{
\renewcommand{\samvatsara}{\blank\see{app:samvatsara_names}}
\renewcommand{\ayane}{(उत्तरायणे/दक्षिणायने)}
\renewcommand{\rtu}{\blank}
\renewcommand{\masa}{\blank}
\renewcommand{\paksha}{(शुक्ल / कृष्ण)}
\renewcommand{\tithau}{\blank}
\renewcommand{\vasara}{(इन्दु / भौम / सौम्य / गुरु / भृगु / स्थिर / भानु)}
\renewcommand{\nakshatra}{\blank\see{app:nakshatra_names}}
\renewcommand{\yoga}{\blank\see{app:yoga_names}}
\renewcommand{\karana}{\blank\see{app:karanam_names}}
\renewcommand{\regularSankalpa}{अस्माकं सहकुटुम्बानां क्षेमस्थैर्य-धैर्य-वीर्य-विजय-आयुरारोग्य-ऐश्वर्याभिवृद्ध्यर्थं
धर्मार्थकाममोक्ष\-चतुर्विधफलपुरुषार्थसिद्ध्यर्थं पुत्रपौत्राभि\-वृद्ध्यर्थम् इष्टकाम्यार्थसिद्ध्यर्थं
मम इहजन्मनि पूर्वजन्मनि जन्मान्तरे च सम्पादितानां ज्ञानाज्ञानकृतमहा\-पातकचतुष्टय-व्यतिरिक्तानां
रहस्यकृतानां प्रकाशकृतानां सर्वेषां पापानां सद्य अपनोदनद्वारा सकल-पापक्षयार्थं}
\renewcommand{\additionalSankalpa}{}
\renewcommand{\kaale}{}
\renewcommand{\prakaarena}{}
\renewcommand{\prityartham}{} 
\renewcommand{\pujaam}{}   
}

\newcommand{\sankalpa}{
\dnsub{सङ्कल्पः}

ममोपात्त-समस्त-दुरित-क्षयद्वारा श्री-परमेश्वर-प्रीत्यर्थं शुभे शोभने मुहूर्ते अद्य ब्रह्मणः
द्वितीयपरार्धे श्वेतवराहकल्पे वैवस्वतमन्वन्तरे अष्टाविंशतितमे कलियुगे प्रथमे पादे
जम्बूद्वीपे भारतवर्षे भरतखण्डे मेरोः दक्षिणे पार्श्वे शकाब्दे अस्मिन् वर्तमाने व्यावहारिकाणां 
प्रभवादीनां षष्ट्याः संवत्सराणां मध्ये
\textbf{\samvatsara} नाम संवत्सरे
\textbf{\ayane{}}
\textbf{\rtu}-ऋतौ
\textbf{\masa}-मासे 
\textbf{\paksha}पक्षे
\textbf{\tithau} शुभतिथौ
\textbf{\vasara}-वासरयुक्तायां
\textbf{\nakshatra}-नक्षत्र-%
\textbf{\yoga}-योग-%
\textbf{\karana}-करण-युक्तायां
च एवं गुण-विशेषण-विशिष्टायाम् अस्याम्\\
\textbf{\tithau{}} शुभतिथौ 
\regularSankalpa{}
\additionalSankalpa{}
\prityartham{}
\kaale{}
\prakaarena{}
यथाशक्ति-ध्यान-आवाहनादि-षोडशो\-पचारैः 
\pujaam{} करिष्ये।\\
तदङ्गं कलशपूजां च करिष्ये। 
}
\newcommand{\vighneshvaraYathasthanam}{
श्रीविघ्नेश्वराय नमः यथास्थानं प्रतिष्ठापयामि। शोभनार्थे क्षेमाय पुनरागमनाय च।\\
(गणपति-प्रसादं शिरसा गृहीत्वा)}

\newcommand{\aavaahitobhava}{
आवाहितो भव। स्थापितो भव। सन्निहितो भव। सन्निरुद्धो भव। अवकुण्ठितो भव।

प्रीतो भव। सुप्रसन्नो भव। सुमुखो भव। वरदो भव। प्रसीद प्रसीद॥

\twolineshloka*
{स्वामिन् सर्वजगन्नाथ यावत्पूजावसानकम्}
{तावत् त्वं प्रीतिभावेन बिम्बेऽस्मिन् सन्निधिं कुरु} 

\centerline{॥इति प्राणप्रतिष्ठा॥}
}

\newcommand{\hiranyagarbha}{
\twolineshloka*
{हिरण्यगर्भगर्भस्थं हेमबीजं विभावसोः}
{अनन्तपुण्यफलदम् अतः शान्तिं प्रयच्छ मे}
}

\newcommand{\kshama}[1]{\dnsub{अपराध-क्षमापनम्}

\twolineshloka*
{यस्य स्मृत्या च नामोक्त्या तपः-पूजा-क्रियादिषु}
{न्यूनं सम्पूर्णतां याति सद्यो वन्दे #1}

\fourlineindentedshloka*
{कायेन वाचा मनसेन्द्रियैर्वा}
{बुद्‌ध्याऽऽत्मना वा प्रकृतेः स्वभावात्}
{करोमि यद्यत् सकलं परस्मै}
{नारायणायेति समर्पयामि}

\centerline{सर्वं तत्सद्ब्रह्मार्पणमस्तु।}}

\newcommand{\OM}{\ifbool{veda}{ॐ}{}}

\newcommand{\OMshri}{\ifbool{veda}{ॐ~}{श्री-}}