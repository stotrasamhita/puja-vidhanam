% !TeX program = XeLaTeX
% !TeX root = ../pujavidhanam.tex

\setlength{\parindent}{0pt}
\chapt{श्री-सङ्कष्टचतुर्थीव्रत-सिद्धिविनायक-पूजा}
\centerline{\small{(मूलम्—श्री-व्रतराजः)}}

श्रावणकृष्णचतुर्थ्यां सङ्कष्टचतुर्थीव्रतम्। तच्च चन्द्रोदयव्यापिन्यां कार्यम्। श्रावणे बहुले पक्षे चतुर्थ्यां तु विधूदये। गणेशं पूजयित्वा तु चन्द्रायार्घ्यं प्रदापयेत् इति कथायां तत्र व्रतपूजाविधानात्।

द्विनद्वये तद्व्याप्तौ पूर्वेव। “मातृविद्धा गणेश्वर” इति वचनात्। दिनद्वयेऽव्याप्तौ परैव। हेमाद्रौ चन्द्रोदयाभावे चतुर्थी निशि षट्घटिकाव्याप्ता परैव व्रते। इति।

\input{purvanga/vighneshwara-puja}

\sect{प्रधान-पूजा — श्री-सिद्धिविनायक-पूजा}

\twolineshloka*
{शुक्लाम्बरधरं विष्णुं शशिवर्णं चतुर्भुजम्}
{प्रसन्नवदनं ध्यायेत् सर्वविघ्नोपशान्तये}

प्राणान् आयम्य। ॐ भूः + भूर्भुवः॒ सुव॒रोम्।

\dnsub{सङ्कल्पः}

ममोपात्त-समस्त-दुरित-क्षयद्वारा श्री-परमेश्वर-प्रीत्यर्थं शुभे शोभने मुहूर्ते अद्य ब्रह्मणः
द्वितीयपरार्धे श्वेतवराहकल्पे वैवस्वतमन्वन्तरे अष्टाविंशतितमे कलियुगे प्रथमे पादे
जम्बूद्वीपे भारतवर्षे भरतखण्डे मेरोः दक्षिणे पार्श्वे शकाब्दे अस्मिन् वर्तमाने व्यावहारिके
प्रभवादि षष्टिसंवत्सराणां मध्ये \mbox{(~~~)}\see{app:samvatsara_names} नाम संवत्सरे दक्षिणायने 
वर्ष-ऋतौ ( )-मासे कृष्णपक्षे चतुर्थ्यां शुभतिथौ \mbox{(~~~)} वासरयुक्तायाम्
\mbox{(~~~)}\see{app:nakshatra_names} नक्षत्र \mbox{(~~~)}\see{app:yoga_names} नाम योग 
\mbox{(~~~)}\see{app:karanam_names}-करण युक्तायां च एवं गुण विशेषण विशिष्टायाम्
अस्यां चतुर्थ्यां शुभतिथौ 
विद्याधनपुत्रपौत्रप्राप्त्यर्थं समस्तरोगमुक्तिकामः श्री-गणेश-प्रीत्यर्थं
अस्माकं सकुटुम्बानां क्षेमस्थैर्य-धैर्य-वीर्य-विजय-आयुरारोग्य-ऐश्वर्याभिवृद्ध्यर्थम्
धर्मार्थकाममोक्ष\-चतुर्विधफलपुरुषार्थसिद्ध्यर्थं पुत्रपौत्राभि\-वृद्ध्यर्थम् इष्टकाम्यार्थसिद्ध्यर्थं
मम इहजन्मनि पूर्वजन्मनि जन्मान्तरे च सम्पादितानां ज्ञानाज्ञानकृतमहा\-पातकचतुष्टय-व्यतिरिक्तानां 
रहस्यकृतानां प्रकाशकृतानां सर्वेषां पापानां सद्य अपनोदनद्वारा 
सकल-पापक्षयार्थं श्री-सिद्धिविनायक-प्रसादसिद्ध्यर्थं 
यथाशक्ति-ध्यानावाहनादिषोडशोपचारैः श्री-सिद्धिविनायक-पूजां सङ्कष्टचतुर्थीव्रतमहं करिष्ये।
तदङ्गं कलशपूजां च करिष्ये। 
% तत्रादौ स्वस्तिवाचनं गणपतिपूजनं कलशार्चनं करिष्ये।

श्रीविघ्नेश्वराय नमः यथास्थानं प्रतिष्ठापयामि। शोभनार्थे क्षेमाय पुनरागमनाय च।\\
(गणपति-प्रसादं शिरसा गृहीत्वा)

\input{purvanga/aasana-puja}

\input{purvanga/ghanta-puja}

\begingroup
\renewcommand{\devaName}{देव}
\input{purvanga/kalasha-puja}
\endgroup

\input{purvanga/aatma-puja}

\input{purvanga/pitha-puja}

\input{purvanga/guru-dhyanam}

\dnsub{प्राण-प्रतिष्ठा}

\sect{षोडशोपचार-पूजा}
सौवर्णरौप्यताम्रमृन्मयाद्यन्यतमां गणपतिमूर्तिं कृत्वा जलपूर्णं पूर्णपात्रं वस्त्रयुतकुम्भोपरि स्थापयित्वा
षोडशोपचारैः पूजयेत्।

\renewcommand{\devAya}{श्री-सिद्धिविनायकाय}
\begin{center}

\twolineshloka*
{लम्बोदरं चतुर्बाहुं त्रिनेत्रं रक्तवर्णकम्}
{नानारत्नैः सुवेषाढ्यं प्रसन्नास्यं विचिन्तयेत्}

\twolineshloka*
{ध्यायेद्गजाननं देवं तप्तकाञ्चनसुप्रभम्}
{चतुर्बाहुं महाकायं सूर्यकोटिसमप्रभम्}

\textbf{अस्मिन् बिम्बे/प्रतिमायां/चित्रपटे श्री-सिद्धिविनायकं ध्यायामि।}

\twolineshloka*
{आगच्छ त्वं जगन्नाथ सुरासुरनमस्कृत}
{अनाथनाथ सर्वज्ञ विघ्नराज कृपां कुरु}

\twolineshloka*
{स॒हस्र॑शीर्‌षा॒ पुरु॑षः। स॒ह॒स्रा॒क्षः स॒हस्र॑पात्}
{स भूमिं॑ वि॒श्वतो॑ वृ॒त्वा। अत्य॑तिष्ठद्दशाङ्गु॒लम्}
\textbf{गजास्याय नमः, अस्मिन् बिम्बे/प्रतिमायां/चित्रपटे गजास्यमावाहयामि।}
\medskip

\twolineshloka*
{गोप्ता त्वं सर्वलोकानामिन्द्रादीनां विशेषतः}
{भक्तदारिद्र्यविच्छेत्ता एकदन्त नमोऽस्तु ते}

\twolineshloka*
{पुरु॑ष ए॒वेदꣳ सर्वम्᳚। यद्भू॒तं यच्च॒ भव्यम्᳚}
{उ॒तामृ॑त॒त्वस्येशा॑नः। यदन्ने॑नाति॒रोह॑ति}
\textbf{विघ्नराजाय नमः, आसनं समर्पयामि।}
\medskip

\twolineshloka*
{मोदकान्धारयन्हस्ते भक्तानां वरदायक}
{देवदेव नमस्तेऽस्तु भक्तानां फलदो भव}

\twolineshloka*
{ए॒तावा॑नस्य महि॒मा। अतो॒ ज्यायाꣴ॑श्च॒ पूरु॑षः}
{पादो᳚ऽस्य॒ विश्वा॑ भू॒तानि॑। त्रि॒पाद॑स्या॒मृतं॑ दि॒वि}
\textbf{लम्बोदराय नमः, पाद्यं समर्पयामि।}
\medskip

\twolineshloka*
{महाकाय महारूप अनन्तफलदो भव}
{देवदेव नमस्तेऽस्तु सर्वेषां पापनाशन}

\twolineshloka*
{त्रि॒पादू॒र्ध्व उदै॒त्पुरु॑षः। पादो᳚ऽस्ये॒हाऽऽभ॑वा॒त्पुनः॑}
{ततो॒ विश्व॒ङ्व्य॑क्रामत्। सा॒श॒ना॒न॒श॒ने अ॒भि}
\textbf{शङ्करसूनवे नमः, अर्घ्यं समर्पयामि।}
\medskip

\twolineshloka*
{कुरुष्वाचमनं देव सुरवन्द्य सुवाहन}
{सर्वाघदलनस्वामिन्नीलकण्ठ नमोऽस्तु ते}

\twolineshloka*
{तस्मा᳚द्वि॒राड॑जायत। वि॒राजो॒ अधि॒ पूरु॑षः}
{स जा॒तो अत्य॑रिच्यत। प॒श्चाद्भूमि॒मथो॑ पु॒रः}
\textbf{उमासुताय नमः, आचमनीयं समर्पयामि।}
\medskip

\twolineshloka*
{स्नानं पञ्चामृतेनैव गृहाण गणनायक}
{अनाथनाथ सर्वज्ञ नमो मूषकवाहन}

\twolineshloka*
{पयो-दधि-घृतं चैव शर्करामधुसंयुतम्}
{पञ्चामृतेन स्नपनं प्रीत्यर्थं प्रतिगृह्यताम्}

\twolineshloka*
{यत्पुरु॑षेण ह॒विषा᳚। दे॒वा य॒ज्ञमत॑न्वत}
{व॒स॒न्तो अ॑स्याऽऽसी॒दाज्यम्᳚। ग्री॒ष्म इ॒ध्मः श॒रद्ध॒विः}
\textbf{वक्रतुण्डाय नमः, पञ्चामृतस्नानम् समर्पयामि।}
\medskip

\twolineshloka*
{गङ्गा च यमुना चैव गोदावरिसरस्वती}
{नर्मदा सिन्धुकावेरी जलं स्नानाय कल्पितम्}

\twolineshloka*
{स॒प्तास्या॑ऽऽसन् परि॒धयः॑। त्रिः स॒प्त स॒मिधः॑ कृ॒ताः}
{दे॒वा यद्य॒ज्ञं त॑न्वा॒नाः। अब॑ध्न॒न् पु॑रुषं प॒शुम्}
\textbf{हेरम्बाय नमः, शुद्धोदकस्नानं समर्पयामि। स्नानानन्तरम् आचमनीयं समर्पयामि।}
\medskip

\twolineshloka*
{रक्तवस्त्रसुयुग्मं च देवानामपि दुर्लभम्}
{गृहाण मङ्गलं देव लम्बोदर हरात्मज}

\twolineshloka*
{तं य॒ज्ञं ब॒र्{}हिषि॒ प्रौक्षन्॑। पुरु॑षं जा॒तम॑ग्र॒तः}
{तेन॑ दे॒वा अय॑जन्त। सा॒ध्या ऋष॑यश्च॒ ये}
\textbf{शूर्पकर्णाय नमः, वस्त्रं समर्पयामि।}
\medskip

\twolineshloka*
{ब्रह्मसूत्रं सोत्तरीयं गृहाण गणनायक}
{आरक्तं ब्रह्मसूत्रं च कनकस्योतरीयकम्}

\twolineshloka*
{तस्मा᳚द्य॒ज्ञाथ्स॑र्व॒हुतः॑। सम्भृ॑तं पृषदा॒ज्यम्}
{प॒शूꣴस्ताꣴश्च॑क्रे वाय॒व्यान्॑। आ॒र॒ण्यान्ग्रा॒म्याश्च॒ ये}
\textbf{कुब्जाय नमः, यज्ञोपवीतं समर्पयामि।}
\medskip

\twolineshloka*
{उद्यद्भास्करसङ्काशं सन्ध्यावदरुणं प्रभो}
{वीरालङ्करणं दिव्यं सिन्दूरं प्रतिगृह्यताम्}
\textbf{\devAya{} नमः, सिन्दूरं समर्पयामि।}\footnotemark
\medskip

\twolineshloka*
{नानाविधानि दिव्यानि नानारत्नोज्ज्वलानि च}
{भूषणानि गृहाणेश पार्वतीप्रियनन्दन}
\textbf{\devAya{} नमः, आभरणानि समर्पयामि।}\footnotemark[\value{footnote}]
\medskip

\twolineshloka*
{गृहाणेश्वर सर्वज्ञ दिव्यचन्दनमुत्तमम्}
{करुणाकर गुञ्जाक्ष गौरीसुत नमोऽस्तु ते}

\twolineshloka*
{तस्मा᳚द्य॒ज्ञाथ्स॑र्व॒हुतः॑। ऋचः॒ सामा॑नि जज्ञिरे}
{छन्दाꣳसि जज्ञिरे॒ तस्मा᳚त्। यजु॒स्तस्मा॑दजायत}
\textbf{गणेश्वराय नमः, दिव्यपरिमलगन्धान् धारयामि। गन्धस्योपरि हरिद्राकुङ्कुमं समर्पयामि।}
\medskip

\twolineshloka*
{अक्षताश्च सुरश्रेष्ठ कुङ्कुमाक्ताः सुशोभिताः}
{मया निवेदिता भक्त्या गृहाण गणनायक}
\textbf{\devAya{} नमः, अक्षतान् समर्पयामि।}
\medskip

\twolineshloka*
{सुगन्धिदिव्यमालां च गृहाण गणनायक}
{विनायक नमस्तुभ्यं शिवसूनो नमोऽस्तु ते\footnotemark[\value{footnote}]}
\footnotetext{अयं श्लोकः/उपचारः श्रीव्रतराजान्तर्गत-सिद्धिविनायक-पूजाया उद्धृतः}

\twolineshloka*
{माल्यादीनि सुगन्धीनि मालत्यादीनि मे प्रभो}
{मयाऽऽहृतानि पुष्पाणि पूजार्थं प्रतिगृह्यताम्}
\textbf{\devAya{} नमः, पुष्पमालां समर्पयामि।}

\twolineshloka*
{करवीरैर्जातिकुसुमैश्चम्पकैर्बकुलैः शुभैः}
{शतपत्रैश्च कह्लारैरर्चयेद् गणनायकम्}

\twolineshloka*
{तस्मा॒दश्वा॑ अजायन्त। ये के चो॑भ॒याद॑तः}
{गावो॑ ह जज्ञिरे॒ तस्मा᳚त्। तस्मा᳚ज्जा॒ता अ॑जा॒वयः॑}
\textbf{विघ्ननाशिने नमः, पुष्पाणि समर्पयामि।} पुष्पैः पूजयामि।

विघ्ननाशिने नमः, दूर्वाकुङ्कुमादि समर्पयामि। 
[अनेनैव नाम्ना दूर्वाकुङ्कुमादि दद्यात्।]

\dnsub{अङ्ग-पूजा}
\begin{longtable}{ll@{~नमः — }l@{~पूजयामि}}
    १. & गणेश्वराय  & पादौ\\
    २. & विघ्नराजाय  & जानुनी\\
    ३. & आखुवाहनाय  & ऊरू\\
    ४. & हेरम्बाय  & कटिं\\
    ५. & कामारिसूनवे  & नाभिं\\
    ६. & लम्बोदराय  & उदरं\\
    ७. & गौरीसुताय  & स्तनौ\\
    ८. & गणनायकाय  & हृदयं\\
    ९. & स्थूलकण्ठाय  & कण्ठं\\
    १०. & स्कन्दाग्रजाय  & स्कन्धौ\\
    ११. & पाशहस्ताय  & हस्तौ\\
    १२. & गजवक्त्राय  & वक्त्रं\\
    १३. & विघ्नहर्त्रे  & ललाटं\\
    १४. & सर्वेश्वराय  & शिरः\\
    १५. & श्रीगणाधिपाय  & सर्वाण्यङ्गानि\\
\end{longtable}

\dnsub{एकविंशति-दूर्वायुग्म-पूजा\footnotemark}
\footnotetext{इयं पूजा सिद्धिविनायकपूजाया उद्धृतः}
\begin{longtable}{ll@{~नमः — दूर्वायुग्मं समर्पयामि।}}
    १. & गणाधिपाय \\
    २. & पाशाङ्कुशधराय \\
    ३. & आखुवाहनाय \\
    ४. & विनायकाय \\
    ५. & ईशपुत्राय \\
    ६. & सर्वसिद्धि-प्रदाय \\
    ७. & एकदन्ताय \\
    ८. & इभवक्त्राय \\
    ९. & मूषिकवाहनाय \\
    १०. & कुमारगुरवे \\
    ११. & कपिलवर्णाय \\
    १२. & ब्रह्मचारिणे \\
    १३. & मोदकहस्ताय \\
    १४. & सुरश्रेष्ठाय \\
    १५. & गजनासिकाय \\
    १६. & कपित्थफल-प्रियाय \\
    १७. & गजमुखाय \\
    १८. & सुप्रसन्नाय \\
    १९. & सुराग्रजाय \\
    २०. & उमापुत्राय \\
    २१. & स्कन्दप्रियाय \\
\end{longtable}

\needspace{3em}
\begingroup
\setlength{\columnseprule}{1pt}
\let\chapt\sect
\input{../namavali-manjari/100/Ganapati_108.tex}

\endgroup

\twolineshloka*
{दशाङ्गं गुग्गुलं धूपमुत्तमं गणनायक}
{गृहाण देव देवेश उमासुत नमोऽस्तु ते}

\twolineshloka*
{यत्पुरु॑षं॒ व्य॑दधुः। क॒ति॒धा व्य॑कल्पयन्}
{मुखं॒ किम॑स्य॒ कौ बा॒हू। कावू॒रू पादा॑वुच्येते}
\textbf{विकटाय नमः, धूपमाघ्रापयामि।}
\medskip

\twolineshloka*
{सर्वज्ञ सर्वरत्नाढ्य सर्वेश विबुधप्रिय}
{गृहाण मङ्गलं दीपं घृतवर्तिसमन्वितम्}

\twolineshloka*
{ब्रा॒ह्म॒णो᳚ऽस्य॒ मुख॑मासीत्। बा॒हू रा॑ज॒न्यः॑ कृ॒तः}
{ऊ॒रू तद॑स्य॒ यद्वैश्यः॑। प॒द्भ्याꣳ शू॒द्रो अ॑जायत}
\textbf{वामनाय नमः, अलङ्कारदीपं सन्दर्शयामि।\\
दीपानन्तरम् आचमनीयं समर्पयामि।}
\medskip

ॐ भूर्भुवः॒ सुवः॑। + ब्र॒ह्मणे॒ स्वाहा᳚।

\twolineshloka*
{नैवेद्यं गृह्यतां देव नानामोदकसंयुतम्}
{पक्वान्नफलसंयुक्तं षड्रसैश्च समन्वितम्}

\twolineshloka*
{च॒न्द्रमा॒ मन॑सो जा॒तः। चक्षोः॒ सूर्यो॑ अजायत}
{मुखा॒दिन्द्र॑श्चा॒ग्निश्च॑। प्रा॒णाद्वा॒युर॑जायत}
\textbf{सर्वदेवाय नमः, (	) निवेदयामि, \\}
अमृतापिधानमसि।

\twolineshloka*
{कृष्णावेण्यागौतमीनां पयोष्णीनर्मदाजलैः}
{आचम्यतां विघ्नराज प्रसन्नो भव सर्वदा}
निवेदनानन्तरम् आचमनीयं समर्पयामि।\medskip

% \twolineshloka*
% {एलोशीरलवङ्गादिकर्पूरपरिवासितम्}
% {प्राशनार्थं कृतं तोयं गृहाण गणनायक}
% मध्ये मध्ये पानीयं समर्पयामि। उत्तरापोशनं मुखप्रक्षालनं च समर्पयामि।\medskip

% \twolineshloka*
% {मलयाचलसम्भूतं कर्पूरेण समन्वितम्}
% {करोद्वर्तनकं चारु गृह्यतां जगतः पते} 
% करोद्वर्तनम् समर्पयामि।

% \twolineshloka*
% {बीजपूराम्रपनसखजूरीकदलीफलम्} 
% {नारिकेलफलं दिव्यं गृहाण गणनायक} 

\twolineshloka*
{फलान्यमृतकल्पानि सुगन्धीन्यघनाशन}
{आनीतानि यथाशक्त्या गृहाण गणनायक}
\textbf{सर्वार्तिनाशिने नमः, फलं समर्पयामि।}
\medskip

\twolineshloka*
{एकविंशतिसङ्ख्याकान् मोदकान् घृतपाचितान्} 
{नैवेद्यं सफलं दद्यान्नमस्ते विघ्ननाशिने} 
\textbf{\devAya{} नमः, मोदकान् समर्पयामि।\footnote{अयं श्लोकः/उपचारः श्रीव्रतराजान्तर्गत-सिद्धिविनायक-पूजाया उद्धृतः}
}
\medskip

\twolineshloka*
{ताम्बूलं गृह्यतां देव नागवल्ल्या दलैर्युतम्}
{कर्पूरेण समायुक्तं सुगन्धं मुखभूषणम्}


\twolineshloka*
{नाभ्या॑ आसीद॒न्तरि॑क्षम्। शी॒र्ष्णो द्यौः सम॑वर्तत}
{प॒द्भ्यां भूमि॒र्दिशः॒ श्रोत्रा᳚त्। तथा॑ लो॒काꣳ अ॑कल्पयन्}
\textbf{विघ्नहर्त्रे नमः, कर्पूरताम्बूलं समर्पयामि।}
\medskip

% \twolineshloka*
% {हिरण्यगर्भगर्भस्थं हेमबीजं विभावसोः}
% {अनन्तपुण्यफलदम् अतः शान्तिं प्रयच्छ मे}

\twolineshloka*
{सर्वदेवाधिदेव त्वं सर्वसिद्धिप्रदायक}
{भक्त्या दत्तां मया देव गृहाण दक्षिणां विभो}
\textbf{सर्वेश्वराय नमः, दक्षिणां समर्पयामि।}
\medskip

\twolineshloka*
{पञ्चवर्तिसमायुक्तं वह्निना योजितं मया}
{गृहाण मङ्गलं दीपं विघ्नराज नमोऽस्तु ते}

\twolineshloka*
{चन्द्रादित्यौ च धरणी विद्युदग्निस्तथैव च}
{त्वमेव सर्वतेजांसि आर्तिक्यं प्रतिगृह्यताम्} 

\twolineshloka*
{वेदा॒हमे॒तं पुरु॑षं म॒हान्तम्᳚। आ॒दि॒त्यव॑र्णं॒ तम॑स॒स्तु पा॒रे}
{सर्वा॑णि रू॒पाणि॑ वि॒चित्य॒ धीरः॑। नामा॑नि कृ॒त्वाऽभि॒वद॒न्॒ यदास्ते᳚}
\textbf{\devAya{} नमः, समस्त-अपराध-क्षमापनार्थं कर्पूरनीराजनं दर्शयामि।\\}
कर्पूरनीरजनानन्तरम् आचमनीयं समर्पयामि।\\
रक्षां धारयामि। पुष्पैः पूजयामि।
\medskip

\twolineshloka*
{धा॒ता पु॒रस्ता॒द्यमु॑दाज॒हार॑। श॒क्रः प्रवि॒द्वान् प्र॒दिश॒श्चत॑स्रः}
{तमे॒वं वि॒द्वान॒मृत॑ इ॒ह भ॑वति। नान्यः पन्था॒ अय॑नाय विद्यते}

यो॑ऽपां पुष्पं॒ वेद॑। पुष्प॑वान् प्र॒जावा᳚न् पशु॒मान् भ॑वति।\\
च॒न्द्रमा॒ वा अ॒पां पुष्पम्᳚। पुष्प॑वान् प्र॒जावा᳚न् पशु॒मान् भ॑वति।\\
य ए॒वं वेद॑। यो॑ऽपामा॒यत॑नं॒ वेद॑। आ॒यत॑नवान् भवति।\medskip

ओं᳚ तद्ब्र॒ह्म। ओं᳚ तद्वा॒युः। ओं᳚ तदा॒त्मा।\\
ओं᳚ तथ्स॒त्यम्‌। ओं᳚ तथ्सर्वम्᳚‌। ओं᳚ तत्पुरो॒र्नमः॥\medskip

अन्तश्चरति॑ भूते॒षु॒ गुहायां वि॑श्वमू॒र्तिषु।\\
त्वं यज्ञस्त्वं वषट्कारस्त्वमिन्द्रस्त्वꣳ\\
रुद्रस्त्वं विष्णुस्त्वं ब्रह्म त्वं॑ प्रजा॒पतिः।\\
त्वं त॑दाप॒ आपो॒ ज्योती॒ रसो॒ऽमृतं॒ ब्रह्म॒ भूर्भुवः॒ सुव॒रोम्‌॥\medskip

\twolineshloka*
{यो वेदादौ स्व॑रः प्रो॒क्तो॒ वे॒दान्ते॑ च प्र॒तिष्ठि॑तः}
{तस्य॑ प्र॒कृति॑लीन॒स्य॒ यः॒ परः॑ स म॒हेश्व॑रः}
\textbf{\devAya{} नमः, वेदोक्तमन्त्रपुष्पाञ्जलिं समर्पयामि।}
\medskip


\twolineshloka*
{यानि कानि च पापानि जन्मान्तरकृतानि च}
{तानि तानि विनश्यन्ति प्रदक्षिण-पदे पदे}
\textbf{प्रदक्षिणं कृत्वा।}
\medskip

\twolineshloka*
{नमस्ते विघ्नसंहत्रे नमस्ते ईप्सितप्रद} 
{नमस्ते देवदेवेश नमस्ते गणनायक} 

\twolineshloka*
{विनायकेशपुत्रस्त्वं गजराज सुरोत्तम}
{देहि मे सकलान् कामान् वन्दे सिद्धिविनायक} 

\twolineshloka*
{नमोऽस्त्वनन्ताय सहस्रमूर्तये सहस्रपादाक्षिशिरोरुबाहवे}
{सहस्रनाम्ने पुरुषाय शाश्वते सहस्रकोटियुगधारिणे नमः}

\twolineshloka*
{स॒प्तास्या॑ऽऽसन् परि॒धयः॑। त्रिः स॒प्त स॒मिधः॑ कृ॒ताः}
{दे॒वा यद्य॒ज्ञं त॑न्वा॒नाः। अब॑ध्न॒न् पु॑रुषं प॒शुम्}
\textbf{\devAya{} नमः, नमस्कारान् समर्पयामि।}
\medskip

\twolineshloka*
{य॒ज्ञेन॑ य॒ज्ञम॑यजन्त दे॒वाः। तानि॒ धर्मा॑णि प्रथ॒मान्या॑सन्}
{ते ह॒ नाकं॑ महि॒मानः॑ सचन्ते। यत्र॒ पूर्वे॑ सा॒ध्याः सन्ति॑ दे॒वाः}
\textbf{\devAya{} नमः, छत्रचामर-नृत्त-गीत-वाद्यादि समस्तराजोपचारान् समर्पयामि।}
\medskip

\twolineshloka*
{यन्मयाऽऽचरितं देव व्रतमेतत् सुदुर्लभम्}
{गणेश त्वं प्रसन्नः सन् सफलं कुरु सर्वदा}

\twolineshloka*
{विनायक गणेशान सर्वदेवनमस्कृत}
{पार्वतीप्रिय विघ्नेश मम विघ्नान्निवारय} 

\twolineshloka*
{नमो नमो गणेशाय नमस्ते विश्वरूपिणे}
{निर्विघ्नं कुरु मे कामं नमामि त्वां गजानन}

\twolineshloka*
{अगजाननपद्मार्कं गजाननमहर्निशम्}
{अनेकदं तं भक्तानाम् एकदन्तमुपास्महे}

\twolineshloka* 
{विनायक वरं देहि महात्मन् मोदकप्रिय}
{अविघ्नं कुरु मे देव सर्वकार्येषु सर्वदा}

\textbf{प्रार्थनाः समर्पयामि।}

\twolineshloka*
{एवं पूजा प्रकर्तव्या षोडशैरुपचारकैः}
{मोदकान्कारयेन्मातस्तिलजान्दश पार्वति}

\twolineshloka*
{देवाग्रे स्थापयेत्पञ्च पञ्च विप्राय कल्पयेत्}
{पूजयित्वा तु तं विप्रं भक्तिभावेन देववत्}

\twolineshloka*
{दक्षिणां च यथाशक्त्या दत्त्वा वै पञ्चमोदकान्}
{पूजयेन्निशि चन्द्रं च अर्घ्यं दत्त्वा यथाविधि}


\dnsub{अर्घ्यम्}
\resetShloka
ममोपात्त-समस्त-दुरित-क्षयद्वारा श्री-सिद्धिविनायक-प्रीत्यर्थं श्री-सङ्कष्टचतुर्थीव्रत-पूजान्ते क्षीरार्घ्यप्रदानं करिष्ये॥

(हस्ते साक्षतपुष्पं क्षीरं गृहीत्वा)

\twolineshloka
{क्षीरसागरसम्भूत सुधारूप निशाकर}
{गृहाणार्घ्यं मया दत्तं गणेशप्रीतिवर्धनम्}
\textbf{रोहिणीसहितचन्द्रमसे नमः, इदमर्घ्यम् इदमर्घ्यम् इदमर्घ्यम्।}

\twolineshloka
{क्षीरोदार्णवसम्भूत सुधारूप निशाकर}
{गृहाणार्घ्यं मया दत्तं रोहिण्या सहितः शशिन्}
\textbf{रोहिणीसहितचन्द्रमसे नमः, इदमर्घ्यम् इदमर्घ्यम् इदमर्घ्यम्।}

\twolineshloka
{गणेशाय नमस्तुभ्यं सर्वसिद्धिप्रदायक}
{सङ्कष्टं हर मे देव गृहाणार्घ्यं नमोऽस्तु ते}
\textbf{सङ्कष्टहरगणेशाय नमः, इदमर्घ्यम् इदमर्घ्यम् इदमर्घ्यम्।}

\twolineshloka
{कृष्णपक्षे चतुर्थ्यां तु पूजितोऽसि विधूदये}
{क्षिप्रं प्रसादितो देव गृहाणार्घ्यं नमोऽस्तु ते}
\textbf{सङ्कष्टहरगणेशाय नमः, इदमर्घ्यम् इदमर्घ्यम् इदमर्घ्यम्।}

\twolineshloka
{तिथीनामुत्तमे देवि गणेशप्रियवल्लभे}
{सर्वसङ्कष्टनाशाय चतुर्थ्यर्घ्यं नमोऽस्तु ते}
\textbf{चतुर्थ्यै नमः, इदमर्घ्यम् इदमर्घ्यम् इदमर्घ्यम्।}


\dnsub{उपायन-दानम्}
\centerline{\textbf{आचार्य पूजा}}

अद्यपूर्वोक्त-एवङ्गुण-विशेषण-विशिष्टायामस्यां चतुर्थ्यां शुभतिथौ श्री-सिद्धिविनायक-पूजा-फलसिद्ध्यर्थं ब्राह्मणपूजाम् उपायन-दानं च करिष्ये॥ 
श्री-महागणपति-स्वरूपस्य ब्राह्मणस्य इदमासनम्। गन्धादि-सकलाराधनैः स्वर्चितम्॥

\twolineshloka*
{[अथैकविंशतिं गृह्य मोदकान् घृतपाचितान्}
{स्थापयित्वा गणाध्यक्षसमीपे कुरुनन्दन}

\twolineshloka*
{दश विप्राय दातव्याः स्थापयेद् दश आत्मनि}
{एकं गणाधिपे दद्यात् सघृतं मोदकं शुभम्]}

\centerline{\textbf{वायनमन्त्रः}}

\threelineshloka*
{विप्रवर्य नमस्तुभ्यं मोदकान्वै ददाम्यहम्}
{मोदकान्सफलान्पञ्च दक्षिणाभिः समन्वितान्}
{आपदुद्धरणार्थाय गृहाण द्विजसत्तम}

\centerline{\textbf{प्रार्थना}}

\twolineshloka*
{अबुद्धमतिरिक्तं वा द्रव्यहीनं मया कृतम्}
{सत्सर्वं पूर्णतां यातु विप्ररूप गणेश्वर}

\twolineshloka*
{ब्राह्मणान् भोजयेद्देवि यथान्नेन यथासुखम्}
{स्वयं भुञ्जीत पञ्चैव मोदकान्फलसंयुतान्}

\twolineshloka*
{अशक्तश्चैकमन्नं वा भुञ्जीत दधिसंयुतम्}
{अथवा भोजनं कार्यमेकवारं हिमागजे}

\twolineshloka*
{प्रतिमां गुरवे दद्यादाचार्याय सदक्षिणाम्}
{वस्त्रकुम्भसमायुक्तामादौ मन्त्रमिमं जपेत्}

ॐ नमो हेरम्ब मदमोहित सङ्कष्टान्निवारय निवारय।\\
इति मूलमन्त्रमेकविंशतिवारं जपेत्।
\medskip


\twolineshloka*
{हिरण्यगर्भगर्भस्थं हेमबीजं विभावसोः}
{अनन्तपुण्यफलदम् अतः शान्तिं प्रयच्छ मे}

श्री-सङ्कष्टचतुर्थीव्रत-पूजा-पुण्यकाले अस्मिन् मया क्रियमाण-श्री-सिद्धिविनायक-पूजायां
यद्देयमुपायनदानं तत्प्रत्याम्नायार्थं हिरण्यं श्री-सिद्धिविनायक-प्रीतिं
कामयमानः मनसोद्दिष्टाय ब्राह्मणाय सम्प्रददे नमः न मम।

\textbf{विसर्जनमन्त्रः}
\twolineshloka*
{गच्छ गच्छ सुरश्रेष्ठ स्वस्थाने त्वं गणेश्वर}
{व्रतेनानेन देवेश यथोक्तफलदो भव}

अनया पूजया श्री-सिद्धिविनायकः प्रीयताम्। 

\sect{अपराध-क्षमापनम्}

\twolineshloka*
{यस्य स्मृत्या च नामोक्त्या तपः पूजाक्रियादिषु} 
{न्यूनं सम्पूर्णतां याति सद्यो वन्दे गजाननम्} 

अनया पूजया श्री-सिद्धिविनायकः प्रीयताम्॥ 

\fourlineindentedshloka*
{कायेन वाचा मनसेन्द्रियैर्वा}
{बुद्‌ध्याऽऽत्मना वा प्रकृतेः स्वभावात्}
{करोमि यद्यत् सकलं परस्मै}
{नारायणायेति समर्पयामि}

ॐ तत्सद्ब्रह्मार्पणमस्तु।

\end{center}

\ifbool{katha}{\sect{सङ्कष्ट-चतुर्थी-व्रत-कथा}
\centerline{\small{(मूलम्—श्री-व्रतराजः)}}

\uvacha{ऋषय ऊचुः}

\twolineshloka
{दारिद्र्यशोककष्टाद्यैः पीडितानां च वैरिभिः}
{राज्यभ्रष्टैर्नृपैः सर्वैः क्रियते किं शुभार्थिभिः} %१


\twolineshloka
{धनहीनैर्नरैः स्कन्द सर्वोपद्रवपीडितैः}
{विद्यापुत्रगृहभ्रष्टै रोगयुक्तैः शुभार्थिभिः} %२

\onelineshloka*
{कर्तव्यं किं वदोपायं पुनः क्षेमार्थसिद्धये}

\uvacha{स्कन्द उवाच}
\onelineshloka
{शृणुध्वं मुनयः सर्वे व्रतानामुत्तमं व्रतम्} %३


\twolineshloka
{सङ्कष्टतरणं नामामुत्रेह सुखदायकम्}
{येनोपायेन सङ्कष्टं तरन्ति भुवि देहिनः} %४


\twolineshloka
{यद्व्रतं देवकीपुत्रः कृष्णो धर्माय दत्तवान्}
{अरण्ये क्लिश्यमानाय पुनः क्षेमार्थसिद्धये} %५


\twolineshloka
{यथा कथितवान् पूर्वं गणेशो मातरं प्रति}
{तथा कथितवाञ्छ्रीशो द्वापरे पाण्डवान्प्रति} %६

\uvacha{ऋषय ऊचुः}

\twolineshloka
{कथं कथितवानम्बां पार्वतीं श्रीगणेश्वरः}
{यथा पृच्छन्ति मुनयो लोकानुग्रहकाङ्क्षिणः} %७

\uvacha{स्कन्द उवाच}

\twolineshloka
{पुरा कृतयुगे पुण्ये हिमाचलसुता सती}
{तपस्तप्तवती भूरि तेनालब्धः शिवः पतिः\footnotemark}\footnotetext{न दृष्टः शङ्करः पतिरित्यपि पाठः} %८


\twolineshloka
{तदाऽस्मरत्सा हेरम्बं गणेशं पूर्वजं सुतम्}
{तत्क्षणादागतं दृष्ट्वा गणेशं परिपृच्छति} %९

\uvacha{पार्वत्युवाच}

\twolineshloka
{तपस्तप्तं मया घोरं दुश्चरं लोमहर्षणम्}
{न प्राप्तः स मया कान्तो गिरीशो मम वल्लभः} %१०


\twolineshloka
{सङ्कष्टतरणं दिव्यं व्रतं नारद उक्तवान्}
{त्वदीयं यद्व्रतं तावत् कथयस्व पुरातनम्} %११


\twolineshloka
{तच्छ्रुत्वा पार्वतीवाक्यं सङ्कष्टतरणं व्रतम्}
{प्रीत्या कथितवान् देवो गणेशो ज्ञानसिद्धिदः} %१२

\uvacha{गणेश उवाच}

\twolineshloka
{श्रावणे बहुले पक्षे चतुर्थ्यां तु विधूदये}
{गणेशं पूजयित्वा तु चन्द्रायार्घ्यं प्रदापयेत्} %१३

\uvacha{पार्वत्युवाच}

\twolineshloka
{क्रियते केन विधिना किं कार्य किं च पूजनम्}
{उद्यापनं कदा कार्यं मन्त्राः के स्युस्तु पूजने} %१४

\onelineshloka*
{किं ध्यानं श्रीगणेशस्य गणेश वद विस्तरात्}

\uvacha{गणेश उवाच}
\onelineshloka
{चतुर्थ्यां प्रातरुत्थाय दन्तधावनपूर्वकम्} %१५


\twolineshloka
{ग्राह्यं व्रतमिदं पुण्यं सङ्कष्टतरणं शुभम्}
{कर्तव्यमिति सङ्कल्प्य व्रतेऽस्मिन् गणपं स्मरेत्} %१६


स्वीकारमन्त्रः—\hfill 
\twolineshloka
{निराहारोऽस्मि देवेश यावच्चन्द्रोदये भवेत्}
{भोक्ष्यामि पूजयित्वाऽहं सङ्कष्टात्तारयस्व माम्} %१७


\twolineshloka
{एवं सङ्कल्प्य राजेन्द्र स्नात्वा कृष्णतिलैः शुभैः}
{आह्निकं तु विधायैव पश्चात्पूज्यो गणाधिपः} %१८


\twolineshloka
{त्रिभिर्माषैस्तदर्धेन तृतीयांशेन वा पुनः}
{यथाशक्त्या तु वा हैमी प्रतिमा क्रियते मम} %१९


\twolineshloka
{हेमाभावे तु रौप्यस्य ताम्रस्यापि यथासुखम्}
{सर्वथैव दरिद्रेण क्रियते मृन्मयी शुभा} %२०


\twolineshloka
{वित्तशाठ्यं न कर्तव्यं कृते कार्यं विनश्यति}
{जलपूर्णं वस्त्रयुतं कुम्भं तदुपरि न्यसेत्} %२१


\twolineshloka
{पूर्णपात्रं तत्र पद्मं लिखेदष्टदलं शुभम्}
{देवतां तत्र संस्थाप्य गन्धपुष्पैः प्रपूजयेत्} %२२


\twolineshloka
{एवं व्रतं प्रकर्तव्यं प्रतिमासं त्वयाऽद्रिजे}
{यावज्जीवं तु वा वर्षाण्येकविंशतिमेव वा} %२३


\twolineshloka
{अशक्तोऽप्येकवर्षं वा प्रतिवर्षमथापि वा}
{उद्यापनं तु कर्तव्यं चतुर्थ्यां श्रावणेऽसिते} %२४


\twolineshloka
{स्वीकारश्च तथा कार्यः सङ्कष्टहरणे तिथौ}
{गाणपत्यं तथाऽऽचार्यं सर्वशास्त्रविशारदम्} %२५


\twolineshloka
{श्रद्धया प्रार्थयेदादौ तेनोक्तं विधिमाचरेत्}
{एकविंशतिविप्रांश्च वस्त्रालङ्कारभूषणैः} %२६


\twolineshloka
{पूजयेद्गोहिरण्याद्यैर्हुत्वाऽग्नौ विधिपूर्वकम्}
{होमद्रव्यं मोदकाश्च तिलयुक्ता घृतप्लुताः} %२७


\twolineshloka
{अष्टोत्तरसहस्रं वा नोचेदष्टोत्तरं शतम्}
{अष्टाविंशतिसङ्ख्याकान्मोदकान्वा सशर्करान्} %२८


\twolineshloka
{अशक्तोऽष्टौ शुभान् स्थूलाञ्जुहुयाज्जातवेदसि}
{वैदिकेन च मन्त्रेण आगमोक्तेन वा तथा} %२९


\twolineshloka
{अथवा नाममन्त्रेण होमं कुर्याद्यथाविधि}
{पुष्पमण्डपिका कार्या गणेशाह्लादकारिणी} %३०


\twolineshloka
{पूजयेत्तत्र गणपं भक्तसङ्कष्टनाशनम्}
{गीतवादित्रनिनदैर्भक्तिभावपुरस्कृतैः} %३१


\twolineshloka
{पुराणवेदनिर्घोषैस्तोषयेच्च गणेश्वरम्}
{एवं जागरणं कार्यं शक्त्या दानादिकं तथा} %३२


\twolineshloka
{सपत्नीकमथाऽऽचार्यं तोषयेद्वस्त्रभूषणैः}
{उपानच्छत्रगोदानकमण्डलुफलादिभिः} %३३


\twolineshloka
{शय्यावाहनभूदानं धनधान्यगृहादिभिः}
{यथाशक्त्या प्रकर्तव्यं दारिद्र्याभावमिच्छता} %३४


\twolineshloka
{एकविंशतिविप्रांश्च भोजयेन्नामभिर्मम}
{गजास्यो विघ्नराजश्च लम्बोदरशिवात्मजौ} %३५


\twolineshloka
{वक्रतुण्डः शूर्पकर्णः कुब्जश्चैव विनायकः}
{विघ्ननाशो हि विकटो वामनः सर्वदैवतः} %३६


\twolineshloka
{सर्वार्तिनाशी भगवान् विघ्नहर्ता च धूम्रकः}
{सर्वदेवाधिदेवश्च सर्वे षोडश वै स्मृताः} %३७


\twolineshloka
{एकदन्तः कृष्णपिङ्गो भालचन्द्रो गणेश्वरः}
{गणपश्चैकविंशश्च सर्व एते गणेश्वराः} %३८

\twolineshloka
{दुर्गोपेन्द्रश्च रुद्रश्च कुलदेव्याधिकं भवेत्}
{विशेषेणाष्टसङ्ख्याकैर्मोदकैर्हवनं स्मृतम्} %३९


\twolineshloka
{एवं कृते विधानेन प्रसन्नोऽहं न संशयः}
{ददामि वाञ्छितान् कामांस्तद्व्रतं मत्प्रियं कुरु} %४०

\uvacha{श्रीकृष्ण उवाच}

\twolineshloka
{एवं तु कथितं सर्वं गणेशेन स्वयं नृप}
{पार्वत्या तत्कृतं राजन् व्रतं सङ्कष्टनाशनम्} %४१


\twolineshloka
{व्रतेनानेन सा प्राप महादेवं पतिं स्वकम्}
{तत्कुरुष्व महाराज व्रतं सङ्कष्टनाशनम्} %४२


\twolineshloka
{चतुर्थी सङ्कटा नाम स्कन्देन कथिता ऋषीन्}
{ऋषिभिर्लोककामैस्तैर्लोके ततमिदं व्रतम्} %४३

\uvacha{सूत उवाच}

\twolineshloka
{कृतं युधिष्ठिरेणैतद्राज्यकामेन वै द्विज}
{तेन शत्रून्निहत्याऽऽजौ स्वराज्यं प्राप्तवान् स्वयम्} %४४


\twolineshloka
{तस्मात्सर्वप्रयत्नेन व्रतं कार्यं विचक्षणैः}
{येन धर्मार्थकामाश्च मोक्षश्चापि भवेत्किल} %४५


\twolineshloka
{यः करोति व्रतं विप्राः सर्वकामार्थसिद्धिदम्}
{स वाञ्छितफलं प्राप्य पश्चाद्गणपतां व्रजेत्} %४६


\twolineshloka
{यदा यदा परं विप्रा नरः प्राप्नोति सङ्कटम्}
{तदा तदा प्रकर्तव्यं व्रतं सङ्कष्टनाशनम्} %४७


\twolineshloka
{त्रिपुरं हन्तुकामेन कृतं देवेन शूलिना}
{त्रैलोक्यभूतिकामेन महेन्द्रेण तथा कृतम्} %४८


\twolineshloka
{रावणेन कृतं पूर्वं वालिबन्धनसङ्कटे}
{स्वकीयं प्राप्तवान्राज्यं गणेशस्य प्रसादतः} %४९


\twolineshloka
{सीतान्वेषणकामेन कृतं वायुसुतेन च}
{सङ्कल्प्य दृष्टवान्सोऽयं सीतां रामप्रियां पुरा} %५०


\threelineshloka
{दमयन्त्या कृतं पूर्वं नलान्वेषणकारणात्}
{सा पतिं नैषधं लेभे पुण्यश्लोकं द्विजोत्तमाः}
{अहल्याऽपि पतिं लेभे गौतमं प्राणवल्लभम्} %५१

\twolineshloka
{विद्यार्थी लभते विद्यां धनार्थी धनमाप्नुयात्}
{पुत्रार्थी पुत्रमाप्नोति रोगी रोगात्प्रमुच्यते} %५२

\centerline{॥इति श्रीस्कन्दपुराणोक्तं सङ्कष्टचतुर्थीव्रतम्॥} %


}{}
