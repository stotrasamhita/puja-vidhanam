% !TeX program = XeLaTeX
% !TeX root = ../pujavidhanam.tex

\setlength{\parindent}{0pt}
\chapt{श्री-स्कन्द-षष्ठी-पूजा}

\dnsub{पूर्वाङ्ग-विघ्नेश्वर-पूजा}

\graphicspath{{purvanga/}{../purvanga/}}

\centerline{\includegraphics[width=1cm]{ganesha.pdf}}

(आचम्य)

\twolineshloka*
{शुक्लाम्बरधरं विष्णुं शशिवर्णं चतुर्भुजम्}
{प्रसन्नवदनं ध्यायेत् सर्वविघ्नोपशान्तये}
 
प्राणान्  आयम्य।

(अप उपस्पृश्य, पुष्पाक्षतान् गृहीत्वा)\\

\twolineshloka*
{तदेव लग्नं सुदिनं तदेव ताराबलं चन्द्रबलं तदेव}
{विद्याबलं दैवबलं तदेव लक्ष्मीपतेरङ्घ्रियुगं स्मरामि}
 
ममोपात्त-समस्त-दुरित-क्षयद्वारा \\
श्री-परमेश्वर-प्रीत्यर्थं करिष्यमाणस्य कर्मणः\\
अविघ्नेन परिसमाप्त्यर्थम् आदौ विघ्नेश्वरपूजां करिष्ये।

(अप उपस्पृश्य)

\ifbool{veda}{
\twolineshloka*
{ॐ ग॒णानां᳚ त्वा ग॒णप॑तिꣳ हवामहे क॒विं क॑वी॒नामु॑प॒मश्र॑वस्तमम्}
{ज्ये॒ष्ठ॒राजं॒ ब्रह्म॑णां ब्रह्मणस्पत॒ आ नः॑ शृ॒ण्वन्नू॒तिभिः॑ सीद॒ साद॑नम्}
}{}

\twolineshloka*
{अगजानानपद्मार्कं गजाननमहर्निशम्}
{अनेकदं तं भक्तानाम् एकदन्तमुपास्महे}

\ifbool{veda}{भूर्भुवः॒ सुव॒रोम्।}{} अस्मिन् हरिद्राबिम्बे सुमुखं महागणपतिं ध्यायामि, आवाहयामि।\\

\renewcommand{\devAya}{\OMshri महागणपतये}

\devAya{} नमः, आसनं समर्पयामि।\\
\devAya{} नमः, पादयोः पाद्यं समर्पयामि।\\
\devAya{} नमः, अर्घ्यं समर्पयामि।\\
\devAya{} नमः, आचमनीयं समर्पयामि।\\
\devAya{} नमः, मधुपर्कं समर्पयामि।\\
\ifbool{veda}{ॐ भूर्भुवः॒ सुवः॑।}{}
\devAya{} नमः, शुद्धोदकस्नानं समर्पयामि। स्नानानन्तरमाचमनीयं समर्पयामि।\\
\devAya{} नमः, वस्त्रार्थमक्षतान् समर्पयामि।\\
\devAya{} नमः, यज्ञोपवीताभरणार्थे अक्षतान् समर्पयामि।\\
\devAya{} नमः, दिव्यपरिमलगन्धान् धारयामि। गन्धस्योपरि हरिद्राकुङ्कुमं समर्पयामि। \\
\devAya{} नमः, अक्षतान् समर्पयामि। \\
\devAya{} नमः, पुष्पमालिकां समर्पयामि। पुष्पैः पूजयामि।

\dnsub{अर्चना}
\begin{enumerate}
\begin{minipage}{0.475\linewidth}   
  \item सुमुखाय नमः
  \item एकदन्ताय नमः
  \item कपिलाय नमः
  \item गजकर्णकाय नमः
  \item लम्बोदराय नमः
  \item विकटाय नमः
  \item विघ्नराजाय नमः
  \item विनायकाय नमः
\end{minipage}
\begin{minipage}{0.525\linewidth}
  \item धूमकेतवे नमः
  \item गणाध्यक्षाय नमः
  \item फालचन्द्राय नमः
  \item गजाननाय नमः
  \item वक्रतुण्डाय नमः
  \item शूर्पकर्णाय नमः
  \item हेरम्बाय नमः
  \item स्कन्दपूर्वजाय नमः
\end{minipage}
\end{enumerate}
\devAya{} नमः, नानाविधपरिमलपत्रपुष्पाणि समर्पयामि॥\\
\devAya{} नमः, धूपमाघ्रापयामि।\\
अलङ्कारदीपं सन्दर्शयामि।\\
% नैवेद्यम्।\\
\ifbool{veda}{ॐ भूर्भुवः॒ सुवः॑। + ब्र॒ह्मणे॒ स्वाहा᳚।}{}
\devAya{} नमः, \blank{} (नैवेद्यं) निवेदयामि। 
\ifbool{veda}{अ॒मृ॒ता॒पि॒धा॒नम॑सि॥}{}
निवेदनान्तरम् आचमनीयं समर्पयामि।\\
\devAya{} नमः, ताम्बूलं समर्पयामि।\\
\devAya{} नमः, कर्पूरनीराजनं दर्शयामि। कर्पूरनीराजनानन्तरमाचमनीयं समर्पयामि।\\
\devAya{} नमः, मन्त्रपुष्पं समर्पयामि। स्वर्णपुष्पं समर्पयामि।\\

\twolineshloka*
{अभीप्सितार्थसिद्ध्यर्थं पूजितो यः सुरैरपि}
{सर्वविघ्नच्छिदे तस्मै गणाधिपतये नमः}

\twolineshloka*
{गजाननं भूतगणादिसेवितं कपित्थ-जम्बूफल-सार-भक्षितम्}
{उमासुतं शोकविनाशकारणं नमामि विघ्नेश्वरपादपङ्कजम्}

\centerline{अनन्तकोटिप्रदक्षिणनमस्कारान् समर्पयामि।}

\centerline{छत्त्रचामरादिसमस्तोपचारान् समर्पयामि।}

\twolineshloka*
{वक्रतुण्डमहाकाय कोटिसूर्यसमप्रभ}
{अविघ्नं कुरु मे देव सर्वकार्येषु सर्वदा}

\twolineshloka*
{सुमुखश्चैकदन्तश्च कपिलो गजकर्णकः}
{लम्बोदरश्च विकटो विघ्नराजो गणाधिपः}

\twolineshloka*
{धूमकेतुर्गणाध्यक्षो फालचन्द्रो गजाननः}
{वक्रतुण्डः शूर्पकर्णो हेरम्भः स्कन्दपूर्वजः}

\threelineshloka*
{षोडशैतानि नामानि यः पठेच्छृणुयादपि}
{विद्यारम्भे विवाहे च प्रवेशे निर्गमे तथा}
{सङ्ग्रामे सर्वकार्येषु विघ्नस्तस्य न जायते}

\centerline{प्रार्थनाः समर्पयामि।}

\closesub

\sect{प्रधान-पूजा — स्कन्द-पूजा}

\twolineshloka*
{शुक्लाम्बरधरं विष्णुं शशिवर्णं चतुर्भुजम्}
{प्रसन्नवदनं ध्यायेत् सर्वविघ्नोपशान्तये}

प्राणान् आयम्य। ॐ भूः + भूर्भुवः॒ सुव॒रोम्।


\dnsub{सङ्कल्पः}

ममोपात्तसमस्तदुरितक्षयद्वारा श्रीपरमेश्वरप्रीत्यर्थं शुभे शोभने मुहूत्ते अद्यब्रह्मणः
द्वितीयपरार्द्धे श्वेतवराहकल्पे वैवस्वतमन्वन्तरे अष्टाविंशतितमे कलियुगे प्रथमे पादे
जम्बूद्वीपे भारतवर्षे भरतखण्डे मेरोः दक्षिणेपार्श्वे शकाब्दे अस्मिन् वर्तमाने व्यावहारिके
 प्रभवादि षष्टिसंवत्सराणां मध्ये (  )\see{app:samvatsara_names} नाम संवत्सरे दक्षिनायने 
शरद्-ऋतौ  तुला-मासे शुक्लपक्षे षष्ठ्यां शुभतिथौ
(इन्दु / भौम / बुध / गुरु / भृगु / स्थिर / भानु) वासरयुक्तायाम्
(  )\see{app:nakshatra_names} नक्षत्र (  )\see{app:yoga_names} नाम  योग  (  ) करण युक्तायां च एवं गुण विशेषण विशिष्टायाम् अस्यां षष्ठ्यां शुभतिथौ 

श्री-वल्ली-देवसेना-समेत-सुब्रह्मण्य-प्रीत्यर्थं प्रसाद-सिद्ध्यर्थम्
अस्माकं सहकुटुम्बानां क्षेमस्थैर्य-धैर्य-वीर्य-विजय-आयुरारोग्य-ऐश्वर्याणाम् अभिवृद्ध्यर्थं
धर्मार्थ\-काम\-मोक्ष\-चतुर्विध\-फल\-पुरुषार्थ\-सिद्ध्यर्थं पुत्र\-पौत्रा\-भि\-वृद्ध्यर्थम् इष्ट\-काम्यार्थ\-सिद्ध्यर्थं
मम इहजन्मनि पूर्वजन्मनि जन्मान्तरे च सम्पादितानां ज्ञानाज्ञानकृतमहा\-पातकचतुष्टय-व्यतिरिक्तानां रहस्यकृतानां प्रकाशकृतानां सर्वेषां पापानां सद्य अपनोदनद्वारा 
सकल-पाप\-क्षयार्थं गो-भू-धन-धान्य-पुत्र-पौत्रादि अनविच्छिन्न-सन्तति स्थिर-लक्ष्मी-कीर्ति-लाभ शत्रु-पराजयादि सदभीष्ट-सिद्ध्यर्थं दिव्यज्ञान-सिद्ध्यर्थं

यावच्छक्ति-ध्यानावाहनादि षोडशोपचारैः कल्पोक्त-प्रकारेण श्री-वल्ली-देवसेना-समेत-सुब्रह्मण्य-पूजाराधनं करिष्ये। तदङ्गं कलश\-पूजां च करिष्ये।


श्रीविघ्नेश्वराय नमः, यथास्थानं प्रतिष्ठापयामि।\\
(गणपति प्रसादं शिरसा गृहीत्वा)


\dnsub{घण्टापूजा}
\twolineshloka*
{आगमार्थं तु देवानां गमनार्थं तु रक्षसाम्}
{कुरु घण्टारवं तत्र देवताऽऽह्वानलाञ्चनम्}


\dnsub{कलशपूजा}
(कलशं गन्धपुष्पाक्षतैः अभ्यर्च्य)

गङ्गायै नमः। यमुनायै नमः। गोदावर्यै नमः। सरस्वत्यै नमः। नर्मदायै नमः। सिन्धवे नमः। कावेर्यै नमः।\\
सप्तकोटिमहातीर्थान्यावाहयामि। \\

(अथ कलशं स्पृष्ट्वा जपं कुर्यात्।)

\twolineshloka*
{कलशस्य मुखे विष्णुः कण्ठे रुद्रः समाश्रितः}
{मूले तत्र स्थितो ब्रह्मा मध्ये मातृगणाः स्मृताः}

\threelineshloka*
{कुक्षौ तु सागराः सर्वे सप्तद्वीपा वसुन्धरा}
{ऋग्वेदोऽथ यजुर्वेदः सामवेदोऽप्यथर्वणः}
{अङ्गैश्च सहिताः सर्वे कलशाम्बुसमाश्रिताः}

\twolineshloka*
{गङ्गे च यमुने चैव गोदावरि सरस्वति}
{नर्मदे सिन्धुकावेरि जलेऽस्मिन् सन्निधिं कुरु}

\twolineshloka*
{सर्वे समुद्राः सरितः तीर्थानि च ह्रदा नदाः}
{आयान्तु विष्णुपूजार्थं दुरितक्षयकारकाः}

% \centerline{ॐ भूर्भुवः॒ सुवो॒ भूर्भुवः॒ सुवो॒ भूर्भुवः॒ सुवः॑।}

(इति कलशजलेन सर्वोपकरणानि आत्मानं च प्रोक्ष्य।)

\dnsub{आत्मपूजा}
आत्मने नमः, दिव्यगन्धान् धारयामि। 


\dnsub{मण्टप-पूजा}

ॐ ह्रीं श्रीं मण्डूकादि-परतत्त्वात्म-पर्यन्त-पीठ-शक्ति-देवताभ्यो नमः।\\
ॐ ह्रीं श्रीं शं शकुन्यै नमः।\\
ॐ ह्रीं श्रीं रें रेवत्यै नमः।\\
ॐ ह्रीं श्रीं पूं पूताय नमः।\\
ॐ ह्रीं श्रीं मं महापूतायै नमः।\\
ॐ ह्रीं श्रीं निं निशीथिन्यै नमः।\\
ॐ ह्रीं श्रीं मां मालिन्यै नमः।\\
ॐ ह्रीं श्रीं शीं शीतलायै नमः।\\
ॐ ह्रीं श्रीं शुं शुद्धायै नमः।\\
ॐ ह्रीं श्रीं विं विश्वतोमुख्यै नमः।\\

% \dnsub{दीप-पूजा}

% \twolineshloka*
% {दीपदेवि महादेवि शुभं भवतु मे सदा}
% {यावत्पूजासमाप्तिः स्यात् तावत् प्रज्वल सुस्थिर}


\section{षोडशोपचारपूजा}

\centering

\fourlineindentedshloka*
{सिन्धूरारुणमिन्दुकान्तिवदनं केयूरहारादिभिः}
{दिव्यैराभरणैर्विभूषिततनुं स्वर्गादिसौख्यप्रदम्}
{अम्भोजाभयशक्तिकुक्कुटधरं रक्ताङ्गराकोज्ज्वलं}
{सुब्रह्मण्यमुपास्महे प्रणमतां भीतिप्रणाशोद्यतम्}
\nobreak%\hfill{}
अस्मिन् कुम्भे सपरिवारं\\
श्री-वल्ली-देवसेना-समेत-सुब्रह्मण्य-स्वामिनम् ध्यायामि।

\fourlineindentedshloka*
{षड्वक्त्रं शिखिवाहनं त्रिनयनं चित्राम्बरालङ्कृतम्}
{वज्रं शक्तिमसिं त्रिशूलमभयं खेटं धनुश्चक्रकम्}
{पाशं कुक्कुटमङ्कुशं च वरदं दोर्भिर्दधानं सदा}
{ध्यायेदीप्सितसिद्धिदं शिवसुतं स्कन्दं सुराराधितम्}

\nobreak%\hfill{}
अस्मिन् कुम्भे सपरिवारं\\
श्री-वल्ली-देवसेना-समेत-सुब्रह्मण्य-स्वामिनम् आवाहयामि। 

आवाहिता भव। संस्थापिता भव।\\
सन्निहिता भव। सन्निरुद्धा भव।\\
अवकुण्ठिता भव। सुप्रीता भव।\\
सुप्रसन्ना भव। वरदा भव।\\

\twolineshloka*
{स्वामिन् सर्वजगन्नाथ यावत्पूजावसानकम्}
{तावत् त्वं प्रीतिभावेन दीपेऽस्मिन् सन्निधिं कुरु}

\twolineshloka*
{देवदेवे महाराज प्रियेश्वर प्रजापते}
{आसनं दिव्यमीशान दास्येयं परमेश्वर}
\nobreak%\hfill{}
श्री-वल्ली-देवसेना-समेत-सुब्रह्मण्य-स्वामिने
नमः आसनं समर्पयामि।

\twolineshloka*
{यद्भक्तिलेशसम्पर्कात् परमानन्दविग्रह}
{तस्मै ते शरणाब्जाय पाद्यं शुद्धाय कल्पये}
\nobreak%\hfill{}
श्री-वल्ली-देवसेना-समेत-सुब्रह्मण्य-स्वामिने
नमः पाद्यं समर्पयामि।

\twolineshloka*
{तापत्रयहरं दिव्यं परमानन्दलक्षणम्}
{तापत्रयविनिर्मुक्तं तवार्घ्यं कल्पयाम्यहम्}
\nobreak%\hfill{}
 श्री-वल्ली-देवसेना-समेत-सुब्रह्मण्य-स्वामिने
 नमः अर्घ्यं समर्पयामि।

\twolineshloka*
{वेदानामपि वेद्याय देवानां देवतात्मने}
{आचामं कल्पयामीश शुद्धानां शुद्धिहेतवे}
\nobreak%\hfill{}
श्री-वल्ली-देवसेना-समेत-सुब्रह्मण्य-स्वामिने
नमः आचमनीयं समर्पयामि।

\twolineshloka*
{तरुपुष्पसमुद्भूतं सुस्वादु मधुरं मधु}
{तेजःपुष्टिकरं दिव्यं प्रतिगृह्णीष्व देवेश}
\nobreak%\hfill{}
श्री-वल्ली-देवसेना-समेत-स्सुब्रह्मण्य-स्वामिने
नमः मधुपर्कं समर्पयामि।

\twolineshloka*
{पयोदधिघृतं चैव मधु च शर्करायुतम्}
{पञ्चामृतं मयाऽऽनीतं स्नानार्थं प्रतिगृह्यताम्}
\nobreak%\hfill{}
श्री-वल्ली-देवसेना-समेत-सुब्रह्मण्य-स्वामिने
नमः पञ्चामृत-स्नानं समर्पयामि।

\twolineshloka*
{कामधेनुसमुत्पन्नं सर्वेषां जीवनं परम्}
{पावनं यज्ञहेतुश्च पयः स्नानार्थमर्पितम्}
\nobreak%\hfill{}
श्री-वल्ली-देवसेना-समेत-सुब्रह्मण्य-स्वामिने
नमः क्षीरस्नानं समर्पयामि।

\twolineshloka*
{भागीरथी यमुना चैव गौतमी च सरस्वती}
{तासां सुसलिलमादाय करोमि त्वामभिषेचनम्}
\nobreak%\hfill{}
श्री-वल्ली-देवसेना-समेत-सुब्रह्मण्य-स्वामिने
नमः स्नानं समर्पयामि। स्नानान्तरमाचमनीयं समर्पयामि।

\twolineshloka*
{सर्वभूषाधिके सौम्ये लोकलज्जानिवारणे}
{मयोपपादिते तुभ्यं वाससी प्रतिगृह्यताम्}
\nobreak%\hfill{}
श्री-वल्ली-देवसेना-समेत-सुब्रह्मण्य-स्वामिने
नमः वस्त्रं समर्पयामि।

\twolineshloka*
{नवभिस्तन्तुभिर्युक्तं त्रिगुणं देवतात्मकम्}
{उपवीतं प्रदास्यामि गृहाण परमेश्वर}
\nobreak%\hfill{}
श्री-वल्ली-देवसेना-समेत-सुब्रह्मण्य-स्वामिने
नमः यज्ञोपवीतं समर्पयामि।

\twolineshloka*
{मुक्ता-माणिक्य-वैडूर्य-रत्न-हेमादि-निर्मितम्}
{नानाभरणं दास्यामि स्वीकुरुष्व दयानिधे}
\nobreak%\hfill{}
श्री-वल्ली-देवसेना-समेत-सुब्रह्मण्य-स्वामिने
नमः नवमणि-मकुटादि नानाभरणम् समर्पयामि।

\twolineshloka*
{चन्दनागरुकर्पूरकस्तूरीकुङ्कुमान्वितम्}
{विलेपनं सुरश्रेष्ठ प्रीत्यर्थं प्रतिगृह्यताम्}
\nobreak%\hfill{}
श्री-वल्ली-देवसेना-समेत-सुब्रह्मण्य-स्वामिने
नमः गन्धान् धारयामि। गन्धस्योपरि हरिद्रा-कुङ्कुमं समर्पयामि।


\twolineshloka*
{अक्षतांश्च सुरश्रेष्ठ कुङ्कुमाक्ता सुशोभिताः}
{मया निवेदिता भक्त्या गृह्यतां परमेश्वर}
\nobreak%\hfill{}
श्री-वल्ली-देवसेना-समेत-सुब्रह्मण्य-स्वामिने
नमः अक्षतान् समर्पयामि।

\twolineshloka*
{मन्दार-पारिजाताब्ज-केतक्युत्पल-पाटलैः}
{मल्लिका-जाति-वकुलैः पुष्पैस्त्वां पूजयाम्यहम्}
\nobreak%\hfill{}
श्री-वल्ली-देवसेना-समेत-सुब्रह्मण्य-स्वामिने
नमः मल्लिकादि-सर्वर्तु-पुष्पमालाः समर्पयामि।

\dnsub{अङ्गपूजा}

\begin{tabular}{lll}
शरवणोद्भूताय नमः & - पादौ पूजयामि।\\
रौद्रेयाय नमः &- जङ्घे पूजयामि।\\
सहस्रपदे नमः &- जानुनी पूजयामि।\\
भयनाशनाय नमः & - ऊरू पूजयामि।\\
बालग्रहाच्छाटनाय नमः & - मेढ्रं पूजयामि \\
भक्तपालनाय नमः & - गुह्यं पूजयामि।\\
गुणनिधये नमः & - कटिं पूजयामि।\\
महनीयाय नमः & - नाभिं पूजयामि।\\
सर्वाभीष्टप्रदाय नमः & - हृदयं पूजयामि।\\
विशालवक्षसे नमः & - वक्षस्थलं पूजयामि।\\
शक्तिधराय नमः & - हस्तान् पूजयामि।\\
अभयप्रदानाय नमः & - बाहून् पूजयामि।\\
नीलकण्ठ-तनयाय नमः & - कण्ठान् पूजयामि।\\
पतित-पावनाय नमः & - चुबुकानि पूजयामि।\\
पुरुष-श्रेष्ठाय नमः & - नासिकानि पूजयामि\\
कमललोचनाय नमः & - लोचनानि पूजयामि\\
पुण्यमूर्तये नमः & - श्रोत्राणि पूजयामि\\
कस्तूरी-तिलकाञ्चित-फालाय नमः & - ललाटानि पूजयामि\\
षडाननाय नमः & - मुखानि पूजयामि\\
त्रिलोकगुरवे नमः & - ओष्ठानि पूजयामि।\\
सहस्रशीर्ष्णे नमः & - शिरांसि पूजयामि।\\
भस्मोद्धूलित-विग्रहाय नमः & - सर्वाण्यङ्गानि पूजयामि। \\
\end{tabular}

\dnsub{षोडश-नामपूजा}
\begin{multicols}{2}
\begin{enumerate}
\item ॐ ज्ञानशक्त्यात्मने नमः
\item ॐ स्कन्दाय नमः
\item ॐ अग्निभुवे नमः
\item ॐ बाहुलेयाय नमः
\item ॐ गाङ्गेयाय नमः
\item ॐ शरवणोद्भवाय नमः
\item ॐ कार्त्तिकेयाय नमः
\item ॐ कुमाराय नमः
\item ॐ षण्मुखाय नमः
\item ॐ कुक्कुटध्वजाय नमः
\item ॐ शक्तिधराय नमः
\item ॐ गुहाय नमः
\item ॐ ब्रह्मचारिणे नमः
\item ॐ षण्मातुराय नमः
\item ॐ क्रौञ्चभित्रे नमः
\item ॐ शिखिवाहनाय नमः
\end{enumerate}
\end{multicols}

\dnsub{सुब्रह्मण्याष्टोत्तरशतनामावलिः}
\begin{multicols}{2}
\begin{flushleft}
ॐ स्कन्दाय~नमः\\
ॐ गुहाय~नमः\\
ॐ षण्मुखाय~नमः\\
ॐ फालनेत्रसुताय~नमः\\
ॐ प्रभवे~नमः\\
ॐ पिङ्गलाय~नमः\\
ॐ कृत्तिकासूनवे~नमः\\
ॐ शिखिवाहाय~नमः\\
ॐ द्विषड्भुजाय~नमः\\
ॐ द्विषण्णेत्राय~नमः\hfill\devanumber{10}\\
ॐ शक्तिधराय~नमः\\
ॐ पिशिताशप्रभञ्जनाय~नमः\\
ॐ तारकासुरसंहारिणे~नमः\\
ॐ रक्षोबलविमर्दनाय~नमः\\
ॐ मत्ताय~नमः\\
ॐ प्रमत्ताय~नमः\\
ॐ उन्मत्ताय~नमः\\
ॐ सुरसैन्यसुरक्षकाय~नमः\\
ॐ देवसेनापतये~नमः\\
ॐ प्राज्ञाय~नमः\hfill\devanumber{20}\\
ॐ कृपालवे~नमः\\
ॐ भक्तवत्सलाय~नमः\\
ॐ उमासुताय~नमः\\
ॐ शक्तिधराय~नमः\\
ॐ कुमाराय~नमः\\
ॐ क्रौञ्चदारणाय~नमः\\
ॐ सेनानिने~नमः\\
ॐ अग्निजन्मने~नमः\\
ॐ विशाखाय~नमः\\
ॐ शङ्करात्मजाय~नमः\hfill\devanumber{30}\\
ॐ शिवस्वामिने~नमः\\
ॐ गणस्वामिने~नमः\\
ॐ सर्वस्वामिने~नमः\\
ॐ सनातनाय~नमः\\
ॐ अनन्तमूर्तये~नमः\\
ॐ अक्षोभ्याय~नमः\\
ॐ पार्वतीप्रियनन्दनाय~नमः\\
ॐ गङ्गासुताय~नमः\\
ॐ शरोद्भूताय~नमः\\
ॐ आहूताय~नमः\hfill\devanumber{40}\\
ॐ पावकात्मजाय~नमः\\
ॐ जृम्भाय~नमः\\
ॐ प्रजृम्भाय~नमः\\
ॐ उज्जृम्भाय~नमः\\
ॐ कमलासनसंस्तुताय~नमः\\
ॐ एकवर्णाय~नमः\\
ॐ द्विवर्णाय~नमः\\
ॐ त्रिवर्णाय~नमः\\
ॐ सुमनोहराय~नमः\\
ॐ चतुर्वर्णाय~नमः\hfill\devanumber{50}\\
ॐ पञ्चवर्णाय~नमः\\
ॐ प्रजापतये~नमः\\
ॐ अहस्पतये~नमः\\
ॐ अग्निगर्भाय~नमः\\
ॐ शमीगर्भाय~नमः\\
ॐ विश्वरेतसे~नमः\\
ॐ सुरारिघ्ने~नमः\\
ॐ हरिद्वर्णाय~नमः\\
ॐ शुभकराय~नमः\\
ॐ वटवे~नमः\hfill\devanumber{60}\\
ॐ पटुवेषभृते~नमः\\
ॐ पूष्णे~नमः\\
ॐ गभस्तये~नमः\\
ॐ गहनाय~नमः\\
ॐ चन्द्रवर्णाय~नमः\\
ॐ कलाधराय~नमः\\
ॐ मायाधराय~नमः\\
ॐ महामायिने~नमः\\
ॐ कैवल्याय~नमः\\
ॐ शङ्करात्मजाय~नमः\hfill\devanumber{70}\\
ॐ विश्वयोनये~नमः\\
ॐ अमेयात्मने~नमः\\
ॐ तेजोयोनये~नमः\\
ॐ अनामयाय~नमः\\
ॐ परमेष्ठिने~नमः\\
ॐ परब्रह्मणे~नमः\\
ॐ वेदगर्भाय~नमः\\
ॐ विराट्सुताय~नमः\\
ॐ पुलिन्दकन्याभर्त्रे~नमः\\
ॐ महासारस्वतावृताय~नमः\hfill\devanumber{80}\\
ॐ आश्रिताखिलदात्रे~नमः\\
ॐ चोरघ्नाय~नमः\\
ॐ रोगनाशनाय~नमः\\
ॐ अनन्तमूर्तये~नमः\\
ॐ आनन्दाय~नमः\\
ॐ शिखण्डिने~नमः\\
ॐ कृतकेतनाय~नमः\\
ॐ डम्भाय~नमः\\
ॐ परमडम्भाय~नमः\\
ॐ महाडम्भाय~नमः\hfill\devanumber{90}\\
ॐ वृषाकपये~नमः\\
ॐ कारणोत्पत्ति-देहाय~नमः\\
ॐ कारणातीत-विग्रहाय~नमः\\
ॐ अनीश्वराय~नमः\\
ॐ अमृताय~नमः\\
ॐ प्राणाय~नमः\\
ॐ प्राणायामपरायणाय~नमः\\
ॐ विरुद्धहन्त्रे~नमः\\
ॐ वीरघ्नाय~नमः\\
ॐ रक्तश्यामगलाय~नमः\hfill\devanumber{100}\\
ॐ सुब्रह्मण्याय~नमः\\
ॐ गुहाय~नमः\\
ॐ प्रीताय~नमः\\
ॐ ब्रह्मण्याय~नमः\\
ॐ ब्राह्मणप्रियाय~नमः\\
ॐ वंशवृद्धिकराय~नमः\\
ॐ वेदवेद्याय~नमः\\
ॐ अक्षयफलप्रदाय~नमः\\
\end{flushleft}
\end{multicols}

॥इति श्री सुब्रह्मण्याष्टोत्तरशतनामावलिः सम्पूर्णा॥


\dnsub{वल्ली अष्टोत्तरशतनामावलिः}
\begin{flushleft}
\begin{multicols}{2}
ॐ महावल्ल्यै~नमः\\
ॐ वन्द्यायै~नमः\\
ॐ वनवासायै~नमः\\
ॐ वरलक्ष्म्यै~नमः\\
ॐ वरप्रदायै~नमः\\
ॐ वाणीस्तुतायै~नमः\\
ॐ वीतमोहायै~नमः\\
ॐ वामदेवसुतप्रियायै~नमः\\
ॐ वैकुण्ठतनयायै~नमः\\
ॐ वर्यायै~नमः\hfill\devanumber{10}\\
ॐ वनेचरसमादृतायै~नमः\\
ॐ दयापूर्णायै~नमः\\
ॐ दिव्यरूपायै~नमः\\
ॐ दारिद्र्यभयनाशिन्यै~नमः\\
ॐ देवस्तुतायै~नमः\\
ॐ दैत्यहन्त्र्यै~नमः\\
ॐ दोषहीनायै~नमः\\
ॐ दयाम्बुधये~नमः\\
ॐ दुःखहन्त्र्यै~नमः\\
ॐ दुष्टदूरायै~नमः\hfill\devanumber{20}\\
ॐ दुरितघ्न्यै~नमः\\
ॐ दुरासदायै~नमः\\
ॐ नाशहीनायै~नमः\\
ॐ नागनुतायै~नमः\\
ॐ नारदस्तुतवैभवायै~नमः\\
ॐ लवलीकुञ्जसम्भूतायै~नमः\\
ॐ ललितायै~नमः\\
ॐ ललनोत्तमायै~नमः\\
ॐ शान्तदोषायै~नमः\\
ॐ शर्मदात्र्यै~नमः\hfill\devanumber{30}\\
ॐ शरजन्मकुटुम्बिन्यै~नमः\\
ॐ पद्मिन्यै~नमः\\
ॐ पद्मवदनायै~नमः\\
ॐ पद्मनाभसुतायै~नमः \\
ॐ परायै~नमः\\
ॐ पूर्णरूपायै~नमः\\
ॐ पुण्यशीलायै~नमः\\
ॐ प्रियङ्गुवनपालिन्यै~नमः\\
ॐ सुन्दर्यै~नमः\\
ॐ सुरसंस्तुतायै~नमः\hfill\devanumber{40}\\
ॐ सुब्रह्मण्यकुटुम्बिन्यै~नमः\\
ॐ मान्यायै~नमः\\
ॐ मनोहरायै~नमः\\
ॐ मायायै~नमः\\
ॐ महेश्वरसुतप्रियायै~नमः\\
ॐ कुमार्यै~नमः\\
ॐ करुणापूर्णायै~नमः\\
ॐ कार्त्तिकेयमनोहरायै~नमः\\
ॐ पद्मनेत्रायै~नमः\\
ॐ परानन्दायै~नमः\hfill\devanumber{50}\\
ॐ पार्वतीसुतवल्लभायै~नमः\\
ॐ महादेव्यै~नमः\\
ॐ महामायायै~नमः\\
ॐ मल्लिकाकुसुमप्रियायै~नमः\\
ॐ चन्द्रवक्त्रायै~नमः\\
ॐ चारुरूपायै~नमः\\
ॐ चाम्पेयकुसुमप्रियायै~नमः\\
ॐ गिरिवासायै~नमः\\
ॐ गुणनिधये~नमः\\
ॐ गतावन्यायै~नमः\hfill\devanumber{60}\\
ॐ गुहप्रियायै~नमः\\
ॐ कलिहीनायै~नमः\\
ॐ कलारूपायै~नमः\\
ॐ कृत्तिकासुतकामिन्यै~नमः\\
ॐ गतदोषायै~नमः\\
ॐ गीतगुणायै~नमः\\
ॐ गङ्गाधरसुतप्रियायै~नमः\\
ॐ भद्ररूपायै~नमः\\
ॐ भगवत्यै~नमः\\
ॐ भाग्यदायै~नमः\hfill\devanumber{70}\\
ॐ भवहारिण्यै~नमः\\
ॐ भवहीनायै~नमः\\
ॐ भव्यदेहायै~नमः\\
ॐ भवात्मजमनोहरायै~नमः\\
ॐ सौम्यायै~नमः\\
ॐ सर्वेश्वर्यै~नमः\\
ॐ सत्यायै~नमः\\
ॐ साध्व्यै~नमः\\
ॐ सिद्धसमर्चितायै~नमः\\
ॐ हानिहीनायै~नमः\hfill\devanumber{80}\\
ॐ हरिसुतायै~नमः\\
ॐ हरसूनुमनःप्रियायै~नमः\\
ॐ कल्याण्यै~नमः\\
ॐ कमलायै~नमः\\
ॐ कल्यायै~नमः\\
ॐ कुमारसुमनोहरायै~नमः\\
ॐ जन्महीनायै~नमः\\
ॐ जन्महन्त्र्यै~नमः\\
ॐ जनार्दनसुतायै~नमः\\
ॐ जयायै~नमः\hfill\devanumber{90}\\
ॐ रमायै~नमः\\
ॐ रामायै~नमः\\
ॐ रम्यरूपायै~नमः\\
ॐ राज्ञ्यै~नमः\\
ॐ राजवरादृतायै~नमः\\
ॐ नीतिज्ञायै~नमः\\
ॐ निर्मलायै~नमः\\
ॐ नित्यायै~नमः\\
ॐ नीलकण्ठसुतप्रियायै~नमः\\
ॐ शिवरूपायै~नमः\hfill\devanumber{100}\\
ॐ सुधाकारायै~नमः\\
ॐ शिखिवाहनवल्लभायै~नमः\\
ॐ व्याधात्मजायै~नमः\\
ॐ व्याधिहन्त्र्यै~नमः\\
ॐ विविधागमसंस्तुतायै~नमः\\
ॐ हर्षदात्र्यै~नमः\\
ॐ हरिभवायै~नमः\\
ॐ हरसूनुप्रियङ्गनायै~नमः\\
\end{multicols}
\end{flushleft}
॥इति श्री वल्ल्यष्टोत्तरशतनामावलिः सम्पूर्णा॥



\dnsub{देवसेना अष्टोत्तरशतनामावलिः}
\begin{flushleft}
\begin{multicols}{2}
ॐ देवसेनायै~नमः\\
ॐ देवलोकजनन्यै~नमः\\
ॐ दिव्यसुन्दर्यै~नमः\\
ॐ देवपूज्यायै~नमः\\
ॐ दयारूपायै~नमः\\
ॐ दिव्याभरणभूषितायै~नमः\\
ॐ दारिद्र्यनाशिन्यै~नमः\\
ॐ देव्यै~नमः\\
ॐ दिव्यपङ्कजधारिण्यै~नमः\\
ॐ दुःस्वप्ननाशिन्यै~नमः\hfill\devanumber{10}\\
ॐ दुष्टशमन्यै~नमः\\
ॐ दोषवर्जितायै~नमः\\
ॐ पीताम्बरायै~नमः\\
ॐ पद्मवासायै~नमः\\
ॐ परानन्दायै~नमः\\
ॐ परात्परायै~नमः\\
ॐ पूर्णायै~नमः\\
ॐ परमकल्याण्यै~नमः\\
ॐ प्रकटायै~नमः\\
ॐ पापनाशिन्यै~नमः\hfill\devanumber{20}\\
ॐ प्राणेश्वर्यै~नमः\\
ॐ परायै शक्त्यै~नमः\\
ॐ परमायै~नमः\\
ॐ परमेश्वर्यै~नमः\\
ॐ महावीर्यायै~नमः\\
ॐ महाभोगायै~नमः\\
ॐ महापूज्यायै~नमः\\
ॐ महाबलायै~नमः\\
ॐ माहेन्द्र्यै~नमः\\
ॐ महत्यै~नमः\hfill\devanumber{30}\\
ॐ मायायै~नमः\\
ॐ मुक्ताहारविभूषितायै~नमः\\
ॐ ब्रह्मानन्दायै~नमः\\
ॐ ब्रह्मरूपायै~नमः\\
ॐ ब्रह्माण्यै~नमः\\
ॐ ब्रह्मपूजितायै~नमः\\
ॐ कार्त्तिकेयप्रियायै~नमः\\
ॐ कान्तायै~नमः\\
ॐ कामरूपायै~नमः\\
ॐ कलाधरायै~नमः\hfill\devanumber{40}\\
ॐ विष्णुपूज्यायै~नमः\\
ॐ विश्ववेद्यायै~नमः\\
ॐ वेदवेद्यायै~नमः\\
ॐ वज्रिजातायै~नमः\\
ॐ वरप्रदायै~नमः\\
ॐ विशाखकान्तायै~नमः\\
ॐ विमलायै~नमः\\
ॐ विशालाक्ष्यै~नमः\\
ॐ सत्यसन्धायै~नमः\\
ॐ सत्प्रभावायै~नमः\hfill\devanumber{50}\\
ॐ सिद्धिदायै~नमः\\
ॐ स्कन्दवल्लभायै~नमः\\
ॐ सुरेश्वर्यै~नमः\\
ॐ सर्ववन्द्यायै~नमः\\
ॐ सुन्दर्यै~नमः\\
ॐ साम्यवर्जितायै~नमः\\
ॐ हतदैत्यायै~नमः\\
ॐ हानिहीनायै~नमः\\
ॐ हर्षदात्र्यै~नमः\\
ॐ हतासुरायै~नमः\hfill\devanumber{60}\\
ॐ हितकर्त्र्यै~नमः\\
ॐ हीनदोषायै~नमः\\
ॐ हेमाभायै~नमः\\
ॐ हेमभूषणायै~नमः\\
ॐ लयहीनायै~नमः\\
ॐ लोकवन्द्यायै~नमः\\
ॐ ललितायै~नमः\\
ॐ ललनोत्तमायै~नमः\\
ॐ लम्बवामकरायै~नमः\\
ॐ लभ्यायै~नमः\hfill\devanumber{70}\\
ॐ लज्जाढ्यायै~नमः\\
ॐ लाभदायिन्यै~नमः\\
ॐ अचिन्त्यशक्त्यै~नमः\\
ॐ अचलायै~नमः\\
ॐ अचिन्त्यरूपायै~नमः\\
ॐ अक्षरायै~नमः\\
ॐ अभयायै~नमः\\
ॐ अम्बुजाक्ष्यै~नमः\\
ॐ अमराराध्यायै~नमः\\
ॐ अभयदायै~नमः\hfill\devanumber{80}\\
ॐ असुरभीतिदायै~नमः\\
ॐ शर्मदायै~नमः\\
ॐ शक्रतनयायै~नमः\\
ॐ शङ्करात्मजवल्लभायै~नमः\\
ॐ शुभायै~नमः\\
ॐ शुभप्रदायै~नमः\\
ॐ शुद्धायै~नमः\\
ॐ शरणागतवत्सलायै~नमः\\
ॐ मयूरवाहनदयितायै~नमः\\
ॐ महामहिमशालिन्यै~नमः\hfill\devanumber{90}\\
ॐ मदहीनायै~नमः\\
ॐ मातृपूज्यायै~नमः\\
ॐ मन्मथारिसुतप्रियायै~नमः\\
ॐ गुणपूर्णायै~नमः\\
ॐ गणाराद्ध्यायै~नमः\\
ॐ गौरीसुतमनःप्रियायै~नमः\\
ॐ गतदोषायै~नमः\\
ॐ गतावद्यायै~नमः\\
ॐ गङ्गाजातकुटुम्बिन्यै~नमः\\
ॐ चतुरायै~नमः\hfill\devanumber{100}\\
ॐ चन्द्रवदनायै~नमः\\
ॐ चन्द्रचूडभवप्रियायै~नमः\\
ॐ रम्यरूपायै~नमः\\
ॐ रमावन्द्यायै~नमः\\
ॐ रुद्रसूनुमनःप्रियायै~नमः\\
ॐ मङ्गलायै~नमः\\
ॐ मधुरालापायै~नमः\\
ॐ महेशतनयप्रियायै~नमः\\
\end{multicols}
\end{flushleft}
॥इति श्री देवसेना अष्टोत्तरशतनामावलिः सम्पूर्णा॥




श्री-वल्ली-देवसेना-समेत-सुब्रह्मण्य-स्वामिने नमः नानाविध-परिमल-पत्र-पुष्पाणि समर्पयामि।


\twolineshloka*
{दशाङ्गं च पटीरं च एला-कुङ्कुम-संयुतम्}
{धूपं गृहाण देवेश सुब्रह्मण्य नमोऽस्तु ते}
%\hfill{}
श्री-वल्ली-देवसेना-समेत-सुब्रह्मण्य-स्वामिने नमः धूपम् आघ्रापयामि।

\twolineshloka*
{इन्द्वर्कवह्निनेत्राय देवसेनापतये नमः}
{घृतवर्तिसुसंयुक्तं दीपोऽयम् अवलोक्यताम्}
%\hfill{}
श्री-वल्ली-देवसेना-समेत-सुब्रह्मण्य-स्वामिने नमः दीपं दर्शयामि। धूप-दीपानन्तरम् आचमनीयं समर्पयामि।

\twolineshloka
{सत्पात्रसिद्धं सुहविर्विविधानेक-भक्षणम्}
{निवेदयामि देवेश सानुगाय गृहाण तत्}
%\hfill{}
श्री-वल्ली-देवसेना-समेत-सुब्रह्मण्य-स्वामिने नमः () महानैवेद्यं निवेदयामि। 
मध्ये मध्ये अमृतपानीयं समर्पयामि। हस्त-प्रक्षालनं समर्पयामि। गण्डूषं समर्पयामि। पुनः हस्त-प्रक्षालनं समर्पयामि।
 पाद-प्रक्षालनं समर्पयामि। आचमनीयं समर्पयामि।

\twolineshloka*
{पूगीफलसमायुक्तं नागवल्लीदलैर्युतम्}
{कर्पूरचूर्णसंयुक्तं ताम्बूलं प्रतिगृह्यताम्}
%\hfill{}
श्री-वल्ली-देवसेना-समेत-सुब्रह्मण्य-स्वामिने नमः ताम्बूलं समर्पयामि।

\twolineshloka*
{नीराजनं देवदेव सूर्यकोटि-समप्रभ}
{अहं भक्त्या प्रदास्यामि स्वीकुरुष्व दयानिधे}
%\hfill{}
श्री-वल्ली-देवसेना-समेत-सुब्रह्मण्य-स्वामिने नमः कर्पूर-नीराजनं दर्शयामि। 
पुष्पाञ्जलिं समर्पयामि। आचमनीयं समर्पयामि। रक्षां धारयामि।

\twolineshloka*
{सर्व-पापौघ-विध्वंस साक्षाद्धर्मस्वरूपक}
{पुष्पाञ्जलिं प्रदास्यामि गृहाण भुवनेश्वर}
%\hfill{}
श्री-वल्ली-देवसेना-समेत-सुब्रह्मण्य-स्वामिने नमः मन्त्रपुष्पाञ्जलिं समर्पयामि। स्वर्णपुष्पं समर्पयामि।

\twolineshloka*
{यानि कानि च पापानि जन्मान्तरकृतानि च}
{तानि तानि विनश्यन्ति प्रदक्षिण पदे पदे}

\twolineshloka*
{षण्मुखं पार्वतीपुत्रं क्रौञ्चशैलविमर्दनम्}
{देवसेनापतिं देवं स्कन्दं वन्दे शिवात्मजम्}
\twolineshloka*
{तारकासुर-हन्तारं मयूरोपरि संस्थितम्}
{शक्तिपाणिं च देवेशं स्कन्दं वन्दे शिवात्मजम्}
%\hfill{}
श्री-वल्ली-देवसेना-समेत-सुब्रह्मण्य-स्वामिने नमः प्रदक्षिण-नमस्कारान् समर्पयामि।

\fourlineindentedshloka*
{नमः केकिने शक्तये चापि तुभ्यम्}
{नमश्छाग तुभ्यं नमः कुक्कुटाय}
{नमः सिन्धवे सिन्धुदेशाय तुभ्यम्}
{पुनः स्कन्दमूर्ते नमस्ते नमोऽस्तु}

\fourlineindentedshloka*
{जयाऽऽनन्दभूमन् जयापारधामन्}
{जयामोघकीर्ते जयाऽऽनन्दमूर्ते}
{जयाऽऽनन्दसिन्धो जयाशेषबन्धो}
{जय त्वं सदा मुक्तिदानेशसूनो}

श्री-वल्ली-देवसेना-समेत-सुब्रह्मण्य-स्वामिने नमः प्रार्थनाः समर्पयामि।

श्री-वल्ली-देवसेना-समेत-सुब्रह्मण्य-स्वामिने नमः छत्त्रं समर्पयामि।
चामरयुगलं वीजयामि।\\
दर्पणं दर्शयामि। गीतं श्रावयामि। \\
नृत्तं दर्शयामि। आन्दोलिकाम् आरोहयामि।\\
गजम् आरोहयामि। अश्वम् आरोहयामि।\\
रथम् आरोहयामि। समस्त-राजोपचार-देवोपचार-पूजाः समर्पयामि।


अनेन पूजनेन श्री-वल्ली-देवसेना-समेत-सुब्रह्मण्य-स्वामिनः प्रीयन्ताम्। \\


\fourlineindentedshloka*
{कायेन वाचा मनसेन्द्रियैर्वा}
{बुद्‌ध्याऽऽत्मना वा प्रकृतेः स्वभावात्}
{करोमि यद्यत् सकलं परस्मै}
{नारायणायेति समर्पयामि}

ॐ तत्सद्ब्रह्मार्पणमस्तु।