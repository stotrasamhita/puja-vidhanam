% !TeX program = XeLaTeX
% !TeX root = ../pujavidhanam.tex

\setlength{\parindent}{0pt}
\chapt{श्री-शङ्कर-भगवत्पाद-पूजा}

\input{purvanga/vighneshwara-puja}

\sect{प्रधान-पूजा — श्री-शङ्कर-भगवत्पाद-पूजा}

\twolineshloka*
{शुक्लाम्बरधरं विष्णुं शशिवर्णं चतुर्भुजम्}
{प्रसन्नवदनं ध्यायेत् सर्वविघ्नोपशान्तये}
 
प्राणान्  आयम्य।  ॐ भूः + भूर्भुवः॒ सुव॒रोम्।

\dnsub{सङ्कल्पः}

ममोपात्त-समस्त-दुरित-क्षयद्वारा श्री-परमेश्वर-प्रीत्यर्थं शुभे शोभने मुहूर्ते अद्य ब्रह्मणः
द्वितीयपरार्धे श्वेतवराहकल्पे वैवस्वतमन्वन्तरे अष्टाविंशतितमे कलियुगे प्रथमे पादे
जम्बूद्वीपे भारतवर्षे भरतखण्डे मेरोः दक्षिणे पार्श्वे शकाब्दे अस्मिन् वर्तमाने व्यावहारिकाणां प्रभवादीनां षष्ट्याः संवत्सराणां मध्ये \mbox{(~~~)}\see{app:samvatsara_names} नाम संवत्सरे उत्तरायणे
वसन्तऋतौ  मेष/वृषभ-वैशाख-मासे शुक्लपक्षे पञ्चम्यां शुभतिथौ
(इन्दु / भौम / बुध / गुरु / भृगु / स्थिर / भानु) वासरयुक्तायाम्
(आर्द्रा/?)\see{app:nakshatra_names}-नक्षत्र \mbox{(~~~)}\see{app:yoga_names} योग  \mbox{(~~~)} करण-युक्तायां च एवं गुण विशेषण विशिष्टायाम्
अस्यां पञ्चम्यां  
शुभतिथौ श्रीपरमेश्वरप्रीत्यर्थम्

\begin{itemize}
\item उत्तराषाढा-नक्षत्रे धनूराशौ आविर्भू\-तानां श्रीमत्-शङ्कर-विजयेन्द्र-सरस्वती-श्रीपादानां, शतभिषङ्-नक्षत्रे कुम्भ-राशौ आविर्भूतानां श्रीमत्-सत्य-चन्द्रशेखरेन्द्र-सरस्वती-श्रीपादानाम् अस्माकं जगद्गुरूणां दीर्घ-आयुः-आरोग्य-सिद्ध्यर्थं,

\item तैः सङ्कल्पितानां सर्वेषां लोक-क्षेमार्थ-कार्याणां वेद-शास्त्रादि-सम्प्रदाय-पोषण-कार्याणां विविध-क्षेत्र-यात्रायाश्च अविघ्नतया सम्पूर्त्यर्थं

\item कामकोटि-गुरु-परम्परायां कामकोटि-भक्त-जनानाम् अचञ्चल-भावशुद्ध-दृढतर-भक्ति-सिद्ध्यर्थं, परस्पर-ऐकमत्य-सिद्ध्यर्थं

\item भारतीयानां महाजनानां विघ्न-निवृत्ति-पूर्वक-सत्कार्य-प्रवृत्ति-द्वारा ऐहिक-आमुष्मिक-अभ्युदय-प्राप्त्यर्थम्, असत्कार्येभ्यः निवृत्त्यर्थं

\item भारतीयानां सन्ततेः सनातन-सम्प्रदाये श्रद्धा-भक्त्योः अभिवृद्ध्यर्थं

\item सर्वेषां द्विपदां चतुष्पदाम् अन्येषां च प्राणि-वर्गाणाम् आरोग्य-युक्त-सुख-जीवन-अवाप्त्यर्थम्

\item अस्माकं सहकुटुम्बानां क्षेमस्थैर्य-धैर्य-वीर्य-विजय-आयुरारोग्य-ऐश्वर्याभिवृद्ध्यर्थम्
 धर्मार्थकाममोक्ष\-चतुर्विधफलपुरुषार्थसिद्ध्यर्थं पुत्रपौत्राभि\-वृद्ध्यर्थम् इष्टकाम्यार्थसिद्ध्यर्थं विवेक-वैराग्य-सिद्ध्यर्थम्
मम इहजन्मनि पूर्वजन्मनि जन्मान्तरे च सम्पादितानां ज्ञानाज्ञानकृतमहा\-पातकचतुष्टय-व्यतिरिक्तानां रहस्यकृतानां प्रकाशकृतानां सर्वेषां पापानां सद्य अपनोदनद्वारा सकल-पापक्षयार्थं 

\end{itemize}

श्रीमत्-शङ्करभगवत्पाद-प्रीत्यर्थं
श्री-शङ्कर-जयन्ती-महोत्सवे
यथाशक्ति-ध्यान-आवाहनादि-षोडशो\-पचारैः श्रीमत्-शङ्कर-भगवत्पादाचार्य-पूजां करिष्ये। तदङ्गं कलशपूजां च करिष्ये।

श्रीविघ्नेश्वराय नमः यथास्थानं प्रतिष्ठापयामि।\\
(गणपति-प्रसादं शिरसा गृहीत्वा)
\renewcommand{\devaName}{श्रीमत्-शङ्कर-भगवत्पादाचार्याः}

\input{purvanga/aasana-puja}

\input{purvanga/ghanta-puja}

\input{purvanga/kalasha-puja}

\input{purvanga/aatma-puja}

\input{purvanga/pitha-puja}

\input{purvanga/guru-dhyanam}

\begin{center}

\sect{षोडशोपचार-पूजा}

\dnsub{प्रधान-पूजा}
\begingroup
\centering
\twolineshloka*
{श्रुति-स्मृति-पुराणानाम् आलयं करुणालयम्}
{नमामि भगवत्पाद-शङ्करं लोक-शङ्करम्}


%अस्मिन् चित्रपटे/विग्रहे
श्रीमत्-शङ्कर-भगवत्पादाचार्यान् ध्यायामि।

\fourlineindentedshloka*
{अज्ञानान्तर्गहन-पतितान् आत्म-विद्योपदेशैः}
{त्रातुं लोकान् भव-दव-शिखा-ताप-पापच्यमानान्}
{मुक्त्वा मौनं वट-विटपिनो मूलतो निष्पतन्ती}
{शम्भोर्मूर्तिश्चरति भुवने शङ्कराचार्य-रूपा}

नम॑स्ते रुद्र म॒न्यव॑ उ॒तो त॒ इष॑वे॒ नमः॑। नम॑स्ते अस्तु॒ धन्व॑ने बा॒हुभ्या॑मु॒त ते॒ नमः॑॥\\
ॐ ह्रीं न॒मः शि॒वाय॑। स॒द्योजा॒तं प्र॑पद्यामि।

\twolineshloka*
{यमाश्रिता गिरां देवी नन्दयत्यात्म-संश्रितान्}
{तमाश्रये श्रिया जुष्टं शङ्करं करुणा-निधिम्}

श्रीमत्-शङ्कर-भगवत्पादाचार्यान् आवाहयामि।

या त॒ इषुः॑ शि॒वत॑मा शि॒वं ब॒भूव॑ ते॒ धनुः॑। शि॒वा श॑र॒व्या॑ या तव॒ तया॑ नो रुद्र मृडय॥\\
ॐ ह्रीं न॒मः शि॒वाय॑। स॒द्योजा॒ताय॒ वै नमो॒ नमः॑। 

\twolineshloka*
{श्री-गुरुं भगवत्पादं शरण्यं भक्त-वत्सलम्}
{शिवं शिव-करं शुद्धम् अप्रमेयं नमाम्यहम्}

श्रीमत्-शङ्कर-भगवत्पादाचार्येभ्यो~नमः, आसनं समर्पयामि।

\twolineshloka*
{नित्यं शुद्धं निराकारं निराभासं निरञ्जनम्}
{नित्य-बोधं चिदानन्दं गुरुं ब्रह्म नमाम्यहम्}

श्रीमत्-शङ्कर-भगवत्पादाचार्येभ्यो~नमः, स्वागतं व्याहरामि। पूर्ण-कुम्भं समर्पयामि।

या ते॑ रुद्र शि॒वा त॒नूरघो॒राऽपा॑पकाशिनी। तया॑ नस्त॒नुवा॒ शन्त॑मया॒ गिरि॑शन्ता॒\-भिचा॑कशीहि॥\\
ॐ ह्रीं न॒मः शि॒वाय॑। भ॒वे भ॑वे॒ नाति॑ भवे भवस्व॒ माम्।

\twolineshloka*
{सर्व-तन्त्र-स्व-तन्त्राय सदात्माद्वैत-रूपिणे}
{श्रीमते शङ्करार्याय वेदान्त-गुरवे नमः}

श्रीमत्-शङ्कर-भगवत्पादाचार्येभ्यो~नमः, पाद्यं समर्पयामि।

यामिषुं॑ गिरिशन्त॒ हस्ते॒ बिभ॒र्ष्यस्त॑वे। शि॒वां गि॑रित्र॒ तां कु॑रु॒ मा हिꣳ॑सीः॒ पुरु॑षं॒ जग॑त्॥\\
ॐ ह्रीं न॒मः शि॒वाय॑। भ॒वोद्भ॑वाय॒ नमः॑॥ 

\twolineshloka*
{वेदान्तार्थाभिधानेन सर्वानुग्रह-कारिणम्}
{यति-रूप-धरं वन्दे शङ्करं लोक-शङ्करम्}

श्रीमत्-शङ्कर-भगवत्पादाचार्येभ्यो~नमः, अर्घ्यं समर्पयामि।\\

शि॒वेन॒ वच॑सा त्वा॒ गिरि॒शाच्छा॑वदामसि। यथा॑ नः॒ सर्व॒मिज्जग॑दय॒क्ष्मꣳ सु॒मना॒ अस॑त्॥\\
ॐ ह्रीं न॒मः शि॒वाय॑। वा॒म॒दे॒वाय॒ नमः॑। 

\twolineshloka*
{संसाराब्धि-निषण्णाज्ञ-निकर-प्रोद्दिधीर्षया}
{कृत-संहननं वन्दे भगवत्पाद-शङ्करम्}

श्रीमत्-शङ्कर-भगवत्पादाचार्येभ्यो~नमः, आचमनीयं समर्पयामि।\\
श्रीमत्-शङ्कर-भगवत्पादाचार्येभ्यो~नमः, मधुपर्कं समर्पयामि।\\

अध्य॑वोचदधिव॒क्ता प्र॑थ॒मो दैव्यो॑ भि॒षक्। अहीꣴ॑श्च॒ सर्वा᳚ञ्ज॒म्भय॒-न्थ्सर्वा᳚श्च यातुधा॒न्यः॑॥\\
ॐ ह्रीं न॒मः शि॒वाय॑। ज्ये॒ष्ठाय॒ नमः॑। 

\twolineshloka*
{यत्-पाद-पङ्कज-ध्यानात् तोटकाद्या यतीश्वराः}
{बभूवुस्तादृशं वन्दे शङ्करं षण्मतेश्वरम्}

श्रीमत्-शङ्कर-भगवत्पादाचार्येभ्यो~नमः, स्नपयामि। (श्रीरुद्र-चमक-पुरुषसूक्त-उपनिषद्भिः स्नापयित्वा) स्नानानन्तरम् आचमनीयं समर्पयामि।\\


अ॒सौ यस्ता॒म्रो अ॑रु॒ण उ॒त ब॒भ्रुः सु॑म॒ङ्गलः॑। ये चे॒माꣳ रु॒द्रा अ॒भितो॑ दि॒क्षु श्रि॒ताः स॑हस्र॒शोऽवै॑षा॒ꣳ॒ हेड॑ ईमहे॥\\
ॐ ह्रीं न॒मः शि॒वाय॑। श्रे॒ष्ठाय॒ नमः॑। 

\twolineshloka*
{नमः श्री-शङ्कराचार्य-गुरवे शङ्करात्मने}
{शरीरिणां शङ्कराय शङ्कर-ज्ञान-हेतवे}

श्रीमत्-शङ्कर-भगवत्पादाचार्येभ्यो~नमः, वस्त्रं समर्पयामि।\\

अ॒सौ यो॑ऽव॒सर्प॑ति॒ नील॑ग्रीवो॒ विलो॑हितः। उ॒तैनं॑ गो॒पा अ॑दृश॒न्न॒दृ॑शन्नुदहा॒र्यः॑। उ॒तैनं॒ विश्वा॑ भू॒तानि॒ स दृ॒ष्टो मृ॑डयाति नः॥\\
ॐ ह्रीं न॒मः शि॒वाय॑। रु॒द्राय॒ नमः॑।

\twolineshloka*
{हर-लीलावताराय शङ्कराय वरौजसे}
{कैवल्य-कलना-कल्प-तरवे गुरवे नमः}

श्रीमत्-शङ्कर-भगवत्पादाचार्येभ्यो~नमः, यज्ञोपवीतं समर्पयामि। \\

\twolineshloka*
{प्रचार्यं सर्व-लोकेषु सञ्चार्यं हृदयाम्बुजे}
{विचार्यं सर्व-वेदान्तैः आचार्यं शङ्करं भजे}

श्रीमत्-शङ्कर-भगवत्पादाचार्येभ्यो~नमः, भस्मोद्धूलनं रुद्राक्ष-मालिकां च समर्पयामि।\\

नमो॑ अस्तु॒ नील॑ग्रीवाय सहस्रा॒क्षाय॑ मी॒ढुषे᳚। अथो॒ ये अ॑स्य॒ सत्वा॑नो॒ऽहं तेभ्यो॑ऽकरं॒ नमः॑॥\\
ॐ ह्रीं न॒मः शि॒वाय॑। काला॑य॒ नमः॑। 

\twolineshloka*
{याऽनुभूतिः स्वयं-ज्योतिः आदित्येशान-विग्रहा}
{शङ्कराख्या च तं नौमि सुरेश्वर-गुरुं परम्}

श्रीमत्-शङ्कर-भगवत्पादाचार्येभ्यो~नमः, दिव्य-परिमल-गन्धान् धारयामि।\\ गन्धस्योपरि हरिद्रा-कुङ्कुमं समर्पयामि।\\

\twolineshloka*
{आनन्द-घनमद्वन्द्वं निर्विकारं निरञ्जनम्}
{भजेऽहं भगवत्पादं भजतामभय-प्रदम्}

श्रीमत्-शङ्कर-भगवत्पादाचार्येभ्यो~नमः, दण्डं समर्पयामि।\\

प्र मु॑ञ्च॒ धन्व॑न॒स्त्वमु॒भयो॒रार्त्नि॑यो॒र्ज्याम्। याश्च॑ ते॒ हस्त॒ इष॑वः॒ परा॒ ता भ॑गवो वप॥\\
ॐ ह्रीं न॒मः शि॒वाय॑। कल॑विकरणाय॒ नमः॑। 

\twolineshloka*
{तं वन्दे शङ्कराचार्यं लोक-त्रितय-शङ्करम्}
{सत्-तर्क-नखरोद्गीर्ण-वावदूक-मतङ्गजम्}

श्रीमत्-शङ्कर-भगवत्पादाचार्येभ्यो~नमः, अक्षतान् समर्पयामि।

अ॒व॒तत्य॒ धनु॒स्त्वꣳ सह॑स्राक्ष॒ शते॑षुधे। नि॒शीर्य॑ श॒ल्यानां॒ मुखा॑ शि॒वो नः॑ सु॒मना॑ भव॥\\
ॐ ह्रीं न॒मः शि॒वाय॑। बल॑विकरणाय॒ नमः॑। 

\twolineshloka*
{नमामि शङ्कराचार्य-गुरु-पाद-सरोरुहम्}
{यस्य प्रसादान्मूढोऽपि सर्व-ज्ञो भवति स्वयम्}

श्रीमत्-शङ्कर-भगवत्पादाचार्येभ्यो~नमः, पुष्प-मालां समर्पयामि। पुष्पैः पूजयामि।\\
\endgroup

\section{श्री-शङ्कर-चतुर्विंशति-नामावल्या अङ्ग-पूजा}
\begin{tabular}{lll}
१. & अष्ट-वर्ष-चतुर्वेदिने~नमः &  पादौ पूजयामि\\
२. & द्वादशाखिल-शास्त्र-विदे~नमः &  गुल्फौ पूजयामि\\
३. & सर्व-लोक-ख्यात-शीलाय~नमः &  जङ्घे पूजयामि\\
४. & प्रस्थान-त्रय-भाष्य-कृते~नमः &  जानुनी पूजयामि\\
५. & पद्मपादादि-सच्छिष्याय~नमः &  ऊरू पूजयामि\\
६. & पाषण्ड-ध्वान्त-भास्कराय~नमः &  कटिं पूजयामि\\
७. & अद्वैत-स्थापनाचार्याय~नमः &  गुह्यं पूजयामि\\
८. & द्वैतादि-द्विप-केसरिणे~नमः &  नाभिं पूजयामि\\
९. & व्यास-नन्दित-सिद्धान्ताय~नमः &  उदरं पूजयामि\\
१०. & वाद-निर्जित-मण्डनाय~नमः &  वक्षःस्थलं पूजयामि\\
११. & षण्मत-स्थापनाचार्याय~नमः &  हृदयं पूजयामि\\
१२. & षड्-गुणैश्वर्य-मण्डिताय~नमः &  कण्ठं पूजयामि\\
१३. & सर्व-लोकानुग्रह-कृते~नमः &  स्कन्धौ पूजयामि\\
१४. & सर्व-ज्ञ-त्वादि-भूषणाय~नमः &  हस्तौ पूजयामि\\
१५. & श्रुति-स्मृति-पुराणार्थाय~नमः &  वक्त्रं पूजयामि\\
१६. & श्रुत्येक-शरण-प्रियाय~नमः &  चिबुकं पूजयामि\\
१७. & सकृत्-स्मरण-सन्तुष्टाय~नमः &  ओष्ठौ पूजयामि\\
१८. & शरणागत-वत्सलाय~नमः &  कपोलौ पूजयामि\\
१९. & निर्व्याज-करुणा-मूर्तये~नमः &  नासिकां पूजयामि\\
२०. & निरहम्भाव-गोचराय~नमः &  नेत्रे पूजयामि\\
२१. & संशान्त-भक्त-हृत्-तापाय~नमः &  कर्णौ पूजयामि\\
२२. & सर्व-ज्ञान-फल-प्रदाय~नमः &  ललाटं पूजयामि\\
२३. & सदसद्-वस्तु-विमुखाय~नमः &  शिरः पूजयामि\\
२४. & सत्ता-सामान्य-विग्रहाय~नमः& सर्वाण्यङ्गानि पूजयामि\\
\end{tabular}


\begingroup
\centering
\setlength{\columnseprule}{1pt}
\let\chapt\sect
\input{../namavali-manjari/100/AdiShankaracharya_108.tex}
\endgroup




\section{आचार्यपरम्परानामावलिः}
\label{sec:ParamparaNamavali}

\medskip
    \begin{flushleft}
\centerline{\bfseries ॥ पूर्वाचार्याः ॥}
\begin{enumerate}%\itemsep -0.9ex
\item श्रीमते दक्षिणामूर्तये~नमः\\
\item श्रीमते विष्णवे~नमः\\
\item श्रीमते ब्रह्मणे~नमः\\
\item श्रीमते वसिष्ठाय~नमः\\
\item श्रीमते शक्तये~नमः\\
\item श्रीमते पराशराय~नमः\\
\item श्रीमते व्यासाय~नमः\\
\item श्रीमते शुकाय~नमः\\
\item श्रीमते गौडपादाय~नमः\\
\item श्रीमते गोविन्द-भगवत्पादाय~नमः\\
\item श्रीमते शङ्कर-भगवत्पादाय~नमः\\
\end{enumerate}

\medskip

\centerline{\bfseries ॥ भगवत्पादशिष्याः ॥}
\begin{enumerate}
    \item श्रीमते पद्मपादाचार्याय~नमः
    \item श्रीमते सुरेश्वराचार्याय~नमः
    \item श्रीमते हस्तामलकाचार्याय~नमः
    \item श्रीमते तोटकाचार्याय~नमः
    \item श्रीमते पृथिवीधवाचार्याय~नमः
    \item श्रीमते सर्वज्ञात्म-इन्द्रसरस्वत्यै~नमः
    \item अन्येभ्यः भगवत्पाद-शिष्येभ्यो~नमः
\end{enumerate}

\medskip

\centerline{\bfseries ॥ कामकोटि-आचार्याः ॥}
\begin{enumerate}
\item श्रीमते शङ्कर-भगवत्पादाय~नमः
\item श्रीमते सुरेश्वराचार्याय~नमः
\item श्रीमते सर्वज्ञात्म-इन्द्रसरस्वत्यै~नमः
\item श्रीमते सत्यबोध-इन्द्रसरस्वत्यै~नमः
\item श्रीमते ज्ञानानन्द-इन्द्रसरस्वत्यै~नमः
\item श्रीमते शुद्धानन्द-इन्द्रसरस्वत्यै~नमः
\item श्रीमते आनन्दज्ञान-इन्द्रसरस्वत्यै~नमः
\item श्रीमते कैवल्यानन्द-इन्द्रसरस्वत्यै~नमः
\item श्रीमते कृपाशङ्कर-इन्द्रसरस्वत्यै~नमः
\item श्रीमते विश्वरूप-सुरेश्वर-इन्द्रसरस्वत्यै~नमः
\item श्रीमते शिवानन्द-चिद्घन-इन्द्रसरस्वत्यै~नमः
\item श्रीमते सार्वभौम-चन्द्रशेखर-इन्द्रसरस्वत्यै~नमः
\item श्रीमते काष्ठमौन-सच्चिद्घन-इन्द्रसरस्वत्यै~नमः
\item श्रीमते भैरवजिद्-विद्याघन-इन्द्रसरस्वत्यै~नमः
\item श्रीमते गीष्पति-गङ्गाधर-इन्द्रसरस्वत्यै~नमः
\item श्रीमते उज्ज्वलशङ्कर-इन्द्रसरस्वत्यै~नमः
\item श्रीमते गौड-सदाशिव-इन्द्रसरस्वत्यै~नमः
\item श्रीमते सुर-इन्द्रसरस्वत्यै~नमः
\item श्रीमते मार्तण्ड-विद्याघन-इन्द्रसरस्वत्यै~नमः
\item श्रीमते मूकशङ्कर-इन्द्रसरस्वत्यै~नमः
\item श्रीमते जाह्नवी-चन्द्रचूड-इन्द्रसरस्वत्यै~नमः
\item श्रीमते परिपूर्णबोध-इन्द्रसरस्वत्यै~नमः
\item श्रीमते सच्चित्सुख-इन्द्रसरस्वत्यै~नमः
\item श्रीमते कोङ्कण-चित्सुख-इन्द्रसरस्वत्यै~नमः
\item श्रीमते सच्चिदानन्दघन-इन्द्रसरस्वत्यै~नमः
\item श्रीमते प्रज्ञाघन-इन्द्रसरस्वत्यै~नमः
\item श्रीमते चिद्विलास-इन्द्रसरस्वत्यै~नमः
\item श्रीमते महादेव-इन्द्रसरस्वत्यै~नमः
\item श्रीमते पूर्णबोध-इन्द्रसरस्वत्यै~नमः
\item श्रीमते भक्तियोग-बोध-इन्द्रसरस्वत्यै~नमः
\item श्रीमते शीलनिधि-ब्रह्मानन्दघन-इन्द्रसरस्वत्यै~नमः
\item श्रीमते चिदानन्दघन-इन्द्रसरस्वत्यै~नमः
\item श्रीमते भाषापरमेष्ठि-सच्चिदानन्दघन-इन्द्रसरस्वत्यै~नमः
\item श्रीमते चन्द्रशेखर-इन्द्रसरस्वत्यै~नमः
\item श्रीमते बहुरूप-चित्सुख-इन्द्रसरस्वत्यै~नमः
\item श्रीमते चित्सुखानन्द-इन्द्रसरस्वत्यै~नमः
\item श्रीमते विद्याघन-इन्द्रसरस्वत्यै~नमः
\item श्रीमते धीरशङ्कर-इन्द्रसरस्वत्यै~नमः
\item श्रीमते सच्चिद्विलास-इन्द्रसरस्वत्यै~नमः
\item श्रीमते शोभन-महादेव-इन्द्रसरस्वत्यै~नमः
\item श्रीमते गङ्गाधर-इन्द्रसरस्वत्यै~नमः
\item श्रीमते ब्रह्मानन्दघन-इन्द्रसरस्वत्यै~नमः
\item श्रीमते आनन्दघन-इन्द्रसरस्वत्यै~नमः
\item श्रीमते पूर्णबोध-इन्द्रसरस्वत्यै~नमः
\item श्रीमते परमशिव-इन्द्रसरस्वत्यै~नमः
\item श्रीमते सान्द्रानन्द-बोध-इन्द्रसरस्वत्यै~नमः
\item श्रीमते चन्द्रशेखर-इन्द्रसरस्वत्यै~नमः
\item श्रीमते अद्वैतानन्दबोध-इन्द्रसरस्वत्यै~नमः
\item श्रीमते महादेव-इन्द्रसरस्वत्यै~नमः
\item श्रीमते चन्द्रचूड-इन्द्रसरस्वत्यै~नमः
\item श्रीमते विद्यातीर्थ-इन्द्रसरस्वत्यै~नमः
\item श्रीमते शङ्करानन्द-इन्द्रसरस्वत्यै~नमः
%\vspace{-0.9ex}
\begin{itemize}%\itemsep -0.9ex
\item श्रीमते अद्वैतब्रह्मानन्दाय~नमः
\item श्रीमते विद्यारण्याय~नमः
\item अन्येभ्यः विद्यातीर्थ-शङ्करानन्द-शिष्येभ्यो~नमः
\end{itemize}
\item श्रीमते पूर्णानन्द-सदाशिव-इन्द्रसरस्वत्यै~नमः
\item श्रीमते व्यासाचल-महादेव-इन्द्रसरस्वत्यै~नमः
\item श्रीमते चन्द्रचूड-इन्द्रसरस्वत्यै~नमः
\item श्रीमते सदाशिवबोध-इन्द्रसरस्वत्यै~नमः
\item श्रीमते परमशिव-इन्द्रसरस्वत्यै~नमः
%\vspace{-0.9ex}
\begin{itemize}%\itemsep -0.9ex
\item श्रीमते सदाशिवब्रह्म-इन्द्रसरस्वत्यै~नमः
\end{itemize}
\item श्रीमते विश्वाधिक-आत्मबोध-इन्द्रसरस्वत्यै~नमः
\item श्रीमते भगवन्नाम-बोध-इन्द्रसरस्वत्यै~नमः
\item श्रीमते अद्वैतात्मप्रकाश-इन्द्रसरस्वत्यै~नमः
\item श्रीमते महादेव-इन्द्रसरस्वत्यै~नमः
\item श्रीमते शिवगीतिमाला-चन्द्रशेखर-इन्द्रसरस्वत्यै~नमः
\item श्रीमते महादेव-इन्द्रसरस्वत्यै~नमः
\item श्रीमते चन्द्रशेखर-इन्द्रसरस्वत्यै~नमः
\item श्रीमते सुदर्शन-महादेव-इन्द्रसरस्वत्यै~नमः
\item श्रीमते चन्द्रशेखर-इन्द्रसरस्वत्यै~नमः
\item श्रीमते महादेव-इन्द्रसरस्वत्यै~नमः
\item श्रीमते चन्द्रशेखर-इन्द्रसरस्वत्यै~नमः
\item श्रीमते जयेन्द्रसरस्वत्यै~नमः
\item श्रीमते शङ्करविजयेन्द्रसरस्वत्यै~नमः
\item श्रीमते सत्य-चन्द्रशेखरेन्द्रसरस्वत्यै~नमः
\end{enumerate}

    \end{flushleft}



श्रीमत्-शङ्कर-भगवत्पादाचार्येभ्यो~नमः, नानाविध-परिमल-पत्र-पुष्पाणि समर्पयामि।\\


विज्यं॒ धनुः॑ कप॒र्दिनो॒ विश॑ल्यो॒ बाण॑वाꣳ उ॒त। अने॑शन्न॒\-स्येष॑व आ॒भुर॑स्य निष॒ङ्गथिः॑॥\\
ॐ ह्रीं न॒मः शि॒वाय॑। बला॑य॒ नमः॑। 

\twolineshloka*
{संसार-सागरं घोरम् अनन्त-क्लेश-भाजनम्}
{त्वामेव शरणं प्राप्य निस्तरन्ति मनीषिणः}

श्रीमत्-शङ्कर-भगवत्पादाचार्येभ्यो~नमः, धूपम् आघ्रापयामि।\\


या ते॑ हे॒तिर्मी॑ढुष्टम॒ हस्ते॑ ब॒भूव॑ ते॒ धनुः॑। तया॒ऽस्मान् वि॒श्वत॒स्त्वम॑य॒क्ष्मया॒ परि॑ब्भुज॥\\
ॐ ह्रीं न॒मः शि॒वाय॑। बल॑प्रमथनाय॒ नमः॑। 

\twolineshloka*
{नमस्तस्मै भगवते शङ्कराचार्य-रूपिणे}
{येन वेदान्त-विद्येयम् उद्धृता वेद-सागरात्}

श्रीमत्-शङ्कर-भगवत्पादाचार्येभ्यो~नमः, दीपं दर्शयामि।\\


ॐ भूर्भुवः॒ सुवः॑। + ब्र॒ह्मणे॒ स्वाहा᳚। नम॑स्ते अ॒स्त्वायु॑धा॒याना॑तताय धृ॒ष्णवे᳚। उ॒भाभ्या॑मु॒त ते॒ नमो॑ बा॒हुभ्यां॒ तव॒ धन्व॑ने॥\\
ॐ ह्रीं न॒मः शि॒वाय॑। सर्व॑भूतदमनाय॒ नमः॑। 

\twolineshloka*
{भगवत्पाद-पादाब्ज-पांसवः सन्तु सन्ततम्}
{अपारासार-संसार-सागरोत्तार-सेतवः}

श्रीमत्-शङ्कर-भगवत्पादाचार्येभ्यो~नमः, अमृतं महानैवेद्यं पानीयं च निवेदयामि। मध्ये मध्ये अमृतपानीयं समर्पयामि। अमृतापिधानमसि।\\
हस्तप्रक्षालनं समर्पयामि। पादप्रक्षालनं समर्पयामि। निवेदनानन्तरम् आचमनीयं समर्पयामि।\\


परि॑ ते॒ धन्व॑नो हे॒तिर॒स्मान्वृ॑णक्तु वि॒श्वतः॑। अथो॒ य इ॑षु॒धिस्तवा॒ऽ॒ऽ॒रे अ॒स्मन्नि धे॑हि॒ तम्॥\\
ॐ ह्रीं न॒मः शि॒वाय॑। म॒नोन्म॑नाय॒ नमः॑। 


श्रीमत्-शङ्कर-भगवत्पादाचार्येभ्यो~नमः, ताम्बूलं समर्पयामि।\\

नम॑स्ते अस्तु भगवन् विश्वेश्व॒राय॑ महादे॒वाय॑ त्र्यम्ब॒काय॑ त्रिपुरान्त॒काय॑ त्रिकाग्निका॒लाय॑ कालाग्निरु॒द्राय॑ नीलक॒ण्ठाय॑ मृत्युञ्ज॒याय॑ सर्वेश्व॒राय॑ सदाशि॒वाय॑ श्रीमन्महादे॒वाय॒ नमः॑॥

\twolineshloka*
{अज्ञान-तिमिरान्धस्य ज्ञानाञ्जन-शलाकया}
{चक्षुरुन्मीलितं येन तस्मै श्री-गुरवे नमः}

श्रीमत्-शङ्कर-भगवत्पादाचार्येभ्यो~नमः, नीराजनं दर्शयामि। नीराजनानन्तरम् आचमनीयं समर्पयामि।\\

श्रीमत्-शङ्कर-भगवत्पादाचार्येभ्यो~नमः, समस्तोपचारान् समर्पयामि।\\


\twolineshloka*
{यानि कानि च पापानि जन्मान्तर-कृतानि च}
{तानि तानि विनश्यन्ति प्रदक्षिण-पदे पदे}
\textbf{प्रदक्षिणं कृत्वा।}
\medskip

श्रीमत्-शङ्कर-भगवत्पादाचार्येभ्यो~नमः, प्रदक्षिणं करोमि।\\

\twolineshloka*
{आचार्यान् भगवत्पादान् षण्मत-स्थापकान् हितान्}
{परहंसान् नुमोऽद्वैत-स्थापकान् जगतो गुरून्}

श्रीमत्-शङ्कर-भगवत्पादाचार्येभ्यो~नमः, नमस्कारान् समर्पयामि।\\

\twolineshloka*
{गुरुर्ब्रह्मा गुरुर्विष्णुर्गुरुर्देवो महेश्वरः}
{गुरुरेव परं ब्रह्म तस्मै श्री-गुरवे नमः}

\twolineshloka*
{अखण्ड-मण्डलाकारं व्याप्तं येन चराचरम्}
{तत्-पदं दर्शितं येन तस्मै श्री-गुरवे नमः}

\twolineshloka*
{अनेक-जन्म-सम्प्राप्त-कर्म-बन्ध-विदाहिने}
{आत्म-ज्ञान-प्रदानेन तस्मै श्री-गुरवे नमः}

\twolineshloka*
{विशुद्ध-विज्ञान-घनं शुचिं हार्द-तमोनुदम्}
{दया-सिन्धुं लोक-बन्धुं शङ्करं नौमि सद्-गुरुम्}

\twolineshloka*
{देह-बुद्ध्या तु दासोऽस्मि जीव-बुद्ध्या त्वदंशकः}
{आत्म-बुद्ध्या त्वमेवाहमिति मे निश्चिता मतिः}

\fourlineindentedshloka*
{एकः शाखी शङ्कराख्यश्चतुर्धा}
{स्थानं भेजे ताप-शान्त्यै जनानाम्}
{शिष्य-स्कन्धैः शिष्य-शाखैर्महद्भिः}
{ज्ञानं पुष्पं यत्र मोक्षः प्रसूतिः}

\fourlineindentedshloka*
{गामाक्रम्य पदेऽधिकाञ्चि निबिडं स्कन्धैश्चतुर्भिस्तथा}
{व्यावृण्वन् भुवनान्तरं परिहरंस्तापं स-मोह-ज्वरम्}
{यः शाखी द्विज-संस्तुतः फलति तत् स्वाद्यं रसाख्यं फलं}
{तस्मै शङ्कर-पादपाय महते तन्मस्त्रि-सन्ध्यं नमः}

श्रीमत्-शङ्कर-भगवत्पादाचार्येभ्यो~नमः, स्तोत्रं समर्पयामि।\\

प्रार्थनाः समर्पयामि।\\

{गुरु-पादोदक-प्राशनम्\textsf{---}\hfill}

\twolineshloka*
{अविद्या-मूल-नाशाय जन्म-कर्म-निवृत्तये}
{ज्ञान-वैराग्य-सिद्ध्यर्थं गुरु-पादोदकं शुभम्}

\closesection

\input{../stotra-sangrahah/stotras/dhyanam/KanchiSwastiVachanam.tex}

\input{../stotra-sangrahah/stotras/shiva/Totakashtakam.tex}

\input{kathas/shankara-jayanti/shankara-prashasti}

\fourlineindentedshloka*
{जय जय शङ्कर हर हर शङ्कर}
{जय जय शङ्कर हर हर शङ्कर}
{काञ्ची-शङ्कर कामकोटि-शङ्कर}
{हर हर शङ्कर जय जय शङ्कर}

\fourlineindentedshloka*
{कायेन वाचा मनसेन्द्रियैर्वा}
{बुद्‌ध्याऽऽत्मना वा प्रकृतेः स्वभावात्}
{करोमि यद् यत् सकलं परस्मै}
{नारायणायेति समर्पयामि}

अनेन पूजनेन श्रीमत्-शङ्कर-भगवत्पादाचार्याः प्रीयन्ताम्। \\

ॐ तत् सद् ब्रह्मार्पणमस्तु।

\closesection


\input{kathas/shankara-jayanti/avatara-ghatta}
\input{kathas/shankara-jayanti/sarvajnapitharohana-ghatta}

\end{center}
\closesection

