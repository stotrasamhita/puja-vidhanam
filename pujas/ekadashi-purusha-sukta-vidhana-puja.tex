% !TeX program = XeLaTeX
% !TeX root = ../pujavidhanam.tex

\setlength{\parindent}{0pt}
\chapt{एकादशीव्रतम् — श्री-महाविष्णुपूजा}

\input{purvanga/vighneshwara-puja}

\sect{प्रधान-पूजा — एकादशीपूजा}

\twolineshloka*
{शुक्लाम्बरधरं विष्णुं शशिवर्णं चतुर्भुजम्}
{प्रसन्नवदनं ध्यायेत् सर्वविघ्नोपशान्तये}

प्राणान्  आयम्य।  ॐ भूः + भूर्भुवः॒ सुव॒रोम्।

\dnsub{सङ्कल्पः}

ममोपात्त-समस्त-दुरित-क्षयद्वारा श्री-परमेश्वर-प्रीत्यर्थं शुभे शोभने मुहूर्ते अद्य ब्रह्मणः
द्वितीयपरार्धे श्वेतवराहकल्पे वैवस्वतमन्वन्तरे अष्टाविंशतितमे कलियुगे प्रथमे पादे
जम्बूद्वीपे भारतवर्षे भरतखण्डे मेरोः दक्षिणे पार्श्वे शकाब्दे अस्मिन् वर्तमाने व्यावहारिके
प्रभवादि षष्टिसंवत्सराणां मध्ये \mbox{(~~~)}\see{app:samvatsara_names} नाम संवत्सरे उत्तरायणे / दक्षिणायने
(ग्रीष्म / वर्ष / शरद् / हेमन्त / शिशिर / वसन्त) ऋतौ  (मेष / वृषभ / मिथुन / कर्कटक / सिंह / कन्या / तुला /
वृश्चिक / धनुर् / मकर / कुम्भ / मीन) मासे (शुक्ल / कृष्ण) पक्षे (एकादश्यां / द्वादश्यां) शुभतिथौ
(इन्दु / भौम / बुध / गुरु / भृगु / स्थिर / भानु) वासरयुक्तायाम्
\mbox{(~~~)}\see{app:nakshatra_names} नक्षत्र \mbox{(~~~)}\see{app:yoga_names} नाम  योग  \mbox{(~~~)} करण युक्तायां च एवं\-गुण\-विशेषण\-विशिष्टायाम्
अस्याम् (एकादश्यां / द्वादश्यां) शुभतिथौ
अस्माकं सहकुटुम्बानां क्षेमस्थैर्य-धैर्य-वीर्य-विजय-आयुरारोग्य-ऐश्वर्याभिवृद्ध्यर्थम्
धर्मार्थकाममोक्ष\-चतुर्विधफलपुरुषार्थसिद्ध्यर्थं पुत्रपौत्राभि\-वृद्ध्यर्थम् इष्टकाम्यार्थसिद्ध्यर्थम्
मम इहजन्मनि पूर्वजन्मनि जन्मान्तरे च सम्पादितानां ज्ञानाज्ञानकृतमहा\-पातकचतुष्टय-व्यतिरिक्तानां रहस्यकृतानां प्रकाशकृतानां सर्वेषां पापानां सद्य अपनोदनद्वारा सकल-पापक्षयार्थं श्री-भूमि-नीला-समेत-श्री-महाविष्णु-प्रीत्यर्थं यावच्छक्ति ध्यानावाहनादि
षोडशोपचार-पूजां करिष्ये तदङ्गं कलशपूजां च करिष्ये।


श्रीविघ्नेश्वराय नमः यथास्थानं प्रतिष्ठापयामि।\\
(गणपति-प्रसादं शिरसा गृहीत्वा)

\renewcommand{\devaName}{विष्णु}
\input{purvanga/aasana-puja}

\input{purvanga/ghanta-puja}

\input{purvanga/kalasha-puja}

\input{purvanga/aatma-puja}

\input{purvanga/pitha-puja}

\input{purvanga/guru-dhyanam}

\begin{center}

\sect{षोडशोपचार-पूजा}
\phantomsection\label{sec:start_ekadashi_puja}
\renewcommand{\devAya}{श्री-भूमि-नीला-समेत-महा-विष्णवे नमः,}
\threelineshloka*
{ध्यायेत् चतुर्भुजं देवं शङ्खचक्रगदाधरम्}
{पीताम्बरयुगोपेतं लक्ष्मीयुक्तं विभूषितम्}
{लसत्कौस्तुभशोभाढ्यं मेघश्यामं सुलोचनम्}
\textbf{अस्मिन् बिम्बे श्री-भूमि-नीला-समेतं महाविष्णुं ध्यायामि।}
\medskip

\twolineshloka*
{स॒हस्र॑शीर्‌षा॒ पुरु॑षः। स॒ह॒स्रा॒क्षः स॒हस्र॑पात्}
{स भूमिं॑ वि॒श्वतो॑ वृ॒त्वा। अत्य॑तिष्ठद्दशाङ्गु॒लम्}
\textbf{अस्मिन् बिम्बे श्री-भूमि-नीला-समेतं महाविष्णुम् आवाहयामि।}
\medskip

\twolineshloka*
{पुरु॑ष ए॒वेदꣳ सर्वम्᳚। यद्भू॒तं यच्च॒ भव्यम्᳚}
{उ॒तामृ॑त॒त्वस्येशा॑नः। यदन्ने॑नाति॒रोह॑ति}
\textbf{\devAya{} आसनं समर्पयामि।}
\medskip

\twolineshloka*
{ए॒तावा॑नस्य महि॒मा। अतो॒ ज्यायाꣴ॑श्च॒ पूरु॑षः}
{पादो᳚ऽस्य॒ विश्वा॑ भू॒तानि॑। त्रि॒पाद॑स्या॒मृतं॑ दि॒वि}
\textbf{\devAya{} पाद्यं समर्पयामि।}
\medskip

\twolineshloka*
{त्रि॒पादू॒र्ध्व उदै॒त्पुरु॑षः। पादो᳚ऽस्ये॒हाऽऽभ॑वा॒त्पुनः॑}
{ततो॒ विश्व॒ङ्व्य॑क्रामत्। सा॒श॒ना॒न॒श॒ने अ॒भि}
\textbf{\devAya{} अर्घ्यं समर्पयामि।}
\medskip

\twolineshloka*
{तस्मा᳚द्वि॒राड॑जायत। वि॒राजो॒ अधि॒ पूरु॑षः}
{स जा॒तो अत्य॑रिच्यत। प॒श्चाद्भूमि॒मथो॑ पु॒रः}
\textbf{\devAya{} आचमनीयं समर्पयामि।}
\medskip

\twolineshloka*
{यत्पुरु॑षेण ह॒विषा᳚। दे॒वा य॒ज्ञमत॑न्वत}
{व॒स॒न्तो अ॑स्याऽऽसी॒दाज्यम्᳚। ग्री॒ष्म इ॒ध्मः श॒रद्ध॒विः}
\textbf{\devAya{} मधुपर्कं समर्पयामि।}
\medskip

\twolineshloka*
{स॒प्तास्या॑ऽऽसन्  परि॒धयः॑। त्रिः स॒प्त स॒मिधः॑ कृ॒ताः}
{दे॒वा यद्य॒ज्ञं त॑न्वा॒नाः। अब॑ध्न॒न् पु॑रुषं प॒शुम्}
\textbf{\devAya{} शुद्धोदकस्नानं समर्पयामि। स्नानानन्तरम् आचमनीयं समर्पयामि।}
\medskip

\twolineshloka*
{तं य॒ज्ञं ब॒र्{}हिषि॒ प्रौक्षन्॑। पुरु॑षं जा॒तम॑ग्र॒तः}
{तेन॑ दे॒वा अय॑जन्त। सा॒ध्या ऋष॑यश्च॒ ये}
\textbf{\devAya{} वस्त्रं समर्पयामि।}
\medskip

\twolineshloka*
{तस्मा᳚द्य॒ज्ञाथ्स॑र्व॒हुतः॑। सम्भृ॑तं पृषदा॒ज्यम्}
{प॒शूꣴस्ताꣴश्च॑क्रे वाय॒व्यान्॑। आ॒र॒ण्यान्ग्रा॒म्याश्च॒ ये}
\textbf{\devAya{} यज्ञोपवीतं समर्पयामि।}
\medskip

\twolineshloka*
{तस्मा᳚द्य॒ज्ञाथ्स॑र्व॒हुतः॑। ऋचः॒ सामा॑नि जज्ञिरे}
{छन्दासि जज्ञिरे॒ तस्मा᳚त्। यजु॒स्तस्मा॑दजायत}
\textbf{\devAya{} दिव्यपरिमलगन्धान् धारयामि।\\
गन्धस्योपरि हरिद्राकुङ्कुमं समर्पयामि। अक्षतान् समर्पयामि।}
\medskip

\twolineshloka*
{तस्मा॒दश्वा॑ अजायन्त। ये के चो॑भ॒याद॑तः}
{गावो॑ ह जज्ञिरे॒ तस्मा᳚त्। तस्मा᳚ज्जा॒ता अ॑जा॒वयः॑}
\textbf{\devAya{} पुष्पाणि समर्पयामि।  पुष्पैः पूजयामि।}

\dnsub{अङ्ग-पूजा}
\begin{longtable}{ll@{— }l}
१.&	ॐ वराहाय नमः & पादौ पूजयामि	\\
२.&	सङ्कर्षणाय नमः & गुल्फौ पूजयामि\\
३.&	कालात्मने नमः & जानुनी पूजयामि	\\
४.&	विश्वरूपाय नमः & जङ्घे पूजयामि\\
५.&	क्रोढाय नमः & ऊरू पूजयामि	\\
६.&	भोक्त्रे नमः & कटिं पूजयामि	\\
७.&	विष्णवे नमः & मेढ्रं पूजयामि		\\
८.&	हिरण्यगर्भाय नमः & नाभिं पूजयामि\\
९.&	श्रीवत्सधारिणे नमः & कुक्षिं पूजयामि	\\
१०.& परमात्मने नमः & हृदयं पूजयामि\\
११.& सर्वास्त्रधारिणे नमः & वक्षः पूजयामि	\\
१२.& वनमालिने नमः & कण्ठं पूजयामि\\
१३.& सर्वात्मने नमः & मुखं पूजयामि	\\
१४.& सहस्राक्षाय नमः & नेत्राणि पूजयामि\\
१५.& सुप्रभाय नमः & ललाटं पूजयामि	\\
१६.& चम्पकनासिकाय नमः & नासिकां पूजयामि	\\
१७.& सर्वेशाय नमः & कर्णौ पूजयामि	\\
१८.& सहस्रशिरसे नमः & शिरः पूजयामि\\
१९.& नीलमेघनिभाय नमः & केशान् पूजयामि	\\
२०.& महापुरुषाय नमः & सर्वाणि अङ्गानि पूजयामि	\\
\end{longtable}

\dnsub{चतुर्विंशति नामपूजा}
\begin{multicols}{2}
\begin{enumerate}
\item ॐ केशवाय नमः
\item ॐ नारायणाय नमः
\item ॐ माधवाय नमः
\item ॐ गोविन्दाय नमः
\item ॐ विष्णवे नमः	
\item ॐ मधुसूदनाय नमः
\item ॐ त्रिविक्रमाय नमः
\item ॐ वामनाय नमः
\item ॐ श्रीधराय नमः
\item ॐ हृषीकेशाय नमः
\item ॐ पद्मनाभाय नमः
\item ॐ दामोदराय नमः
\item ॐ सङ्कर्षणाय नमः
\item ॐ वासुदेवाय नमः
\item ॐ प्रद्युम्नाय नमः
\item ॐ अनिरुद्धाय नमः
\item ॐ पुरुषोत्तमाय नमः
\item ॐ अधोक्षजाय नमः
\item ॐ नृसिंहाय नमः
\item ॐ अच्युताय नमः
\item ॐ जनार्दनाय नमः
\item ॐ उपेन्द्राय नमः
\item ॐ हरये नमः
\item ॐ श्रीकृष्णाय नमः
\end{enumerate}
\end{multicols}
\clearpage

\begingroup
\setlength{\columnseprule}{1pt}
\let\chapt\sect
\begingroup
\input{../namavali-manjari/1000/Vishnu_1000.tex}
\endgroup
\input{../namavali-manjari/100/Krishna_108.tex}
\endgroup

\closesub

\sect{उत्तराङ्ग-पूजा}

\twolineshloka*
{यत्पुरु॑षं॒ व्य॑दधुः। क॒ति॒धा व्य॑कल्पयन्}
{मुखं॒ किम॑स्य॒ कौ बा॒हू। कावू॒रू पादा॑वुच्येते}
\textbf{\devAya{} धूपमाघ्रापयामि।}
\medskip

\twolineshloka*
{ब्रा॒ह्म॒णो᳚ऽस्य॒ मुख॑मासीत्। बा॒हू रा॑ज॒न्यः॑ कृ॒तः}
{ऊ॒रू तद॑स्य॒ यद्वैश्यः॑। प॒द्भ्याꣳ शू॒द्रो अ॑जायत}
उद्दी᳚प्यस्व जातवेदोऽप॒घ्नन्निर्ऋ॑तिं॒ मम॑।\\
प॒शूꣴश्च॒ मह्य॒माव॑ह॒ जीव॑नं च॒ दिशो॑ दिश॥ \\
मा नो॑ हिꣳसीज्जातवेदो॒ गामश्वं॒ पुरु॑षं॒ जग॑त्।\\
अबि॑भ्र॒दग्न॒ आग॑हि श्रि॒या मा॒ परि॑पातय॥ \\
\textbf{\devAya{} अलङ्कारदीपं सन्दर्शयामि।}
\medskip

ॐ भूर्भुवः॒ सुवः॑। + ब्र॒ह्मणे॒ स्वाहा᳚।
\twolineshloka*
{च॒न्द्रमा॒ मन॑सो जा॒तः। चक्षोः॒ सूर्यो॑ अजायत}
{मुखा॒दिन्द्र॑श्चा॒ग्निश्च॑। प्रा॒णाद्वा॒युर॑जायत}
\textbf{\devAya{} \mbox{(~~~)} निवेदयामि।}\\
{अमृतापिधानमसि। निवेदनानन्तरम् आचमनीयं समर्पयामि।}
\medskip

\twolineshloka*
{नाभ्या॑ आसीद॒न्तरि॑क्षम्। शी॒र्ष्णो द्यौः सम॑वर्तत}
{प॒द्भ्यां भूमि॒र्दिशः॒ श्रोत्रा᳚त्। तथा॑ लो॒काꣳ अ॑कल्पयन्}

\twolineshloka*
{पूगीफलसमायुक्तं नागवल्लीदलैर्युतम्}
{कर्पूरचूर्णसंयुक्तं ताम्बूलं प्रतिगृह्यताम्}
\textbf{\devAya{} कर्पूरताम्बूलं समर्पयामि।}
\medskip

\twolineshloka*
{वेदा॒हमे॒तं पुरु॑षं म॒हान्तम्᳚। आ॒दि॒त्यव॑र्णं॒ तम॑स॒स्तु पा॒रे}
{सर्वा॑णि रू॒पाणि॑ वि॒चित्य॒ धीरः॑। नामा॑नि कृ॒त्वाऽभि॒वद॒न्॒ यदास्ते᳚}
\textbf{\devAya{} समस्त-अपराध-क्षमापनार्थं कर्पूरनीराजनं दर्शयामि।}\\
कर्पूरनीरजनानन्तरम् आचमनीयं समर्पयामि।\medskip

\twolineshloka*
{धा॒ता पु॒रस्ता॒द्यमु॑दाज॒हार॑। श॒क्रः प्रवि॒द्वान्  प्र॒दिश॒श्चत॑स्रः}
{तमे॒वं वि॒द्वान॒मृत॑ इ॒ह भ॑वति। नान्यः पन्था॒ अय॑नाय विद्यते}

यो॑ऽपां पुष्पं॒ वेद॑। पुष्प॑वान् प्र॒जावा᳚न् पशु॒मान् भ॑वति।\\
च॒न्द्रमा॒ वा अ॒पां पुष्पम्᳚। पुष्प॑वान् प्र॒जावा᳚न् पशु॒मान् भ॑वति।\\
य ए॒वं वेद॑। यो॑ऽपामा॒यत॑नं॒ वेद॑। आ॒यत॑नवान् भवति।\medskip

ओं᳚ तद्ब्र॒ह्म। ओं᳚ तद्वा॒युः। ओं᳚ तदा॒त्मा।\\ ओं᳚ तथ्स॒त्यम्‌।
ओं᳚ तथ्सर्वम्᳚‌। ओं᳚ तत्पुरो॒र्नमः॥\medskip

अन्तश्चरति॑ भूते॒षु॒ गुहायां वि॑श्वमू॒र्तिषु। \\
त्वं यज्ञस्त्वं वषट्कारस्त्वमिन्द्रस्त्वꣳ\\ रुद्रस्त्वं विष्णुस्त्वं ब्रह्म त्वं॑ प्रजा॒पतिः। \\
त्वं त॑दाप॒ आपो॒ ज्योती॒ रसो॒ऽमृतं॒ ब्रह्म॒ भूर्भुवः॒ सुव॒रोम्‌॥\\
\textbf{\devAya{} वेदोक्तमन्त्रपुष्पाञ्जलिं समर्पयामि।}
\medskip

\twolineshloka*
{सुवर्णरजतैर्युक्तं चामीकरविनिर्मितम्}
{स्वर्णपुष्पं प्रदास्यामि गृह्यतां मधुसूदन}
\textbf{\devAya{} स्वर्णपुष्पं समर्पयामि।}
\medskip
\medskip

\twolineshloka*
{प्रदक्षिणं करोम्यद्य पापानि नुत माधव}
{मयार्पितान्यशेषाणि परिगृह्य कृपां कुरु}

\twolineshloka*
{यानि कानि च पापानि जन्मान्तरकृतानि च}
{तानि तानि विनश्यन्ति प्रदक्षिण-पदे पदे}
\textbf{प्रदक्षिणं कृत्वा।}
\medskip

\twolineshloka*
{नमस्ते देवदेवेश नमस्ते भक्तवत्सल}
{नमस्ते पुण्डरीकाक्ष वासुदेवाय ते नमः}

\twolineshloka*
{नमः सर्वहितार्थाय जगदाधाररूपिणे}
{साष्टाङ्गोऽयं प्रणामोऽस्तु जगन्नाथ मया कृतः}
\textbf{\devAya{} अनन्तकोटिप्रदक्षिणनमस्कारान् समर्पयामि।}
\medskip

\twolineshloka*
{य॒ज्ञेन॑ य॒ज्ञम॑यजन्त दे॒वाः। तानि॒ धर्मा॑णि प्रथ॒मान्या॑सन्}
{ते ह॒ नाकं॑ महि॒मानः॑ सचन्ते। यत्र॒ पूर्वे॑ सा॒ध्याः सन्ति॑ दे॒वाः}
\textbf{\devAya{} छत्त्रचामरादिसमस्तोपचारान् समर्पयामि।}
\medskip

\dnsub{अर्घ्यप्रदानम्}
ममोपात्त-समस्त-दुरित-क्षयद्वारा श्रीपरमेश्वरप्रीत्यर्थम् एकादशीपुण्यकाले महाविष्णुपूजान्ते क्षीरार्घ्यप्रदानं करिष्ये॥
\medskip

\twolineshloka*
{एकादश्यामुपोष्यैव पारणात् पूर्वकालतः}
{इदमर्घ्यं प्रदास्यामि गृहाण सुरवन्दित}
\hfill महाविष्णवे नमः इदमर्घ्यमिदमर्घ्यमिदमर्घ्यम्॥\medskip

\twolineshloka*
{नमोऽस्तु केशवादिभ्यः सर्वलोकैकवन्दिताः}
{इदमर्घ्यं प्रदास्यामि सुप्रीतो भव सर्वदा}
\hfill केशवादिभ्यः इदमर्घ्यमिदमर्घ्यमिदमर्घ्यम्।\medskip

\twolineshloka*
{कूर्मरूपाय देवाय मत्स्यरूप नमोऽस्तुते}
{नीलमेघस्वरूपाय अर्घ्यं दत्तं मया प्रभो}
\hfill विष्णवे नमः इदमर्घ्यमिदमर्घ्यमिदमर्घ्यम्॥\medskip

\twolineshloka*
{क्षीरोद्भवे महालक्ष्मि सुप्रसन्ने सुरेश्वरि}
{सर्वप्रदे जगद्वन्द्ये गृह्णीदार्घ्यमिदं रमे}
\hfill महालक्ष्म्यै नमः इदमर्घ्यमिदमर्घ्यमिदमर्घ्यम्।\\
अनेन अर्घ्यप्रदानेन भगवान् सर्वात्मकः\\ श्री-भूमि-नीला-समेतः श्री-महाविष्णुः प्रीयताम्।\medskip

\twolineshloka*
{हिरण्यगर्भगर्भस्थं हेमबीजं विभावसोः}
{अनन्तपुण्यफलदम् अतः शान्तिं प्रयच्छ मे}

एकादशीपुण्यकाले अस्मिन् मया क्रियमाण\\
महाविष्णुपूजायां यद्देयमुपायनदानं तत्प्रत्याम्नायार्थं हिरण्यं\\
श्री-भूमि-नीला-समेत-श्री-महाविष्णु-प्रीतिं कामयमानः\\
मनसोद्दिष्टाय ब्राह्मणाय सम्प्रददे नमः न मम।\\
अनया पूजया श्री-भूमि-नीला-समेतः श्री-महाविष्णुः प्रीयताम्।

\dnsub{विसर्जनम्}

\twolineshloka*
{यस्य स्मृत्या च नामोक्त्या तपः-पूजा-क्रियादिषु}
{न्यूनं सम्पूर्णतां याति सद्यो वन्दे तमच्युतम्}

\twolineshloka*
{इदं व्रतं मया देव कृतं प्रीत्यै तव प्रभो}
{न्यूनं सम्पूर्णतां यातु त्वत्प्रसादाज्जनार्द्दन}
\medskip

अस्मात् बिम्बात् श्री-भूमि-नीला-समेत-श्री-महाविष्णुं यथास्थानं प्रतिष्ठापयामि (अक्षतानर्पित्वा देवमुत्सर्जयेत्।)\\
अनया पूजया श्री-भूमि-नीला-समेतः श्री-महाविष्णुः प्रीयताम्।\medskip

\fourlineindentedshloka*
{कायेन वाचा मनसेन्द्रियैर्वा}
{बुद्‌ध्याऽऽत्मना वा प्रकृतेः स्वभावात्}
{करोमि यद्यत् सकलं परस्मै}
{नारायणायेति समर्पयामि}


ॐ तत्सद्ब्रह्मार्पणमस्तु।\medskip

\twolineshloka*
{सालग्रामशिलावारि पापहारि शरीरिणाम्}
{आजन्मकृतपापानां प्रायश्चित्तं दिने दिने}

\twolineshloka*
{अकालमृत्युहरणं सर्वव्याधिनिवारणम्}
{सर्वपापक्षयकरं विष्णुपादोदकं शुभम्}
इति तीर्थं पीत्वा शिरसि प्रसादं धारयेत्।

\end{center}

\dnsub{उत्तरस्मिन् दिने पारणम्}

\twolineshloka*
{अज्ञानतिमिरान्धस्य व्रतेनानेन केशव}
{प्रसीद सुमुखो नाथ ज्ञानदृष्टिप्रदो भव}

\hyperref[sec:start_ekadashi_puja]{\closesection}

\clearpage

\ifbool{individual}{
\begin{center}
    \centerline{\LARGE \bfseries पद्मपुराणान्तर्गत-एकादशी-माहात्म्यानि}
    \phantomsection\label{sec:ekadashi_mahatmyam_padma_puranam}
    \begin{itemize}
        \item \hyperref[sec:padma-unmilani]{उन्मीलनीव्रतम्}
        \item \hyperref[sec:padma-pakshavardhini]{पक्षवर्धिनी-एकादशी-माहात्म्यम्}
        \item \hyperref[sec:padma-jagaranamahima]{द्वादशी-एकादशी-जागरणमहिमावर्णनम्}
        \item \hyperref[sec:padma-margashirsha-krishna-utpanna]{मार्गशीर्ष-कृष्ण-उत्पन्ना-एकादशी-कृत-मुरवधः}
        \item \hyperref[sec:padma-margashirsha-shukla-mokshada]{मार्गशीर्ष-शुक्ल-मोक्षदा-एकादशी-माहात्म्यम्}
        \item \hyperref[sec:padma-pausha-krishna-saphala]{पौष-कृष्ण-सफला-एकादशी-माहात्म्यम्}
        \item \hyperref[sec:padma-pausha-shukla-putrada]{पौष-शुक्ल-पुत्रदा-एकादशी-माहात्म्यम्}
        \item \hyperref[sec:padma-magha-krishna-shattila]{माघ-कृष्ण-षट्तिला-एकादशी-माहात्म्यम्}
        \item \hyperref[sec:padma-magha-shukla-jaya]{माघ-शुक्ल-जया-एकादशी-माहात्म्यम्}
        \item \hyperref[sec:padma-phalguna-krishna-vijaya]{फाल्गुन-कृष्ण-विजया-एकादशी-माहात्म्यम्}
        \item \hyperref[sec:padma-phalguna-shuklamalaki]{फाल्गुन-शुक्लामलकी-एकादशी-माहात्म्यम्}
        \item \hyperref[sec:padma-chaitra-krishna-papamochani]{चैत्र-कृष्ण-पापमोचनी-एकादशी-माहात्म्यम्}
        \item \hyperref[sec:padma-chaitra-shukla-kamada]{चैत्र-शुक्ल-कामदा-एकादशी-माहात्म्यम्}
        \item \hyperref[sec:padma-vaishakha-krishna-varuthini ekadashi]{वैशाख-कृष्ण-वरूथिनी-एकादशी-माहात्म्यम्}
        \item \hyperref[sec:padma-vaishakha-shukla-mohini]{वैशाख-शुक्ल-मोहिनी-एकादशी-माहात्म्यम्}
        \item \hyperref[sec:padma-jyeshtha-krishnapara]{ज्येष्ठ-कृष्णापरा-एकादशी-माहात्म्यम्}
        \item \hyperref[sec:padma-jyeshtha-shukla-nirjala]{ज्येष्ठ-शुक्ल-निर्जला-एकादशी-माहात्म्यम्}
        \item \hyperref[sec:padma-ashadha-krishna-yogini]{आषाढ-कृष्ण-योगिनी-एकादशी-माहात्म्यम्}
        \item \hyperref[sec:padma-ashadha-shukla-shayani]{आषाढ-शुक्ल-शयनी-एकादशी-माहात्म्यम्}
        \item \hyperref[sec:padma-shravana-krishna-kamika]{श्रावण-कृष्ण-कामिका-एकादशी-माहात्म्यम्}
        \item \hyperref[sec:padma-shravana-shukla-putrada]{श्रावण-शुक्ल-पुत्रदा-एकादशी-माहात्म्यम्}
        \item \hyperref[sec:padma-bhadrapada-krishnaja]{भाद्रपद-कृष्णाजा-एकादशी-माहात्म्यम्}
        \item \hyperref[sec:padma-bhadrapada-shukla-padma]{भाद्रपद-शुक्ल-पद्मा-एकादशी-माहात्म्यम्}
        \item \hyperref[sec:padma-ashvina-krishnendira]{आश्विन-कृष्णेन्दिरा-एकादशी-माहात्म्यम्}
        \item \hyperref[sec:padma-ashvina-shukla-pashankusha]{आश्विन-शुक्ल-पाशाङ्कुशा-एकादशी-माहात्म्यम्}
        \item \hyperref[sec:padma-karttika-krishna-rama]{कार्त्तिक-कृष्ण-रमा-एकादशी-माहात्म्यम्}
        \item \hyperref[sec:padma-karttika-shukla-prabodhini]{कार्त्तिक-शुक्ल-प्रबोधिनी-एकादशी-माहात्म्यम्}
        \item \hyperref[sec:padma-purushottama-masasya krishna-kamala]{पुरुषोत्तम-मासस्य कृष्ण-कमला-एकादशी-माहात्म्यम्}
        \item \hyperref[sec:padma-purushottama-masasya shukla-kamada]{पुरुषोत्तम-मासस्य शुक्ल-कामदा-एकादशी-माहात्म्यम्}
    \end{itemize}
    \clearpage
    \sect{उन्मीलनीव्रतम्}
\label{sec:padma-unmilani}


\uvacha{महादेव उवाच}

\twolineshloka
{अतस्त्वां सम्प्रवक्ष्यामि उन्मीलनीमनुत्तमाम्}
{यस्याः श्रवणमात्रेण जन्मसंसारबन्धनात्}% ॥१॥

\twolineshloka
{पापात्मा मुच्यते पापैः स्वर्गलोके महीयते}
{देवताः पितरश्चैव तस्या गतिमवाप्नुयुः}% ॥२॥

\twolineshloka
{विद्यार्थी लभते विद्यां सर्वकामानवाप्नुयात्}
{तस्या व्रतान्न सन्देहः स्वर्गलोके महीयते}% ॥३॥

\twolineshloka
{स्वस्थानं तत्र वै प्राप्तः शिवलोके महीयते}
{अतस्त्वं कुरुषे राजन्वैष्णवानां तु पूजनम्}% ॥४॥

\twolineshloka
{वैष्णवानां तु ये राजन्सेवा कुर्वन्ति नित्यशः}
{तेषां दण्डं च कुरुषे नो वा तेषां नराधिपः}% ॥५॥

\twolineshloka
{भोजनानन्तरं तेषां भोजनं कुरुते नृप}
{तैरेव पूजितो विष्णुर्यैर्भक्त्या तु प्रपूजितः}% ॥६॥

\twolineshloka
{शालग्रामशिलाभूतां दत्त्वा मूर्धनि प्रत्यहम्}
{त्वं धारयसि भूपाल कण्ठे नित्यं सुभक्तितः}% ॥७॥

\twolineshloka
{धूपशेषं तु वै विष्णोर्भक्त्या भजसि भूपते}
{आरार्तिकं सदा कृत्वा भक्तानां वेदयेर्नृप}% ॥८॥

\twolineshloka
{शङ्खोदकं हरेर्मूर्ध्नि भ्रामयित्वा तु भक्तितः}
{नित्यं बिभर्षि शिरसि शेषं यच्छसि वैष्णवान्}% ॥९॥

\twolineshloka
{नैवेद्यं प्रत्यहं कृत्वा सर्वोपस्करसंयुतम्}
{विष्वक्सेनाय दत्वा वै स्वयं भुनक्षि वाडव}% ॥१०॥

\twolineshloka
{विष्णोर्निवेदितं चान्नं वैष्णवैः सह भुज्यते}
{नित्यं नामसहस्रेण भक्त्या स्तौषि जनार्दनम्}% ॥११॥

\twolineshloka
{दीपार्घदानं वै विष्णोः कुरुषे गीतनर्तनम्}
{श्यामाङ्कुरैः पूजयसे पूज्यन्ते नृपसत्तम}% ॥१२॥

\twolineshloka
{श्यामाङ्कुरैः सदा वत्स पूजनं चाति दुर्ल्लभम्}
{पृथ्वीदानसमं पुण्यं दूर्वया पूजने कृते}% ॥१३॥

\twolineshloka
{अतो वै नास्ति लोकेऽस्मिन्दूर्वायाः सदृशं भुवि}
{तया वै पूजनं कार्यं विष्णुसायुज्यमिच्छता}% ॥१४॥

\twolineshloka
{अतस्त्वं कुरुषे नित्यं पूजनं दूर्वया सह}
{यवाक्षतैर्विशेषेण पूजनं कुरुषेन वा}% ॥१५॥

\twolineshloka
{पक्षेपक्षे नृपश्रेष्ठ विधिवद्द्वादशीव्रतम्}
{यत्कृतं तु महाराज महापापप्रणाशनम्}% ॥१६॥

\twolineshloka
{मोक्षदं सुखदं चैव तथायुष्यप्रदं सदा}
{एतद्विष्णुव्रतं प्रोक्तं वैष्णवानां तु मोक्षदम्}% ॥१७॥

\twolineshloka
{गृहस्थानां तु सुखदं यतीनां मुक्तिदायकम्}
{सर्वरोगादिशमनं पवित्रं कायशोधनम्}% ॥१८॥

\twolineshloka
{व्रतमेतच्च कुरुषे नो वा चैव नराधिप}
{दशमीवेधरहितं कुरुषे जागरान्वितम्}% ॥१९॥

\twolineshloka
{तुलसीपत्रनिकरैर्नित्यं पूजयसे हरिम्}
{गोपीचन्दनजं पत्रं भाले वा नृपसत्तम}% ॥२०॥

\twolineshloka
{धारितं सर्वलोकानां पवित्रीकरणं नृप}
{अतस्त्वं च धारयसे गोपीचन्दनसम्भवम्}% ॥२१॥

\twolineshloka
{ब्रह्महा हेमहारी च मद्यपानी तथैव च}
{अगम्यगो महापापी तथा ह्यनृतभाषितः}% ॥२२॥

\twolineshloka
{ते सर्वे मुक्तिमायान्ति तिलकं धारणादृताः}
{बिभर्षि कण्ठे नित्यं त्वं धात्रीफलसमुद्भवाम्}% ॥२३॥

\twolineshloka
{मालां मुख्यायुतसमां तुलसीपत्रसम्भवाम्}
{शालग्रामशिलायुक्तां द्वारकायां समुद्भवाम्}% ॥२४॥

\twolineshloka
{नित्यं पूजयसे भूप भुक्तिमुक्तिफलप्रदाम्}
{पद्मसंज्ञं पुराणं वै पठसे पुरतो हरेः}% ॥२५॥

\twolineshloka
{चरितं दैत्यराज्यस्य प्रह्लादस्य च भूपते}
{वासरं वासुदेवस्य सवेधं कुर्वतो नरान्}% ॥२६॥

\twolineshloka
{निवारयसि भूपाल शास्त्रं दृष्ट्वा प्रयत्नतः}
{सवेधं वासरं विष्णोर्यस्मिन्राष्ट्रे प्रवर्तते}% ॥२७॥

\threelineshloka
{लिप्यते तेन पापेन राजा भवति नारकी}
{वेधं चतुर्विधं त्यक्त्वा समुपोष्य हरेर्दिनम्}
{कुलकोटिं समुद्धृत्य विष्णुलोके महीयते}% ॥२८॥

\uvacha{गौतम उवाच}

\twolineshloka
{शृणु भूपाल वक्ष्यामि वैष्णवाख्यं महाव्रतम्}
{यं श्रुत्वा पापिनः सर्वे मुक्तिमायान्ति तत्क्षणात्}% ॥२९॥

\onelineshloka
{द्वादशीसम्भवं पुण्यं मया ख्यातं न कस्यचित्}% ॥३०॥

\twolineshloka
{वैष्णवोऽसि महाराज भक्तो भागवते नृणाम्}
{वैष्णवं तु महागुह्यं तद्व्रतं त्वं निशामय}% ॥३१॥

\twolineshloka
{उन्मीलनी नाम पुरा भक्त्या मे माधवेन तु}
{कथिता सुप्रसन्नेन तां ते भूप वदाम्यहम्}% ॥३२॥

\twolineshloka
{एकादशी अहोरात्रं प्रभाते घटिका भवेत्}
{उन्मीलनी तु सा ज्ञेया विशेषेण हरिप्रिया}% ॥३३॥

\twolineshloka
{त्रैलोक्ये यानि तीर्थानि पुण्यान्यायतनानि च}
{कोट्यंशे नैव तुल्यानि मखा वेदास्तपांसि च}% ॥३४॥

\twolineshloka
{उन्मीलनी समं किञ्चिन्न भूतं न भविष्यति}
{प्रयागो न कुरुक्षेत्रं न काशी न च पुष्करः}% ॥३५॥

\twolineshloka
{शैलो हिमाचलो नैव न मेरुर्गन्धमादनः}
{शैलो न नीलनिषधो न विन्ध्यो नैव नैमिषम्}% ॥३६॥

\twolineshloka
{गोदावरी न कावेरी चन्द्रभागा न वेदिका}
{न तापी न पयोष्णी च न क्षिप्रा नैव चन्दना}% ॥३७॥

\twolineshloka
{चर्मण्वती च सरयूश्चन्द्रभागा न गण्डिका}
{गोमती च विपाशा च शोणाख्यश्च महानदः}% ॥३८॥

\twolineshloka
{किमत्र बहुनोक्तेन भूयोभूयो नराधिप}
{उन्मीलनीसमं किञ्चिन्न देवः केशवात्परः}% ॥३९॥

\twolineshloka
{उन्मीलनीमनु प्राप्य यैः कृतं केशवार्चनम्}
{पापचक्रसमूहस्य राशयः पतिताः क्षणात्}% ॥४०॥

\twolineshloka
{यस्मिन्मासे महीपाल तिथिरुन्मीलनी भवेत्}
{तन्मासनाम्ना गोविन्दः पूजनीयः प्रयत्नतः}% ॥४१॥

\twolineshloka
{जातरूपमयः कार्यं मासनाम्ना तु माधवः}
{स्वशक्त्या विश्वरूपस्तु श्रद्धाभक्तिसमन्वितः}% ॥४२॥

\twolineshloka
{पवित्रोदकसंयुक्तं पञ्चरत्नसमन्वितम्}
{गन्धपुष्पाक्षतैर्युक्तं कुम्भं स्रग्दामभूषितम्}% ॥४३॥

\twolineshloka
{पात्रं च सोदकं कार्यं गोधूमैश्चापि पूरितम्}
{नानारत्नैश्च संयुक्तं नानागन्धैः प्रपूजितम्}% ॥४४॥

\twolineshloka
{मल्लिकामोदसंयुक्तं जातीपुष्पैः प्रपूजितम्}
{श्वेताख्यैस्तन्दुलैश्चैव पूरणीयः प्रयत्नतः}% ॥४५॥

\twolineshloka
{प्रदद्याद्वस्त्रयुग्मं तु उपवीतं तु सोत्तरम्}
{उपानहौ तु राजर्षे आतपत्रं शिरोपरि}% ॥४६॥

\twolineshloka
{भोजनं जलपात्रं च सप्तधान्यं तिलैः सह}
{रूप्यं चैव तु कार्पासं पायसं मुद्रिका हरेः}% ॥४७॥

\twolineshloka
{धेनुर्वात्र तु दातव्या वत्सालङ्कारसंयुता}
{सुवर्णशृङ्गी रौप्यखुरी ताम्रपृष्ठी तथैव च}% ॥४८॥

\twolineshloka
{कांस्यदोहीं रत्नपुच्छीं दद्याद्वै गुरवे तदा}
{शय्यां सोपस्करां दद्यात्साधवे भक्तिपूर्वकम्}% ॥४९॥

\twolineshloka
{धूपं दीपं तु नैवेद्यं फलपत्रं निवेदयेत्}
{पूजनीयो महाभक्तैर्मन्त्रैरेभिस्तु केशवः}% ॥५०॥

\twolineshloka
{तुलसीपत्रसंयुक्तैः पुष्पैः कालोद्भवैस्तथा}
{मासनाम्नैव चरणौ जानुनी विष्णुरूपिणे}% ॥५१॥

\twolineshloka
{गुह्ये तु गुह्यपतये कटे वै पीतवाससे}
{ब्रह्ममूर्तिभृते नाभावुदरे विश्वयोनये}% ॥५२॥

\twolineshloka
{हृदये ज्ञानगम्याय कण्ठे वैकुण्ठमूर्तये}
{ऊर्ध्वगाय ललाटे तु बाहौ दक्षान्तकारिणे}% ॥५३॥

\twolineshloka
{उत्तमाङ्गे सुरेशाय सर्वाङ्गे सर्वमूर्तये}
{स्वनाम्ना चायुधादीनि पूजनीयानि भक्तितः}% ॥५४॥

\twolineshloka
{अर्घदानं प्रकर्तव्यं नालिकेरादिभिः समम्}
{शङ्खोपरि जले कृत्वा गन्धपुष्पाक्षतान्वितम्}% ॥५५॥

\twolineshloka
{सूत्रेणावेष्टितं कृत्वा दद्यादर्घं विधानतः}
{देवदेव महादेव श्रीकेशव जनार्दन}% ॥५६॥

\twolineshloka
{सुब्रह्मण्य नमस्तेऽस्तु पुण्यराशिविवर्धन}
{शोकमोहमहापापान्मामुद्धर भवार्णवात्}% ॥५७॥

\twolineshloka
{सुकृतं न कृतं किचिज्जन्मकोटिशतैरपि}
{तथापि मां महास्वामिन्मामुद्धर भवार्णवात्}% ॥५८॥

\twolineshloka
{व्रतेनानेन देवेश ये चान्ये मम पूर्वजाः}
{वियोनिं च गताश्चान्ये पापमृत्युवशं गताः}% ॥५९॥

\twolineshloka
{ये भविष्यन्ति येऽतीताः प्रेतलोकान्समुद्धर}
{श्रान्तोस्म्यहमधीनस्य भक्तिरव्यभिचारिणी}% ॥६०॥

\twolineshloka
{दत्तमर्घं मया तुभ्यं भक्त्या गृह्ण गदाधर}
{दत्वार्घं धूपदीपाद्यैर्नैवेद्यैर्विष्णुसम्भवैः}% ॥६१॥

\twolineshloka
{स्तोत्रैर्नीराजनैर्गीतैर्भृत्यैः सन्तोषयेद्धरिम्}
{वस्त्रैर्दानैश्च गोदानैर्भोजनैस्तोषयेद्गुरुम्}% ॥६२॥

\twolineshloka
{तथातथा विधातव्यं गुरुर्वै प्रीतिमाप्नुयात्}
{लोकानां तारणार्थाय धात्रा सृष्टो गुरुर्यतः}% ॥६३॥

\twolineshloka
{अतो वै गुरुपूजा च कर्तव्या वै प्रयत्नतः}
{अहितं यो नाशयति स्वहितं दर्शयेत्सदा}% ॥६४॥

\twolineshloka
{स गुरुः स च विज्ञेयः सर्वधर्मार्थकोविदः}
{अकुर्वन्वित्तशाठ्यं तु गुरवे तं निवेदयेत्}% ॥६५॥

\twolineshloka
{गुरोर्निवेदिते भूप परिपूर्णं भवेद्व्रतम्}
{कृत्वा दिवातनं कर्म भोजनं ब्राह्मणैः सह}% ॥६६॥

\twolineshloka
{कर्तव्यं नृपशार्दूल दिनं नेयं कथानकैः}
{अनेन विधिना यस्तु कुर्यादुन्मीलनीव्रतम्}% ॥६७॥

\onelineshloka
{कल्पकोटिसहस्राणि वसते विष्णुसन्निधौ}% ॥६८॥

॥इति श्रीपाद्मे महापुराणे पञ्चपञ्चाशत्साहस्र्यां संहितायामुत्तरखण्डे उमापतिनारदसंवादान्तर्गतकृष्णयुधिष्ठिरसंवादे उन्मीलनीव्रतं नाम सप्तत्रिंशोऽध्यायः॥३७॥


\hyperref[sec:ekadashi_mahatmyam_padma_puranam]{\closesub}
\clearpage

\sect{पक्षवर्धिनी-एकादशी-माहात्म्यम्}
\label{sec:padma-pakshavardhini}


\uvacha{नारद उवाच}

\twolineshloka
{कीदृशी स्यान्महादेव पक्षवर्धनिसंज्ञका}
{यया वै कृतया जन्तुर्महपापात्प्रमुच्यते}% ॥१॥

\uvacha{श्रीमहादेव उवाच}

\onelineshloka*
{अमा वा यदि वा पूर्णा सम्पूर्णा जायते तदा}

\twolineshloka
{भूत्वा वै नाडिकाषष्ठिर्जायते प्रतिपद्दिने}
{अश्वमेधायुतैस्तुल्या सा भवेत्पक्षवर्धिनी}% ॥२॥

\uvacha{नारद उवाच}

\twolineshloka
{पूजाविधिं तु पृच्छामि साम्प्रतं देवसत्तम}
{यत्कृते तु महादेव महाफलमवाप्नुयात्}% ॥३॥

\uvacha{महादेव उवाच}

\twolineshloka
{पूजाविधिं प्रवक्ष्यामि साम्प्रतं द्विजनन्दन}
{पूजिते चार्चिते विष्णौ फलं प्राप्नोत्यसंशयः}% ॥४॥

\twolineshloka
{येन पूजाविधानेन तुष्टिं प्राप्नोति माधवः}
{अव्रणं जलपूर्णं च कुम्भं चन्दनचर्चितम्}% ॥५॥

\twolineshloka
{पञ्चरत्नसमायुक्तं पुष्पमालाभिवेष्टितम्}
{स्थाप्यं ताम्रमयं पात्रं सगोधूमं घटोपरि}% ॥६॥

\twolineshloka
{सौवर्णं कारयेद्देवं माससंज्ञाभिनामकम्}
{पञ्चामृतेन विधिना स्नपनं सुमनोरमम्}% ॥७॥

\twolineshloka
{कारयेद्देवदेवेशं जगन्नाथं जगत्पतिम्}
{विलेपनं तु कर्तव्यं कुङ्कुमागरुचन्दनैः}% ॥८॥

\twolineshloka
{वस्त्रयुग्मं च दातव्यं छत्रोपानहसंयुतम्}
{पूजयेद्देवताधीशं कुम्भपात्रोपरिस्थितम्}% ॥९॥

\twolineshloka
{पद्मनाभाय वै पादौ जानुनी विश्वमूर्तये}
{ऊरू वै ज्ञानगम्याय कटी दानप्रदाय च}% ॥१०॥

\twolineshloka
{उदरं विश्वनाथाय हृदयं श्रीधराय च}
{कण्ठं कौस्तुभकण्ठाय बाहू क्षत्रान्तकारिणे}% ॥११॥

\twolineshloka
{ललाटं व्योममूर्ध्ने तु शिरो वै सर्वरूपिणे}
{स्वनाम्ना चैव कमलां सर्वाङ्गीं दिव्यरूपिणीम्}% ॥१२॥

\twolineshloka
{एवं विधिवत्सम्पूज्य ततोऽर्घं दापयेत्सुधीः}
{नालिकेरेण शुभ्रेण देवदेवस्य चक्रिणः}% ॥१३॥

\twolineshloka
{अनेनैवार्घदानेन सम्पूर्णं जायते व्रतम्}
{संसारार्णवमग्नं मां समुद्धर जगत्पते}% ॥१४॥

\twolineshloka
{त्वमीशः सर्वलोकानां त्वं साक्षाच्च जगत्पतिः}
{गृहाणार्घं मया दत्तं पद्मनाभ नमोस्तु ते}% ॥१५॥

\twolineshloka
{नैवेद्यानि सुहृद्यानि षड्रसानि विशेषतः}
{देयानि तु विशेषेण केशवाय सुभक्तितः}% ॥१६॥

\twolineshloka
{नागपत्रं सकर्पूरं दद्याद्देवस्य भक्तितः}
{घृतेन दीपकं कुर्यात्तिलतैलेन वा पुनः}% ॥१७॥

\twolineshloka
{कृत्वा सम्यग्विधानेन गुरोः पूजां प्रकारयेत्}
{वस्त्राणि चैव चोष्णीषं कञ्चुकं च प्रदापयेत्}% ॥१८॥

\twolineshloka
{दक्षिणां च यथाशक्त्या गुरुवे सम्प्रदापयेत् }
{भोजनं चैव ताम्बूलं दत्त्वा चार्घं प्रदापयेत्}% ॥१९॥

\twolineshloka
{स्ववित्तस्यानुमानेन यथाशक्त्या तु निर्धनैः}
{कार्या सम्यक्प्रयत्नेन द्वादशी पक्षवर्धिनी}% ॥२०॥

\twolineshloka
{ततो जागरणं कुर्याद्गीतनृत्यसमन्वितम्}
{पुराणपाठसहितं हास्याह्लादसमन्वितम्}% ॥२१॥

\twolineshloka
{स्तुवन्ति च प्रशंसन्ति जागरं चक्रपाणिनः}
{नित्योत्सवो भवेत्तेषां गृहे वै दशजन्मसु}% ॥२२॥

\twolineshloka
{अतो धन्यतमा चेयं कर्तव्या पक्षवर्धनी}
{कृत्वा तु सकलं पुण्यं फलं प्राप्नोत्यसंशयम्}% ॥२३॥

\twolineshloka
{पक्षवर्धनी माहात्म्यं ये शृण्वन्ति मनीषिणः}
{तैः कृतं सत्कृतं सर्वं यावदाभूतसम्प्लवम्}% ॥२४॥

\twolineshloka
{पञ्चाग्निसाधने पुण्यं यच्च स्यात्तीर्थसाधने}
{तत्पुण्यं समवाप्नोति विष्णोर्जागरकारणात्}% ॥२५॥

\twolineshloka
{पक्षवर्धनिका पुण्या पवित्रा पापनाशिनी}
{उपवासकृता विप्र हत्याकोटिविनाशिनी}% ॥२६॥

\twolineshloka
{वसिष्ठेन कृता पूर्वं भारद्वाजेन वै मुने}
{ध्रुवेण चाम्बरीषेण कृतेयं विष्णुवल्लभा}% ॥२७॥

\twolineshloka
{इयं काशीसमा पुण्या इयं वै द्वारका समा}
{उपोषिता च भक्तेन वाञ्छितं च ददात्यसौ}% ॥२८॥

\twolineshloka
{इयं धन्या धन्यतमा हत्यायुतविनाशिनी}
{कर्तव्या तु विशेषेण वैष्णवैर्ज्ञानतत्परैः}% ॥२९॥

\twolineshloka
{अहो सर्वेश्वरो देवः संसेव्यो व्रततत्परैः}
{किमन्यद्बहुनोक्तेन कर्तव्यं व्रतमुत्तमम्}% ॥३०॥

\twolineshloka
{यथा च वर्धते चन्द्रः सिते पक्षे विशेषतः}
{तथा वै वर्धते भक्त कारणात्पक्षवर्धिनी}% ॥३१॥

\twolineshloka
{सूर्योदये यथा ध्वान्तं नश्यते तत्क्षणादपि}
{तथाघं नाशमाप्नोति करणात्पक्षवर्धिनी}% ॥३२॥

॥इति श्रीपाद्मे महापुराणे पञ्चपञ्चाशत्साहस्र्यां संहितायामुत्तरखण्डे उमापतिनारदसंवादान्तर्गतकृष्णयुधिष्ठिरसंवादे पक्षवर्धिनी-एकादशी-माहात्म्यं नाम नामाष्टत्रिंशोऽध्यायः॥३८॥


\hyperref[sec:ekadashi_mahatmyam_padma_puranam]{\closesub}
\clearpage

\sect{द्वादशी-एकादशी-जागरणमहिमावर्णनम्}
\label{sec:padma-jagaranamahima}


\uvacha{महादेव उवाच}

\twolineshloka
{शृणु नारद वक्ष्यामि माहात्म्यं जागरस्य च}
{यच्छ्रुत्वा मुक्तिमाप्नोति महापापी न संशयः}% ॥१॥

\uvacha{नारद उवाच}

\twolineshloka
{अहो विश्वेश्वरो विष्णुः पवित्रीकरणः सदा}
{तस्योपवासमाहात्म्यं श्रुतं हि त्वन्मुखाच्छिव}% ॥२॥

\twolineshloka
{तथापि श्रोतुमिच्छामि माहात्म्यं जागरस्य तु }
{कीदृक्जागरमाहात्म्यं रात्रो भक्तिस्तु  कीदृशी}% ॥३॥

\threelineshloka
{प्रहरेषु च या पूजा वद विश्वेश्वर प्रभो}
{त्वं लोकेषु सदा पूज्यस्त्वं हि देवो जनार्दनः}
{त्वं हि विश्वेश्वरो देवो यतो भक्तिर्जनार्दने}% ॥४॥

\twolineshloka
{सर्वेषां चैव भक्तानां त्वं च श्रेष्ठ उमापतिः}
{लोकेस्मिन्सर्वदा भक्त्या तवाख्या वर्तते सदा}% ॥५॥

\twolineshloka
{अतो येन प्रकारेण लोकानां मुक्तिरेव च}
{विश्वेश्वर वद त्वं तु माहात्म्यं जागरस्य तु}% ॥६॥

\uvacha{महादेव उवाच}

\twolineshloka
{एकादश्यां जनो विष्णुं रात्रौ सम्पूज्य भक्तितः}
{कुर्याज्जागरणं विष्णोः पुरतो वैष्णवैः सह}% ॥७॥

\twolineshloka
{गीतं वाद्यं तथा नृत्यं पुराणपठनं तथा}
{धूपं दीपं च नैवेद्यं पुष्पं गन्धानुलेपनम्}% ॥८॥

\twolineshloka
{फलमर्घं तथा श्रद्धा दानमिन्द्रियसंयमम्}
{सत्यान्वितं च विप्रेन्द्र वचोयुक्तं क्रियान्वितम्}% ॥९॥

\twolineshloka
{विनिद्रं च मुदायुक्तो यः करोति नरः सदा}
{सर्वपापविनिर्मुक्तो जायते विष्णुवल्लभः}% ॥१०॥

\twolineshloka
{रात्रौ जागरणे प्राप्ते निद्रां कुर्वन्ति वैष्णवाः}
{हारितं चोपवासं तैर्व्रतं वै विष्णुसंज्ञकम्}% ॥११॥

\twolineshloka
{ये कुर्वन्ति नराः प्राज्ञ जागरे विष्णुसंज्ञके}
{जागरं कृष्णभावेन न स्वपन्ति कदाचन}% ॥१२॥

\twolineshloka
{कृष्णस्य नाम मनसा वदन्ति च पुनः पुनः}
{ते तु धन्यतमा ज्ञेया अस्यां रात्रौ विशेषतः}% ॥१३॥

\twolineshloka
{क्षणेक्षणे तु गोदानं घट्यां चैव चतुर्गुणम्}
{प्रहरे कोटिगुणितं चतुर्यामेष्वसङ्ख्यकम्}% ॥१४॥

\twolineshloka
{जागरे निमिषार्धे तु केशवाग्रे विशेषतः}
{तत्फलं कोटिगुणितं तस्य सङ्ख्या न विद्यते}% ॥१५॥

\twolineshloka
{नर्तनं कुरुते यस्तु केशवाग्रे नरोत्तमः}
{न फलं हीयते तस्य आजन्ममरणान्तिकम्}% ॥१६॥

\twolineshloka
{साश्चर्यं चैव सोत्साहं पापालापादिवर्जितम्}
{प्रदक्षिणसमायुक्तं नमस्कारपुरःसरम्}% ॥१७॥

\twolineshloka
{नीराजनसमायुक्तमनिर्विण्णेन चेतसा}
{यामेयामे महाभाग कुर्यादारार्तिकं हरेः}% ॥१८॥

\twolineshloka
{षड्विंशगुणसंयुक्तमेकादश्यां च जागरम्}
{यः करोति नरो भक्त्या न पुनर्जायते भुवि}% ॥१९॥

\twolineshloka
{य एवं कुरुते भक्त्या वित्तशाठ्यविवर्जितः}
{जागरं वासरे विष्णोर्लीयते परमात्मनि}% ॥२०॥

\twolineshloka
{धनवान्वित्तशाठ्येन यः करोति प्रजागरम्}
{तेनात्माहारितो नूनं कितवेन दुरात्मना}% ॥२१॥

\twolineshloka
{विष्णुजागरणे प्राप्ते उपहासं करोति यः}
{षष्टिवर्षसहस्राणि विष्ठायां जायते कृमिः}% ॥२२॥

\twolineshloka
{वेदविद्ब्राह्मणो यस्तु नर्तनेन विशेषतः}
{उपहासपरः प्राप्तः स वै चाण्डाल उच्यते}% ॥२३॥

\twolineshloka
{निमिषं निमिषार्धं वा यः करोति प्रजागरम्}
{धर्मार्थकाममोक्षाणां प्राप्नोति पदमव्ययम्}% ॥२४॥

\twolineshloka
{वेदशास्त्ररतो नित्यं नित्यं वै यज्ञयाजकः}
{रात्रौ जागरणे प्राप्ते निन्दां कुर्वन्व्रजत्यधः}% ॥२५॥

\twolineshloka
{मम पूजां प्रकुर्वाणो विष्णुनिन्दासु तत्परः}
{एकविंशत्कुलेनैव नरकं प्रतिपद्यते}% ॥२६॥

\threelineshloka
{विष्णुः शिवः शिवो विष्णुरेकमूर्तिर्द्विधा स्थिताः}
{तस्मात्सर्वप्रकारेण नैव निन्दां प्रकारयेत्}
{दष्टाः कलिभुजङ्गेन स्वपन्ति मधुहाहनि}% ॥२७॥

\twolineshloka
{कुर्वन्ति जागरं नैव मायया ते च मोहिताः}
{प्राप्ता एकादशी येषां कलौ जागरणं विना}% ॥२८॥

\twolineshloka
{ते विनष्टा न सन्देहो यस्माज्जीवितमध्रुवम्}
{उद्धृतं नेत्रयुग्मं तु दत्त्वा वै वैष्णवं पदम्}% ॥२९॥

\twolineshloka
{कृतं ये नैव पश्यन्ति पापिनो हरिजागरम्}
{अभावे वाचकस्याथ गीतं नृत्यं तु कारयेत्}% ॥३०॥

\twolineshloka
{वाचके सति देवर्षे पुराणं प्रथमं पठेत्}
{अश्वमेधसहस्रस्य वाजपेयायुतस्य च}% ॥३१॥

\twolineshloka
{पुण्यं कोटिगुणं वत्स विष्णोर्जागरणे कृते}
{पितृपक्षे मातृपक्षे भार्यापक्षे तु वाडव}% ॥३२॥

\twolineshloka
{कुलानुद्धरते चैतान्कृत्वा जागरणं हरेः}
{उपोषणदिने विद्धे जागरं पूजनं हरेः}% ॥३३॥

\twolineshloka
{वृथादानादिकं सर्वं कृतघ्नेषु कृतं यथा}
{उपोषणदिने विद्धे प्रारब्धे जागरे स्थितिम्}% ॥३४॥

\twolineshloka
{विहाय स्थानं तद्विष्णुः शापं दत्वा प्रगच्छति}
{अविद्धे वासरे विष्णोर्ये कुर्वन्ति प्रजागरम्}% ॥३५॥

\twolineshloka
{तेषां मध्ये तु तुष्टः सन्नृत्यं तु कुरुते हरिः}
{यावद्दिनानि कुरुते जागरं केशवाग्रतः}% ॥३६॥

\twolineshloka
{युगानि तानि तावन्ति विष्णुलोके महीयते}
{यावद्दिनानि वसते विना जागरणं हरेः}% ॥३७॥

\twolineshloka
{तावद्वर्षसहस्राणि रौरवान्न निवर्तते}
{एकदश्यां शयानस्तु विना जागरणं हरेः}% ॥३८॥

\twolineshloka
{मूकवत्तिष्ठते यो वै गानं पाठं न वाचरेत्}
{सप्तजन्मानि मूकत्वं जायतेऽजागरे हरेः}% ॥३९॥

\twolineshloka
{यो न नृत्यति मूढात्मा पुरतो जागरे हरेः}
{पङ्गुत्वं तस्य जानीयात्सप्तजन्मानि वाडव}% ॥४०॥

\twolineshloka
{यः पुनः कुरुते गीतं नृत्यं जागरणं हरेः}
{ब्राह्मं पदं मदीयं च सत्यं वै तस्य वैष्णवम्}% ॥४१॥

\twolineshloka
{यः प्रबोधयते लोकान्विष्णोर्जागरणे रतः}
{वसेच्चिरं तु वैकुण्ठे पितृभिः सह वैष्णवः}% ॥४२॥

\twolineshloka
{मतिं प्रयच्छते यस्तु हरेर्जागरणं प्रति}
{षष्टिवर्षसहस्राणि श्वेतद्वीपे वसेन्नरः}% ॥४३॥

\twolineshloka
{यत्किञ्चित्क्रियते पापं कोटिजन्मनि मानवैः}
{श्रीकृष्णजागरे सर्वं रात्रौ नश्यति वाडव}% ॥४४॥

\twolineshloka
{शालग्रामशिलाग्रे ये कुर्वन्ति प्रतिजागरम्}
{यामेयामे फलं प्रोक्तं कोट्यैन्दवसमुद्भवम्}% ॥४५॥

\twolineshloka
{सम्प्राप्ते वासरे विष्णोर्ये न कुर्वन्ति जागरम्}
{वृथा स्यात्तत्कृतं तेषां वैष्णवानां च निन्दया}% ॥४६॥

\twolineshloka
{कामार्थौ सम्पदः पुत्राः कीर्तिर्लोकाश्च शाश्वताः}
{यज्ञायुतैर्न लभ्यन्ते द्वादशीजागरं विना}% ॥४७॥

\twolineshloka
{मतिर्न जायते यस्य द्वादश्यां जागरं प्रति}
{न हि तस्याधिकारोऽस्ति पूजने केशवस्य हि}% ॥४८॥

\twolineshloka
{यावत्पदानि चलति केशवायतनं प्रति}
{अश्वमेधसमानि स्युः जागरार्थं प्रगच्छतः}% ॥४९॥

\twolineshloka
{पादयोः पतितं यावद्धरण्यां पांशु गच्छताम्}
{तावद्वर्षसहस्राणि जागरो वसते दिवि}% ॥५०॥

\twolineshloka
{तस्माद्गृहात्प्रगन्तव्यं जागरे केशवालये}
{कलौ मलविनाशाय द्वादशीद्वादशीषु च}% ॥५१॥

\twolineshloka
{परापवादसंयुक्तं मनः प्रासाद वर्जितम्}
{शास्त्रहीनमगान्धर्वं तथा दीपविवर्जितम्}% ॥५२॥

\twolineshloka
{शक्त्योपचाररहितमुदासीनं सनिन्दनम्}
{कलियुक्तं विशेषेण जागरं नवधा मतम्}% ॥५३॥

\twolineshloka
{सशास्त्रं जागरं यच्च नृत्यगान्धर्वसंयुतम्}
{सवाद्यं तालासंयुक्तं सदीपं मधुभिर्युतम्}% ॥५४॥

\twolineshloka
{उच्चारैस्तु समायुक्तं यथोक्तैर्भक्तिभावितैः}
{प्रसन्नं तुष्टिजननं सम्मूढं लोकरञ्जनम्}% ॥५५॥

\twolineshloka
{गुणैर्द्वादशभिर्युक्तं जागरं माधवप्रियम्}
{कर्तव्यं तत्प्रयत्नेन पक्षयोः शुक्लकृष्णयोः}% ॥५६॥

\twolineshloka
{किं व्रतैर्बहुभिश्चीर्णैस्तीर्थवासेन तस्य किम्}
{द्वादशीवासरे प्राप्ते न कुर्याज्जागरं हरेः}% ॥५७॥

\twolineshloka
{प्रवासेन त्यजेद्यस्तु पथिस्विन्नोऽपि वाडव}
{जागरं वासुदेवस्य द्वादश्यां तु समे प्रियः}% ॥५८॥

\twolineshloka
{मद्भक्तो न हरेः कुर्याज्जागरं पापमोहितः}
{व्यर्थं मत्पूजनं तस्य मत्पूज्यं यो न पूजयेत्}% ॥५९॥

\twolineshloka
{न शैवो न च सौरोऽसौ न शाक्तो गणसेवकः}
{यो भुङ्क्ते वासरे विष्णोर्ज्ञेयः पश्वधिको हि सः}% ॥६०॥

\twolineshloka
{विप्रियं च कृतं तेन दुष्टेनैव च पापिना}
{मद्भक्तिबलमाश्रित्य यो भुङ्क्ते वै हरेर्दिने}% ॥६१॥

\twolineshloka
{सबाह्याभ्यन्तरं देहं वेष्टितं पापकोटिभिः}
{मुच्यन्ते वासरे विष्णोर्ये कुर्वन्ति प्रजागरम्}% ॥६२॥

\twolineshloka
{कूर्परं यमदूतानां दत्तं तेन यमस्य च}
{कृत्वा जागरणं विष्णोरविद्धं द्वादशीव्रतम्}% ॥६३॥

\twolineshloka
{स्वर्गापेक्षा मुनिश्रेष्ठ मुक्ता ते नैव संशयः}
{वाञ्छितं नारकं सौख्यं विद्धं कृत्वा हरेर्दिनम्}% ॥६४॥

\twolineshloka
{निहताः पितरस्तेन देवानां वै वधः कृतः}
{दत्तं राज्यं तु दैत्यानां कृत्वा विद्धं हरेर्दिनम्}% ॥६५॥

\twolineshloka
{यो नृत्यति प्रहृष्टात्मा कृत्वा वै करताडनम्}
{गीतं कुर्वन्मुखेनापि दर्शयन्कौतुकान्बहून्}% ॥६६॥

\twolineshloka
{पुरतो वासुदेवस्य रात्रौ जागरणे स्थितः}
{पठन्कृष्णचरित्राणि रञ्जयन्वैष्णवान्गणान्}% ॥६७॥

\twolineshloka
{मुखेन कुरुते वाद्यं सम्प्रहृष्टतनूरुहः}
{दर्शयन्विविधान्भृत्यान्स्वेच्छालापान्प्रकारयन्}% ॥६८॥

\twolineshloka
{भावैरेतैर्नरो यस्तु कुरुते जागरे हरेः}
{निमिषेनिमिषे पुण्यं तीर्थकोटिफलं स्मृतम्}% ॥६९॥

\twolineshloka
{अनुद्विग्नमना यस्तु धूपनीराजनं हरेः}
{कुरुते जागरे रात्रौ सप्तद्वीपाधिपो भवेत्}% ॥७०॥

\twolineshloka
{यानि कानि च पापानि ब्रह्महत्यासमानि च}
{कृष्णा ह जागरात्तानि विलयं यान्ति खण्डशः}% ॥७१॥

\twolineshloka
{एकतः क्रतवः सर्वे समाप्तवरदक्षिणाः}
{एकतो देवदेवस्य जागरः कृष्णवल्लभः}% ॥७२॥

\twolineshloka
{तत्र काशी पुष्करं च प्रयागं नैमिषं गया}
{शालग्राममहाक्षेत्रमर्बुदारण्यमेव च}% ॥७३॥

\twolineshloka
{पौष्करं मथुरा तत्र सर्वतीर्थानि चैव हि}
{यज्ञा वेदाश्च चत्वारो व्रजन्ति हरिजागरम्}% ॥७४॥

\twolineshloka
{गङ्गा सरस्वती तापी यमुना च शतद्रुका}
{चन्द्रभागा वितस्ता च नद्यः सर्वास्तु तत्र वै}% ॥७५॥

\twolineshloka
{सरांसि च ह्रदाः सर्वे समुद्राः सर्व एव हि}
{एकादश्यां द्विजश्रेष्ठ गच्छन्ति कृष्णजागरम्}% ॥७६॥

\twolineshloka
{स्पृहणीया हि देवानां ये नराः कृष्णजागरे}
{नृत्यन्ति गीतं कुर्वन्ति वीणावाद्यप्रहर्षिताः}% ॥७७॥

\twolineshloka
{एवं जागरणं कृत्वा सम्पूज्य च महाहरिम्}
{द्वादश्यां पारणं कृत्वा स्वशक्त्या वैष्णवैः सह}% ॥७८॥

\uvacha{महादेव उवाच}

\twolineshloka
{शृणु ब्रह्मन्प्रवक्ष्यामि द्वादशीमाहात्म्यमुत्तमम्}
{द्वादशी तु सदा ज्ञेया पुत्रदा मोक्षदायिनी}% ॥७९॥

\twolineshloka
{प्रातः स्नात्वा हरिं पूज्य उपवासं समर्पयेत्}
{अज्ञानतिमिरान्धस्य व्रतेनानेन केशव}% ॥८०॥

\twolineshloka
{प्रसीद सुमुखो भूत्वा ज्ञानदृष्टिप्रदो भव}
{पारणं च ततः कुर्याद्यथासम्भवमग्रतः}% ॥८१॥

\twolineshloka
{अत ऊर्ध्वं यथेष्टं तु कारयेच्च यथाविधि}
{यदा तु द्वादशी स्वल्पा पारणेन भवेद्दिवज}% ॥८२॥

\twolineshloka
{तदा रात्रौ तु कर्तव्यं पारणं मुक्तिमिच्छता}
{तदा न रात्रिदोषः स्यान्निषिद्धं न भवेत्क्वचित्}% ॥८३॥

\twolineshloka
{यदुक्तं निशि न स्नायान्महानिशि न भोजयेत्}
{तत्पूर्वपरयामाभ्यां दिनवत्कर्म कारयेत्}% ॥८४॥

\twolineshloka
{यदा भवति स्वल्पा तु द्वादशी पारणे दिने}
{उषःकाले  द्वयं कुर्यात्प्रातर्मध्याह्निकं तथा}% ॥८५॥

\twolineshloka
{द्वादशी साधिता येन नरेण भुवि सर्वदा}
{तस्य पुण्यमहं वक्तुं न समर्थो विशेषतः}% ॥८६॥

\twolineshloka
{साधयित्वाखिलान्कामान्प्राप्नुयुश्च महाजनाः}
{अम्बरीषादयः सर्वे ये भक्ता भुवि विश्रुताः}% ॥८७॥

\twolineshloka
{द्वादशीं साधयित्वा तु ते गता विष्णुसद्मनि}
{सत्यं सत्यं पुनः सत्यं यदुक्तं तु मया तव}% ॥८८॥

\twolineshloka
{नास्ति विष्णुसमो देवो न तिथिर्द्वादशीसमा}
{अत्र दत्तं च भुक्तं च तथा पूजादिकं च यत्}% ॥८९॥

\twolineshloka
{तत्सर्वं पूर्णतां याति पूजिते माधवे सति}
{किं पुनर्बहुनोक्तेन भक्तानां वल्लभो हरिः}% ॥९०॥

\twolineshloka
{प्रददात्यखिलान्कामान्यावदाभूतसम्प्लवम्}
{द्वादश्यां चैव यद्दत्तं तत्सर्वं सफलं भवेत्}% ॥९१॥

\twolineshloka
{कुरुक्षेत्रेषु यद्दत्तं निष्फलं नैव जायते}
{तद्वच्च द्वादशीदत्तं भवेद्देवर्षिसत्तम}% ॥९२॥

॥इति श्रीपाद्मे महापुराणे पञ्चपञ्चाशत्साहस्र्यां संहितायामुत्तरखण्डे उमापतिनारदसंवादान्तर्गतकृष्णयुधिष्ठिरसंवादे द्वादशी-एकादशी-जागरणमहिमावर्णनं नाम नामैकोनचत्वारिंशोऽध्यायः॥३९॥


\hyperref[sec:ekadashi_mahatmyam_padma_puranam]{\closesub}
\clearpage

\sect{मार्गशीर्ष-कृष्ण-उत्पन्ना-एकादशी-कृत-मुरवधः}
\label{sec:padma-margashirsha-krishna-utpanna}

\uvacha{महादेव उवाच}

\twolineshloka
{एकस्मिन्समये पुत्र गतोऽहं विष्णुसन्निधौ}
{तत्र पृष्टं मया पूर्वं माहात्म्यं द्वादशीभवम्}% ॥१॥

\uvacha{नारद उवाच}

\twolineshloka
{कीदृशी स्यान्महादेव महाद्वादशिका परा}
{तस्या व्रते फलं कीदृग्वद सर्वेश्वर प्रभो}% ॥२॥

\uvacha{शिव उवाच}

\twolineshloka
{इयमेकादशी ब्रह्मन्महापुण्यफलप्रदा}
{ऋक्षयोगैश्च संयुक्ता कर्तव्या मुनिसत्तमैः}% ॥३॥

\twolineshloka
{जया च विजया चैव जयन्ती पापनाशिनी}
{सर्वपापहराश्चैताः कर्तव्याः फलकाङ्क्षिभिः}% ॥४॥

\twolineshloka
{एकादश्यां यदा ऋक्षं शुक्लपक्षे पुनर्वसुः}
{नाम्ना सा च जया ख्याता तिथीनामुत्तमा तिथिः}% ॥५॥

\twolineshloka
{तामुपोष्य नरः पापान्मुच्यते नात्र संशयः}
{यदा च शुक्लद्वादश्यां नक्षत्रं श्रवणं भवेत्}% ॥६॥

\twolineshloka
{विजया सा समाख्याता तिथीनामुत्तमा तिथिः}
{सहस्रगुणितं दानं यस्यां वै विप्र भोजनम्}% ॥७॥

\twolineshloka
{होमस्तथोपवासश्च सहस्राधिफलप्रदः}
{यदा च शुक्लद्वादश्यां प्राजापत्यं हि जायते}% ॥८॥

\twolineshloka
{जयन्ती नाम सा प्रोक्ता सर्वपापहरा तिथिः}
{सप्तजन्मकृतं पापं स्वल्पं वा यदि वा बहु}% ॥९॥

\twolineshloka
{प्रक्षालयति गोविन्दस्तस्यामभ्यर्चितो ध्रुवम्}
{यदा वै शुक्लद्वादश्यां पुष्यं भवति कर्हिचित्}% ॥१०॥

\twolineshloka
{तदा तु सा महापुण्या भविता पापनाशिनी}
{यो ददाति तिलप्रस्थं नित्यं संवत्सरं प्रति}% ॥११॥

\twolineshloka
{उपवासं च यस्तस्यां करोत्येतत्समं स्मृतम्}
{तस्यां जगत्पतिर्देवस्तुष्टः सर्वेश्वरो हरिः}% ॥१२॥

\twolineshloka
{प्रत्यक्षतां प्रयात्येव तत्रानन्तफलं स्मृतम्}
{सगरेण ककुत्स्थेन नहुषेण च साधितः}% ॥१३॥

\twolineshloka
{तस्यामाराधितः कृष्णो दत्तवानखिलं भुवि}
{वाचिकान्मानसाद्वाऽपि कायजाच्च विशेषतः}% ॥१४॥

\twolineshloka
{सप्तजन्मकृतात्पापान्मुच्यते नात्र संशयः}
{तामेकां समुपोष्याथ पुण्यनक्षत्रसंयुताम्}% ॥१५॥

\twolineshloka
{एकादशीसहस्रस्य फलं प्राप्नोति मानवः}
{स्नानं दानं जपो होमः स्वाध्यायो देवतार्चनम्}% ॥१६॥

\twolineshloka
{यत्तस्यां क्रियते किञ्चित्तदक्षयफलं स्मृतम्}
{तस्मादेषा प्रकर्तव्या यत्नेन फलकाङ्क्षिभिः}% ॥१७॥

\twolineshloka
{पञ्चमेनाश्वमेधेन यदि स्नातो युधिष्ठिरः}
{पर्यपृच्छत धर्मात्मा कृष्णं यदुकुलोद्वहम्}% ॥१८॥

\uvacha{युधिष्ठि रउवाच}

\twolineshloka
{उपवासस्य नक्तस्य त्वेकभुक्तस्य मे प्रभो}
{किं पुण्यं किं फलं तस्य ब्रूहि सर्वं जनार्दन}% ॥१९॥

\uvacha{श्रीभगवानुवाच}
\onelineshloka*
{हेमन्ते चैव सम्प्राप्ते मासे मार्गे च शोभने}

\twolineshloka
{कृष्णपक्षे च या पार्थ द्वादशी तामुपोषयेत्}
{दशम्यां चैकभक्तश्च शुद्धचित्तो दृढव्रतः}% ॥२०॥

\twolineshloka
{नक्तं चैव तथा ज्ञात्वा दशम्यां नियतः सदा}
{दिवसस्याष्टमे भागे मन्दीभूते दिवाकरे}% ॥२१॥

\twolineshloka
{नक्तं च तद्विजानीयान्न नक्तं निशिभोजनम्}
{नक्षत्रदर्शनान्नक्तं गृहस्थस्य विधीयते}% ॥२२॥

\twolineshloka
{यतेर्दिनाष्टमे भागे रात्रौ तस्य निषेधनम्}
{ततः प्रभातसमये कृत्वा च नियमं व्रती}% ॥२३॥

\twolineshloka
{मध्याह्ने च तथा पार्थ स्नानं शुचिः समाचरेत्}
{अधमं कूपके स्नानं वाप्यां स्नानं च मध्यमम्}% ॥२४॥

\twolineshloka
{तडागे चोत्तमं स्नानं नद्यां स्नानं ततः परम्}
{पीड्यन्ते जन्तवो यत्र जलमध्ये व्यवस्थिते}% ॥२५॥

\twolineshloka
{तत्र स्नाने कृते पार्थ पापं पुण्यं समं भवेत्}
{गृहे चैवोत्तमं स्नानं जलं चैव विशोधयेत्}% ॥२६॥

\twolineshloka
{तस्मात्तु पाण्डवश्रेष्ठ गृहे स्नानं समाचरेत्}
{अश्वक्रान्ते रथक्रान्ते विष्णुक्रान्ते वसुन्धरे}% ॥२७॥

\twolineshloka
{मृत्तिके हर मे पापं यन्मया पूर्वसञ्चितम्}
{क्रोधलोभौ परित्यज्य चैकचित्तो दृढव्रतः}% ॥२८॥

\twolineshloka
{नालापेतान्त्यजं चैव तथा पाखण्डिनो नरान्}
{मिथ्यावादरतांश्चैव तथा ब्राह्मणनिन्दकान्}% ॥२९॥

\twolineshloka
{अन्यांश्चैव दुराचारानगम्यागमनेरतान्}
{परद्रव्यापहारांश्च परदाराभिगामिनः}% ॥३०॥

\twolineshloka
{केशवं पूजयित्वा तु नैवेद्यं तत्र कारयेत्}
{दीपं दद्याद्गृहे तत्र भक्तियुक्तेन चेतसा}% ॥३१॥

\twolineshloka
{तद्दिने वर्जयेत्पार्थ निद्रां चैव तु मैथुनम्}
{धर्मशास्त्रविनोदेन दिनं सर्वं निवारयेत्}% ॥३२॥

\twolineshloka
{रात्रौ जागरणं कृत्वा भक्तियुक्तो नृपोत्तम}
{विप्रेभ्यो दक्षिणां दद्यात्प्रणिपत्य क्षमापयेत्}% ॥३३॥

\twolineshloka
{यथा कृष्णा तथा शुक्ला विधिनैवं प्रकारयेत्}
{एकादशीं द्विजः पार्थ विभेदं नैव कारयेत्}% ॥३४॥

\twolineshloka
{एवं हि कुरुते यस्तु शृणु तस्य हि यत्फलम्}
{शङ्खोद्धारे नरः स्नात्वा दृष्ट्वा देवं गदाधरम्}% ॥३५॥

\twolineshloka
{एकादश्युपवासस्य कलां नार्हति षोडशीम्}
{सङ्क्रान्तिषु चतुर्लक्षं यो ददाति नृपोत्तम}% ॥३६॥

\twolineshloka
{एकादश्युपवासस्य कलां नार्हति षोडशीम्}
{प्रभासक्षेत्रे यत्पुण्यं ग्रहणे चन्द्र सूर्ययोः}% ॥३७॥

\twolineshloka
{तत्फलं जायते नूनमेकादश्युपवासिनः}
{केदारे चोदकं पीत्वा पुनर्जन्म न विद्यते}% ॥३८॥

\twolineshloka
{तथा चैकादशी पार्थ गर्भवासक्षयङ्करी}
{अश्वमेधस्य यज्ञस्य पृथिव्यां यत्फलं लभेत्}% ॥३९॥

\twolineshloka
{तस्माच्छतगुणं पुण्यमेकादश्युपवासिनः}
{तपस्विनो गृहे यस्य भुञ्जते च द्विजोत्तमाः}% ॥४०॥

\twolineshloka
{तत्फलं जायते नूनमेकादश्युपवासिनः}
{गोसहस्रेण यत्पुण्यं दत्वा वेदान्तपारगे}% ॥४१॥

\twolineshloka
{तस्माच्छतगुणं पुण्यमेकादश्युपवासिनः}
{येषां देहे त्रयो देवा ब्रह्मविष्णुमहेश्वराः}% ॥४२॥

\twolineshloka
{वसन्ति तेषां ते तुल्या एकादश्युपवासिनः}
{ते नराः पुण्यकर्माणो ये भक्ता हरिपूजकाः}% ॥४३॥

\twolineshloka
{एकादशीव्रतस्यापि पुण्यसङ्ख्या न विद्यते}
{एतत्पुण्यं भवेत्तस्य यत्सुरैरपि दुर्ल्लभम्}% ॥४४॥

\twolineshloka
{एतस्मादर्धपुण्यं तु प्राप्यते नक्तभोजनात्}
{नक्तस्यार्धं भवेत्पुण्यमेकभक्तेन देहिनाम्}% ॥४५॥

\twolineshloka
{तावद्गर्जन्ति तीर्थानि दानानि नियमानि च}
{यावन्नोपोषयेज्जन्तुर्वासरं विष्णुवल्लभम्}% ॥४६॥

\twolineshloka
{तस्मात्त्वं पाण्डवश्रेष्ठ व्रतमेतत्समाचर}
{पुण्यसङ्ख्यां न जानामि यत्त्वं पृच्छसि पाण्डव}% ॥४७॥

\twolineshloka
{एतद्धि कथितं पार्थ यद्गोप्यं व्रतमुत्तमम्}
{एकादशीसमं नास्ति कृत्वा यज्ञसहस्रकम्}% ॥४८॥

\uvacha{युधिष्ठिर उवाच}

\twolineshloka
{उत्पन्ना सा कथं देव पुण्या चैकादशी तिथिः}
{कथं पवित्रा विश्वेऽस्मिन्कथं वै देवताप्रिया}% ॥४९॥

\uvacha{श्रीभगवानुवाच}

\twolineshloka
{पुरा कृतयुगे पार्थ मुरनामेति दानवः}
{अत्यद्भुतो महारौद्र सर्वदेवभयङ्करः}% ॥५०॥

\twolineshloka
{इन्द्रोऽपि निर्जितस्तेन सर्वदेवास्तथा नृप}
{महासुरेण तेनैव मृत्युना च दुरात्मना}% ॥५१॥

\twolineshloka
{स्वर्गान्निराकृतास्तेन विचरन्ति महीतले}
{सशङ्का भयभीताश्च सर्वे गत्वा महेश्वरम्}% ॥५२॥

\twolineshloka
{इन्द्रेण कथितं सर्वमीश्वरस्यापि चाग्रतः}
{स्वर्गलोकपरिभ्रष्टा विचरन्ति  महीतले}% ॥५३॥

\twolineshloka
{मर्त्येषु संस्थिता देवा न शोभन्ते महेश्वर}
{उपायं ब्रूहि मे देव ह्यमरा यान्ति कां गतिम्}% ॥५४॥

\uvacha{महादेव उवाच}

\twolineshloka
{देवराजसुरश्रेष्ठयत्रास्तेगरुडध्वजः}
{शरण्यश्च जगन्नाथः परित्राता परायणः}% ॥५५॥

\twolineshloka
{तत्र गच्छ सुरश्रेष्ठ स वो रक्षां विधास्यति}
{ईश्वरस्य वचः श्रुत्वा देवराजो महामतिः॥५६                          ॥}

\twolineshloka
{त्रिदशैः सहितो यत्र गतस्तत्र युधिष्ठिर}
{जलमध्ये प्रसुप्तं तं दृष्ट्वा देवं गदाधरम्}% ॥५७॥

\onelineshloka
{कृताञ्जलिपुटो भूत्वा इन्द्रः स्तोत्रमुदीरयत्}% ॥५८॥

\uvacha{इन्द्र उवाच}

\twolineshloka
{ॐनमो देवदेवेश देवदानववन्दित}
{दैत्यारे पुण्डरीकाक्ष त्राहि नो मधुसूदन}% ॥५९॥

\twolineshloka
{सुराः सर्वे समायाता भयभीताश्च दानवात्}
{शरणं त्वां जगन्नाथ त्राहि मां भक्तवत्सल}% ॥६०॥

\twolineshloka
{त्राहि नो देवदेवेश त्राहि त्राहि जनार्दन}
{त्राहि वै पुण्डरीकाक्ष दानवानां विनाशक}% ॥६१॥

\twolineshloka
{त्वत्समीपं गताः सर्वे त्वमेव शरणं प्रभो}
{चशरणागतानां देवानां सहायं कुरु वै प्रभो}% ॥६२॥

\twolineshloka
{त्वं पतिस्त्वं मतिर्देव त्वं कर्ता त्वं च कारणम्}
{त्वं माता सर्वलोकानां त्वमेव जगतः पिता}% ॥६३॥

\twolineshloka
{भगवन्देवदेवेश शरणागतवत्सल}
{शरणं तव चायाता भयभीताश्च देवताः}% ॥६४॥


\threelineshloka
{देवता निर्जिताः सर्वाः स्वर्गभ्रष्टाः कृताः प्रभो}
{अत्युग्रेण हि दैत्येन मुरनाम्ना महौजसा}
{इन्द्रस्य वचनं श्रुत्वा विष्णुर्वचनमब्रवीत्}% ॥६५॥

\uvacha{श्रीभगवानुवाच}

\onelineshloka*
{कीदृशो दानवः शक्र किं रूपं कीदृशं बलम्}

\twolineshloka
{क्व स्थानं तस्य दुष्टस्य किं वीर्यं कः पराक्रमः}
{किं वरं तस्य दुष्टस्य ममाख्याहि महामते}% ॥६६॥

\uvacha{इन्द्र उवाच}

\twolineshloka
{पूर्वं बभूव देवेश ब्रह्मवंशसमुद्भवः}
{तालजङ्घस्तु नाम्ना च अत्युग्रोऽपि महासुरः}% ॥६७॥

\twolineshloka
{तस्य पुत्रो हि विख्यातो मुरनामेति दानवः}
{अत्युत्कटो महावीर्यो देवतानां भयङ्करः}% ॥६८॥

\twolineshloka
{पुरी चन्द्रावती नाम्ना तत्र स्थाने वसत्यसौ}
{निर्जिता देवताः सर्वा स्वर्गात्तेन विवासिताः}% ॥६९॥

\twolineshloka
{इन्द्रोऽन्यो रोपितस्तेन वातश्चैव हुताशनः}
{चन्द्रसूर्यो कृतौ चान्यौ वायुर्वरुण एव च}% ॥७०॥

\twolineshloka
{सर्वमात्मकृतं तेन सत्यं सत्यं जनार्दन}
{देवलोकः कृतस्तेन सर्वस्थानविवर्जितः}% ॥७१॥

\twolineshloka
{तस्य तद्वचनं श्रुत्वा कोपमानो जनार्दनः}
{हनिष्ये दानवं दुष्टं देवतानां भयङ्करम्}% ॥७२॥

\twolineshloka
{त्रिदशैः सहितो देवो गतश्चन्द्रावतीं पुरीम्}
{दृष्टो देवैस्तु दैत्येन्द्रो गर्जमानः पुनः पुनः}% ॥७३॥

\twolineshloka
{तेन सर्वे जिता देव गताश्चैव दिशो दश}
{हरिं निरीक्ष्य प्रोवाच तिष्ठतिष्ठेति दानवः}% ॥७४॥

\onelineshloka*
{भगवानब्रवीत्तं च क्रोधसंरक्तलोचनः}

\uvacha{श्रीभगवानुवाच}

\onelineshloka
{रे दानव दुराचार मम बाहुं निरीक्षय}% ॥७५॥

\twolineshloka
{ततस्ते सम्मुखाः सर्वे विष्णुना दुष्टदानवाः}
{हता बाणैः पुनर्दिव्यैर्जाताश्च भयविह्वलाः}% ॥७६॥

\twolineshloka
{चक्रं मुक्तं च कृष्णेन दैत्यसैन्येषु पाण्डव}
{तेनच्छिन्नास्तु शतशो बहवो निधनं गताः}% ॥७७॥

\twolineshloka
{एकोपि दानवस्तत्र युध्यमानो मुहुर्मुहुः}
{नष्टाः सर्वे सुरास्तेन निर्जितो मधुसूदनः}% ॥७८॥

\twolineshloka
{निर्जितं तेन दैत्येन बाहुयुद्धमजायत}
{बाहुयुद्धं कृतं तेन दिव्यं वर्षसहस्रकम्}% ॥७९॥

\twolineshloka
{विष्णुश्चिन्तां प्रपन्नश्च नष्टाः सर्वाश्च देवताः}
{विष्णुश्च निर्जितस्तेन गतो बदरिकाश्रमम्}% ॥८०॥

\twolineshloka
{गुहा सिंहावती नाम तत्र सुप्तो जनार्दनः}
{योजनद्वादशवती एकद्वारा च पाण्डव}% ॥८१॥

\twolineshloka
{तस्यां विष्टः प्रसुप्तश्च दानवो हन्तुमुद्यतः}
{महायुद्धेन तेनैव श्रान्तोऽसौ योगमायया}% ॥८२॥

\twolineshloka
{दानवः पृष्ठतो लग्नो प्रविष्टः स तदा गुहाम्}
{प्रसुप्तं तत्र मां दृष्ट्वा दानवो हर्षमागतः}% ॥८३॥

\twolineshloka
{इत्थं मां निर्जितं मत्वा प्रविष्टं शङ्कया हरिम्}
{निःसन्देहं हनिष्यामि दानवानां भयङ्करम्}% ॥८४॥

\twolineshloka
{निर्गता कन्यका तत्र विष्णुदेहाद्युधिष्ठिर}
{रूपवती सुसौभाग्या दिव्यप्रहरणायुधा}% ॥८५॥

\twolineshloka
{तस्य तेजोंऽशसम्भूता महाबलपराक्रमा}
{दृष्टा सा दानवेन्द्रेण मुरनाम्ना धनञ्जय}% ॥८६॥

\twolineshloka
{युद्धं समाहितं तेन स्त्रिया चैव प्रयाचितम्}
{कन्यका युध्यते तत्र सर्वयुद्धविशारदा}% ॥८७॥

\twolineshloka
{हुङ्कारैर्भस्मसाज्जातो मुर नामा महासुरः}
{निहते दानवे तस्मिंस्तत्र देवस्त्वबुध्यत}% ॥८८॥

\twolineshloka
{पतितं दानवं दृष्ट्वा ततो विस्मयमागतः}
{केनायं च हतो रौद्रो ह्यत्युग्रो मम शात्रवः}% ॥८९॥

\onelineshloka*
{अत्युग्रं च कृतं कर्म मम कारुण्यतः कृतम्}

\uvacha{कन्योवाच}
\onelineshloka
{तेन देवाश्च गन्धर्वा सयक्षोरगराक्षसाः}% ॥९०॥

\twolineshloka
{इन्द्राद्याः सकला जित्वा स्वर्गाच्चैव निराकृताः}
{हरिः सुप्तो मया दृष्टो मुरः पृष्ठे समागतः}% ॥९१॥

\twolineshloka
{संहरिष्यति त्रैलोक्यं सुप्ते चैव जर्नादने}
{तस्यास्तद्वचनं श्रुत्वा विष्णुर्वचनमब्रवीत्}% ॥९२॥
\onelineshloka*
{अहं च निर्जितो येन कथं सोऽपि त्वया जितः}

\uvacha{एकादश्युवाच}
\onelineshloka
{त्वत्प्रसादाच्च भो स्वामिन्महादैत्यो मया हतः}% ॥९३॥

\uvacha{श्रीभगवानुवाच}

\threelineshloka
{आनन्दं त्रिषु लोकेषु मुनयो देवता गताः}
{ब्रूहि त्वं च मया भद्रे यत्ते मनसि रोचते}
{ददामि च न सन्देहो यत्सुरैरपि दुर्ल्लभम्}% ॥९४॥

\uvacha{एकादश्युवाच}

\twolineshloka
{यदि तुष्टोऽसि मे देव सत्यमुक्तं जनार्दन}
{वरमेकं तु वाञ्च्छामि हृदये च जगत्पते}% ॥९५॥

\twolineshloka
{प्रार्थयामि च देवेश ईप्सितं च मया प्रभो}
{यदि सत्यं जगन्नाथ तिस्रो वाचो ददासि मे}% ॥९६॥

\uvacha{श्रीभगवानुवाच}

\twolineshloka
{सत्यं सत्यं मया प्रोक्तं अवश्यं तव सुव्रते}
{तिस्रो वाचो मया दत्ता न चावाक्यं भवेदिह}% ॥९७॥

\uvacha{एकादश्युवाच}

\twolineshloka
{त्रिभुवनेषु च देवेश चतुर्युगेषु साम्प्रतम्}
{त्रिषु लोकेषु सर्वत्र तादृशं कुरु मे प्रभो}% ॥९८॥

\twolineshloka
{सर्वतीर्थप्रधानं हि सर्वविघ्नविनाशिनी}
{सर्वसिद्धिकरी देवी त्वत्प्रसादाद्भवाम्यहम्}% ॥९९॥

\twolineshloka
{मामुपोष्यन्ति ये भक्त्या तव भक्त्या जनार्दन}
{सर्वसिद्धिर्भवेत्तेषां यदि तुष्टोऽसि मे प्रभो}% ॥१००॥

\twolineshloka
{उपवासं च नक्तं च एकभक्तं करोति च}
{तस्य वित्तं च धर्मं च मोक्षं वै देहि माधव}% ॥१०१॥

\uvacha{विष्णुरुवाच}

\twolineshloka
{यत्त्वं वदसि कल्याणि तत्सर्वं च भविष्यति}
{सर्वान्मनोरथान्भद्रे दास्यसि त्वं च नान्यथा}% ॥१०२॥

\twolineshloka
{मम भक्ताश्च ये लोके ये च भक्तास्तु कार्त्तिके}
{चतुर्युगेषु विख्यातास्त्रिषु लोकेषु वै प्रभो}% ॥१०३॥

\twolineshloka
{त्वां च शक्तिमहं मन्ये एकादशीव्रतस्थिताः}
{मम पूजां करिष्यन्ति मोक्षगास्ते न संशयः}% ॥१०४॥

\threelineshloka
{तृतीया चाष्टमी चैव नवमी च चतुर्दशी}
{एकादशी विशेषेण तिथिरेषा हरिप्रिया}
{सर्वतीर्थाधिकं पुण्यं सत्यं सत्यं न संशयः}% ॥१०५॥

\twolineshloka
{इदं दत्त्वा वरं तस्यै तिस्रो वाचो न संशयः}
{हृष्टा पुष्टा च सञ्जाता एकादशी महाव्रता}% ॥१०६॥

\twolineshloka
{शत्रुं हंसि परां तस्य ददासि परमां गतिम्}
{त्वं हंसि सर्वविघ्नानि सर्वसिद्धिवरप्रदा}% ॥१०७॥

\twolineshloka
{उभयोः पक्षयोः पार्थ तुल्या एकादशी शुभा}
{न शुक्ला नैव कृष्णा च विभेदं नैव कारयेत्}% ॥१०८॥

\twolineshloka
{विभेदो नैव कर्तव्यः समस्तव्रतकारिभिः}
{दिवा वा यदि वा रात्रौ शृणोति भक्तितत्परः}% ॥१०९॥

\twolineshloka
{तिथिरेका भवेत्सर्वा पक्षयोरुभयोरपि}
{उभयैकादशी स्वल्पा ह्यन्ते चैव त्रयोदशी}% ॥११०॥

\twolineshloka
{मध्ये तु द्वादशी पूर्णा त्रिस्पृशा सा हरिप्रिया}
{एकामुपोषयेत्तां वै सहस्रैकादशीफलम्}% ॥१११॥

\twolineshloka
{सहस्रगुणितं ह्येवं द्वादश्यां पारणे कृते}
{अष्टम्येकादशी षष्ठी तृतीया च चतुर्दशी}% ॥११२॥

\twolineshloka
{पूर्वविद्धा न कर्तव्या परविद्धामुपोषयेत्}
{एकादशी ह्यहोरात्रं प्रभाते घटिका भवेत्}% ॥११३॥

\twolineshloka
{सा तिथिः परिहर्तव्या उपोष्या द्वादशीयुता}
{एवंविधा मया प्रोक्ता पक्षयोरुभयोरपि}% ॥११४॥

\twolineshloka
{एकादश्यां प्रकुर्वीत ह्युपवासं न संशयः}
{ते यान्ति वैष्णवं स्थानं यत्रास्ते गरुडध्वजः}% ॥११५॥

\twolineshloka
{धन्यास्ते मानवा लोके विष्णुभक्तिपरायणाः}
{एकादश्याश्च माहात्म्यं सर्वकालेषु यः पठेत्}% ॥११६॥

\twolineshloka
{गोसहस्रफलं सोऽपि पुण्यं प्राप्नोति मानवः}
{दिवा वा यदि वा रात्रौ ये वै शृण्वन्ति भक्तितः}% ॥११७॥

\threelineshloka
{ब्रह्महत्यादिपापेभ्यो मुच्यन्ते नात्र संशयः}
{विष्णुधर्मसमं नास्ति गीतार्थं च नृपोत्तम}
{एकादशीसमं नास्ति व्रतं पापप्रणाशनम्}% ॥११८॥

॥इति श्रीपाद्मे महापुराणे पञ्चपञ्चाशत्साहस्र्यां संहितायामुत्तरखण्डे उमापतिनारदसंवादान्तर्गतकृष्णयुधिष्ठिरसंवादे मार्गशीर्ष-कृष्ण-उत्पन्ना-एकादशी-कृत-मुरवधो नाम चत्वारिंशोऽध्यायः॥४०॥


\hyperref[sec:ekadashi_mahatmyam_padma_puranam]{\closesub}
\clearpage

\sect{मार्गशीर्ष-शुक्ल-मोक्षदा-एकादशी-माहात्म्यम्}
\label{sec:padma-margashirsha-shukla-mokshada}


\uvacha{युधिष्ठिर उवाच}

\twolineshloka
{वन्दे विष्णुं विभुं साक्षाल्लोकत्रयसुखावहम्}
{विश्वेशं विश्वकर्तारं पुराणं पुरुषोत्तमम्}% ॥१॥

\twolineshloka
{पृच्छामि देवदेवेश संशयोऽस्ति महान्मम}
{लोकानां च हितार्थाय पापानां क्षयहेतवे}% ॥२॥

\threelineshloka
{मार्गशीर्षेसिते पक्षे भवेदेकादशी तु या}
{किं नाम को विधिस्तस्याः को देवस्तत्र पूज्यते}
{एतदाचक्ष्व मे स्वामिन्विस्तरेण यथातथम्}% ॥३॥

\uvacha{श्रीकृष्ण उवाच}

\twolineshloka
{साधु पृष्टं त्वया राजन्साधु ते विमलं यशः}
{कथयिष्यामि राजेन्द्र हरिवासरमुत्तमम्}% ॥४॥

\twolineshloka
{उत्पन्ना चासिते पक्षे द्वादशी मम वल्लभा}
{मार्गशीर्षोत्पत्तिरिति मम देह समुद्भवा}% ॥५॥

\twolineshloka
{सुरासुरवधार्थाय ह्युत्पन्ना भरतर्षभ}
{कथिता च मया सा वै तवाग्रे राजसत्तम}% ॥६॥

\twolineshloka
{पूर्वा चैकादशी राजंस्त्रैलोक्ये सचराचरे}
{मार्गशीर्षे सिते पक्षे उत्पत्तिरिति नामतः}% ॥७॥

\twolineshloka
{अतः परं प्रवक्ष्यामि मार्गशीर्षे सिता तु या}
{यस्याः श्रवणमात्रेण वाजपेयफलं लभेत्}% ॥८॥

\twolineshloka
{मोक्षा नामेति सा प्रोक्ता सर्वपापहरा परा}
{देवं दामोदरं राजन्पूजयेच्च प्रयत्नतः}% ॥९॥

\twolineshloka
{तुलस्या मञ्जरीभिश्च धूपैर्दीपैः प्रयत्नतः}
{पूर्वेण विधिना चैव दशम्येकादशी तथा}% ॥१०॥

\twolineshloka
{मोक्षा चैकादशी नाम्ना महापातकनाशिनी}
{रात्रौ जागरणं कार्यं नृत्यगीतस्तवैर्मम}% ॥११॥

\twolineshloka
{शृणु राजन्प्रवक्ष्यामि दिव्यां पौराणिकीकथाम्}
{यस्याः श्रवणमात्रेण सर्वपापक्षयो भवेत्}% ॥१२॥

\twolineshloka
{अधोयोनिगतश्चैव पितरो यस्य पापतः}
{अस्याश्च पुण्यदानेन मोक्षं यान्ति न संशयः}% ॥१३॥

\twolineshloka
{चम्पके नगरे रम्ये वैष्णवैश्च विभूषिते}
{वैखानसो नाम नृपः पुत्रवत्पालयेत्प्रजाः}% ॥१४॥

\twolineshloka
{वसन्ति बहवो विप्रा वेदवेदाङ्गपारगाः}
{ऋद्धिमत्यः प्रजास्तस्य राज्ञो वैखानसस्य हि}% ॥१५॥

\twolineshloka
{एवं राज्यं प्रकुर्वाणो रात्रौ स्वप्नस्य मध्यतः}
{स्वकीयपितरो दृष्ट्वा अधोयोनिगता नृप}% ॥१६॥

\twolineshloka
{एवं दृष्ट्वा च तान्सर्वान्विस्मयाविष्टमानसः}
{कथयामास वृत्तान्तं स्वप्नजातं द्विजान्प्रति}% ॥१७॥

\uvacha{राजोवाच}

\twolineshloka
{मया स्वपितरो दृष्ट्वा नरकोपगता द्विजाः}
{तारयेति तनूजत्वमस्मान्निरयसागरात्}% ॥१८॥

\twolineshloka
{एवं ब्रुवाणास्ते नूनं रोदमाना मुहुर्मुहुः}
{मया दृष्टा द्विजश्रेष्ठा एतस्माच्च न मे सुखम्}% ॥१९॥

\twolineshloka
{एतद्राज्यं मम महत्सुखदायि न विद्यते}
{अश्वा गजास्तथा सर्वे रोचन्ते मे न भो द्विजाः}% ॥२०॥

\twolineshloka
{न दारा न सुता मह्यं रोचन्ते द्विजसत्तमाः}
{किं करोमि क्व गच्छामि हृदयं मेऽवरुध्यते}% ॥२१॥

\twolineshloka
{तद्व्रतं तं तपोयोगं येनैव मम पूर्वजाः}
{मोक्षं प्रयान्ति सद्यो वै कथ्यतां च द्विजोत्तमाः}% ॥२२॥

\twolineshloka
{पुत्रे तु जीवितप्राये बलीयसि महात्मनि}
{पितास्ति नरके घोरे तस्य पुत्रस्य किं फलम्}% ॥२३॥

\uvacha{ब्राह्मणा ऊचुः}

\twolineshloka
{पर्वतस्य मुने राजन्निकटे चाश्रमो महान्}
{गम्यतां राजशार्दूल भूतं भव्यं विजानतः}% ॥२४॥

\twolineshloka
{तेषां श्रुत्वा ततो वाक्यं राजा वैखानसो महान्}
{जगाम चाशु तत्रैव चाश्रमं पर्वतस्य च}% ॥२५॥

\twolineshloka
{ब्राह्मणैर्वेष्टितो राजा राजभिश्च समन्वितः}
{आश्रमं विपुलं तस्य सम्प्राप्तो राजसत्तमः}% ॥२६॥

\twolineshloka
{तत्रर्ग्वेद यजुर्वेद सामाध्ययनकोविदैः}
{वेष्टितं मुनिभिश्चैव द्वितीयं ब्रह्मणो यथा}% ॥२७॥

\twolineshloka
{दृष्ट्वा तं मुनिशार्दूलं राजा वैखानसस्तथा}
{दण्डवत्प्रणतिं कृत्वा पस्पर्श चरणौ मुनेः}% ॥२८॥

\twolineshloka
{पप्रच्छ कुशलं तस्य सप्तस्वङ्गेष्वसौ मुनिः}
{राज्ये निष्कण्टकत्वं च राज्ञः सौख्यसमन्वितम्}% ॥२९॥

\uvacha{राजोवाच}

\twolineshloka
{तव प्रसादाद्भो स्वामिन्कुशलं मेऽङ्गसप्तसु}
{भक्ता ये विष्णुविप्रेषु कथं तेषां च विघ्नता}% ॥३०॥

\twolineshloka
{मया स्वपितरो दृष्टाः स्वप्ने च नरके स्थिताः}
{कस्य पुण्यस्य सामर्थ्यान्मोक्षं यान्ति द्विजोत्तम}% ॥३१॥

\twolineshloka
{अयं मे संशयः स्वामिन्प्रष्टुं तं त्वामुपागतः}
{उपायः कश्चिदेवात्र कर्तव्यो मुनिसत्तम}% ॥३२॥

\twolineshloka
{एतद्वाक्यं ततः श्रुत्वा पर्वतो मुनिसत्तमः}
{ध्यानस्तिमितनेत्रोऽभूत्तपस्वी ब्रह्मसन्निभः}% ॥३३॥

\twolineshloka
{मुहूर्तमेकं ध्यानस्थो भूपतिं प्रत्युवाच ह}
{ज्ञातं हि तव राजेन्द्र पितॄणां पूर्वचेष्टितम्}% ॥३४॥

\twolineshloka
{पूर्वजन्मनि तातस्ते क्षत्रियो राज्यगर्वितः}
{सपत्न्या ऋतुकाले तु राजधर्मप्रवर्तितः}% ॥३५॥

\twolineshloka
{गतो ग्रामे तु तां त्यक्त्वा कार्यार्थी निजयोषितम्}
{तव पित्रा तु तस्याश्च न दत्तमृतुदानकम्}% ॥३६॥

\twolineshloka
{तेन पापप्रभावेन नरके पितृभिः सह}
{पतितो राजशार्दूल तव तातः सुदारुणे}% ॥३७॥

\twolineshloka
{ततः पुनरुवाचेदं राजा वैखानसो मुनिम्}
{केन व्रतप्रभावेन मोक्षस्तेषां भवेन्मुने}% ॥३८॥

\uvacha{मुनिरुवाच}

\twolineshloka
{मार्गशीर्षे सिते पक्षे मोक्षा नामेति नामतः}
{सर्वैश्चेतद्व्रतं कार्यं पित्रे पुण्यं प्रदीयताम्}% ॥३९॥

\twolineshloka
{तेन पुण्यप्रभावेन मोक्षस्तेषां भविष्यति}
{सत्यमेतन्महाभाग ब्रह्मणो वचनं यथा}% ॥४०॥

\twolineshloka
{मुनेर्वाक्यं ततः श्रुत्वा स्वगृहं पुनरागतः}
{मार्गशीर्षस्तथा मासः प्राप्तः कष्टेन तेन वै}% ॥४१॥

\twolineshloka
{मुनेर्वाक्येन तत्कृत्वा व्रतं वैखानसो नृपः}
{अददत्पुण्यमखिलैः सार्धं पित्रे स भूमिपः}% ॥४२॥

\twolineshloka
{दत्ते पुण्यक्षणेनैव पुष्पवृष्टिरभूद्दिवि}
{वैखानसस्य तातो वै पितृभिर्मोक्षमाविशत्}% ॥४३॥

\twolineshloka
{राजानं चान्तरिक्षे स गिरं पुण्यामुवाच ह}
{स्वस्तिस्वस्तीति ते पुत्र प्रोच्य चैवं दिवं गतः}% ॥४४॥

\twolineshloka
{एवं यः कुरुते राजन्मोक्षामेकादशीं शुभाम्}
{तस्य पापानि नश्यन्ति मृतो मोक्षमवाप्नुयात्}% ॥४५॥

\twolineshloka
{नातः परतरा काचित्मोक्षदैकादशी भवेत्}
{पुण्यसङ्ख्यां न जानामि राजन्मे प्रियकृद्व्रतम्}% ॥४६॥

\twolineshloka
{चिन्तामणिसमा ह्येषा नृणां मोक्षप्रदायिनी}
{पठनाच्छ्रवणादस्या वाजपेयफलं लभेत्}% ॥४७॥

॥इति श्रीपाद्मे महापुराणे पञ्चपञ्चाशत्साहस्र्यां संहितायामुत्तरखण्डे उमापतिनारदसंवादान्तर्गतकृष्णयुधिष्ठिरसंवादे मार्गशीर्ष-शुक्ल-मोक्षदा-एकादशी-माहात्म्यं नाम एकचत्वारिंशोऽध्यायः॥४१॥


\hyperref[sec:ekadashi_mahatmyam_padma_puranam]{\closesub}
\clearpage

\sect{पौष-कृष्ण-सफला-एकादशी-माहात्म्यम्}
\label{sec:padma-pausha-krishna-saphala}


\uvacha{युधिष्ठिर उवाच}

\threelineshloka
{पौषस्य  कृष्णपक्षे तु किं नामैकादशी भवेत्}
{किं नाम को विधिस्तस्या एतद्विस्तरतो वद}
{एतदाख्याहि भो स्वामिन्को देवस्तत्र पूज्यते}% ॥१॥

\uvacha{श्रीकृष्ण उवाच}

\twolineshloka
{कथयिष्यामि राजेन्द्र भवतः स्नेहबन्धनात्}
{तुष्टिर्मे न तथा राजन्यज्ञैर्बहुलदक्षिणैः}% ॥२॥

\twolineshloka
{यथा मे तुष्टिरायाति ह्येकादशीव्रतेन वै}
{तस्मात्सर्वप्रयत्नेन कर्तव्यो हरिवासरः}% ॥३॥

\twolineshloka
{सत्यमेतन्न वै मिथ्या धर्मिष्ठानां विशारद}
{पौषस्य कृष्णपक्षे या सफला नाम नामतः}% ॥४॥

\twolineshloka
{तस्यां नारायणं देवं पूजयेच्च यथाविधि}
{पूर्वेणैव विधानेन कर्तव्यैकादशी शुभा}% ॥५॥

\twolineshloka
{नागानां च यथा शेषो पक्षिणां पन्नगाशनः}
{देवानां च यथा विष्णुर्द्विपदानां यथा द्विजः}% ॥६॥

\twolineshloka
{व्रतानां च यथा राजन्श्रेष्ठा चैकादशी तिथिः}
{ते जनाश्चैव भो राजन्पूज्या वै सर्वदा मम}% ॥७॥

\twolineshloka
{हरिवासरसंलीनाः कुर्वन्त्येकादशीव्रतम्}
{इहैव धनसंयुक्ता मृता मोक्षं लभन्ति ते}% ॥८॥

\twolineshloka
{सफलायां फलै राजन्पूजयेन्नामतो हरिम्}
{नारिकेलफलैश्चैव क्रमुकैर्बीजपूरकैः}% ॥९॥

\twolineshloka
{जम्बीरैर्दाडिमैश्चैव तथा धात्रीफलैः शुभैः}
{लवङ्गैर्बदरीभिश्च तथाम्रैश्च विशेषतः}% ॥१०॥

\twolineshloka
{पूजयेद्देवदेवेशं धूपदीपैस्तथैव च}
{सफलायां विशेषेण दीपदानं तु कारयेत्}% ॥११॥

\twolineshloka
{रात्रौ जागरणं चैव कर्तव्यं सह वैष्णवैः}
{यावन्निमेषो नेत्रस्य तावज्जागर्ति यो निशि}% ॥१२॥

\twolineshloka
{एकाग्रमनसो राजन्तस्य पुण्यं शृणुष्व हि}
{तत्समो नास्ति यज्ञो वै तीर्थं वा तत्समं नहि}% ॥१३॥

\twolineshloka
{सर्वव्रतानि राजेन्द्र कलां नार्हन्ति षोडशीम्}
{एवं वर्षसहस्राणि तपसा नैव यत्फलम्}% ॥१४॥

\twolineshloka
{तत्फलं समवाप्नोति यः करोति हि जागरम्}
{श्रूयतां राजशार्दूल सफलायाः कथा शुभा}% ॥१५॥

\twolineshloka
{चम्पावतीति विख्याता पुरी माहिष्मतस्य च}
{बभूवुस्तस्य राजर्षेः पुत्राः पञ्च कुमारकाः}% ॥१६॥

\twolineshloka
{तेषां मध्ये तु ज्येष्ठो वै महापापरतः सदा}
{परदाराभिचारी च वेश्यासक्तश्च मद्यपः}% ॥१७॥

\twolineshloka
{पितुर्द्रव्यं तु तेनैव गमितं पापकर्मणा}
{असद्वृत्तिरतो नित्यं भूसुराणां तु निन्दकः}% ॥१८॥

\twolineshloka
{वैष्णवानां च देवानां नित्यं निन्दां करोति सः}
{ईदृशं तु ततो दृष्ट्वा पुत्रं माहिष्मतो नृपः}% ॥१९॥

\twolineshloka
{नाम्ना तु लुम्पक इति राजपुत्रेषु चापठत्}
{राज्यान्निष्कासितस्तेन पित्रा चैव तु बन्धुभिः}% ॥२०॥

\twolineshloka
{स चैवं परिवारैस्तु त्यक्तश्च परिपन्थिवत्}
{लुम्पकोऽपि तथा त्यक्तश्चिन्तयामास वै तदा}% ॥२१॥

\twolineshloka
{त्यक्तोऽहं बान्धवैः पित्रा राज्यान्निष्कासितः किल}
{इति सञ्चिन्त्यमानोऽसौ मतिं पापे तदाकरोत्}% ॥२२॥

\twolineshloka
{मया गन्तव्यमेवास्तु दारुणे गहने वने}
{तस्माच्चैव पुरं सर्वं लुम्पयिष्यामि वै पितुः}% ॥२३॥

\twolineshloka
{इत्येवं स मतिं कृत्वा लुम्पको दैवयोगतः}
{निर्जगाम पुरात्तस्माद्गतोऽसौ गहने वने}% ॥२४॥

\twolineshloka
{जीवघातरतो नित्यं स्तेयद्यूतकलानिधिः}
{सर्वं च नगरं तेन मुषितं पापकर्मणा}% ॥२५॥

\twolineshloka
{स्तेयाभिगामी नगरे गृहीतः स निशाचरैः}
{उवाच तान्सुतोऽहं वै राज्ञो माहिष्मतस्य च}% ॥२६॥

\twolineshloka
{स तैर्मुक्तः पापकर्मा चागतो विपिनं पुनः}
{आमिषाभिरतो नित्यं तरोर्वै फलभक्षणे}% ॥२७॥

\twolineshloka
{आश्रमस्तस्य दुष्टस्य वासुदेवस्य सन्निधौ}
{अश्वत्थो वर्तते तत्र जीर्णश्च बहुवार्षिकः}% ॥२८॥

\twolineshloka
{देवत्वं तस्य वृक्षस्य विपिने वर्तते महत्}
{तत्रैव निवसंश्चैव लुम्पकः पापबुद्धिमान्}% ॥२९॥

\twolineshloka
{गते बहुतिथे काले कस्यचित्पुण्यसञ्चयात्}
{पौषस्य कृष्णपक्षे तु दशम्यां दिवसे तथा}% ॥३०॥

\twolineshloka
{फलानि भुक्त्वा वृक्षाणां रात्रौ शीतेन पीडितः}
{लुम्पको नाम पापिष्ठो वस्त्रहीनो गतेक्षणः}% ॥३१॥

\twolineshloka
{पीड्यमानोऽतिशीतेन हरिवृक्षसमीपतः}
{न निद्रा न सुखं तस्य गतप्राण इवाभवत्}% ॥३२॥

\twolineshloka
{आछाद्य दशनैरास्यमेवं नीता निशाखिला}
{भानूदयेऽपि पापिष्ठो न लेभे चेतनां तदा}% ॥३३॥

\twolineshloka
{लुम्पको गतसंज्ञस्तु सफलाया दिने तथा}
{रवौ मध्यङ्गते चैव संज्ञां लेभे स लुम्पकः}% ॥३४॥

\twolineshloka
{इतस्ततो विलोक्याथ व्यथितश्च तदासनात्}
{स्खलत्पद्भ्यां प्रचलितः खञ्जन्निव मुहुर्मुहुः}% ॥३५॥

\twolineshloka
{वनमध्ये गतस्तत्र क्षुत्क्षामः पीडितोऽभवत्}
{न शक्तिर्जीवघाते तु लुम्पकस्य दुरात्मनः}% ॥३६॥

\twolineshloka
{फलानि च तदा राजन्नाजहार स लुम्पकः}
{यावत्समागतस्तत्र तावदस्तं गतो रविः}% ॥३७॥

\twolineshloka
{किं भविष्यति तातेति स विलापं चकार ह}
{फलानि तत्र भूरीणि वृक्षमूले न्यवेशयत्}% ॥३८॥

\twolineshloka
{इत्युवाच फलैरेभिः श्रीपतिस्तुष्यतां हरिः}
{इत्युक्त्वा लुम्पकश्चैव निद्रां लेभे न वै निशि}% ॥३९॥

\twolineshloka
{रात्रौ जागरणं मेने विष्णुस्तस्य दुरात्मनः}
{फलैस्तु पूजनं मेने सफलायास्तथानघ}% ॥४०॥

\twolineshloka
{अकस्माद्व्रतमेवैतत्कृतवान्वै  स लुम्पकः}
{तेन पुण्यप्रभावेन प्राप्तं राज्यमकण्टकम्}% ॥४१॥

\twolineshloka
{सूर्यस्योदयनं यावत्तावद्विष्णुर्जगाम ह}
{दिवि तत्कालमुत्पन्ना वागुवाचाशरीरिणी}% ॥४२॥

\twolineshloka
{राज्यं प्राप्स्यसि पुत्रत्वं सफलायाः प्रसादतः।}
{तथेत्युक्ते तु वचसि दिव्यरूपधरोऽभवत्}% ॥४३॥

\twolineshloka
{मतिरासीत्ततस्तस्य परमा वैष्णवी नृप}
{दिव्याभरणशोभाढ्यो लेभे राज्यमकण्टकम्}% ॥४४॥

\twolineshloka
{कृतं राज्यं तु तेनैवं वर्षाणि दशपञ्च च}
{मनोदास्तस्यपुत्रास्तु दाराः कृष्णप्रसादतः}% ॥४५॥

\twolineshloka
{आशु राज्यं परित्यज्य पुत्रे चैव समर्प्य च}
{गतः कृष्णस्य सान्निध्यं यत्र गत्वा न शोचति}% ॥४६॥

\twolineshloka
{एवं यः कुरुते राजन्सफलाव्रतमुत्तमम्}
{इहलोके सुखं प्राप्य मृतो मोक्षमवाप्नुयात्}% ॥४७॥

\twolineshloka
{धन्यास्ते मानवा लोके सफलायां च ये रताः}
{तेषां च सफलं जन्म नात्र कार्या विचारणा}% ॥४८॥

\twolineshloka
{पठनाच्छ्रवणाच्चैव करणाच्च विशाम्पते}
{राजसूयस्य यज्ञस्य फलमाप्नोति मानवः}% ॥४९॥

॥इति श्रीपाद्मे महापुराणे पञ्चपञ्चाशत्साहस्र्यां संहितायामुत्तरखण्डे उमापतिनारदसंवादान्तर्गतकृष्णयुधिष्ठिरसंवादे पौष-कृष्ण-सफला-एकादशी-माहात्म्यं नाम द्विचत्वारिंशोऽध्यायः॥४२॥

\hyperref[sec:ekadashi_mahatmyam_padma_puranam]{\closesub}
\clearpage

\sect{पौष-शुक्ल-पुत्रदा-एकादशी-माहात्म्यम्}
\label{sec:padma-pausha-shukla-putrada}


\uvacha{युधिष्ठिर उवाच}

\twolineshloka
{कथिता वै त्वया कृष्ण सफलैकादशी शुभा}
{कथयस्व प्रसादेन शुक्लपक्षस्य या भवेत्}% ॥१॥

\twolineshloka
{किन्नाम को विधिस्तस्याः को देवस्तत्र पूज्यते}
{कस्मै तुष्टो हृषीकेशस्त्वमेव पुरुषोत्तमः}% ॥२॥

\uvacha{श्रीकृष्ण उवाच}

\twolineshloka
{शृणु राजन्प्रवक्ष्यामि शुक्ला पौषस्य या भवेत्}
{कथयामि महाराज लोकानां हितकाम्यया}% ॥३॥

\twolineshloka
{पूर्वेण विधिना राजन्कर्तव्यैषा प्रयत्नतः}
{पुत्रदा नाम नाम्ना सा सर्वपापहरा परा}% ॥४॥

\twolineshloka
{नारायणोऽधिदेवोऽस्या कामदः सिद्धिदायकः}
{नातः परतरा काचित्त्रैलोक्ये सचराचरे}% ॥५॥

\twolineshloka
{विद्यावन्तं यशस्वन्तं करोति च नरं हरिः}
{शृणु राजन्प्रवक्ष्यामि कथां पापहरां पराम्}% ॥६॥

\twolineshloka
{भद्रावत्यां पुरा ह्यासीत्पुर्यां राजा सुकेतुमान्}
{तस्य राज्ञस्तथाराज्ञी चम्पका नाम वर्तते}% ॥७॥

\twolineshloka
{पुत्रहीनेन राज्ञा च काले नीतो मनोरथैः}
{नैवात्मजं नृपो लेभे वंशकर्तारमेव च}% ॥८॥

\twolineshloka
{तेनैव राज्ञा धर्मेण चिन्तितं बहुकालतः}
{किं करोमि क्व गच्छामि सुतप्राप्तिः कथं भवेत्}% ॥९॥

\twolineshloka
{न राष्ट्रे न पुरे सौख्यं लेभे राजा सुकेतुमान्}
{साध्व्य स्वकान्तया सार्धं प्रत्यहं दुःखितोऽभवत्}% ॥१०॥

\twolineshloka
{तावुभौ दम्पती नित्यं चिन्ताशोकपरायणौ}
{पितरोऽस्य जलं दत्तं कवोष्णमुपभुञ्जते}% ॥११॥

\twolineshloka
{राज्ञः पश्चान्न पश्यामो योऽस्मान्सन्तर्पयिष्यति}
{इत्येवं संस्मरन्तोऽस्य दुःखिताः पितरोऽभवन्}% ॥१२॥

\twolineshloka
{न बान्धवा न मित्राणि नामात्याः सुहृदस्तथा}
{रोचयन्त्यस्य भूपस्य न गजाश्वाः पदातयः}% ॥१३॥

\twolineshloka
{नैराश्यं भूपतेस्तस्य नित्यं मनसि वर्तते}
{नरस्य पुत्रहीनस्य नास्ति वै जन्मनः फलम्}% ॥१४॥

\twolineshloka
{अपुत्रस्य गृहं शून्यं हृदयं दुःखितं सदा}
{पितृदेवमनुष्याणां नानृणत्वं सुतं विना}% ॥१५॥

\twolineshloka
{तस्मात्सर्वप्रयत्नेन सुतमुत्पादयेन्नरः}
{इह लोके यशस्तेषां परलोके शुभा गतिः}% ॥१६॥

\twolineshloka
{येषां तु पुण्यकर्तॄणां पुत्रजन्मगृहे भवेत्}
{आयुरारोग्यसम्पत्तिस्तेषां गेहे प्रवर्तते}% ॥१७॥

\twolineshloka
{पुत्रजन्म गृहे येषां लोकानां पुण्यकारिणाम्}
{पुण्यं विना न च प्राप्तिर्विष्णुभक्तिं विना नृप}% ॥१८॥

\twolineshloka
{पुत्राश्च सम्पदो वाऽपि निश्चयादिति मे मतिः}
{एवं विचिन्त्यमानोऽसौ न शर्म लभते नृपः}% ॥१९॥

\twolineshloka
{प्रत्यूषे चिन्तयद्राजा निशीथे चिन्तयत्ततः}
{स्वयमात्मविनाशं च चिन्तयामास केतुमान्}% ॥२०॥

\twolineshloka
{आत्मघाते दुर्गतिं च चिन्तयित्वा तदा नृपः}
{दृष्ट्वात्मदेहं पतितमपुत्रत्वं तथैव च}% ॥२१॥

\twolineshloka
{पुनर्विचार्यात्मबुद्ध्या आत्मनो हितकारणम्}
{अश्वारूढस्ततो राजा जगाम गहनं वनम्}% ॥२२॥

\twolineshloka
{पुरोहितादयः सर्वे न जानन्ति गतं नृपम्}
{गम्भीरे विपिने राजा मृगपक्षिनिषेविते}% ॥२३॥

\twolineshloka
{विचचार तदा राजा वनवृक्षान्विलोकयन्}
{वटानश्वत्थबिल्वांश्च खर्जूरान्पनसांस्तथा}% ॥२४॥

\twolineshloka
{बकुलान्सप्तपर्णांश्च तिन्दुकांस्तिलकानपि}
{शालांस्तालांस्तमालांश्च ददर्श सरलान्नृपः}% ॥२५॥

\twolineshloka
{इङ्गुदी ककुभांश्चैव श्लेष्मातकनगांस्तथा}
{शल्लकान्करमर्दांश्च पाटलान्बदरानपि}% ॥२६॥

\twolineshloka
{अशोकांश्च पलाशांश्च शृगालाञ्शशकानपि}
{वनमार्जारमहिषान्शल्लकांश्चमरानपि}% ॥२७॥

\twolineshloka
{ददर्श भुजगान्राजा वल्मीकादर्धनिःसृतान्}
{तथा वनगजान् मत्तान् कलभैः सह सङ्गतान्}% ॥२८॥

\twolineshloka
{यूथपांश्च चतुर्दन्तान्करिणीयूथमध्यगान्}
{तान्दृष्ट्वा चिन्तयामास आत्मनः स गजान्नृपः}% ॥२९॥

\twolineshloka
{तेषां स विचरन्मध्ये राजा शोभामवाप ह}
{महदाश्चर्यसंयुक्तं ददृशे विपिनं नृपः}% ॥३०॥

\twolineshloka
{मार्गे शिवारुतान्शृण्वन्नुलूकविरुतं तथा}
{तांस्तानृक्षमृगान्पश्यन्बभ्राम वनमध्यतः}% ॥३१॥

\twolineshloka
{एवं ददर्श गहनं नृपो मध्यगते रवौ}
{क्षुत्तृड्भ्यां पीडितो राजा इतश्चेतश्च धावति}% ॥३२॥

\twolineshloka
{नृपतिश्चिन्तयामास संशुष्कगलकन्धरः}
{मया तु किं कृतं कर्म प्राप्तं दुःखं यदीदृशम्}% ॥३३॥

\twolineshloka
{मया वै तोषिता देवा यज्ञैः पूजाभिरेव च}
{तथैव ब्राह्मणा दानैस्तोषिता मिष्टभोजनैः}% ॥३४॥

\twolineshloka
{प्रजाश्चैव सदा कालं पुत्रवत्पालिता भृशम्}
{कस्माद्दुःखं मया प्राप्तमीदृशं दारुणं महत्}% ॥३५॥

\twolineshloka
{इति चिन्तापरो राजा जगामैवाग्रतो वनम्}
{सुकृतस्य प्रभावेन सरो दृष्टमनुत्तमम्}% ॥३६॥

\twolineshloka
{मीनसंस्पर्शमानं च पद्मैश्चापरशोभितम्}
{कारण्डैश्चक्रवाकैश्च राजहंसैश्च शोभितम्}% ॥३७॥

\twolineshloka
{मकरैर्बहुभिर्मत्स्यैरन्यैर्जलचरैर्युतम्}
{समीपे सरसस्तस्य मुनीनामाश्रमान्बहून्}% ॥३८॥

\twolineshloka
{ददर्श राजा लक्ष्मीवान्शकुनैः शुभशंसिभिः}
{दक्षिणं प्रास्फुरन्नेत्रमथ सव्येतरः करः}% ॥३९॥

\twolineshloka
{प्रास्फुरन्नृपतेस्तस्य कथयञ्शोभनं फलम्}
{तस्य तीरे मुनीन्दृष्ट्वा कुर्वाणन्नैगमं जपम्}% ॥४०॥

\twolineshloka
{हर्षेण महताविष्टो बभूव नृपनन्दनः}
{अवतीर्य हयात्तस्मान्मुनीनामग्रतः स्थितः}% ॥४१॥

\twolineshloka
{पृथक्पृथग्ववन्देऽसौ मुनींस्तान्शंसितव्रतान्}
{कृताञ्जलिपुटो भूत्वा दण्डवच्च पुनः पुनः}% ॥४२॥

\onelineshloka*
{प्रत्यूचुस्तेऽपि मुनयः प्रसन्ना नृपते वयम्}

\uvacha{राजोवाच}

\twolineshloka
{के भवन्तोऽत्र कथ्यन्तां का चाख्या भवतामपि}
{किमर्थं सङ्गता यूयं सत्यं वदत मेऽग्रतः}% ॥४३॥

\uvacha{मुनय ऊचुः}

\twolineshloka
{विश्वेदेवा वयं राजन्स्नानार्थमिह चागताः}
{माघो निकटमायात एतस्मात्पञ्चमेऽहनि}% ॥४४॥

\twolineshloka
{अद्य चैकादशी राजन्पुत्रदा नाम नामतः}
{पुत्रं ददात्यसौ विष्णुः पुत्रदा कारिणां नृणाम्}% ॥४५॥

\uvacha{राजोवाच}

\twolineshloka
{एष वै संशयो मह्यं पुत्रस्योत्पादने महान्}
{यदि तुष्टा भवन्तो वै पुत्रं मे दीयतां तदा}% ॥४६॥

\uvacha{मुनिरुवाच}

\twolineshloka
{अद्यैव दिवसे राजन्पुत्रदा नाम वर्तते}
{एकादशीति विख्यातं क्रियतां व्रतमुत्तमम्}% ॥४७॥

\twolineshloka
{अभिषेकात्ततोऽस्माकं केशवस्य प्रसादतः}
{अवश्यं तव राजेन्द्र पुत्रप्राप्तिर्भविष्यति}% ॥४८॥

\twolineshloka
{इत्येवं वचनात्तेषां कृतं राज्ञा व्रतोत्तमम्}
{मुनीनामुपदेशेन पुत्रदा या विधानतः}% ॥४९॥

\twolineshloka
{द्वादश्यां पारणं कृत्वा मुनीन्नत्वा पुनः पुनः}
{आजगाम गृहं राजा राज्ञी गर्भमथादधौ}% ॥५०॥

\twolineshloka
{पुत्रो जातः सूतिकाले तेजस्वी पुण्यकर्मणा}
{पितरं तोषयामास प्रजापालो बभूव सः}% ॥५१॥

\twolineshloka
{तस्माद्राजन्प्रकर्तव्यं पुत्रदा व्रतमुत्तमम्}
{लोकानां तु हितार्थाय तवाग्रे कथितं मया}% ॥५२॥

\threelineshloka
{एकचित्तास्तु ये मर्त्याः कुर्वन्ति पुत्रदा व्रतम्}
{पुत्रान्प्राप्येह लोके तु मृतास्ते स्वर्गगामिनः}
{पठनाच्छ्रवणाद्राजन्नग्निष्टोमफलं लभेत्}% ॥५३॥

॥इति श्रीपाद्मे महापुराणे पञ्चपञ्चाशत्साहस्र्यां संहितायामुत्तरखण्डे उमापतिनारदसंवादान्तर्गतकृष्णयुधिष्ठिरसंवादे पौष-शुक्ल-पुत्रदा-एकादशी-माहात्म्यं नाम त्रिचत्वारिंशोऽध्यायः॥४३॥


\hyperref[sec:ekadashi_mahatmyam_padma_puranam]{\closesub}
\clearpage

\sect{माघ-कृष्ण-षट्तिला-एकादशी-माहात्म्यम्}
\label{sec:padma-magha-krishna-shattila}


\uvacha{युधिष्ठिर उवाच}

\twolineshloka
{साधु कृष्ण जगन्नाथ आदिदेव जगत्पते}
{कथयस्व प्रसादेन कृपां कुरु ममोपरि}% ॥१॥

\twolineshloka
{माघस्य कृष्णपक्षे तु का वा चैकादशी भवेत्}
{किं नाम को विधिस्तस्या  एतद्विस्तरतो वद}% ॥२॥

\uvacha{श्रीभगवानुवाच}

\twolineshloka
{शृणु त्वं नृपशार्दूल कृष्णमाघस्य या भवेत्}
{षट्तिला नाम विख्याता सर्वपापप्रणाशिनी}% ॥१॥

\twolineshloka
{षट्तिलायाः शृणुष्व त्वं कथां पापहरां शुभाम्}
{यां पुलस्त्यो मुनिश्रेष्ठो दालभ्यं प्रति चोक्तवान्}% ॥२॥

\uvacha{दालभ्य उवाच}

\twolineshloka
{मर्त्यलोकमनुप्राप्ताः पापं कुर्वन्ति जन्तवः}
{ब्रह्महत्यादिपापैश्च युक्ता ये विविधादिभिः}% ॥३॥

\twolineshloka
{परद्रव्यापहाराश्च परव्यसनमोहिताः}
{कथं न यान्ति नरकं ब्रह्मंस्तद्ब्रूहि तत्त्वतः}% ॥४॥

\twolineshloka
{अनायासेन भगवन्दानेनाल्पेन केनचित्}
{पापं प्रशमनं याति एतन्मे वक्तुमर्हसि}% ॥५॥

\uvacha{पुलस्त्य उवाच}

\twolineshloka
{साधुसाधु महाभाग गुह्यमेतत्सुदुर्ल्लभम्}
{यन्न कस्यचिदाख्यातं विष्णुब्रह्मेन्द्रदैवतैः}% ॥६॥

\twolineshloka
{तदहं कथयिष्यामि त्वया पृष्टो द्विजोत्तम}
{माघमासे तु सम्प्राप्ते शुचिस्नातो जितेन्द्रियः}% ॥७॥

\twolineshloka
{कामक्रोधाभिमानेर्ष्यालोभ पैशुन्य वर्जितः}
{देवदेवं च संस्मृत्य पादौ प्रक्षाल्य वारिभिः}% ॥८॥

\twolineshloka
{भूमावपतितं गृह्य गोमयं तत्र मानवः}
{तिलान्प्रक्षिप्य कार्पांसं पिण्डिकाश्चैव कारयेत्}% ॥९॥

\twolineshloka
{अष्टोत्तरशतं चैव नात्र कार्या विचारणा}
{ततो माघे च सम्प्राप्ते ह्याषाढर्क्षं भवेद्यदि}% ॥१०॥

\twolineshloka
{मूलं वा कृष्णपक्षस्यैकादशी नियमांस्ततः}
{गृह्णीयात्पुण्यकाले च विधानं तत्र मे शृणु}% ॥११॥

\twolineshloka
{देवदेवं समभ्यर्च्य सुस्नातः प्रयतः शुचिः}
{कृष्णनामानि सङ्कीर्त्य पुनः प्रस्खलितादिषु}% ॥१२॥

\twolineshloka
{रात्रौ जागरणं कुर्यादादौ होमं च कारयेत्}
{अर्चयेद्देवदेवेशं द्वितीयेऽह्नि पुनर्हरिम्}% ॥१३॥

\twolineshloka
{चन्दनागुरुकर्पूरैर्नैवेद्यं कृसरं तथा}
{संस्मृत्य नाम्ना च ततः कृष्णाख्येन पुनः पुनः}% ॥१४॥

\threelineshloka
{कूष्माण्डैर्नारिकेरैश्च ह्यथवा बीजपूरकैः}
{सर्वाभावेऽपि विप्रेन्द्र शस्तपूगफलैर्वृतम्}
{अर्घं दद्याद्विधानेन पूजयित्वा जनार्दनम्}% ॥१५॥

\twolineshloka
{कृष्णकृष्ण कृपालुस्त्वमगतीनां गतिर्भव}
{संसारार्णव मग्नानां प्रसीद पुरुषोत्तम}% ॥१६॥

\threelineshloka
{नमस्ते पुण्डरीकाक्ष नमस्ते विश्वभावन}
{सुब्रह्मण्य नमस्तेऽस्तु महापुरुषपूर्वज}
{गृहाणार्घ्यं मया दत्तं लक्ष्म्या सह जगत्पते}% ॥१७॥
[इत्यर्घमन्त्रः]

\twolineshloka
{ततस्तु पूजयेद्विप्रमुदकुम्भं प्रदापयेत्}
{छत्रोपानहवस्त्रैश्च कृष्णो मे प्रीयतामिति}% ॥१८॥

\twolineshloka
{कृष्णा धेनुः प्रदातव्या यथाशक्ति द्विजोत्तमे}
{तिलपात्रं द्विजश्रेष्ठ दद्यात्पात्रविचक्षणः}% ॥१९॥

\threelineshloka
{स्नाने प्राशनके शस्तास्तथा कृष्णतिला मुने}
{तान्प्रदद्यात्प्रयत्नेन यथाशक्ति द्विजोत्तमे}
{तिलप्ररोहजाः क्षत्रे यावत्सङ्ख्यास्तिला द्विज}% ॥२०॥

\twolineshloka
{तावद्वर्षसहस्राणि स्वर्गलोके महीयते}
{तिलस्नायी तिलोद्वर्ती तिलहोमी तिलोदकी}% ॥२१॥

\onelineshloka*
{तिलदाता च भोक्ता च षट्तिलाः पापनाशनाः}

\uvacha{नारद उवाच}

\onelineshloka
{कृष्ण कृष्ण महाबाहो नमस्ते विश्वभावन}% ॥२२॥

\twolineshloka*
{षट्तिलैकादशीभूतं कीदृशं फलमस्ति वै}
{सोपाख्यानं मम ब्रूहि यदि तुष्टोऽसि यादव}

\uvacha{श्रीकृष्ण उवाच}

\onelineshloka*
{शृणु राजन्यथावृत्तं दृष्टं तत्कथयामि ते}

\twolineshloka
{मर्त्यलोके पुरा ह्यासीद्ब्राह्मण्येका च नारद}
{व्रतचर्यारता नित्यं देवपूजारता सदा}% ॥२३॥

\twolineshloka
{मासोपवासनिरता मम भक्ता च सर्वदा}
{कृष्णोपवाससंयुक्ता मम पूजापरायणा}% ॥२४॥

\twolineshloka
{शरीरं क्लेशितं चैव उपवासैर्द्विजोत्तम}
{देवानां ब्राह्मणानां च कुमारीणां च भक्तितः}% ॥२५॥

\twolineshloka
{गृहादिकं प्रयच्छन्ती सर्वकालं महासती}
{अतिकृच्छ्ररता सा तु सर्वकालं तु वै द्विज}% ॥२६॥

\onelineshloka
{न दत्ता भिक्षुके भिक्षा ब्राह्मणा न च तर्पिताः}% ॥२७॥

\twolineshloka
{ततः कालेन महता मया वै चिन्तितं द्विज}
{शुद्धमस्याः शरीरं हि व्रतैः कृच्छ्रैर्न संशयः}% ॥२८॥

\twolineshloka
{अर्चितो वैष्णवो लोकः कायक्लेशेन  वै तया}
{न दत्तमन्नदानं हि येन तृप्तिः परा भवेत्}% ॥२९॥

\twolineshloka
{एवं ज्ञात्वा अहं ब्रह्मन्मर्त्यलोकमुपागतः}
{कापालं रूपमास्थाय भिक्षापात्रे च याचिता}% ॥३०॥

\twolineshloka
{कस्मात्त्वमागतो ब्रह्मन्क्व यासि वद मेऽग्रतः}
{पुनरेव मया प्रोक्तं देहि भिक्षां च सुन्दरि}% ॥३१॥

\twolineshloka
{तया कोपेन महता मृत्पिण्डस्ताम्रभाजने}
{क्षिप्तो यावदहं ब्रह्मन्पुनः स्वर्गं गतो द्विज}% ॥३२॥

\twolineshloka
{ततः कालेन महता तापसी सुमहाव्रता}
{सदेहा स्वर्गमायाता व्रतचर्या प्रभावतः}% ॥३३॥

\twolineshloka
{मृत्पिण्डिकाप्रदानेन गृहं प्राप्तं मनोरमम्}
{सञ्जातं चैव विप्रर्षे धान्यराशि विवर्जितम्}% ॥३४॥

\twolineshloka
{गृहं यावन्निरीक्षेत न किञ्चित्तत्र पश्यति}
{तावद्गृहाद्विनिष्क्रान्ता ममान्ते चागता द्विज}% ॥३५॥

\twolineshloka
{क्रोधेन महताविष्टमिदं वचनमब्रवीत्}
{मया व्रतैश्च कृच्छ्रैश्च उपवासैरनेकशः}% ॥३६॥

\twolineshloka
{पूजयाराधितो देवः सर्वलोकस्य पालकः}
{न तत्र  दृश्यते किञ्चिद्गृहे मम जनार्दन}% ॥३७॥

\twolineshloka
{ततश्चोक्तं मया तस्यै गृहं गच्छ महाव्रते}
{आगमिष्यन्ति सुतरां कौतूहल समन्विताः}% ॥३८॥

\twolineshloka
{देवपत्न्यो हि द्रष्टुं त्वां विस्मयाभिसमन्विताः}
{द्वारं नोद्घाटय विना षट्तिलापुण्यवाचनात्}% ॥३९॥

\twolineshloka
{एवमुक्ता मया सा तु गता वै मानुषी तदा}
{अत्रान्तरे समायाता देवपत्न्यश्च वाडव}% ॥४०॥

\twolineshloka
{ताभिश्च कथितं तत्र त्वां द्रष्टुं हि समागताः}
{द्वारमुद्घाटयस्वाद्य त्वां प्रपश्याम शोभने}% ॥४१॥

\uvacha{मानुष्युवाच}

\twolineshloka
{यदि मद्दर्शनं कार्यं सत्यं वाच्यं विशेषतः}
{षट्तिलाया व्रतं पुण्यं द्वारोद्घाटनकारणात्}% ॥४२॥

\uvacha{श्रीकृष्ण उवाच}

\twolineshloka
{एकापिनावदत्तत्र षट्तिलैकादशीव्रतम्}
{अन्यया कथितं तत्र द्रष्टव्या मानुषी मया}% ॥४३॥

\twolineshloka
{ततो द्वारं समुद्घाट्य दृष्टा ताभिश्च मानुषी}
{न देवी न च गन्धर्वी नासुरी न च पन्नगी}% ॥४४॥

\twolineshloka
{दृष्टा पूर्वं तथा नारी यादृशीयं द्विजर्षभ}
{देवीनामुपदेशेन षट्तिलाया व्रतं कृतम्}% ॥४५॥

\twolineshloka
{मानुष्या सत्यव्रतया भुक्तिमुक्तिफलप्रदम्}
{रूपकान्तिसमायुक्ता क्षणेन समवाप सा}% ॥४६॥

\twolineshloka
{धनं धान्यं च वस्त्रादि सुवर्णं रौप्यमेव च}
{भवनं सर्वसम्पन्नं षट्तिलायाः प्रभावतः}% ॥४७॥

\onelineshloka
{रूपकान्तिसमायुक्ता क्षणेन समपद्यत}% ॥४८॥

\twolineshloka
{अतितृष्णा न कर्तव्या वित्तशाठ्यं विवर्जयेत्}
{आत्मवित्तानुसारेण तिलान्वस्त्राणि दापयेत्}% ॥४९॥

\twolineshloka
{लभते चैवमारोग्यं नरो जन्मनि जन्मनि}
{न दारिद्र्यं न कष्टत्वं न च दौर्भाग्यमेव च}% ॥५०॥

\twolineshloka
{सम्भवेद्वै द्विजश्रेष्ठ षट्तिलासमुपोषणात्}
{अनेन विधिना भूप तिलदाता न संशयः}% ॥५१॥

\threelineshloka
{मुच्यते पातकैः सर्वैरनायासेन मानवः}
{दानं च विधिवत्पात्रे सर्वपातकनाशनम्}
{नानर्थः कश्चिन्नायासः शरीरे नृपसत्तम}% ॥५२॥


॥इति श्रीपाद्मे महापुराणे पञ्चपञ्चाशत्साहस्र्यां संहितायामुत्तरखण्डे उमापतिनारदसंवादान्तर्गतकृष्णयुधिष्ठिरसंवादे माघ-कृष्ण-षट्तिला-एकादशी-माहात्म्यं नाम चतुश्चत्वारिंशोऽध्यायः॥४४॥
\hyperref[sec:ekadashi_mahatmyam_padma_puranam]{\closesub}
\clearpage

\sect{माघ-शुक्ल-जया-एकादशी-माहात्म्यम्}
\label{sec:padma-magha-shukla-jaya}


\uvacha{युधिष्ठिर उवाच}

\twolineshloka
{साधु कृष्ण त्वया प्रोक्तमादिदेवो भवान्प्रभो}
{स्वेदजा अण्डजाश्चैव उद्भिज्जाश्च जरायुजाः}% ॥१॥

\twolineshloka
{तेषां कर्ता विकर्ता त्वं पालकः क्षयकारकः}
{माघस्य कृष्णपक्षे तु षट्तिला कथिता त्वया}% ॥२॥

\twolineshloka
{शुक्ले च का भवेद्देव कथयस्व प्रसादतः}
{किं नाम को विधिस्तस्याः को देवस्तत्र पूज्यते}% ॥३॥

\uvacha{श्रीकृष्ण उवाच}

\twolineshloka
{कथयिष्यामि राजेन्द्र शुक्ले माघस्य या भवेत्}
{जया नामेति विख्याता सर्वपापहरा परा}% ॥४॥

\twolineshloka
{पवित्रा पापहन्त्री च कामदा मोक्षदा नृणाम्}
{ब्रह्महत्यापहन्त्री च पिशाचत्वविनाशिनी}% ॥५॥

\twolineshloka
{नैव तस्या व्रते चीर्णे प्रेतत्वं जायते नृणाम्}
{नातः परतरा काचित्पापघ्नी मोक्षदायिनी}% ॥६॥

\twolineshloka
{एतस्मात्कारणाद्राजन्कर्तव्या सा प्रयत्नतः}
{श्रूयतां राजशार्दूल कथा पौराणिकी शुभा}% ॥७॥

\twolineshloka
{पङ्कजे च पुराणेऽस्या महिमा कथितो मया}
{एकदा नाकलोके वै इन्द्रो राज्यं चकार ह}% ॥८॥

\twolineshloka
{देवास्तत्र सुखेनैव निवसन्ति मनोरमे}
{पीयूषपाननिरता अप्सरोगणसेविताः}% ॥९॥

\twolineshloka
{नन्दनं तु वनं तत्र पारिजातोपसेवितम्}
{रमयन्ति रमन्तेऽत्र अप्सरोभिर्दिवौकसः}% ॥१०॥

\twolineshloka
{एकदा रममाणोऽसौ देवेन्द्रः स्वेच्छया नृप}
{नर्तयामास वै हर्षात्पञ्चाशत्कोटिनायकः}% ॥११॥

\twolineshloka
{गन्धर्वास्तत्र गायन्ति गन्धर्वः पुष्पदन्तकः}
{चित्रसेनस्तु तत्रैव चित्रसेनसुता तथा}% ॥१२॥

\twolineshloka
{मालिनीति च नाम्ना तु चित्रसेनस्य योषिता}
{मालिन्यास्तु समुत्पन्ना पुष्पदन्ती च नामतः}% ॥१३॥

\twolineshloka
{पुष्पदन्तस्य पुत्रोऽसौ माल्यवान्नाम नामतः}
{पुष्पदन्त्याश्च रूपेण माल्यवानतिमोहितः}% ॥१४॥

\twolineshloka
{तया ह्येवं कटाक्षैश्च माल्यवांश्च वशीकृतः}
{लावण्यं रूपसम्पन्नं तस्या रूपं निशामय}% ॥१५॥

\twolineshloka
{बाहू तस्याश्च कामेन कण्ठपाशौ कृताविव}
{कर्णायते तु नयने रक्तान्ते घूर्णिते तथा}% ॥१६॥

\twolineshloka
{कर्णौ तु शोभनौ तस्याः कुण्डलाभ्यां नृपोत्तम}
{कम्बुग्रीवायुता सैव दिव्याभरणभूषिता}% ॥१७॥

\twolineshloka
{पीनोन्नतौ कुचौ तस्यास्तौ हेमकलशाविव}
{मध्यं क्षामं च चार्वङ्ग्या मुष्टिग्राह्यमनुत्तमम्}% ॥१८॥

\twolineshloka
{नितम्बौ विस्तृतौ चास्या विस्तीर्णा जघनस्थली}
{चरणौ शोभमानौ च रक्तोत्पलसमद्युती}% ॥१९॥

\twolineshloka
{ईदृश्या पुष्पवत्या स माल्यवानतिमोहितः}
{शक्रस्य परितोषाय नृत्यार्थं तौ समागतौ}% ॥२०॥

\twolineshloka
{गायमानौ तु तौ तत्र अप्सरोगणसेवितौ}
{मदनाभिपरीताङ्गौ पुष्पदन्ती च माल्यवान्}% ॥२१॥

\twolineshloka
{परस्परानुरागेण व्यामोहवशमागतौ}
{न शुद्धगानं गायेतां चित्तभ्रमसमन्वितौ}% ॥२२॥

\twolineshloka
{बद्धदृष्टी तथान्योन्यं कामबाणवशं गतौ}
{ज्ञात्वा लेखर्षभस्तत्र सङ्गतं मानसं तयोः}% ॥२३॥

\twolineshloka
{तालक्रियामानलोपात्तथा गीतविसर्जनात्}
{चिन्तयित्वा तु मघवा ह्यवमानं तथात्मनः}% ॥२४॥

\twolineshloka
{कुपितश्च तयोरर्थे शापं दास्यन्निदं जगौ}
{धिग्धिग्वां पतितौ मूढावाज्ञाभङ्ग कृतौ मम}% ॥२५॥

\twolineshloka
{युवां पिशाचौ भवतां दम्पतीभावधारिणौ}
{मर्त्यलोकमनुप्राप्तौ भुञ्जानौ कर्मणः फलम्}% ॥२६॥

\twolineshloka
{एवं मघवता शप्तावुभौ दुःखितमानसौ}
{हिमवन्तं गिरिं प्राप्ताविन्द्रशापाद्विमोहितौ}% ॥२७॥

\twolineshloka
{उभौ पिशाचतां प्राप्तौ दारुणं दुःखमेव च}
{सन्तप्तमानसौ तत्र हिमकृच्छ्रगतावुभौ}% ॥२८॥

\twolineshloka
{गन्धर्वत्वमप्सरस्त्वं न जानीतो विमौहितौ}
{पीड्यमानौ निदाघेन देहपातकजेन च}% ॥२९॥

\twolineshloka
{न निशायां सुखं शान्तिं लभेते कर्मपीडितौ}
{परस्परं वादमानौ चेरतुर्गिरिगह्वरे}% ॥३०॥

\twolineshloka
{पीड्यमानौ तु शीतेन तुषारप्रभवेन तौ}
{दन्तघर्षं प्रकुर्वाणौ रोमाञ्चितवपुर्धरौ}% ॥३१॥

\twolineshloka
{ऊचे पिशाचः स तदा तां पत्नीं स्वां पिशाचिकाम्}
{किमनल्पकृतं पापं दारुणं रोमहर्षणम्}% ॥३२॥

\twolineshloka
{येन प्राप्तं पिशाचत्वं स्वेन दुष्कृतकर्मणा}
{नरकं दारुणं मत्वा पिशाचत्वं च दुःखदम्}% ॥३३॥

\twolineshloka
{तस्मात्सर्वप्रयत्नेन पातकं न समाचरेत्}
{इति चिन्तापरौ तत्र तावास्तां दुःखकर्षितौ}% ॥३४॥

\twolineshloka
{दैवयोगात्तयोः प्राप्ता माघस्यैकादशी तिथिः}
{जयानामेति विख्याता तिथीनामुत्तमा तिथिः}% ॥३५॥

\twolineshloka
{तस्मिन्दिने तु सम्प्राप्ते तावाहारविवर्जितौ}
{आसाते तत्र नृपते जलपानविवर्जितौ}% ॥३६॥

\twolineshloka
{न कृतो जीवघातश्च न पत्रफलभक्षणम्}
{अश्वत्थस्य समीपे तौ सर्वदा दुःखसंयुतौ}% ॥३७॥

\twolineshloka
{रविरस्तङ्गतो राजंस्तथैव स्थितयोस्तयोः}
{प्राप्ता चैव निशा घोरा दारुणा प्राणहारिणी}% ॥३८॥

\twolineshloka
{वेपमानौ ततस्तौ तु ततः सुषुपतः क्षितौ}
{परस्परेण संलग्नौ गात्रयोर्भुजयोरपि}% ॥३९॥

\twolineshloka
{न निद्रा न रतं तत्र न तौ सौख्यमविन्दताम्}
{एवं तौ राजशार्दूल शापेनेन्द्रस्य पीडितौ}% ॥४०॥

\twolineshloka
{इत्थं तयोर्दुःखितयोर्निर्जगाम निशीथिनी}
{मार्तण्ड उदयं प्राप्तो द्वादशी दिवसागमे}% ॥४१॥

\twolineshloka
{मया तु राजशार्दूल तयोर्मुक्तिर्धृता हृदि}
{जयायाः सुव्रतं चीर्णं रात्रौ जागरणं कृतम्}% ॥४२॥

\twolineshloka
{तस्माद्व्रतप्रभावाच्च यथाजातं तथा शृणु}
{द्वादशीदिवसे प्राप्ते तथा चीर्णे जया व्रते}% ॥४३॥

\twolineshloka
{विष्णोः प्रभावान्नृपते पिशाचत्वं तयोर्गतम्}
{पुष्पदन्ती माल्यवतौ पूर्वरूपौ बभूवतुः}% ॥४४॥

\twolineshloka
{पुरातनस्नेहयुतौ  पूर्वालङ्कारधारिणौ}
{विमानमधिरूढौ तौ गतौ नाके मनोरमे}% ॥४५॥

\twolineshloka
{देवेन्द्रस्याग्रतो गत्वा प्रणामं चक्रतुर्मुदा}
{तथाविधौ तु तौ दृष्ट्वा मघवा विस्मितोऽब्रवीत्}% ॥४६॥

\uvacha{इन्द्र उवाच}

\twolineshloka
{वद तं केन पुण्येन पिशाचत्वं हि वां गतौ}
{मम शापं च सम्प्राप्तौ केन देवेन मोचितौ}% ॥४७॥

\uvacha{माल्यवानुवाच}

\twolineshloka
{वासुदेवप्रसादेन जयायास्तु व्रतेन च}
{पिशाचत्वं गतं स्वामिंस्तव भक्तिप्रभावतः}% ॥४८॥

\twolineshloka
{इति श्रुत्वा तु मघवा प्रत्युवाच पुनस्तथा}
{पवित्रौ पावनौ जातौ वन्दनीयौ ममापि च}% ॥४९॥

\twolineshloka
{हरिवासरकर्तारौ विष्णुभक्तिपरायणौ}
{हरिवासरसँल्लीना ये च कृष्णपरायणाः}% ॥५०॥

\twolineshloka
{अस्माकमपि मर्त्यास्ते पूज्याश्चैव न संशयः}
{विहरस्व यथासौख्यं पुष्पदन्त्या सुरालये}% ॥५१॥

\uvacha{कृष्ण उवाच}

\twolineshloka
{एतस्मात्कारणाद्राजन्कर्तव्यो हरिवासरः}
{जया तु राजशार्दूल ब्रह्महत्यापहारिणी}% ॥५२॥

\twolineshloka
{सर्वदानानि तेनैव सर्वयज्ञा अशेषतः}
{दत्तानि कारिताश्चैव जयायास्तु व्रतं कृतम्}% ॥५३॥

\twolineshloka
{कल्पकोटिर्भवेत्तावद्वैकुण्ठे मोदते ध्रुवम्}
{पठनाच्छ्रवणाद्राजन्नग्निष्टोमफलं लभेत्}% ॥५४॥

॥इति श्रीपाद्मे महापुराणे पञ्चपञ्चाशत्साहस्र्यां संहितायामुत्तरखण्डे उमापतिनारदसंवादान्तर्गतकृष्णयुधिष्ठिरसंवादे माघ-शुक्ल-जया-एकादशी-माहात्म्यं नाम पञ्चचत्वारिंशोऽध्यायः॥४५॥


\hyperref[sec:ekadashi_mahatmyam_padma_puranam]{\closesub}
\clearpage

\sect{फाल्गुन-कृष्ण-विजया-एकादशी-माहात्म्यम्}
\label{sec:padma-phalguna-krishna-vijaya}


\uvacha{युधिष्ठिर उवाच}

\twolineshloka
{फाल्गुनस्यासिते पक्षे किन्नामैकादशी भवेत्}
{कथयस्व प्रसादेन वासुदेव ममाग्रतः}% ॥१॥

\uvacha{श्रीकृष्ण उवाच}

\threelineshloka
{नारदः परिपप्रच्छ ब्रह्माणं कमलासनम्}
{फाल्गुनस्यासिते पक्षे विजयानाम नामतः}
{तस्याः पुण्यं द्विजश्रेष्ठ कथयस्व प्रसादतः}% ॥२॥

\uvacha{ब्रह्मोवाच}

\twolineshloka
{शृणु नारद वक्ष्यामि कथां पापहरां पराम्}
{यन्न कस्यचिदाख्यातं मयैतद्विजयाव्रतम्}% ॥३॥

\twolineshloka
{पुरातनं व्रतं ह्येतत्पवित्रं पापनाशनम्}
{जयं ददाति विजया नृपाणां वै न संशयः}% ॥४॥

\twolineshloka
{पुरा रामो वनं यातो वर्षाण्येव चतुर्दश}
{न्यवसत्पञ्चवट्यां तु  सहसीतः सलक्ष्मणः}% ॥५॥

\twolineshloka
{तत्रैव वसतस्तस्य रामस्य विजयात्मनः}
{रावणेन हृता लौल्याद्भार्या सीता यशस्विनी}% ॥६॥

\twolineshloka
{तेन दुःखेन रामोऽपि मोहमभ्यागतस्तदा}
{भ्रमन्जटायुषमथो ददर्श विगतायुषम्}% ॥७॥

\twolineshloka
{कबन्धो निहतः पश्चाद्भ्रमतारण्यमध्यतः}
{सुग्रीवेण समं तस्य सखित्वं समपद्यत}% ॥८॥

\twolineshloka
{वानराणामनीकानि रामार्थंसङ्गतानि च}
{ततो हनुमता दृष्टा लङ्कोद्याने तु जानकी}% ॥९॥

\twolineshloka
{रामसंज्ञापनं तस्यै दत्तं कर्म महत्कृतम्}
{पुनः समेत्य रामेण सर्वं तत्र निवेदितम्}% ॥१०॥

\twolineshloka
{अथ श्रुत्वा रामचन्द्रो वाक्यं चैव हनूमतः}
{सुग्रीवानुमतेनैव प्रस्थानं समरोचयत्}% ॥११॥

\threelineshloka
{सौमित्रेकेन पुण्येन तीर्यते वरुणालयः}
{अगाधो नितरामेष यादोभिश्च समाकुलः}
{उपायं नैव पश्यामि येनासौ सुतरो भवेत्}% ॥१२॥

\uvacha{लक्ष्मण उवाच}

\twolineshloka
{आदिदेवस्त्वमेवासि पुराणपुरुषोत्तमः}
{बकदाल्भ्यो मुनिश्चात्र वर्तते द्वीपमध्यतः}% ॥१३॥

\twolineshloka
{अस्मात्स्थानाद्योजनार्धमाश्रमस्तस्य राघव}
{अन्ये च ब्राह्मणास्तत्र बहवो रघुनन्दन}% ॥१४॥

\twolineshloka
{तं पृच्छ गत्वा राजेन्द्र पुराणमृषिपुङ्गवम्}
{इति वाक्यं ततः श्रुत्वा लक्ष्मणस्यातिशोभनम्}% ॥१५॥

\twolineshloka
{जगाम राघवो द्रष्टुं बकदाल्भ्यं महामुनिम्}
{प्रणनाम मुनिं मूर्ध्ना रामो विष्णुमिवामरः}% ॥१६॥

\twolineshloka
{ज्ञात्वा मुनिस्ततो रामं पुराणं पुरुषोत्तमम्}
{केनापि कारणेनैव प्रविष्टो मानुषीं तनुम्}% ॥१७॥

\onelineshloka
{उवाच स ऋषिस्तुष्टः कुतो राम तवागमः}% ॥१८॥

\uvacha{राम उवाच}

\twolineshloka
{त्वत्प्रसादादहं विप्र तीरं नदनदीपतेः}
{आगतोऽस्मि ससैन्योऽत्र लङ्कां जेतुं सराक्षसाम्}% ॥१९॥

\twolineshloka
{भवतश्चानुकूलत्वात्तीर्यतेऽब्धिर्यथा मया}
{तमुपायं वद मुने प्रसादं कुरु साम्प्रतम्}% ॥२०॥

\twolineshloka
{एतस्मात्कारणादेव द्रष्टुं त्वाहमिहागतः}
{रामस्य वचनं श्रुत्वा बकदाल्भ्यो महामुनिः}% ॥२१॥

\twolineshloka
{उवाच सुप्रसन्नात्मा रामं राजीवलोचनम्}
{कर्तव्यमद्य ते राम व्रतानां व्रतमुत्तमम्}% ॥२२॥

\twolineshloka
{कृतेन येन सहसा विजयस्ते भविष्यति}
{लङ्कां जित्वा राक्षसांश्च स्वच्छां कीर्तिमवाप्स्यसि}% ॥२३॥

\twolineshloka
{एकाग्रमानसो भूत्वा व्रतमेतत्समाचर}
{फाल्गुनस्यासिते पक्षे विजयैकादशी भवेत्}% ॥२४॥

\twolineshloka
{तस्या व्रतेन हे राम विजयस्ते भविष्यति}
{निःसंशयं समुद्रं त्वं तरिष्यसि सवानरः}% ॥२५॥

\twolineshloka
{विधिस्तु श्रूयतां राजन्व्रतस्यास्य फलप्रदः}
{दशम्यां दिवसे प्राप्ते कुम्भमेकं तु कारयेत्}% ॥२६॥

\twolineshloka
{हैमं वा राजतं वाऽपि ताम्रं वाप्यथ मृन्मयम्}
{स्थापयेच्छोभितं चैव जलपूर्णं सपल्लवम्}% ॥२७॥

\twolineshloka
{सप्तधान्यान्यधस्तस्य यवानुपरि विन्यसेत्}
{तस्योपरि न्यसेद्देवं हैमं नारायणं प्रभुम्}% ॥२८॥

\twolineshloka
{एकादशीदिने प्राप्ते प्रातः स्नानं समाचरेत्}
{निश्चलं स्थापयेत्कुम्भं कण्ठमाल्यानुलेपनैः}% ॥२९॥

\twolineshloka
{पूगीफलैर्नालिकेरैः पूजयेच्च विशेषतः}
{गन्धैर्धूपैश्चदीपैश्च नैवेद्यैर्विविधैरपि}% ॥३०॥

\twolineshloka
{कुम्भाग्रे तद्दिनं राम नीयते सत्कथादिभिः}
{रात्रौ जागरणं चैव तस्याग्रे कारयेद्बुधः}% ॥३१॥

\twolineshloka
{प्रकाशयेद्घृतदीपमखण्डव्रतहेतवे}
{द्वादशीदिवसे प्राप्ते मार्तण्डस्योदये सति}% ॥३२॥

\twolineshloka
{नीत्वा कुम्भं जलोद्देशे नद्याः प्रस्रवणे तथा}
{तडागे स्थापयित्वा तं पूजयित्वा यथाविधि}% ॥३३॥

\twolineshloka
{दद्यात्सदेवं तं कुम्भं ब्राह्मणे वेदपारगे}
{कुम्भेन सह राजेन्द्र महादानानि दापयेत्}% ॥३४॥

\twolineshloka
{अनेनविधिना राम यूथपैः सह सङ्गतः}
{कुरु व्रतं प्रयत्नेन विजयस्ते भविष्यति}% ॥३५॥

\twolineshloka
{इति श्रुत्वा ततो रामो यथोक्तमकरोत्तदा}
{कृते व्रते स विजयी बभूव रघुनन्दनः}% ॥३६॥

\twolineshloka
{प्राप्ता सीता जिता लङ्का पौलस्त्यो निहतो रणे}
{अनेनविधिना पुत्र ये कुर्वन्ति नरा व्रतम्}% ॥३७॥

\twolineshloka
{इहलोकजयप्राप्तिः परलोकस्तथाक्षयः}
{एतस्मात्कारणात्पुत्र कर्तव्यं विजयाव्रतम्}% ॥३८॥

\twolineshloka
{विजयायाश्च माहात्म्यं सर्वकिल्बिषनाशनम्}
{पठनाच्छ्रवणाच्चैव वाजपेयफलं लभेत्}% ॥३९॥

॥इति श्रीपाद्मे महापुराणे पञ्चपञ्चाशत्साहस्र्यां संहितायामुत्तरखण्डे उमापतिनारदसंवादान्तर्गतकृष्णयुधिष्ठिरसंवादे फाल्गुन-कृष्ण-विजया-एकादशी-माहात्म्यं नाम षट्चत्वारिंशोऽध्यायः॥४६॥


\hyperref[sec:ekadashi_mahatmyam_padma_puranam]{\closesub}
\clearpage

\sect{फाल्गुन-शुक्लामलकी-एकादशी-माहात्म्यम्}
\label{sec:padma-phalguna-shuklamalaki}


\uvacha{श्रीकृष्ण उवाच}

\twolineshloka
{माहात्म्यं विजयायाश्च श्रुतं कृष्ण महत्फलम्}
{फाल्गुनस्यार्जुने पक्षे यन्नाम्नी तां वदाधुना}% ॥१॥

\uvacha{श्रीकृष्ण उवाच}

\twolineshloka
{धर्मपुत्र महाभाग शृणु वक्ष्यामितेऽधुना}
{योक्ता पृष्टेन मान्धात्रा वसिष्ठेन महात्मना}% ॥२॥

\twolineshloka
{फाल्गुनस्य विशेषेण विशेषः कथितो नृप}
{आमलकीव्रतं पुण्यं विष्णुलोकफलप्रदम्}% ॥३॥

\twolineshloka
{आमलक्यास्तले गत्वा जागरं तत्र कारयेत्}
{कृत्वा जागरणं रात्रौ गोसहस्रफलं लभेत्}% ॥४॥

\uvacha{मान्धातोवाच}

\twolineshloka
{आमलकी कदा ह्येषा उत्पन्ना द्विजसत्तम}
{एतत्सर्वं ममाचक्ष्व परं कौतूहलं हि मे}% ॥५॥

\twolineshloka
{कस्मादियं पवित्रा च कस्मात्पापप्रणाशिनी}
{कस्माज्जागरणं कृत्वा गोसहस्रफलं लभेत्}% ॥६॥

\uvacha{वसिष्ठ उवाच}

\twolineshloka
{कथयामि महाभाग यथेयमभवत्क्षितौ}
{आमलकी महावृक्षः सर्वपापप्रणाशनः}% ॥७॥

\twolineshloka
{एकार्णवे पुरा जाते नष्टे स्थावरजङ्गमे}
{नष्टे देवासुरगणे प्रणष्टोरगराक्षसे}% ॥८॥

\twolineshloka
{तत्र देवादिदेवेशः परमात्मा सनातनः}
{जगाम ब्रह्मपरममात्मनः पदमव्ययम्}% ॥९॥

\twolineshloka
{ततोऽस्य जाग्रतो ब्रह्ममुखाच्छशिसमप्रभः}
{ष्ठीवनाद्बिन्दुरुत्पन्नः स भूमौ निपपात ह}% ॥१०॥

\twolineshloka
{तस्माद्बिन्दोः समुत्पन्नः स्वयं धात्री नगो महान्}
{शाखाप्रशाखाबहुलः फलभारेण नामितः}% ॥११॥

\twolineshloka
{सर्वेषां चैव वृक्षाणामादिरोहः प्रकीर्तितः}
{ब्रह्मणाथ ततः पश्चात्संसृष्टाश्च इमाः प्रजाः}% ॥१२॥

\twolineshloka
{देवदानवगन्धर्वयक्षराक्षसपन्नगान्}
{असृजद्भगवान्देवो महर्षींश्च तथामलान्}% ॥१३॥

\twolineshloka
{आजग्मुस्तत्र देवास्ते यत्र धात्री हरिप्रिया}
{तां दृष्ट्वा ते महाभाग परं विस्मयमागताः}% ॥१४॥

\twolineshloka
{न जानीम इमं वृक्षं चिन्तयन्तोऽभिसंस्थिताः}
{एवं चिन्तयतां तेषां वागुवाचाशरीरिणी}% ॥१५॥

\twolineshloka
{आमलकी नगो ह्येष प्रवरो वैष्णवो मतः}
{अस्य संस्मरणादेव लभेद्गोदानजं फलम्}% ॥१६॥

\twolineshloka
{स्पर्शनाद्द्विगुणं पुण्यं त्रिगुणं धारणात्तथा}
{तस्मात्सर्वप्रयत्नेन सेव्या आमलकी सदा}% ॥१७॥

\twolineshloka
{सर्वपापहरा प्रोक्ता वैष्णवी पापनाशिनी}
{तस्या मूले स्थितो विष्णुस्तदूर्ध्वे च पितामहः}% ॥१८॥

\twolineshloka
{स्कन्धे च भगवान्रुद्रः संस्थितः परमेश्वरः}
{शाखासु मुनयः सर्वे प्रशाखासु च देवताः}% ॥१९॥

\twolineshloka
{पर्णेषु चासते देवाः पुष्पेषु मरुतस्तथा}
{प्रजानां पतयः सर्वे फलेष्वेव व्यवस्थिताः}% ॥२०॥

\twolineshloka
{सर्वदेवमयी ह्येषा धात्री च कथिता मया}
{तस्मात्पूज्यतमा ह्येषा विष्णुभक्तिपरायणैः}% ॥२१॥

\uvacha{ऋषय ऊचुः}

\twolineshloka
{को भवान्न हि जानीमः कस्मात्कारणतां गतः}
{देवो वा यदि वा चान्यः कथयस्व यथातथम्}% ॥२२॥

\uvacha{वागुवाच}

\twolineshloka
{यः कर्ता सर्वभूतानां भुवनानां च सर्वशः}
{विस्मितान्विदुषः प्रेक्ष्य सोऽहं विष्णुः सनातनः}% ॥२३॥

\twolineshloka
{तच्छ्रुत्वा देवदेवस्य भाषितं ब्रह्मणः सुताः}
{अनादिनिधनं देवं स्तोतुं तत्र प्रचक्रमुः}% ॥२४॥

\twolineshloka
{नमो भूतात्मभूताय आत्मने परमात्मने}
{अच्युताय नमो नित्यमनन्ताय नमो नमः}% ॥२५॥

\twolineshloka
{दामोदराय कवये यज्ञेशाय नमो नमः}
{एवं स्तुतस्तु ऋषिभिस्तुतोष भगवान्हरिः}% ॥२६॥

\onelineshloka*
{प्रत्युवाच महर्षींस्तानभीष्टं किं ददामि वः}

{ऋषय ऊचुः}

\onelineshloka
{यदि तुष्टोऽसि भगवन्नस्माकं हितकाम्यया}% ॥२७॥

\twolineshloka
{व्रतं किञ्चित्समाख्याहि स्वर्गमोक्षफलप्रदम्}
{धनधान्यप्रदं पुण्यमात्मनस्तुष्टिकारकम्}% ॥२८॥

\twolineshloka
{अल्पायासं बहुफलं व्रतानामुत्तमं व्रतम्}
{कृतेन येन देवेश विष्णुलोके महीयते}% ॥२९॥

\uvacha{विष्णुरुवाच}

\twolineshloka
{फाल्गुने शुक्लपक्षे तु पुष्येण द्वादशी यदि}
{भवेत्सा च महापुण्या महापातकनाशिनी}% ॥३०॥

\twolineshloka
{विशेषस्तत्र कर्तव्यः शृणुध्वं द्विजसत्तमाः}
{आमलकीं च सम्प्राप्य जागरं तत्र कारयेत्}% ॥३१॥

\threelineshloka
{सर्वपापविनिर्मुक्तो गोसहस्रफलं लभेत्}
{एतद्वः कथितं विप्रा व्रतानां व्रतमुत्तमम्}
{अर्चयित्वाच्युतं तस्यां विष्णुलोकान्न मुच्यते}% ॥३२॥

\uvacha{ऋषय ऊचुः}

\twolineshloka
{व्रतस्यास्य विधिं ब्रूहि परिपूर्णं कथं भवेत्}
{के मन्त्राः के नमस्काराः देवता का प्रकीर्तिता}% ॥३३॥

\twolineshloka
{कथं दानं कथं स्नानं कश्च पूजाविधिः स्मृतः}
{अर्घार्चनस्य मन्त्रं तु कथयस्व यथातथम्}% ॥३४॥

\uvacha{विष्णुरुवाच}

\twolineshloka
{श्रूयतां यो विधिः सम्यग्व्रतस्यास्य द्विजर्षभाः}
{एकादश्यां निराहारः स्थित्वा चैव परेऽहनि}% ॥३५॥

\twolineshloka
{भोक्ष्येऽहं पुण्डरीकाक्ष शरणं मे भवाच्युत}
{इति कृत्वा तु नियमं दन्तधावनपूर्वकम्}% ॥३६॥

\twolineshloka
{नालपेत्पतितांश्चौरांस्तथा पाषण्डिनो नरान्}
{दुर्वृत्तान्भिन्नमर्यादान्गुरुदारप्रधर्षकान्}% ॥३७॥

\twolineshloka
{अपराह्णे ततः स्नानं विधिना कारयेद्बुधः}
{नद्यां तडागे कूपे वा गृहे वा नियतात्मवान्}% ॥३८॥

\twolineshloka
{मृत्तिकालम्भनं पूर्वं ततः स्नानं च कारयेत्}
{अश्वक्रान्ते रथक्रान्ते विष्णुक्रान्ते वसुन्धरे}% ॥३९॥

\onelineshloka
{मृत्तिके हर मे पापं यन्मया दुष्कृतं कृतम्}% ॥४०॥
[इति मृत्तिकामन्त्रः]

\twolineshloka
{त्वमम्बु सर्वभूतानां जीवनं तनुरक्षकम्}
{स्वेदजोद्भिज्जजातीनां रसानां पतये नमः}% ॥४१॥

\twolineshloka
{स्नातोऽहं सर्वतीर्थेषु ह्रदप्रस्रवणेषु च}
{नदीषु देवखातेषु इदं स्नानं तु मे भवेत्}% ॥४२॥
[इति स्नानमन्त्रः]

\twolineshloka
{जामदग्न्यं मुनिं चैव कारयित्वा हिरण्मयम्}
{माषकस्य सुवर्णस्य तदर्धार्धेन वा पुनः}% ॥४३॥

\twolineshloka
{गृहमागत्य पूजायाः पूजाहोमं तु कारयेत्}
{ततश्चामलकीं गच्छेत्सर्वोपस्करसंयुतः}% ॥४४॥

\twolineshloka
{आमलकीं ततो गत्वा परिशोध्य समन्ततः}
{स्थापयेत्सततं कुम्भमव्रणं मन्त्रपूर्वकम्}% ॥४५॥

\twolineshloka
{पञ्चरत्नसमोपेतं दिव्यगन्धाधिवासितम्}
{छत्रोपानद्युगोपेतं सितचन्दनचर्चितम्}% ॥४६॥

\twolineshloka
{स्रग्दामलम्बितग्रीवं सर्वधूपविधूपितम्}
{दीपमालाकुलं कुर्यात्सर्वतः सुमनोहरम्}% ॥४७॥

\twolineshloka
{तस्योपरि न्यसेत्पात्रं दिव्यलाजैः प्रपूरितम्}
{पात्रोपरि न्यसेद्देवं जामदग्न्यं महाप्रभम्}% ॥४८॥

\twolineshloka
{विशोकाय नमः पादौ जानुनी विश्वरूपिणे}
{उग्राय च ततोऽप्यूरू कटी दामोदराय च}% ॥४९॥

\twolineshloka
{उदरं पद्मनाभाय उरः श्रीवत्सधारिणे}
{चक्रिणे वामबाहुं च दक्षिणं गदिने नमः}% ॥५०॥

\twolineshloka
{वैकुण्ठाय नमः कण्ठमास्यं यज्ञमुखाय वै}
{नासां विशोकनिधये वासुदेवाय चाक्षिणी}% ॥५१॥

\twolineshloka
{ललाटं वामनायेति रामायेति भ्रुवौ नमः}
{सर्वात्मने तु तच्छीर्षं नम इत्यभिपूजयेत्}% ॥५२॥
[इति पूजामन्त्रः]

\twolineshloka
{ततो देवाधिदेवाय अर्घं चैव प्रदापयेत्}
{फलेन चैव शुभ्रेण भक्तियुक्तेन चेतसा}% ॥५३॥

\twolineshloka
{ततो जागरणं कुर्याद्भक्तियुक्तेन चेतसा}
{नृत्यैर्गीतैश्च वादित्रैर्धर्माख्यानैः स्तवैरपि}% ॥५४॥

\twolineshloka
{वैष्णवैश्च तथाख्यानैः क्षपयेत्सर्वशर्वरीम्}
{प्रदक्षिणां ततः कुर्याद्धात्र्या वै विष्णुनामभिः}% ॥५५॥

\twolineshloka
{अष्टाधिकं शतं चैव अष्टाविंशतिरेव वा}
{ततः प्रभाते समये कृत्वा नीराजनं हरेः}% ॥५६॥

\twolineshloka
{ब्राह्मणं पूजयित्वा तु सर्वं तस्मै निवेदयेत्}
{जामदग्न्य घटे तत्र वस्त्रयुग्ममुपानहौ}% ॥५७॥

\twolineshloka
{जामदग्न्यस्वरूपेण प्रीयतां मम केशवः}
{ततश्चामलकीं स्पृष्ट्वा कृत्वा चैव प्रदक्षिणाम्}% ॥५८॥

\twolineshloka
{स्नानं कृत्वा विधानेन ब्राह्मणान्भोजयेत्ततः}
{ततश्च स्वयमश्नीयात्कुटुम्बेन समावृतः}% ॥५९॥

\twolineshloka
{एवं कृतेन यत्पुण्यं तत्सर्वं कथयामि ते}
{सर्वतीर्थेषु यत्पुण्यं सर्वदानेषु यत्फलम्}% ॥६०॥

\twolineshloka
{सर्वयज्ञाधिकं चैव लभते नात्र संशयः}
{एतद्वः सर्वमाख्यातं व्रतानामुत्तमं व्रतम्}% ॥६१॥

\twolineshloka
{एतावदुक्त्वा देवेशस्तत्रैवान्तरधीयत}
{ते चापि ऋषयः सर्वे चक्रुः सर्वमशेषतः}% ॥६२॥

\twolineshloka
{तथा त्वमपि राजेन्द्र कर्तुमर्हसि सत्तम}
{व्रतमेतद्दुराधर्षं सर्वपापप्रमोचनम्}% ॥६३॥

॥इति श्रीपाद्मे महापुराणे पञ्चपञ्चाशत्साहस्र्यां संहितायामुत्तरखण्डे उमापतिनारदसंवादान्तर्गतकृष्णयुधिष्ठिरसंवादे फाल्गुन-शुक्लामलकी-एकादशी-माहात्म्यं नाम सप्तचत्वारिंशोऽध्यायः॥४७॥


\hyperref[sec:ekadashi_mahatmyam_padma_puranam]{\closesub}
\clearpage

\sect{चैत्र-कृष्ण-पापमोचनी-एकादशी-माहात्म्यम्}
\label{sec:padma-chaitra-krishna-papamochani}


\uvacha{युधिष्ठिर उवाच}

\twolineshloka
{फाल्गुनस्य सिते पक्षे श्रुता चामलकी तथा}
{चैत्रस्य कृष्णपक्षे तु किं नामैकादशी भवेत्}% ॥१॥

\uvacha{श्रीकृष्ण उवाच}

\twolineshloka
{शृणु राजेन्द्र वक्ष्यामि आख्यानं पापनाशनम्}
{यल्लोमशोऽब्रवीत्पृष्टो मान्धात्रा चक्रवर्तिना}% ॥२॥

\uvacha{मान्धातोवाच}

\threelineshloka
{भगवन्श्रोतुमिच्छामि लोकानां हितकाम्यया}
{चैत्रस्य प्रथमे पक्षे का नामैकादशी भवेत्}
{को विधिः किं फलं तस्याः कथयस्व प्रसादतः}% ॥३॥

\uvacha{लोमश उवाच}

\twolineshloka
{चैत्रमास्यसिते पक्षे नाम्ना वै पापमोचनी}
{एकादशी समाख्याता पिशाचत्वविनाशिनी}% ॥४॥

\twolineshloka
{शृणु तस्याः प्रवक्ष्यामि कामदां सिद्धिदां नृप}
{कथां विचित्रां शुभदां पापघ्नीं धर्मदायिनीम्}% ॥५॥

\twolineshloka
{पुरा चैत्ररथोद्देशे अप्सरोगणसेविते}
{वसन्तसमये प्राप्ते षट्पदाकुलिते वने}% ॥६॥

\twolineshloka
{गन्धर्वकन्यावादित्रै रमन्ति सह किन्नरैः}
{पाकशासनमुख्याश्च क्रीडान्ते त्रिदिवौकसः}% ॥७॥

\twolineshloka
{नापरं सुखदं किञ्चिद्विना चैत्ररथाद्वनम्}
{तस्मिन्वने तु मुनयस्तपन्ति बहुलं तपः}% ॥८॥

\twolineshloka
{मेधावि नामानमृषिं तत्रस्थं ब्रह्मचारिणम्}
{अप्सरास्तं मुनिवरं मोहनायोपचक्रमे}% ॥९॥

\twolineshloka
{मञ्जुघोषेति विख्याता भावं तस्य वितन्वती}
{क्रोशमात्रं स्थिता तस्य भयादाश्रमसन्निधौ}% ॥१०॥

\twolineshloka
{गायन्ती मधुरं साधु पीडयन्ती विपञ्चिकाम्}
{गायन्तीं तामथालोक्य पुष्पचन्दनसेविताम्}% ॥११॥

\twolineshloka
{कामोऽपि विजयाकाङ्क्षी शिवभक्तान्मुनीश्वरान्}
{तस्याः शरीरे संवासमकरोन्मनसः सुतः}% ॥१२॥

\twolineshloka
{कृत्वा भ्रुवौ धनुष्कोटिं गुणं कृत्वा कटाक्षकम्}
{मार्गणौ नयने कृत्वा पक्ष्मयुक्ते यथाक्रमम्}% ॥१३॥

\twolineshloka
{कुचौ कृत्वा पटकुटीं विजयायोपचक्रमे}
{मञ्जुघोषाभवत्तस्य कामस्यैव वरूथिनी}% ॥१४॥

\twolineshloka
{मेधाविनं मुनिं दृष्ट्वा सापि कामेन पीडिता}
{यौवनोद्भिन्नदेहोऽसौ मेधाव्यपि विराजते}% ॥१५॥

\twolineshloka
{सितोपवीतसहितो दृष्टः स्मर इवापरः}
{मेधावी वसते चासौ च्यवनस्याश्रमे शुभे}% ॥१६॥

\twolineshloka
{मञ्जुघोषा स्थितं तत्र दृष्ट्वा सा मुनिपुङ्गवम्}
{मदनस्य वशं प्राप्ता मन्दं मन्दमगायत}% ॥१७॥

\twolineshloka
{रणद्वलयसंयुक्तां शिञ्जन्नूपुरमेखलाम्}
{गायन्तीं तां तथाभूतां विलोक्य मुनिपुङ्गवः}% ॥१८॥

\twolineshloka
{मदनेन ससैन्येन नीतो मोहवशं बलात्}
{मञ्जुघोषा समागम्य मुनिं दृष्ट्वा तथाविधम्}% ॥१९॥

\twolineshloka
{हावभावकटाक्षैस्तं मोहयामास चाङ्गना}
{अधः संस्थाप्य वीणां सा सस्वजे तं मुनीश्वरम्}% ॥२०॥

\twolineshloka
{वलितेव लता वृक्षं वातवेगेन कम्पितम्}
{सोऽपि रेमे तया सार्धं मेधावी मुनिपुङ्गवः}% ॥२१॥

\twolineshloka
{तस्मिन्नेव ततो दृष्ट्वा तस्यास्तं देहमुत्तमम्}
{शिवतत्वं गतं तस्य कामतत्त्ववशं गतः}% ॥२२॥

\twolineshloka
{न निशां न दिनं सोऽपि रमन्जानाति कामुकः}
{बहुवर्षं गतः कालो मुनेराचारलोपतः}% ॥२३॥

\threelineshloka
{मञ्जुघोषा देवलोकगमनायोपचक्रमे}
{गच्छन्ती तं प्रत्युवाच रमन्तं मुनिसत्तमम्}
{आदेशो दीयतां ब्रह्मन्स्वदेशगमनाय मे}% ॥२४॥

\uvacha{मेधाव्युवाच}

\threelineshloka
{अद्यैव त्वं समायाता प्रदोषादौ वरानने}
{यावत्प्रभातसन्ध्या स्यात्तावत्तिष्ठ ममान्तिके}
{इति श्रुत्वा मुनेर्वाक्यं भयभीता बभूव सा}% ॥२५॥

\twolineshloka
{पुनर्वै रमयामास तमृषिं नृपसत्तम}
{मुनेः शापभयाद्भीता बहुलान्परिवत्सरान्}% ॥२६॥

\twolineshloka
{वर्षाणां पञ्चपञ्चाशन्नवमासदिनत्रयम्}
{सा रेमे मुनिना तस्य निशार्धमिव चाभवत्}% ॥२७॥

\twolineshloka
{सा तं पुनरुवाचाथ तस्मिन्काले गते मुनिम्}
{आदेशो दीयतां ब्रह्मन्गन्तव्यं स्वगृहे मया}% ॥२८॥

\uvacha{मेधाव्युवाच}

\twolineshloka
{प्रभातमधुना चास्ते श्रूयतां वचनं मम}
{सन्ध्या यावच्च कुर्वेऽहं तावत्त्वं वै स्थिरा भव}% ॥२९॥

\twolineshloka
{इति वाक्यं मुनेः श्रुत्वा जातानन्दसमाकुला}
{स्मितं कृत्वा तु सा किञ्चित्प्रत्युवाच शुचिस्मिता}% ॥३०॥

\uvacha{अप्सरा उवाच}

\twolineshloka
{कियत्प्रमाणा विप्रेन्द्र तव सन्ध्या गतानघ}
{मयि प्रसादं कृत्वा तु गतकालो विचार्यताम्}% ॥३१॥

\twolineshloka
{इति तस्या वचः श्रुत्वा विस्मयोत्फुल्ललोचनः}
{गतकालस्य विप्रेन्द्र प्रमाणमकरोत्तदा}% ॥३२॥

\twolineshloka
{समाश्च सप्तपञ्चाशद्गतास्तस्य तया सह}
{चुक्रोध सततस्तस्यै ज्वालामाली बभूव ह}% ॥३३॥

\twolineshloka
{नेत्राभ्यां विस्फुलिङ्गान्स मुञ्चमानोऽतिकोपनः}
{कालरूपां तु तां दृष्ट्वा तपसः क्षयकारिणीम्}% ॥३४॥

\twolineshloka
{दुःखार्जितं क्षयं नीतं तपोदृष्ट्वातयासह}
{सकम्पोष्ठो मुनिस्तत्रप्रत्युवाचाकुलेन्द्रियः}% ॥३५॥

\twolineshloka
{तां शशापाथमेधावी त्वं पिशाची भवेति च}
{धिक्त्वां पापे दुराचारे कुलटे पातकप्रिये}% ॥३६॥

\threelineshloka
{तस्य शापेन सा दग्धा विनयावनता स्थिता}
{उवाच वचनं सुभ्रूः प्रसादं वाञ्छती मुनिम्}
{प्रसादं कुरु विप्रेन्द्र शापस्यानुग्रहं कुरु}% ॥३७॥

\threelineshloka
{सतां सङ्गो हि भवति वचोभिः सप्तभिः पदैः}
{त्वया सह मम ब्रह्मन्नीता वै बहुवत्सराः}
{एतस्मात्कारणात्स्वामिन्प्रसादं कुरु सुव्रत}% ॥३८॥

\uvacha{मुनिरुवाच}

\twolineshloka
{शृणु मे वचनं भद्रे शापानुग्रहकारकम्}
{किं करोमि त्वया पापे क्षयं नीतं महत्तपः}% ॥३९॥

\twolineshloka
{चैत्रस्य कृष्णपक्षे तु भवेदेकादशी शुभा}
{पापमोचनिका नाम सर्वपापक्षयङ्करी}% ॥४०॥

\twolineshloka
{तस्या व्रते कृते शुभ्रे पिशाचत्वं प्रयास्यति}
{इत्युक्त्वा सोऽपि मेधावी जगाम पितुराश्रमम्}% ॥४१॥

\twolineshloka
{तमागतं समालोक्य च्यवनः प्रत्युवाचतम्}
{किमेतद्विहितं पुत्र त्वया पुण्यं क्षयं कृतम्}% ॥४२॥

\uvacha{मेधाव्युवाच}

\twolineshloka
{पातकं वै कृतं तात रमिता चाप्सरा मया}
{प्रायश्चित्तं ब्रूहि तात येन पापक्षयो भवेत्}% ॥४३॥

\uvacha{च्यवन उवाच}

\twolineshloka
{चैत्रस्य चासिते पक्षे नाम्ना वै पापमोचनी}
{अस्या व्रते कृते पुत्र पापराशिः क्षयं व्रजेत्}% ॥४४॥

\twolineshloka
{इति श्रुत्वा पितुर्वाक्यं कृतं तेन व्रतोत्तमम्}
{गतं पापं क्षयं तस्य तपोयुक्तो बभूव सः}% ॥४५॥

\threelineshloka
{साप्येवं मञ्जुघोषा च कृत्वैतद्व्रतमुत्तमम्}
{पिशाचत्वाद्विनिर्मुक्ता पापमोचनिकाव्रतात्}
{दिव्यरूपधरा सा वै गता नाके वराप्सराः}% ॥४६॥

\uvacha{लोमश उवाच}

\twolineshloka
{पापमोचनिकां राजन्ये कुर्वन्ति नरोत्तमाः}
{तेषां पापं च यत्किञ्चित्तत्सर्वं च क्षयं व्रजेत्}% ॥४७॥

\twolineshloka
{पठनाच्छ्रवणाद्राजन्गोसहस्रफलं लभेत्}
{ब्रह्महा हेमहारी च सुरापो गुरुतल्पगः}% ॥४८॥

\twolineshloka
{व्रतस्य चास्य करणात्पापमुक्ता भवन्ति ते}
{बहुपुण्यप्रदं ह्येतत्करणाद्व्रतमुत्तमम्}% ॥४९॥

॥इति श्रीपाद्मे महापुराणे पञ्चपञ्चाशत्साहस्र्यां संहितायामुत्तरखण्डे उमापतिनारदसंवादान्तर्गतकृष्णयुधिष्ठिरसंवादे चैत्र-कृष्ण-पापमोचनी-एकादशी-माहात्म्यं नाम अष्टचत्वारिंशोऽध्यायः॥४८॥


\hyperref[sec:ekadashi_mahatmyam_padma_puranam]{\closesub}
\clearpage

\sect{चैत्र-शुक्ल-कामदा-एकादशी-माहात्म्यम्}
\label{sec:padma-chaitra-shukla-kamada}


\uvacha{युधिष्ठिर उवाच}

\twolineshloka
{वासुदेव नमस्तुभ्यं कथयस्व ममाग्रतः}
{चैत्रस्य शुक्लपक्षे तु किं नामैकादशी भवेत्}% ॥१॥

\uvacha{श्रीकृष्ण उवाच}

\twolineshloka
{शृणुष्वैकमना राजन्कथां पुण्यां पुरातनीम्}
{वसिष्ठो यामकथयत्प्राग्दिलीपाय पृच्छते}% ॥२॥

\uvacha{दिलीप उवाच}

\twolineshloka
{भगवञ्छ्रोतुमिच्छामि कथयस्व प्रसादतः}
{चैत्रमासि सिते पक्षे किन्नामैकादशी भवेत्}% ॥३॥

\uvacha{वसिष्ठ उवाच}

\twolineshloka
{साधु पृष्टं त्वया राजन्कथयामि तवाग्रतः}
{चैत्रस्य शुक्लपक्षे तु कामदा नाम नामतः}% ॥४॥

\twolineshloka
{एकादशी पुण्यतमा पापेन्धनदवानलः}
{शृणु राजन्कथामेतां पापघ्नीं पुण्यदायिनीम्}% ॥५॥

\twolineshloka
{पुरा नागपुरे रम्ये हेमरत्नविभूषिते}
{पुण्डरीकमुखा नागा निवसन्ति महोत्कटाः}% ॥६॥

\twolineshloka
{तस्मिन्पुरे पुण्डरीको राजा राज्यं चकार सः}
{गन्धर्वैः किन्नरैश्चैव अप्सरोभिश्च सेव्यते}% ॥७॥

\twolineshloka
{वराप्सरास्तु ललिता गन्धर्वो ललितस्तथा}
{उभौ रागेण संरक्तौ दम्पती कामपीडितौ}% ॥८॥

\twolineshloka
{रेमाते स्वगृहे रम्ये धनधान्ययुते तदा}
{ललितायाश्च हृदये पतिर्वसति सर्वदा}% ॥९॥

\twolineshloka
{हृदये तस्य ललिता नित्यं वसति भामिनी}
{एकदा पुण्डरीकोऽथ क्रीडते सदसि स्थितः}% ॥१०॥

\twolineshloka
{गीतं गानं प्रकुरुते ललितो दयितां विना}
{पदबन्धस्खलज्जिह्वो बभूव ललितां स्मरन्}% ॥११॥

\twolineshloka
{मनोभावं विदित्वास्य कर्कटो नागसत्तमः}
{पदबन्धच्युतिं तस्य पुण्डरीके न्यवेदयत्}% ॥१२॥

\twolineshloka
{श्रुत्वा कर्कोटकवचः पुण्डरीको भुजङ्गराट्}
{क्रोधसंरक्तनयनो बभूवातिभयङ्करः}% ॥१३॥

\twolineshloka
{शशाप ललितं तत्र गायन्तं मदनातुरम्}
{राक्षसो भव दुर्बुद्धे क्रव्यादः पुरुषादकः}% ॥१४॥

\twolineshloka
{यतः पत्नीवशोपेतो गायमानो ममाग्रतः}
{वचनात्तस्य राजेन्द्र रक्षोरूपो बभूव सः}% ॥१५॥

\twolineshloka
{रौद्राननो विरूपाक्षो दृष्टमात्रो भयङ्करः}
{बाहूयोजनविस्तीर्णौ मुखं कन्दरसन्निभम्}% ॥१६॥

\twolineshloka
{चन्द्रसूर्यनिभे नेत्रे ग्रीवापर्वतसन्निभा}
{नासारन्ध्रे तु विवरे अधरौ योजनायतौ}% ॥१७॥

\twolineshloka
{शरीरं तस्य राजेन्द्र उत्थितं योजनाष्टकम्}
{ईदृशो राक्षसो भूत्वा भुञ्जानः कर्मणः फलम्}% ॥१८॥

\twolineshloka
{ललिता तु तथालोक्य स्वपतिं विकृताकृतिम्}
{चिन्तयामास मनसा दुःखेन महतार्दिता}% ॥१९॥

\twolineshloka
{किं करोमि क्व गच्छामि पतिः शापेन पीडितः}
{इति संस्मृत्य संस्मृत्य मनसा शर्म नालभत्}% ॥२०॥

\twolineshloka
{चचार पतिना सार्धं ललिता गहने वने}
{बभ्राम विपिने दुर्गे कामरूपी स राक्षसः}% ॥२१॥

\twolineshloka
{निर्घृणः पापनिरतो विरूपः पुरुषादकः}
{न सुखं लभते रात्रौ न दिवा पापपीडितः}% ॥२२॥

\twolineshloka
{ललिता दुःखितातीव पतिं दृष्ट्वा तथाविधम्}
{बभ्राम तेन सार्धं सा रुदती गहने वने}% ॥२३॥

\twolineshloka
{दृष्ट्वाश्रमपदं रम्यं मुनिं संशान्तविग्रहम्}
{शीघ्रं जगाम ललिता नमस्कृत्याग्रतः स्थिता}% ॥२४॥

\twolineshloka
{तां दृष्ट्वा स मुनिः प्राह दुःखितां हि दयापरः}
{का त्वं कस्मादिहायाता सत्यं वद ममाग्रतः}% ॥२५॥

\uvacha{ललितोवाच}

\twolineshloka
{वीरधन्वेति गन्धर्वः सुता तस्य महात्मनः}
{ललितां नाम मां विद्धि पत्यर्थमिह चागताम्}% ॥२६॥

\twolineshloka
{भर्ता मे पापदोषेण राक्षसोऽभून्महामुने}
{रौद्ररूपो दुराचारस्तं दृष्ट्वा नास्ति मे सुखम्}% ॥२७॥

\twolineshloka
{साम्प्रतं शाधि मां ब्रह्मन्यत्कृत्यं तद्वद प्रभो}
{कुरुष्व तद् व्रतं भद्रे विधिपूर्वं मयोदितम्}% ॥२८॥

\uvacha{ऋषिरुवाच}

\twolineshloka
{चैत्रमासस्य रम्भोरु शुक्लपक्षोऽस्ति साम्प्रतम्}
{कामदैकादशीनाम पापघ्नी ललिते परा}% ॥२९॥

\twolineshloka
{कुरुष्व तद्व्रतं भद्रे विधिपूर्वं मयोदितम्}
{अस्य व्रतस्य यत्पुण्यं तत्स्वभर्त्रे प्रदीयताम्}% ॥३०॥

\twolineshloka
{दत्ते पुण्येक्षणात्तस्य शापदोषः प्रयास्यति}
{इति श्रुत्वा मुनेर्वाक्यं ललिता हर्षिताभवत्}% ॥३१॥

\twolineshloka
{उपोष्यैकादशीं राजन्द्वादशीदिवसे तथा}
{विप्रस्यैव समीपे तद्वासुदेवस्य चाग्रतः}% ॥३२॥

\twolineshloka
{वाक्यमुवाच ललिता स्वपत्युस्तारणाय वै}
{मया तु तद्व्रतं चीर्णं कामदाया उपोषणम्}% ॥३३॥

\twolineshloka
{तस्य पुण्यप्रभावेन गच्छत्वस्य पिशाचता}
{ललितावचनादेव वर्तमानोऽपि तत्क्षणे}% ॥३४॥

\twolineshloka
{गतपापः स ललितो दिव्यदेहो बभूव ह}
{राक्षसत्वं गतं तस्य प्राप्ता गन्धर्वता पुनः}% ॥३५॥

\twolineshloka
{हेमरत्नसमाकीर्णो रेमे ललितया सह}
{विमानवरमारूढौ पूर्वरूपाधिकौ च तौ}% ॥३६॥

\twolineshloka
{दम्पती अत्यशोभेतां कामदायाः प्रभावतः}
{इति ज्ञात्वा नृपश्रेष्ठ कर्तव्यैषा प्रयत्नतः}% ॥३७॥

\twolineshloka
{लोकानां तु हितार्थाय तवाग्रे कथिता मया}
{ब्रह्महत्यादि पापघ्नी पिशाचत्वविनाशनी}% ॥३८॥

\twolineshloka
{नातः परतरा काचित्त्रैलोक्ये सचराचरे}
{पठनाच्छ्रवणाद्राजन्वाजपेयफलं लभेत्}% ॥३९॥

॥इति श्रीपाद्मे महापुराणे पञ्चपञ्चाशत्साहस्र्यां संहितायामुत्तरखण्डे उमापतिनारदसंवादान्तर्गतकृष्णयुधिष्ठिरसंवादे चैत्र-शुक्ल-कामदा-एकादशी-माहात्म्यं नाम एकोनपञ्चाशत्तमोऽध्यायः॥४९॥


\hyperref[sec:ekadashi_mahatmyam_padma_puranam]{\closesub}
\clearpage

\sect{वैशाख-कृष्ण-वरूथिनी-एकादशी-माहात्म्यम्}
\label{sec:padma-vaishakha-krishna-varuthini ekadashi}


\uvacha{युधिष्ठिर उवाच}

\twolineshloka
{वैशाखस्यासिते पक्षे किन्नामैकादशी भवेत्}
{महिमानं कथय मे वासुदेव नमोऽस्तु ते}% ॥१॥

\uvacha{श्रीकृष्ण उवाच}

\twolineshloka
{सौभाग्यदायिनी राजन्निहलोके परत्र च}
{वैशाखकृष्णपक्षे तु नाम्ना चैव वरूथिनी}% ॥२॥

\twolineshloka
{वरूथिन्या व्रतेनैव सौख्यं भवति सर्वदा}
{पापहानिश्च भवति सौभाग्यप्राप्तिरेव च}% ॥३॥

\twolineshloka
{दुर्भगा या करोत्येनां सा स्त्री सौभाग्यमाप्नुयात्}
{लोकानां चैव सर्वेषां भुक्तिमुक्तिप्रदायिनी}% ॥४॥

\twolineshloka
{सर्वपापहरा नॄणां गर्भवासनिकृन्तनी}
{वरूथिन्या व्रतेनैव मान्धाता स्वर्गतिं गतः}% ॥५॥

\twolineshloka
{धुन्धुमारादयश्चान्ये राजानो बहवस्तथा}
{ब्रह्मकपालनिर्मुक्तो बभूव भगवान्भवः}% ॥६॥

\threelineshloka
{दशवर्षसहस्राणि तपस्तप्यति यो नरः}
{कुरुक्षेत्रे  रविग्रहे स्वर्णभारं ददाति यः}
{तत्तुल्यम्फलमाप्नोतिवरूथिन्याव्रतञ्चरन्}% ॥७॥

\twolineshloka
{श्रद्धावान्यस्तु कुरुते वरूथिन्या व्रतं नरः}
{वाञ्छितं लभते सोऽपि इहलोके परत्र च}% ॥८॥

\twolineshloka
{पवित्रा पावनी ह्येषा महापातकनाशिनी}
{भुक्तिमुक्तिप्रदा चैव कर्तॄणां नृपसत्तम}% ॥९॥

\twolineshloka
{अश्वदानान्नृपश्रेष्ठ गजदानं विशिष्यते}
{गजदानाद्भूमिदानं तिलदानं ततोऽधिकम्}% ॥१०॥

\twolineshloka
{तस्माच्च स्वर्णदानं वै अन्नदानं ततोऽधिकम्}
{अन्नदानात्परं दानं न भूतं न भविष्यति}% ॥११॥

\twolineshloka
{पितृदेवमनुष्याणां तृप्तिरन्नेन जायते}
{तत्समं कविभिः प्रोक्तं कन्यादानं नृपोत्तम}% ॥१२॥

\twolineshloka
{धेनुदानं च तत्तुल्यमित्याह भगवान्स्वयम्}
{प्रोक्तेभ्यः सर्वदानेभ्यो विद्यादानं विशिष्यते}% ॥१३॥

\twolineshloka
{तत्फलं समवाप्नोति नरः कृत्वा वरूथिनीम्}
{कन्यावित्तेन जीवन्ति ये नराः पापमोहिताः}% ॥१४॥

\twolineshloka
{पुण्यक्षयं ते गच्छन्ति निरयं यातनामयम्}
{तस्मात्सर्वप्रयत्नेन न ग्राह्यं कन्यकाधनम्}% ॥१५॥

\twolineshloka
{यश्च गृह्णाति लोभेन कन्यां क्रीत्वा च तद्धनम्}
{सोऽन्यजन्मनि राजेन्द्र ओतुर्भवति निश्चितम्}% ॥१६॥

\twolineshloka
{कन्यां पुण्येन यो दद्याद्यथाशक्ति स्वलङ्कृताम्}
{तत्पुण्यसङ्ख्यां नृपते चित्रगुप्तो न शक्नुयात्}% ॥१७॥

\twolineshloka
{तत्तुल्यं फलमाप्नोति नरः कृत्वा वरूथिनीम्}
{कांस्यं मांसं मसूरांश्च चणकान्कोद्रवांस्तथा}% ॥१८॥

\twolineshloka
{शाकं मधु परान्नं च पुनर्भोजन मैथुने}
{वैष्णवो व्रतकर्ता च दशम्यां दश वर्जयेत्}% ॥१९॥

\twolineshloka
{द्यूतं क्रीडां च निद्रां च ताम्बूलं दन्तधावनम्}
{परापवादं पैशुन्यं स्तेयं हिंसां तथा रतिम्}% ॥२०॥

\twolineshloka
{क्रोधं चैवानृतं वाक्यमेकादश्यां विवर्जयेत्}
{कांस्यं मांसं सुरां क्षौद्रं  तैलं पतितभाषणम्}% ॥२१॥

\twolineshloka
{व्यायामं च प्रवासं च पुनर्भोजनमैथुनम्}
{वृषपृष्ठं मसूरान्नं द्वादश्यां परिवर्जयेत्}% ॥२२॥

\threelineshloka
{अनेन विधिना राजन्विहिता यैर्वरूथिनी}
{सर्वपापक्षयं कृत्वा दद्यात्प्रान्तेऽक्षयां गतिम्}
{रात्रौ जागरणं कृत्वा पूजितो मधुसूदनः}% ॥२३॥

\twolineshloka
{सर्वपापविनिर्मुक्तास्ते यान्ति परमां गतिम्}
{तस्मात्सर्वप्रयत्नेन कर्तव्या पापभीरुभिः}% ॥२४॥

\threelineshloka
{क्षपारितनयाद्भीतो नरः कुर्याद्वरूथिनीम्}
{पठनाच्छ्रवणाद्राजन्गोसहस्रफलं लभेत्}
{सर्वपापविनिर्मुक्तो विष्णुलोके महीयते}% ॥२५॥

॥इति श्रीपाद्मे महापुराणे पञ्चपञ्चाशत्साहस्र्यां संहितायामुत्तरखण्डे उमापतिनारदसंवादान्तर्गतकृष्णयुधिष्ठिरसंवादे वैशाख-कृष्ण-वरूथिनी-एकादशी-माहात्म्यं नाम पञ्चाशत्तमोऽध्यायः॥५०॥


\hyperref[sec:ekadashi_mahatmyam_padma_puranam]{\closesub}
\clearpage

\sect{वैशाख-शुक्ल-मोहिनी-एकादशी-माहात्म्यम्}
\label{sec:padma-vaishakha-shukla-mohini}


\uvacha{युधिष्ठिर उवाच}

\twolineshloka
{वैशाख शुक्लपक्षे तु किन्नामैकादशी भवेत्}
{किं फलं को विधिस्तत्र कथयस्व जनार्दन}% ॥१॥

\uvacha{श्रीकृष्ण उवाच}

\twolineshloka
{इदमेव पुरा पृष्टं रामचन्द्रेण धीमता}
{वसिष्ठं प्रति राजेन्द्र यत्त्वं मामनुपृच्छसि}% ॥२॥

\uvacha{राम उवाच}

\twolineshloka
{भगवञ्छ्रोतुमिच्छामि व्रतानामुत्तमं व्रतम्}
{सर्वपापक्षयकरं सर्वदुःखनिकृन्तनम्}% ॥३॥

\twolineshloka
{मया दुःखानि भुक्तानि सीताविरहजानि तु}
{ततोऽहं भयभीतोऽस्मि पृच्छामि त्वां महामुने}% ॥४॥

\uvacha{वसिष्ठ उवाच}

\twolineshloka
{साधु पृष्टं त्वया राम तवैषा नैष्ठिकी मतिः}
{त्वन्नामग्रहणेनैव पूतो भवति मानवः}% ॥५॥

\twolineshloka
{तथापि कथयिष्यामि लोकानां हितकाम्यया}
{पवित्रं पावनानां च व्रतानामुत्तमं व्रतम्}% ॥६॥

\twolineshloka
{वैशाखस्य सिते पक्षे राम चैकादशी भवेत्}
{मोहिनी नाम सा प्रोक्ता सर्वपापहरापराः}% ॥७॥

\twolineshloka
{मोहजालात्प्रमुच्यन्ते पातकानां समूहतः}
{अस्या व्रत प्रभावेन सत्यं सत्यं वदाम्यहम्}% ॥८॥

\twolineshloka
{अतः कारणतो राम कर्तव्यैषा भवादृशैः}
{पातकानां क्षयकरी महादुःखविनाशिनी}% ॥९॥

\twolineshloka
{शृणुष्वैकमना राम कथां पापहरां पराम्}
{यस्याः श्रवणमात्रेण महापापं प्रणश्यति}% ॥१०॥

\twolineshloka
{सरस्वत्यास्तटे रम्ये पुरी भद्रावती शुभा}
{द्युतिमान्नाम नृपतिस्तत्र राज्यं करोति वै}% ॥११॥

\twolineshloka
{चन्द्रवंशोद्भवो नाम धृतिमान्सत्यसङ्गरः}
{तत्र वैश्यो निवसति धनधान्यसमृद्धिमान्}% ॥१२॥

\twolineshloka
{धनपाल इति ख्यातः पुण्यकर्मप्रवर्तकः}
{प्रपा कूप मठाराम तडाग गृहकारकः}% ॥१३॥

\twolineshloka
{विष्णुभक्तिरतः शान्तस्तस्यासन्पञ्चपुत्रकाः}
{सुमना द्युतिमांश्चैव मेधावी सुकृतस्तथा}% ॥१४॥

\twolineshloka
{पञ्चमो धृष्टबुद्धिश्च महापापरतः सदा}
{परस्त्रीसङ्गनिरतो विटगोष्ठी विशारदः}% ॥१५॥

\twolineshloka
{द्यूतादि व्यसनासक्तः परस्त्री रतिलालसः}
{न च देवार्चने बुद्धिर्नपितॄन्न द्विजान्प्रति}% ॥१६॥

\twolineshloka
{अन्यायवर्ती दुष्टात्मा पितुर्द्रव्यक्षयङ्करः}
{अभक्ष्यभक्षकः पापी सुरापाने रतः सदा}% ॥१७॥

\twolineshloka
{वेश्याकण्ठे क्षिप्तबाहुर्भ्रमन्दुष्टश्चतुष्पथे}
{पित्रा निष्कासितो गेहात्परित्यक्तश्च बान्धवैः}% ॥१८॥

\twolineshloka
{स्वदेहभूषणान्येव क्षयं नीतानि तेन वै}
{गणिकाभिः परित्यक्तो निन्दितश्च धनक्षयात्}% ॥१९॥

\twolineshloka
{ततश्चिन्तापरो जातो वस्त्रहीनः क्षुधार्दितः}
{किं करोमि क्वगच्छामि केनोपायेन जीव्यते}% ॥२०॥

\twolineshloka
{तस्करत्वं समारब्धं तत्रैव नगरे पितुः}
{गृहीतो राजपुरुषैर्मुक्तश्च पितृगौरवात्}% ॥२१॥

\twolineshloka
{पुनर्बद्धः पुनस्त्यक्तः पुनर्बद्धः ससम्भ्रमैः}
{धृष्टबुद्धिर्दुराचारो निबध्य निगडै र्दृढैः}% ॥२२॥

\twolineshloka
{कशाघातैस्ताडितश्च पीडितश्च पुनः पुनः}
{न स्थातव्यं हि मन्दात्मंस्त्वया मद्देशगोचरे}% ॥२३॥

\twolineshloka
{एवमुक्त्वा ततो राज्ञा मोचितो दृढबन्धनात्}
{निर्जगाम भयात्तस्य गतोऽसौ गहनं वनम्}% ॥२४॥

\twolineshloka
{क्षुत्तृषापीडितश्चायमितश्चेतश्च धावति}
{सिंहवन्निजघानासौ मृग शूकर चित्रलान्}% ॥२५॥

\twolineshloka
{आमिषाहार निरतो वने तिष्ठति सर्वदा}
{करे शरासनं कृत्वा निषङ्गं पृष्ठ सङ्गतम्}% ॥२६॥

\twolineshloka
{अरण्यचारिणो हन्ति पक्षिणश्च पदाचरन्}
{चकोरांश्च मयूरांश्च कङ्क तित्तिर मूषिकान्}% ॥२७॥

\twolineshloka
{एतानन्यान्हिनस्त्यन्धो धृष्टबुद्धिस्तु निर्घृणः}
{पूर्वजन्मकृतैः पापैर्निमग्नः पापकर्दमे}% ॥२८॥

\twolineshloka
{दुःखशोकसमाविष्टः पीड्यमानो दिवानिशम्}
{कौण्डिन्यस्याश्रमपदं प्राप्तः पुण्यागमात्क्वचित्}% ॥२९॥

\twolineshloka
{माधवे मासि जाह्नव्याः कृतस्नानं तपोधनम्}
{आससाद धृष्टबुद्धिः शोकभारेण पीडितः}% ॥३०॥

\twolineshloka
{तद्वस्त्रबिन्दुस्पर्शेन गतपापो हताशुभः}
{कौण्डिन्यस्याग्रतः स्थित्वा प्रत्युवाच कृताञ्जलि}% ॥३१॥

\uvacha{धृष्टबुद्धिरुवाच}

\twolineshloka
{भो भो ब्रह्मन्द्विजश्रेष्ठ दयां कृत्वा ममोपरि}
{येन पुण्यप्रभावेन मुक्तिर्भवति तद्वद}% ॥३२॥

\uvacha{कौण्डिन्य उवाच}

\twolineshloka
{शृणुष्वैकमनाभूत्वा येन पापक्षयस्तव}
{वैशाखस्य सिते पक्षे मोहिनी नाम विश्रुता}% ॥३३॥

\twolineshloka
{एकादशीव्रतं तस्याः कुरु मद्वाक्यनोदितः}
{मेरुतुल्यानि पापानि क्षयं गच्छन्ति देहिनाम्}% ॥३४॥

\twolineshloka
{बहुजन्मार्जितान्येषा मोहिनी समुपोषिता}
{इति वाक्यं मुनेः श्रुत्वा धृष्टबुद्धिः प्रसन्नधीः}% ॥३५॥

\twolineshloka
{व्रतं चकार विधिवत्कौण्डिन्यस्योपदेशतः}
{कृते व्रते नृपश्रेष्ठ गतपापो बभूव सः}% ॥३६॥

\twolineshloka
{दिव्यदेहस्ततो भूत्वा गरुडोपरिसंस्थितः}
{जगाम वैष्णवं लोकं सर्वोपद्रव वर्जितम्}% ॥३७॥

\twolineshloka
{इतीदृशं रामचन्द्र उत्तमं मोहिनी व्रतम्}
{नातः परतरं किञ्चित्त्रैलोक्ये सचराचरे}% ॥३८॥

\twolineshloka
{यज्ञादितीर्थदानानि कलां नार्हन्ति षोडशीम्}
{पठनाच्छ्रवणाद्राजन्गोसहस्र फलं लभेत्}% ॥३९॥

॥इति श्रीपाद्मे महापुराणे पञ्चपञ्चाशत्साहस्र्यां संहितायामुत्तरखण्डे उमापतिनारदसंवादान्तर्गतकृष्णयुधिष्ठिरसंवादे वैशाख-शुक्ल-मोहिनी-एकादशी-माहात्म्यं नाम नामैकपञ्चाशत्तमोऽध्यायः॥५१॥


\hyperref[sec:ekadashi_mahatmyam_padma_puranam]{\closesub}
\clearpage

\sect{ज्येष्ठ-कृष्णापरा-एकादशी-माहात्म्यम्}
\label{sec:padma-jyeshtha-krishnapara}


\uvacha{युधिष्ठिर उवाच}

\twolineshloka
{ज्येष्ठस्य कृष्णपक्षे तु किं नामैकादशी भवेत् 	।}
{श्रोतुमिच्छामि माहात्म्यं तद्वदस्व जनार्दन}% ॥१॥

\uvacha{श्रीकृष्ण उवाच}

\twolineshloka
{साधु पृष्टं त्वया राजन्लोकानां हितकाम्यया}
{बहुपुण्यप्रदा ह्येषा महापातकनाशिनी}% ॥२॥

\twolineshloka
{अपरा नाम राजेन्द्र अपरा पुत्रदायिनी}
{लोके प्रसिद्धितां याति अपरां यस्तु सेवते}% ॥३॥

\twolineshloka
{ब्रह्महत्याभिभूतोऽपि गोत्रहा भ्रूणहा तथा}
{परापवादवादी च परस्त्री रसिकोऽपि च}% ॥४॥

\twolineshloka
{अपरा सेवनाद्राजन्विपाप्मा भवति ध्रुवम्}
{कूटसाक्ष्यं कूटमानं तुलाकूटं करोति यः}% ॥५॥

\twolineshloka
{कूटवेदं पठेद्यस्तु कूटशास्त्रं तथैव च}
{ज्योतिषी गणकः कूटः कूटायुर्वैदिको भिषक्}% ॥६॥

\twolineshloka
{कूटसाक्ष्य समायुक्तो विज्ञेया नरकौकसः}
{अपरा सेवनाद्राजन्पापैर्मुक्ता भवन्ति ते}% ॥७॥

\twolineshloka
{क्षत्त्रियः क्षात्रधर्मं यस्त्यक्त्वा युद्धात्पलायते}
{स याति नरकं घोरं स्वीयधर्मबहिष्कृतः}% ॥८॥

\twolineshloka
{अपरा सेवनात्सोऽपि पापं त्यक्त्वा दिवं व्रजेत्}
{विद्यावान्यः स्वयं शिष्यो गुरुनिन्दां करोति च}% ॥९॥

\twolineshloka
{स महापातकैर्युक्तो निरयं याति दारुणम्}
{अपरा सेवनात्सोऽपि सद्गतिं प्राप्नुयान्नरः}% ॥१०॥

\twolineshloka
{महिमानमपरायाः शृणु राजन्वदाम्यहम्}
{मकरस्थे रवौ माघे प्रयागे यत्फलं नृणाम्}% ॥११॥

\twolineshloka
{काश्यां यत्प्राप्यते पुण्यमुपरागे निमज्जनात्}
{गयायां पिण्डदानेन पितॄणां तृप्तिदो यथा}% ॥१२॥

\twolineshloka
{सिंहस्थिते देवगुरौ गौतम्यां स्नातको नरः}
{कन्यागते गुरौ राजन्कृष्णवेणी निमज्जनात्}% ॥१३॥

\twolineshloka
{यत्फलं समवाप्नोति कुम्भकेदार दर्शनात्}
{बदर्याश्रमयात्रायां तत्तीर्थसेवनादपि}% ॥१४॥

\twolineshloka
{यत्फलं समवाप्नोति कुरुक्षेत्रे रविग्रहे}
{गजाश्व हेमदानेन यज्ञं कृत्वा सदक्षिणम्}% ॥१५॥

\twolineshloka
{तादृशं फलमाप्नोति अपरा व्रतसेवनात्}
{अर्धप्रसूतां गां दत्वा सुवर्णं वसुधां तथा}% ॥१६॥

\twolineshloka
{नरो यत्फलमाप्नोति अपराया व्रतेन तत्}
{पापद्रुमकुठारीयं पापेन्धन दवानलः}% ॥१७॥

\twolineshloka
{पापान्धकारतरणिः पापसारङ्ग केसरी}
{बुद्बुदा इव तोयेषु पुत्तिका इव जन्तुषु}% ॥१८॥

\twolineshloka
{जायन्ते मरणायैव एकादश्या व्रतं विना}
{अपरां समुपोष्यैव पूजयित्वा त्रिविक्रमम्}% ॥१९॥

\threelineshloka
{सर्वपापविनिर्मुक्तो विष्णुलोके महीयते}
{लोकानां च हितार्थाय तवाग्रे कथितं मया}
{पठनाच्छ्रवणाद्राजन्गोसहस्रफलं लभेत्}% ॥२०॥

॥इति श्रीपाद्मे महापुराणे पञ्चपञ्चाशत्साहस्र्यां संहितायामुत्तरखण्डे उमापतिनारदसंवादान्तर्गतकृष्णयुधिष्ठिरसंवादे ज्येष्ठ-कृष्णापरा-एकादशी-माहात्म्यं नाम द्विपञ्चाशत्तमोऽध्यायः॥५२॥


\hyperref[sec:ekadashi_mahatmyam_padma_puranam]{\closesub}
\clearpage

\sect{ज्येष्ठ-शुक्ल-निर्जला-एकादशी-माहात्म्यम्}
\label{sec:padma-jyeshtha-shukla-nirjala}


\uvacha{युधिष्ठिर उवाच}

\twolineshloka
{अपरायाश्च माहात्म्यं श्रुतं सर्वं जनार्दन}
{ज्येष्ठस्य शुक्लपक्षे तु स्याद्या तां वद मानद}% ॥१॥

\uvacha{श्रीकृष्ण उवाच}

\twolineshloka
{एनां वक्ष्यति धर्मात्मा व्यासः सत्यवतीसुतः}
{सर्वशास्त्रार्थतत्त्वज्ञो वेदवेदाङ्गपारगः}% ॥२॥

\uvacha{युधिष्ठिर उवाच}

\twolineshloka
{श्रुता मे मानवा धर्मा वासिष्ठाश्च श्रुता मया}
{द्वैपायन यथावत्त्वं वैष्णवान्वक्तुमर्हसि}% ॥३॥

\uvacha{श्रीवेदव्यास उवाच}

\twolineshloka
{श्रुतास्तु मानवा धर्मा वैदिकाश्च श्रुतास्त्वया}
{कलौ युगे न शक्यन्ते ते वै कर्तुं नराधिप}% ॥४॥

\twolineshloka
{सुखोपायमल्पधनमल्पक्लेशं महाफलम्}
{पुराणानां च सर्वेषां सारभूतं महामते}% ॥५॥

\twolineshloka
{एकादश्यां न भुञ्जीत पक्षयोरुभयोरपि}
{द्वादश्यां तु शुचिर्भूत्वा पुष्पैः सम्पूज्य केशवम्}% ॥६॥

\twolineshloka
{अन्नं भुञ्जीत सत्कृत्य पश्चाद्विप्रपुरःसरम्}
{सूतकेऽपि न भोक्तव्यं नाशौचे च जनाधिप}% ॥७॥

\twolineshloka
{यावज्जीवं व्रतमिदं कर्तव्यं पुरुषर्षभ}
{स्वर्गतिं प्राप्तुमिच्छद्भिरत्र नैवास्ति संशयः}% ॥८॥

\twolineshloka
{आपपापा दुराचाराः पापिष्ठा धर्मवर्जिताः}
{एकादश्यां न भुञ्जाना न ते यान्ति यमान्तिकम्}% ॥९॥

\twolineshloka
{इति तद्वचनं श्रुत्वा कम्पितोऽश्वत्थपत्रवत्}
{भीमसेनो महाबाहुर्नत्वोवाच गुरुं प्रति}% ॥१०॥

\uvacha{भीमसेन उवाच}

\twolineshloka
{पितामह महाबुद्धे शृणु मे परमं वचः}
{युधिष्ठिरश्च कुन्ती च तथा द्रुपदनन्दिनी}% ॥११॥

\twolineshloka
{अर्जुनो नकुलश्चैव सहदेवस्तथैव च}
{एकादश्यां न भुञ्जन्ति कदाचिदपि सुव्रताः}% ॥१२॥

\twolineshloka
{ते मां ब्रुवन्ति वै नित्यं मा भुङ्क्ष्व त्वं वृकोदर}
{अहं तानब्रुवं तात बुभुक्षा दुःसहा मम}% ॥१३॥

\twolineshloka
{दानं दास्यामि विधिवत्पूजयिष्यामि केशवम्}
{भीमसेनवचः श्रुत्वा व्यासो वचनमब्रवीत्}% ॥१४॥

\uvacha{व्यास उवाच}

\twolineshloka
{यदि स्वर्गमभीष्टं ते नरकं दुष्टमेव च}
{एकादश्यां न भोक्तव्यं पक्षयोरुभयोरपि}% ॥१५॥

\uvacha{भीमसेन उवाच}

\twolineshloka
{पितामह महाबुद्धे कथयामि तवाग्रतः}
{एकभक्ते न शक्नोमि उपवासे कुतः प्रभो}% ॥१६॥

\twolineshloka
{वृकोऽपिनाम यो वह्निः स सदा जठरे मम}
{अतिवेलं यदाश्नामि तदा समुपशाम्यति}% ॥१७॥

\threelineshloka
{नैकं शक्नोम्यहं कर्तुमुपवासं महामुने}
{येनैव प्राप्यते स्वर्गस्तत्कर्ताऽस्मि यथातथम्}
{तदेकं वद निश्चित्य येन श्रेयोऽहमाप्नुयाम्}% ॥१८॥

\uvacha{व्यास उवाच}

\twolineshloka
{वृषस्थे मिथुनस्थे वा यदा चैकादशी भवेत्}
{ज्येष्ठमासे प्रयत्नेन सोपोष्योदकवर्जिता}% ॥१९॥

\twolineshloka
{गण्डूषाचमनं वारि वर्जयित्वोदकं बुधः}
{उपभुञ्जीत नैवेह व्रतभङ्गोऽन्यथा भवेत्}% ॥२०॥

\twolineshloka
{उदयादुदयं यावद्वर्जयित्वोदकं नरः}
{श्रूयतां समवाप्नोति द्वादशद्वादशी फलम्}% ॥२१॥

\twolineshloka
{ततः प्रभाते विमले द्वादश्यां स्नानमाचरेत्}
{जलं सुवर्णं दत्त्वा च द्विजातिभ्यो यथाविधि}% ॥२२॥

\twolineshloka
{भुञ्जीत कृतकृत्यस्तु ब्राह्मणैः सहितो वशी}
{एवं कृते च यत्पुण्यं भीमसेन शृणुष्व तत्}% ॥२३॥

\twolineshloka
{संवत्सरे तु याश्चैव एकादश्यो भवन्ति हि}
{तासां फलमवाप्नोति ह्यत्र मे नास्ति संशयः}% ॥२४॥

\twolineshloka
{इति मां केशवः प्राह शङ्खचक्रगदाधरः}
{सर्वान्परित्यज्य पुमान्मामेकं शरणं व्रजेत्}% ॥२५॥

\twolineshloka
{एकादश्यां निराहारस्ततः पापात्प्रमुच्यते}
{द्रव्यशुद्धिः कलौ नास्ति संस्कारः स्मार्त एव च}% ॥२६॥

\twolineshloka
{वैदिकस्तु कुतश्चापि प्राप्ते दुष्टे कलौ युगे}
{किं नु ते बहुनोक्तेन वायुपुत्र पुनः पुनः}% ॥२७॥

\twolineshloka
{एकादश्यां न भुञ्जीत पक्षयोरुभयोरपि}
{एकादश्यां सिते पक्षे ज्येष्ठेमास्युदकं विना}% ॥२८॥

\twolineshloka
{पुण्यं फलमवाप्नोति तच्छृणुष्व वृकोदर}
{संवत्सरे तु या प्रोक्ताः शुक्लः कृष्णा वृकोदर}% ॥२९॥

\twolineshloka
{उपोषिता हि सर्वाः स्युरेकादश्यो न संशयः}
{धनधान्यप्रदा पुण्या पुत्रारोग्यशुभप्रदा}% ॥३०॥

\twolineshloka
{उपोषिता नरव्याघ्र इति सत्यं ब्रवीमि ते}
{यमदूता महाकायाः करालाः कृष्णरूपिणः}% ॥३१॥

\twolineshloka
{दण्डपाशधरा रौद्रा नोपसर्पन्ति तं नरम्}
{पीताम्बरधरा सौम्याश्चक्रहस्ता मनोजवाः}% ॥३२॥

\twolineshloka
{अन्तकालेन यन्त्येते वैष्णवान्वैष्णवीपुरीम्}
{तस्मात्सर्वप्रयत्नेन उपोष्योदकवर्जिता}% ॥३३॥

\twolineshloka
{जलधेनुं तदा दत्त्वा सर्वपापैः प्रमुच्यते}
{ततस्त्वमस्यां कौन्तेय सोपवासोऽर्चनं हरेः}% ॥३४॥

\twolineshloka
{कुरु सर्वप्रयत्नेन सर्वपापप्रशान्तये}
{स्वप्नेन मेऽपराधोस्ति दन्तरागतयापि वा}% ॥३५॥

\twolineshloka
{भोक्ष्येपरेऽह्नि देवेश ह्यशनं वासराद्धरेः}
{इत्युच्चार्य ततो मन्त्र उपवासपरो भवेत्}% ॥३६॥

\twolineshloka
{सर्वपापविनाशाय श्रद्धा दम समन्वितः}
{मेरुमन्दरमात्राघं स्त्रिया पुंसा च यत्कृतम्}% ॥३७॥

\twolineshloka
{सर्वं तद्भस्मतां याति एकादश्याः प्रभावतः}
{न शक्नुवन्ति ये दातुं जलधेनुं नराधिप}% ॥३८॥

\twolineshloka
{सकाञ्चनः प्रदातव्यो घटको वस्त्रसंयुतः}
{तोयस्य नियमं योऽस्या कुरुते वै स पुण्यभाक्}% ॥३९॥

\twolineshloka
{फलं कोटिसुवर्णस्य यामेयामे श्रुतं फलम्}
{स्नानं दानं जपं होमं यदस्यां कुरुते नरः}% ॥४०॥

\twolineshloka
{तत्सर्वं चाक्षयं प्राप्तमेतत्कृष्णप्रभाषितम्}
{किं वापरेण धर्मेण निर्जलैकादशीं विना}% ॥४१॥

\twolineshloka
{उपोष्य सम्यग्विधिवद्वैष्णवं पदमाप्नुयात्}
{सुवर्णमन्नं वासो वा यदस्यां सम्प्रदीयते}% ॥४२॥

\twolineshloka
{तदस्य कुरुशार्दूल सर्वं चाप्यक्षयं भवेत्}
{एकादश्यां दिने योऽन्नं भुङ्क्ते पापं भुनक्ति सः}% ॥४३॥

\twolineshloka
{इहलोके स चाण्डालो मृतः प्राप्नोति दुर्गतिम्}
{ये च दास्यन्ति दानानि द्वादश्यां समुपोषिताः}% ॥४४॥

\twolineshloka
{ज्येष्ठमासे सिते पक्षे प्राप्स्यन्ति परमं पदम्}
{ब्रह्महा मद्यपः स्तेनो गुरुद्वेषी सदानृती}% ॥४५॥

\twolineshloka
{मुच्यन्ते पातकैः सर्वैर्निर्जलायैरुपोषिता}
{विशेषं शृणु कौन्तेय निर्जलैकादशी दिने}% ॥४६॥

\twolineshloka
{यत्कर्तव्यं नरैः स्त्रीभिर्दानं श्रद्धासमन्वितैः}
{जलशायी च सम्पूज्यो देया धेनुस्तथाम्मयी}% ॥४७॥

\twolineshloka
{प्रत्यक्षा वा नृपश्रेष्ठ घृतधेनुरथापि वा}
{दक्षिणाभिः सुपुष्टाभिर्मिष्टान्नैश्च पृथग्विधैः}% ॥४८॥

\twolineshloka
{तोषणीयाः प्रयत्नेन द्विजा धर्मभृतां वर}
{तुष्टा भवन्ति वै विप्रास्तैस्तुष्टैर्मोक्षदो हरिः}% ॥४९॥

\twolineshloka
{आत्मद्रोहः कृतस्तैर्हि यैरेषा न ह्युपोषिता}
{पापात्मानो दुराचारा मुष्टास्ते नात्र संशयः}% ॥५०॥

\twolineshloka
{कुलानां शतमागामि अतीतानां तथा शतम्}
{आत्मना सह तैर्नीतं वासुदेवस्य मन्दिरम्}% ॥५१॥

\twolineshloka
{शान्तैर्दान्तैर्दानपरैरर्चयद्भिस्तथा हरिम्}
{कुर्वद्भिर्जागरं रात्रौ यैरेषा समुपोषिता}% ॥५२॥

\twolineshloka
{अन्नं वस्त्रं तथा गावो जलं शय्यासनं शुभम्}
{कमण्डलुं तथा छत्रं दातव्यं निर्जला दिने}% ॥५३॥

\twolineshloka
{उपानहौ यो ददाति पात्रभूते द्विजोत्तमे}
{स सौवर्णेन यानेन स्वर्गलोके महीयते}% ॥५४॥

\twolineshloka
{यश्चेमां शृणुयाद्भक्त्या यश्चापि परिकीर्तयेत्}
{उभौ तौ स्वर्गमाप्नोति नात्र कार्या विचारणा}% ॥५५॥

\twolineshloka
{यत्फलं सन्निहत्यायां राहुग्रस्ते दिवाकरे}
{कृत्वा श्राद्धं लभेन्मर्त्यस्तदस्याः श्रवणादपि}% ॥५६॥

\twolineshloka
{नियमं च प्रकर्तव्यं दन्तधावनपूर्वकम्}
{एकादश्यां निराहारो वर्जयिष्यामि वै जलम्}% ॥५७॥

\twolineshloka
{केशवप्रीणनार्थाय अन्यदाचमनादृते}
{द्वादश्यां देवदेवेशः पूजनीयस्त्रिविक्रमः}% ॥५८॥

\twolineshloka
{गन्धैर्धूपैस्तथा पुष्पैर्वासोभिः प्रियदर्शनैः}
{पूजयित्वा विधानेन मन्त्रमेतमुदीरयेत्}% ॥५९॥

\twolineshloka
{देवदेव हृषीकेश संसारार्णवतारक}
{उदकुम्भप्रदानेन नय मां परमां गतिम्}% ॥६०॥

\twolineshloka
{ज्येष्ठे मासि तु वै भीम या शुक्लैकादशी शुभा}
{निर्जला समुपोष्यात्र जलकुम्भान्सशर्करान्}% ॥६१॥

\twolineshloka
{प्रदाय विप्रमुख्येभ्यो मोदते विष्णुसन्निधौ}
{ततः कुम्भाः प्रदातव्या ब्राह्मणानां  च भक्तितः}% ॥६२॥

\twolineshloka
{भोजयित्वा ततो विप्रान्स्वयं भुञ्जीत तत्परः}
{एवं यः कुरुते पूर्णां द्वादशीं पापनाशिनीम्}% ॥६३॥

\threelineshloka
{सर्वपापविनिर्मुक्तो पदं गच्छत्यनामयम्}
{ततः प्रभृति भीमेन कृता ह्येकादशी शुभा}
{पाण्डवद्वादशी नाम्ना लोके ख्याता बभूव ह}% ॥६४॥

॥इति श्रीपाद्मे महापुराणे पञ्चपञ्चाशत्साहस्र्यां संहितायामुत्तरखण्डे उमापतिनारदसंवादान्तर्गतकृष्णयुधिष्ठिरसंवादे ज्येष्ठ-शुक्ल-निर्जला-एकादशी-माहात्म्यं नाम त्रिपञ्चाशत्तमोऽध्यायः॥५३॥


\hyperref[sec:ekadashi_mahatmyam_padma_puranam]{\closesub}
\clearpage

\sect{आषाढ-कृष्ण-योगिनी-एकादशी-माहात्म्यम्}
\label{sec:padma-ashadha-krishna-yogini}


\uvacha{युधिष्ठिरउवाच}

\twolineshloka
{आषाढकृष्णपक्षे तु किं वै एकादशी भवेत्}
{कथयस्व प्रसादेन वासुदेव ममाग्रतः}% ॥१॥

\uvacha{श्रीकृष्ण उवाच}

\twolineshloka
{व्रतानामुत्तमं राजन्कथयामि तवाग्रतः}
{सर्वपापक्षयकरं सर्वमुक्तिप्रदायकम्}% ॥२॥

\twolineshloka
{आषाढस्यासिते पक्षे योगिनी नाम नामतः}
{एकादशी नृपश्रेष्ठ महापातकनाशिनी}% ॥३॥

\twolineshloka
{संसारार्णवमग्नानां पोतभूता सनातनी}
{जगत्त्रये सारभूता योगिनी व्रतकारिणाम्}% ॥४॥

\twolineshloka
{कथयामि तवाग्रेऽहं कथां पौराणिकीं शुभाम्}
{अलकायां राजराजः शिवभक्तिपरायणः}% ॥५॥

\twolineshloka
{तस्यासीत्पुष्पबटुको हेममालीति नामतः}
{तस्य पत्नी सुरूपा च विशालाक्षीति नामतः}% ॥६॥

\twolineshloka
{स तस्यां चासक्तमना कामपाशवशं गतः}
{मानसात्पुष्पनिचयमानीय स्वगृहे स्थितः}% ॥७॥

\twolineshloka
{पत्नीप्रेमरसासक्तो न कुबेरालयं गतः}
{कुबेरो देवसदने करोति शिवपूजनम्}% ॥८॥

\twolineshloka
{मध्याह्नसमये राजन्पुष्पागमसमीक्षकः}
{हेममाली स्वभवने रमते कान्तया सह}% ॥९॥

\threelineshloka
{यक्षराट्प्रत्युवाचाथ कालातिक्रमकोपितः}
{कस्मान्नायाति भो यक्षा हेममाली दुरात्मवान्}
{निश्चयः क्रियतामस्य इत्युवाच पुनः पुनः}% ॥१०॥

\uvacha{यक्षा ऊचुः}

\twolineshloka
{वनिताकामुको गेहे रमते स्वेच्छया नृप}
{तेषां वाक्यं समाकर्ण्य कुबेरः कोपपूरितः}% ॥११॥

\twolineshloka
{आह्वयामास तं तूर्णं बटुकं हेममालिनम्}
{ज्ञात्वा कालात्ययं सोऽपि भयव्याकुललोचनः}% ॥१२॥

\twolineshloka
{अस्नात एव आगत्य कुबेरस्याग्रतः स्थितः}
{तं दृष्ट्वा धनदः क्रुद्धः क्रोधसंरक्तलोचनः}% ॥१३॥
\onelineshloka*
{प्रत्युवाच रुषाविष्टः कोपप्रस्फुरिताधरः}

\uvacha{धनद उवाच}
\onelineshloka
{आः पाप दुष्ट दुर्वृत्त कृतवान्देवहेलनम्}% ॥१४॥

\twolineshloka
{अष्टादशकुष्ठवृतो वियुक्तः कान्तया तया}
{अस्मात्स्थानादपध्वस्तो गच्छ स्वप्नमथाधम}% ॥१५॥

\twolineshloka
{इत्युक्ते वचने तस्य तस्मात्स्थानात्पपात सः}
{महादुःखाभिभूतश्च कुष्ठैः पीडितविग्रहः}% ॥१६॥

\twolineshloka
{न सुखं दिवसे तस्य न निद्रां लभते निशि}
{छायायां पीडिततनुर्निदाघेऽत्यन्तपीडितः}% ॥१७॥

\twolineshloka
{शिवपूजाप्रभावेन स्मृतिस्तस्य न लुप्यते}
{पातकेनाभिभूतोऽपि पूर्वं कर्म स्मरत्यसौ}% ॥१८॥

\twolineshloka
{भ्रममाणस्ततो गच्छन्हिमाद्रिं पर्वतोत्तमम्}
{तत्रापश्यन्मुनिवरं मार्कण्डेयं तपोनिधिम्}% ॥१९॥

\twolineshloka
{यस्यायुर्विद्यते राजन्ब्रह्मणो वयसा समम्}
{ववन्दे चरणौ तस्य दूरतः पापकर्मकृत्}% ॥२०॥

\twolineshloka
{मार्कण्डेयो मुनिवरो दृष्ट्वा तं कम्पितं तथा}
{परोपकरणार्थाय समाहूयेदमब्रवीत्}% ॥२१॥

\twolineshloka
{कस्मात्कुष्ठाभिभूतस्त्वं कुतो निन्द्यतरो ह्यसि}
{इत्युक्तः स प्रत्युवाच मार्कण्डेयं महामुनिम्}% ॥२२॥

\uvacha{हेममाल्युवाच}

\twolineshloka
{राजराजस्यानुचरो हेममालीति नामतः}
{मानसात्पद्मनिचयमानीय प्रत्यहं मुने}% ॥२३॥

\twolineshloka
{शिवपूजनवेलायां कुबेराय समर्पये}
{एकस्मिन्दिवसे चैव कालश्चाविदितो मया}% ॥२४॥

\twolineshloka
{पत्नीसौख्यप्रसक्तेन शोकव्याकुलचेतसा}
{ततः क्रुद्धेन शप्तोऽस्मि राजराजेन वै मुने}% ॥२५॥

\twolineshloka
{कुष्ठाभिभूतः सञ्जातो वियुक्तः कान्तया तया}
{अधुना तव सान्निध्यं प्राप्तोऽस्मि शुभकर्मणा}% ॥२६॥

\twolineshloka
{सतां स्वभावतश्चित्तं परोपकरणे क्षमम्}
{इति ज्ञात्वा मुनिश्रेष्ठ मां प्रशाधि कृतागसम्}% ॥२७॥

\uvacha{मार्कण्डेय उवाच}

\twolineshloka
{त्वया सत्यमिह प्रोक्तं नासत्यं भाषितं यतः}
{अतो व्रतोपदेशं ते कथयामि शुभप्रदम्}% ॥२८॥

\twolineshloka
{आषाढे कृष्णपक्षे तु योगिनीव्रतमाचर}
{अस्य व्रतस्य पुण्येन कुष्ठं यास्यति वै ध्रुवम्}% ॥२९॥

\twolineshloka
{इति वाक्यमृषेः श्रुत्वा दण्डवत्पतितो भुवि}
{उत्थापितः स मुनिना बभूवातीव हर्षितः}% ॥३०॥

\twolineshloka
{मार्कण्डेयोपदेशेन व्रतं तेन कृतं यथा}
{अष्टादशेव कुष्ठानि गतानि तस्य सर्वशः}% ॥३१॥

\twolineshloka
{मुनेर्वाचा ततः सम्यग्व्रते चीर्णेऽभवत्सुखी}
{ईदृग्विधं नृपश्रेष्ठ कथितं योगिनीव्रतम्}% ॥३२॥

\twolineshloka
{अष्टाशीतिसहस्राणि द्विजान्भोजयते तु यः}
{तत्समं फलमाप्नोति योगिनीव्रतकृन्नरः}% ॥३३॥

\twolineshloka
{महापापप्रशमनं महापुण्यफलप्रदम्}
{पठनाच्छ्रवणान्मर्त्यः सर्वपापैः प्रमुच्यते}% ॥३४॥

॥इति श्रीपाद्मे महापुराणे पञ्चपञ्चाशत्साहस्र्यां संहितायामुत्तरखण्डे उमापतिनारदसंवादान्तर्गतकृष्णयुधिष्ठिरसंवादे आषाढ-कृष्ण-योगिनी-एकादशी-माहात्म्यं नाम चतुष्पञ्चाशत्तमोऽध्यायः॥५४॥


\hyperref[sec:ekadashi_mahatmyam_padma_puranam]{\closesub}
\clearpage

\sect{आषाढ-शुक्ल-शयनी-एकादशी-माहात्म्यम्}
\label{sec:padma-ashadha-shukla-shayani}


\uvacha{युधिष्ठिर उवाच}

\twolineshloka
{आषाढस्य सिते पक्षे का च एकादशी भवेत्}
{किं नाम को विधिस्तस्या एतद्विस्तरतो वद}% ॥१॥

\uvacha{श्रीकृष्ण उवाच}

\twolineshloka
{कथयामि महापुण्यां स्वर्गमोक्षप्रदायिनीम्}
{शयनीं नामनामेति सर्वपापहरां पराम्}% ॥२॥

\twolineshloka
{यस्याः श्रवणमात्रेण वाजपेयफलं लभेत्}
{सत्यं सत्यं मया प्रोक्तं नातः परतरं नृणाम्}% ॥३॥

\twolineshloka
{पापिनां पापनाशाय सृष्टा धात्रा महोत्तमा}
{अतः परा न राजेन्द्र वर्तते मोक्षदायिनी}% ॥४॥

\twolineshloka
{एतस्मात्कारणाद्राजन्श्रूयतां गतिरुत्तमा}
{भवेन्नराणां श्रोतॄणां कथायाः श्रवणादपि}% ॥५॥

\twolineshloka
{ते सदा वैष्णवा राजन्मम भक्तिपरायणाः}
{आषाढे वामनश्चैव पूज्यते परमेश्वरः}% ॥६॥

\twolineshloka
{वामनः पूजितो येन कमलैः कमलेक्षणः}
{आषाढस्य सिते पक्षे कामिकाया दिने तथा}% ॥७॥

\twolineshloka
{तेनार्चितं जगत्सर्वं त्रयो देवाः सनातनाः}
{कृता चैकादशी येन हरिवासरमुत्तमम्}% ॥८॥

\uvacha{युधिष्ठिर उवाच}

\twolineshloka
{संशयोऽस्ति महान्मेऽत्र श्रूयतां पुरुषोत्तम}
{कथं सुप्तोऽसि देवेश कथं च बलिमाश्रितः}% ॥९॥

\twolineshloka
{कथं च भूमौ संवेशः किं कुर्वन्ति जनाः परे}
{एतद्वद महाप्राज्ञ संशयोऽस्ति महान्मम}% ॥१०॥

\uvacha{श्रीकृष्ण उवाच}

\twolineshloka
{श्रूयतां राजशार्दूल कथां पापहरां पराम्}
{यस्याः श्रवणमात्रेण सर्वपापक्षयो भवेत्}% ॥११॥

\twolineshloka
{बलिनामा पूर्वमासीद्दैत्यस्त्रेतायुगे नृप}
{पूजयंश्चैव मां नित्यं मद्भक्तो मत्परायणः}% ॥१२॥

\twolineshloka
{यज्ञैस्तु विधिवद्दैत्यो यजते मां सनातनम्}
{भक्त्या च परया राजन्यज्ञकृद्व्रतकृत्तथा}% ॥१३॥

\twolineshloka
{परं विचार्य बहुधा मघोना चैव सूक्तिभिः}
{गुरुणा दैवतैः सार्धं बहुधा पूजितोऽप्यहम्}% ॥१४॥

\twolineshloka
{ततो वामनरूपेण अवतारे च पञ्चमे}
{अत्युग्ररूपेण तदा सर्वब्रह्माण्डरूपिणा}% ॥१५॥

\twolineshloka
{वाक्छलेन जिता दैत्याः सत्यमाश्रित्य संस्थितः}
{शुक्रस्तं वारयामास यन्नारायण इत्ययम्}% ॥१६॥

\twolineshloka
{याचिता वसुधा राजन्सार्धत्रयपदी मया}
{सङ्कल्पोदकमात्रे तु करे तेनैव चार्पिते}% ॥१७॥

\twolineshloka
{रूपमीदृग्विधं राजंस्तदा शृणु मया कृतम्}
{भूर्लोके चरणौ न्यस्य भुवर्लोके तु जानुनी}% ॥१८॥

\twolineshloka
{स्वर्लोके च कटिं न्यस्य महर्लोके तथोदरम्}
{जनलोके च हृदयं तपोलोके तु कण्ठकम्}% ॥१९॥

\twolineshloka
{सत्यलोके मुखं स्थाप्य मस्तकं च तदूर्ध्वकम्}
{चन्द्रसूर्यग्रहाश्चैव नक्षत्राणि तथैव च}% ॥२०॥

\twolineshloka
{देवाः सेन्द्राश्च नागाश्च यक्षगन्धर्वकिन्नराः}
{स्तुवन्तो वेदसम्भूतैः सूक्तैश्च विविधैस्तथा}% ॥२१॥

\twolineshloka
{करे गृहीत्वा च बलिं त्रिपदैः पूरिता मही}
{अर्धं च तस्य पृष्ठे च पदं न्यस्तं मया तदा}% ॥२२॥

\twolineshloka
{गतो रसातलं राजन्दानवो मम पूजकः}
{क्षिप्तोऽधो दानवश्चैव किमकुर्वं ततः परम्}% ॥२३॥

\twolineshloka
{विनयेनानतोसौ वै सुप्रसन्नो जनार्दनः}
{आषाढशुक्लपक्षे तु कामिका हरिवासरः}% ॥२४॥

\twolineshloka
{तस्यामेका च मूर्तिर्मे बलिमाश्रित्य तिष्ठति}
{द्वितीया शेषपृष्ठे वै क्षीरसागरमध्यतः}% ॥२५॥

\twolineshloka
{स्वपित्येव महाराज यावदागामि कार्तिकी}
{तावद्भवेत्सुधर्मात्मा सर्वधर्मोत्तमोत्तमः}% ॥२६॥

\twolineshloka
{व्रतं च कुरुते मर्त्यः स याति परमां गतिम्}
{एतस्मात्कारणाद्राजन्कर्तव्या च प्रयत्नतः}% ॥२७॥

\twolineshloka
{नातः परतरा काचित्पवित्रा पापनाशिनी}
{यस्यां स्वपिति देवेशः शङ्खचक्रगदाधरः}% ॥२८॥

\twolineshloka
{तस्यां च पूजयेद्देवं शङ्खचक्रगदाधरम्}
{रात्रौ जागरणं कृत्वा भक्त्या चैव विशेषतः}% ॥२९॥

\twolineshloka
{नास्याः पुण्यस्य सङ्ख्यानं कर्तुं शक्तश्चतुर्मुखः}
{एवं यः कुरुते राजन्नेकादश्या व्रतोत्तमम्}% ॥३०॥

\twolineshloka
{सर्वपापहरं चैव भुक्तिमुक्तिप्रदायकम्}
{स च लोके मम सदा श्वपचोऽपि प्रियङ्करः}% ॥३१॥

\twolineshloka
{दीपदानेन पालाशपत्रे भुक्त्या व्रतेन च}
{चातुर्मास्यं नयन्तीह ते नरा मम वल्लभाः}% ॥३२॥

\twolineshloka
{चातुर्मास्ये हरौ सुप्ते भूमिशायी भवेन्नरः}
{श्रावणे वर्जयेच्छाकं  दधि भाद्रपदे तथा}% ॥३३॥

\twolineshloka
{दुग्धमाश्वयुजि त्याज्यं कार्तिके द्विदलं त्यजेत्}
{अथवा ब्रह्मचर्यस्थः स याति परमां गतिम्}% ॥३४॥

\twolineshloka
{एकादश्या व्रतेनैव पुमान्पापैर्विमुच्यते}
{कर्तव्या सर्वदा राजन्विस्मर्तव्या न कर्हिचित्}% ॥३५॥

\twolineshloka
{शयनी बोधिनी मध्ये या कृष्णैकादशीभवेत्}
{सैवोपोष्या गृहस्थस्य नान्या कृष्णा कदाचन}% ॥३६॥

\twolineshloka
{शृणुयाच्चैव यो राजन्कथां पापहरां पराम्}
{अश्वमेधस्य यज्ञस्य फलं प्राप्नोति मानवः}% ॥३७॥

॥इति श्रीपाद्मे महापुराणे पञ्चपञ्चाशत्साहस्र्यां संहितायामुत्तरखण्डे उमापतिनारदसंवादान्तर्गतकृष्णयुधिष्ठिरसंवादे आषाढ-शुक्ल-शयनी-एकादशी-माहात्म्यं नाम पञ्चपञ्चाशत्तमोऽध्यायः॥५५॥


\hyperref[sec:ekadashi_mahatmyam_padma_puranam]{\closesub}
\clearpage

\sect{श्रावण-कृष्ण-कामिका-एकादशी-माहात्म्यम्}
\label{sec:padma-shravana-krishna-kamika}


\uvacha{युधिष्ठिर उवाच}

\twolineshloka
{श्रावणस्य सिते पक्षे किन्नामैकादशीभवेत्}
{तन्नः कथय गोविन्द वासुदेव नमोऽस्तु ते}% ॥१॥

\uvacha{श्रीकृष्ण उवाच}

\twolineshloka
{शृणु राजन्प्रवक्ष्यामि आख्यानं पापनाशनम्}
{यत्प्रोक्तं ब्रह्मणा पूर्वं पृच्छते नारदाय वै}% ॥२॥

\uvacha{नारद उवाच}

\twolineshloka
{भगवन्श्रोतुमिच्छामि त्वत्तोऽहं कमलासन}
{श्रावणस्यासिते पक्षे किं नामैकादशी भवेत्}% ॥३॥

\twolineshloka
{को देवः को विधिस्तस्याः किं पुण्यं कथय प्रभो}
{इति तस्य वचः श्रुत्वा ब्रह्मा वचनमब्रवीत्}% ॥४॥

\uvacha{ब्रह्मोवाच}

\twolineshloka
{शृणु नारद ते वच्मि लोकानां हितकाम्यया}
{श्रावणैकादशी कृष्णा कामिका नाम नामतः}% ॥५॥

\twolineshloka
{अस्याः श्रवणमात्रेण वाजपेयफलं लभेत्}
{अस्यां यजति देवेशं शङ्खचक्रगदाधरम्}% ॥६॥

\twolineshloka
{श्रीधराख्यं हरिं विष्णुं माधवं मधुसूदनम्}
{पूजयेद्ध्यायते यो वै तस्य पुण्यम्फलं शृणु}% ॥७॥

\twolineshloka
{न गङ्गायां न काश्यां च नैमिषे न च पुष्करे}
{तत्फलं समवाप्नोति यत्फलं कृष्णपूजनात्}% ॥८॥

\twolineshloka
{गोदावर्यां गुरौ सिंहे व्यतीपाते च दण्डके}
{यत्फलं समवाप्नोति तत्फलं कृष्णपूजनात्}% ॥९॥

\twolineshloka
{ससागरवनोपेतां यो ददाति वसुन्धराम्}
{कामिकाव्रतकारी च ह्युभौ समफलौ स्मृतौ}% ॥१०॥

\twolineshloka
{प्रसूयमानां यो धेनुं दद्यात्सोपस्करां नरः}
{तत्फलं समवाप्नोति कामिकाव्रतकारकः}% ॥११॥

\twolineshloka
{श्रावणे श्रीधरं देवं पूजयेद्यो नरोत्तमः}
{तेनैव पूजिता देवा गन्धर्वोरगपन्नगाः}% ॥१२॥

\twolineshloka
{तस्मात्सर्वप्रयत्नेन कामिकादिवसे हरिः}
{पूजनीयो यथाशक्ति मानुषैः पापभीरुभिः}% ॥१३॥

\twolineshloka
{ये संसारार्णवे मग्नाः पापपङ्कसमाकुले}
{तेषामुद्धरणार्थाय कामिकाव्रतमुत्तमम्}% ॥१४॥

\twolineshloka
{नातः परतरा काचित्पवित्रा पापहारिणी}
{एवं नारद जानीहि स्वयमाह परो हरिः}% ॥१५॥

\twolineshloka
{अध्यात्मविद्या निरतैर्यत्फलं प्राप्यते नरैः}
{ततो बहुतरं विद्धि कामिकाव्रतसेविनाम्}% ॥१६॥

\twolineshloka
{रात्रौ जागरणं कृत्वा कामिकाव्रतकृन्नरः}
{न पश्यति यमं रौद्रं नैव गच्छति दुर्गतिम्}% ॥१७॥

\twolineshloka
{न पश्यति कुयोनिं च कामिकाव्रतसेवनात्}
{कामिकाया व्रते चीर्णे कैवल्यं योगिनो गताः}% ॥१८॥

\twolineshloka
{तस्मात्सर्वप्रयत्नेन कर्तव्या नियतात्मभिः}
{तुलसीप्रभवैः पत्रैः यो नरः पूजयेद्धरिम्}% ॥१९॥

\twolineshloka
{न लिप्यते स पापेन पद्मपत्रमिवाम्भसा}
{सुवर्णभारमेकं तु रजतं च चतुर्गुणम्}% ॥२०॥

\twolineshloka
{तत्फलं समवाप्नोति तुलसीदलपूजनात्}
{रत्न मौक्तिक वैडूर्य प्रवालादिभिरर्चितः}% ॥२१॥

\twolineshloka
{न तुष्यति तथा विष्णुस्तुलसीदलतो यथा}
{तुलसीमञ्जरीभिश्च पूजितो येन केशवः}% ॥२२॥

\fourlineindentedshloka
{या दृष्टा निखिलाघसङ्घशमनी स्पृष्टा वपुः पावनी}
{रोगाणामभिवन्दितानि रसिनी सिक्तान्तकत्रासिनी}
{प्रत्यासत्तिविधायिनी भगवतः कृष्णस्य संरोपिता}
{न्यस्ता तच्चरणे विमुक्तिफलदा तस्यै तुलस्यै नमः}% ॥२३॥

\twolineshloka
{दीपं ददाति यो मर्त्यो दिवारात्रं हरेर्दिने}
{तस्य पुण्यस्य सङ्ख्यातुं चित्रगुप्तो न वेत्त्यलम्}% ॥२४॥

\twolineshloka
{कृष्णाग्रे दीपको यस्य ज्वलत्येकादशीदिने}
{पितरस्तस्य तृप्यन्ति अमृतेन दिवि स्थिताः}% ॥२५॥

\twolineshloka
{घृतेन दीपं प्रज्वाल्य तिलतैलेन वा पुनः}
{प्रयाति सूर्यलोकं च दीपकोटिशतार्चितः}% ॥२६॥

\twolineshloka
{अयं तवाग्रे कथितः कामिकामहिमा मया}
{अतो नरैः प्रकर्तव्या सर्वपातकहारिणी}% ॥२७॥

\twolineshloka
{ब्रह्महत्यापहरणी भ्रूणहत्याविनाशिनी}
{वैष्णवस्थानदात्री च महापुण्यफलप्रदा}% ॥२८॥

\twolineshloka
{श्रुत्वा माहात्म्यमेतस्या नरः श्रद्धासमन्वितः}
{विष्णुलोकमवाप्नोति सर्वपापैः प्रमुच्यते}% ॥२९॥

॥इति श्रीपाद्मे महापुराणे पञ्चपञ्चाशत्साहस्र्यां संहितायामुत्तरखण्डे उमापतिनारदसंवादान्तर्गतकृष्णयुधिष्ठिरसंवादे श्रावण-कृष्ण-कामिका-एकादशी-माहात्म्यं नाम षट्पञ्चाशत्तमोऽध्यायः॥५६॥


\hyperref[sec:ekadashi_mahatmyam_padma_puranam]{\closesub}
\clearpage

\sect{श्रावण-शुक्ल-पुत्रदा-एकादशी-माहात्म्यम्}
\label{sec:padma-shravana-shukla-putrada}

युधिष्ठिर उवाच

\twolineshloka
{श्रावणस्य सिते पक्षे किं नामैकादशीभवेत्}
{कथयस्व प्रसादेन ममाग्रे मधुसूदन}% ॥१॥

\uvacha{श्रीकृष्ण उवाच}

\twolineshloka
{शृणुष्वावहितो राजन्कथां पापहरां पराम्}
{यस्याः श्रवणमात्रेण वाजपेयफलं भवेत्}% ॥२॥

\twolineshloka
{द्वापरस्य युगस्यादौ पुरा माहिष्मती पुरे}
{राजा महीजिदाख्यातो राज्यं पालयति स्वकम्}% ॥३॥

\twolineshloka
{पुत्रहीनस्य तस्यैव न तद्राज्यं सुखप्रदम्}
{अपुत्रस्य सुखं नास्ति इहलोके परत्र च}% ॥४॥

\twolineshloka
{चिन्तयास्य सुतस्यैवं कालो बहुतरो गतः}
{न प्राप्तश्च सुतो राज्ञा सर्वसौख्यप्रदो नृणाम्}% ॥५॥

\twolineshloka
{दृष्ट्वात्मानं प्रवयसं राजा चिन्तापरोऽभवत्}
{तदागतः प्रजामध्ये इदं वचनमब्रवीत्}% ॥६॥

\twolineshloka
{इहजन्मनि भो लोका न मया पातकं कृतम्}
{अन्यायोपार्जितं वित्तं क्षिप्तं कोशे मया न हि}% ॥७॥

\threelineshloka
{ब्रह्मस्वं देवद्रविणं न गृहीतं मया क्वचित्}
{न्यासापहारो न कृतः परस्य बहुपापदः}
{पुत्रवत्पालितो लोको धर्मेण विजिता मही}% ॥८॥

\twolineshloka
{दुष्टेषु पातितो दण्डो बन्धुपुत्रोपमेष्वपि}
{शिष्टास्तु पूजिता नित्यं न द्वेष्याश्च मया जनाः}% ॥९॥

\twolineshloka
{इत्येवं ब्रुवतो मार्गं धर्मयुक्तं द्विजोत्तमाः}
{कस्मान्मम गृहे पुत्रो न जातस्तद्विमृश्यताम्}% ॥१०॥

\twolineshloka
{इति वाक्यं द्विजाः श्रुत्वा सप्रजाः सपुरोहिताः}
{मन्त्रयित्वा नृपहितं जग्मुस्ते गहनं वनम्}% ॥११॥

\twolineshloka
{इतस्ततश्च पश्यन्त आश्रमानृषिसेवितान्}
{नृपतेर्हितमिच्छन्तो ददृशुर्मुनिसत्तमम्}% ॥१२॥

\twolineshloka
{तप्यमानं तपो घोरं निरालम्बं निरामयम्}
{निराहारं जितात्मानं जितक्रोधं सनातनम्}% ॥१३॥

\twolineshloka
{लोमशं धर्मतत्वज्ञं सर्वशास्त्रविशारदम्}
{दीर्घायुषं महात्मानं सकेशं ब्रह्मसम्मितम्}% ॥१४॥

\twolineshloka
{कल्पेकल्पे गते तस्य एकं लोम विशीर्यते}
{अतो लोमशनामायं त्रिकालज्ञो महामुनिः}% ॥१५॥

\twolineshloka
{तं दृष्ट्वा हर्षिताः सर्वे आजग्मुस्तस्य सन्निधिम्}
{यथान्यायं यथार्हं ते नमश्चक्रुर्यथोदितम्}% ॥१६॥

\threelineshloka
{विनयावनताः सर्वे ऊचुस्ते च परस्परम्}
{अस्मद्भाग्यवशादेव प्राप्तोऽयं मुनिसत्तमः}
{तास्तथा स प्रजा वीक्ष्य उवाच ऋषिसत्तमः}% ॥१७॥

\uvacha{लोमश उवाच}

\twolineshloka
{किमर्थमिह सम्प्राप्तः कथयध्वं सकारणम्}
{दर्शनाद्धृष्टमनसः स्तुवन्तश्चैव मां किमु}% ॥१८॥

\twolineshloka
{असंशयं करिष्यामि भवतां यद्धितं भवेत्}
{परोपकृतये जन्म मादृशानां न संशयः}% ॥१९॥

\uvacha{जना ऊचुः}

\twolineshloka
{श्रूयतामभिधास्यामो वयं स्वागमकारणम्}
{संशयच्छेदनार्थाय तव सान्निध्यमागताः}% ॥२०॥

\twolineshloka
{पद्मयोनेः परतरस्त्वत्तः श्रेष्ठो न विद्यते}
{अतः कार्यवशात्प्राप्ताः समीपं भवतो वयम्}% ॥२१॥

\twolineshloka
{महीजिन्नाम राजासौ पुत्रहीनोऽस्ति साम्प्रतम्}
{वयं तस्य प्रजा ब्रह्मन्पुत्रवत्तेन पालिताः}% ॥२२॥

\twolineshloka
{तं पुत्ररहितं दृष्ट्वा तस्य दुःखेन दुःखिताः}
{तपः कर्तुमिहायाता मतिं कृत्वा तु नैष्ठिकीम्}% ॥२३॥

\twolineshloka
{तस्य भाग्येन दृष्टोऽसि ह्यस्माभिस्त्वं द्विजोत्तम}
{महतां दर्शनेनैव कार्यसिद्धिर्भवेन्नृणाम्}% ॥२४॥

\threelineshloka
{उपदेशं वद मुने राज्ञः पुत्रो यथा भवेत्}
{इति तेषां वचः श्रुत्वा मुहूर्तं ध्यानमास्थितः}
{प्रत्युवाच मुनिर्ज्ञात्वा तस्य जन्म पुरातनम्}% ॥२५॥

\uvacha{लोमश उवाच}

\twolineshloka
{पुरा जन्मनि वैश्योऽयं धनहीनो नृशंसकृत्}
{वाणिज्यकर्मनिरतो ग्रामाद्ग्रामान्तरं भ्रमन्}% ॥२६॥

\twolineshloka
{ज्येष्ठे मासि सिते पक्षे दशमी दिवसे तथा}
{मध्यगे द्युमणौ प्राप्ते ग्रामसीम्नि जलाशयम्}% ॥२७॥

\twolineshloka
{कूपिकां सजलां दृष्ट्वा जलपाने मनोदधे}
{सद्यस्ततः सवत्सा च धेनुस्तत्र समागता}% ॥२८॥

\twolineshloka
{तृष्णातुरा निदाघार्ता तस्यामम्बुपपौ तु सा}
{पिबन्तीं वारयित्वा तामसौ तोयं पपौ स्वयम्}% ॥२९॥

\twolineshloka
{कर्मणा तेन पापेन पुत्रहीनो नृपो भवेत्}
{कस्यापि जन्मनः पुण्यात्प्राप्तं राज्यमकण्टकम्}% ॥३०॥

\uvacha{लोका ऊचुः}

\threelineshloka
{पुण्यात्पापं क्षयं याति पुराणे श्रूयते मुने}
{पुण्योपदेशं कथय येन पापक्षयो भवेत्}
{यथा भवत्प्रसादेन पुत्रो भवति भूपतेः}% ॥३१॥

\uvacha{लोमश उवाच}

\twolineshloka
{श्रावणे शुक्लपक्षे तु पुत्रदा नाम विश्रुता}
{एकादशी वाञ्च्छितदा कुरुध्वं तद्व्रतं जनाः}% ॥३२॥

\twolineshloka
{इति श्रुत्वा नमस्कृत्य मुनिमेत्य पुरं व्रतम्}
{यथाविधियथान्यायं कृतं तैर्जागरान्वितम्}% ॥३३॥

\twolineshloka
{तस्य पुण्यं सुविमलं दत्तं नृपतये जनैः}
{दत्ते पुण्येऽथ सा राज्ञी गर्भमाधत्त शोभनम्}% ॥३४॥

\twolineshloka
{प्राप्तो प्रसवकाले सा सुषुवे पुत्रमूर्जितम्}
{श्रावणस्य सिते पक्षे कर्कटस्थे दिवाकरे}% ॥३५॥

\twolineshloka
{द्वादश्यां वासुदेवाय पवित्रारोपणं स्मृतम्}
{हेम रौप्य ताम्र क्षौमैः सूत्रैः कौशेयपद्मजैः}% ॥३६॥

\twolineshloka
{कुशैः काशैश्च कार्पासैर्ब्राह्मण्या कर्तितैः शुभैः}
{स्नात्वा त्रिगुणितं सूत्रं त्रिगुणीकृत्य शोधयेत्}% ॥३७॥

\twolineshloka
{गोदोहान्तरिते काले पूर्वेद्युरधिवासनम्}
{ब्राह्मणेभ्यो नमस्कृत्य गुरुपादौ प्रणम्य च}% ॥३८॥

\twolineshloka
{गीतमङ्गलनिर्घोषः कुर्याज्जागरणं ततः}
{ब्राह्मणाः क्षत्रिया वैश्या भिल्लाः शूद्रास्तथैव च}% ॥३९॥

\twolineshloka
{स्वधर्मावस्थिताः सर्वे भक्त्या कुर्युः पवित्रकम्}
{ततः पवित्रं गुरवे दद्याद्वै विधिपूर्वकम्}% ॥४०॥

\twolineshloka
{ब्राह्मणान्वैष्णवांश्चैव गन्धपुष्पादिनार्चयेत्}
{अतो देवेति मन्त्रेण द्विजो विष्णौ निवेदयेत्}% ॥४१॥

\twolineshloka
{शूद्रस्तु मूलमन्त्रेण यथा विष्णौ तथा शिवे}
{वर्षेवर्षे प्रकर्तव्यं पवित्रारोपणं नरैः}% ॥४२॥

\twolineshloka
{भुक्तिं मुक्तिं च इच्छद्भिः संसारे शोकसागरे}
{न करोति विधानेन पवित्रारोपणं तु यः}% ॥४३॥

\threelineshloka
{तस्य सांवत्सरी पूजा निष्फला वैष्णवस्य तु}
{श्रुत्वा माहात्म्यमेतस्या नरः पापात्प्रमुच्यते}
{इह पुत्रसुखं प्राप्य परत्र स्वर्गतिं भवेत्}% ॥४४॥

॥इति श्रीपाद्मे महापुराणे पञ्चपञ्चाशत्साहस्र्यां संहितायामुत्तरखण्डे उमापतिनारदसंवादान्तर्गतकृष्णयुधिष्ठिरसंवादे श्रावण-शुक्ल-पुत्रदा-एकादशी-माहात्म्यं नाम सप्तपञ्चाशत्तमोऽध्यायः॥५७॥


\hyperref[sec:ekadashi_mahatmyam_padma_puranam]{\closesub}
\clearpage

\sect{भाद्रपद-कृष्णाजा-एकादशी-माहात्म्यम्}
\label{sec:padma-bhadrapada-krishnaja}


\uvacha{युधिष्ठिर उवाच}

\twolineshloka
{भाद्रस्य कृष्णपक्षे तु किन्नामैकादशीभवेत्}
{एतदिच्छाम्यहं श्रोतुं कथयस्व जनार्दन}% ॥१॥

\uvacha{श्रीकृष्ण उवाच}

\twolineshloka
{शृणुष्वैकमना राजन्कथयिष्यामि विस्तरात्}
{अजेति नामतः प्रोक्ता सर्वपापप्रणाशिनी}% ॥२॥

\twolineshloka
{पूजयित्वा हृषीकेशं व्रतमस्यां करोति यः}
{पापानि तस्य नश्यन्ति व्रतस्य श्रवणादपि}% ॥३॥

\twolineshloka
{नातः परतरा राजन्लोकद्वयहिताय वै}
{सत्यमुक्तं मया ह्येतन्नासत्यं मम भाषितम्}% ॥४॥

\twolineshloka
{हरिश्चन्द्र इति ख्यातो बभूव नृपतिः पुरा}
{चक्रवर्ती सत्यसन्धः समस्ताया भुवः पतिः}% ॥५॥

\twolineshloka
{कस्यापि कर्मणः प्राप्तौ राज्यभ्रष्टो बभूव सः}
{विक्रीतौ वनितापुत्रौ स चकारात्मविक्रयम्}% ॥६॥

\twolineshloka
{पुल्कसस्य च दासत्वं गतो राजा स पुण्यकृत्}
{सत्यमालम्ब्य राजेन्द्र मृतचैलापहारकः}% ॥७॥

\twolineshloka
{सोऽभवन्नृपतिश्रेष्ठो न सत्याच्चलितस्तथा}
{एवं च तस्य नृपतेर्बहवो वत्सरा गताः}% ॥८॥

\twolineshloka
{ततश्चिन्तापरो राजा स बभूवातिदुःखितः}
{किं करोमि क्व गच्छामि निष्कृतिर्मे कथं भवेत्}% ॥९॥

\twolineshloka
{इति चिन्तयतस्तस्य मग्नस्य वृजिनार्णवे}
{आजगाम मुनिः कश्चिज्ज्ञात्वा राजानमातुरम्}% ॥१०॥

\twolineshloka
{परोपकारणार्थाय निर्मिता ब्रह्मणा द्विजाः}
{स तं दृष्ट्वा द्विजवरं ननाम नृपसत्तमः}% ॥११॥

\twolineshloka
{कृताञ्जलिपुटो भूत्वा गौतमस्याग्रतः स्थितः}
{कथयामास वृत्तान्तमात्मनो दुःखसंयुतम्}% ॥१२॥

\twolineshloka
{श्रुत्वा नृपतिवाक्यानि गौतमो विस्मयान्वितः}
{उपदेशं नृपतये व्रतस्यास्य ददौ मुनिः}% ॥१३॥

\twolineshloka
{मासि भाद्रपदे राजन्कृष्णपक्षेति शोभना}
{एकादशी समायाता  अजा नामेति पुण्यदा}% ॥१४॥

\twolineshloka
{अस्याः कुरु व्रतं राजन्पापस्यान्तो भविष्यति}
{तव भाग्यवशादेषा सप्तमेऽह्नि समागता}% ॥१५॥

\twolineshloka
{उपवासपरो भूत्वा रात्रौ जागरणं कुरु}
{एवमस्या व्रते चीर्णे तव पापक्षयो ध्रुवम्}% ॥१६॥

\twolineshloka
{तव पुण्यप्रभावेण चागतोऽहं नृपोत्तम}
{इत्येवं कथयित्वा च मुनिरन्तरधीयत}% ॥१७॥

\twolineshloka
{मुनिवाक्यं नृपः श्रुत्वा चकार व्रतमुत्तमम्}
{कृते तस्मिन्व्रते राज्ञः पापस्यान्तोऽभवत्क्षणात्}% ॥१८॥

\twolineshloka
{श्रूयतां राजशार्दूल प्रभावोऽस्य व्रतस्य च}
{यद्दुःखम्बहुभिर्वर्षैर्भोक्तव्यन्तत्क्षयोभवेत्}% ॥१९॥

\twolineshloka
{निस्तीर्णदुःखो राजासीद्व्रतस्यास्य प्रभावतः}
{पत्न्या सह समायोगं पुत्रजीवनमाप सः}% ॥२०॥

\twolineshloka
{दिवि दुन्दुभयो नेदुः पुष्पवर्षमभूद्दिवः}
{एकादश्याः प्रभावेन प्राप्यराज्यमकण्टकम्}% ॥२१॥

\twolineshloka
{स्वर्गं लेभे हरिश्चन्द्रः सपुरः सपरिच्छदः}
{ईदृग्विधं व्रतंराजन्ये कुर्वन्ति च मानवाः}% ॥२२॥

\twolineshloka
{सर्वपापविनिर्मुक्तास्त्रिदिवं यान्ति ते नृप}
{पठनाच्छ्रवणाद्वाऽपि अश्वमेधफलं लभेत्}% ॥२३॥

॥इति श्रीपाद्मे महापुराणे पञ्चपञ्चाशत्साहस्र्यां संहितायामुत्तरखण्डे उमापतिनारदसंवादान्तर्गतकृष्णयुधिष्ठिरसंवादे भाद्रपद-कृष्णाजा-एकादशी-माहात्म्यं नाम अष्टपञ्चाशत्तमोऽध्यायः॥५८॥


\hyperref[sec:ekadashi_mahatmyam_padma_puranam]{\closesub}
\clearpage

\sect{भाद्रपद-शुक्ल-पद्मा-एकादशी-माहात्म्यम्}
\label{sec:padma-bhadrapada-shukla-padma}


\uvacha{युधिष्ठिर उवाच}

\twolineshloka
{नभस्यस्य सिते पक्षे किन्नामैकादशी भवेत्}
{को देवः को विधिस्तस्य एतदाख्याहि केशव}% ॥१॥

\uvacha{श्रीकृष्ण उवाच}

\twolineshloka
{कथयामि महीपाल कथामाश्चर्यकारिणीम्}
{कथयामास यां ब्रह्मा नारदाय महात्मने}% ॥२॥

\uvacha{नारद उवाच}

\threelineshloka
{कथयस्व प्रसादेन चतुर्मुख नमोऽस्तु ते}
{नभस्य शुक्लपक्षे तु किं नामैकादशी भवेत्}
{एतदिच्छाम्यहं श्रोतुं विष्णोराराधनाय वै}% ॥३॥

\uvacha{ब्रह्मोवाच}

\twolineshloka
{वैष्णवोऽसि मुनिश्रेष्ठ साधुपृष्टं किल त्वया}
{नातः परतरा लोके पवित्रा हरिवासरात्}% ॥४॥

\twolineshloka
{पद्मा नामेति विख्याता नभस्यैकादशी सिता}
{हृषीकेशः पूज्यतेऽस्यां कर्तव्यं व्रतमुत्तमम्}% ॥५॥

\twolineshloka
{कथयामि तवाग्रेऽहं कथां पौराणिकीं शुभाम्}
{यस्याः श्रवणमात्रेण महापापं प्रणश्यति}% ॥६॥

\twolineshloka
{मान्धाता नाम राजर्षिर्विवस्वद्वंशसम्भवः}
{बभूव चक्रवर्ती स सत्यसन्धः प्रतापवान्}% ॥७॥

\twolineshloka
{धर्मतः पालयामास प्रजाः पुत्रानिवौरसान्}
{न तस्य राज्ये दुर्भिक्षं नाधयो व्याधयस्तथा}% ॥८॥

\twolineshloka
{निरातङ्काः प्रजास्तस्य धनधान्यसमेधिताः}
{न्यायेनोपार्जितं वित्तं तस्य कोशे महीपते}% ॥९॥

\twolineshloka
{स्वस्वधर्मे प्रवर्तन्ते सर्वे वर्णाश्रमास्तथा}
{कामधेनुसमाभूमिस्तस्य राज्ये महीपतेः}% ॥१०॥

\twolineshloka
{तस्यैवं कुर्वतो राज्यं बहुवर्षगणा गताः}
{अथैकस्मिंश्च सम्प्राप्ते विपाकः कर्मणः खलु}% ॥११॥

\twolineshloka
{वर्षत्रयं तद्विषये न ववर्ष बलाहकः}
{तेन भग्नाः प्रजास्तस्य बभूवुः क्षुधयार्दिताः}% ॥१२॥

\threelineshloka
{स्वाहा स्वधा वषट्कार वेदाध्ययनवर्जिताः}
{बभूव विषयस्तस्या भाग्येन दैवपीडितः}
{अथ प्रजाः समागम्य राजानमिदमब्रुवन्}% ॥१३॥

\uvacha{प्रजा ऊचुः}

\twolineshloka
{श्रोतव्यं नृपशार्दूल प्रजानां वचनं त्वया}
{आपो नारा इति प्रोक्ताः पुराणेषु मनीषिभिः}% ॥१४॥

\twolineshloka
{अयनं भगवतस्तस्मान्नारायण इति स्मृतः}
{पर्जन्यरूपो भगवान्विष्णुः सर्वगतः स्थितः}% ॥१५॥

\threelineshloka
{स एवं कुरुते वृष्टिं वृष्टेरन्नं ततः प्रजाः}
{तदभावे नृपश्रेष्ठ क्षयं गच्छन्ति वै प्रजाः}
{तथा कुरु नृपश्रेष्ठ योगः क्षेमो यथा भवेत्}% ॥१६॥

\uvacha{राजोवाच}

\twolineshloka
{सत्यमुक्तं भवद्भिश्च न मिथ्याभिहितं क्वचित्}
{अन्नं ब्रह्म यतः प्रोक्तमन्ने सर्वं प्रतिष्ठितम्}% ॥१७॥

\twolineshloka
{अन्नाद्भवन्ति भूतानि जगदन्नेन वर्तते}
{इत्येवं श्रूयते लोके पुराणे बहुविस्तरे}% ॥१८॥

\twolineshloka
{नृपाणामपचारेण प्रजानां पीडनं भवेत्}
{नाहं पश्याम्यात्मकृतमेवं बुद्ध्या विचारयन्}% ॥१९॥

\twolineshloka
{तथापि प्रयतिष्यामि प्रजानां हितकाम्यया}
{इति कृत्वा मतिं राजा परिमेयपरिच्छदः}% ॥२०॥

\twolineshloka
{नमस्कृत्य विधातारं जगाम गहनं वनम्}
{चचार मुनिमुख्यांश्च आश्रमान्तापसैः श्रितान्}% ॥२१॥

\twolineshloka
{ददर्शाथ ब्रह्मसुतमृषिमाङ्गिरसं नृपः}
{तेजसा द्योतितदिशं द्वितीयमिव पद्मजम्}% ॥२२॥

\twolineshloka
{तं दृष्ट्वा हर्षितो राजा अवतीर्य स्ववाहनात्}
{नमश्चक्रेऽस्य चरणौ कृताञ्जलिपुटो वशी}% ॥२३॥

\twolineshloka
{मुनिस्तमभिनन्द्याथ स्वस्तिवाचनपूर्वकम्}
{पप्रच्छ कुशलं राज्ये सप्तस्वङ्गेषु भूपतेः}% ॥२४॥

\twolineshloka
{निवेदयित्वा कुशलं पप्रच्छानामयं नृपः}
{दत्तासनो गृहीतार्घ्य उपविष्टोऽस्य सन्निधौ}% ॥२५॥

\onelineshloka
{प्रत्युवाच मुनिं राजा पृष्टो ह्यागमकारणम्}% ॥२६॥

\uvacha{राजोवाच}

\twolineshloka
{भगवन्धर्मविधिना मम पालयतो महीम्}
{अनावृष्टिश्च संवृत्ता नाहं वेद्म्यत्र कारणम्}% ॥२७॥

\twolineshloka
{संशयच्छेदनायात्र आगतोऽहं तवान्तिके}
{योगक्षेमविधानेन प्रजानां कुरु निर्वृतिम्}% ॥२८॥

\uvacha{ऋषिरुवाच}

\twolineshloka
{एतत्कृतयुगं राजन्युगानामुत्तमं मतम्}
{अत्र ब्रह्मपरा लोका धर्मश्चात्र चतुष्पदः}% ॥२९॥

\twolineshloka
{अस्मिन्युगे तपोयुक्ता ब्राह्मणा नेतरा जनाः}
{विषये तव राजेन्द्र वृषलोऽयं तपस्यति}% ॥३०॥

\twolineshloka
{एतस्मात्कारणाच्चैव न वर्षति बलाहकः}
{कुरु तस्य वधे यत्नं येन दोषः प्रशाम्यति}% ॥३१॥

\uvacha{राजोवाच}

\twolineshloka
{नाहमेनं वधिष्यामि तपस्यन्तमनागसम्}
{धर्मोपदेशं कथय उपसर्गविनाशनम्}% ॥३२॥

\uvacha{ऋषिरुवाच}

\twolineshloka
{यद्येवं तर्हि नृपते कुरुष्वैकादशीव्रतम्}
{नभस्यस्य सिते पक्षे पद्मा नामेति विश्रुता}% ॥३३॥

\twolineshloka
{तस्या व्रतप्रभावेन सुवृष्टिर्भविता ध्रुवम्}
{सर्वसिद्धिप्रदा ह्येषा सर्वोपद्रवनाशिनी}% ॥३४॥

\twolineshloka
{अस्या व्रतं कुरु नृप सप्रजः सपरिच्छदः}
{इति वाक्यमृषेः श्रुत्वा राजा स्वगृहमागतः}% ॥३५॥

\twolineshloka
{भाद्रमासे सिते पक्षे पद्माव्रतमथाकरोत्}
{प्रजाभिः सह सर्वाभिश्चातुर्वर्ण्यसमन्वितः}% ॥३६॥

\twolineshloka
{एवं व्रते कृते राजन्प्रववर्ष बलाहकः}
{जलेन प्लाविता भूमिरभवत्सस्यशालिनी}% ॥३७॥

\twolineshloka
{ऋषीश्वरप्रभावेन लोकाः सौख्यं प्रपेदिरे}
{एतस्मात्कारणादेवं कर्तव्यं व्रतमुत्तमम्}% ॥३८॥

\twolineshloka
{दध्योदनयुतं तस्यां जलपूर्णं घटं द्विजे}
{वस्त्रसंवेष्टितं दत्त्वा छत्रोपानहमेव च}% ॥३९॥

\twolineshloka
{नमो नमस्ते गोविन्द बुधश्रवणसंज्ञक}
{अघौघसङ्क्षयं कृत्वा सर्वसौख्यप्रदो भव}% ॥४०॥

\twolineshloka
{भुक्तिमुक्तिप्रदश्चैव लोकानां सुखदायकः}
{पठनाच्छ्रवणाद्राजन्सर्वपापैः प्रमुच्यते}% ॥४१॥

॥इति श्रीपाद्मे महापुराणे पञ्चपञ्चाशत्साहस्र्यां संहितायामुत्तरखण्डे उमापतिनारदसंवादान्तर्गतकृष्णयुधिष्ठिरसंवादे भाद्रपद-शुक्ल-पद्मा-एकादशी-माहात्म्यं नाम एकोनषष्टितमोऽध्यायः॥५९॥


\hyperref[sec:ekadashi_mahatmyam_padma_puranam]{\closesub}
\clearpage

\sect{आश्विन-कृष्णेन्दिरा-एकादशी-माहात्म्यम्}
\label{sec:padma-ashvina-krishnendira}


\uvacha{युधिष्ठिर उवाच}

\twolineshloka
{कथयस्व प्रसादेन ममाग्रे मधुसूदन}
{इषस्य कृष्णपक्षे तु किन्नामैकादशीभवेत्}% ॥१॥

\uvacha{श्रीकृष्ण उवाच}

\twolineshloka
{आश्विने कृष्णपक्षे तु इन्दिरा नाम नामतः}
{तस्या व्रतप्रभावेन महापापं प्रणश्यति}% ॥२॥

\twolineshloka
{अधोयोनि गतानां च पितॄणां गतिदायिनी}
{शृणुष्वावहितो राजन्कथां पापहरां पराम्}% ॥३॥

\twolineshloka
{यस्याः श्रवणमात्रेण वाजपेयफलं लभेत्}
{पुरा कृतयुगे राजन्बभूव नृपनन्दनः}% ॥४॥

\twolineshloka
{इन्द्रसेन इति ख्यातः पुरा माहिष्मतीपतिः}
{स राजा पालयामास धर्मेण यशसान्वितः}% ॥५॥

\twolineshloka
{पुत्रपौत्रसमायुक्तो धनधान्यसमन्वितः}
{माहिष्मत्यधिपो राजा विष्णुभक्तिपरायणः}% ॥६॥

\twolineshloka
{जपन्गोविन्दनामानि मुक्तिदानि नराधिपः}
{कालं नयति विधिवदध्यात्मध्यानचिन्तकः}% ॥७॥

\twolineshloka
{एकस्मिन्दिवसे राज्ञि सुखासीने सदो गते}
{अवतीर्यागमत्तत्र ह्यम्बरान्नारदो मुनिः}% ॥८॥

\twolineshloka
{तमागतमभिप्रेक्ष्य प्रत्युत्थाय कृताञ्जलि}
{पूजयित्वाऽथ विधिना चासने सन्न्यवेशयत्}% ॥९॥
\onelineshloka*
{सुखोपविष्टं स मुनिं प्रत्युवाच नृपोत्तमः}

\uvacha{राजोवाच}
\onelineshloka
{त्वत्प्रसादान्मुनिश्रेष्ठ सर्वं च कुशलं मम}% ॥१०॥

\twolineshloka
{अद्य क्रतुक्रियाः सर्वाः सफलास्तव दर्शनात्}
{प्रसादं कुरु देवर्षे ब्रूह्यागमनकारणम्}% ॥११॥

\uvacha{नारद उवाच}

\twolineshloka
{श्रूयतां नृपशार्दूल मद्वचो विस्मयप्रदम्}
{ब्रह्मलोकादहं प्राप्तो यमलोकं नृपोत्तम}% ॥१२॥

\twolineshloka
{शमनेनार्चितो भक्त्या उपविष्टो वरासने}
{धर्मशीलः सत्यवांस्तु भास्करिं समुपासते}% ॥१३॥

\twolineshloka
{बहुपुण्यप्रकर्ता च व्रतवैकल्यदोषतः}
{सभायां श्राद्धदेवस्य मया दृष्टः पिता तव}% ॥१४॥

\twolineshloka
{कथितस्तेन सन्देशस्तं निबोध जनेश्वर}
{इन्द्रसेन इति ख्यातो राजा माहिष्मतीपतिः}% ॥१५॥

\twolineshloka
{तस्याग्रे कथयब्रह्मन्स्थितं मां यमसन्निधौ}
{केनापि चान्तरायेण पूर्वजन्मोद्भवेन च}% ॥१६॥

\twolineshloka
{स्वर्गं प्रेषय मां पुत्र इन्दिरा पुण्य दानतः}
{इत्युक्तोऽहं समायातः समीपं तव पार्थिव}% ॥१७॥

\twolineshloka
{पितुः स्वर्गकृते राजन्निन्दिरा व्रतमाचर}
{तेन व्रतप्रभावेन स्वर्गं यास्यति ते पिता}% ॥१८॥

\uvacha{राजोवाच}

\twolineshloka
{कथयस्व प्रसादेन भगवन्निन्दिरा व्रतम्}
{विधिना केन कर्तव्यं कस्मिन्पक्षे तिथौ तथा}% ॥१९॥

\uvacha{नारदउवाच}

\twolineshloka
{शृणु राजेन्द्र ते वच्मि व्रतस्यास्य विधिं शुभम्}
{आश्विनस्यासिते पक्षे दशमी दिवसे शुभे}% ॥२०॥

\twolineshloka
{प्रातःस्नानं प्रकुर्वीत श्रद्धायुक्तेन चेतसा}
{ततो मध्याह्नसमये स्नानं कृत्वा समाहितः}% ॥२१॥

\threelineshloka
{पितॄणां प्रीतये श्राद्धं कुर्याच्छ्रद्धा समन्वितः}
{एकभक्तं ततः कृत्वा रात्रौ भूमौ शयीत च}
{प्रभाते विमले जाते प्राप्ते चैकादशी दिने}% ॥२२॥

\twolineshloka
{मुख प्रक्षालनं कुर्याद्दन्तधावन वर्जितम्}
{उपवासस्य नियमं गृह्णीयाद्भक्तिभावतः}% ॥२३॥

\twolineshloka
{अद्यस्थित्वा निराहारः सर्वभोग विवर्जितः}
{श्वो भोक्ष्ये पुण्डरीकाक्ष शरणं मे भवाच्युत}% ॥२४॥

\twolineshloka
{इत्येवं नियमं कृत्वा मध्याह्न समये तथा}
{शालग्राम शिलाग्रे तु स्नानं कुर्याद्यथाविधि}% ॥२५॥

\twolineshloka
{पूजयित्वा हृषीकेशं धूप गन्धादिभिस्तथा}
{रात्रौ जागरणं कुर्यात्केशवस्य समीपतः}% ॥२६॥

\twolineshloka
{ततः प्रभातसमये प्राप्ते वै द्वादशी दिने}
{अर्चयित्वा हरिं भक्त्या श्राद्धं कुर्याद्यथाविधि}% ॥२७॥

\twolineshloka
{पितॄणां प्रीतये श्राद्धं कुर्याच्छ्रद्धा समन्वितः}
{गोधूम चूर्णैर्यछ्राद्धं कृतं मेध्यकृतं भवेत्}% ॥२८॥

\twolineshloka
{यवैर्व्रीहितिलैर्माषैर्गोधूमैश्चणकैस्तथा}
{ब्राह्मणान्भोजयेद्राजन् दक्षिणाभिः प्रपूजितान्}% ॥२९॥

\twolineshloka
{बन्धु दौहित्र पुत्राद्यैः स्वयं भुञ्जीत वाग्यतः}
{अनेन विधिना राजन्कुरुव्रतमतन्द्रितः}% ॥३०॥

\twolineshloka
{विष्णुलोकं प्रयास्यन्ति पितरस्तव भूपते}
{इत्युक्त्वा नृपतिं राजन्मुनिरन्तरधीयत}% ॥३१॥

\twolineshloka
{यथोक्त विधिना राजा चकार व्रतमुत्तमम्}
{अन्तःपुरेण सहितः पुत्रभृत्य समन्वितः}% ॥३२॥

\twolineshloka
{कृते व्रते तु कौन्तेय पुष्पवृष्टिरभूद्दिवः}
{तत्पिता गरुडारूढो जगाम हरिमन्दिरम्}% ॥३३॥

\twolineshloka
{इन्द्रसेनोऽपिराजर्षिः कृत्वा राज्यमकण्टकम्}
{राज्ये निवेश्य तनयं जगाम त्रिदिवं स्वयम्}% ॥३४॥

\twolineshloka
{इन्दिरा व्रत माहात्म्यं तवाग्रे कथितं मया}
{पठनाच्छ्रवणाद्राजन् सर्वपापैः प्रमुच्यते}% ॥३५॥

\onelineshloka
{भुक्त्वेह निखिलान्भोगान् विष्णुलोके वसेच्चिरम्}% ॥३६॥

॥इति श्रीपाद्मे महापुराणे पञ्चपञ्चाशत्साहस्र्यां संहितायामुत्तरखण्डे उमापतिनारदसंवादान्तर्गतकृष्णयुधिष्ठिरसंवादे आश्विन-कृष्णेन्दिरा-एकादशी-माहात्म्यं नाम षष्टितमोऽध्यायः॥६०॥



\hyperref[sec:ekadashi_mahatmyam_padma_puranam]{\closesub}
\clearpage

\sect{आश्विन-शुक्ल-पाशाङ्कुशा-एकादशी-माहात्म्यम्}
\label{sec:padma-ashvina-shukla-pashankusha}


\uvacha{युधिष्ठिर उवाच}

\twolineshloka
{कथयस्व प्रसादेन ममाग्रे मधुसूदन}
{इषस्य शुक्लपक्षे तु किं नामैकादशीभवेत्}% ॥१॥

\uvacha{श्रीकृष्णउवाच}

\twolineshloka
{शृणुराजेन्द्र वक्ष्यामि माहात्म्यं पाप नाशनम्}
{शुक्लपक्षे चाश्विनस्य भवेदेकादशी तु या}% ॥२॥

\twolineshloka
{पापाङ्कुशेति विख्याता सर्वपापहरा परा}
{पद्मनाभाभिधानं मां पूजयेत्तत्र मानवः}% ॥३॥

\twolineshloka
{सर्वाभीष्ट फलं प्राप्तौ स्वर्गमोक्षप्रदं नृणाम्}
{तपस्तप्त्वा पुनस्तीव्रं चिरं सुनियतेन्द्रियः}% ॥४॥

\twolineshloka
{यत्फलं समवाप्नोति तन्न त्वा गरुडध्वजम्}
{कृत्वाऽपि बहुशःपापं नरो मोह समन्वितः}% ॥५॥

\twolineshloka
{न याति नरकं नत्वा सर्वपापहरं हरिम्}
{पृथिव्यां यानि तीर्थानि पुण्यान्यायतनानि च}% ॥६॥

\twolineshloka
{तानि सर्वाण्यवाप्नोति विष्णोर्नामानुकीर्तनात्}
{देवं शार्ङ्गधरं विष्णुं ये प्रपन्ना जनार्दनम्}% ॥७॥

\threelineshloka
{न तेषां यमलोकस्य यातना जायते क्वचित्}
{उपोष्यैकादशीमेकां प्रसङ्गेनापि मानवः}
{न याति यातनां यामीं पापं कृत्वाऽपि दारुणम्}% ॥८॥

\twolineshloka
{वैष्णवः पुरुषो भूत्वा शिवनिन्दां करोति यः}
{न विन्देद्वैष्णवं लोकं स याति नरकं ध्रुवम्}% ॥९॥

\twolineshloka
{शैवः पाशुपतो भूत्वा विष्णुनिन्दां करोति चेत्}
{रौरवे पच्यते घोरे यावदिन्द्राश्चतुर्दश}% ॥१०॥

\twolineshloka
{नेदृशं पावनं किञ्चित्त्रिषु लोकेषु विद्यते}
{यादृशं पद्मनाभस्य व्रतं पातकनाशनम्}% ॥११॥

\twolineshloka
{तावत्पापानि देहेऽस्मिन्तिष्ठन्ति मनुजाधिपम्}
{यावन्नोपवसेज्जन्तुः पद्मनाभदिनं शुभम्}% ॥१२॥

\twolineshloka
{अश्वमेधसहस्राणि राजसूयशतानि च}
{एकादश्युपवासस्य कलां नार्हन्ति षोडशीम्}% ॥१३॥

\twolineshloka
{एकादशीसमं किञ्चिद्व्रतं लोके न विद्यते}
{व्याजेनापि कृता यैश्च न ते यान्ति हि भास्करिम्}% ॥१४॥

\twolineshloka
{स्वर्गमोक्षप्रदा ह्येषा शरीरारोग्यदायिनी}
{कलत्रसुतदा ह्येषा धनमित्रप्रदायिनी}% ॥१५॥

\twolineshloka
{न गङ्गा न गया राजन्न च काशी च पुष्करम्}
{न चापि कौरवं क्षेत्रं पुण्यं भूप हरेर्दिनात्}% ॥१६॥

\twolineshloka
{रात्रौ जागरणं कृत्वा समुपोष्य हरेर्दिनम्}
{अनायासेन भूपाल प्राप्यते वैष्णवं पदम्}% ॥१७॥

\twolineshloka
{दशैव मातृके पक्षे राजेन्द्र दश पैतृके}
{प्रियाया दशपक्षे तु पुरुषानुद्धरेन्नरः}% ॥१८॥

\twolineshloka
{चतुर्भुजा दिव्यरूपा नागारिकृतकेतनाः}
{स्रग्विणः पीतवस्त्राश्च प्रयान्ति हरिमन्दिरम्}% ॥१९॥

\twolineshloka
{बालत्वे यौवनत्वे च वृद्धत्वे च नृपोत्तम}
{उपोष्यैकादशीं नूनं नैव प्राप्नोति दुर्गतिम्}% ॥२०॥

\twolineshloka
{पापाङ्कुशामुपोष्यैव आश्विनस्य सिते नरः}
{सर्वपापविनिर्मुक्तो हरिलोकं स गच्छति}% ॥२१॥

\twolineshloka
{दत्त्वा हेमतिलान्भूमिं गामन्नमुदकं तथा}
{उपानच्छत्रवस्त्रादीन्न पश्यति यमं नरः}% ॥२२॥

\twolineshloka
{यस्य पुण्यविहीनानि दिनान्यायान्ति यान्ति च}
{स लोहकारभस्त्रेव श्वसन्नपि न जीवति}% ॥२३॥

\twolineshloka
{अवन्ध्यं दिवसं कुर्याद्दरिद्रोऽपि नृपोत्तम}
{सदाचरन्यथाशक्ति स्नानदानादिकाः क्रियाः}% ॥२४॥

\twolineshloka
{होमस्नानजपध्यानसत्रादिपुण्यकर्मणाम्}
{कर्तारो नैव पश्यन्ति घोरां तां यमयातनाम्}% ॥२५॥

\twolineshloka
{दीर्घायुषो धनाढ्याश्च कुलीना रोगवर्जिताः}
{दृश्यन्ते मानवा लोके पुण्यकर्तार ईदृशाः}% ॥२६॥

\twolineshloka
{किमत्र बहुनोक्तेन यान्त्यधर्मेण दुर्गतिम्}
{आरोहन्ति दिवं धर्मैर्नात्रकार्या विचारणा}% ॥२७॥

\twolineshloka
{इति ते कथितं राजन्यत्पृष्टोऽहं त्वयानघ}
{पापाङ्कुशाया माहात्म्यं किमन्यच्छ्रोतुमिच्छसि}% ॥२८॥

॥इति श्रीपाद्मे महापुराणे पञ्चपञ्चाशत्साहस्र्यां संहितायामुत्तरखण्डे उमापतिनारदसंवादान्तर्गतकृष्णयुधिष्ठिरसंवादे आश्विन-शुक्ल-पाशाङ्कुशा-एकादशी-माहात्म्यं नाम एकषष्टितमोऽध्यायः॥६१॥


\hyperref[sec:ekadashi_mahatmyam_padma_puranam]{\closesub}
\clearpage

\sect{कार्त्तिक-कृष्ण-रमा-एकादशी-माहात्म्यम्}
\label{sec:padma-karttika-krishna-rama}


\uvacha{युधिष्ठिर उवाच}

\twolineshloka
{कथयस्व प्रसादेन मयि स्नेहाज्जनार्दन}
{कार्त्तिकस्यासिते पक्षे किन्नामैकादशी भवेत्}% ॥१॥

\uvacha{श्रीकृष्ण उवाच}

\twolineshloka
{श्रूयतां राजशार्दूल कथयामि तवाग्रतः}
{कार्त्तिके कृष्णपक्षे तु रमा नाम सुशोभना}% ॥२॥

\twolineshloka
{एकादशी समाख्याता महापापहरा परा}
{अस्याः प्रसङ्गतो राजन्माहात्म्यं प्रवदामि ते}% ॥३॥

\twolineshloka
{मुचुकुन्द इति ख्यातो बभूव नृपतिः पुरा}
{देवेन्द्रेण समं तस्य मित्रत्वमभवन्नृप}% ॥४॥

\twolineshloka
{यमेन वरुणेनैव कुबेरेणापि सर्वथा}
{विभीषणेन यस्यैव सखित्वमभवन्नृप}% ॥५॥

\twolineshloka
{विष्णुभक्तः सत्यसन्धो बभूव नृपतिः परः}
{तस्यैवं शासतो राजन्राज्यं निहतकण्टकम्}% ॥६॥

\twolineshloka
{बभूव दुहिता गेहे चन्द्रभागा सरिद्वरा}
{शोभनाय च सा दत्ता चन्द्रसेनसुताय वै}% ॥७॥

\twolineshloka
{स कदाचित्समायातः श्वशुरस्य गृहे नृप}
{एकादशीव्रतदिनं समायातं सुपुण्यदम् }% ॥८॥

\twolineshloka
{समागते व्रतदिने चन्द्रभागा त्वचिन्तयत्}
{किं भविष्यति देवेश मम भर्तातिदुर्बलः}% ॥९॥

\twolineshloka
{क्षुधां न क्षमते सोढुं पिता चैवोग्रशासनः}
{पटहस्ताड्यते यस्य सम्प्राप्ते दशमीदिने}% ॥१०॥

\twolineshloka
{न भोक्तव्यं न भोक्तव्यं न भोक्तव्यं हरेर्दिने}
{श्रुत्वा पटहनिर्घोषं शोभनस्त्वब्रवीत्प्रियाम्}% ॥११॥
\onelineshloka*
{किं कर्तव्यं मया कान्ते देहि शिक्षां वरानने}

\uvacha{चन्द्रभागोवाच}
\onelineshloka
{मत्पितुर्वेश्मनि प्रभो भोक्तव्यं नाद्य केनचित्}% ॥१२॥

\twolineshloka
{गजैरश्वैश्च कलभैरन्यैः पशुभिरेव  च}
{तृणमन्नं तथा वारि न भोक्तव्यं हरेर्दिने}% ॥१३॥

\twolineshloka
{मानवैश्च कुतः कान्त भुज्यते हरिवासरे}
{यदि त्वं भोक्ष्यसे कान्त ततो गर्हां प्रयास्यसि}% ॥१४॥
\onelineshloka*
{एवं विचार्य मनसा सुदृढं मानसं कुरु}

\uvacha{शोभन उवाच}
\onelineshloka
{सत्यमेतत्प्रिये वाक्यं करिष्येऽहमुपोषणम्}% ॥१५॥

\twolineshloka
{दैवेन विहितं यद्धि तत्तथैव भविष्यति}
{इति दृढां मतिं कृत्वा चकार नियमं व्रते}% ॥१६॥

\twolineshloka
{क्षुधया पीडिततनुः स बभूवातिदुःखितः}
{इति चिन्तयतस्तस्य आदित्योऽस्तमगाद्गिरिम्}% ॥१७॥

\twolineshloka
{वैष्णवानां नराणां सा निशा हर्षविवर्धनी}
{हरिपूजारतानां च जागरासक्तचेतसाम्}% ॥१८॥

\twolineshloka
{बभूव नृपशार्दूल शोभनस्यातिदुःखदा}
{रवेरुदयवेलायां शोभनः पञ्चतां गतः}% ॥१९॥

\twolineshloka
{दाहयामास राजा तं राजयोग्यैश्च दारुभिः}
{चन्द्रभागा नात्मदेहं ददाह पतिना सह}% ॥२०॥

\twolineshloka
{कृत्वौर्ध्वदैहिकं तस्य तस्थौ जनकवेश्मनि}
{शोभनश्च नृपश्रेष्ठ रमाव्रतप्रभावतः}% ॥२१॥

\twolineshloka
{प्राप्तो देवपुरं रम्यं मन्दराचलसानुनि}
{अनुत्तममनाधृष्यमसङ्ख्येयगुणान्वितम्}% ॥२२॥

\twolineshloka
{हेमस्तम्भमयैसौधै रत्नवैडूर्यमण्डितैः}
{स्फाटिकैर्विविधाकारैर्विचित्रैरुपशोभितैः}% ॥२३॥

\threelineshloka
{सिंहासनसमारूढः सुश्वेतच्छत्रचामरः}
{किरीटकुण्डलयुतो हारकेयूरभूषितः}
{स्तूयमानश्च गन्धर्वैरप्सरोगणसेवितः }% ॥२४॥

\twolineshloka
{शोभनः शोभते यत्र राजराजोपरो यथा}
{सोमशर्मेति विख्यातो मुचुकुन्दपुरेऽभवत्}% ॥२५॥

\twolineshloka
{तीर्थयात्राप्रसङ्गेन भ्रमन्विप्रो ददर्श तम्}
{नृपजामातरं ज्ञात्वा तत्समीपं जगाम सः}% ॥२६॥

\twolineshloka
{शोभनोऽपि तदा ज्ञात्वा सोमशर्माणमागतम्}
{आसनादुत्थितः शीघ्रं नमश्चक्रे द्विजोत्तमम्}% ॥२७॥

\twolineshloka
{चकार कुशलप्रश्नं श्वशुरस्य नृपस्य च}
{कान्तायाश्चन्द्रभागायास्तथैव नगरस्य च}% ॥२८॥

\uvacha{सोमशर्मोवाच}

\twolineshloka
{कुशलं वर्तते राजन्श्वशुरस्य गृहे तव}
{चन्द्रभागा कुशलिनी सर्वतः कुशलं पुरे}% ॥२९॥

\twolineshloka
{स्ववृत्तं कथ्यतां राजन्नाश्चर्यं विद्यतेऽद्भुतम्}
{पुरं विचित्रं रुचिरं न दृष्टं केनचित्क्वचित्}% ॥३०॥
\onelineshloka*
{एतदाचक्ष्व नृपते कुतः प्राप्तमिदं त्वया}

\uvacha{शोभन उवाच}
\onelineshloka
{कार्तिकस्यासिते पक्षे या नामैकादशी रमा}% ॥३१॥

\twolineshloka
{तामुपोष्य मया प्राप्तं द्विजेन्द्र पुरमध्रुवम्}
{ध्रुवं भवति येनैव तत्कुरुष्व द्विजोत्तमः}% ॥३२॥

\uvacha{द्विज उवाच}

\twolineshloka
{कथमध्रुवमेतद्धि कथं हि भवति ध्रुवम्}
{तत्त्वं कथय राजेन्द्र तत्करिष्यामि नान्यथा}% ॥३३॥

\uvacha{शोभन उवाच}

\twolineshloka
{मयैतद्विहितं विप्र श्रद्धाहीनं व्रतोत्तमम्}
{तेनाहमध्रुवं मन्ये ध्रुवं भवति तच्छृणु}% ॥३४॥

\twolineshloka
{मुचुकुन्दस्य दुहिता चन्द्रभागा सुशोभना}
{तस्यै कथय वृत्तान्तं ध्रुवमेतद्भविष्यति}% ॥३५॥

\uvacha{कृष्ण उवाच}

\twolineshloka
{स श्रुत्वा वचनं तस्य मुचुकुन्दपुरं गतः}
{उवाच सर्वं वृत्तान्त्तं चन्द्रभागाग्रतो द्विजः}% ॥३६॥

\uvacha{सोमशर्मोवाच}

\threelineshloka
{प्रत्यक्षं दयितः कान्तस्तव दृष्टो मया शुभे}
{इन्द्रतुल्यमनाधृष्यं दृष्टं तस्य पुरं मया}
{अध्रुवं तेन तत्प्रोक्तं ध्रुवं भवति तत्कुरु}% ॥३७॥

\uvacha{चन्द्रभागोवाच}

\twolineshloka
{तत्र मां नय विप्रर्षे पतिदर्शनलालसाम्}
{आत्मनो व्रतपुण्येन करिष्यामि पुरं ध्रुवम्}% ॥३८॥

\twolineshloka
{आवयोर्द्विजसंयोगो यथा भवति तत्कुरु}
{प्राप्यते हि महत्पुण्यं कृत्वा योगं वियुक्तयोः}% ॥३९॥

\twolineshloka
{इति श्रुत्वा सह तया सोमशर्मा जगाम ह}
{आश्रमं वामदेवस्य मन्दराचलसन्निधौ}% ॥४०॥

\twolineshloka
{वामदेवोऽशृणोत्सर्वं वृत्तान्तं कथितं तयोः}
{अभ्यषिञ्चच्चन्द्रभागां वेदमन्त्रैरथोज्ज्वलाम्}% ॥४१॥

\twolineshloka
{ऋषिमन्त्रप्रभावेन विष्णुवासरसेवनात्}
{दिव्यदेहा बभूवासौ दिव्यां गतिमवाप ह}% ॥४२॥

\twolineshloka
{पत्युः समीपमगमत्प्रहर्षोत्फुल्ललोचना}
{जहर्ष शोभनोऽतीव दृष्ट्वा कान्तां समागताम्}% ॥४३॥

\twolineshloka
{समाहूय स्वके वामे पार्श्वेतां सन्न्यवेशयत्}
{अथोवाच प्रियं हर्षाच्चन्द्रभागा प्रियं वचः}% ॥४४॥

\twolineshloka
{शृणु कान्त हितं वाक्यं यत्पुण्यं विद्यते मयि}
{अष्टवर्षाधिका जाता यदाहं पितृवेश्मनि}% ॥४५॥

\threelineshloka
{ततः प्रभृति यच्चीर्णं मया चैकादशीव्रतम्}
{यथोक्तविधिसंयुक्तं श्रद्धायुक्तेन चेतसा}
{तेन पुण्यप्रभावेन भविष्यति पुरं ध्रुवम्}% ॥४६॥

\twolineshloka
{सर्वकामसमृद्धं च यावदाभूतसम्प्लवम्}
{एवं सा नृपशार्दूल रमते पतिना सह}% ॥४७॥

\twolineshloka
{दिव्यभोगा दिव्यरूपा दिव्याभरणभूषिता}
{शोभनोऽपि तया सार्धं रमते दिव्यविग्रहः}% ॥४८॥

\twolineshloka
{रमाव्रतप्रभावेन मन्दराचलसानुनि}
{चिन्तामणिसमा ह्येषा कामधेनुसमाथ वा}% ॥४९॥

\twolineshloka
{रमाभिधाना नृपते तवाग्रे कथिता मया}
{तस्या माहात्म्यमनघ श्रुतं सर्वं त्वया नृप}% ॥५०॥

\twolineshloka
{मया तवाग्रे कथितं माहात्म्यं पापनाशनम्}
{एकादशीव्रतानां च पक्षयोरुभयोरपि}% ॥५१॥

\twolineshloka
{यथा कृष्णा तथा शुक्ला विभेदं नैव कारयेत्}
{सेवितैकादशी नॄणां भुक्तिमुक्तिप्रदायिनी}% ॥५२॥

\twolineshloka
{धेनुः श्वेता यथा कृष्णा उभयोः सदृशं पयः}
{तथैव तुल्यफलदं स्मृतमेकादशीद्वयम्}% ॥५३॥

\twolineshloka
{एकादशीव्रतानां यो माहात्म्यं शृणुते नरः}
{सर्वपापविनिर्मुक्तो विष्णुलोके महीयते}% ॥५४॥

॥इति श्रीपाद्मे महापुराणे पञ्चपञ्चाशत्साहस्र्यां संहितायामुत्तरखण्डे उमापतिनारदसंवादान्तर्गतकृष्णयुधिष्ठिरसंवादे कार्त्तिक-कृष्ण-रमा-एकादशी-माहात्म्यं नाम द्विषष्टिमोऽध्यायः॥६२॥


\hyperref[sec:ekadashi_mahatmyam_padma_puranam]{\closesub}
\clearpage

\sect{कार्त्तिक-शुक्ल-प्रबोधिनी-एकादशी-माहात्म्यम्}
\label{sec:padma-karttika-shukla-prabodhini}


\uvacha{युधिष्ठिर उवाच}

\twolineshloka
{श्रुतं रमाया माहात्म्यं त्वत्तः कृष्ण यथातथम्}
{कार्तिके शुक्लपक्षे या तां मे कथय मानद}% ॥१॥

\uvacha{श्रीकृष्ण उवाच}

\twolineshloka
{शृणु राजन्प्रवक्ष्यामि शुक्ले चोर्जदले तु या}
{सा यथा नारदे प्रोक्ता ब्रह्मणा लोककारिणा}% ॥२॥

\uvacha{नारद उवाच}

\twolineshloka
{प्रबोधिन्याश्च माहात्म्यं वद विस्तरतो मम}
{यस्यां जागर्ति गोविन्दो धर्मकर्मप्रवर्तकः}% ॥३॥

\uvacha{ब्रह्मोवाच}

\twolineshloka
{प्रबोधिन्याश्च माहात्म्यं पापघ्नं पुण्यवर्धनम्}
{मुक्तिप्रदं सुबुद्धीनां शृणुष्व मुनिसत्तम}% ॥४॥

\twolineshloka
{तावद्गर्जन्ति तीर्थानि आसमुद्रसरांसि च}
{यावत्प्रबोधिनी विष्णोस्तिथिर्नाऽऽयाति कार्तिके}% ॥५॥

\twolineshloka
{तावद्गर्जन्ति विप्रेन्द्र गङ्गाभागीरथी क्षितौ}
{यावन्नायाति पापघ्नी कार्तिके हरिबोधिनी}% ॥६॥

\twolineshloka
{अश्वमेधसहस्राणि राजसूयशतानि च}
{एकेनैवोपवासेन प्रबोधिन्या लभेन्नरः}% ॥७॥

\twolineshloka
{यद्दुर्ल्लभं यदप्राप्यं त्रैलोक्यस्य न गोचरम्}
{तदप्यप्रार्थितं पुत्रं ददाति हरिबोधिनी}% ॥८॥

\twolineshloka
{ऐश्वर्यं सम्पदं प्रज्ञां राज्यं च सुखसम्पदः}
{ददात्युपोषिता भक्त्या जनेभ्यो हरिबोधिनी}% ॥९॥

\twolineshloka
{मेरुमन्दरमात्राणि पापान्युक्तानि यानि च}
{एकेनैवोपवासेन दहते पापनाशिनी}% ॥१०॥

\twolineshloka
{पूर्वजन्मसहस्रेषु यत्पापं समुपार्जितम्}
{निशि जागरणं चास्या दहते तूलराशिवत्}% ॥११॥

\twolineshloka
{उपवासं प्रबोधिन्यां यः करोति स्वभावतः}
{विधिवन्मुनिशार्दूल यथोक्तं लभते फलम्}% ॥१२॥

\twolineshloka
{यथोक्तं कुरुते यस्तु विधिवत्सुकृतं नरः}
{स्वल्पं मुनिवरश्रेष्ठ मेरुतुल्यं भवेत्फलम्}% ॥१३॥

\twolineshloka
{विधिहीनं तु यः कुर्यात्सुकृतं मेरुमात्रकम्}
{अणुमात्रं तदाप्नोति फलं धर्मस्य नारद}% ॥१४॥

\twolineshloka
{ये ध्यायन्ति मनोवृत्या ये करिष्यन्ति बोधिनीम्}
{वसन्ति पितरो हृष्टा विष्णुलोके च तस्य वै}% ॥१५॥

\twolineshloka
{विमुक्ता नारकैर्दुःखैर्याति विष्णोः परं पदम्}
{कृत्वा तु पातकं घोरं ब्रह्महत्यादिकं नरः}% ॥१६॥

\twolineshloka
{कृत्वा तु जागरं विष्णोर्धौतपापो भवेन्नरः}
{दुष्प्राप्यं यत्फलं विप्र अश्वमेधादिकैर्मखैः}% ॥१७॥

\twolineshloka
{प्राप्यते तत्सुखेनैव प्रबोधिन्यास्तु जागरे}
{आप्लुत्य सर्वतीर्थेषु प्रदत्त्वा काञ्चनं महीम्}% ॥१८॥

\twolineshloka
{तत्फलं समवाप्नोति यत्कृत्वा जागरं हरेः}
{जातः स एव सुकृती कुलं तेनैव पावितम्}% ॥१९॥

\twolineshloka
{कार्तिके मुनिशार्दूल कृता येन प्रबोधिनी}
{यथा ध्रुवं नृणां मृत्युर्धनं गात्रं तथाऽध्रुवम्}% ॥२०॥

\twolineshloka
{इति ज्ञात्वा मुनिश्रेष्ठ कर्तव्यं वैष्णवं दिनम्}
{यानि कानि च तीर्थानि त्रैलोक्ये सम्भवन्ति च}% ॥२१॥

\twolineshloka
{तानि तस्य गृहे सम्यग्यः करोति प्रबोधिनीम्}
{किं तस्य बहुभिः पुण्यैः कृता येन प्रबोधिनी}% ॥२२॥

\twolineshloka
{पुत्रपौत्रप्रदा ह्येषा कार्तिके हरिबोधिनी}
{स ज्ञानी च स योगी च स तपस्वी जितेन्द्रियः}% ॥२३॥

\twolineshloka
{भोगो मोक्षश्च तस्यास्ति उपास्ते हरिबोधिनीम्}
{विष्णोः प्रियतरा ह्येषा धर्मसारसहायिनी}% ॥२४॥

\twolineshloka
{यः करोति नरो भक्त्या भुक्तिभाक्स भवेन्नरः}
{प्रबोधिनीमुपोषित्वा गर्भे न विशते नरः}% ॥२५॥

\twolineshloka
{सर्वधर्मान्परित्यज्य तस्मात्कुर्वीत नारद}
{कर्मणा मनसा वाचा पापं यत्समुपार्जितम्}% ॥२६॥

\twolineshloka
{तत्क्षालयति गोविन्दः प्रबोधिन्यां तु जागरे}
{स्नानं दानं जपः पूजां समुद्दिश्य जनार्दनम्}% ॥२७॥

\twolineshloka
{नरो यत्कुरुते वत्स प्रबोधिन्यां तदक्षयम्}
{येऽर्चयन्ति नरास्तस्यां भक्त्या देवं च माधवम्}% ॥२८॥

\twolineshloka
{समुपोष्य प्रमुच्यन्ते पापैस्तैः शतजन्मजैः}
{महाव्रतमिदं पुत्र महापापौघनाशनम्}% ॥२९॥

\twolineshloka
{प्रबोधवासरं विष्णोर्विधिवत्समुपोषयेत्}
{व्रतेनानेन देवेशं परितोष्य जनार्दनम्}% ॥३०॥

\twolineshloka
{विराजयन्दिशः सर्वाः प्रयाति हरिमन्दिरम्}
{कर्तव्यैषा प्रयत्नेन नरैः कान्तिधनार्थिभिः}% ॥३१॥

\twolineshloka
{बाल्ये यत्सञ्चितं पापं यौवने वार्धके तथा}
{शतजन्मकृतं पापं स्वल्पं वा यदि वा बहु}% ॥३२॥

\twolineshloka
{तत्क्षालयति गोविन्दश्चास्यामभ्यर्चितो नृणाम्}
{धनधान्यवहा पुण्या सर्वपापहरा परा}% ॥३३॥

\twolineshloka
{तामुपोष्य हरेर्भक्त्या दुर्ल्लभं न भवेत्क्वचित्}
{चन्द्रसूर्योपरागे च यत्फलं परिकीर्तितम्}% ॥३४॥

\twolineshloka
{तत्सहस्रगुणं प्रोक्तं प्रबोधिन्यां प्रजागरे}
{स्नानं दानं जपो होमः स्वाध्यायोऽभ्यर्चनं हरेः}% ॥३५॥

\twolineshloka
{तत्सर्वं कोटिगुणितं प्रबोधिन्यां कृतं तु यत्}
{जन्मप्रभृतियत्पुण्यं नरेणोपार्जितं भवेत्}% ॥३६॥

\twolineshloka
{वृथा भवति तत्सर्वमकृत्वा कार्तिके व्रतम्}
{अकृत्वा नियमं विष्णोः कार्तिकं यः क्षिपेन्नरः}% ॥३७॥

\twolineshloka
{न जन्मार्जितपुण्यस्य फलं प्राप्नोति नारद}
{तस्मात्सर्वप्रयत्नेन देवदेवं जनार्दनम्}% ॥३८॥

\twolineshloka
{उपसेवेत विप्रेन्द्र सर्वकामफलप्रदम्}
{परान्नं वर्जयेद्यस्तु कार्तिके विष्णुतत्परः}% ॥३९॥

\twolineshloka
{परान्नवर्जनाद्वत्स चान्द्रायणफलं लभेत्}
{नित्यं शास्त्रविनोदेन कार्तिके मधुसूदनः}% ॥४०॥

\twolineshloka
{स दहेत्सर्वपापानि यज्ञायुतफलं लभेत्}
{न तथा तुष्यते यज्ञैर्न दानैर्वाजपादिभिः}% ॥४१॥

\twolineshloka
{यथा शास्त्रकथालापैः कार्तिके मधुसूदनः}
{ये कुर्वन्ति कथां विष्णोर्ये शृण्वन्ति शुभान्विताः}% ॥४२॥

\twolineshloka
{श्लोकार्धं श्लोकपादं वा कार्तिके गोशतं फलम्}
{सर्वधर्मान्परित्यज्य कार्तिके केशवाग्रतः}% ॥४३॥

\twolineshloka
{शास्त्रावधारणं कार्यं श्रोतव्यं च महामुने}
{श्रेयसा लोभबुद्ध्या च यः करोति हरेः कथाम्}% ॥४४॥

\twolineshloka
{कार्तिके मुनिशार्दूल कुलानां तारयेच्छतम्}
{नियमेन नरो यस्तु शृणुते वैष्णवीं कथाम्}% ॥४५॥

\twolineshloka
{कार्तिके तु विशेषेण गोसहस्रफलं लभेत्}
{प्रबोधवासरे विष्णोः शृणुते यो हरेः कथाम्}% ॥४६॥

\twolineshloka
{सप्तद्वीपवती दाने तत्फलं लभते मुने}
{श्रुत्वा विष्णुकथां दिव्यां येऽर्चयन्ति कथाविदम्}% ॥४७॥

\twolineshloka
{स्वशक्त्या मुनिशार्दूल तेषां लोकोऽक्षयः स्मृतः}
{गीतशास्त्रविनोदेन कार्तिकं यो नयेन्नरः}% ॥४८॥

\twolineshloka
{न तस्य पुनरावृत्तिर्मया दृष्टा कलिप्रिय}
{गीतं नृत्यं च वाद्यं च भव्यां विष्णुकथां मुने}% ॥४९॥

\twolineshloka
{यः करोति स पुण्यात्मा त्रैलोक्योपरि संस्थितः}
{बहुपुष्पैर्बहुफलैः कर्पूरागुरुकुङ्कुमैः}% ॥५०॥

\twolineshloka
{हरेः पूजा विधातव्या कार्तिके बोधवासरे}
{यस्मात्पुण्यमसङ्ख्यातं प्राप्यते मुनिसत्तम}% ॥५१॥

\twolineshloka
{फलैर्नानाविधैर्द्रव्यैः प्रबोधिन्यां तु जागरे}
{शङ्खे तोयं समादाय अर्घो देयो जनार्दने}% ॥५२॥

\twolineshloka
{यत्फलं सर्वतीर्थेषु सर्वदानेषु यत्फलम्}
{तत्फलं कोटिगुणितं दत्त्वाऽर्घं बोधवासरे}% ॥५३॥

\twolineshloka
{गुरुपूजा ततः कार्या भोजनाच्छादनादिभिः}
{दक्षिणाभिश्च देवर्षे तुष्ट्यर्थं चक्रपाणिनः}% ॥५४॥

\twolineshloka
{भागवतं शृणुते यस्तु पुराणं च पठेन्नरः}
{प्रत्यक्षरं भवेत्तस्य कपिलादानजं फलम्}% ॥५५॥

\twolineshloka
{कार्त्तिके मुनिशार्दूल स्वशक्त्या वैष्णवं व्रतम्}
{यः करोति यथोक्तं तु मुक्तिस्तस्य सुनिश्चला}% ॥५६॥

\twolineshloka
{केतक्या एकपत्रेण पूजितो गरुडध्वजः}
{समाः सहस्रं सुप्रीतो भवति मधुसूदनः}% ॥५७॥

\twolineshloka
{अगस्तिकुसुमैर्देवं पूजयेद्यो जनार्दनम्}
{दर्शनात्तस्य देवर्षे नरकाग्निः प्रणश्यति}% ॥५८॥

\twolineshloka
{मुनिपुष्पार्चितो विष्णुः कार्तिके पुरुषोत्तमः}
{ददात्यभिमतान्कामान्शशिसूर्यग्रहे यथा}% ॥५९॥

\twolineshloka
{विहाय सर्वपुष्पाणि मुनिपुष्पेण केशवम्}
{कार्तिके योऽर्चयेद्भक्त्या वाजिमेधफलं लभेत्}% ॥६०॥

\twolineshloka
{तुलसीदलानि पुष्पाणि ये यच्छन्ति जनार्दने}
{कार्तिके सकलं वत्स पापं जन्मायुतं दहेत्}% ॥६१॥

\twolineshloka
{दृष्टा स्पृष्टाऽथ वा ध्याता कीर्तिता नामतस्तु ता}
{रोपिता सिञ्चिता नित्यं पूजिता तुलसी शुभा}% ॥६२॥

\twolineshloka
{नवधा तुलसीभक्तिं ये कुर्वन्ति दिनेदिने}
{युगकोटिसहस्राणि तन्वन्ति सुकृतं मुने}% ॥६३॥

\twolineshloka
{यावच्छाखाप्रशाखाभिर्बीजपुष्पदलैर्मुने}
{रोपिता तुलसी पुम्भिर्वर्धते वसुधातले}% ॥६४॥

\twolineshloka
{तेषां वंशे तु ये जाता ये भविष्यन्ति ये गताः}
{आकल्पवर्षसाहस्रं तेषां वासो हरेर्गृहे}% ॥६५॥

\twolineshloka
{यत्फलं सर्वपुष्पेषु सर्वपत्रेषु नारद}
{तुलसीदलेन चैकेन कार्तिके प्राप्यते तु तत्}% ॥६६॥

\twolineshloka
{सम्प्राप्तं कार्त्तिकं दृष्ट्वा नियमेन जनार्दनः}
{पूजनीयो महाविष्णुः कोमलैस्तुलसीदलैः}% ॥६७॥

\twolineshloka
{इष्ट्वा क्रतुशतैर्देवान्दत्त्वा दानान्यनेकशः}
{तुलसीदलैस्तु तत्पुण्यं कार्तिके केशवार्चने}% ॥६८॥

॥इति श्रीपाद्मे महापुराणे पञ्चपञ्चाशत्साहस्र्यां संहितायामुत्तरखण्डे उमापतिनारदसंवादान्तर्गतकृष्णयुधिष्ठिरसंवादे कार्त्तिक-शुक्ल-प्रबोधिनी-एकादशी-माहात्म्यं नाम त्रिषष्टितमोऽध्यायः॥६३॥


\hyperref[sec:ekadashi_mahatmyam_padma_puranam]{\closesub}
\clearpage

\sect{पुरुषोत्तम-मासस्य कृष्ण-कमला-एकादशी-माहात्म्यम्}
\label{sec:padma-purushottama-masasya krishna-kamala}


\uvacha{युधिष्ठिर उवाच}

\twolineshloka
{भगवञ्छ्रोतुमिच्छामि व्रतानामुत्तमं व्रतम्}
{सर्वपापहरं विष्णोः फलदं व्रतिनां च यत्}% ॥१॥

\twolineshloka
{पुरुषोत्तममासस्य कथां ब्रूहि जनार्दन}
{को विधिः किं फलं तस्य को देवस्तत्र पूज्यते}% ॥२॥

\twolineshloka
{अधिमासे च सम्प्राप्ते व्रतं ब्रूहि जनार्दन}
{कस्य दानस्य किं पुण्यं किं कर्तव्यं नृभिः प्रभो}% ॥३॥

\twolineshloka
{कथं स्नानं च किं जाप्यं कथं पूजाविधिः स्मृतः}
{किं भोज्यमुत्तमं चान्नं मासेऽस्मिन्पुरुषोत्तमे}% ॥४॥

\uvacha{श्रीकृष्ण उवाच}

\twolineshloka
{कथयिष्यामि राजेन्द्र भवतः स्नेहकारणात्}
{पुरुषोत्तममासस्य माहात्म्यं पापनाशनम्}% ॥५॥

\twolineshloka
{अधिमासे तु सम्प्राप्ते भवेदेकादशी तु या}
{कमला नाम सा नाम्ना तिथीनामुत्तमा तिथिः}% ॥६॥

\twolineshloka
{तस्या व्रतप्रभावेन कमलाभिमुखी भवेत्}
{ब्राह्मे मुहूर्ते चोत्थाय संस्मृत्य पुरुषोत्तमम्}% ॥७॥

\twolineshloka
{स्नात्वा चैव विधानेन नियमं कारयेद्व्रती}
{गृहे त्वेकगुणं जाप्यं नद्यां तु द्विगुणं स्मृतम्}% ॥८॥

\twolineshloka
{गवां गोष्ठे सहस्रोर्ध्वमग्न्यागारे शतान्वितम्}
{शिवक्षेत्रेषु तीर्थेषु देवतानां च सन्निधौ}% ॥९॥

\twolineshloka
{लक्षं तुलस्याः सान्निध्ये ह्यनन्तं विष्णुसन्निधौ}
{अवन्त्यामभवत्कश्चिच्छिवशर्मा द्विजोत्तमः}% ॥१०॥

\twolineshloka
{तस्यात्मजास्तु पञ्चासन्कनिष्ठो दोषवानभूत्}
{तदा पित्रा परित्यक्तस्त्यक्तः स्वजनबान्धवैः}% ॥११॥

\twolineshloka
{कुकर्मणः प्रभावेन गतो दूरतरं वनम्}
{एकदा दैवयोगेन तीर्थराजं समागतः}% ॥१२॥

\twolineshloka
{क्षुत्क्षामो दीनवदनस्त्रिवेण्यां स्नानमाचरत्}
{मुनीनामाश्रमांस्तत्र विचिन्वन्क्षुधयार्दितः}% ॥१३॥

\twolineshloka
{हरिमित्रमुनेस्तत्र ददर्शाश्रममुत्तमम्}
{पुरुषोत्तममासे वै जनानां च समागमे}% ॥१४॥

\twolineshloka
{तत्राश्रमे कथयतां कथां कल्मषनाशिनीम्}
{ब्राह्मणानां मुखात्तेन श्रद्धया कमला श्रुता}% ॥१५॥

\twolineshloka
{एकादशी पुण्यतमा भुक्तिमुक्तिप्रदायिनी}
{जयशर्मा विधानेन श्रुत्वेमां कमला तिथिम्}% ॥१६॥

\twolineshloka
{एकादशीपुण्यतमा भुक्तिमुक्तिप्रदायिनी}
{व्रतं कृतं च तैः सार्धं स्थित्वा शून्यालये तदा}% ॥१७॥

\twolineshloka
{निशीथे समनुप्राप्ते लक्ष्मीस्तत्र समागता}
{वरं ददामि भो विप्र कमलायाः प्रभावतः}% ॥१८॥

\uvacha{जयशर्मोवाच}

\twolineshloka
{का त्वं कस्यासि रम्भोरु प्रसन्ना च कथं मम}
{इन्द्राणी सुरनाथस्य भवानी शङ्करस्य वा}% ॥१९॥

\twolineshloka
{गन्धर्वी किन्नरी वाऽथ वधूर्वा चन्द्रसूर्ययोः}
{त्वत्सदृक्षा न दृष्टा च न श्रुता च शुभानने}% ॥२०॥

\uvacha{लक्ष्मीरुवाच}

\twolineshloka
{प्रसन्ना साम्प्रतं जाता वैकुण्ठादहमागता}
{प्रेरिता देवदेवेन कमलायाः प्रभावतः}% ॥२१॥

\twolineshloka
{पुरुषोत्तममासस्य या पक्षे प्रथमे भवेत्}
{तस्या व्रतं त्वया चीर्णं प्रयागे मुनिसन्निधौ}% ॥२२॥

\twolineshloka
{व्रतस्यास्य प्रभावेन वशगाहं न संशयः}
{तववंशे भविष्यन्ति मानवा द्विजसत्तम}% ॥२३॥
\onelineshloka*
{मत्प्रसादादवाप्स्यन्ति सत्यं ते व्याहृतं मया}

\uvacha{ब्राह्मण उवाच}
\onelineshloka
{प्रसन्ना यदि मे पद्मे व्रतं विस्तरतो वद}% ॥२४॥
\onelineshloka*
{यत्कथासु प्रवर्तन्ते साधवो ये जना द्विजाः}

\uvacha{लक्ष्मीरुवाच}
\onelineshloka
{श्रोतॄणां परमं श्राव्यं पवित्राणामनुत्तमम्}% ॥२५॥

\twolineshloka
{दुःस्वप्ननाशनं पुण्यं श्रोतव्यं यत्नतस्ततः}
{उत्तमः श्रद्धया युक्तः श्लोकं श्लोकार्धमेव वा}% ॥२६॥

\twolineshloka
{पठित्वा मुच्यते सद्यो महापातककोटिभिः}
{मासानां परमो मासः पक्षिणां गरुडो यथा}% ॥२७॥

\twolineshloka
{नदीनां च यथा गङ्गा तिथीनां द्वादशी तिथिः}
{अद्यापि निर्जराः सर्वे भारते जन्मलिप्सवः}% ॥२८॥

\twolineshloka
{तमर्चयन्ति विविधा नारायणमनामयम्}
{ये यजन्ति सदा भक्त्या देवं नारायणं प्रभुम्}% ॥२९॥

\twolineshloka
{तानर्चयन्ति सततं ब्रह्माद्या देवतागणाः}
{येऽपि नामपरा ये च हरिकीर्तनतत्पराः}% ॥३०॥

\twolineshloka
{हरिपूजापरा ये च ते कृतार्थाः कलौ युगेः}
{शुक्ले वा यदि वा कृष्णे भवेदेकादशी द्वयम्}% ॥३१॥

\threelineshloka
{गृहस्थानां भवेत्पूर्वा यतीनामुत्तरा स्मृता}
{एकादशी द्वादशी च रात्रिशेषे त्रयोदशी}
{तत्र क्रतुशतं पुण्यं त्रयोदश्यां तु पारणे}% ॥३२॥

\twolineshloka
{एकादश्यां निराहारः स्थित्वाहमपरेऽहन्नि}
{भोक्ष्यामि पुण्डरीकाक्ष शरणं मे भवाच्युत}% ॥३३॥

\twolineshloka
{अमुं मन्त्रं समुच्चार्य देवदेवस्य चक्रिणः}
{भक्तिभावेन तुष्टात्मा उपवासं समाचरेत्}% ॥३४॥

\twolineshloka
{कुर्याद्देवस्य पुरतो जागरं नियतो व्रती}
{गीतैर्वाद्यैश्च नृत्यैश्च पुराणपठनादिभिः}% ॥३५॥

\twolineshloka
{ततः प्रातः समुत्थाय द्वादशीदिवसे व्रती}
{स्नात्वा विष्णुं समभ्यर्च्य विधिवत्प्रयतेन्द्रियः}% ॥३६॥

\twolineshloka
{पञ्चामृतेन संस्नाप्य एकादश्यां जनार्दनः}
{द्वादश्यां च पयःस्नानाद्धरेः सारूप्यमश्नुते}% ॥३७॥

\twolineshloka
{अज्ञानतिमिरान्धस्य व्रतेनानेन केशव}
{प्रसीद सन्मुखो भूत्वा ज्ञानदृष्टिप्रदो भव}% ॥३८॥

\twolineshloka
{एवं विज्ञाप्य देवेशं देवदेवं गदाधरम्}
{ब्राह्मणान्भोजयेद्भक्त्या तेभ्यो दद्याच्च दक्षिणाम्}% ॥३९॥

\twolineshloka
{ततः स्वबन्धुभिः सार्धं नारायणपरायणः}
{कृत्वा पञ्चमहायज्ञान्स्वयं भुञ्जीत वाग्यतः}% ॥४०॥

\twolineshloka
{एवं यः प्रयतः कुर्यात्पुण्यमेकादशीव्रतम्}
{स याति विष्णुभवनं पुनरावृत्तिदुर्ल्लभम्}% ॥४१॥

\twolineshloka
{इत्युक्त्वा कमला तस्मै वरं दत्त्वा तिरोदधे}
{सोऽपि विप्रो धनी भूत्वा पितुर्गेहं समागतः}% ॥४२॥

\uvacha{श्रीकृष्ण उवाच}

\twolineshloka
{एवं यः कुरुते राजन्कमलाव्रतमुत्तमम्}
{शृणुयाद्वासरे विष्णोः सर्वपापैः प्रमुच्यते}% ॥४३॥

॥इति श्रीपाद्मे महापुराणे पञ्चपञ्चाशत्साहस्र्यां संहितायामुत्तरखण्डे उमापतिनारदसंवादान्तर्गतकृष्णयुधिष्ठिरसंवादे पुरुषोत्तम-मासस्य कृष्ण-कमला-एकादशी-माहात्म्यं नाम चतुःषष्टितमोऽध्यायः॥६४॥


\hyperref[sec:ekadashi_mahatmyam_padma_puranam]{\closesub}
\clearpage

\sect{पुरुषोत्तम-मासस्य शुक्ल-कामदा-एकादशी-माहात्म्यम्}
\label{sec:padma-purushottama-masasya shukla-kamada}


\uvacha{युधिष्ठिर उवाच}

\twolineshloka
{श्रुतानि बहुधर्माणि व्रतानि च जगत्प्रभो}
{एकादशीसमं किञ्जिच्छ्रुतं नैव जनार्दन}% ॥१॥

\twolineshloka
{पुनस्त्वेकादशीं ब्रूहि पापघ्नीं पुण्यदायिनीम्}
{यां कृत्वा मनुजो लोके प्राप्नुयात्परमं पदम्}% ॥२॥

\uvacha{श्रीकृष्ण उवाच}

\twolineshloka
{शुक्ले वा यदि वा कृष्णे यदा चैकादशी भवेत्}
{न त्याज्या जगतीपाल मोक्षसौख्यविवर्धनी}% ॥३॥

\twolineshloka
{एकादशी कलौ राजन्भवबन्धविमोचनी}
{कामदा सर्वकामानां पापानां पापहा भुवि}% ॥४॥

\twolineshloka
{रविवारेऽथ माङ्गल्ये सङ्क्रमे वा नृपोत्तम}
{एकादशी सदोपोष्या पुत्रपौत्रविवर्धनी}% ॥५॥

\twolineshloka
{एकादशीव्रतं क्वापि न त्याज्यं विष्णुवल्लभैः}
{आयुः कीर्तिप्रदं नित्यं सन्तानारोग्यवित्तदम्}% ॥६॥

\twolineshloka
{मोक्षदं रूपदं राज्यं नित्यमेकादशीव्रतम्}
{ये कुर्वन्ति महीपाल श्रद्धया परया युताः}% ॥७॥

\twolineshloka
{यथोक्तविधिना लोके ते नरा विष्णुरूपिणः }
{जीवन्मुक्तास्तु भूपाल दृश्यन्ते नात्र संशयः}% ॥८॥

\uvacha{युधिष्ठिर उवाच}

\twolineshloka
{जीवन्मुक्ताः कथं कृष्ण विष्णुरूपाः कथं पुनः}
{पापरूपाश्च दृश्यन्ते परं कौतूहलं हि मे}% ॥९॥

\uvacha{श्रीकृष्ण उवाच}

\twolineshloka
{ये च राजन्कलौ भक्त्या निर्जलं व्रतमुत्तमम्}
{एकादश्याः प्रकुर्वन्ति विधिदृष्टेन कर्मणा}% ॥१०॥

\twolineshloka
{न कथं विष्णुरूपास्ते जीवन्मुक्ताः कथं नहि}
{सर्वपापहरं पुण्यं व्रतमेकादशीसमम्}% ॥११॥

\twolineshloka
{न किञ्चिद्विद्यते राजन्सर्वकामप्रदं नृणाम्}
{एकाशनं दशम्यां च नन्दायां निर्जलं व्रतम्}% ॥१२॥

\twolineshloka
{पारणं चैव भद्रायां कृत्वा विष्णुसमा नराः}
{श्रद्धावान्यस्तु कुरुते कामदाया व्रतं शुभम्}% ॥१३॥

\twolineshloka
{वाञ्छितं लभते सोऽपि इहलोके परत्र च}
{पवित्रा पावनी ह्येषा महापातकनाशिनी}% ॥१४॥

\twolineshloka
{भुक्तिमुक्तिप्रदा चैव कर्तॄणां नृपसत्तम}
{कामदायां विधानेन पूजयेत्पुरुषोत्तमम्}% ॥१५॥

\twolineshloka
{पुष्पधूपादिभिश्चैव नैवेद्यैर्विविधैस्तथा}
{कांस्य मांसमसूरांश्च चणकान्कोद्रवांस्तथा}% ॥१६॥

\twolineshloka
{शाकं मधु परान्नं च पुनर्भोजन मैथुनम्}
{वैष्णवो व्रतकर्ता च दशम्यां दश वर्जयेत्}% ॥१७॥

\twolineshloka
{द्यूतं क्रीडां तथा निद्रा ताम्बूलं दन्तधावनम्}
{परापवादं पैशुन्यं स्तेयं हिंसां तथा रतिम्}% ॥१८॥

\twolineshloka
{क्रोधं च वितथं वाक्यमेकादश्यां विवर्जयेत्}
{कांस्यं मांसं मसूरांश्च तैलं वितथभाषणम्}% ॥१९॥

\twolineshloka
{व्यायामं च प्रवासं च पुनर्भोजनमैथुनम्}
{वृषपृष्ठं परान्नं च शाकं च द्वादशीदिने}% ॥२०॥

\twolineshloka
{अनेन विधिना राजन्विहिता यैश्च कामदा}
{रात्रौ जागरणं कृत्वा पूजितः पुरुषोत्तमः}% ॥२१॥

\twolineshloka
{सर्वपापविनिर्मुक्तास्ते यान्ति परमां गतिम्}
{पठनाच्छ्रवणाद्राजन्गोसहस्रफलं लभेत्}% ॥२२॥

॥इति श्रीपाद्मे महापुराणे पञ्चपञ्चाशत्साहस्र्यां संहितायामुत्तरखण्डे उमापतिनारदसंवादान्तर्गतकृष्णयुधिष्ठिरसंवादे पुरुषोत्तम-मासस्य शुक्ल-कामदा-एकादशी-माहात्म्यं नाम पञ्चषष्टितमोऽध्यायः॥६५॥

\hyperref[sec:ekadashi_mahatmyam_padma_puranam]{\closesub}
\end{center}

\clearpage

\begin{center}
    \centerline{\LARGE \bfseries व्रतराजतः विविधपुराणान्तर्गत-एकादशी-माहात्म्यानि}
    \phantomsection\label{sec:ekadashi_mahatmyam_vrata_raja}
    \begin{itemize}
        \item \hyperref[sec:vrata-raja-margashirsha-krishna]{मार्गशीर्ष-कृष्ण-एकादशी-माहात्म्यम्}
        \item \hyperref[sec:vrata-raja-margashirsha-shukla-mokshada]{मार्गशीर्ष-शुक्ल-मोक्षदा-एकादशी-माहात्म्यम्}
        \item \hyperref[sec:vrata-raja-pausha-krishna-saphala]{पौष-कृष्ण-सफला-एकादशी-माहात्म्यम्}
        \item \hyperref[sec:vrata-raja-pausha-shukla-putrada]{पौष-शुक्ल-पुत्रदा-एकादशी-माहात्म्यम्}
        \item \hyperref[sec:vrata-raja-magha-krishna-shattila]{माघ-कृष्ण-षट्तिला-एकादशी-माहात्म्यम्}
        \item \hyperref[sec:vrata-raja-magha-shukla-jaya]{माघ-शुक्ल-जया-एकादशी-माहात्म्यम्}
        \item \hyperref[sec:vrata-raja-phalguna-krishna-vijaya]{फाल्गुन-कृष्ण-विजया-एकादशी-माहात्म्यम्}
        \item \hyperref[sec:vrata-raja-phalguna-shuklamalaki]{फाल्गुन-शुक्लामलकी-एकादशी-माहात्म्यम्}
        \item \hyperref[sec:vrata-raja-chaitra-krishna-papamochani]{चैत्र-कृष्ण-पापमोचनी-एकादशी-माहात्म्यम्}
        \item \hyperref[sec:vrata-raja-chaitra-shukla-kamada]{चैत्र-शुक्ल-कामदा-एकादशी-माहात्म्यम्}
        \item \hyperref[sec:vrata-raja-vaishakha-krishna-varuthini]{वैशाख-कृष्ण-वरूथिनी-एकादशी-माहात्म्यम्}
        \item \hyperref[sec:vrata-raja-vaishakha-shukla-mohini]{वैशाख-शुक्ल-मोहिनी-एकादशी-माहात्म्यम्}
        \item \hyperref[sec:vrata-raja-jyeshtha-krishnapara]{ज्येष्ठ-कृष्णापरा-एकादशी-माहात्म्यम्}
        \item \hyperref[sec:vrata-raja-jyeshtha-shukla-nirjala]{ज्येष्ठ-शुक्ल-निर्जला-एकादशी-माहात्म्यम्}
        \item \hyperref[sec:vrata-raja-ashadha-krishna-yogini]{आषाढ-कृष्ण-योगिनी-एकादशी-माहात्म्यम्}
        \item \hyperref[sec:vrata-raja-ashadha-shukla-shayani]{आषाढ-शुक्ल-शयनी-एकादशी-माहात्म्यम्}
        \item \hyperref[sec:vrata-raja-ashadha-shukla-shayani-bhavishya]{आषाढ-शुक्ल-शयनी-एकादशी-माहात्म्यम् (भविष्य-पुराणम्)}
        \item \hyperref[sec:vrata-raja-shravana-krishna-kamika]{श्रावण-कृष्ण-कामिका-एकादशी-माहात्म्यम्}
        \item \hyperref[sec:vrata-raja-shravana-shukla-putrada]{श्रावण-शुक्ल-पुत्रदा-एकादशी-माहात्म्यम्}
        \item \hyperref[sec:vrata-raja-bhadrapada-krishnaja]{भाद्रपद-कृष्णाजा-एकादशी-माहात्म्यम्}
        \item \hyperref[sec:vrata-raja-bhadrapada-shukla-padma]{भाद्रपद-शुक्ल-पद्मा-एकादशी-माहात्म्यम्}
        \item \hyperref[sec:vrata-raja-ashvina-krishnendira]{आश्विन-कृष्णेन्दिरा-एकादशी-माहात्म्यम्}
        \item \hyperref[sec:vrata-raja-ashvina-shukla-pashankusha]{आश्विन-शुक्ल-पाशाङ्कुशा-एकादशी-माहात्म्यम्}
        \item \hyperref[sec:vrata-raja-karttika-krishna-rama]{कार्त्तिक-कृष्ण-रमा-एकादशी-माहात्म्यम्}
        \item \hyperref[sec:vrata-raja-karttika-shukla-prabodhini]{कार्त्तिक-शुक्ल-प्रबोधिनी-एकादशी-माहात्म्यम्}
        \item \hyperref[sec:vrata-raja-purushottama-krishna-kamala]{पुरुषोत्तम-मासस्य कृष्ण-कमला-एकादशी-माहात्म्यम्}
        \item \hyperref[sec:vrata-raja-purushottama-shukla-kamada]{पुरुषोत्तम-मासस्य शुक्ल-कामदा-एकादशी-माहात्म्यम्}
    \end{itemize}
    \clearpage
    \sect{मार्गशीर्ष-कृष्ण-एकादशी-माहात्म्यम्}
\label{sec:vrata-raja-margashirsha-krishna}

\uvacha{सूत उवाच}

\twolineshloka
{एवं प्रीत्या पुरा विप्राः श्रीकृष्णेन परं व्रतम्}
{माहात्म्यविधिसंयुक्तमुपदिष्टं विशेषतः} %॥१॥

\twolineshloka
{उत्पत्तिं यः शृणोत्येवमेकादश्यां द्विजोत्तम}
{भुक्त्वा भोगाननकांस्तु विष्णुलोकं प्रयाति सः} %॥२॥

\uvacha{पार्थ उवाच}

\twolineshloka
{उपवासस्य नक्तस्य एकभक्तस्य च प्रभो}
{किं पुण्यं किं विधानं हि बेहि सर्व जनार्दन} %॥३॥

\uvacha{श्रीकृष्ण उवाच}

\twolineshloka
{हेमन्ते चैव सम्प्राप्ते मासि मार्गशिरे शुभे}
{शुक्लपक्षे तथा पार्थ एकादश्या मुपोषयेत्} %॥४॥

\twolineshloka
{नक्तं दशम्यां कुर्यात्तु दन्तधावनपूर्वकम्}
{दिवसस्याष्टमे भागे अन्दीभूते दिवाकरे} %॥५॥

\twolineshloka
{तत्र नक्तं विजानीयान नक्तं निशिभोजनम्}
{ततः प्रभातसमये लइल्प नियनश्चरेत्} %॥६॥

\twolineshloka
{मध्याहे च तथा पार्थ शुचिः स्नानः समानः}
{नद्यां नडागे वाप्यां वा घुत्तमं मध्यम त्वधः} %॥७॥

\twolineshloka
{क्रमाज्ज्ञेयं तथा कूपे तदनावे प्रशस्यत}
{अश्वक्रान्ने रथक्रान्तं विष्णुकान्ते वसुन्धरे} %॥८॥

\twolineshloka
{मृत्तिके हर मे पापं यन्मया पूर्वसञ्चितम्}
{क्या हतेन पापेन गच्छामि परमां गतिम्} %॥९॥

\twolineshloka
{अनेन मृत्तिकास्नान विदध्यातु व्रती नरः}
{नालपेत्पतितैश्चोरैस्तथा पाखण्डिभिः सह} %॥१०॥

\twolineshloka
{मिथ्यापवादिनो देववेदब्राह्मणनिन्दकान्}
{अन्यांश्चैव दुराचारानगम्यागामिनस्तथा} %॥११॥

\twolineshloka
{परद्रव्यापहर्तृश्च देवद्रव्यापहारिणः}
{न सम्भाषेत दृष्ट्वापि भास्करं चावलोकयेत्} %॥१२॥

\twolineshloka
{ततो गोविन्दमभ्यर्च्य नैवेद्यादिभिरादरात्}
{दीपं दद्याद्गृहे चैव भक्तियुक्तेन चेतसा} %॥१३॥

\twolineshloka
{तदिने वर्जयेत्पार्थ निद्रां मैथुनमेव च}
{गीतशास्त्रविनोदेन दिवारानं नयेद्रती} %॥१४॥

\twolineshloka
{रात्रौ जागरणं कृत्वा भक्तियुक्तेन चेतसा}
{विप्रेभ्यो दक्षिणां दत्त्वा प्रणिपत्य क्षमापयेत्} %॥५॥

\twolineshloka
{यथा शुक्ला तथा कृष्णा मान्या वै धर्मतत्परैः}
{एकादश्योयो राजन्विभेदं नैव कारयत्} %॥१६॥

\twolineshloka
{एवं हि कुरुते यस्तु शृणु तस्यापि यत्फलम्}
{शङ्खोद्धारे नरः स्नात्वा दृष्ट्वा देवं गदाधरम्} %॥१७॥

\twolineshloka
{एकादश्युपवासस्य कलां नाप्नोति षोडशीम्}
{व्यतीपाते च दानस्य लक्षमें फलं स्मृतम्} %॥१८॥

\twolineshloka
{सङ्क्रान्ति चतुर्लक्षं दानस्य च धनञ्जय}
{कुरुक्षेत्रे च यत्पुण्यं ग्रहणे चन्द्रसूर्ययोः} %॥१९॥

\twolineshloka
{तत्सर्व लभते यस्तु टङ्कादश्यामुपोषितः}
{अश्वमेधस्य यज्ञस्य करणाद्यत्फलं लभत्} %॥२०॥

\twolineshloka
{ततः शतगुणं पुण्यमेकादश्युपवासतः}
{तपस्विनो गृहे नित्यं लक्ष यस्य च भुञ्जते} %॥२१॥

\twolineshloka
{षष्टिवर्षसहस्राणि तस्य पुण्यं च यद्भवेत्}
{एकादश्युपवासेन फलं प्राप्नोति मानवः} %॥२२॥

\twolineshloka
{गोसहस्रे च यत्पुण्यं दत्ते वेदाङ्गपारगे}
{तस्मात्पुण्यं दशगुणमेकादश्युपवासिनाम्} %॥२३॥

\twolineshloka
{नित्यं च भुञ्जते यस्य गृहे दश द्विजोत्तमाः}
{यत्पुण्यं तद्दशगुणं भोजने ब्रह्मचारिणः} %॥२४॥

\twolineshloka
{एतत्सहस्रं भूदाने कन्यादाने तु तत्स्मृतम्}
{तस्मादशगुणं प्रोक्त विद्यमाने तथैव च} %॥२५॥

\twolineshloka
{विद्यादशगुणं चान्नं यो ददाति बुभुक्षिते}
{अन्नदानसमं दानं न भूतं न भविष्यति} %॥२६॥

\twolineshloka
{तृप्तिमायान्ति कौन्तेय स्वर्गस्थाः पितृदेवताः}
{एकादश्या व्रतस्यापि पुण्यसङ्ख्या न विद्यते} %॥२७॥

\twolineshloka
{एतत्पुण्यप्रभावश्च यत्सुरैरपि दुर्लभः}
{नक्तस्याद्रफलं तस्य एकभक्तस्य सत्तम} %॥२८॥

\twolineshloka
{एकभक्तं च नक्तं च उपवासस्तथैव च}
{एतेष्वन्यतमं वापि व्रतं कुर्याद्धरोर्दन} %॥२९॥

\twolineshloka
{तावद्गर्जन्ति तीर्थानि दानानि नियमा यमाः}
{एकादशी न सम्प्राप्ता यावत्तावन्मखा अपि} %॥३०॥

\twolineshloka
{तस्मादेकादशी सर्वैरुपोष्या भवभीरुभिः}
{न शङ्खन पिवेत्तोयं न खादेन्मत्स्यसूकरौ} %॥३१॥

\twolineshloka
{एकादश्यां न भुञ्जीत यन्मां त्वं पृच्छसेऽर्जुन}
{एतत्ते कथितं सर्व व्रतानामुत्तमं व्रतम्} %॥३२॥

\onelineshloka*
{एकादशीसमं नास्ति कृत्वा यज्ञसहस्रकम्}

\uvacha{अर्जुन उवाच}

\onelineshloka
{उक्ता त्वया कथं देव पुण्येयं सर्वतस्तिथिः} %॥३३॥

\onelineshloka*
{सर्वभ्योऽपि पवित्रेयं कथं होकादशी तिथिः}

\uvacha{श्रीकृष्ण उवाच}

\onelineshloka
{पुरा कृतयुगे पार्थ मुरनामा हि दानवः} %॥३४॥

\twolineshloka
{अत्यद्भुतो महारौद्रः सर्वदेवभयङ्करः}
{इन्द्रो विनिर्जितस्तेन ह्याद्यो देवः पुरन्दरः} %॥३५॥

\twolineshloka
{आदित्या वसवो ब्रह्मा वायुरग्निस्तथैव च}
{देवता निर्जितास्तेन अत्युप्रेण च पाण्डव} %॥३६॥

\twolineshloka
{इन्द्रेण कथितः सर्वो वृत्तान्तः शङ्कराय वै}
{स्वर्गलोकपरिभ्रष्टा विचरामो महीतले} %॥३७॥

\onelineshloka*
{उपायं ब्रूहि मे देव अमराणां तु का गतिः}

\uvacha{ईश्वर उवाच}

\onelineshloka
{गच्छ गच्छ सुरश्रेष्ठ यत्राम्नि गरुडध्वजः} %॥३८॥

\twolineshloka
{शरण्यश्च जगन्नाथः परित्राणपरायणः}
{ईशस्य वचनं श्रुत्वा देवराजो महामनाः} %॥३९॥

\twolineshloka
{त्रिदशैः सहितः सर्वैर्गतस्तत्र धनञ्जय}
{यत्र देवो जगन्नाथः प्रसुप्तो हि जनार्दनः} %॥४०॥

\twolineshloka
{जलमध्ये प्रसुप्तं तु दृष्ट्वा देवं जगत्पतिम्}
{कृताञ्जलिपुटो भूत्वा इदं स्तोत्रमुदीरयत} %॥४१॥

\twolineshloka
{ओं नमो देवदेवाय देवदेवः सुवन्दित}
{दैत्यारे पुण्डरीकाक्ष त्राहि नो मधुसूदन} %॥४२॥

\twolineshloka
{दैत्यभीता इमे देवा मया सह समागताः}
{शरणं त्वं जगन्नाथ त्वं कर्ता त्वं च कारकः} %॥४३॥

\twolineshloka
{त्वं माता सर्वलोकानां त्वमेव जगतः पिता}
{त्वं स्थितिस्त्वं तथोत्पत्तिस्त्वं च संहारकारकः} %॥४४॥

\twolineshloka
{सहायस्त्वं च देवानां त्वं च शान्तिकरः प्रभो}
{त्वं धरा च त्वमाकाशः सर्वविश्वोपकारकः} %॥४५॥

\twolineshloka
{भवस्त्वं च स्वयं ब्रह्मा त्रैलोक्यप्रतिपालक}
{त्वं रविस्त्वं शशाङ्कश्च त्वं च देवो हुताशनः} %॥४६॥

\twolineshloka
{हव्यं होमो हुतस्त्वं च मन्त्रतन्त्रविजो जपः}
{यजमानश्च यज्ञस्त्वं फलभोक्ता त्वमीश्वरः} %॥४७॥

\twolineshloka
{न त्वया रहितं किञ्चित्रलोक्ये सचराचरे}
{भगवन्देवदेवेश शरणागतवत्सल} %॥४८॥

\twolineshloka
{त्राहि त्राहि महायोगिन्भीतानां शरणं भव}
{दानवैर्विजिता देवाः स्वर्गभ्रष्टाः कृता विभो} %॥४९॥

\twolineshloka
{स्थानभ्रष्टा जगन्नाथ विचरन्ति महीतले}
{इन्द्रस्य वचनं श्रुत्वा विष्णुर्वचनमब्रवीत्} %॥५०॥

\uvacha{श्रीभगवानुवाच}

\twolineshloka
{कोऽसौ दैत्यौ महा यो देवा येन विनिर्जिताः}
{किं स्थानं तस्य किं नाम किं बलं कस्तदाश्रयः} %॥५१॥

\onelineshloka*
{एतत्सर्व समाचक्ष्व मघवन्निर्भयो भव}

\uvacha{इन्द्र उवाच}

\onelineshloka
{भगवन्देवदेवेश भक्तानुग्रहकारक} %॥५२॥

\twolineshloka
{दैत्यः पूर्व महानासीन्नाडीजव इति स्मृतः}
{ब्रह्मवंशसमुद्भूतो महोग्रः सुरसूदनः} %॥५३॥

\twolineshloka
{तस्य पुत्रोऽतिविख्यातो मुरनामा महासुरः}
{तस्य चन्द्रवतीनाम नगरी च गरीयसी} %॥५४॥

\twolineshloka
{तस्यां वसन्स दुष्टात्मा विश्व निर्जित्य वीर्यवान्}
{तुरान्स्ववशमानिन्ये निराकृत्य त्रिविष्टपात्} %॥५५॥

\twolineshloka
{इन्द्राग्नियमवाय्वीशसोमानिऋतिपाशिनाम्}
{पदेषु स्वयमेवासीत्सूर्यो भूत्वा तपत्यपि} %॥५६॥

\twolineshloka
{पर्जन्यः स्वयमेवासीदजेयः सर्वदेवतैः}
{जहि तं दानवं विष्णो सुराणां जयमावह} %॥५७॥

\twolineshloka
{तस्य तद्वचनं श्रुत्वा कोपाविष्टो जनादनः}
{उवाच शत्रु देवेन्द्र हनिष्ये तं महाबलम्} %॥५८॥

\twolineshloka
{प्रयान्तु सहिताः सर्वे चन्द्रवत्यां महाबलाः}
{इत्युक्ताः प्रययुः सर्वे पुरस्कृत्य हरिं सुराः} %॥५९॥

\twolineshloka
{दृष्टो देवैस्तु दैत्येन्द्रो गर्जमानस्तु दानवैः}
{असङ्ख्यातसहस्रेस्तु दिव्यप्रहरणायुधैः} %॥६०॥

\twolineshloka
{हन्यमानास्तदा देवा असुरैर्बाहुशालिभिः}
{सङ्ग्रामं ते समुत्सृज्य पलायन्त दिशो दश} %॥६१॥

\twolineshloka
{ततो दृष्ट्वा हृषीकेशं सङ्ग्रामे समुपस्थितम्}
{अन्वधावन्नाभिकुद्धा विविधायुधपाणयः} %॥६२॥

\twolineshloka
{अथ तान्प्रगुतान्दृष्ट्वा शङ्खचक्रगदाधरः}
{विव्याध सर्वगात्रेषु शरैराशीविषोपमै} %॥६३॥

\twolineshloka
{तेनाहतास्ते शनशो दानवा निधनं गताः}
{एकाङ्गो दानवः स्थित्वा युध्यमानो मुहुर्मुहुः} %॥६४॥

\twolineshloka
{तस्योपरि हृषीकेशो यद्यदायुधमुत्सृजत्}
{पुष्पवत्तत्समभ्येति कुण्ठितं तस्य तेजसा} %॥६५॥

\twolineshloka
{शस्त्रास्त्रविध्यमानोऽपि यदा जेतुं न शक्यते}
{युयोध च तदा क्रुद्धो बाहुभिः परिघोपमैः} %॥६६॥

\twolineshloka
{वाहुयुद्धं कृतं तेन दिव्यवर्ष लहाकन्द}
{तेन श्रान्तः स भगवान् गतो बदरिकाश्रमम्} %॥६७॥

\twolineshloka
{तत्र हैमवती नाम्नी गुहा परमशोजना}
{तां प्राविशन्महायोगी शयनार्थ जगत्पतिः} %॥६८॥

\twolineshloka
{योजनबादशायामः पत्रद्वारा धनञ्जय}
{अहं तत्र प्रसुप्तोस्मि भयभीतो न संशयः} %॥६९॥

\twolineshloka
{महायुद्धेन तेनैव श्रान्तोऽहं गाडुनन्दन}
{दानवः पृष्ठतो लग्नः प्रविवेश स तां गुहाम्} %॥७०॥

\twolineshloka
{प्रसुप्तं मां तदा दृष्ट्वाऽचिन्नयदानो हृदि}
{हरिमेनं हनिष्येऽहं दानवानां क्षयावहम्} %॥७१॥

\twolineshloka
{एवं सुदुनेम्नाय व्यवसायं व्यवस्य च}
{समुद्भूता ममाङ्गेभ्यः कन्यैका च महाप्रभा} %॥७२॥

\twolineshloka
{दिव्यप्रहरणा देवी युद्धाय समुपस्थिता}
{मुरेण दानवेन्द्रेण ईक्षिता पाण्डुनन्दन} %॥७३॥

\twolineshloka
{युद्धं समीरितं तेन स्त्रिया तत्र प्रयाचितम्}
{तेनायुध्यत सा नित्यं तां दृष्ट्वा विस्मयं गतः} %॥७४॥

\twolineshloka
{केनेयं निर्मिता रौद्रा अत्युप्राशनिपातिनी}
{इत्युक्त्वा दानवेन्द्रोऽसौ युयुधे कन्यया तया} %॥७५॥

\twolineshloka
{ततस्तया महादेव्या त्वरया दानवो बली}
{छित्त्वा सर्वाणि शस्त्राणि क्षणेन विरथः कृतः} %॥७६॥

\twolineshloka
{बाहुप्रहरणोपेतो धावमानो महाबलात्}
{तलेनाहत्यहृदये तया देव्या निपातितः} %॥७॥

\twolineshloka
{पुनरुत्थाय सोचावकन्याह ननकाङ्क्षया}
{दानवं पुनरायान्तं रोषणाहत्य तच्छिरः} %॥७८॥

\twolineshloka
{क्षणानिपातयामास भूमौ तच्च समुज्ज्वलत्}
{दैत्यः कृत्तशिराः सोथ ययौ वैवस्वतालयम्} %॥७९॥

\twolineshloka
{शेषा भयादिना दीनाः पातालं विविशुद्धिषः}
{तत्तः समुत्थितो देवः पुरो दृष्ट्वाऽसुरं हतम्} %॥८०॥

\twolineshloka
{कन्यां पुरः स्थिता चापि कृताञ्जलिपुटां नताम्}
{विस्मयोत्फुल्लनयनः प्रोवाच जगतां पतिः} %॥८१॥

\twolineshloka
{केनायं निहतः सङ्ख्ये दानवो दुष्टमानसः}
{येन देवाः सगन्धर्वाः सेन्द्राश्च समस्हणाः} %॥८२॥

\twolineshloka
{सनागाः सहलोकेशा लीलयैव विनिर्जिताः}
{येनाहं निर्जितो भीतः श्रान्तः सुप्तो गुहामिमाम्} %॥८३॥

\onelineshloka*
{केन कारुण्यभावेन राक्षतोऽहं पलायितः}

\uvacha{कन्योवाच}

\onelineshloka
{मया विनिहतो दैत्यस्त्वदंशोद्भूतया प्रभो} %॥८४॥

\twolineshloka
{दृष्ट्वा सुप्तं हरे त्वां तु दैत्यो हन्तुं समुद्यतः}
{त्रैलोक्यकण्टकस्येत्यं व्यवसाय प्रबुध्य च} %॥८५॥

\twolineshloka
{हतो मया दुरात्माऽसौ देवता निर्भयाः कृताः}
{तवैवाहं महाशक्तिः सर्वशत्रुभयङ्करी} %॥८६॥

\twolineshloka
{त्रैलोक्यरक्षणार्थाय हतो लोकभयङ्करः}
{निहतं दानवं दृष्ट्वा किमाश्चर्य वद प्रभो} %॥८७॥

\uvacha{श्रीभगवानुवाच}

\twolineshloka
{निहते दानवेन्द्रेऽस्मिन्सन्तुष्टोऽहं त्वयानघे}
{हृष्टाः पुष्टाश्च वै देवा आनन्दः समजायत} %॥८८॥

\twolineshloka
{आनन्दस्त्रिषु लोकेषु देवानां यस्त्वया कृतः}
{प्रसन्नोस्म्यनचे तुभ्यं वरं वरय सुव्रते} %॥८९॥

\onelineshloka*
{ददामि तन्न सन्देहो यत्सुरैरपि दुर्लभम्}

\uvacha{कन्योवाच}

\onelineshloka
{यदि तुष्टोऽसि मे देव यदि दयो वरो मम} %॥९०॥

\twolineshloka
{तारयेहं महापापादुपवासपरं नरम्}
{उपवालस्य यत्पुण्यं तस्यार्द्ध नक्तभोजने} %॥९१॥

\twolineshloka
{तदर्द्ध च भवेत्तस्य एकभुक्तं करोति यः}
{यः करोति व्रतं भक्त्या दिने मम जितेन्द्रियः} %॥९२॥

\twolineshloka
{स गत्वा वैष्णवं स्थानं कल्पकोटिशतानि च}
{भुञ्जानो विविधान्भोगानुपवासी जितेन्द्रियः} %॥९३॥

\twolineshloka
{भगवंस्त्वत्प्रसादेन भवत्वेष वरो मम}
{उपवासं च नक्तं च एकभुक्तं करोति यः} %॥९४॥

\onelineshloka*
{तस्य धर्म च वित्तं च मोक्षं देहि जनार्दन}

\uvacha{श्रीभगवानुवाच}

\onelineshloka
{यत्त्वं वदसि कल्याणि तत्सर्वं च भविष्यति} %॥९५॥

\twolineshloka
{मम भक्ताश्च ये लोकास्तव भक्ताश्च ये नराः}
{त्रिषु लोकेषु विख्याताः प्राप्स्यन्ति मम सनिधिम्} %॥९६॥

\twolineshloka
{एकादश्यां समुत्पन्ना मम शक्तिः परा यतः}
{अत एकादशीत्येवं तव नाम भविष्यति} %॥९॥

\twolineshloka
{दग्ध्वा पापानि सर्वाणि दास्यामि पदमव्ययम्}
{तृतीया चाष्टमी चैव नवमी च चतुर्दशी} %॥९८॥

\twolineshloka
{एकादशी विशेषेण तिथयो मे महाप्रियाः}
{सर्वतीर्थाधिकं पुण्यं सर्वदानाधिकं फलम्} %॥९९॥

\twolineshloka
{सर्वव्रताधिक चैव सत्यं सत्यं वदामि ते}
{एवं दत्त्वा वरं तस्यास्तत्रैवान्तरधीयत} %॥१००॥

\twolineshloka
{दृष्टा तुष्टा तु सा जाता तदा एकादशीतिथिः}
{इमामेकादशी पार्थ करिष्यन्ति नरास्तु ये} %॥१॥

\twolineshloka
{तेषां शत्रु हनिष्यामि दास्यामि परमां गतिम्}
{अन्येऽपि ये करिष्यन्ति एकादश्या महाव्रतम्} %॥२॥

\twolineshloka
{हरामि तेषां विनांश्च सर्वसिद्धिं ददामि च}
{एवमुक्का समुत्पत्तिरेकादश्याः पृथासुत} %॥३॥

\twolineshloka
{इयमेकादशी नित्या सर्वपापक्षयङ्करी}
{एकैव च महापुण्या सर्वपापनिषूदनी} %॥४॥

\twolineshloka
{उदिता सर्वलोकेषु सर्वसिद्धिकरी तिथिः}
{शुक्का वाप्यथवा कृष्णा इति भेदं न कारयेत्} %॥५॥

\twolineshloka
{कर्तव्ये तु उभे पार्थ न तुल्या द्वादशीतिथिः}
{अन्तरं नैव कर्तव्यं समस्तैर्वतकारिभिः} %॥६॥

\twolineshloka
{तिथिरेका भवेत्सर्वा पक्षयोरुभयोरपि}
{ते यान्ति परमं स्थानं यत्राले गरुडध्वजः} %॥७॥

\twolineshloka
{धन्यान्ते मानवा लोके विष्णुभक्तिपरायणाः}
{एकादश्यास्तु माहात्म्यं सर्वकालं तु यः पठेत्} %॥८॥

\twolineshloka
{अश्वमेधस्य यत्पुण्यं तदाप्नोति न संशय}
{यः शृणोनि दिवारात्रौ नरो विष्णुपरायणः} %॥९॥

\twolineshloka
{तद्भक्तमुखनिष्पन्नां कथां विष्णोः सुमङ्गलाम्}
{कुलकोटिसमायुक्तो विष्णुलोके महीयते} %॥१०॥

\twolineshloka
{एकादश्याश्च माहात्म्यं पादमेकं शृणोति यः}
{ब्रह्महत्यादिकं पापं नदयते नात्र संशयः} %॥११॥

\twolineshloka
{विष्णुधर्मः समो नास्ति गीतार्थेन धनञ्जय}
{एकादशी समं नास्ति व्रतं नाम सनातनम्} %॥११२॥

॥इति श्रीकृष्णार्जुनसंवादे मार्गशीर्षकृष्णैकादशीमाहात्म्यं सम्पूर्णम्॥


\hyperref[sec:ekadashi_mahatmyam_vrata_raja]{\closesub}
\clearpage

\sect{मार्गशीर्ष-शुक्ल-मोक्षदा-एकादशी-माहात्म्यम्}
\label{sec:vrata-raja-margashirsha-shukla-mokshada}

\uvacha{युधिष्ठिर उवाच}

\twolineshloka
{वन्दे विष्णु प्रभु साक्षाल्लोकत्रयसुखप्रदम्}
{विश्वेशं विश्वकर्तारं पुराणं पुरुषोत्तमम्} %॥१॥

\twolineshloka
{पृच्छामि देवदेवेश संशयोऽस्ति महान्मम}
{लोकानां तु हितार्थाय पापानां च क्षयाय च} %॥२॥

\twolineshloka
{मार्गशीर्षे सिते पक्षे किन्नामैकादशी भवेत्}
{कीदृशश्च विधिस्तस्याः को देवस्तत्र पूज्यते} %॥३॥

\onelineshloka*
{एतदाबश्व मे स्वामिविस्तरेण यथातथम्}

\uvacha{श्रीकृष्ण उवाच}

\onelineshloka
{सम्यक् पृष्टं त्वया राजन् साधु ते विमला मतिः} %॥४॥

\twolineshloka
{कथयिष्यामि राजेन्द्र हरिवासरमुत्तमम्}
{उत्पन्ना सा सिते पक्षे द्वादशी मम वल्लभा} %॥५॥

\twolineshloka
{मार्गशीर्षे समुत्पन्ना मम देहान्नराधिप}
{मुरस्य च वधार्थाय प्रख्याता मम वल्लभा} %॥६॥

\twolineshloka
{कथिना सा मया चैव त्वद्ने राजसत्तम}
{पूर्वमेकादशी राजन् त्रैलोक्ये सचराचरे} %॥७॥

\twolineshloka
{मार्गशीर्षेसिते पक्षे चोत्पत्तिरिति नामतः}
{अतः पर प्रवक्ष्यामि मार्गशीर्षलितां तथा} %॥८॥

\twolineshloka
{मोक्षाना-नातिविख्यातां सर्वपापहरां पराम्}
{देव दामोदरं तस्या पूजयेञ्च प्रयत्नतः} %॥९॥

\twolineshloka
{गन्धपुष्पादिभिश्चैव गीतनृत्यः सुमङ्गलैः}
{शृणु राजेन्द्र वक्ष्यामि कथा पौराणिकी शुभाम्} %॥१०॥

\twolineshloka
{यस्याः श्रवणमात्रेण वाजपेयफलं लभेत्}
{अधोगतिं गता ये व पितृमातृसुतादयः} %॥११॥

\twolineshloka
{अस्याः पुण्यप्रभावेण स्वर्ग यान्ति न संशयः}
{एतत्मात्कारणाद्राजन्महिमान शृणुष्व तम्} %॥१२॥

\twolineshloka
{पुरा वै नगरे रम्ये गोकुलै न्यबसन्नृपः}
{वैखानसेति राजर्षिः पुत्रवत्पालयन्प्रजाः} %॥१३॥

\twolineshloka
{द्विजाश्च न्यवसंस्तत्र चतुर्वेदपरायणाः}
{एवं स राज्यं कुर्वाणो रात्रौ तु स्वप्नमध्यतः} %॥१४॥

\twolineshloka
{ददर्श जनकं स्वं तु अधोयोनिगतं नृपः}
{एवं दृष्ट्वा तु तं तन्त्र विस्मयो फुललोचनः} %॥१५॥

\onelineshloka*
{कथयामास वृत्तान्तं द्विजाने स्वप्नसम्भवम्}

\uvacha{राजो वाच}

\onelineshloka
{मया तु स्वपिता दृष्टो नरके पतितो द्विजाः} %॥१६॥

\twolineshloka
{तारयस्वेति मां तात अधोयोनिगतं सुत}
{इति ब्रुवाणः स तदा मया दृष्टः पिता स्वयम्} %॥१८॥

\twolineshloka
{तदाप्रभृति भो विप्रा नाहं शर्म लभाम्यहो}
{एतद्राज्यं मम महदसह्यमसुखं तथा} %॥१८॥

\twolineshloka
{अश्वा गजा रथाश्चैव न मां रोचन्ति सर्वथा}
{न कोशोऽपि सुखायालं न किञ्चित्सुखदं मम} %॥१९॥

\twolineshloka
{न दारा न सुता मह्यं रोचन्ते द्विसत्तम}
{किं करोमि व गच्छामि शरीरं मे तु दह्यते} %॥२०॥

\twolineshloka
{दानं व्रतं तपो योगो येनैव मम पूर्वजाः}
{मोक्षमायान्ति विप्रेन्द्रास्तदेव कययन्तु मे} %॥२१॥

\twolineshloka
{किं तेन जीवता लोके सुपुत्रेण बलीयला}
{पिता तु यस्य नरके तस्य जन्म निरर्थकम्} %॥२२॥

\uvacha{ब्राह्मणा ऊचुः}

\twolineshloka
{पर्वतस्य मुनेरत्र आश्रमो निकटे नृप}
{गम्यतां राजशार्दूल भूतं भव्यं विजानतः} %॥२३॥

\twolineshloka
{तेषां श्रुत्वा ततो वाक्यं विषण्णो राजसत्तमः}
{जगाम नत्र यत्रासौ आश्रमे पर्वतो मुनिः} %॥२४॥

\twolineshloka
{ब्राह्मणैर्वेष्टितः शान्तैः प्रजाभिश्च समन्ततः}
{आश्रमो विपुलस्तस्य मुनिभिः सन्निषेवितः} %॥२५॥

\twolineshloka
{ऋग्वेदिभिर्याजुषैश्च सामावणकोविदः}
{वेष्टितो मुनिभिस्तत्र द्वितीय इव पद्मजः} %॥२६॥

\twolineshloka
{दृष्ट्वा तं मुनिशार्दूलं राजा वैखानसस्तदा}
{जगाम चावनिं मूर्ना दण्डवत् प्रणनाम च} %॥२७॥

\twolineshloka
{पप्रच्छ कुशलं तस्य सप्तस्वङ्गेष्वसौ मुनिः}
{राज्ये निष्कण्टकत्वं च राजसौख्यसमन्वितम्} %॥२८॥

\uvacha{राजोवाच}

\twolineshloka
{तव प्रसादान्कुशलमङ्गेषु मम सप्ततु}
{विभवेष्वनुकूलेषु कश्चिद्विन्न उपस्थितः} %॥२९॥

\twolineshloka
{एवं मे संशयं ब्रह्मन् प्रष्टुं त्वामहमागतः}
{एवं श्रुत्वा नृपवचः पर्वतो मुनिसत्तमः} %॥३०॥

\twolineshloka
{ध्यानस्तिमितनेत्रोऽसौ भूतं भव्यं व्यचिन्तयत}
{मुहूर्तमेकं ध्यात्वा च प्रत्युवाच नृपोत्तमम्} %॥३१॥

\uvacha{मुनिरुवाच}

\twolineshloka
{जानेऽहं तव राजेन्द्र पितुः पापं विकमणः}
{पूर्वजन्मनि ते पित्रा स्वपत्नीद्रयमध्यतः} %॥३२॥

\twolineshloka
{कामासक्तेन चैकस्या ऋतुभङ्गः कृतः स्त्रियः}
{त्राहि देहीति जल्पत्या अन्यस्याश्च नराधिप} %॥३३॥

\onelineshloka*
{कर्मणा तेन सततं नरके पतितो ह्ययम्}

\uvacha{राजोवाच}

\onelineshloka
{केन व्रतेन दानेन मोक्षस्तस्य भवेन्मुने} %॥३४॥

\onelineshloka*
{निरयात्पापसंयुक्तातन्ममाचक्ष्व पृच्छतः}

\uvacha{मुनिरुवाच}

\onelineshloka
{मार्गशीर्षे सिते पक्षे मोक्षानाम्नी हरेस्तिथिः} %॥३५॥

\twolineshloka
{सर्वैस्तु तद्वतं कृत्वा पित्रे पुण्यं प्रदीयताम्}
{तस्य पुण्यप्रभावेण मोक्षस्तस्य भविष्यति} %॥३॥

\twolineshloka
{मुनेर्वाक्यं ततः श्रुत्वा नृपः स्वगृहमागतः}
{आग्रहायणिकी शुक्ला प्राप्ता भरतसत्तम} %॥३७॥

\twolineshloka
{अन्तःपुरचरैः सर्वैः पुत्ररस्तदा नृपः}
{व्रतं कृत्वा विधानेन पित्रे पुण्यं ददौ नृपः} %॥३८॥

\twolineshloka
{तस्मिन्दत्ते तदा पुण्ये पुष्पवृष्टिरभूदिवः}
{वैखानसपिता तेन गतः स्वर्ग स्तुतो गणैः} %॥३९॥

\twolineshloka
{राजानमन्तरिक्षाच शुद्धां गिरमभाषत}
{स्वस्त्यस्तु ते पुत्र सदेत्यथ स त्रिदिवं गतः} %॥४०॥

\twolineshloka
{एवं यः कुरुते राजन् मोक्षामेकादशीमिमाम्}
{तस्य पापं क्षयं याति मृतो मोक्षमवाप्नुयात्} %॥४१॥

\twolineshloka
{नातः परतरा काचिन्मोझदा विमला शुभा}
{पुण्यसङ्ख्यां तु तेषां वै न जानेऽहं तु यः कृता} %॥४२॥

\twolineshloka
{पठनाच्छ्वणाचास्या वाजपेयफलं लभेत्}
{चिन्तामणिसमा ह्येषा स्वर्गमोक्षप्रदायिनी} %॥४३॥

॥इति श्रीब्रह्माण्डपुराणे मार्गशीर्षशुक्लैकादश्या मोक्षानाम्न्या माहात्म्यं सम्पूर्णम्॥


\hyperref[sec:ekadashi_mahatmyam_vrata_raja]{\closesub}
\clearpage

\sect{पौष-कृष्ण-सफला-एकादशी-माहात्म्यम्}
\label{sec:vrata-raja-pausha-krishna-saphala}

\uvacha{युधिष्ठिर उवाच}

\twolineshloka
{पौषस्य कृष्णपक्षे तु द्वादशी या भवेत् प्रभो}
{किन्नाम को विधिस्तस्याः को देवस्तत्र पूज्यते} %॥१॥

\onelineshloka*
{एतदाचक्ष्व मे स्वामिन्विस्तरेण जनार्दन}

\uvacha{श्रीकृष्ण उवाच}

\onelineshloka
{कथयिष्यामि राजेन्द्र भवतः स्नेहकारणात्} %॥२॥

\twolineshloka
{तथा तुष्टिर्न मे राजन् ऋतुभिश्चाप्तदक्षिणैः}
{यथा तुष्टिर्भवन्मह्यमेकादश्या व्रतेन वै} %॥३॥

\twolineshloka
{तस्मात्सर्वप्रयत्नेन कर्तव्यो हरिवासरः}
{पौषस्य कृष्णपक्षे तु द्वादशी या भवेन्नृप} %॥४॥

\twolineshloka
{तस्याश्चैव च माहात्म्यं शृणुष्वैकापमानसः}
{गदितायाश्च चै राजन्नैकादश्यो भवन्ति हि} %॥५॥

\twolineshloka
{तासामपि हि सर्वासां विकल्पं नैव कारयेत्}
{अतः परं प्रवक्ष्यामि पौधे कृष्णा हि द्वादशी} %॥६॥

\twolineshloka
{तस्या विधिं नृपश्रेष्ठ लोकानां हितकाम्यया}
{पौषस्य कृष्णपक्षे या सफलानाम नामतः} %॥७॥

\twolineshloka
{नारायणोऽधिदेवोऽस्याः पूजयेत्तं प्रयत्नतः}
{पूर्वेण विधिना राजन् कर्तव्यकादशी जनः} %॥८॥

\twolineshloka
{नागानां च यथा शेषः पक्षिणां गरुडो यथा}
{यथाश्वमेधो यज्ञानां नदीनां जाह्नवी यथा} %॥९॥

\twolineshloka
{देवानां च यथाविष्णुर्द्विपदां ब्राह्मणो यथा}
{व्रतानां च तथा राजन् प्रवरैकादशी तिथिः} %॥१०॥

\twolineshloka
{ते जना भरतश्रेष्ठ मम पूज्याश्च सर्वशः}
{हरिवासरसंसक्ता वर्तन्ते ये भृशं नृप} %॥११॥

\twolineshloka
{सफलानाम या प्रोक्ता तस्याःपूजाविध शृणु}
{फलैर्मा पूजयेत्तत्र कालदेशोद्भवैः शुभैः} %॥१२॥

\twolineshloka
{नारिकेलफल शुद्धस्तथा वै बीजपूरकैः}
{जम्बीरैर्दाडिमैश्चैव तथा पूगफलैरपि} %॥१३॥

\twolineshloka
{लवङ्गैविविधैश्चान्यैस्तथा चाम्रफलादिभिः}
{पूजयेदेवदेवेशं धूपै-पैर्यथाक्रमम्} %॥१४॥

\twolineshloka
{सफलायां दीपदानं विशेषेण प्रकीर्तितम्}
{रात्री जागरणं तत्र कर्तव्यं च प्रयत्नतः} %॥१५॥

\twolineshloka
{यावदुन्मिषते नेत्रं तावजागर्ति यो निशि}
{एकाग्रमानसो भूत्वा तस्य पुण्यफलं शृणु} %॥१६॥

\twolineshloka
{तत्समो नास्ति वै यज्ञस्तीर्थ नत्सदृशं न हि}
{तत्समं न व्रतं किञ्चिदिह लोके नराधिप} %॥१७॥

\twolineshloka
{पञ्चवर्षसहस्राणि तपस्तप्त्वा च यत्फलम्}
{तत्फलं समवाप्नोति सफलाजागरेण वै} %॥१८॥

\twolineshloka
{श्रूयतां राजशार्दूल सफलायाः कथानकम्}
{चम्पावतीति विख्याता पुरी माहिष्मतस्य च} %॥१९॥

\twolineshloka
{माहिष्मतव्य राजर्वेश्चन्वारश्वाभवन्सुताः}
{तेषां मध्ये तु यो ज्येष्ठः स महापापसंयुतः} %॥२०॥

\twolineshloka
{परदाराभिगामी च यतवेश्यारतः सदा}
{पितुर्द्रव्यं स पापिष्ठो गमयामास सर्वशः} %॥२१॥

\twolineshloka
{असवृत्तिरतो नित्यं देवताद्विजनिन्दकः}
{वैष्णवानां च देवानां नित्यं निन्दारतः स वै} %॥२२॥

\twolineshloka
{ईदृग्विधं तदा दृष्ट्वा पुत्र माहिष्मतो नृपः}
{राज्यानिष्कासयामास लुम्पकं नाम नामतः} %॥२३॥

\twolineshloka
{राज्यानिष्कासितस्तेन पित्रा चैवापि बन्धुभिः}
{परिवारजनैः सर्वैस्त्यक्तो राज्ञो भयात्तदा} %॥२४॥

\twolineshloka
{लुम्पकोऽपि तदा त्यक्तश्चिन्तयामास चैकलः}
{मयात्र किं प्रकर्तव्यं त्यक्तेन पितृबान्धवैः} %॥२५॥

\twolineshloka
{इति चिन्तापरो भृत्वा मतिं पापे तदाकरोत्}
{मया तु गमनं कार्य बने त्यक्त्वा पुरं पितुः} %॥२६॥

\twolineshloka
{तस्मानापितुः सर्व व्यापयिष्ये पुरं निशि}
{दिवा वने चरिष्यामि रात्रावपि पितुः पुरे} %॥२७॥

\twolineshloka
{इत्येवं स मति कृत्वा लुम्पको दैवपातितः}
{निर्जगाम पुरात्तस्माद्गतोऽसौ गहनं वनम्} %॥२८॥

\twolineshloka
{जीवघातकरो नित्यं नित्यं स्तेयपरायणः}
{सर्व च नगरं तेन मुषितं पापकर्मणा} %॥२९॥

\twolineshloka
{गृहीतश्च परिल्यको लोके राज्ञो भयात्तदा}
{जन्मान्तरीयपापेन राज्यभ्रष्टः स पापकृत्} %॥३०॥

\twolineshloka
{आमिषाभिरतो नित्यं नित्यं वै फलभक्षकः}
{आश्रमस्तस्य दुष्टस्य वासुदेवस्य सम्मतः} %॥३१॥

\twolineshloka
{अश्वत्थो वर्तते तत्र जीर्णो बहुलवार्षिकः}
{देवत्वं तस्य वृक्षस्य वर्तते तद्ने महत्} %॥३२॥

\twolineshloka
{तत्रैव न्यबसचासौ लुम्पकः पापबुद्धिमान्}
{एवं कालक्रमेणैव वसतस्तस्य पापिनः} %॥३३॥

\twolineshloka
{दुष्कर्मनिरतस्यास्य कुर्वतः कर्म निन्दितम्}
{पौषस्य कृष्णपक्षे तु पूवस्मिन् सफलादिनात्} %॥३४॥

\twolineshloka
{दशमीदिवसे राजन्निशायां शीतपीडितः}
{लुम्पको वस्त्रहीनो वै निश्चेष्टो ह्यभवत्तदा} %॥३५॥

\twolineshloka
{पीडयमानस्तु शीतेन अश्वत्थस्य समीपगः}
{न निद्रा न सुखं तस्य गतप्राण इवाभवत्} %॥३६॥

\twolineshloka
{पीडयन्दशनैर्दन्तानवं सोऽगमयन्निशाम्}
{भानूदयेऽपि तस्याथ न सञ्ज्ञा समजायत} %॥३७॥

\twolineshloka
{लुम्पको गतसञ्ज्ञस्तु सफलादिवसे ततः}
{मध्याह्नसमये प्राप्ते सञ्ज्ञा लेभे स पार्थिव} %॥३८॥

\twolineshloka
{प्राप्तसञ्ज्ञो मुहूर्तेन चोत्थितोसौ तदासनात्}
{प्रस्खलंश्च पदस्यास्पैः पडूगुवञ्चलितो मुहुः} %॥३९॥

\twolineshloka
{वनमध्ये गतस्तत्र क्षुत्तृषापीडितोऽभवत्}
{न शक्तिविघातस्य लुप्पकस्य दुरात्मनः} %॥४०॥

\twolineshloka
{फलानि भूमौ पतितान्याहृत्य च स लुपकः}
{यावत्स चागतस्तत्र तावदस्तमगाद्रविः} %॥४१॥

\twolineshloka
{किं भविष्यति तातेति विललापातिदुःखितः}
{फलानि तानि सर्वाणि वृक्षमूले निवेदयन्} %॥४२॥

\twolineshloka
{इत्युवाच फलैरेभिः प्रीयतां भगवान् हरिः}
{उपविष्टो लुपकश्च निद्रां लेभे न वै निशि} %॥४३॥

\twolineshloka
{तेन जागरणं मेने भगवान्मधुसूदनः}
{फलश्च पूजनं मेने सफलायां तथानघ} %॥४४॥

\twolineshloka
{कृतमेवं लुपकेन ह्यकस्माद्वतमुत्तमम्}
{तेन व्रतप्रभावेण प्राप्त राज्यमकण्टकम्} %॥४५॥

\twolineshloka
{पुण्याकुरोदयाद्राजन् यथाप्राप्तं तथा शृणु}
{रवेरुदयवेलायां दिव्योऽश्वश्चाजगाम ह} %॥४६॥

\twolineshloka
{दिव्यवस्तुपरीवारो लुम्पकस्य समीपतः}
{तस्थौ स तुरगो राजन वागुवाचाशरीरिणाम्} %॥४७॥

\twolineshloka
{प्राप्तुहि त्वं नृपसुत स्वराज्यं हतकण्टकम्}
{वासुदेवप्रसादेन सफलायाः प्रभावतः} %॥४८॥

\twolineshloka
{पितुः समीपं गच्छ त्वं भुंश्व राज्यमकण्टकम्}
{तथेत्युक्त्वा त्वसौ तत्र दिव्यरूपधरोऽभवत्} %॥४९॥

\twolineshloka
{कृष्णे मतिश्च तस्यासीत्परमा वैष्णवी तथा}
{दिव्याभरणशोभाढयस्तातं नत्वा स्थितो गृहे} %॥५०॥

\twolineshloka
{वैष्णवाय ततो दत्तं पित्रा राज्यमकण्टकम्}
{कृतं राज्यं तु तेनैव वर्षाणि सुबहून्यपि} %॥५१॥

\twolineshloka
{हरिवासरसलीनो विष्णुभक्तिरतः सदा}
{मनोज्ञास्त्वस्य पुत्राः स्युर्दाराः कृष्णप्रसादतः} %॥५२॥

\twolineshloka
{ततः स वार्द्ध के प्राप्त राज्यं पुत्रे निवेश्य च}
{वनं गतः संयतात्मा विष्णुभक्तिपरायणः} %॥५३॥

\twolineshloka
{साधयित्वा तथात्मानं विष्णुलोकं जगाम ह}
{एवं ये वै प्रकुर्वन्ति सफलैकादशीव्रतम्} %॥५४॥

\twolineshloka
{इह लोके यशः प्राप्य मोक्षं यास्यत्यसंशयम्}
{धन्यास्ते मानवा लोके सफलातकारिणः} %॥५५॥


\threelineshloka
{तस्मिन्नन्मनि ते मोक्षं लभन्ते नात्र संशयः}
{सफलायाश्च माहात्म्यश्रवणाद्धि विशाम्पते}
{राजसूयफलं प्राप्य वसेत्स्वर्गे च मानवः} %॥५६॥

॥इति पौषकृष्ण-एकादश्याः सफलानाम्न्या माहात्म्यं सम्पूर्णम्॥


\hyperref[sec:ekadashi_mahatmyam_vrata_raja]{\closesub}
\clearpage

\sect{पौष-शुक्ल-पुत्रदा-एकादशी-माहात्म्यम्}
\label{sec:vrata-raja-pausha-shukla-putrada}

\uvacha{युधिष्ठिर उवाच}

\twolineshloka
{कथिता वै त्वया कृष्ण सफलैकादशी शुभा}
{कथयस्व प्रसादेन शुक्ला पौषस्य या भवेत्} %॥१॥

\twolineshloka
{किन्नाम को विधिस्तस्याः को देवस्तत्र पूज्यते}
{कस्मै तुष्टो हृषीकेश त्वमेव पुरुषोत्तम} %॥२॥

\uvacha{श्रीकृष्ण उवाच}

\twolineshloka
{शृणु राजन् प्रवक्ष्यामि शुक्ला पौषस्य या भवेत्}
{तस्या विधि महारान लोकानां च हिताय वै} %॥३॥

\twolineshloka
{पूर्वेण विधिना राजन् कर्तव्यैषा प्रयत्नतः}
{पुत्रदेति च नाम्नासौ सर्वपापहरा वरा} %॥४॥

\twolineshloka
{नारायणोऽधिदेवोऽम्याः कामदः सिद्विदायकः}
{नातःपरतरा काचित्रलोक्ये सचराचरे} %॥५॥

\twolineshloka
{विद्यावन्तं यशस्वन्तं लक्ष्मीवन्तं करोत्यसौ}
{शृणु राजन् प्रवक्ष्यामि कथां पापहरां पराम्} %॥६॥

\twolineshloka
{पुरी भद्रावती नाम्नी राजा तत्र सुकेतुमान्}
{तस्य राज्ञोऽथ राज्ञी च शैब्या नाम्नीति विश्रुता} %॥७॥

\twolineshloka
{पुत्रहीनेन राज्ञा च कालो नीतो मनोरथैः}
{नैवात्मजं नृपो लेभे वंशकर्तारमेव च} %॥८॥

\twolineshloka
{तेनैव राज्ञा धर्मेण चिन्तितं बहुकालतः}
{किं करोमि छ गच्छामि सुतप्रातिः कथं भवेत्} %॥९॥

\twolineshloka
{न राष्ट्र न पुरे सौख्यं लेभे राजा सुकेतुमान्}
{शैव्यया कान्तया साई प्रत्यहं दुःखितोऽभवत्} %॥१०॥


\threelineshloka
{तावुभौ दम्पती नित्यं चिन्ताशोकपरायणौ}
{पितरोऽन्य जलं दत्तं कवोष्णमुपभुनते}
{राजः पश्चान्न पश्यामो योऽस्मान् सन्तर्पयिष्यति} %॥११॥

\twolineshloka
{इत्येवं संस्मरन्तोऽस्य पितरो दुःखिनोभवन्}
{न बान्धवा न मित्राणि नामात्याः सुहृदस्तथा} %॥१२॥

\twolineshloka
{रोचन्ते तस्य भूपस्य न गजाश्वपदातयः}
{नैराश्यं भूपतेस्तस्य मनस्येवमजायत} %॥१३॥

\twolineshloka
{नरस्य पुत्रहीनस्य नास्ति वै जन्मनः फलम्}
{अपुत्रस्य गृहं शून्यं हृदयं दुःखितं सदा} %॥१४॥

\twolineshloka
{पितृदेवमनुष्याणां नानृगित्वं सुतं विना}
{तस्मात्सर्वप्रयत्नेन सुतमुत्पादयेन्नरः} %॥१५॥

\twolineshloka
{इहलोके यशस्तेषां परलोके शुभा गतिः}
{येषां तु पुण्यकर्तृणां पुत्रजन्म गृहे भवेत्} %॥१६॥

\twolineshloka
{आयुरारोग्यसम्पत्तिस्तेषां गेहे प्रवर्तते}
{पुत्राः पौत्राश्च लोकाश्च भवेयुः पुण्यकर्मणाम्} %॥१७॥

\twolineshloka
{पुण्यं विना न च प्राप्तिर्विष्णुभक्तिं विना तथा}
{पुत्राणां सम्पदो वापि विद्यापाश्चेति मे मतिः} %॥१८॥

\twolineshloka
{एवं चिन्तयमानोसो राजा शर्म न लब्धवान्}
{प्रत्यूषेऽचिन्तयद्राजा निशीथेऽचिन्तयत्तथा} %॥१९॥

\twolineshloka
{ततश्चात्मविनाशं व विचार्याथ सुकेतुमान्}
{आत्मघाते दुर्गातं च चिन्तयित्वा तदा नृपः} %॥२०॥

\twolineshloka
{दृष्ट्वान्मदेई प्रक्षीणमपुत्रत्वं तथैव च}
{पुनर्विचार्यात्मबुद्धचा ह्यात्मनो हितकारणम्} %॥२१॥

\twolineshloka
{अश्वारूढस्ततो राजा जगाम गहनं वनम्}
{पुरोहितादयः सर्वे न जानन्ति गतं नृपम्} %॥२२॥

\twolineshloka
{गम्भीरे विपिन राजा मृगपक्षिनिषेविते}
{विचचार तदा तस्मिन्वनवृक्षान्विलोकयन्} %॥२३॥

\twolineshloka
{वटानश्वत्थविल्वांश्च खजूरापनसांस्तथा}
{बकुलांश्च सदापर्णास्तिन्दुकांस्तिलकानपि} %॥२४॥

\twolineshloka
{शालांस्तालास्तमालांश्च ददर्श सरलानृपः}
{इगुदीककुभांश्चैव श्लेष्मातकविभीतकान्} %॥२५॥

\twolineshloka
{शल्लकीकरमर्दाश्च पाटलान् स्खदिरानपि}
{शाकांश्चैव पलाशांश्च शोभितान् ददृशे पुनः} %॥२६॥

\twolineshloka
{मृगव्याध्रवराहांश्च सिंहाशाखामृगानपि}
{गवयान् कृष्णसारांश्च सृगालाशशकानपि} %॥२७॥

\twolineshloka
{वनमार्जार कान् क्रूराञ्शल्लकांश्चमरानपि}
{ददर्श भुजगान् राजा वल्मीकादभिनिन्मृतान्} %॥२८॥

\twolineshloka
{तथा वनगजान्मत्तान्कलभैः सह सङ्गतान्}
{यूथपांश्च चतुर्दन्तान्करिणीगणमध्यगान्} %॥२९॥

\twolineshloka
{तान् दृष्ट्वा चिन्तयामास ह्यात्मनः स गजान्नृपा}
{तेषां स विचरन्मध्ये राजा शोभामवाप ह} %॥३०॥

\twolineshloka
{महदाश्चर्यसंयुक्तं ददर्श विपिनं नृपः}
{कञ्चिच्छिवारुतं शृण्वन्तुलूकविरुतं तथा} %॥३१॥

\twolineshloka
{तांस्तान् क्षिमृगान् पश्यन्बभ्राम वनमध्यगः}
{एवं ददर्श गहनं नृपो मध्यङ्गते रवौ} %॥३२॥

\twolineshloka
{क्षुत्तृभ्यां पीडितो राजा इतश्चेतश्च धावति}
{चिन्तयामास नृपतिः संशुष्कगलकन्धरः} %॥३३॥

\twolineshloka
{मया तु किं कृतं कर्म प्राप्तं दुःखं यदीदृशम्}
{मया वै तोविता देवा यज्ञैः पूजाभिरेव च} %॥३४॥

\twolineshloka
{तथैव ब्राह्मगा दानस्तोषिता मृष्टभोजनैः}
{प्रजाश्चैव यथाकालं पुत्रवन्परिपालिताः} %॥३५॥

\twolineshloka
{कस्मादुःखं मया प्रातमीदृशं दारुणं महत्}
{इति चिन्तापरो राजा जगामाथागतो वनम्} %॥३६॥

\twolineshloka
{सुकृतस्य प्रभावेण सरो दृष्टं मनोरमम्}
{मानसेन स्पर्द्धमानं पद्मिनी परिशोभितम्} %॥३७॥

\twolineshloka
{कारण्डवैश्चक्रवाकै राजहंसैश्च नादितम्}
{मकरहुभिर्मत्स्यैरन्य जलचरैर्युतम्} %॥३८॥

\twolineshloka
{समीपे सरसस्तत्र मुनीनामाश्रमान् बहून्}
{ददर्श राजा लक्ष्मीवानिमित्तैः शुभशंसिभिः} %॥३९॥

\twolineshloka
{सव्यात्परतरं चक्षुरपसव्यस्तथा करः}
{प्रास्फुरन्नृपतेस्तस्य कथयञ्शोभनं फलम्} %॥४०॥

\twolineshloka
{तस्य तीरे मुनीन् दृष्ट्वा कुर्वाणान्नैगमं जपम्}
{अवतीर्य हयात्तस्मान्मुनीनामग्रतः स्थितः} %॥४१॥

\twolineshloka
{पृथक् पृथग्ववन्दे स मुनीस्तान् संशितव्रतान्}
{कृताञ्जलिपुटो भूत्वा दण्डवञ्च प्रणम्य सः} %॥४२॥

\twolineshloka
{हर्षेण महताविष्टो बभूव नृपसत्तमः}
{तमूचुस्तेऽपि मुनयः प्रसन्नाः स्मो वयं तव} %॥४३॥

\onelineshloka*
{कथयस्वाद्य वै राजन्यत्ते मनसि वर्तते}

\uvacha{राजोवाच}

\onelineshloka
{के यूयमुग्रतपसः का आख्या भवतामपि} %॥४४॥

\onelineshloka*
{किमर्थ सङ्गता यूयं वदन्तु मम तत्त्वतः}

\uvacha{मुनय ऊचुः}

\onelineshloka
{विश्वेदेवा वयं राजन् स्नानार्थमिह चागताः} %॥४५॥

\twolineshloka
{माघो निकटमायात एतस्मात्पञ्चमेहनि}
{अद्य ह्येकादशी राजन पुत्रदा नाम नामतः} %॥४६॥

\onelineshloka*
{पुत्रं ददात्यसौ शुक्ला पुत्रदा पुत्रमिच्छताम्}

\uvacha{राजोवाच}

\onelineshloka
{ममापि यत्नो मुनयः सुतस्योत्पादने महान्} %॥४७॥

\onelineshloka
{यदि तुष्टा भवन्तो मे पुत्रो धै दीयतां शुभः}

\uvacha{मुनय ऊचुः}

\onelineshloka
{अस्मिन्नेव दिने राजन् पुत्रदा नाम वर्तते} %॥४८॥

\twolineshloka
{एकादशी तिथिः ख्याता क्रियतां व्रतमुत्तमम्}
{आशीर्वादेन चास्माकं केशवस्य प्रसादतः} %॥४९॥

\twolineshloka
{अवश्यं तव राजेन्द्र पुत्रप्राप्तिर्भविष्यति}
{इत्येवं वचनात्तेषां कृतं राज्ञा व्रतं शुभम्} %॥५०॥


\twolineshloka
{द्वादश्यां पारणं कृत्वा मुनीनत्वा पुनः पुनः}
{आजगाम गृहं राजा राज्ञी गर्भ समादधे} %॥५१॥

\twolineshloka
{मुनीनां वचनेनैव पुत्रदायाः प्रसादतः}
{पुत्रो जातस्तथा काले तेजस्वी पुण्यकर्मकृत्} %॥५२॥

\twolineshloka
{पितरं तोषयामास प्रजापालो बभूव सः}
{एतस्मात्कारणाद्राजन्कर्तव्यं पुत्रदावतम्} %॥५३॥

\twolineshloka
{लोकानां च हितार्थाय तवाने कथितं मया}
{एतद्वतं तु ये माः कुर्वन्ति पुत्रदाभिधम्} %॥५४॥

\twolineshloka
{पुत्र प्राप्येह लोके तु मृतास्ते स्वर्गगामिनः}
{पठनाच्छ्रवणाद्राजन्नश्वमेधफलं लभेत्} %॥५५॥

॥इति श्रीभविष्योत्तरपुराणे पौषशुक्लैकादश्याः पुत्रदानाम्न्या माहात्म्यं सम्पूर्णम्॥


\hyperref[sec:ekadashi_mahatmyam_vrata_raja]{\closesub}
\clearpage

\sect{माघ-कृष्ण-षट्तिला-एकादशी-माहात्म्यम्}
\label{sec:vrata-raja-magha-krishna-shattila}

\uvacha{दाल्भ्य उवाच}

\twolineshloka
{मर्त्यलोके तु सम्प्राप्ताः पापं कुर्वन्ति जन्तवः}
{ब्रह्महत्यादिपापैश्च ह्यन्यैश्च विविधैर्युताः} %॥१॥

\twolineshloka
{परद्रव्यापहर्तारः परव्यसनमोहिताः}
{कथं नायान्ति नरकान्ब्रह्मस्तवहि तत्त्वतः} %॥२॥

\twolineshloka
{अनायासेन भगवन् दानेनाल्पेन केनचित्}
{पापं प्रशममायाति येन तद्वक्तुमर्हसि} %॥३॥

\uvacha{पुलस्त्य उवाच}

\twolineshloka
{साधु साधु महाभाग गुह्यमेतत्तुदुर्लभम्}
{यन्न कस्यचिदाख्यातं ब्रह्मविष्ण्विन्द्रदैवतैः} %॥४॥

\twolineshloka
{तदहं कथयिष्यामि त्वया पृष्टो द्विजोत्तम}
{पौषमासे तु सम्प्राप्ते शुचिः स्नातो जितेन्द्रियः} %॥५॥

\twolineshloka
{कामक्रोधाभिमानेालोभपैशुन्यवर्जितः}
{देवदेवं च संस्मृत्य पादौ प्रक्षाल्य वारिणा} %॥६॥

\twolineshloka
{पुष्यक्षेण तु सङ्गृह्य गोमयं तत्र मानकः}
{तिलान्प्रक्षिप्य कार्यासं पिण्डकांश्चैव कारयेत्} %॥७॥

\twolineshloka
{अष्टोत्तरशतं होमो नात्र कार्या विचारणा}
{माघमासे तु सम्प्राप्त ह्याषाढः भवेद्यदि} %॥८॥

\twolineshloka
{मूलं वा कृष्णपक्षस्य द्वादश्यां नियमं ततः}
{गृहीयात्पुण्यफलदं विधानं तस्य मे शृणु} %॥९॥

\twolineshloka
{देवदेवं समभ्यर्च सुनातः प्रयतः शुचिः}
{कृष्णनामानि सङ्कीर्त्य एकादश्यामपोषितः} %॥१०॥

\twolineshloka
{रात्री जागरणं कुर्याद्रात्रौ होमं च कारयेत्}
{अर्चयेद्देवदेवशं द्वितीयेदि पुनहरिम्} %॥११॥

\twolineshloka
{चन्दनागुरुकर्परेनैवेद्यं कृसरं तथा}
{संस्तुत्य नाम्ना तेनैव कृष्णाख्येन पुनः पुनः} %॥१२॥

\twolineshloka
{कूष्माण्डेनारिकेले व ह्यथवा बीजपूरकैः}
{सर्वाभावे तु विप्रेन्द्र शस्तपूगीफलैर्युतम्} %॥१३॥

\twolineshloka
{अर्थ दद्याद्विधानेन पूजयित्वा जनादनम्}
{कृष्ण कृष्ण कृपालुस्त्वमगतीनां गतिर्भव} %॥१४॥

\twolineshloka
{संसारार्णवमन्नानां प्रसीद परमेश्वर}
{नमस्ते पुण्डरीकाक्ष नमस्ते विश्वभावन} %॥१५॥

\twolineshloka
{सुब्रह्मण्य नमस्तेऽस्तु महापुरुषपूर्वज}
{गृहाणायं मया दत्तं लक्ष्म्या सह जगत्पते} %॥१६॥

\twolineshloka
{ततस्तु पूजयेद्विप्रनुदकुम्भं प्रदापयेत्}
{छत्रोपानयुगैः सार्ध कृष्णो मे प्रीयतामिति} %॥१७॥

\twolineshloka
{कृष्णा धेनुः प्रदातव्या यथाशक्त्या द्विजोत्तम}
{तिलपात्र द्विजश्रेष्ठ दात्तत्र विचक्षणः} %॥१८॥

\twolineshloka
{स्नानप्राशनयोः शस्ताः श्वेताः कृष्णास्तिला मुने}
{तान्प्रदद्यात्प्रयत्नेन यथाशक्त्या द्विजोत्तम} %॥१९॥

\twolineshloka
{तिलप्ररोहजाः क्षेत्रे यावत्सङ्ख्यास्तिला द्विज}
{तावद्वर्षसहस्राणि स्वर्गलोके महीयते} %॥२०॥

\twolineshloka
{तिलस्नायी तिलोद्वौ तिलहोमी तिलोदकी}
{तिलभुक् तिलदाता च षट्तिलाः पापनाशकाः} %॥२१॥

[इयमेव पन्तलाख्या।]

\uvacha{नारद उवाच}

\twolineshloka
{कृष्ण कृष्ण महाबाहो नमस्ते विश्वभावन}
{षटतिलकादशीभूतं कीदृशं फलमश्नुते} %॥२२॥

\onelineshloka*
{सोपाख्यानं मम ब्रूहि यदि तुष्टोसि यादव}

\uvacha{श्रीकृष्ण उवाच}

\onelineshloka
{शृणु ब्रह्मन् यथावृत्तं दृष्टं तत्कथयामि ते} %॥२३॥

\twolineshloka
{मृत्युलोके पुरा ह्यासीद्राह्मण्येका च नारद}
{व्रतचर्यारता नित्यं देवपूजारता सदा} %॥२४॥

\twolineshloka
{मासोपंवासन्निरता मम भक्ता च सर्वदा}
{कृष्णोपवाससंयुक्ता मम पूजापरायणा} %॥२५॥

\twolineshloka
{शरीरं क्लेशितं नित्यमुपवासैस्तया द्विज}
{दीनानां ब्राह्मणानाञ्च कुमारीणां च भक्तितः} %॥२६॥

\twolineshloka
{गृहादिकं प्रयच्छन्ती सर्वकालं महामतिः}
{अतिकृच्छ्रता सा तु सर्वकालेषु वै द्विजा} %॥२७॥

\twolineshloka
{ब्राह्मणा नान्नदानेन तर्पिता देवता न च}
{ततःकालेन महता मया ये चिन्तितं द्विज} %॥२८॥

\twolineshloka
{शुद्धमस्या शरीरं हि व्रतः कृच्छर्न संशयः}
{जितो वैष्णवो लोकः कायक्लेशेन वै तया} %॥२९॥

\twolineshloka
{न दत्तमन्नदानं हि येन तृप्तिःपरा भवेत्}
{विचिन्त्यैवं मया ब्रह्मन् मृत्युलोकमुपेत्य च} %॥३०॥

\onelineshloka*
{कापालं रूपमास्थाय भिक्षा पात्रेण याचिता}

\uvacha{ब्राह्मण्युवाच}

\onelineshloka
{कस्मात्त्वमागतो ब्रह्मन् वद सत्यं ममाप्रतः} %॥३१॥

\twolineshloka
{पुनरेव मयाप्रोक्तं दहि भिक्षां च सुन्दरि}
{तया कोपेन महता मृत्पिण्डस्ताम्रभाजने} %॥३२॥

\twolineshloka
{क्षिप्तो यावदहं ब्रह्मन् पुनः स्वर्गः गतो द्विज}
{ततः कालेन महता तापसी तुमहाव्रता} %॥३३॥

\twolineshloka
{सदेहा स्वर्गमायाता व्रतचर्याप्रभावतः}
{मृत्पिण्डस्य प्रभावेण गृहं प्राप्त मनोरमम्} %॥३४॥

\twolineshloka
{परं तच्चैव विप्रर्षे धान्यकोशविवर्जितम्}
{गृहं यावत्प्रविश्यैषा न किञ्चितत्र पक्ष्यति} %॥३५॥

\twolineshloka
{तावद्गृहाद्विनिष्क्रम्य ममान्ते चागता द्विज}
{क्रोधेन महताविष्टा इदै वचनमब्रवीत्} %॥३६॥

\twolineshloka
{मया वतैश्च कृछैच ह्युपवासैरनेकशः}
{पूजयाराधितो देवः सर्वलो कल्य भावनः} %॥३७॥

\twolineshloka
{न तत्र दश्यते किश्चिद्गृहे मम जनार्दन}
{ततश्चोक्ता मया सा तु गृह गच्छ यथागतम्} %॥३८॥

\twolineshloka
{आगमिष्यन्ति सुतरां कौतूहलसमन्विताः}
{द्रष्टुं त्वां देवपत्न्यस्तु दिव्यरूपसमन्विताः} %॥३९॥


\threelineshloka
{द्वारं नोद्धाटय विना षट्तिलापुण्यवाचनात्}
{एवमुक्ता गता सा तु याव? मानुषी गृहम्}
{अवान्तरे समायाता देवपत्त्यश्च नारद} %॥४०॥

\twolineshloka
{ताभिश्च कथितं तत्र त्वां द्रष्टुं हि समागताः}
{द्वारमुद्घाटय त्वं च पश्यामस्त्वां शुभानने} %॥४१॥

\uvacha{मानुष्युवाच}

\twolineshloka
{यदि द्रष्टुं समायाताः सत्यं वाच्यं विशेषतः}
{षट्तिलाया व्रतं पुण्यं द्वारोद्धाटनकारणात्} %॥४२॥

\twolineshloka
{एकापि नावदत्तत्र षट्तिलैकादशीव्रतम्}
{अन्यया कथितं तत्र द्रष्टव्या मानुषी मया} %॥४३॥

\twolineshloka
{ततो द्वारं समुदाय दृष्टा ताभिश्च मानुषी}
{न देवी न च गन्धर्वी नासुरी न च पन्नगी} %॥४४॥

\twolineshloka
{दृष्टा पूर्व तथा नारी यादृशीयं द्विजर्षभ}
{देवीनामुपदेशेन षट्तिलाया व्रतं कृतम्} %॥४५॥

\twolineshloka
{मानुष्या सत्यवतया भुक्तिमुक्तिफलप्रदम्}
{रूपकान्तिसमायुक्ता क्षणेन समवाप सा} %॥४६॥

\twolineshloka
{धनं धान्यं च वस्त्रादि सुवर्ण रौप्यमेव च}
{भवनं सर्वसम्पन्नं षटतिलायाः प्रसादतः} %॥४७॥

\twolineshloka
{अतितृष्णा न कर्तव्या वित्तशाठयं विवर्जयेत्}
{आत्मवित्तानुसारेण तिलान् वस्त्रादि दापयेत्} %॥४८॥

\twolineshloka
{लभते चैवमारोग्यं ततो जन्मनि जन्मनि}
{दारिद्यं न च कष्टं च न च दौर्भाग्यमेव च} %॥४९॥

\twolineshloka
{न भवढे द्विजश्रेष्ठ पतिलायामुपोषणात्}
{अनेन विधिना ब्रह्मस्तिलदानान्न संशयः} %॥५०॥

\threelineshloka
{मुच्यते पातकैः सर्वैर्नात्र कार्या विचारणा}
{दानं च विधिना सम्यक् सर्वपापप्रणाशनम्}
{नानर्थः कश्चिन्नायासः शरीरे मुनिसत्तम} %॥५१॥

॥इति श्रीभविष्योत्तरपुराणे माघकृष्णैकादश्याः षट्तिलानाम्न्या माहात्म्यं सम्पूर्णम्॥


\hyperref[sec:ekadashi_mahatmyam_vrata_raja]{\closesub}
\clearpage

\sect{माघ-शुक्ल-जया-एकादशी-माहात्म्यम्}
\label{sec:vrata-raja-magha-shukla-jaya}

\uvacha{युधिष्ठिर उवाच}

\twolineshloka
{कृष्ण कृष्णाप्रमेयात्मवादिदेव जगत्पते}
{स्वेदजा अण्ड जाश्चैव उद्भिजाश्च जरायुजाः} %॥१॥

\twolineshloka
{तेषां कर्ता विकर्ता त्वं पालकः क्षयकारकः}
{माघस्य कृष्णपक्षे तु षट्तिला कथिता त्वया} %॥२॥

\twolineshloka
{शुक्ले यैकादशी तां च कथयस्व प्रसादतः}
{किन्नामा कोविधिस्तस्याः को देवस्तत्र पूज्यते} %॥३॥

\uvacha{श्रीकृष्ण उवाच}

\twolineshloka
{कथयिष्यामि राजेन्द्र शुक्ले माघन्य या भवेत्}
{जयानाम्रीति विख्याता सर्वपापहरा परा} %॥४॥


\threelineshloka
{पवित्रा पापहन्त्री च कामदा मोक्षदा नृणाम्}
{ब्रह्महत्यापहन्त्री च पिशाचत्वाविनाशिनी}
{नैव तस्या ते चीणे प्रेतत्वं जायते नृणाम्} %॥५॥

\twolineshloka
{नातः परतरा काचिपापन्नी मोक्षदायनी}
{एतस्मात्कारणाद्राजन् कर्तव्येयं प्रयत्नतः} %॥६॥

\twolineshloka
{श्रूयतां राजशार्दूल कथा पौराणिकी शुभा}
{पङ्कजाख्यपुराणेऽस्या महिमा कथितो मया} %॥७॥

\twolineshloka
{एकदा नाकलोके वै इन्द्रो राज्यं चकार ह}
{देवाश्च तत्र सौख्येन निवसन्ति मनोरमे} %॥८॥

\twolineshloka
{पीयूष पाननिरता ह्यप्सरोगणसेविताः}
{नन्दनं तु वनं तत्र पारिजातोपशोभितम्} %॥९॥

\twolineshloka
{रमयन्ति रमन्त्यत्र ह्यप्सरोभिर्दिवौकसः}
{एकदा रममाणोऽसौ देवेन्द्रः स्वेच्छया नृप} %॥१०॥

\twolineshloka
{नर्तयामास हर्षात्स पञ्चाशत्कोटिनायिकाः}
{गन्धर्वास्तत्र गायन्ति गन्धर्वः पुष्पदन्तकः} %॥११॥

\twolineshloka
{चित्रसनश्च नत्रैव चित्रसेनसुता तथा}
{मालिनीति च नाना तु चित्रसेनस्य कामिनी} %॥१२॥

\twolineshloka
{मालिन्यां तु समुत्पन्नः पुष्पवानिति नामतः}
{तस्य पुष्पवतः पुत्रो माल्यवान्नाम नामतः} %॥१३॥

\twolineshloka
{गन्धर्वी पुष्पवत्याख्या माल्यवत्यतिमोहिता}
{कामस्य च शरैस्तीक्ष्णैर्विद्धाङ्गी सा बभूव ह} %॥१४॥

\twolineshloka
{तया भावकटाक्षश्च माल्यवांस्तु वशीकृतः}
{लावण्यरूपसम्पत्या तस्या रूपं नृप शृणु} %॥१५॥

\twolineshloka
{बाहू तस्यास्तु कामेन कण्ठपाशौ कृताविव}
{चन्द्रबद्वदनं तस्या नयने श्रवणायते} %॥१६॥

\twolineshloka
{कर्णी तु शोभितौ तस्याः कुण्डलाभ्यां नृपोत्तम}
{कण्ठो प्रैवेयसंयुक्तो दिव्याभरणभूषितः} %॥१७॥

\twolineshloka
{पीनोन्नतौ कुचौ तस्यास्तौ हेमकलशाविव}
{अतिक्षामं तदुदर मुष्टिमात्रं च मध्यमम्} %॥१८॥

\twolineshloka
{नितम्बौ विपुलौ तस्या विस्तीर्ण जघनस्थलम्}
{चरणौ शोभमानौ तौ रक्तोत्पलसमद्युती} %॥१९॥

\twolineshloka
{ईदृश्यां पुष्पवत्यां स माल्यवानपि मोहितः}
{शक्रस्य परितोषाय नृत्यार्थ तौ समागतौ} %॥२०॥

\twolineshloka
{गायमानौ च तो तत्र ह्यप्सरोगणसङ्गतौ}
{न शुद्धगानं गायेतां चित्तभ्रमसमन्वितौ} %॥२१॥

\twolineshloka
{बद्धदृष्टी तथान्योन्यं कामबाणवशं गतौ}
{ज्ञात्वा लेखर्षभस्तत्र सङ्गतं मानसं तयोः} %॥२२॥

\twolineshloka
{कालक्रियाणां संलोपात्तथा गीतावभञ्जनात्}
{चिन्तयित्वा तु मघवानवज्ञानं तथात्मनः} %॥२३॥

\twolineshloka
{कुपितश्च तयोरित्थं शापं दास्यन्निदं जगौ}
{धिग्वां पापगतौ मूढावाजाभङ्गकरौ मम} %॥२४॥

\twolineshloka
{युवां पिशाचौ भवतं दम्पतीरूपधारिणौ}
{मृत्युलोकमनुप्राप्तौ भुनानौ कर्मणः फलम्} %॥२५॥

\twolineshloka
{एवं मघवता शप्तावुभौ दुःखितमानसौ}
{हिमवन्तमनुप्राप्ताविन्द्रशापविमोहितौ} %॥२६॥

\twolineshloka
{उभी पिशाचतां प्राप्तौ दारुणं दुःखमेव च}
{सन्तप्तमानसौ तत्र महाकृच्छ्रगतावुभौ} %॥२७॥

\twolineshloka
{गन्धं रसं च स्पर्श च न जानीतो विमोहितौ}
{पीडयमानौ तु दाहेन देहपातकरेण च} %॥२८॥

\twolineshloka
{तो ननिद्रामुखं प्राप्तौ कर्मणा तेन पीडितौ}
{परस्परं खादमानौ चरेतुर्गिरिगह्वरम्} %॥२९॥

\twolineshloka
{पीडयमानौ तु शीतेन तुषारप्रभवेण तौ}
{दन्तघर्ष प्रकुर्वाणौ रोमाश्चितवपुर्धरौ} %॥३०॥

\twolineshloka
{ऊचे पिशाचः शीतार्तः स्वपत्नी तु पिशाचिकाम्}
{किमावाभ्यां कृतं पापमत्यन्तं दुःखदायकम्} %॥३१॥

\twolineshloka
{येन प्राप्तं पिशाचत्वं स्वेन दुष्कृतकर्मणा}
{नरकं दारुणं मन्ये पिशाचत्वं च गर्हितम्} %॥३२॥

\twolineshloka
{तस्मात्सर्वप्रयत्नेन पापं नैव समाचरेत्}
{इति चिन्तापरौ तत्र ह्यास्तां दुःखेन कर्शितौ} %॥३३॥

\twolineshloka
{दैवयोगात्तयोः प्राप्ता माघस्यैकादशी सिता}
{जया नाम्नीति विख्याता तिथीनामुत्तमा तिथिः} %॥३४॥

\twolineshloka
{तस्मिन्दिने तु सम्प्राप्ते तावाहारविवर्जितौ}
{आसाते तत्र नृपते जलपानविवर्जितो} %॥३५॥

\twolineshloka
{न कृतो जीवघातश्च न पेत्रफलभक्षणम्}
{अश्वत्थस्य समीपे तु पतितौ दुःखसंयुतो} %॥३६॥

\twolineshloka
{रविरस्तं गतो राजस्तथैव स्थितयोस्तयोः}
{प्राप्ता चैव निशा घोरा दारुणा शीतकारिणी} %॥३७॥

\twolineshloka
{वेपमानौ तु तो तत्र हिमेन च जडीकृतौ}
{परस्परेण संलग्नौ गात्रयो(जयोरपि} %॥३८॥

\twolineshloka
{न निद्रां न रतिं तत्र न तौ सौस्यमविन्दताम्}
{एवं तौ राजशार्दूल शापेनेन्द्रस्य पीडितौ} %॥३९॥

\twolineshloka
{इत्थं तयोर्दुःखितयोर्निर्जमाम तदा निशा}
{जयायास्तु व्रते चीर्ण रात्री जागरणे कृते} %॥४०॥

\twolineshloka
{तयोव्रतप्रभावेण यथा ह्यासीत्तथा शृणु}
{द्वादशीदिवसे प्राप्ते ताभ्यां चीर्णे जयाव्रते} %॥४१॥

\twolineshloka
{विष्णोः प्रभावान्नृपते पिशाचत्वं तयोर्गतम्}
{पुष्पवनीमाल्यवांश्च पूर्वरूपौ बभूवतुः} %॥४२॥

\twolineshloka
{पुरातनस्नेहयुतौ पूर्वालङ्कारसंयुनौ}
{विमानमधिरूढौ तावप्सरोगणसेवितो} %॥४३॥

\twolineshloka
{स्तूयमानौ तु गन्धर्वैस्तुम्बुरुप्रमुम्बैस्तथा}
{हावभावसमायुनौ गतौ नाके मनोरमे} %॥४४॥

\twolineshloka
{देवेन्द्रस्याग्रतो गत्वा प्रणाम चक्रतुर्मुदा}
{तथाविधौ तु तो दृष्ट्रा मघवा विस्मितोऽब्रवीत्} %॥४५॥

\uvacha{इन्द्र उवाच}

\twolineshloka
{वदतं केन पुण्यन पिशाचत्वं विनिर्गतम}
{मम शापवशं प्राप्तौ केन देवेन मोचितौ} %॥४६॥

\uvacha{माल्यवानुवाच}

\twolineshloka
{वासुदेवप्रसादेन जयायाः सुव्रतेन च}
{पिशाचत्वं गतं स्वामिन्सत्यं भक्तिप्रभावतः} %॥४७॥

\twolineshloka
{इति श्रुत्वा वचस्तस्य प्रत्युवाच सुरेश्वरः}
{पवित्रौ पावनौ जातो वन्दनीयौ ममापि च} %॥४८॥

\twolineshloka
{हरिवासरकारी विष्णुभक्तिपरायणौ}
{हरिभक्तिरता ये च शिवभक्तिरतास्तथा} %॥४९॥

\twolineshloka
{अस्माकमपि ते माः पूज्या वन्द्या न संशयः}
{विहरस्व यथासौख्यं पुष्पवत्या सुरालये} %॥५०॥

\twolineshloka
{एतस्मात्कारणाद्राजन् कर्तव्यो हरिवासरः}
{जया नामेति राजेन्द्र ब्रह्महत्यापहारकः} %॥५१॥

\twolineshloka
{सर्वदानानि दत्तानि यज्ञास्तेन कृता नृप}
{सर्वतीर्थेषु सुनातः कृतं येन जयाव्रतम्} %॥५२॥

\twolineshloka
{य करोति नरो भक्त्या श्रद्धायुक्तो जयावतम्}
{कल्पकोटिशतं यावद्वैकुण्ठे मोदते ध्रुवम्} %॥५३॥

\onelineshloka
{पठनाच्यणाद्राजन्नाग्निष्टोमफलं लभेत्} %॥५४॥

॥इति श्रीभविष्योत्तरपुराणे माघशुक्क्लैकादश्या जयाया माहात्म्यं सम्पूर्णम्॥


\hyperref[sec:ekadashi_mahatmyam_vrata_raja]{\closesub}
\clearpage

\sect{फाल्गुन-कृष्ण-विजया-एकादशी-माहात्म्यम्}
\label{sec:vrata-raja-phalguna-krishna-vijaya}

\uvacha{युधिष्ठिर उवाच}

\twolineshloka
{फाल्गुनस्यासिते पक्षे किन्नामैकादशी भवेत्}
{वासुदेव कृपासिन्धो कथयस्व प्रसादतः} %॥१॥

\uvacha{श्रीकृष्ण उवाच}

\twolineshloka
{कथयिष्यामि राजेन्द्र कृष्णा या फाल्गुनी भवेत्}
{विजयेति च सा प्रोक्ता कर्तृणां जयदा सदा} %॥२॥

\twolineshloka
{तस्याश्च व्रतमाहात्म्यं सर्वपापहरं परम्}
{नारदः परिपप्रच्छ ब्रह्माणं कमलासनम्} %॥३॥

\twolineshloka
{फाल्गुनस्यासिते पक्षे विजयानाम या तिथिः}
{तस्या व्रतं सुरश्रेष्ठ कथयस्व प्रसादतः} %॥४॥

\onelineshloka*
{इति पृष्टो मारदेन प्रत्युवाच पितामहः}

\uvacha{ब्रह्मोवाच}

\onelineshloka
{शृणु नारद वक्ष्यामि कथां पापहरां पराम्} %॥५॥

\twolineshloka
{पुरातनं व्रतं ह्येतत्पवित्रं पापनाशनम्}
{यन्न कस्यचिदाख्यातं मयैतद्विजयाव्रतम्} %॥६॥

\twolineshloka
{जयं ददाति विजया नृणां चैवन संशयः}
{रामस्तपोवनं यातो वर्षाण्येव चतुर्दश} %॥७॥

\twolineshloka
{न्यवसत्पञ्चवट्यां तु ससीतश्च सल क्ष्मणः}
{तत्रैव वसतस्तस्य राघवस्य महात्मनः} %॥८॥

\twolineshloka
{रावणेन हृता भार्या सीतानाम्नी तपस्विनी}
{तेन दुःखेन रामोऽसौ मोहमभ्यागतस्तदा} %॥९॥

\twolineshloka
{भ्रमञ्जटायुषं तत्र ददर्श विगतायुषम्}
{कवन्धो निहतः पश्चाद्रमतारण्यमध्यतः} %॥१०॥

\twolineshloka
{राजे विज्ञाप्य तत्सर्व सोऽपि मृत्युवशं गतः}
{सुग्रीवेण समं सख्यमजयं समजायत} %॥११॥

\twolineshloka
{वानराणामनीकानि रामार्थ सङ्गतानि वै}
{ततो हनूमता दृष्टा लोद्याने तु जानकी} %॥१२॥

\twolineshloka
{रामसञ्ज्ञापनं तस्यै दत्तं कर्म महत्कृतम्}
{समेत्य रामेण पुनः सर्व तत्र निवेदितम्} %॥१३॥

\twolineshloka
{अथ श्रुत्वा रामचन्द्रो वाक्यं चैव हनूमतः}
{सुग्रीवानुमतेनैव प्रस्थानं समरोचयत्} %॥१४॥

\twolineshloka
{स गत्वा वानरैः सार्द्ध तीरं नदनदीपतेः}
{दृष्ट्वाब्धि दुस्तरं रामो विस्मितोऽभूत्कपिप्रियः} %॥१५॥

\twolineshloka
{प्रोत्फुल्ललोचनो भूत्वा लक्ष्मणं वाक्यमब्रवीत्}
{सौमित्र केन पुण्येन तीर्यते वरुणालयः} %॥१६॥

\twolineshloka
{अगाधसलिलैः पूर्णो नीमैः समाकुलः}
{उपायं नैव पश्यामि येनैव सुतरो भवेत्} %॥१७॥

\uvacha{लक्ष्मण उवाच}

\twolineshloka
{आदिदेवस्त्वमेवासि पुराणपुरुषोत्तम}
{बकदाल्भ्यो मुनिश्चात्र वर्तते द्वीपमध्यतः} %॥१८॥

\twolineshloka
{अस्मात्स्थानाद्योजनार्द्धमाश्रमस्तस्य राघव}
{अनेन दृष्टा ब्रह्माणो बहवो रघुनन्दन} %॥१९॥

\twolineshloka
{तं पृच्छ गत्वा राजेन्द्र पुराणमृषिपुङ्गवम्}
{इति वाक्यं ततः श्रुत्वा लक्ष्मणस्यातिशोभनम्} %॥२०॥

\twolineshloka
{जगाम राघवो द्रष्टुं बदाल्भ्यं महामुनिम्}
{त्रण नाम मुनि मूर्ना रामो विष्णुमिवामराः} %॥२१॥

\twolineshloka
{मुनित्विा ततो राम पुराणपुरुषोत्तमम्}
{केनापि कारणे व प्रविष्टं मानुषी तनुम्} %॥२२॥

\onelineshloka*
{उवाच स ऋषिस्तत्र कुतो राम तवागमः}

\uvacha{राम उवाच}

\onelineshloka
{त्वत्प्रसादादहो विप्र वरुणालयसनिधिम्} %॥२३॥

\twolineshloka
{आगतोऽस्मि ससैन्योऽत्र लङ्कां जेतुं सराक्षसाम्}
{भवतश्चानुकूल्येन तीर्यतेन्धिर्यथा मया} %॥२४॥

\twolineshloka
{तमुपायं वद मुने प्रसादं कुरु सुव्रत}
{एतस्मात्कारणादेव द्रष्टुं त्वाहमुपागतः} %॥२५॥

\uvacha{मुनिरुवाच}

\twolineshloka
{कथयिष्याम्यहं राम बतानामुत्तमं व्रतम्}
{कृतेन येन सहसा विजयस्ते भविष्यति} %॥२६॥

\twolineshloka
{लड़ां जित्वा राक्षसांश्च दीवी कीर्तिमवाप्स्यसि}
{एकाप्रमानसो भूत्वा प्रतमेतत्समाचर} %॥२७॥

\twolineshloka
{फाल्गुनस्यासिते पक्षे विजयैकादशी भवेत्}
{तस्या व्रते कृते राम विजयस्ते भविष्यति} %॥२८॥

\twolineshloka
{निःसंशयं समुद्रं च तरिष्यसि सवानरः}
{विधिस्तु श्रूयतां राम व्रतस्यास्य फलप्रदः} %॥२९॥

\twolineshloka
{दशमीदिवसे प्राप्ते कुम्भमेकं च कारयेत्}
{हेमं वा राजतं वापि तानं वाप्यथ मृन्मयम्} %॥३॥

\twolineshloka
{स्थापयेत्स्थण्डिले कुम्भं जलपूर्ण सपल्लवम्}
{सतधान्यान्य वस्तस्य यवातुपरि विन्यसेत्} %॥३१॥

\twolineshloka
{तस्योपरि न्यसेदेवं हेमं नारायणं प्रभुम्}
{एकादशीदिने प्राते प्रातःस्नानं समाचरेत्} %॥३२॥

\twolineshloka
{निश्चले स्थापित कुम्भे गन्धमाल्यातुलपिते}
{गन्धैर्धपैस्तथा दीपनैवेद्युविविधैरपि} %॥३३॥

\twolineshloka
{दाडिमालिकेरैश्च पूजयेच्च विशेषतः}
{कुम्भाने तहिन राम नेतव्यं भक्तिभावतः} %॥३४॥

\twolineshloka
{रात्रौ जागरणं तब तस्याने कारयेद्बुधः}
{द्वादशीदिवसे प्राप्त मार्तण्डस्योदये नृप} %॥३५॥

\twolineshloka
{नीत्वा कुम्भं जलोद्देशे नद्यां प्रस्रवणे तथा}
{तडागे स्थापयित्वा वा पूजयित्वा यथाविधि} %॥३६॥

\twolineshloka
{दद्यात्सदैवतं कुम्भं ब्राह्मणे वेदपारगे}
{कुम्भेन सह राजेन्द्र महादानानि दापयेत्} %॥३७॥

\twolineshloka
{अनेन विधिना राम यूथपैः सह सङ्गतः}
{कुरु व्रतं प्रयत्नेन विजयस्ते भविष्यति} %॥३८॥

\twolineshloka
{इति श्रुत्वा वचो रामो यथोक्तमकरोत्तथा}
{कृते व्रते स विजयी बभूव रघुनन्दनः} %॥३९॥

\twolineshloka
{अनेन विधिना राजन्ये कुर्वन्ति नरा व्रतम्}
{इहलोके जयस्तेषां परलोकस्तथाऽक्षयः} %॥४०॥


\threelineshloka
{एतस्मात्कारणात्पुत्र कर्तव्यं विजयावतम्}
{विजयायाश्च माहात्म्यं सर्वकिल्बिषनाशनम्}
{पठनाच्छवणात्तस्य वाजपेयफलं लभेत्} %॥४१॥

॥इति श्रीस्कन्दपुराणे फाल्गुनकृष्णैकादश्या विजयानाम्न्या माहात्म्यं सम्पूर्णम्॥


\hyperref[sec:ekadashi_mahatmyam_vrata_raja]{\closesub}
\clearpage

\sect{फाल्गुन-शुक्लामलकी-एकादशी-माहात्म्यम्}
\label{sec:vrata-raja-phalguna-shuklamalaki}

\uvacha{मान्धातोवाच}

\twolineshloka
{वद ब्रह्मन्महाभाग येन श्रेयो भवेन्मम}
{कृपया तब्रह्मयोने यधनुग्राह्यतो मयि} %॥१॥

\onelineshloka*
{सरहस्यं सेतिहासं प्रतानामुत्तमं व्रतम्}

\uvacha{वसिष्ठ उवाच}

\onelineshloka
{कथयाम्यधुना तुभ्यं सर्ववतफलप्रदम्} %॥२॥

\twolineshloka
{, आमलक्या व्रतं राजन् महापातकनाशनम्}
{मोक्षदं सर्वलोकानां गोसहस्नफलप्रदम्} %॥३॥

\twolineshloka
{अत्रयोदाहरन्तीममितिहासं पुरातनम्}
{यथामुक्तिमनुप्राप्तो व्याधो हिंसासमन्वितः} %॥४॥

\twolineshloka
{वैदिशं नाम नगरं हृष्टपुष्टजनावृतम्}
{ब्राह्मणैः क्षत्रियैश्यः शूद्रेश्च समलङ्कृतम्} %॥५॥

\twolineshloka
{रुचिरं नृपशार्दूल ब्रह्मघोषनिनादितम्}
{न नास्तिको दुष्कृतिकस्तस्मिन्पुरवरे सदा} %॥६॥

\twolineshloka
{तत्र सोमान्वयो राजा विख्यातः शशिबिन्दवः}
{राजा चैत्ररथो नाम धर्मात्मा सत्यसङ्गरः} %॥७॥

\twolineshloka
{नागायुतबलः श्रीमाञ्छस्त्रशास्त्रार्थपारगः}
{तस्मिन्छासति धर्मज्ञे धर्मात्मनि धरां प्रभो} %॥८॥

\twolineshloka
{कृपणो नैव कुत्रापि दृश्यते नैव निर्धनः}
{सुकालः क्षेममारोग्यं न दुर्भिक्षं न चेतयः} %॥९॥

\twolineshloka
{विष्णुभक्तिरता लोकास्तस्मिन्पुरवरे सदा}
{हरिपूजारताश्चैव राजा चापि विशेषतः} %॥१०॥

\twolineshloka
{न शुक्लां नैव कृष्णां च द्वादशी भुञ्जते जनाः}
{सर्वधर्मान्परित्यज्य हरिभक्तिपरायणाः} %॥११॥

\twolineshloka
{एवं संवत्सरा जग्मुर्बहवो राजसत्तम}
{जनस्य सौख्ययुक्तस्य हरिभक्तिरतस्य च} %॥१२॥

\twolineshloka
{अथ कालेन सम्प्राप्ता द्वादशी पुण्यसंयुता}
{फाल्गुनस्य सिते पक्षे नाम्ना ह्यामलकी स्मृता} %॥१३॥

\twolineshloka
{तामवाप्य जनाः सर्वे बालकाः स्थविरा नृप}
{नियमं चोपवासं च सर्वे चकुर्नरा विभो} %॥१४॥

\twolineshloka
{प्रहालं व्रतं ज्ञात्वा स्नानं कृत्वा नदीजल}
{नत्र देवालये राजा लोकयुक्तो महाप्रभुः} %॥१५॥

\twolineshloka
{पूर्णकुम्भमवस्थाप्य छोपानह संयुतम्}
{पञ्चरत्नसमायुतं दिव्यगधाधिवालितम्} %॥६॥

\twolineshloka
{दीपमालावितं चैव जामदग्न्यसमन्वितम्}
{पूजयामातुरव्यमा धात्री च मुनिभिर्जनाः} %॥१७॥

\twolineshloka
{जामदग्न्य नमस्तेऽस्तु रेगुकानन्दवर्धन}
{मलकीतच्छाय भुक्तिमुक्तिवरप्रद} %॥१८॥

\twolineshloka
{धात्रि धातृसमुद्भूते सर्वपातकनाशिनि}
{आमलकि नमस्तुभ्यं सहामा यो दकं मम} %॥१९॥

\twolineshloka
{धात्रि ब्रह्म स्वरूपासि त्वं तु रामे ग पूजिता}
{प्रदक्षिण विधानेन सर्वनापहरा भव} %॥२०॥

\twolineshloka
{तत्र जागरणं चक्रुर्जनः सर्वे स्वभक्तिः}
{एतस्मिन्नेव काले तु व्याधस्तत्र समागतः} %॥२१॥

\twolineshloka
{क्षुधाश्रमपरिव्यातो महामारण पीडितः}
{कुटुम्बार्थ जीवनाती सर्वधर्मबहि कृतः} %॥२२॥

\twolineshloka
{जागरं तत्र सोऽपश्यदामलक्यां झुधान्वितः}
{दीपमालाकुलं दृष्ट्वा तत्रैव निषसाद सः} %॥२३॥

\twolineshloka
{किमेतदिनि सञ्चिन्त्य प्रातबाविस्मयं भृशम्}
{ददर्श कुम्भं तत्रस्थं देवं दामोदरं तथा} %॥२४॥

\twolineshloka
{ददर्शामलकीवृक्षं तत्रस्थाश्चैव दीपकान्}
{वैष्णवं च तथाऽख्यानं शुश्राव पठतां नृणाम्} %॥२५॥

\twolineshloka
{एकादश्याश्च माहात्म्यं शुश्राव क्षुधितोऽपि सन्}
{जाग्रतस्तस्य सा रात्रिर्गता विस्मितचेतसः} %॥२६॥

\twolineshloka
{ततः प्रभातसमये विविशुनगरं जनाः}
{व्याधोऽपि गृहमा गत्य बुजे प्रीतमानसः} %॥२७॥

\twolineshloka
{ततः कालेन महता व्यायः पञ्चत्वमागतः}
{एकादश्याः प्रभावेण रात्री जागरणेन च} %॥२८॥

\twolineshloka
{राज्यं प्रपेदे सुमहच्चतुरङ्गवलाधिनम्}
{जय-नीनाम नगरी तब राजा विदूरथः} %॥२९॥

\twolineshloka
{तस्मात्स तनयो जज्ञ नाम्ना वतुरथो बली}
{चतुरङ्गबलोपेतो धनधान्यसमन्वितः} %॥३०॥

\twolineshloka
{दशायुतानि ग्रामागां बुभुजे भयवर्जिनः}
{तेजसादित्यसदृशः कान्त्या चन्द्रसमप्रभः} %॥३१॥

\twolineshloka
{पराक्रमे विष्णुप्तमः क्षमया पृथिवीलमः}
{धार्मिकः सत्यवादी शक्तिपरायणः} %॥३२॥

\twolineshloka
{ब्रह्मज्ञः कर्मशीलश्च प्रजापालनतत्परः}
{यजते विविधान कमान राजा परदर्पहा} %॥३३॥

\twolineshloka
{दानानि विविधान्येव प्रददाति च सर्वदा}
{एकदा मृगयां यानो देवान्मार्गपरिच्युतः} %॥३४॥

\twolineshloka
{न दिशो नैव विदिशो वेत्ति तत्र महीपतिः}
{उपधाय च दोलनकाकी गहने वने} %॥३५॥

\twolineshloka
{श्रान्तश्च क्षुधितोऽत्यन्तं संविवेश महीपतिः}
{अवान्तरे न पर्वतान्तरवासभाक्} %॥३६॥

\twolineshloka
{आययौ तत्र यत्रास्ते राजा परबलार्दनः}
{कृतवैरास्ते राज्ञा सर्वदेवोपतापिताः} %॥३७॥

\twolineshloka
{परिवार्य ततस्तस्थू राजानं भूरिदक्षिणम्}
{हन्यता हाला वायं पूर्व वरविरुद्धधीः} %॥३८॥

\twolineshloka
{अनेन निहताः पूर्व पितरौ भ्रातरः सुताः}
{पौवाश्च व मातुलाश्च निपातिताः} %॥३९॥


\threelineshloka
{निष्कासिताश्च स्वस्थानाद्विक्षिप्ताश्च दिशो दश}
{एनपदुःस्वा ते सर्वे तत्रैनं हन्तुमुद्यताः}
{पाशश्च पट्टिशैः खङ्गैर्बाणैर्धनुषि संस्थितैः} %॥४०॥

\twolineshloka
{शस्त्राणि समापतन्ति न वै शरीरे प्रविशन्ति}
{तस्यातिचापि सर्वे हतशस्त्रसङ्घा म्लेच्छा अरिजीवदेहाः} %॥४१॥

\twolineshloka
{यदापि चलितुं तत्र न शेकुल्तेरयो भृशम्}
{शस्त्रागि कुण्ठता जग्गः तवेषां हतचेतसाम्} %॥४२॥

\twolineshloka
{दीना बभूवुस्ते सर्वे ये तं हन्तुं समागताः}
{एतस्मिन्नेव काले तु तस्य राज्ञः शरीरतः} %॥४३॥

\onelineshloka
{निन्मृता प्रमदा ह्येका सर्वावयवशोभना} %॥४४॥

\twolineshloka
{दिव्य पता दिव्याभरणभूषिता}
{दिव्यमाल्याम्बरधरा भृकुटीकुटिलानना} %॥४५॥

\twolineshloka
{स्फुलि गाभ्यो च नेत्राभ्यां पावकं वमती बहु}
{चक्रोद्यतकरा चैव कालरात्रिरिवापरा} %॥४६॥

\twolineshloka
{अभ्यधामा सङ्कुद्धा म्लेच्छानत्यन्तदुःखितान्}
{निहताश्च यदा म्लेच्छास्ते विकर्मरतास्तथा} %॥४॥

\twolineshloka
{ततो राजा विबुद्धः सन् ददर्श महदद्भतम्}
{हतान् म्लेच्छगणान् दृष्ट्वा राजा हर्षमवाप सः} %॥४८॥

\twolineshloka
{इह न हता म्लेच्छा अत्यन्तं वैरिणो मम}
{केन चेदं महत्कर्म कृतमस्मद्धितार्थिना} %॥४९॥

\twolineshloka
{समय काले तु वागुवाचाशरीरिणी}
{तं स्थितं नृपतिं दृष्ट्वा निकामं विस्मयान्वितम्} %॥५०॥

\twolineshloka
{शरणं केशवादन्यो नास्ति कोपि द्वितीयकः}
{इति श्रुत्वाकाशवाणी विस्मयोत्तर} %॥५१॥

\twolineshloka
{वनात्तस्मात्स कुशली समायातः स भूमिभुक्}
{राज्यं चकार धर्मात्मा धरायो तेशवत्} %॥५२॥

\uvacha{वसिष्ठ उवाच}

\twolineshloka
{तस्मादामलकी राजन् ये कुर्वन्ति नरोत्तमाः}
{ते यान्ति के लोकं नात्र कार्या विचारणा} %॥५३॥

॥इति श्रीब्रह्माण्डपुराणे आमलक्याख्यफाल्गुनशुक्लैकादशीव्रतम्॥


\hyperref[sec:ekadashi_mahatmyam_vrata_raja]{\closesub}
\clearpage

\sect{चैत्र-कृष्ण-पापमोचनी-एकादशी-माहात्म्यम्}
\label{sec:vrata-raja-chaitra-krishna-papamochani}

\uvacha{युधिष्ठिर उवाच}

\twolineshloka
{फाल्गुनस्य सिते पक्षे श्रुता साऽमलकी मया}
{चैत्रस्य कृष्णपक्षेन किं नामैकादशी भवेत्} %॥१॥

\twolineshloka
{को विधिः किं फलं तस्या ब्रूहि कृष्ण ममाननः}
{श्रीकृष्ण शृणु राजेन्द्र वक्ष्यामि पापमोचनिकाव्रतम्} %॥२॥

\onelineshloka*
{यल्लोमशोऽब्रवीत्पृष्टो मान्धात्रा चक्र}

\uvacha{मान्धातोवाच}

\onelineshloka
{भगवञ्छोतुमिच्छामि लोकानां हितकाम्यया} %॥३॥

\twolineshloka
{चैत्रमास्यसिते पक्ष नामैकादशी भवेत्}
{को विधिः किं फलं तस्याः कथयस्व प्रसादतः} %॥४॥

\uvacha{लोमश उवाच}

\twolineshloka
{चैत्रमास्यसिते पक्षे नाम्ना वै पापमोचनी}
{एकादशी समाख्याता पिशाचत्वविनाशिनी} %॥५॥

\twolineshloka
{शृणु तस्याः प्रवक्ष्यामि क मदां सिद्धिदां नृप}
{कथां विचित्रां शुभदा पापी धर्मका} %॥६॥

\twolineshloka
{पुरा चैत्ररथोद्देशे अप्सरोगणसेविते}
{वसन्तसमये प्राते पुष्पैराकुलिते वने} %॥७॥

\twolineshloka
{गन्धर्वकन्यास्तत्रैव रमन्ति सह किन्नरैः}
{पाकशालनमुख्याश्च क्रीडन्ते च दिवौकसः} %॥८॥

\twolineshloka
{नापरं सुन्दर किश्चिदनाच्चैत्ररथानम्}
{तस्मिन्वने तु मुनयस्त पन्ति बहुलं तपः} %॥९॥

\twolineshloka
{वैस्तु मघवा रमते मधुमाधवौ}
{एको मुनिवरस्तव मेधावी नाम नामतः} %॥१०॥

\twolineshloka
{ॐ मुनिवरं मोहनायोपचक्रमे}
{मजुयोति विख्याना भावं तस्य विचिन्वती} %॥११॥

\twolineshloka
{क्रोशमात्र स्थिता तस्य भयदाश्रमसन्निधौ}
{गायन्ती मधुरं साधु पीडयन्ती विपश्चिकाम्} %॥१२॥

\twolineshloka
{गायानी तोमथालोक्य पुष्पचन्दनवेष्टिताम्}
{कामोऽपि विजयाकाङ्क्षी शिवभक्तं मुनीश्वरम्} %॥१३॥

\twolineshloka
{तस्याः शरीरसंसर्ग शिववैरमनुस्मरन्}
{कृत्वा भ्रुवौ धनुष्कोटी गुणं कृत्वा कटाक्षकम्} %॥१४॥

\twolineshloka
{मार्गणो नयने कृत्वा पक्षयुक्तौ यथाक्रमम्}
{कुचौ कृत्वा पटकुटी विजयायोपसंस्थितः} %॥१५॥

\twolineshloka
{मञ्जुघोषाभवत्तत्र कामस्येव वरूथिनी}
{मेधाविनं मुनिं दृष्ट्वा सापि कामेन पीडिता} %॥१६॥

\twolineshloka
{यौवनोदिनदेहोऽसौ मेधाव्यतिविराजते}
{सितोपवीतसहितो दण्डी स्मर इवापरः} %॥१७॥

\twolineshloka
{मञ्जुघोषा स्थिता तत्र दृष्ट्वा तं मुनिपुङ्गवम्}
{मदनस्य वशं प्राप्ता मन्दं मन्दमगायत} %॥१८॥

\twolineshloka
{रणद्रलयसंयुक्तां शिञ्जपुरमेखलाम्}
{गायन्ती भावसंयुक्तां विलोक्य मुनिपुङ्गवः} %॥१९॥

\twolineshloka
{मदनेन ससैन्येन नीतो मोहवशं बलात्}
{मञ्जुघोषा समागम्य मुनिं दृष्ट्वा तथाविधम्} %॥२०॥

\twolineshloka
{हावभावकटाक्षैस्तु मोहयामास चाङ्गना}
{अधः संस्थाप्य वीणां सा सस्वजे तं मुनीश्वरम्} %॥२१॥

\twolineshloka
{वल्लीवाकुलिता वृक्षं वातवेगेन वेपिता}
{सोऽपि रेमे तया सार्द्ध मेधावी मुनिपुङ्गवः} %॥२२॥

\twolineshloka
{तस्मिन्नेव वनोदेशे दृष्ट्वा तदेहमुत्तमम्}
{शिवतत्त्वं स विस्मृत्य कामतत्त्ववशं गतः} %॥२३॥

\twolineshloka
{न निशां न दिनं सोऽपि रमजानाति कामुकः}
{बहुलश्च गतः कालो मुनराचारलोपकः} %॥२४॥

\twolineshloka
{मजुघोषा देवलोकगमनायोपचक्रमे}
{गच्छन्ती प्रत्युवाचाथ रमन्तं मुनिपुङ्गवम्} %॥२५॥

\onelineshloka*
{आदशो दीयतां ब्रह्मन् स्वधामगमनाय मे}

\uvacha{मेधाव्युवाच}

\onelineshloka
{अव त्वं समायाता प्रदोषादौ वरानने} %॥२६॥

\twolineshloka
{यावत्प्रभातसन्ध्या स्यात्तावत्तिष्ठं ममान्तिके}
{इति श्रुत्वा मुने वाक्यं भयभीता बभूव सा} %॥२७॥

\twolineshloka
{पुनर्वै रमयामास तं मुनि नृपसत्तम}
{मुनिशापभयागीता बहुलान्परिवत्सरान्} %॥२८॥

\twolineshloka
{वर्षाणि सप्तपञ्चाशनवमासान् दिनत्रयम्}
{सा रेमे सुनिना तस्य निशार्द्धमिव चाभवत्} %॥२९॥

\twolineshloka
{सा तं पुनरुवाचाथ तस्मिन्काले गत मुनिम्}
{आदेशो दीयता ब्रह्मन गन्तव्यं स्वगृहे मया} %॥३०॥

\uvacha{मेधाव्युवाच}

\twolineshloka
{प्रातःकालोऽधुनेवारते श्रूयतां वचनं मम}
{कुर्वे सन्ध्यां दिनं यावत्तावत्त्वं सुस्थिरा भव} %॥३१॥

\twolineshloka
{इति वाक्यं मुनेः श्रुत्वा भयानन्दसमाकुलम्}
{स्मितं कृत्वा तु सा किश्चित्प्रत्युवाच सुविस्मिता} %॥३२॥

\uvacha{अप्सरा उवाच}

\twolineshloka
{कियत्प्रमाणा विप्रेन्द्र तव सन्ध्या गताः किल}
{मयि प्रसादं कृत्वा तु गतः कालो विचार्यताम्} %॥३३॥

\twolineshloka
{इति तस्या वचः श्रुत्वा विस्मयोत्फुल्ललोचनः}
{सध्यात्वा हृदि विप्रेन्द्रःप्रणाममकरोत्तदा} %॥३४॥

\twolineshloka
{समाश्च सप्तपञ्चाशद्गता मम तया सह}
{नेत्राभ्यां विस्फुल्लिङ्गान्स मुश्चमानोऽतिकोपनः} %॥३५॥

\twolineshloka
{कालरूपां च तां दृष्ट्वा तपसः क्षयकारिणीम्}
{दुःखार्जितं मम तपो नीतं तदनया क्षयम्} %॥३६॥

\twolineshloka
{विचार्येत्थं स कम्पोष्ठो मुनिस्तु व्याकुलेन्द्रियः}
{तां शशाप च मेधावी त्वं पिशाची भवेति हि} %॥३७॥

\twolineshloka
{धिक्त्वा पापे दुराचारे कुलटे पातकप्रिये}
{तस्य शापेन सा दग्धा विनयावनता स्थिता} %॥३८॥

\twolineshloka
{उवाच वचनं सुभूः प्रसादं वाञ्छती मुनिम्}
{कृत्वा प्रसादं विप्रेन्द्र शापस्योपशमं कुरु} %॥३९॥

\twolineshloka
{सतां सङ्गेहि भवति मित्रत्वं सप्तमे पद}
{त्वया सह मम ब्रह्मन् गताः सुबहवः समाः} %॥४०॥

\onelineshloka*
{एतस्मात्कारणात्स्वामिन् प्रसादं कुरु सुव्रत}

\uvacha{मुनिरुवाच}

\onelineshloka
{शृणु मे वचनं भद्रे शापानुग्रहकारकम्} %॥४१॥

\twolineshloka
{किं करोमि त्वया पापे क्षयं नीतं महत्तपः}
{चैत्रस्य कृष्णपक्षे या भवेदेकादशी शुभा} %॥४२॥

\twolineshloka
{पापमोचनिका नाम सर्वपापक्षयङ्करी}
{तस्या व्रते कृते सुभ्रु पिशाचत्वं प्रयास्यति} %॥४३॥

\twolineshloka
{इत्युक्त्वा तां स मधावी जगाम पितुराश्रमम्}
{तमागतं समालोक्य च्यवनःप्रत्युवाच ह} %॥४४॥

\onelineshloka*
{किमेतद्विहितं पुत्र त्वया पुण्यक्षयः कृतः}

\uvacha{मेधाव्युवाच}

\onelineshloka
{पापं कृतं महत्तात रमिता चाप्सरा मया} %॥४५॥

\onelineshloka*
{प्रायश्चित्तं ब्रूहि मम येन पापक्षयो भवेत्}

\uvacha{च्यवन उवाच}

\onelineshloka
{चैत्रस्य चासिते पक्ष नाम्रा वै पापमोचनी} %॥४६॥

\twolineshloka
{अस्या व्रते कृते पुत्र पापराशिः क्षयं व्रजेत्}
{इति श्रुत्वा पितुः वाक्यं कृतं तेन व्रतोत्तमम्} %॥४७॥

\twolineshloka
{गतं पापं क्षयं तस्य पुण्ययुक्तो बभूव सा}
{साप्येवं मञ्जुघोषा च कृत्वा तद्रूतमुत्तमम्} %॥४८॥

\twolineshloka
{पिशाचत्वविनिर्मुक्ता पापमोचनिकाव्रतात्}
{दिव्यरूपधरा भूत्वा गता नाकं वराप्सराः} %॥४९॥

\uvacha{लोमश उवाच}

\twolineshloka
{इत्थम्भूतप्रभावं हि पापमोचनिकाव्रतम्}
{पापमोचनिको राजन् ये कुर्वन्तीह मानवाः} %॥५०॥

\twolineshloka
{तेषां पापं च यत्किञ्चित्तत्सर्व क्षीणतां व्रजेत्}
{पठनाच्छ्रवणादस्या गोसहस्रफलं लभेत्} %॥५१॥


\threelineshloka
{ब्रह्महा हेमहारी च सुरापो गुरुतल्पगः}
{व्रतस्य चास्य करणात् पापमुक्ता भवन्ति ते}
{बहुपुण्यप्रदं ह्येतत्करणागतमुत्तमम्} %॥५२॥

॥इति श्रीभविष्यपुराणे पापमोचनिकाख्यचैत्रकृष्णैकादशीमाहात्म्यं सम्पूर्णम्॥


\hyperref[sec:ekadashi_mahatmyam_vrata_raja]{\closesub}
\clearpage

\sect{चैत्र-शुक्ल-कामदा-एकादशी-माहात्म्यम्}
\label{sec:vrata-raja-chaitra-shukla-kamada}

\uvacha{युधिष्ठिर उवाच}

\twolineshloka
{वासुदेव नमस्तुभ्यं कथयस्व ममाग्रतः}
{चैत्रस्य शुक्लपक्षे तु किन्नामकादशी भवेत्} %॥१॥

\uvacha{श्रीकृष्ण उवाच}

\twolineshloka
{शृणुष्वैकमना राजन् कथामेको पुरातनीम्}
{वसिष्ठो यामकथयत्प्राग्दिलीपाय पृच्छते} %॥२॥

\uvacha{दिलीप उवाच}

\twolineshloka
{भगकछोतुमिच्छामि कथयस्व प्रसा दतः}
{चैत्रे मासि सिते पक्षे किन्नामैकादशी भवेत्} %॥३॥

\uvacha{वसिष्ठ उवाच}

\twolineshloka
{साधु पृष्टं नृपश्रेष्ठ कथयामि तवाग्रतः}
{चत्रस्य शुक्लपक्षे तु कामदा नाम नामतः} %॥४॥

\twolineshloka
{एकादशी पुण्यतमा पापेन्धनदवानलः}
{शृणु राजन् कथामेतां पापन्नी पुत्रदायिनीम्} %॥५॥

\twolineshloka
{पुरा भोगिपुरे रम्ये हेमरत्नविभूषिते}
{पुण्डरीकमुखा नागा निवसान्त मदोत्कटाः} %॥६॥

\twolineshloka
{तस्मिन्पुरे पुण्डरीको राजा राज्यं करोति च}
{गन्धर्वैः किन्नरैश्चैव ह्यप्सरोभिः स सेव्यते} %॥७॥

\twolineshloka
{वराप्सरा तु ललिता गन्धर्वो ललितस्तथा}
{उभौ रागेण संयुक्तौ दम्पती कामपीडितौ} %॥८॥

\twolineshloka
{रेमाते स्वगृहे रम्ये धनधान्ययुते सदा}
{ललितायास्तु हृदये पतिर्वसति सर्वदा} %॥९॥

\twolineshloka
{हृदयें तस्य ललिता नित्यं वसति भामिनी}
{एकदा पुण्डरीकाद्याः क्रीडन्तः सदसि स्थिताः} %॥१०॥

\twolineshloka
{गीतगानं प्रकुरुते ललितो दयितां विना}
{पदबन्धे स्खलजिह्वो बभूव ललितां स्मरन्} %॥११॥

\twolineshloka
{मनोभावं विदित्वाऽस्य कोटो नागसत्तमः}
{पदबन्धच्युतिं तस्य पुण्डरीके न्यवेदयत्} %॥१२॥

\twolineshloka
{क्रोधसंरक्तनयनः पुण्डरीकोऽभवत्तदा}
{शशाप ललितं तत्र मन्दनातुरचेतसम्} %॥१३॥

\twolineshloka
{राक्षसो भव दुर्बुद्धे क्रव्यादः पुरुषादकः}
{यतः पत्नीवशो जातो गायश्चैव ममाग्रतः} %॥१४॥

\twolineshloka
{वचनात्तस्य राजेन्द्र रक्षोरूपो बभूव ह}
{रौद्राननो विरूपाक्षो दृष्टमात्रो भयङ्करः} %॥१५॥

\twolineshloka
{बाहू योजनविस्तीर्णी मुखकन्दरसन्निभम्}
{चन्द्रसूर्यनिभे नेत्रे ग्रीवा पर्वतसन्निभा} %॥१६॥

\twolineshloka
{नासारन्ध्रे तु विवरे चाधरौ योजनार्द्धकौ}
{शरीरं तस्य राजेन्द्र उत्थितं योजनाष्टकम्} %॥१७॥

\twolineshloka
{ईदृशो राक्षसः सोऽभूद्भुञ्जानः कर्मणः फलम्}
{ललिता तमथालोक्य स्वपति विकृताकृतिम्} %॥१८॥

\twolineshloka
{चिन्तयामास मनसा दुःखेन महतार्दिता}
{किं करोमि व गच्छामि पतिः शापेन पीडितः} %॥१९॥

\twolineshloka
{इति संस्मृत्य मनसा न शर्म लभते तु सा}
{चचार पतिना सार्द्ध ललिता गहने वन} %॥२०॥

\twolineshloka
{बभ्राम विपिने दुर्गे कामरूपः स राक्षसः}
{निर्वृणः पापनिरतो विरूपः पुरुषदकः} %॥२१॥

\twolineshloka
{न सुखं लभते रात्रौ न दिवा तापपीडितः}
{ललिता दुःखितातीव पतिं दृष्ट्वा तथाविधम्} %॥२२॥

\twolineshloka
{भ्रमन्ती तेन सार्द्ध सा रुदती गहने वने}
{कदाचिदगमद्विन्ध्यशिखरे बहुकौतुके} %॥२३॥

\twolineshloka
{ऋष्यशृङ्गमुनेस्तत्र दृष्ट्वाश्रमपदं शुभम्}
{शीघ्रं जगाम ललिता विनयावनता स्थिता} %॥२४॥

\twolineshloka
{प्रत्युवाच मुनिदृष्ट्वा का त्वं कस्य सुता शुभम्}
{किमर्थ त्वमिहायाता सत्यं वद ममाप्रतः} %॥२६॥

\uvacha{ललितोवाच}

\twolineshloka
{वीरधन्तेति गन्धर्वः सुतां तस्य महात्मनः}
{ललितां नाम मां विद्धि पत्यर्थमिह चागताम्} %॥२६॥

\twolineshloka
{भर्ता में शापदोषेण राक्षसोऽभून्महामुने}
{रौद्ररूपो दुराचारस्तं दृष्ट्वा नास्ति मे सुखम्} %॥२७॥

\twolineshloka
{साम्प्रतं शाधि मां ब्रह्मन् प्रायश्चित्तं करोमि तत्}
{येन पुण्येन मे भर्ता राक्षसत्वाद्विमुच्यते} %॥२८॥

\uvacha{ऋषिरुवाच}

\twolineshloka
{चैत्रमासस्य रम्भोरु शुक्लपक्षेऽस्ति साम्प्रतम्}
{कामदैकादशी नाम्ना या कृता कामदा नृणाम्} %॥२९॥

\twolineshloka
{कुरुष्व तद्वतं भद्रे विधिपूर्व मयोदितम्}
{तस्य व्रतस्य यत्पुग्यं तत्स्वभत्रै प्रदीयताम्} %॥३०॥

\twolineshloka
{दत्ते पुण्ये क्षणात्तस्य शापदोषः प्रशाम्यति}
{इति श्रुत्वा मुनेर्वाक्यं ललिता हर्विपाभवत्} %॥३१॥

\twolineshloka
{उपोष्यैकादशी राजन्द्वादशी दिवसे तदा}
{विप्रस्यैव समीपे तु वासुदेवाग्रतः स्थिता} %॥३२॥

\twolineshloka
{वाक्यमूचे तु ललिता स्वपत्युत्तारणाय वै}
{मया तु यद्वतं चीर्ण कामदाया उपोषणम्} %॥३३॥

\twolineshloka
{तस्य पुण्यप्रभावेण गच्छत्वस्य पिशाचता}
{ललितावचनादेवं वर्तमानोपि तत्क्षगे} %॥३४॥

\twolineshloka
{गतपापः सललितो दिव्य देहो बभूव ह}
{राक्षसत्वं गतं तस्य प्राप्तो गन्धर्षतां पुनः} %॥३५॥

\twolineshloka
{हेमरत्नसमाकीणों रेमे ललितया सह}
{तौ विधानं समारूढौ पूर्वरूपाधिकावुभौ} %॥३६॥

\twolineshloka
{दम्पती चापि शोभेता कामदायाः प्रभावतः}
{इति ज्ञात्वा नृपश्रेष्ठ कर्तव्यैषा प्रयत्नतः} %॥३७॥

\twolineshloka
{लोकानां च हितार्थाय तबाने कथिता मया}
{ब्रह्महत्यादिपापनी पिशाचत्वविनाशिनी} %॥३८॥

\twolineshloka
{नातः परतरा काचित्रैलोक्ये सचराचरे}
{पठनाच्छ्रवणाद्वापि वाजपेयफलं लभेत्} %॥३९॥

॥इति श्रीवाराहपुराणे कामदा नाम चैत्रशुक्लैकादशीमाहात्म्यं सम्पूर्णम्॥


\hyperref[sec:ekadashi_mahatmyam_vrata_raja]{\closesub}
\clearpage

\sect{वैशाख-कृष्ण-वरूथिनी-एकादशी-माहात्म्यम्}
\label{sec:vrata-raja-vaishakha-krishna-varuthini}

\uvacha{युधिष्ठिर उवाच}

\twolineshloka
{वैशाखस्यासिते पक्षे किन्नामैकादशी भवेत्}
{महिमानं कथय मे वासुदेव नमोऽस्तु ते} %॥१॥

\uvacha{श्रीकृष्ण उवाच}

\twolineshloka
{सौभाग्यदायिनी राजनिह लोके परत्र च}
{वैशाखकृष्णपक्षे तु नाम्ना चैव वरूथिनी} %॥२॥

\twolineshloka
{वरूथिन्या व्रतेनैव सौख्यं भवति सर्वदा}
{पापहानिश्च भवति सौभाग्यप्राप्तिरेव च} %॥३॥

\twolineshloka
{दुर्भगा या करोत्येना सा स्त्री सौभाग्यमाप्नुयात्}
{लोकानां चैव सर्वेषां भुक्तिमुक्तिप्रदायिनी} %॥४॥

\twolineshloka
{सर्वपापहरा नृणां गर्भवासनिकृन्तनी}
{वरूथिन्या व्रतेनैव मान्धाता स्वर्गतिं गतः} %॥५॥

\twolineshloka
{धुन्धुमारादयश्चान्ये राजानो बहवस्तथा}
{ब्रह्मकपालनिर्मुक्तो बभूव भगवान्भवः} %॥६॥

\twolineshloka
{दशवर्षसहस्राणि तपस्तप्यति यो नरः}
{तचुल्यं फलमाप्नोति वरूथिन्या व्रतादपि} %॥७॥

\twolineshloka
{कुरुक्षेत्रे रविग्रहे स्वर्णभारं ददाति यः}
{तत्तुल्यं फलमाप्नोति वरूथिन्या व्रतान्नरः} %॥८॥

\twolineshloka
{श्रद्धावान्यस्तु कुरुते वरूथिन्या व्रतं नरः}
{वाञ्छितं लभते सोऽपि इह लोके परत्र च} %॥९॥

\twolineshloka
{पवित्रा पावनी ह्येषा महापातकनाशिनी}
{भुक्तिमुक्तिप्रदा चापि कर्मणां नृपसत्तम} %॥१०॥

\twolineshloka
{अश्वदानान्नृपश्रेष्ठ गजदानं विशिष्यते}
{गजदानाद्भूमिदानं तिलदानं ततोऽधिकम्} %॥११॥

\twolineshloka
{ततः सुवर्णदानं तु अन्नदानं ततोऽधिकम्}
{अन्नदानात्पर दानं न भूतं न भविष्यति} %॥१२॥

\twolineshloka
{पितृदेवमनुष्याणां तृप्तिरनेन जायते}
{तत्सम कविभिः प्रोक्तं कन्यादानं नृपोत्तम} %॥१३॥

\twolineshloka
{धेनुदानं च तत्तुल्यमित्याह भगवान् स्वयम्}
{प्रोक्तेभ्यः सर्वदानेभ्यो विद्यादानं विशिष्यते} %॥१४॥

\twolineshloka
{तत्फलं समवाप्नोति नरः कृत्वा वरूथिनीम्}
{कन्यावित्तेन जीवन्ति ये नराः पापमोहिताः} %॥१५॥

\twolineshloka
{ते नरा नरकं यान्ति यावदाभूतसम्प्लवम्}
{तस्मात्सर्वप्रयत्नेन न ग्राह्यं कन्यकाधनम्} %॥१६॥

\twolineshloka
{यञ्च गृहाति लोभेन कन्या क्रीत्वा च तद्धनम्}
{'सोन्यजन्मनि राजेन्द्र ओतुर्भवति निश्चि} %॥१७॥

\twolineshloka
{कन्यां वित्तेन यो दद्याद्यथाशक्ति स्वलङ्कृताम्}
{तत्पुण्यसङ्ख्यां कर्तुं हि चित्रगुप्तो न वेत्यलम्} %॥१८॥


\threelineshloka
{तत्फलं समवाप्नोति नरः कृत्वा वरूथिनीम्}
{कांस्यं मांसं मसूरानं चणकान् कोद्रवांस्तथा}
{शाकं मधु परान्नं च पुनर्भोजनमैथुने} %॥१९॥

\twolineshloka
{वैष्णवव्रतकर्ता च दशम्यां दश वर्जयेत्}
{द्यूतक्रीडां च निद्रां च ताम्बूलं दन्तधावनम्} %॥२०॥

\twolineshloka
{परापवादं पैशुन्यं पतितैः सह भाषणम्}
{क्रोध चैवानृतं वाक्यमेकादश्यां विवर्जयेत्} %॥२१॥

\twolineshloka
{कांस्यं मांसं मसूरांश्च क्षौद्रं वितथभाषणम्}
{व्यायामश्च प्रयासं च पुनर्भोजनमैथुने} %॥२२॥


\threelineshloka
{क्षौर तैलं परानं च द्वादश्यां परिवर्जयेत्}
{अनेन विधिना राजन्विहिता यैर्वरूथिनी}
{सर्वपापक्षयं कृत्वा दद्यात्प्रान्तेऽक्षयां गतिम्} %॥२३॥

\twolineshloka
{रात्री जागरणं कृत्वा पूजितो यर्जनार्दनः}
{सर्वपापविनिर्मुक्तास्ते यान्ति परमो गतिम्} %॥२४॥

\twolineshloka
{तस्मात्सर्वप्रयत्नेन कर्तव्या पापभीरुभिः}
{क्षपारितनयादीतैर्नरदेव वरूथिनीम्} %॥२५॥

\twolineshloka
{पठनाच्छ्रवणाद्राजन् गोसहस्रकलं लभेत्}
{सर्वपापविनिर्मुक्तो विष्णुलोके महीयते} %॥२६॥

॥इति श्रीभविष्यपुराणे वैशाखकृष्णैकादश्या वरूथिन्याख्याया माहात्म्यं सम्पूर्णम्॥


\hyperref[sec:ekadashi_mahatmyam_vrata_raja]{\closesub}
\clearpage

\sect{वैशाख-शुक्ल-मोहिनी-एकादशी-माहात्म्यम्}
\label{sec:vrata-raja-vaishakha-shukla-mohini}

\uvacha{युधिष्ठिर उवाच}

\twolineshloka
{वैशाखशुक्लपक्षे तु किन्नामैकादशी भवेत्}
{किं फलं को विधिस्तस्याः कथयस्व जनार्दन} %॥१॥

\uvacha{श्रीकृष्ण उवाच}

\twolineshloka
{कथयामि कथामेतां शृणु त्वं धर्मनन्दन}
{वसिष्ठो यामकथयत्पुरा रामाय पृच्छते} %॥२॥

\uvacha{राम उवाच}

\twolineshloka
{भगवन् श्रोतुमिच्छामि व्रतानामुत्तमं व्रत}
{सर्वपापक्षयकरं सर्वदुःखनिकृन्तनम्} %॥३॥

\twolineshloka
{मया दुःखानि भुक्तानि सीताविरहजानि व}
{ततोऽहं भयभीतोऽस्मि पृच्छामि त्वां महामुने} %॥४॥

\uvacha{वसिष्ठ उवाच}

\twolineshloka
{साधु पृष्टं त्वया राम तवैषा नैष्ठिकी मतिः}
{त्वन्नामग्रहणेनैव पूतो भवति मानवः} %॥५॥

\twolineshloka
{तथापि कथयिष्यामि लोकानां हितकाम्यया}
{पवित्रं प्रावनानां च व्रतानामुत्तमं व्रतम्} %॥६॥

\twolineshloka
{वैशाखस्य सिते पक्षे द्वादशी राम या भवेत्}
{मोहिनीनाम सा प्रोक्ता सर्व पापहरा परा} %॥७॥

\twolineshloka
{मोहजालात्प्रमुच्येत पातकानां समूहन्तः}
{अस्या व्रतप्रभावेण सत्यं सत्यं वदाम्यहम्} %॥८॥

\twolineshloka
{अतस्तु कारणाद्राम कर्तव्यैषा भवादशैः}
{पातकानां क्षयकरी महादुःखविनाशिनी} %॥९॥

\twolineshloka
{शृणुष्वैकमना राम कथां पुण्यप्रदां शुभाम्}
{यस्याः श्रवणमात्रण महापापं प्रणश्यति} %॥१०॥

\twolineshloka
{सरस्वत्यास्तटे रम्ये पुरी भद्रावती शुभा}
{द्युतिमानाम नृपतिस्तत्र राज्यं करोति वै} %॥११॥

\twolineshloka
{सोमवंशोद्भवो राम धृतिमान्सत्यसङ्गरः}
{तत्र वैश्यो निवसति धनधान्यसमृद्धिमान्} %॥१२॥

\twolineshloka
{धनपाल इति ख्यात पुण्यकर्मप्रवर्तकः}
{प्रपासवाद्यायतनतडागारामकारकः} %॥१३॥

\twolineshloka
{विष्णुभक्तिपरः शान्तस्तस्यांसपञ्चपुत्रकाः}
{सुमना द्युतिमांश्चैव मेधावी सुकृती तथा} %॥१४॥

\twolineshloka
{पञ्चमो धृष्टबुद्धिश्च महापापरतः सदा}
{वारस्त्रीसङ्गनिरतो विटगोष्ठीविशारदः} %॥१५॥

{द्यूतादिव्यसनासक्तः परस्त्रीरतिलालसः}
{न देवांश्चातिथीन्वृद्धान्पितॄश्चा विजानपि} %॥१६॥

\twolineshloka
{अन्यायकर्ता दुष्टात्मा पितृद्रव्यक्षयङ्करः}
{अभक्ष्यभक्षकः पाप सुरापानरतः सदा} %॥१७॥

\twolineshloka
{वेश्याकण्ठक्षिप्तवाहुर्धमदृष्टिश्चतुष्पथे}
{पित्रा निष्कासितो गेहात्परित्यक्तश्च बान्धवैः} %॥१८॥

\twolineshloka
{स्वदेहभूषणान्येवं क्षयं नीतानि तेन वै}
{मणिकाभिः परित्यक्तो निन्दितश्च धनक्षयात्} %॥१९॥

\twolineshloka
{ततश्चिन्तापरो जातो वस्त्रहीनः क्षुधार्दितः}
{किं करोमि क गच्छानि केनोपायेन जीव्यते} %॥२०॥

\twolineshloka
{तस्करत्वं समारब्धं तत्रैव नगरे ततः}
{गृहीतो राजपुरुषैर्मुक्तश्च पितृगौरवात्} %॥२१॥

\twolineshloka
{पुनर्बद्धः पुनर्मुक्तः पुनर्मुक्तः स वै भटैः}
{धृष्टबुद्धिर्दुराचारो निबद्धो निगडैद्वैः} %॥२२॥

\twolineshloka
{कशाघातस्ताडितश्च पीडितश्च पुनः पुनः}
{न स्थातव्यं हि मन्दात्मंस्त्वया मद्देश गोचरे} %॥२३॥

\twolineshloka
{एवमुक्त्वा ततो राज्ञा मोचितो दृढबन्धनात्}
{निर्जगाम भयात्तस्य गतोऽसौ गहनं वनम्} %॥२४॥

\twolineshloka
{क्षुत्तृषापीडितश्चायमितश्चेतश्च धावति}
{सिंहवनिजघानासौ मृगसूकरचित्तलान्} %॥२५॥

\twolineshloka
{आमिषाहारनिरतो वने तिष्ठति सर्वदा}
{शरासने शरं कृत्वा निषङ्गं पृष्ठसङ्गतम्} %॥२६॥

\twolineshloka
{अरण्यचारिणो हन्ति दक्षिणश्च चतुष्पदान्}
{चकोरांश्च मयूरांश्च कास्तित्तिरिमूषकान्} %॥२७॥

\twolineshloka
{एतानन्यान् हन्ति नित्यं धृष्टबुद्धिः स निर्वृणः}
{पूर्वजन्मकृतैः पापैनिमग्नः पापकर्दमे} %॥२८॥

\twolineshloka
{दुःखशोकसमाविष्टश्चिन्तयन् सोऽप्यहर्निशम्}
{कौण्डिन्यस्याश्रमं प्राप्तः कस्माञ्चित्पुण्यगौरवात्} %॥२९॥

\twolineshloka
{माधवे मासि जाहल्या कृतस्नानं तपोधनम्}
{आससाद धृष्टबुद्धिः शोकभारेण पीडितः} %॥३०॥

\twolineshloka
{तद्वस्त्रविन्दुस्पर्शन गतपाप्मा हताशुभः}
{कौण्डिन्यस्याप्रतः स्थित्वा प्रत्युवाच कृताञ्जलिः} %॥३१॥

\uvacha{धृष्टबुद्धिरुवाच}

\twolineshloka
{प्रायश्चित्तं वद ब्रह्मन्विना वित्तेन यद्भवेत्}
{आजन्मकृतपापस्य नास्ति वित्तं ममाधुना} %॥३२॥

\uvacha{ऋषिरुवाच}

\twolineshloka
{शृणुष्वकमना भूत्वा येन पापक्षयस्तव}
{वैशाखस्य सिते पक्षे मोहिनी नाम नामतः} %॥३३॥

\twolineshloka
{एकादशीवतं तस्याः कुरु मद्वाक्यनोदितः}
{मेरुतुल्यानि पापानि क्षयं नयति देहिनाम्} %॥३४॥

\twolineshloka
{बहुजमार्जितान्येषा मोहिनी समुपोषिता}
{इति वाक्यं मुनेः श्रुत्वा धृष्टबुद्धिः प्रसन्नहृत्} %॥३५॥

\twolineshloka
{व्रतं चकार विधिवत्कौण्डिन्यस्योपदेशतः}
{कृते व्रते नृपश्रेष्ठ हतपापो बभूव सः} %॥३६॥

\twolineshloka
{दिव्यदेहस्ततो भूत्वा गरुडोपरि संस्थितः}
{जगाम वैष्णवं लोकं सर्वोपद्रववर्जितम्} %॥३७॥

\twolineshloka
{इतीदृशं रामचन्द्र तमोमोहनिकृन्तनम्}
{नातः परतरं किश्चित्रैलोक्ये सचराचरे} %॥३८॥

\twolineshloka
{यज्ञादितीर्थदानानि कलां नाहन्ति षोडशीम्}
{पठनाच्छवणाद्राजन् गोसहस्रफलं लभेत्} %॥३९॥

॥इति श्रीकूर्मपुराणे मोहिन्याख्यवैशाखशुक्लैकादशीमाहात्म्यं सम्पूर्णम्॥


\hyperref[sec:ekadashi_mahatmyam_vrata_raja]{\closesub}
\clearpage

\sect{ज्येष्ठ-कृष्णापरा-एकादशी-माहात्म्यम्}
\label{sec:vrata-raja-jyeshtha-krishnapara}

\uvacha{युधिष्ठिर उवाच}

\twolineshloka
{ज्येष्ठस्य कृष्णपक्षे तु किन्नामैकादशी भवेत्}
{श्रोतुमिच्छामि माहात्म्य तद्वदस्व जनार्दन} %॥१॥

\uvacha{श्रीकृष्ण उवाच}

\twolineshloka
{साधु पृष्टं त्वया राजैल्लोकानां हितकाम्यया}
{बहुपुण्यप्रदा ह्येषा महापातकनाशिनी} %॥२॥

\twolineshloka
{अपरा नाम राजेन्द्र अपारफलदायिनी}
{लोक प्रसिद्धतां याति अपरां यस्तु सेवते} %॥३॥

\twolineshloka
{ब्रह्महत्याभिभूतोऽपि गोत्रहा भ्रूणहा तथा}
{परापवादवादी च परस्त्रीरसिकोपि च} %॥४॥

\twolineshloka
{अपरासेवनाद्राजन्विपाप्मा भवति ध्रुवम्}
{कूटसाक्ष्य मानकूटं तुलाकूटं करोति यः} %॥५॥

\twolineshloka
{कूटवेदं पठेद्विप्रः कूटशास्त्रं तथैव च}
{ज्योतिषी कूटगणकः कूटायुर्वेदको भिषक्} %॥६॥

\twolineshloka
{कूट साक्षिसमा ह्येते विज्ञेया नरकौकसः}
{अपरासेवनादाजन् पापमुक्ता भवन्ति ते} %॥७॥

\twolineshloka
{क्षत्रियः क्षात्रधर्म यस्त्यक्त्वा युद्धात्पलायते}
{स याति नरकं घोरं स्वीयधर्मबहिष्कृतः} %॥८॥

\twolineshloka
{अपरासेवनात्सोपि पापं त्यक्त्वा दिवं व्रजेत्}
{विद्यामधीत्य यः शिष्यो गुरुनिन्दा करोति च} %॥९॥

\twolineshloka
{महापातकसंयुक्तो निरयं याति दारुणम्}
{अपरासेवनात्सोपि सद्गतिं प्राप्नुयानरः} %॥१०॥

\twolineshloka
{पुष्करत्रितये स्नात्वा कार्तिक्यां यत्फलं लभेत्}
{मकरस्थे रवौ माघे प्रयागे यत्फलं नृणाम्} %॥११॥

\twolineshloka
{काश्यां यत्प्राप्यते पुण्यं शिवरारुपोषणात्}
{गयायां पिण्डदानेन यत्फलं प्राप्यते नृभिः} %॥१२॥

\twolineshloka
{सिंहस्थिते देवगुरौ गौतमीस्नानतो नरः}
{यत्फलं समवाप्नोति कुम्भे केदारदर्शनात्} %॥१३॥

\twolineshloka
{बदर्याश्रमयात्रायास्तत्तीर्थसेवनादपि}
{यत्फलं समवाप्नोति कुरुक्षेत्रे रविग्रहे} %॥१४॥

\twolineshloka
{गजाश्वहेमदानेन यज्ञे कृत्स्नसुवर्णदः}
{तत्फलं समवाप्नोति अपराया व्रतान्नरः} %॥१५॥

\twolineshloka
{अर्धप्रसूतां गां दत्त्वा सुवर्ण वसुधां तथा}
{नरो यत्फलमानोति अपराया व्रतेन तत्} %॥१६॥

\twolineshloka
{पापद्रुमकुठारोऽयं पापेन्धनदवानलः}
{पापान्धकारसूर्योऽयं पापसारङ्गकेसरी} %॥१७॥

\twolineshloka
{अपरैकादशी राजन् कर्तव्या पापभीरुभिः}
{बुदबुदा इव तोयेषु पुत्तिका इव जन्तुषु} %॥१८॥

\twolineshloka
{जायन्ते मरणायैव एकादश्या व्रतं विना}
{अपरां समुपोष्यैव पूजयित्वा त्रिविक्रमम्} %॥१९॥


\threelineshloka
{सर्वपापविनिर्मुक्तो विष्णुलोकं व्रजेन्नरः}
{लोकानां च हितार्थाय तवाने कथितं मया}
{पठनाच्छवणाद्राजन् सर्वपापैः प्रमुच्यते} %॥२०॥

॥इति ब्रह्माण्डपुराणे ज्येष्ठकृष्णापराख्यकादशीमाहात्म्यं सम्पूर्णम्॥


\hyperref[sec:ekadashi_mahatmyam_vrata_raja]{\closesub}
\clearpage

\sect{ज्येष्ठ-शुक्ल-निर्जला-एकादशी-माहात्म्यम्}
\label{sec:vrata-raja-jyeshtha-shukla-nirjala}

\uvacha{भीमसेन उवाच}

\twolineshloka
{पितामह महाबुद्धे शृणु मे परमं वचः}
{युधिष्ठिरश्च कुन्ती च तथा द्रुपद नन्दिनी} %॥१॥

\twolineshloka
{अर्जुनो नकुलश्चैव सहदेवस्तथैव च}
{एकादश्यां न भुनन्ति कदाचिदपि सुव्रत} %॥२॥

\twolineshloka
{ते मां ब्रुवन्ति वै नित्यं मा भुक्ष्व त्वं वृकोदर}
{अहं तानत्रुवं तात बुभुक्षा दुःसहा मम} %॥३॥

\twolineshloka
{दानं दास्यामि विधिवत्पूजयिष्यामि केशवम्}
{विनोपवासं लभ्येत कथमेकादशीव्रतम्} %॥४॥

\onelineshloka*
{भीमसेनवचः श्रुत्वा व्यासो वचनमब्रवीत्}

\uvacha{व्यास उवाच}

\onelineshloka
{यदि स्वर्गोत्यभीष्टस्ते नरकोनिष्ट एव च} %॥५॥

\onelineshloka*
{एकादश्यां न भोक्तव्यं पक्षयोरुभयोरपि}

\uvacha{भीमसेन उवाच}

\onelineshloka
{पितामह महाबुद्धे कथयामि तवाग्रतः} %॥६॥

\twolineshloka
{एकभक्तं न शक्तोऽहमुपवासः कुतो मुने}
{वृको नामास्ति यो वन्द्विः स सदा जठरे मम} %॥७॥

\twolineshloka
{अतीवानं यदानामि तदा समुपशाम्यति}
{एकं शक्तोस्म्यहं कर्तु चोपवासं महामुने} %॥८॥

\onelineshloka*
{तदेकं वद निश्चित्य येन श्रेयोइमाप्नुयाम्}

\uvacha{व्यास उवाच}

\onelineshloka
{श्रुतास्ते मानवा धर्मा वैदिकाश्च श्रुतास्त्वया} %॥९॥

\twolineshloka
{कलौ युगे न शक्यन्ते ते वै कर्तु नराधिप}
{सुखोपायं चाल्पधनमल्पक्लेशं महाफलम्} %॥१०॥

\twolineshloka
{पुराणानां च सर्वेषां सारभूतं वदामि ते}
{एकादश्यां न भुञ्जीत पक्षयोरुभयोरपि} %॥११॥

\twolineshloka
{एकादश्या न भुङ्क्त यो न याति नरकं तु सः}
{व्यासस्य वचनं श्रुत्वा कम्पितोऽश्वत्थपत्रवत्} %॥१२॥

\onelineshloka*
{भीमसेनो महाबाहुर्भातो वाक्यमभाषत}

\uvacha{भीमसेन उवाच}

\onelineshloka
{पितामह न शक्तोऽहमुपवासे करोमि किम्} %॥१३॥

\onelineshloka*
{ततो बहुफलं ब्रूहि व्रतमेक मम प्रभो}

\uvacha{व्यास उवाच}

\onelineshloka
{वृषस्थ मिथुनस्थे के शुक्का यैकादशी भवेत्} %॥१४॥

\twolineshloka
{ज्येष्ठमासे प्रयत्नेन सोपोष्या जलवर्जिता}
{स्नाने चाचमने चैव वर्जयित्वोदकं बुधः} %॥१५॥

\twolineshloka
{उपयुञ्जीत नैवान्यगतभङ्गोऽन्यथा भवेत}
{उदयादुदयं यावद्वर्जयित्वा जलं बुधः} %॥१६॥

\twolineshloka
{अप्रयत्नादवाप्नोति द्वादशद्वादशीफलम्}
{ततः प्रभाते विमले द्वादश्यां स्नानमाचरेत्} %॥१७॥

\twolineshloka
{जलं सुवर्ण दत्त्वा च द्विजातिभ्यो यथाविधि}
{भुञ्जीत कृतकृत्यस्तु ब्राह्मणैः सहितो वशी} %॥१८॥

\twolineshloka
{एवं कृते तु यत्पुण्यं भीमसेन शृणुष्व तत्}
{संवत्स रस्य या मध्ये एकादश्यो भवन्ति वै} %॥१९॥

\twolineshloka
{तासां फलमवाप्नोति अत्र मे नास्ति संशयः}
{इति मां केशवः प्राह शङ्खचक्रगदाधरः} %॥२०॥

\twolineshloka
{एकादश्यां सिते पक्षे ज्येष्ठस्यौदकवर्जितम्}
{उपोष्य फलमाप्नोति तच्छृणुष्व वृतोदर} %॥२१॥

\twolineshloka
{सर्वतीर्थेषु यत्पुण्यं सर्वदानेषु यत्फलम्}
{यत्फेलं समवाप्नोति इमां कृत्वा वृकोदर} %॥२२॥

\twolineshloka
{संवत्सरस्य यावन्त्यः शुक्लाः कृष्णा वृकोदर}
{उपोपितास्ताः सर्वाः स्युरेकादश्यो न संशयः} %॥२३॥

\twolineshloka
{धनधान्यवहाः पुण्याः पुत्रारोग्यफलप्रदाः}
{उपोषिता नरव्याघ्र इति सत्यं वदामि ते} %॥२४॥

\twolineshloka
{यमदूता महाकायाः करालाः कृष्णपिङ्गलाः}
{दण्डपाशधरा रौद्रा मरणे दृष्टिगोचरम्} %॥२५॥

\twolineshloka
{न प्रयान्ति नरव्याघ्र एकादश्यामुपोषणात्}
{पीताम्बरधराः सौम्याश्चक्रहस्ता मनोजवाः} %॥२६॥

\twolineshloka
{अन्नकाले नयन्त्येव मानवं वैष्णवीं पुरीम्}
{तस्मात्सर्वप्रयत्नेन सोपोष्योदकवर्जिता} %॥२७॥

\twolineshloka
{जलधेनुं ततो दत्त्वा सर्वपापैः प्रमुच्यते}
{इति श्रुत्वा तदा चकुः पाण्डवा जनमेजय} %॥२८॥

\twolineshloka
{ततःप्रभृति भीमेन कृतेयं निर्जला शुभा}
{पाण्डवद्वादशोनाम्ना लोके ख्याता बभूव ह} %॥२९॥

\twolineshloka
{तथा त्वमपि भूपाल सोपवासार्चनं हरेः}
{कुरु त्वं च प्रयत्नेन सर्वपापप्रशान्तये} %॥३०॥

\twolineshloka
{करिष्याम्यद्य देवेश जलवर्जमुपोषणम्}
{भोक्ष्ये परेऽति देवेश ह्यन्नं च तव वासरात्} %॥३१॥

\twolineshloka
{इत्युच्चार्य ततो मन्त्रमुपवासपरो भवेत्}
{सर्वपापविनाशाय श्रद्धादमसमन्वितः} %॥३२॥

\twolineshloka
{मेरुमन्दरमानं तु त्रियाथ पुरुषस्य यत्}
{पापं तद्भस्मतां याति एकादश्याः प्रभावतः} %॥३३॥

\twolineshloka
{न शक्रोति च यो दातुं जलधेतुं नराधिप}
{सकाश्चनो घटस्तेन देयो वस्त्रेण संवृतः} %॥३४॥

\twolineshloka
{तोयस्य नियमं योस्यां कुरुते व पुण्यभाक्}
{पलकोटिनुवर्गस्य यामेयामेऽश्नुते फलम्} %॥३५॥

\twolineshloka
{स्नानं दानं जपं होमं यदस्यां कुरुते नरः}
{तत्सर्व चाक्षयं प्रोक्तमेतत्कृष्णस्य भाषितम्} %॥३६॥

\twolineshloka
{किं वापरण धर्मेण निर्जलैकादशीं नृप}
{उपोष्य च नरो भक्त्या वैष्णवं पदमाप्नुयात्} %॥३७॥

\twolineshloka
{सुवर्णमन्नं वासांसि यदस्यां सम्प्रदीयते}
{तदस्य च कुरुश्रेष्ठ सर्वमप्यक्षयं भवेत्} %॥३८॥

\twolineshloka
{एकादशीदिने योऽनं भुक्ते पापं भुनक्ति सः}
{इह लोके स चाण्डालो मृतः प्राप्नोति दुर्गतिम्} %॥३९॥

\twolineshloka
{ये प्रदास्यन्ति दानानि द्वादशी समुपोष्य च}
{ज्येष्ठे मासि सिते पक्षे प्राप्स्यन्ति परमं पदम्} %॥४०॥

\twolineshloka
{ब्रह्महा मद्यपः स्तेनो गुरुद्वेष्टा सदाऽनृती}
{मुच्यन्ते पातकै सर्वैर्निर्जला यरुपोषिता} %॥४१॥

\twolineshloka
{विशेषं शृणु राजेन्द्र निर्जलैकादशीदिने}
{यत्कर्तव्यं नरैः स्त्रीभिः श्रद्धादमसमन्वितैः} %॥४२॥

\twolineshloka
{जलशायी तु सम्पूज्यो देया धेनुश्च तन्मयी}
{प्रत्यक्षा वा नृपश्रेष्ठ घृतधेनुरथापि वा} %॥४३॥

\twolineshloka
{दक्षिणाभिश्च श्रेष्ठाभिदृष्टान्नैश्च पृथग्विधैः}
{तोषणीया प्रयत्नेन द्विजा धर्मभृतां वर} %॥४४॥

\twolineshloka
{तुष्टो भवति वै क्षिप्रं तैस्तुष्टैर्मोक्षदो हरिः}
{आत्मद्रोहः कृतस्तैस्तु यैषा समुपोषिता} %॥४५॥

\twolineshloka
{पापात्मानो दुराचारा दुष्टास्ते नात्र संशयः}
{कुलानां च शतं सायमनाचाररतं सदा} %॥४६॥

\twolineshloka
{आत्मना सह तैनीतं वासुदेवस्य मन्दिरम्}
{शान्तैनपरैश्चैव अर्चद्भिश्च तथा हरिम्} %॥७॥

\twolineshloka
{कुर्वद्भिर्जागरं रात्रौ यैरेषा समुपोषिता}
{अन्नं पानं तथा गावो वस्त्रं शय्यासनं शुभम्} %॥४८॥

\twolineshloka
{कमण्डलु तथा छत्रं दातव्यं निर्जलादिन}
{उपानहीं च यो दद्यात्पात्रभूते द्विजोत्तमे} %॥४९॥

\twolineshloka
{स सौवर्णेन यानेन स्वर्गलोकं व्रजेयुवम्}
{यश्चेमां शृणुयाद्भक्त्या यश्चापि परिकीर्तयेत्} %॥५०॥

\twolineshloka
{उभौ तौ स्वर्गतौ स्याता नात्र कार्या विचारणा}
{यत्फलं सन्निहत्यायां राहुप्रस्ते दिवाकरे} %॥५१॥


\threelineshloka
{कृत्वा श्राद्धं लभेन्मयम्तदस्याः श्रवणादपि}
{एवं यः कुरुते पुण्यां द्वादशी पापनाशिनीम्}
{सर्वपापविनिमुक्तः पदं गच्छत्यनामयम्} %॥५२॥

॥इति श्रीभारतपद्मयोरुक्तं ज्येष्ठशुक्लनिर्जलैकादशीमाहात्म्यं सम्पूर्णम्॥


\hyperref[sec:ekadashi_mahatmyam_vrata_raja]{\closesub}
\clearpage

\sect{आषाढ-कृष्ण-योगिनी-एकादशी-माहात्म्यम्}
\label{sec:vrata-raja-ashadha-krishna-yogini}

\uvacha{युधिष्ठिर उवाच}

\twolineshloka
{ज्येष्ठशुक्ने निर्जलाया माहात्म्यं व श्रुतं मया}
{आषाढकृष्णपक्षे तु किनामैकादशी भवेत्} %॥१॥

\onelineshloka*
{कथयस्व प्रसादेन ममाने मधुसूदन}

\uvacha{श्रीकृष्ण उवाच}

\onelineshloka
{व्रतानामुत्तमं राजन्कथयामि तवाप्रतः} %॥२॥

\twolineshloka
{सर्वपापक्षयकरं भुक्तिमुक्तिप्रदायकम्}
{आषाढस्यासिते पक्षे योगिनीनाम नामतः} %॥३॥

\twolineshloka
{एकादशी नृपश्रेष्ठ महापातकनाशिनी}
{संसारार्णवमन्नानां पोतरूपा सनातनी} %॥४॥

\twolineshloka
{जगत्रये सारभूता योगिनीति नराधिप}
{कथयामि, कथा, तस्याः पौराणी पापहारिणीम्} %॥५॥

\twolineshloka
{अलकाधिपतिर्नाम्ना कुबेरः शिवपूजकः}
{तस्यासीत्पुष्पबदुको हेममालीति नामतः} %॥६॥

\twolineshloka
{तस्य पत्नी सुरूपासीद्विशालाक्षीति नामतः}
{स तस्यां स्नेहसंयुक्तः कामपाशवशं गतः} %॥७॥

\twolineshloka
{मानसात्पुष्पनिचयमानीय स्वगृहे स्थितः}
{पत्नीप्रेमसमायुक्तो न कुबेरालयं गतः} %॥८॥

\twolineshloka
{कुबेरो देवतदने करोति शिवपूजनम्}
{मध्याह्नसमये राजन् पुष्पाणि प्रसमीक्षते} %॥९॥

\twolineshloka
{हेममाली स्वभवने रमते कान्तया सह}
{यक्षराट् प्रत्युवाचाथ कालातिक्रमकोपितः} %॥१०॥

\twolineshloka
{कस्मान्नायाति भो यक्षा हेममाली दुरात्मवान्}
{निश्चयः क्रियतामस्य प्रत्युवाच पुनः पुनः} %॥११॥

\uvacha{यक्षा उचुः}

\twolineshloka
{वनिताकामुको गेहे रमते स्वेच्छया नृप}
{तेषां वाक्यं समाकर्ण्य कुबेरः कोपपूरितः} %॥१२॥

\twolineshloka
{आह्वयामास तं तूर्ण बटुक हेममालिनम्}
{ज्ञात्वा कालात्ययं सोपि भयव्याकुललोचनः} %॥१३॥

\twolineshloka
{आजगाम नमस्कृत्य कुबेरस्याप्रतःस्थितः}
{तं दृष्ट्वा धनदः क्रुद्धः कोपसंरक्तलोचनः} %॥१४॥

\onelineshloka*
{प्रत्युवाच रुषाविष्टः कोपाद्विस्फुरिताधरः}

\uvacha{धनद उवाच}

\onelineshloka
{रे पाप दुष्ट दुर्वृत्त कृतवान् देवहेलनम्} %॥१५॥

\twolineshloka
{अतो भव चित्रयुक्तो वियुक्तः कान्तया सदा}
{अस्मात्स्थानादपध्वस्तो गच्छ स्थानमथाधमम्} %॥१६॥

\twolineshloka
{इत्युक्ते वचने तेन तस्मात्स्था नात्पपातु सः}
{महादुःखाभिभूतश्च कुष्ठपीडितविग्रहः} %॥१७॥

\twolineshloka
{न वै तोयं न भक्ष्यं च बनेरौद्रे लभत्यसौ}
{न सुखं दिवसे तस्य न निद्रा लभते निशि} %॥१८॥

\twolineshloka
{छायायां पीडिततनुर्निदाघेऽत्यन्तपीडितः}
{शिवपूजाप्रभावेण स्मृतिस्तस्य न गच्छति} %॥१९॥

\twolineshloka
{पातकेन्नाभिभूतोऽपि कर्म पूर्वमनुस्मरन्}
{भ्रममाणस्ततोऽगच्छद्धिमाद्रि पर्वतोत्तमम्} %॥२०॥

\twolineshloka
{तत्रापश्यन्मुनिवरं मार्कण्डेयं तपोनिधिम्}
{यस्यायुर्विद्यते राजन् ब्रह्मणो दिनसप्तकम्} %॥२१॥

\twolineshloka
{आश्रम स गतस्तस्य ऋषेब्रह्मसदः समम्}
{ववन्दे चरणौ तस्य दूरतः पापकर्मकृत्} %॥२२॥

\twolineshloka
{मार्कण्डेयो मुनिवरो दृष्ट्वा तं कुष्ठिनं तदा}
{परोपकरणार्थाय समाहूयदमब्रवीत्} %॥२३॥

\uvacha{मार्कण्डेय उवाच}

\twolineshloka
{कस्मात् कुष्ठाभिभूतस्त्वं कुतो निन्द्यतरो ह्यसि}
{इत्युक्तः प्रत्युवाचाथ मार्कण्डेयेन धीमता} %॥२४॥

\uvacha{हेममाल्युवाच}

\twolineshloka
{यक्षराजस्यानुचरो हेममालीति नामतः}
{मानसात्पुष्पनिचयमानीय प्रत्यहं मुने} %॥२५॥

\twolineshloka
{शिवपूजनवेलायां कुबेराय समर्पये}
{एकस्मिन् दिवसे काललोपश्च विहितो मया} %॥२६॥

\twolineshloka
{पत्नीसौख्यप्रसक्तन कामव्याकुलचेतसा}
{ततःक्रुद्धेन शप्तोऽहं राजराजन वै मुन} %॥२७॥

\twolineshloka
{कुष्ठाभिभूतः सञ्जातो वियुक्तः कान्तया सह}
{अधुना तव सान्निध्यं प्राप्तोऽस्मि शुभकर्मणा} %॥२८॥

\twolineshloka
{सतां स्वभावतश्चित्तं परोपकरणक्षमम्}
{इति ज्ञात्वा मुनिश्रेष्ठ शाधि मां च कृतैनसम्} %॥२९॥

\uvacha{मार्कण्डेय उवाच}

\twolineshloka
{त्वया सत्यमिह प्रोक्तं नासत्यं भाषितं यतः}
{अतो व्रतोपदेशं ते करिष्यामि शुभप्रदम्} %॥३०॥

\twolineshloka
{आषाढ कृष्णपक्षे त्वं योगिनीव्रतमाचर}
{अस्य व्रतस्य पुण्येन कुष्ठात्त्वं मुच्यसे ध्रुवम्} %॥३१॥

\twolineshloka
{इति वाक्यं मुनेः श्रुत्वा दण्डवत्पतितो भुवि}
{उत्थापितश्च मुनिना बभूवातीव हर्षितः} %॥३२॥

\twolineshloka
{मार्कण्डेयोपदेशेन कृतं तेन व्रतोत्तमम्}
{तद्वतस्य प्रभावेण देवरूपो बभूव सः} %॥३३॥

\twolineshloka
{संयोग कान्तया लेभे बुभुजे सौख्यमुत्तमम्}
{ईटग्विधं नृपश्रेष्ठ कथितं योगिनीव्रतम्} %॥३४॥

\twolineshloka
{अष्टाशीतिसहस्राणि द्विजान् भोजयते तु यः}
{तत्फलं समवानोति योगिनीव्रतकृन्नरः} %॥३५॥

\twolineshloka
{महापापप्रशमनी महापुण्यफलप्रदा}
{शुचिकृष्णकादशी ते कथिता योगिनी नृप} %॥३६॥

॥इति श्रीब्रह्मवैवर्तपुराणे आषाढकृष्णयोगिन्याख्यैकादशीमाहात्म्यं सम्पूर्णम्॥


\hyperref[sec:ekadashi_mahatmyam_vrata_raja]{\closesub}
\clearpage

\sect{आषाढ-शुक्ल-शयनी-एकादशी-माहात्म्यम्}
\label{sec:vrata-raja-ashadha-shukla-shayani}

\uvacha{युधिष्ठिर उवाच}

\twolineshloka
{आषाढस्य सिते पक्षे किन्नामकादशी भवेत्}
{को देवः को विधिस्तस्या एतदाख्याहि केशव} %॥१॥

\uvacha{श्रीकृष्ण उवाच}

\twolineshloka
{कथयामि महीपाल कथामाश्चर्यकारिणीम्}
{थयामास यां ब्रह्मा नारदाय महात्मने} %॥२॥

\uvacha{नारद उवाच}

\twolineshloka
{कथयस्व प्रसादेन विष्णोराराधनाय मे}
{आषाढशुक्लपक्षे तु किनामैकादशी भवेत्} %॥३॥

\uvacha{ब्रह्मोवाच}

\twolineshloka
{वैष्णवोऽप्ति मुनि श्रेष्ठ साधु पृष्ट कलिप्रिय}
{नातः परतरं लोके पवित्रं हरिवासरात्} %॥४॥

\twolineshloka
{कर्तव्यं तु प्रयत्नेन सर्वपापापनुत्तये}
{तस्मात्तेऽहं प्रवक्ष्यामि शुक्ल एकादशीव्रतम्} %॥५॥

\twolineshloka
{एकादश्या व्रतं पुण्यं पापघ्नं सर्वकामदम्}
{न कृतं यैर्नरैलोक ते नरा निरयेषिणः} %॥६॥

\twolineshloka
{पद्मानामेति विख्याता शुचौ ोकादशी सिता}
{हृषीकेशप्रीतये तु कर्त्तव्यं व्रतमुत्तमम्} %॥७॥

\twolineshloka
{कथयामि तवाग्रेऽहं कथा पौराणिकी शुभाम्}
{यस्याः श्रवणमात्रेण महापापं प्रणश्यति} %॥८॥

\twolineshloka
{मान्धाता नाम राजर्षिर्विवस्वदंशसम्भवः}
{बभूव चक्रवर्ती स सत्यसन्धः प्रतापवान्} %॥९॥

\twolineshloka
{धर्मतः पालयामास प्रजाः पुत्रानिवौरसान्}
{ने तस्य राज्ये दुर्भिक्षं नाधयो व्याधयस्तथा} %॥१०॥

\twolineshloka
{निरातङ्काः प्रजास्तस्य धनधान्यसमन्विताः}
{नान्यायोपार्जितं द्रव्यं कोशे तस्य महीपतेः} %॥११॥

\twolineshloka
{तस्यैवं कुर्वतो राज्य बहुवर्षगणो गतः}
{अथो कदाचित्सम्प्राप्ते विपाके पापकर्मणः} %॥१२॥

\twolineshloka
{वर्षत्रयं तद्विषये न ववर्ष बलाहकः}
{तेनोद्विग्नाः प्रजास्तत्र बभूवुः क्षुधयार्दिताः} %॥१३॥

\twolineshloka
{स्वाहास्वधावषट्कारवेदाध्ययनवर्जिता}
{गावभूबुर्विषयास्तस्य सस्याभावेन पीडिताः} %॥१४॥

\twolineshloka
{अथ प्रजाः समागत्य राजानमिदमब्रुवन्}
{श्रूयतां वचनं राजन् प्रजानां हितकारकम्} %॥१५॥

\twolineshloka
{आपो नारा इति प्रोक्ताः पुराणेषु मनीषिभिः}
{अयनं ता भङ्गंवतस्तेन नारायणः स्मृतः} %॥१६॥

\twolineshloka
{पर्जन्यरूपो भगवान्विष्णुः सर्वगतः सदा}
{स एव कुरुते वीष्ट वृष्टेरन्नं ततः प्रजाः} %॥१७॥

\twolineshloka
{तदभावेन नृपते क्षयं गच्छन्ति वै प्रजाः}
{तथा कुरु नृपश्रेष्ठ योगक्षेमो यथा भवेत्} %॥१८॥

\uvacha{राजोवाच}

\twolineshloka
{सत्यमुक्तं भवद्भिश्च न मिथ्याभिहितं वचः}
{अन्नं ब्रह्ममयं प्रोक्तमन्ने सर्व प्रतिष्ठितम्} %॥१९॥

\twolineshloka
{अन्नाद्भवन्ति भूतानि जगदनेन वर्तते}
{इत्येवं श्रूयते लोके पुराणे बहुविस्तरे} %॥२०॥

\twolineshloka
{नृपाणामपचारेण प्रजानां पीडनं भवेत्}
{नाहं पश्याम्यात्मकृतं दोषं बुद्ध्या विचारयन्} %॥२१॥

\twolineshloka
{तथापि प्रयतिष्यामि प्रजानां हितकाम्यया}
{इति कृत्वा मति राजा परिमेयबलान्वितः} %॥२२॥

\twolineshloka
{नमस्कृत्य विधातारं जगाम गहनं वनम्}
{चचारि मुनिमुख्यानामाश्रमांस्तपसैधितान्} %॥२३॥

\twolineshloka
{शाददर्शाथ ब्रह्मसुतमृषिमगिरसं नृपः}
{तेजसा द्योतितदिशं द्वितीयमिव पद्माजम्} %॥२४॥

\twolineshloka
{तं दृष्ट्वा हर्षितो राजा अवतीर्य च वाहनात्}
{नमश्चक्रेस्य चरणौ कृताञ्जलिपुटो वशी} %॥२५॥

\twolineshloka
{मुनिस्तमभिनन्द्याथ स्वस्तिवाचनपूर्वकम्}
{पप्रच्छ कुशलं राज्ये सप्तस्वङ्गेषु भूपतेः} %॥२६॥

\twolineshloka
{निवेदयित्वा कुशलं पप्रच्छानामयं नृपः}
{ततश्च मुनिना राजो पृष्टागमनकारणः} %॥२७॥

\onelineshloka*
{अब्रवीन्मुनिशार्दूलं स्वस्यागमनकारणम्}

\uvacha{राजोवाच}

\twolineshloka
{भगवन् धर्मविधिना मम पालयतो महीम्}
{अनावृष्टिः सम्प्रवृत्ता नाहं वेदयत्र कारणम्} %॥२८॥

\twolineshloka
{संशयच्छेदनार्थेऽत्र ह्यागतोऽहं तवान्तिकम्}
{योगक्षेमविधानेन प्रजानां निर्वृतिं कुरु} %॥२९॥

\uvacha{ऋषिरुवाच}

\twolineshloka
{एतत्कृतयुगं राजन् युगानामुत्तमं स्मृतम्}
{अत्र ब्रह्मोत्तरा लोका धर्मश्चात्र चतुष्पदः} %॥३०॥

\twolineshloka
{अस्मिन्युगे तपोयुक्ता ब्राह्मणा नेतरे जनाः}
{विषये तव राजेन्द्र वृषलो यत्तपस्यति} %॥३१॥

\twolineshloka
{अकार्यकरणात्तस्य न वर्षति बलाहकः}
{कुरु तस्य वधे यत्नं येन दोषः प्रशाम्यति} %॥३२॥

\uvacha{राजोवाच}

\twolineshloka
{नाहमेनं वधिष्यामि तपस्यन्तमनागसम्}
{धर्मोपदेशं कथय उपसर्गविनाशने} %॥३३॥

\uvacha{ऋषिरुवाच}

\twolineshloka
{यद्येवं तर्हि नृपते कुरुष्वैकादशीव्रतम्}
{शुचिमासे सिते पक्षे पद्मानामेति विश्रुता} %॥३४॥

\twolineshloka
{तस्या व्रतप्रभावेण सुवृष्टिर्भविता ध्रुवम्}
{सर्वसिद्धिप्रदा ह्येषा सर्वोपद्रवनाशिनी} %॥३५॥

\twolineshloka
{अस्या व्रतं कुरु नृप सप्रजः सपरिच्छदः}
{इति वाक्यं मुनेः श्रुत्वा राजा स्वगृहमागतः} %॥३६॥

\twolineshloka
{आषाढमासे सम्प्राप्ते पद्माव्रतमथाकरोत्}
{प्रजाभिः सह सर्वाभिश्चातुर्वर्ण्यसमन्वितः} %॥३७॥

\twolineshloka
{एवं कृते व्रते राजन्प्रववर्ष बलाहकः}
{जलेन प्लाविता भूमिरभवत्सस्यमालिनी} %॥३८॥

\twolineshloka
{हृषीकेशप्रसादेन जनाः सौख्यं प्रपेदिरे}
{एतस्मात्कारणादेव कर्तव्यं व्रतमुत्तमम्} %॥३९॥

\twolineshloka
{भुक्तिमुक्तिप्रदं चैव लोकानां सुखदायकम्}
{पठनाच्छ्रवणादस्याः सर्वपापैः प्रमुच्यते} %॥४०॥

[इयमेव शयन्याख्या। एतस्यां विष्णुशयनव्रतं चातुर्मास्यव्रतग्रहणं चोक्तं भविष्ये।]

॥इति श्रीवाराहपुराणे आषाढशुक्लपद्माख्यैकादशीव्रतमाहात्म्यम्॥


\hyperref[sec:ekadashi_mahatmyam_vrata_raja]{\closesub}
\clearpage

\sect{आषाढ-शुक्ल-शयनी-एकादशी-माहात्म्यम् (भविष्य-पुराणम्)}
\label{sec:vrata-raja-ashadha-shukla-shayani-bhavishya}

\uvacha{कृष्ण उवाच}

\twolineshloka
{इयमेकादशी राजञ्छयनीत्यभिधीयते}
{विष्णोः प्रसादसिद्धयर्थमस्यां च शयनव्रतम्} %॥१॥

\twolineshloka
{कर्तव्यं राजशार्दूल जनैर्मोक्षेच्छुमिः सदा}
{चातुर्मास्यव्रतारम्भोऽप्यस्यामेव विधीयते} %॥२॥

\uvacha{युधिष्ठिर उवाच}

\twolineshloka
{कथं कृष्ण प्रकर्तव्यं श्रीविष्णोः शयनव्रतम्}
{तब्रूहि कृपया देव चातुर्मास्यप्रतानि च} %॥३॥

\twolineshloka
{श्रीकृष्ण उवाचशृणु पार्थ प्रवक्ष्यामि गोविन्दशयनव्रतम्}
{चातुर्मास्ये च यान्युतान्यासंस्तानि व्रतानि च} %॥४॥

\twolineshloka
{कर्कराशिगते सूर्ये शुचौ रुक्के तु पक्षके}
{एकादश्यां जगनाथं स्वापयेन्मधुसूदनम्} %॥५॥

\twolineshloka
{तुलाराशिस्थिते तस्मिन् पुनरुत्थापयेद्धरिम्}
{आषाढल्य सिते पक्षे एकादश्यामुपोषितः} %॥६॥

\twolineshloka
{चातुर्मास्यव्रतानां तु कुर्वीत नियमं ततः}
{स्थापयेत् प्रतिमा विष्णो शङ्खचक्रगदाधराम्} %॥७॥

\twolineshloka
{पीताम्बरधरां सौम्यां पर्यङ्के वै सिते शुभे}
{सितवस्त्रसमाच्छन्ने सोपधाने युधिष्ठिर} %॥८॥

\twolineshloka
{इतिहासपुराणज्ञो ब्राह्मणो वेदपारगः}
{नापयित्वा दधिक्षीरघृतक्षौद्रसिताजलैः} %॥९॥

\twolineshloka
{समालेप्य शुभैर्गन्धैधूपैर्दीपैश्च भूरिशः}
{पूजयेत्कुसुमैः शस्तैमन्त्रेणानेन पाण्डव} %॥१०॥

\twolineshloka
{सुप्ते त्वाय जगन्नाथे जगत्सुप्तं चराचरम्}
{विबुद्धे त्वयि बुध्येत जगत्सर्व चराचरम्} %॥११॥

\twolineshloka
{एवं तां प्रतिमां विष्णोः पूजयित्वा युधिष्ठिर}
{प्रभाषेताग्रतो विष्णोः कृताञ्जलिपुटो नरः} %॥१२॥

\twolineshloka
{चतुरो वार्षिकान्मासान्देवस्योत्थापनावधि}
{ग्रहीष्ये नियमाञ्छुद्वानिर्विघ्रान्कुरु मे प्रभो} %॥१३॥

\twolineshloka
{इति सम्प्रार्थ्य देवेशं प्रहः संशुद्धमानसः}
{स्त्री वा नरो वा मद्भक्तो धर्मार्थ च धृतव्रतः} %॥१४॥

\twolineshloka
{गृहीयानियमानेतान् दन्तधावन पूर्वकम्}
{व्रतप्रारम्भकालास्तु प्रोक्ताः पञ्चैव विष्णुना} %॥१५॥

\twolineshloka
{एकादशी द्वादशी च पौर्णिमा च तथाष्टमी}
{कर्कटाख्या च सङ्क्रान्तिस्तेषु कुर्याद्यथाविधि} %॥१६॥

\twolineshloka
{चतुर्धा गृह्य वै चीर्ण चातुर्मास्यव्रतं नरः}
{कार्तिक शुक्लपक्षे तु द्वादश्यां तत्समापयत्} %॥१७॥

\twolineshloka
{न शैशवं च मौव्यं च शुक्रगुवोर्ने वा तिथेः}
{खण्डत्वं चिन्तयेदादौ चातुर्मास्यविधौ नरः} %॥१८॥

\twolineshloka
{अशुचिर्वा शुचिर्वापि यदि स्त्री यदि वा पुमान्}
{व्रतमेकं नरः कृत्वा मुच्यते सर्वपातकैः} %॥१९॥

\twolineshloka
{प्रतिवर्षे तु यः कुर्याद्रतं वै संस्मरन् हरिम्}
{देहान्तेऽतिप्रदीप्तेन विमानेनार्कतेजसा} %॥२०॥

\twolineshloka
{मोदते विष्णुलोकेऽसौ यावदाभूतसम्प्लवम्}
{तेषां फलानि वक्ष्यामि कर्तृणां त पृथक्पृथक्} %॥२१॥

\twolineshloka
{देवतायतने नित्यं मार्जनं जलसेचनम्}
{प्रलेपनं गोमयेन रङ्गवल्ल्यादिकं तथा} %॥२२॥

\twolineshloka
{यः करोति नरश्रेष्ठश्चातुर्मास्यमतन्द्रितः}
{समाप्तौ च यथाशक्त्या कुर्याद्राह्मणभोजनम्} %॥२३॥

\twolineshloka
{सप्तजन्मसु विप्रेन्द्रः सत्यधर्मपरो भवेत्}
{दना क्षीरेण चाज्येन क्षौद्रेण सितया तथा} %॥२४॥

\twolineshloka
{नापयेद्विधिना देवं चातुर्मास्ये जनाधिप}
{स याति विष्णुसारूप्य सुखमक्षय्यमश्नुते} %॥२५॥

\twolineshloka
{नृपो भूमिं प्रदद्याद्यो यथाशक्त्या च काञ्चनम्}
{विप्राय देवमुद्दिश्य सफलं च सदक्षिणम्} %॥२६॥

\twolineshloka
{अक्षयान् लभते भोगान् स्वर्ग इन्द्र इवापरः}
{लोक स समवाप्नोति विष्णोरत्र न संशयः} %॥२७॥

\twolineshloka
{देवाय हेमपद्मं तु दद्यानैवेद्यसंयुतम्}
{गन्धपुष्पाक्षताद्यैर्यो देवब्राह्मणयोरपि} %॥२८॥

\twolineshloka
{पूजां यः कुरुते नित्यं चातुर्मास्ये व्रती नरः}
{अक्षयं सुखमाप्नोति पुरन्दरपुरं व्रजेत्} %॥२९॥

\twolineshloka
{यस्तु वै चतुरो मासांस्तुलस्या हरिमर्चयेत्}
{तुलसी काञ्चनीं कृत्वा ब्राह्मणाय निवेदयेत्} %॥३०॥

\twolineshloka
{काञ्चनेन विमानेन वैष्णवी लभते गतिम्}
{देवाय गुग्गुलं यो वै दीपं चार्पयते नरः} %॥३१॥

\twolineshloka
{समाप्तौ धूपिको दद्यादीपिकां च महामते}
{स भोगी जायते श्रीमांस्तथा सौभाग्यवानपि} %॥३२॥

\twolineshloka
{प्रदक्षिणास्तु यः कुर्यान्नमस्काराविशेषतः}
{अश्वत्थस्याथवा विष्णोः कार्तिक्यवधि स ध्रुवम्} %॥३३॥

\twolineshloka
{विष्णुलोकमवाप्नोति सत्यं सत्यं न संशयः}
{सन्ध्यादीपप्रदो यस्तु प्राङ्गणे द्विजदेवयोः} %॥३४॥

\twolineshloka
{समाप्तौ दीपिका दद्याद्वस्त्रं चैकं च काचनम्}
{वैकुण्ठं समवाप्नोति तेजस्वी स भवेदिह} %॥३५॥

\twolineshloka
{विष्णुपादोदकं यस्तु पिबेच्छ्राद्धासमन्वितः}
{विष्णोर्लोकमवाप्नोति न चास्मिञ्जायते नरः} %॥३६॥

\twolineshloka
{शतमष्टोत्तरं यस्तु गायत्रीजपमाचरेत्}
{त्रिकालं वैष्णवे हम्र्थे न स पापेन लिप्यते} %॥३७॥

\twolineshloka
{पुराणं शृणुयानित्यं धर्मशास्त्रमथापि वा}
{काश्चनेन युतं वस्त्रं पुस्तकं च निवेदयेत्} %॥३८॥

\twolineshloka
{पुण्यवान् धनवान्भोगी सत्यशौचपरायणः}
{ज्ञानवाँल्लोकविख्यातो बहुशिष्यः सुधार्मिकः} %॥३९॥

\twolineshloka
{नाममन्त्रव्रतपरः शम्भोर्वा केशवस्य च}
{समाप्तौ प्रतिमां दद्यात्तस्य देवस्य काश्चनीम्} %॥४०॥

\twolineshloka
{पुण्यवान् दोषनिर्मुक्तः स भवेच्च गुणालयः}
{कृतनित्यक्रियो भूत्वा सूर्यायाय निवेदयेत्} %॥४१॥

\twolineshloka
{सूर्यमण्डलमध्यस्थं देवं ध्यात्वा जनार्दनम्}
{समाप्तौ काश्चनं दद्याद्रक्तवस्त्रं च गां तथा} %॥४२॥

\twolineshloka
{आरोग्यं पूर्णमायुश्च कीर्ति लक्ष्मी बलं लभेत्}
{तिलहोमं तु यः कुर्याचातुर्मास्ये दिनदिने} %॥४३॥

\twolineshloka
{भक्त्या व्याहृतिभिर्मत्रैर्गायच्या वा व्रतान्वितः}
{अष्टोत्तरशतं चाथ अष्टाविंशतिमेव वा} %॥४४॥

\twolineshloka
{तिलपात्रं समाप्तौ तु दद्याद्विप्राय धीमते}
{वाङ्मन कायजनितः पापैर्मुच्येत् सश्चितैः} %॥४५॥

\twolineshloka
{न रोगरभिभूयेत लभेत्सन्ततिमुत्तमाम्}
{अनहोमं तु यः कुर्याचातुर्मास्यमतन्द्रितः} %॥४६॥

\twolineshloka
{समाप्तौ घृतकुम्भं तु दद्यात्सवस्त्रकाञ्चनम्}
{आरोग्यं कान्तिमतुलां पुत्रसौभाग्यसम्पदः} %॥४७॥

\twolineshloka
{शत्रुक्षयं च लभते ब्रह्मगा प्रतिमो भवेत्}
{अश्वत्थसेवां यः कुर्यात्सर्वपापैः प्रमुच्यते} %॥१८॥

\twolineshloka
{विष्णुभक्तो भवेत्पश्चादन्ते वस्त्रं प्रदादयेत्}
{सकाश्चनं ब्राह्मणाय नैव रोगान् स विन्दते} %॥४९॥

\twolineshloka
{तुलसी धारयेद्यस्तु विष्णुप्रीतिकरां शुभाम्}
{विष्णुलोकमवाप्नोति सर्वपापैः प्रमुच्यते} %॥५०॥

\twolineshloka
{ब्राह्मणान्भोजयेत्पश्चाद्विष्णुमुद्दिश्य पाण्डव}
{यस्तु सुप्ते हृषीकेशे दूर्वामृतसम्भवाम्} %॥५१॥

\twolineshloka
{सदा प्रातर्वहेन्मूनि त्वं दूर्वे इति मन्त्रतः}
{व्रतान्ते च कुरुश्रेष्ठ दूर्वी स्वर्णविनिर्मिताम्} %॥५२॥

\twolineshloka
{दद्याद् दक्षिणया साई मन्त्रेणानेन सुव्रत}
{यथाशाखाप्रशाखाभिर्विस्तृतासि महीतले} %॥५३॥

\twolineshloka
{तथा ममापि सन्तानं देहि त्वमजरामरम्}
{नाशुभं प्राप्नुयाजातु पापेभ्यःप्रविमुच्यते} %॥५४॥

\twolineshloka
{भुक्त्वा तु सकलान् भोगान् स्वर्गलोके महीयते}
{गीतं तु देवदेवस्य केशवस्य शिवस्य वा} %॥५५॥

\twolineshloka
{करोति पुरतो नित्यं जागृतेः फलमाप्नुयात्}
{चातुर्मास्यवती दद्याद् घण्टां देवाय सुस्वराम्} %॥५६॥

\twolineshloka
{सरस्वति जगन्नाथे जगजाड्यापहारिणि}
{साक्षाद्ब्रह्मकलयं च विष्णुरुद्रादिभिः स्तुता} %॥५७॥

\twolineshloka
{गुरोरवज्ञया यच्चानध्यायऽध्ययनं कृतम्}
{तन्ममाध्ययनोत्पन्नं जाड्यं हर वरानने} %॥५८॥

\twolineshloka
{घण्टादानेन तुष्टा त्वं ब्रह्माणी लोकपावनी}
{विप्रपादविनिर्मुक्तं तोयं यःप्रत्यहं पिबेत्} %॥५९॥

\twolineshloka
{चातुर्मास्ये नरो भक्त्या मद्रूपं ब्राह्मणं स्मरन्}
{मनोवाकायजनितैर्मुक्तो भवति किल्बिः} %॥६॥

\twolineshloka
{व्याधिभिर्नामिभूयेत श्रीरायुस्तस्य वर्द्धते}
{समाप्तौ गोयुगं दद्याद्गामेकां वा पयस्विनीम्} %॥६१॥

\twolineshloka
{तत्राप्यशक्तौ राजेन्द्र दद्याद्वासोयुगं व्रती}
{ब्राह्मणं वन्दते यस्तु सर्वदेवमयं श्रुतम्} %॥६२॥

\twolineshloka
{कृतकृत्यो भवेत्सद्यः सर्वपापैः प्रमुच्यते}
{समाप्तौ भोजयेद्विप्रानायुवित्तं च विन्दति} %॥६३॥

\twolineshloka
{संस्पृशेत्कपिलां यो वै नित्यं भक्तिसमन्वितः}
{तामेवालङ्कृतां दद्यात्सवत्सां दक्षिणायुताम् '} %॥६४॥

\twolineshloka
{सार्वभौमो भवेद्राजा दीर्घायुश्च प्रतापवान्}
{स वसेदिन्द्रवत्स्वर्गे वत्सरान् रोमसम्मितान्} %॥६५॥

\twolineshloka
{नमस्करोति यः सूर्य गणेशं वापि नित्यशः}
{आयुरारोग्यमैश्वर्य लभते कान्तिमुत्तमाम्} %॥६६॥

\twolineshloka
{विनराजप्रसादेन प्राप्नुयादीप्सितं फलम्}
{सर्वत्र विजयं चैव नात्र कार्या विचारणा} %॥६७॥

\twolineshloka
{विघ्नेशाकों सुवर्णस्य सिन्दूरारुणसन्निभौ}
{निवेदयेद्राह्मणाय सर्वकामार्थसिद्धये} %॥६८॥

\twolineshloka
{यस्तु रौप्यं शिवप्रीत्यै दद्याद्भक्त्या ऋतुद्वये}
{तानं वा प्रत्यहं दद्यात्स्वशक्त्या शिवतुष्टये} %॥६९॥

\twolineshloka
{सुरूपाल्लभते पुत्रान् रुद्रभक्तिपरायणान्}
{समाप्तौ मधुपूर्ण तु पात्रं राजतमुत्तमम्} %॥७०॥

\twolineshloka
{प्रदद्यात्ताम्रदाने तु ताम्रपात्रं गुडान्वितम्}
{यस्तु सुप्ते हृषीकेशे स्वर्ण दद्याद स्वशक्तितः} %॥७१॥

\twolineshloka
{वस्त्रयुग्मतिलैः सार्द्ध सर्वपापैः प्रमुच्यते}
{इह भुक्त्वा महाभोगानन्ते शिवपुरं व्रजेत्} %॥७२॥

\twolineshloka
{वस्त्र दानं तु यः कुर्याचातुर्मास्ये द्विजातये}
{अभ्यर्च्य गन्धपुष्पायेविष्णुमें प्रीयतामिति} %॥७३॥

\twolineshloka
{शय्यां दद्यात्समाप्तौ तु वासः काञ्चनपट्टिकाम्}
{अक्षय्यं सुखमाप्नोति धनं स धनदोपमम्} %॥७४॥

\twolineshloka
{यो गोपीचन्दनं दद्यानित्यं वर्षासु मानवः}
{श्रीपतिस्तस्य सन्तुष्टो भुक्तिं मुक्तिं ददाति च} %॥७५॥

\twolineshloka
{समाप्तावपि तदद्यात्तुलापरिमितं शुभम्}
{तदर्द्ध वा तदर्द्ध वा सवस्त्रं च सदक्षिणम्} %॥७६॥

\twolineshloka
{यस्तु सुप्ते हृषीकेशे प्रत्यहं तु व्रतान्वितः}
{दद्याद् दक्षिणया सार्द्ध शर्करमथवा गुडम्} %॥७७॥

\twolineshloka
{एवं व्रते तु सम्पूर्ण कुर्वत्रुद्यापनं बुधः}
{प्रत्येक ताम्रपात्राणि पलाष्टकमितानि तु} %॥७८॥

\twolineshloka
{वित्तशाठयमकुर्वाणश्चतुष्पलमितानि वा}
{अष्टचत्वारि चैक वा शर्करापूरितानि च} %॥७९॥

\twolineshloka
{दक्षिणाफलवासोभिः प्रत्येकं संयुतानि च}
{सह धान्यानि विप्रेभ्यः श्रद्धया प्रतिपादयेत्} %॥८०॥

\twolineshloka
{ताम्रपात्रं सवस्त्रं च शर्कराहेमसंयुतम्}
{सूर्यप्रीतिकरं यस्माद्रोगघ्नं पापनाशनम्} %॥८१॥

\twolineshloka
{पुष्टि कीर्तिदं नृणां नित्यं सन्तानकारकम्}
{सर्वकामप्रदं स्वर्ग्यमायुर्वर्द्धनमुत्तमम्} %॥८२॥

\twolineshloka
{तस्मादस्य प्रदानेन कीर्तिरस्तु सदा मम}
{एवं व्रतं तु यः कुर्यात्तस्य पुण्यफलं शृणु} %॥८३॥

\twolineshloka
{गन्धर्वविद्यासम्पन्नः सर्वयोषित्प्रियो भवेत्}
{राजापि लभते राज्यं पुत्रार्थी लभते सुतान्} %॥८४॥

\twolineshloka
{अर्थार्थी प्राप्नुयादर्थ निष्कामो मोक्षमाप्तुयात्}
{यस्तु वै चतुरो मासाञ्छाकमूलफलादिकम्} %॥८५॥

\twolineshloka
{नित्यं ददाति विप्रेभ्यः शक्त्या यत्सम्भवेन्नृप}
{व्रतान्ते वस्त्रयुग्मं च शक्त्या दद्यात्सदक्षिणम्} %॥८६॥

\twolineshloka
{सुखीभूत्वा चिरं काल राजयोगी भवेन्नरः}
{सर्वदेवप्रियं यस्माच्छाकं तृप्तिकरं नृणाम्} %॥८७॥

\twolineshloka
{ददामि तेन देवाद्याः सदा कुर्वन्तु मङ्गलम्}
{यस्तु सुप्ते हृषीकेशे प्रत्यहं तु ऋतुद्ये} %॥८८॥

\twolineshloka
{दद्यात्कटुवयं मयों गृहपर्याप्तमादरात्}
{ब्राह्मणाय सुशीलाय दिनेशप्रीतये नघ} %॥८९॥

\twolineshloka
{दक्षिणावस्त्रसहितं मन्त्रेणानेन सुव्रत}
{कटुत्रयमिदं यस्माद्रोगनं सर्वदेहिनाम्} %॥९०॥

\twolineshloka
{तस्मादस्य प्रदानेन प्रीतो भवतु भास्करः}
{एवं कृत्वा व्रतं सम्यकुर्यादुद्यापनं बुधः} %॥९१॥

\twolineshloka
{कृत्वा स्वर्णमयों शुण्ठी मरीचं मागधीमपि}
{सवस्त्रां दक्षिणायुक्तां दद्याद्विप्राय धीमते} %॥९२॥

\twolineshloka
{एवं व्रतं यः कुरुते स जीवेच्छरदां शतम्}
{प्राप्नुयादीप्सितानानन्ते स्वर्ग ब्रजेनृप} %॥९३॥

\twolineshloka
{मुक्ताफलानि यो दद्यानित्यं विप्राय सन्मतिः}
{अन्नवान्कीर्तिमाञ्छीमाञ्जायते वसुधाधिप} %॥९॥

\twolineshloka
{ताम्बूलदानं यःकुर्याद्वर्जयेता जितेन्द्रियः}
{रक्तवस्त्रद्वयं दद्यात्समाप्तौ च सदक्षिणम्} %॥९५॥

\twolineshloka
{महालावण्यमानोति सर्वरोगविवर्जितः}
{मेधावी तुभगःप्राज्ञो रक्तकण्डव जायते} %॥९६॥

\twolineshloka
{गाधर्वत्वमवानोति स्वर्गलोकं च गच्छति}
{ताम्बूलं श्रीकर भद्रं ब्रह्मविष्णुशिवात्मकम्} %॥९७॥

\twolineshloka
{अस्य प्रदानाद्ब्रह्माद्याः श्रियं ददतु पुष्कलाम्}
{चातुर्मासे प्रतिदिनं सुवासिन्यै द्विजाय च} %॥९८॥

\twolineshloka
{नारीवा पुरुषो वापि हरिद्रां सम्प्रयच्छति}
{लक्ष्मीमुद्दिश्य गौरी वा समाप्तौ राजतं नवन्} %॥९९॥

\twolineshloka
{हरिद्रापूरितं कृत्वा तत्पात्र दक्षिणान्वितम्}
{प्रदद्याद्भक्तिसंयुक्तं देवी मे प्रीयतामिति} %॥१००॥

\twolineshloka
{भ; सह मुखं मुङ्क्त नारी नार्या तथा पुमान्}
{सौभाग्यप्रक्षयं धान्यं धनपुत्रसमुन्नतिम्} %॥१०१॥

\twolineshloka
{सम्प्राप्य रूपलावण्ये देवीलोके महीयते}
{उमामहेशमुद्दिश्य चातुर्मास्ये दिने दिने} %॥१०२॥

\twolineshloka
{सम्पूज्य विप्रमिथुनं तस्मै यश्च स्वशक्तितः}
{दद्यात् सदक्षिणं हेम उमेशः प्रीयतामिति} %॥१०३॥

\twolineshloka
{उमेशप्रतिमा हैमी दयादुद्यापने बुधः}
{पञ्चोपचारैः सम्पूज्य धेन्वा च वृषभेण च} %॥१०४॥

\twolineshloka
{भोजयेदपि मिष्टानं तस्य पुण्यफलं शृणु}
{सम्पत्तिरक्षया कीर्तिर्जायते व्रतवैभवात्} %॥१०५॥

\twolineshloka
{इह भुक्त्वाखिलान्कामानन्ते शिवपुरं व्रजेत्}
{फलदानं तु यः कुर्याश्चातुर्मास्यमतन्द्रितः} %॥१०६॥

\twolineshloka
{समाप्तौ कलधौतानि तानि दद्याट्विजातथे}
{सन्निनोरयान्प्राप्य सन्तति चानपायिनीम्} %॥१०७॥

\twolineshloka
{फलदानस्य माहात्म्यान्मोदते नन्दने बने}
{पुष्पदानव्रते चापि स्वर्णपुष्पादि दापयेत्} %॥१०८॥

\twolineshloka
{स सौभाग्यं परं प्राप्य गन्धर्वपदमाप्नुयात्}
{वासुदेवे प्रसुप्ते तु चातुर्मास्यमतन्द्रितः} %॥१०९॥

\twolineshloka
{नित्यं वामनमुद्दिश्य दध्यन्नं स्वादु षड्सैः}
{भोजये इथवः दद्यादेकादश्यां न भोजयेत्} %॥११०॥

\twolineshloka
{दानमेव प्रकुर्वीत ग्रहणादौ तथैव च}
{अशक्तौ नित्यदाने तु कुर्यात्तश्च पर्वतु} %॥१११॥

\twolineshloka
{भूताष्टम्याममायां च पूर्णिमायां तथैव च}
{प्रत्यकवारमथवा प्रतिभार्गववासरम्} %॥११२॥

\twolineshloka
{एवं कृत्वा समाप्तौ तु यथाशक्ति महीं ददेव}
{अशक्तौ भूमिदाने तु धेतुं दद्यादलङ्कृताम्} %॥११३॥

\twolineshloka
{तत्राप्यशक्तौ वासश्च सरुक्मे पादुके तथा}
{अक्षय्यमनमामोति पुत्रपौत्रादिसम्पदम्} %॥११४॥

\twolineshloka
{सुस्थिरां विष्णुभक्तिं च प्रयाति हरिमन्दिरम्}
{नित्यं पयस्विनी दद्यात्सालङ्कारों शुभावहाम्} %॥११५॥

\twolineshloka
{सवत्सां दक्षिणोपेतां स सर्वज्ञानवान् भवेत्}
{न परप्रेष्यतां याति ब्रह्मलोकं च मच्छति} %॥११६॥

\twolineshloka
{अक्षय्यं सुखमाप्नोति पितृभिः सहितो नरः}
{वार्षिकांश्चतुरो मासान् प्राजापत्यं चरेन्नरः} %॥११७॥

\twolineshloka
{समाप्तौ गोयुग दत्त्वा कृत्वा ब्राह्मणभोजनम्}
{सर्वपापविशुद्धात्मा याति ब्रह्म सनातनम्} %॥११८॥

\twolineshloka
{एकान्तरोपवासे तु सीराण्यष्टौ प्रदापयेत्}
{वस्त्रकाञ्चनयुक्तानि बलीवर्दयुतानि च} %॥११९॥

\twolineshloka
{अनछुट्टयसंयुक्तं लाल कर्षणक्षमम्}
{सर्वोपस्करसंयुक्तं ददामि प्रीतये हरेः} %॥१२०॥

\twolineshloka
{शाकमूलफलैर्वापि चातुर्मास्यं नयेन्नरः}
{समाप्तौ गोप्रदानेन स गच्छेद्विष्णुमन्दिरम्} %॥१२१॥

\twolineshloka
{पयोव्रती तथाप्नोति ब्रह्मलोक सनातनम्}
{व्रतान्त च तथा दद्याद्गामेकां च पयस्विनीम्} %॥१२२॥

\twolineshloka
{नित्यं रम्भापलाशे च ये भुङ्क्त तु ऋतुद्वये}
{वस्त्रयुग्मं च कांस्यं च शक्स्या दत्त्वा सुखी भवेत्} %॥१२३॥

\twolineshloka
{कांस्यं ब्रह्मा शिवो लक्ष्मीः कास्यमेव विभावसुः}
{कांस्यं विष्णुमयं यस्मादतः शान्ति प्रयच्छ मे} %॥१२४॥

\twolineshloka
{नित्यं पलाशभोजी चेत्तैलाभ्यङ्गविवर्जितः}
{स निहन्त्यतिपापानि तूलराशिमिवानलः} %॥१२५॥

\twolineshloka
{ब्रह्मन्नश्च सुरापश्च बालघातकरश्च यः}
{असत्यवादिनो ये च स्त्रीधातिव्रतघातकाः} %॥१२६॥

\twolineshloka
{अगम्यागामिनश्चैव विधवागामिनस्तथा}
{चाण्डालीगामिनश्चैव विप्रस्त्रीगामिनस्तथा} %॥१२७॥

\twolineshloka
{ते सर्वे पापनिर्मुक्ता भवन्त्येतद्रतेन च}
{समाप्तौ कांस्यपात्रं तु चतुःषष्टिपलैर्युतम्} %॥१२८॥

\twolineshloka
{सवत्सां गां च वै दद्यात्सालङ्कारां पयस्विनीम्}
{अलङ्कृताय विदुषे सुवस्त्राय सुत्रेषिणे} %॥१२९॥

\twolineshloka
{भूमौ विलीप्य यो भुक्ते देव नारायणं स्मरन्}
{दद्याद्भूमि यथाशक्ति कृष्यां बहुजलान्वितान्} %॥१३०॥

\twolineshloka
{आरोग्य पुत्रसम्पन्नो राजा भवति धार्मिकः}
{शत्रोभयं न लभते विष्णुलोकं स गच्छति} %॥१३१॥

\twolineshloka
{अयाचिते त्वनड्वाहं सहिरण्यं सचन्दनम्}
{षडसं भोजनं दद्यात्स याति परमां गतिम्} %॥१३२॥

\twolineshloka
{यस्तुसुप्ते हृषीकेशे नक्तं च कुरुते व्रतम्}
{ब्राह्मणान्भोजयेत्पश्चाच्छिक्लो महीयते} %॥१२३॥

\twolineshloka
{एकभक्तं नरः कृत्वा मिताशी च दृढव्रतः}
{योर्चयेचतुरो मासान्वासुदेवं स नाकभाक्} %॥१३४॥

\twolineshloka
{समाप्तौ भोजयेदिप्राञ्छक्त्या दद्याच्च दक्षिणाम्}
{यस्तु सुप्ते हृषीकेशे क्षितिशायी भवेन्नरः} %॥१३५॥

\twolineshloka
{शय्यां सोपस्करां दद्याच्छिवलोके महीयते}
{पादाभ्यनं नरो यस्तु वर्जयेच्च ऋतुद्वये} %॥१३६॥

\twolineshloka
{समाप्तौ च यथाशक्ति कुर्याद्राह्मणभोजनन्}
{दत्त्वा च दक्षिणां शक्त्या स गच्छेद्विष्णुमन्दिरम्} %॥१३७॥

\twolineshloka
{आषाढादिचतुर्मासान्वर्जयेन्नखकृन्तनम्}
{आरोग्यपुत्रसम्पन्नो राजा भवति धार्मिकः} %॥१३८॥

\twolineshloka
{पायसं लवणं चैव मधुसर्पिः फलानि च}
{चातुर्मास्ये वर्जयेद्यो गौरीशङ्करतुष्टये} %॥१३९॥

\twolineshloka
{कार्तिक्यां च पुनस्तानि ब्राह्मणाय निवेदयेत्}
{स रुद्रलोकमाप्नोति रुद्रव्रतनिषेवणात्} %॥१४०॥

\twolineshloka
{यवान्नं भक्षयेद्यस्तु शुभं शाल्यन्नमेव वा}
{पुत्रपौत्रादिभिः सार्द्ध शिवलोके महीयते} %॥१४१॥

\twolineshloka
{तैलाभ्यङ्गपरित्यागी विष्णुभक्तः सदा व्रती}
{वर्षासु विष्णुमभ्यर्च्य वैष्णवी लभते गतिम्} %॥१४२॥

\twolineshloka
{समाप्तौ कांस्यपात्रं च सुवर्णेन समन्वितम्}
{तैलेन पूरितं कृत्वा ब्राह्मणाय निवेदयेत्} %॥१४३॥

\twolineshloka
{वार्षिकांश्चतुरो मासाञ्छाकानि परिवर्जयेत्}
{व्रतान्ते हरिमुद्दिश्य पात्रं राजतमेव हि} %॥१४४॥

\twolineshloka
{वस्त्रेण वेष्टितं शाकदशकेन प्रपूरितम्}
{समभ्यर्च्य यथाशक्त्या ब्राह्मणान्वेदपारगान्} %॥१४५॥

\twolineshloka
{तेभ्यो दद्यादक्षिणया व्रतसम्पूर्तिहेतवे}
{शिवसायुज्यमाप्नोति प्रसादाच्छूलपाणिनः} %॥१४६॥

\twolineshloka
{गोधूमवर्जनं कृत्वा भोजनव्रतमाचरेत्}
{कार्तिके स्वर्णगोधूमान वस्त्रं दत्त्वाऽश्वमेधकृत्} %॥१४७॥

\twolineshloka
{गोधूमाः सर्वजन्तूनां बलपुष्टिविवर्द्धनाः}
{मुख्याश्च हव्यकव्येषु तस्मान्मे ददतु श्रियम्} %॥१४८॥

\twolineshloka
{आषाढादिचतुर्मासान्वृन्ताकं वर्जयेन्नरः}
{कारवेल्लफलं वापि तथालाबू पटोलकम्} %॥१४९॥

\twolineshloka
{यद्यत्फलं प्रियतरं तच्चापि परिवर्जयेत्}
{चातुर्मास्ये ततो वृत्ते रौप्याण्येतानि कारयेत्} %॥१५०॥

\twolineshloka
{मध्ये विद्रुमयुक्तानि ह्ययित्वा तु शक्तितः}
{दद्यादक्षिणया सार्द्ध ब्राह्मणायातिभक्तितः} %॥१५१॥

\twolineshloka
{अभिष्ट देवमुद्दिश्य देवो मे प्रीयतामिति}
{स दीर्घमायुरारोग्यं पुत्रपौत्रान्सुरूपताम्} %॥१५२॥

\twolineshloka
{अक्षय्यो सन्तात कीर्ति लब्ध्वा स्वर्गे महीयते}
{श्रावणे वर्जयेच्छाकं दधि भाद्रपदे तथा} %॥१५३॥

\twolineshloka
{दुग्धमाश्वयुजे मासि कातक द्विदलं त्यजेत्}
{चत्वार्येतानि नित्यानि चातुराश्रमवर्तिनाम्} %॥१५४॥

\twolineshloka
{कूष्माण्डं राजमाषांश्च मूलकं गृञ्जनं तथा}
{करमर्द चक्षुदण्डं चातुर्मास्ये त्यजन्नरः} %॥१५५॥

\twolineshloka
{मसूरं बहुबीजं च वृन्ताङ्क चैव वर्जयेत्}
{नित्यान्येतानि विप्रेन्द्र व्रतान्याहुर्मनीषिणः} %॥१५६॥

\twolineshloka
{विशेषाद्वदरी धात्रीमलाबु चिश्चिणी त्यजेत्}
{वार्षिकांश्चतुरो मासान्प्रसुप्ते च जनार्दने} %॥१५७॥

\twolineshloka
{मञ्चखट्या दिशयनं वर्जयेद्भक्तिमानरः}
{अनृतौ वर्जयेद्भार्यामृतौ गच्छन्न दुष्यति} %॥१५८॥

\twolineshloka
{मधुवल्ली च शिग्रं च चातुर्मास्ये त्यजेनरः}
{वृन्ताकं च कलिङ्गं च बिल्वोदुम्बरभिस्सष्टाः} %॥१५९॥

\twolineshloka
{उदरे यस्य जीर्यन्ते तस्य दूरतरो हरिः}
{उपवासं तथा नक्तमेकभक्तमयाचितम्} %॥११६॥

\twolineshloka
{अशक्तस्तु यथाकुर्यात्सायम्प्रातरखण्डितम्}
{स्नानपूजादिकं यस्तु स नरो हरिलोकभाक्} %॥१६१॥

\twolineshloka
{गीतवाद्यपरो विष्णोर्गान्धर्व लोकमाप्नुयात्}
{मधुत्यामी भवेहाजा पुरुषो गुडवर्जनात्} %॥१६२॥

\twolineshloka
{लभेच्च सन्तति दीर्घा पुत्रपौत्रादिवर्धिनीम्}
{तैलस्य वर्जनाद्राजन् सुदर्शाङ्गः प्रजायते} %॥१६३॥

\twolineshloka
{कौसुम्भतैलसन्त्यागाच्छत्रुनाशमवाप्नुयात्}
{मधूकतेलत्यागाच सुसौभाग्यफलं लभेत्} %॥१६४॥

\twolineshloka
{कटुतिक्ताम्लमधुरकषायलवणान् रसान्}
{वर्जयेत्स च वैरूप्यं दौर्गन्ध्यं नाप्नुयात्सदा} %॥१६५॥

\twolineshloka
{पुष्पादिभोगत्यागेन स्वर्गे विद्याधरो भवेत्}
{योगाभ्यासी भवेद्यस्तु स ब्रह्मपदवीमियात्} %॥१६६॥

\twolineshloka
{ताम्बूलवर्जनाद्रोगी सद्योमुक्तामयो भवेत्}
{पादाभ्यङ्गपरित्यागाच्छिरोऽभ्यङ्गस्य पार्थिव} %॥१६७॥

\twolineshloka
{दीप्तिमान्दीप्तकरणो यक्षद्रव्यपतिर्भवेत्}
{दधिदुग्धपरित्यागी गोलोकं लभते नरः} %॥१६८॥

\twolineshloka
{इन्द्रलोकमवाप्नोति स्थालीपाकविवर्जनात्}
{एकान्तरोपवासेन ब्रह्मलोके महीयते} %॥१६९॥

\twolineshloka
{चतुरो वार्षिकामासानखरोमाणि धारयेत्}
{कल्पस्थायी भवेद्राजन्स नरो नात्र संशयः} %॥१७०॥

\twolineshloka
{नमो नाराय गायेति जपित्वानन्तकं फलम्}
{विष्णुपादाम्बुजस्पर्शात्कृत्यकृत्यो भवेन्नरः} %॥१७१॥

\twolineshloka
{लक्षदक्षिगाभिः सेवते हरिमव्ययम्}
{हंसयुक्तविमानेन स याति वैष्णवीं पुरीम्} %॥१७२॥

\twolineshloka
{त्रिरात्रभोजनत्यागान्मोदते दिवि देववत्}
{परानवर्जनाद्राजन्देवो वै मानुषो भवेत्} %॥१७३॥

\twolineshloka
{प्राजापत्यं चरेद्यो वै चातुर्मास्ये व्रतं नरः}
{मुच्यते पातकै सवैत्रिविधैर्नात्र संशयः} %॥१७४॥

\twolineshloka
{तप्तकृच्छातिकृच्छ्राभ्यां यः क्षिपच्छयनं हरेः}
{स याति परमं स्थानं पुनरावृत्तिवर्जितम्} %॥१७५॥

\twolineshloka
{चान्द्राय गेन यो राजन्मिासचतुष्टयम्}
{दिव्यदेहो भवत्सोऽथ शिवलोकं च गच्छति} %॥१७६॥

\twolineshloka
{चातुर्मास्ये नरो यो वै त्यजेदन्नादिभक्षणम्}
{स गच्छेद्धरिसायुज्यं न भूयस्तु प्रजायते} %॥१७७॥

\twolineshloka
{भिक्षाभोजी नरो यो हि स भवेद्वेदपारगा}
{पयोव्रतेन यो राजन्क्षिपेन्मासचतुष्टयम्} %॥१७८॥

\twolineshloka
{तस्य वंशसमुच्छेदः कदाचिन्नोपपद्यते}
{पञ्चगव्याशनः पार्थ चान्द्रायणफलं लभेत्} %॥१७९॥

\twolineshloka
{दिनत्रयं जलत्यागान्न रोगैरभिभूयते}
{एवमादिवतेः पार्थ तुष्टिमायाति केशवः} %॥१८०॥

\fourlineindentedshloka
{दुग्धाब्धिवीचिशयने भगवाननन्तो}
{यस्मिन्दिने स्वपिति चाथ विबुध्यते च}
{तस्मिन्ननन्यमनसामुपवासमाजां}
{पुंसां ददाति च गतिं गरुडासनोऽसौ} %॥१८१॥

॥इति श्रीभविष्यपुराणे विष्णोः शयन्येकादशीमाहात्म्यं सम्पूर्णम्॥


\hyperref[sec:ekadashi_mahatmyam_vrata_raja]{\closesub}
\clearpage

\sect{श्रावण-कृष्ण-कामिका-एकादशी-माहात्म्यम्}
\label{sec:vrata-raja-shravana-krishna-kamika}

\uvacha{युधिष्ठिर उवाच}

\twolineshloka
{आषाढशुक्लपक्षे तु यदेवशयनव्रतम्}
{तन्मया श्रुतपूर्व हि पुराणे बहुविस्तरम्} %॥१॥

\twolineshloka
{श्रावणे कृष्णपक्षे तु किन्नामैकादशी भवेत्}
{एतत्कथय गोविन्द वासुदेव नमोऽस्तु ते} %॥२॥

\uvacha{श्रीकृष्ण उवाच}

\twolineshloka
{शृणु राजन् प्रवक्ष्यामि व्रतं पापप्रणाशनम्}
{नारदाय पुरा राजन् पृच्छते च पितामहः} %॥३॥

\onelineshloka*
{परं यदुलवांस्तात तदहं ते वदामि च}

\uvacha{नारद उवाच}

\onelineshloka
{भगवञ्छ्रोतुमिच्छामि त्वत्तोऽहं कमलासन} %॥४॥

\twolineshloka
{श्रावणस्यासिते पक्षे किनामैकादशी भवेत्}
{को देवः को विधिस्तस्याः किं पुण्यं कथय प्रभो} %॥५॥

\onelineshloka*
{इति तस्य वचः श्रुत्वा ब्रह्मा वचनमब्रवीत्}

\uvacha{ब्रह्मोवाच}

\onelineshloka
{शृणु नारद ते वच्मि लोकानां हितकाम्यया} %॥६॥

\twolineshloka
{श्रावणैकादशी कृष्णा कामिकेति च नामतः}
{तस्याः श्रवणमात्रेण वाजपेयफलं लभेत्} %॥७॥

\twolineshloka
{तस्यां यः पूजयेद्देवं शङ्खचक्रगदाधरम्}
{श्रीधराख्यं हारं विष्णुं माधवं मधुसूदनम्} %॥८॥

\twolineshloka
{यजते ध्यायतेऽथो वै तस्य पुण्यफलं शृणु}
{न गङ्गायां न काश्यां वै नैमिष न च पुष्करे} %॥९॥

\twolineshloka
{तत्फलं समवाप्नोति यत्फलं विष्णुपूजनात्}
{केदारे च कुरुक्षेत्रे राहुप्रस्ते दिवाकरे} %॥१०॥

\twolineshloka
{न तत्फलमवाप्नोति यत्फलं कृष्णपूजनात्}
{गोदावर्या गुरौ सिंहे व्यतीपाते च गण्डके} %॥११॥

\twolineshloka
{म तकलमवाप्नोति यत्फलं कृष्णपूजनात्}
{ससागरवनोपेतां यो ददाति वसुन्धराम्} %॥१२॥

\twolineshloka
{कामिकानतकारी च ह्युभौ समकलौ स्मृतौ}
{प्रलूयमानां यो धेनुं दद्यात्सोपस्करां नरः} %॥१३॥

\twolineshloka
{तत्फलं समवाप्नोति कामिकाबलकारकः}
{श्रावणे श्रीधरं देवं पूजयेद्यो नरोत्तमः} %॥१४॥

\twolineshloka
{तेनैव पूजिता देवा गन्धवारगपन्नगाः}
{तस्मात्सर्वप्रयत्नेन कामिकादिवसे हरिः} %॥१५॥

\twolineshloka
{पूजनीयो यथाशक्त्या मनुष्यः पापभीरूभिः}
{संसारार्णवमना ये पारपासनाकुलाः} %॥१६॥

\twolineshloka
{तेषामुद्धरणार्थाय कामिकावतमुत्तमम्}
{नातः परतरा काचित्पवित्रा पापहारिणी} %॥१७॥

\twolineshloka
{एवं नारद जानीहि स्वयमाह पुरा हरिः}
{अध्यात्मविद्यानिरर्यकलं प्राप्यते नरैः} %॥१८॥

\twolineshloka
{ततो बहुलरं विद्धि कामिकात्रतसेवनात्}
{रात्रौ जागरणं कुर्यात्कामिकावनकुन्नरः} %॥१९॥

\twolineshloka
{न पश्यति यम रौद्रं नैव पश्यति दुर्गतिम्}
{न गच्छनि कुयोनि च कामिकाव्रतसेवनात्} %॥२०॥

\twolineshloka
{कामिकाया व्रतेनैव कैवल्यं योगिनो गताः}
{तस्मात्सर्वप्रयलेन कर्नव्या नियतात्मभिः} %॥२१॥

\twolineshloka
{तुलसीप्रभवैः पत्रों नरः पूजयेद्धरिम्}
{न व स लिप्यते पापैः पद्मपत्रमिवाम्भसा} %॥२२॥

\twolineshloka
{सुवर्णभारमेकं तु रजतं च चतुर्गुणम्}
{तत्फलं समवाप्नोति तुलसीदलपूजनात्} %॥२३॥

\twolineshloka
{रत्नमौक्तिकवैदूर्यप्रवालादिमिर्चितः}
{न तुष्यति तथा विष्णुस्तुल लीपूजनाद्यथा} %॥२४॥

\twolineshloka
{तुलसीमनरीभिस्तु पूजितो येन केशवः}
{आजन्मकृतपापस्य तेन सम्मार्जिता लिपिः} %॥२५॥

\fourlineindentedshloka
{या दृष्टा निखिलाघसंवशमनी स्पृष्टा घः पावनी}
{रोगाणामभिवन्दिता निरलिनी सिकान्तकत्रासिनी}
{प्रत्यासतिविधाधिनी भगवतः कुष्गस्य संरोपिना}
{न्यस्ता तचर विनुकिकलदा तस्यै तुलस्यै नमः} %॥२६॥

\twolineshloka
{दीपं ददाति यो भयो दिवारात्रौ हरेर्दिने}
{तस्य पुण्यस्य सङ्ख्यानं चित्रगुप्तोऽपि वेचि न} %॥२७॥

\twolineshloka
{कृष्णाने दीको यस्य ज्वले देकादशीदिने}
{पितरस्तस्य तृप्यन्ति अमृतेन दिवि स्थिताः} %॥२८॥

\twolineshloka
{घृतेन दीप प्रज्वाल्य तिलतलेन वा पुनः}
{प्रयाति सूर्यलोकेऽसौ दीपकोटिशतैर्वृतः} %॥२९॥

\twolineshloka
{अयं तवाग्रे कथितः कामिकामहिमा मया}
{अतो नरेः प्रकर्तव्या सर्वपातकहारिणी} %॥३०॥

\twolineshloka
{ब्रह्महत्यापहरणी भ्रूणहत्याविनाशिनी}
{त्रिदिवस्थानदात्री च महापुण्यफलप्रदा} %॥३१॥

\twolineshloka
{श्रुत्वा माहात्म्यमेतस्या नरः श्रद्धासमन्वितः}
{विष्णुलोकमवाप्नोति सर्वपापैः प्रमुच्यते} %॥३२॥

॥इति श्रीब्रह्मवैवर्तपुराणे श्रावणकृष्णैकादश्याः कामिकाया माहात्म्यं सम्पूर्णम्॥


\hyperref[sec:ekadashi_mahatmyam_vrata_raja]{\closesub}
\clearpage

\sect{श्रावण-शुक्ल-पुत्रदा-एकादशी-माहात्म्यम्}
\label{sec:vrata-raja-shravana-shukla-putrada}

\uvacha{युधिष्ठिर उवाच}

\twolineshloka
{श्रावणस्य सिते पक्षे किनामैकादशी भवेत्}
{कथयस्व प्रसादेन ममाने मधुसूदन} %॥१॥

\uvacha{श्रीकृष्ण उवाच}

\twolineshloka
{शृणुष्वावहितो राजन् कथा पापहरा पराम्}
{यस्याः श्रवणमात्रेण वाजपेयफलं लभेत्} %॥२॥

\twolineshloka
{द्वापरल्य युगस्यादौ पुरा माहिष्मतीपुरे}
{राजा महीजिदाख्यातो राज्यं पालयति स्वकम्} %॥३॥

\twolineshloka
{पुत्रहीनस्य तस्यैव न तद्राज्यं सुखप्रदम्}
{अपुत्रस्य सुखं नास्ति इहलोके परत्र च} %॥४॥

\twolineshloka
{यततोऽस्य सुतप्राप्तौ कालो बहुतरो गतः}
{न प्राप्तश्च मुतो राज्ञा सर्वसौख्यप्रदो नृणाम्} %॥५॥

\twolineshloka
{दृष्ट्वात्मानं प्रवयसं राजा चिन्तापरोऽभवत्}
{सदोगतः प्रजामध्य इदं वचनमब्रवीत्} %॥६॥

\twolineshloka
{इहजन्मनि भो लोका न मया पातकं कृतम्}
{अन्यायोपार्जितं वित्तं क्षिप्तं कोशे मया न हि} %॥७॥

\twolineshloka
{ब्रह्मस्वं देवद्रविणं न गृहीतं मया क्वचित्}
{न्यासापहाग्ने न कृतः परस्य बहुपापदः} %॥८॥

\twolineshloka
{सुतवत्पालिता लोका धर्मेण विजिता मही}
{दुष्टेषु पातितो दण्डो बन्धुपुत्रोपमेष्वपि} %॥९॥


\threelineshloka
{शिष्टाः सुपूजिता लोका द्वेष्याश्चापि महाजनाः}
{इत्येवं व्रजते मार्गे धर्मयुक्ते द्विजोत्तमाः}
{कस्मान्मम गृहे पुत्रो न जातस्तद्विचार्यताम्} %॥१०॥

\twolineshloka
{इति वाक्य द्विजाः श्रुत्वा समजाः सपुरोहिताः}
{मन्त्रयित्वा नृपहितं जग्मुस्ते गहनं वनम्} %॥११॥

\twolineshloka
{इतस्ततश्च पश्यन्तश्चाश्रमानुषिसेवितान्}
{नृपतेहितमिच्छन्तो ददृशुर्मुनिसत्तमम्} %॥१२॥

\twolineshloka
{तप्यमानं तपो घोरं चिदानन्दं निरामयम्}
{निराहारं जितात्मानं जितक्रोध सनातनम्} %॥१३॥

\twolineshloka
{लोमशं धर्मतत्त्वज्ञं सर्वशास्त्रविशारदम्}
{दीर्घायुषं महात्मानमनेकब्रह्मसम्मितम्} %॥१४॥

\twolineshloka
{कल्पे गते यस्यैकस्मिन्नेक लोम विशीर्यते}
{अतो लोमशनामानं त्रिकालज्ञ महामुनिम्} %॥१५॥

\twolineshloka
{तं दृष्ट्वा हर्षिताः सर्वे आजग्मुस्तस्य सन्निधिम्}
{यथान्यायं यथार्ह ते नमश्चकुर्यथोदितम्} %॥१६॥

\twolineshloka
{विनयावनताः सर्वे ऊचुश्चैव परस्परम्}
{अस्मद्भाग्यवशादेव प्राप्तोऽयं मुनिसत्तमः} %॥१७॥

\onelineshloka*
{तांस्तथा प्रणतान्दृष्ट्वा ह्युवाच मुनिसत्तमः}

\uvacha{लोमश उवाच}

\onelineshloka
{किमर्थामह सम्प्राप्ताः कथयध्वं च कारणम्} %॥१८॥

\twolineshloka
{मद्दर्शनालादगिरा भवन्तः स्तुवते किमु}
{असंशयं करिष्यामि भवतां यद्वितं भवेत्} %॥१९॥

\onelineshloka*
{परोपकृतये जन्म मादृशानां न संशयः}

\uvacha{जना ऊचुः}

\onelineshloka
{श्रूयताममिधास्यामो वयमागमकारणम्} %॥२०॥

\twolineshloka
{संशयच्छेदनार्थाय तव सन्निधिमागताः}
{पद्मयोनेः परतरस्त्वत्तः श्रेष्ठो न विद्यते} %॥२१॥

\twolineshloka
{अतः कार्यवशात्प्राप्ताः समीपं भवतो वयम्}
{महीजिनाम राजासौ पुत्रहीनोस्ति साम्प्रतम्} %॥२२॥

\twolineshloka
{वयं तस्य प्रजा ब्रह्मन् पुत्रवत्तेन पालिताः}
{तं पुत्ररहितं दृष्ट्वा तस्य दुःखेन दुःखिताः} %॥२३॥

\twolineshloka
{तपः कर्तुमिहायाता मतिं कृत्वा तु नैष्ठिकीम्}
{तस्य भाग्यवशादृष्टस्त्वमस्माभिर्द्विजोत्तम} %॥२४॥

\twolineshloka
{महतां दर्शनेनैव कार्यसिद्धिर्भवेन्नृणाम्}
{उपदेशं वद मुने राज्ञः पुत्रो यथा भवेत्} %॥२५॥

\twolineshloka
{इति तेषां वचः श्रुत्वा मुहूर्त ध्यानमास्थितः}
{प्रत्युवाच मुनिख़त्वा तस्य जन्म पुरातनम्} %॥२६॥

\uvacha{लोमश उवाच}

\twolineshloka
{पूर्वजन्मनि वैश्योऽयं धनहीनो नृशंसकृत्}
{वाणिज्यकर्मनिरतो मामाद् प्रामान्तरं भ्रमन्} %॥२७॥

\twolineshloka
{ज्येष्ठे मासि सिते पक्ष द्वादशीदिवसे तथा}
{मध्याह्ने ह्युमणौ प्राप्ते ग्रामसीन्नि तृषाकुलः} %॥२८॥

\twolineshloka
{रम्यं जलाशयं दृष्ट्वा जलपाने मनो दधौ}
{सद्यःसूता सवत्सा च धेनुस्तत्र समागता} %॥२९॥

\twolineshloka
{तृषातुरा निदाघार्ता तस्य चापः पपौ तु सा}
{पिबन्ती वारयित्वा तामसी तोयं स्वयं पपौ} %॥३०॥

\twolineshloka
{कर्मणस्तस्य पाकेन पुत्रहीनो नृपोऽभवत्}
{पूर्वजन्मकृतात्पुण्यात्प्राप्तं राज्यमकण्टकम्} %॥३१॥

\uvacha{जना ऊचुः}


\twolineshloka
{पुग्यात्पापं क्षयं याति पुराणे श्रूयते मुने}
{पुण्योपदेशं कथय येन पापक्षयो भवेत्} %॥३२॥

\onelineshloka*
{यथा भवत्प्रसादेन पुत्रोऽस्य भविता तथा}

\uvacha{लोमश उवाच}

\onelineshloka
{श्रावणे शुक्लपक्षे तु पुत्रदानाम विश्रुता} %॥३३॥

\twolineshloka
{एकादशीतिथिश्चास्ति कुरुध्वं तद्वतं जनाः}
{यथाविधि यथान्याय्य यथोक्तं जागरान्वितम्} %॥३४॥

\twolineshloka
{तस्याः पुण्यं सुविमलं देयं नृपतये जनाः}
{एवं कृते सुनिया राजः पुत्रो भविष्यति} %॥३५॥

\threelineshloka
{श्रुत्वा तु लोमशवचस्तं प्रणम्य द्विजोत्तमम्}
{प्रजग्मुः स्वगृहान् सर्वे हषर्कोत्फुल्लविलोचनाः}
{श्रावणं तु समासाद्य स्मृत्वा लोमशभाषितम्} %॥३६॥

\twolineshloka
{राज्ञा सह व्रतं चक्रुः सर्वे श्रद्धासमन्विताः}
{द्वादशीदिवसे पुण्यं ददुपतये जनाः} %॥३७॥

\twolineshloka
{दत्ते पुण्ये तु सा राज्ञी गर्भमाधत्त शोभनम्}
{प्राप्ते प्रसवकाले सा सुषुवे पुत्रमूर्जितम्} %॥३८॥

\twolineshloka
{एवमेषा नृपश्रेष्ठ पुत्रदानाम विश्रुता}
{कर्तव्या सुखमिच्छद्भिरिह लोके परत्र च} %॥३९॥

\twolineshloka
{श्रुत्वा माहात्म्यमेतस्याः सर्वपापैः प्रमुच्यते}
{इह पुत्रसुखं प्राप्य परत्र स्वर्गतिं लभेत्} %॥४०॥

॥इति श्रीभविष्योत्तरपुराणे पुत्रदाख्यश्रावणशुक्कैकादशीमाहात्म्यं सम्पूर्णम्॥


\hyperref[sec:ekadashi_mahatmyam_vrata_raja]{\closesub}
\clearpage

\sect{भाद्रपद-कृष्णाजा-एकादशी-माहात्म्यम्}
\label{sec:vrata-raja-bhadrapada-krishnaja}

\uvacha{युधिष्ठिर उवाच}

\twolineshloka
{भाद्रस्य कृष्णपक्षे तु किन्नामैकादशी भवेत्}
{एनदिच्छाम्यहं श्रोतुं कथयस्व जनार्दन} %॥१॥

\uvacha{श्रीकृष्ण उवाच}

\twolineshloka
{शृणुष्वैकमना राजन् कथयिष्यामि विस्तरात्}
{अजेति नाम्ना विख्याता सर्वपापप्रणाशिनी} %॥२॥

\twolineshloka
{पूजयित्वा हृषीकेशं व्रतं तस्याः करोति यः}
{पापानि तस्य नश्यन्ति व्रतस्य श्रवणादपि} %॥३॥

\twolineshloka
{नातः परतरा राजैल्लोकद्वयहितावहा}
{सत्यमुक्तं मया ह्येतेन्नासत्यं भाषितं मम} %॥४॥

\twolineshloka
{हरिश्चन्द्र इति ख्यातो बभूव नृपतिः पुरा}
{चक्रवर्ती सत्यसन्धः समस्ताया भुवः पतिः} %॥५॥

\twolineshloka
{कस्यापि कर्मणो योगाद्राज्यभ्रष्टो बभूव सः}
{विक्रीय वनितां पुत्रं स चकारात्मविक्रयम्} %॥६॥

\twolineshloka
{पुल्कप्तस्य च दासत्वं गतो राजा स पुण्यकृत्}
{सत्यमालम्ब्य राजेन्द्र मृतचैलापहारकः} %॥७॥

\twolineshloka
{सोऽभवन्नृपतिश्रेष्ठो न सत्याञ्चलितस्तथा}
{एवं गतस्य नृपतेर्बहवो वत्सरा गताः} %॥८॥

\twolineshloka
{ततश्चिन्तापरो राजा बभूवात्यन्तदुःखितः}
{किं करोमि क्व गच्छामि निष्कृतिमें कथं भवेत्} %॥९॥

\twolineshloka
{इति चिन्तयतस्तस्य मनस्य वृजिनार्णवे}
{आजगाम मुनिः कैश्चिज्ज्ञात्वा राजानमातुरम्} %॥१०॥

\twolineshloka
{परोपकरणार्थाय निर्मितो ब्रह्मणा द्विजः}
{स तं दृष्ट्वा द्विजवरं ननाम नृपसत्तमः} %॥११॥

\twolineshloka
{कृताञ्जलिपुटो भूत्वा गौतमस्याग्रतः स्थितः}
{कथयामास वृत्तान्तमात्मनो दुःखसंयुतम्} %॥१२॥

\twolineshloka
{श्रुत्वा नृपतिवाक्यानि गौतमो विस्मयान्वितः}
{उपदेशं नृपतये व्रतस्यास्य मुनिर्ददौ} %॥१३॥

\twolineshloka
{मासि भाद्रपदे राजन् कृष्णपक्षे तु शोभना}
{एकादशी समाख्याता अजानाम्रातिपुण्यदा} %॥१४॥

\twolineshloka
{तस्याः कुरु व्रतं राजपापनाशो भविष्यति}
{तव भाग्यवशादेषा सप्तमेऽति समागता} %॥१५॥

\twolineshloka
{उपवासपरो भूत्वा रात्री जागरणं कुरु}
{एवं तस्या व्रत चीर्णे सर्वपापक्षयो भवेत्} %॥१६॥

\twolineshloka
{तव पुण्यप्रभावेण चागतोऽहं नृपोत्तम}
{इत्येवं कथयित्वा तु मुनिरन्तरधीयत} %॥१७॥

\twolineshloka
{मुनिवाक्यं नृपः श्रुत्वा चकार व्रतमुत्तमम्}
{कृते तस्मिन्नते राज्ञः पापस्यान्तोऽभवत्क्षणात्} %॥१८॥

\twolineshloka
{श्रूयतां राजशार्दूल प्रभावोऽस्य व्रतस्य च}
{यदुःखं बहुभिर्वर्षेभोक्तव्यं तत्क्षयो भवेत्} %॥१९॥

\twolineshloka
{निस्तीर्ण दुःखो राजासीद्वतस्यास्य प्रभावतः}
{पत्न्या सह समायोगं पुत्रजीवनमाप सः} %॥२०॥

\twolineshloka
{देवदुन्दुभयो नेदुः पुष्पवर्षमभद्दिवः}
{एकादश्याः प्रभावेण प्राप्त राज्यमकण्टकम्} %॥२१॥

\twolineshloka
{स्वर्ग लेभे हरिश्चन्द्रः सपुरः सपरिच्छदः}
{ईदृग्विधं व्रतं राजन् ये कुर्वन्ति द्विजोत्तमाः} %॥२२॥

\twolineshloka
{सर्वपापविनिर्मुक्तास्त्रिदिवं यान्ति ते ध्रुवम्}
{पठनाच्छ्रवणाद्राजन्नश्वमेधफलं भवेत्} %॥२३॥

॥इति श्रीब्रह्माण्डपुराणे भाद्रपदकृष्णाया अजानाम्न्या एकादश्या माहात्म्यं सम्पूर्णम्॥


\hyperref[sec:ekadashi_mahatmyam_vrata_raja]{\closesub}
\clearpage

\sect{भाद्रपद-शुक्ल-पद्मा-एकादशी-माहात्म्यम्}
\label{sec:vrata-raja-bhadrapada-shukla-padma}

\uvacha{युधिष्ठिर उवाच}

\twolineshloka
{नभस्य सितपक्ष तु किन्नामैकादशी भवेत्}
{को देवः को विधिस्तस्याः किं पुण्यं च वदस्व नः} %॥१॥

\uvacha{श्रीकृष्ण उवाच}

\twolineshloka
{कथयामि महापुण्यां स्वर्गमोक्षदायिनीम्}
{वामनैकादशी राजन्सर्वपापहारां पराम्} %॥२॥

\twolineshloka
{इमामेव जयन्त्याख्यां प्राहुरेकादशीं नृप}
{यस्य श्रवणमात्रेण सर्वपापक्षयो भवेत्} %॥३॥

\twolineshloka
{पापिनां पापशमनं जयन्तीव्रतमुत्तमम्}
{नातः पर तरा राजन्न वै मोक्षप्रदायिनी} %॥४॥

\twolineshloka
{एतस्मात्कारणाद्राजन्कर्तव्या गतिमिच्छता}
{वैष्णव मम भक्तैस्तु मनुजैर्मत्परायणैः} %॥५॥

\twolineshloka
{नभस्ये वामनो यैस्तु पूजितस्तैर्जगत्रयम्}
{पूजितं नाव सन्देहस्ते यान्ति हरिसन्निधिम्} %॥६॥

\twolineshloka
{वामनः पूजितो येन कमलैः कमलेक्षणः}
{नभस्यसितपक्षे तु जयन्त्येकादशीदिने} %॥७॥

\twolineshloka
{तेनार्चितं जगत्सर्व वयो देवाः सनातनाः}
{एतस्मात्कारणाद्राजन्कर्तव्यो हरिवासरः} %॥८॥

\twolineshloka
{अस्मिन्कृते न कर्तव्यं किञ्चिदस्ति जगत्रये}
{अस्या प्रसुप्तो भगवानेत्यङ्गपरिवर्तनम्} %॥९॥

\onelineshloka*
{तस्मादेनां जनाः सर्वे वदन्ति परिवर्तिनीम्}

\uvacha{युधिष्ठिर उवाच}

\onelineshloka
{संशयोऽस्ति महान्मह्यं श्रूयतां च जनार्दन} %॥१०॥

\twolineshloka
{कथं सुप्तोऽसि देवेश कथं यास्यङ्गवर्तनम्}
{किमर्थ देवदेवेश बलिर्बद्धस्त्वयासुरः} %॥११॥

\twolineshloka
{सन्तुष्टाः पृथिवीदेवाः किमकुर्वञ्जनार्दन}
{को विधिः किं व्रतं चैव चातुर्मास्यमुपासताम्} %॥१२॥

\twolineshloka
{त्वयि मुप्ते जगनाथ किं कुर्वन्ति जनाः प्रभो}
{एतद्विस्तरतो ब्रूहि संशयं हर मे प्रभो} %॥१३॥

\uvacha{श्रीकृष्ण उवाच}

\twolineshloka
{श्रूयतां राजशार्दूल कथां पापहरी पराम्}
{बालवै दानवः पूर्वमासीत्रेतायुगे नृप} %॥१४॥

\twolineshloka
{अपूजयच्च मां नित्यं मद्भक्तो मत्परायणः}
{जपैस्तु विविधैः सूक्तैर्यजते मांस नित्यशः} %॥१५॥

\twolineshloka
{द्विजानां पूजको नित्यं यज्ञकर्मकृताशयः}
{परन्त्विन्द्रकृतद्वेषो देवलोकमजीजयत्} %॥१६॥

\twolineshloka
{महत्तमिह लोकश्च जितस्तेन महात्मना}
{विलोक्य च ततः सर्वे देवाः संहत्य मन्त्र यन्} %॥१७॥

\twolineshloka
{सवर्मिलित्वा गन्तव्यं देवं विज्ञापितुं प्रभुम्}
{ततश्च देवऋषिभिः साकमिन्द्रो गतः प्रभुम्} %॥१८॥

\twolineshloka
{शिरसा ह्यवनीं गत्वा स्तुत इन्द्रेण सूक्तिभिः}
{गुरुणा देवतैः सार्ध बहुधा पूजितो ह्यहम्} %॥१९॥

\twolineshloka
{ततो वामनरूपेण ह्यवतीर्णश्च पञ्चमः}
{अत्युग्ररूपेण तदा सर्वब्रह्माण्डरूपिणा} %॥२०॥

\onelineshloka*
{बालकेन जितः सोऽथ सत्यमालम्ब्य तस्थिवान्}

\uvacha{युधिष्ठिर उवाच}

\onelineshloka
{त्वया वामनरूपेण सोऽसुरश्च जितः कथम्} %॥२१॥

\onelineshloka*
{एतत्कथय देवेश मह्यं भक्ताय विस्तरात}

\uvacha{श्रीकृष्ण उवाच}

\onelineshloka
{मयाऽलीकेन स बालः प्रार्थितो बटुरूपिणा} %॥२२॥

\twolineshloka
{पदत्रयमिता भूमि देहि मे भुवनत्रयम्}
{दत्तं भवति ते राजन्नात्र कार्या विचारणा} %॥२३॥

\twolineshloka
{इत्युक्तश्च मया राजा दत्तवास्त्रिपदां भुवम्}
{सङ्कल्पमात्राद्विवृधे देहविक्रमः परम्} %॥२४॥

\twolineshloka
{भूलोके तु कृतौ पादौ भुवोङ्के तु जानुनी}
{स्वलोङ्के तु कटिं न्यस्य महर्लोके तथोदरम्} %॥२५॥

\twolineshloka
{जनलोके तु हृदयं तपोलोके च कण्ठकम्}
{सत्यलोके मुखं स्थाप्य उत्तमाङ्ग तथोर्ध्वतः} %॥२६॥

\twolineshloka
{चन्द्रसूर्यग्रहाश्चैव भगणो योगसंयुतः}
{सेन्द्राश्चैव तदा देवा नागाः शेषादयः परे} %॥२७॥

\twolineshloka
{अस्तुवन्वेदसम्भूतैः सूक्तैश्च विविधैस्तु माम्}
{करे गृहीत्वा तु बलिमब्रुवं वचनं तदा} %॥२८॥

\twolineshloka
{एकेन पूरिता पृथ्वी द्वितीयेन त्रिविष्टपम्}
{तृतीयस्य तु पादस्य स्थानं देहि ममानघ} %॥२९॥

\twolineshloka
{एवमुक्ते मया सोऽपि मस्तके दत्तवान्बलिः}
{ततो वै मस्तके ह्येकं पदं दत्तं मया तदा} %॥३०॥

\twolineshloka
{क्षिप्तो रसातले राजन्दानवो मम पूजकः}
{विनयावनतं दृष्ट्वा प्रसन्नोऽस्मि जनार्दनः} %॥३१॥

\twolineshloka
{बले वसामि सततं सन्निधौ तव मानद}
{इत्यवोचं महाभागं बालि वैरोचनि तदा} %॥३२॥

\twolineshloka
{नमस्यशुक्लपक्षे तु परिवर्तिनि वासरे}
{ममैका तत्र मूर्तिश्च बलिमाश्रित्य तिष्ठति} %॥३३॥

\twolineshloka
{द्वितीया शेषपृष्ठे वै क्षीराब्धौ सागरोत्तमे}
{सुप्ता भवति भो भूप यावच्चायाति कार्तिकी} %॥३४॥

\twolineshloka
{एतस्मात्कारणाद्राजन्कर्तव्यैषा प्रयत्नतः}
{एकादशी महापुण्या पवित्रा पापहारिणी} %॥३५॥

\twolineshloka
{अस्यां प्रसुप्तो भगवानेत्यङ्गपरिवर्तनम्}
{एतस्यां पूजयेदेवं त्रैलोक्यस्य पितामहम्} %॥३६॥

\twolineshloka
{दधिदानं प्रकर्तव्यं रौप्यतण्डुलसंयुतम्}
{रात्री जागरणं कृत्वा मुक्तो भवति मानवः} %॥३७॥

\twolineshloka
{एवं यः कुरुते राजनेकादश्या व्रतं शुभम्}
{सर्वपापहरं चैव भुक्तिमुक्तिप्रदायकम्} %॥३८॥

\threelineshloka
{स देवलोकं सम्प्राप्य भ्राजते चन्द्रमा यथा}
{शृणुयाच्चैव यो मर्त्यः कथा पापहरां पराम्}
{अश्वमेधसहस्रस्य फलं प्राप्नोति मानवः} %॥३९॥

॥इति श्रीस्कन्दपुराणे भाद्रपदशुक्लायाः परिवर्तिनीनामैकादश्या माहात्म्यं सम्पूर्णम्॥


\hyperref[sec:ekadashi_mahatmyam_vrata_raja]{\closesub}
\clearpage

\sect{आश्विन-कृष्णेन्दिरा-एकादशी-माहात्म्यम्}
\label{sec:vrata-raja-ashvina-krishnendira}

\uvacha{युधिष्ठिर उवाच}

\twolineshloka
{कथयस्व प्रसादेन ममाग्रे मधुसूदन}
{आश्विने कृष्णपक्षे तु किन्नामकादशी भवेत्} %॥१॥

\uvacha{श्रीकृष्ण उवाच}

\twolineshloka
{आश्विनस्यासिते पक्षे इन्दिरानाम नामतः}
{तस्या व्रतप्रभावेण महापापं प्रणश्यति} %॥२॥

\twolineshloka
{अधोयोनिगतानां च पितॄणां गतिदायिनी}
{शृणुष्वावहितो राजन्कथां पापहरी पराम्} %॥३॥

\twolineshloka
{यस्याः श्रवणमात्रेण वाजपेयफलं लभेत्}
{पुरा कृतयुगे राजा बभूव रिपुसूदनः} %॥४॥

\twolineshloka
{इन्द्रसेन इति ख्यातः पुरी माहिष्मती प्रति}
{सराज्यं पालयामास धर्मेण यशसान्वितः} %॥५॥

\twolineshloka
{पुत्रपौत्रसमायुक्तो धनधान्यसमन्वितः}
{माहिष्मत्यधिपो राजा विष्णुभक्तिपरायणः} %॥६॥

\twolineshloka
{जपन् गोविन्दनामानि मुक्तिदानि नराधिपः}
{ध्यानेन कालं नयति नित्यमध्यात्मचिन्तकः} %॥७॥

\twolineshloka
{एकस्मिन् दिवसे राज्ञि सुखासीने सदोगते}
{अवतीर्यागमद्धीमानम्बरानारदो मुनिः} %॥८॥

\twolineshloka
{तमागतमभिप्रेक्ष्य प्रत्युत्थाय कृताञ्जलिः}
{पूजयित्वाविधिना चासने सन्न्यवेशयत्} %॥९॥

\twolineshloka
{सुखोपविष्टः सः मुनिः प्रत्युवाच नृपोत्तमम्}
{कुशलं तव राजेन्द्र सप्तस्वङ्गेषु वर्तते} %॥१०॥

\twolineshloka
{धर्मे मतिर्वर्तते ते विष्णुभाक्तिरतिस्तथा}
{इति वाक्यं तु देवर्षेः श्रुत्वा राजा तमब्रवीत्} %॥११॥

\uvacha{राजोवाच}

\twolineshloka
{त्वत्प्रसादान्मुनिश्रेष्ठ सर्वत्र कुशलं मम}
{अद्य क्रतुक्रियाः सर्वाः सफलास्तव दर्शनात्} %॥१२॥

\twolineshloka
{प्रसादं कुरु विप्रर्षे ब्रह्मागमनकारणम्}
{इति राज्ञो वचः श्रुत्वा देवर्षिर्वाक्यमब्रवीत्} %॥१३॥

\uvacha{नारद उवाच}

\twolineshloka
{श्रूयतां राजशार्दूल मद्वचो विस्मयप्रदम्}
{ब्रह्मलोकादहं प्राप्तो यमलोकं द्विजोत्तम} %॥१४॥

\twolineshloka
{शमनेनार्चितो भक्त्या उपविष्टो वरासने}
{धर्मशील सत्यवांस्तु भास्करिं समुपासते} %॥१५॥

\twolineshloka
{बहुपुण्यप्रकर्ता च व्रतवैकल्यदोषतः}
{सभायां श्राद्धदेवस्य मया दृष्टः पिता तव} %॥१६॥

\twolineshloka
{कथितस्तेन सन्देशस्त निबोध जनेश्वर}
{इन्द्रसन इति ख्यातो राजा माहिष्मतीप्रभुः} %॥१७॥

\twolineshloka
{तस्याने कथय ब्रह्मन् स्थितं मां यमसन्निधौ}
{केनलि चान्तराये गर्वजन्मोद्वेन वै} %॥१८॥

\twolineshloka
{स्वर्ग प्रेषय मां पुत्र इन्दिरावतदानतः}
{इत्युकोऽहं समायातः समीरं तव पार्थिव} %॥१९॥

\twolineshloka
{पितुः स्वर्गतये राजनिन्दिरावतमाचर}
{तेन व्रतमभावेग स्वर्ग यास्यति ते पिता} %॥२०॥

\uvacha{राजोवाच}

\twolineshloka
{कथयस्व प्रसादेन भगवनिन्दिराव्रतम्}
{विधिना केन कर्तव्यं कस्मिन्पक्षे तिथौ तथा} %॥२१॥

\uvacha{नारद उवाच}

\twolineshloka
{शृगु राजन् हितं वच्मि व्रतस्यास्य विधिं शुभम्}
{आश्विनस्यासिते पक्षे दशमीदिवसे शुभे} %॥२२॥

\twolineshloka
{प्रातः स्नानं प्रकुर्वीत श्रद्वायु तेन चेतसा}
{ततो मध्याह्नसमये स्नानं कृत्वा बहिर्जल} %॥२३॥

\twolineshloka
{पितॄणां प्रीतये श्राद्धं कुर्याच्छ्रद्धासमविन्तः}
{एकभक्तं ततः कृत्वा रात्री भूमौ शयीत च} %॥२४॥

\twolineshloka
{प्रभाते विमले जाते प्राप्ते चैकादशीदिने}
{मुखप्रक्षालनं कुर्यादन्तधावनपूर्वकम्} %॥२५॥

\twolineshloka
{उपवासस्य नियम गृहीयाद्भक्तिभावतः}
{अद्य स्थित्वा निराहारः सर्वभोगविवर्जितः} %॥२६॥

\twolineshloka
{श्वो भोक्ष्ये पुण्डरीकाक्ष शरणं मे भवाच्युत}
{इत्येवं नियमं कृत्वा मध्याह्नसमये तथा} %॥२७॥

\twolineshloka
{शालग्रामशिला तु श्राद्धं कृत्वा यथाविधि}
{भोजयित्वा द्विजाछुद्वान्दक्षिणाभिः सुपूजितान्} %॥२८॥

\twolineshloka
{पितृशेष समावाय गवे दद्याद्विचक्षणः}
{पूजयित्वा हृषीकेशं धूपगन्धादिभिस्तथा} %॥२९॥

\twolineshloka
{रात्री जागरणं कुर्यात्केशवस्य समीपतः}
{ततः प्रभातसमये सम्प्राप्ते द्वादशीदिने} %॥३०॥

\twolineshloka
{अर्चयित्वा हरि भक्त्या भोजयित्वा द्विजानथ}
{बन्धुदौहित्रपुत्राद्यैः स्वयं भुञ्जीत वाङ्ग्यतः} %॥३१॥

\twolineshloka
{अनेन विधिना राजकुरु व्रतमतन्द्रितः}
{विष्णुलो के प्रयास्यन्ति पितरस्तव भूपते} %॥३२॥

\twolineshloka
{इत्युक्त्वा नृपतिं राजन् मुनिरन्तरधीयत}
{यथोक्तविधिना राजा चकार व्रतमुत्तमम्} %॥३३॥

\twolineshloka
{अन्तःपुरेण सहितः पुत्रभृत्यसमन्वितः}
{कृते व्रते तु कौन्तेय पुष्पवृष्टिरभूदिवः} %॥३४॥

\twolineshloka
{तत्पिता गरुडारूढो जगाम हरिमन्दिरम्}
{इन्द्रसेनौऽपि राजर्षिः कृत्वा राज्यमकण्टकम्} %॥३५॥

\twolineshloka
{राज्ये निवेश्य तनयं जगाम विदिवं स्वयम्}
{इन्दिरावतमाहात्म्यं तवाग्रे कथितं मया} %॥३६॥

\twolineshloka
{पठनाच्छ्रवणाचास्य सर्वपापैः प्रमुच्यते}
{भुक्त्वेह निखिलान्भोगाविष्णुलोके वसच्चिरम्} %॥३७॥

॥इति श्रीब्रह्मवैवर्तपुराणे आश्विनकृष्णैकादश्या इन्दिरानाम्न्या माहात्म्यं सम्पूर्णम्॥


\hyperref[sec:ekadashi_mahatmyam_vrata_raja]{\closesub}
\clearpage

\sect{आश्विन-शुक्ल-पाशाङ्कुशा-एकादशी-माहात्म्यम्}
\label{sec:vrata-raja-ashvina-shukla-pashankusha}

\uvacha{धुधिष्ठिर उवाच}

\twolineshloka
{कथयस्व प्रसादेन भगवन् मधुसूदन}
{इषस्य शुक्लपक्षे तु किन्नामैकादशी भवेत्} %॥१॥

\uvacha{श्रीकृष्ण उवाच}

\twolineshloka
{शृणु राजेन्द्र वक्ष्यामि माहात्म्यं पापनाशनम्}
{शुक्लपक्षे चाश्वयुजि भवेदकादशी तु या} %॥२॥

\twolineshloka
{पाशाङ्कुशति विख्याता सर्वपापहरा परा}
{पद्मनाभाभिधानं तु पूजयेत्तत्र मानवः} %॥३॥

\twolineshloka
{सर्वाभीष्टफलप्रात्प्यै स्वर्गमोक्षप्रदं नृणाम्}
{तपस्तप्त्वा नरस्ती चिरं सुनियतेन्द्रियः} %॥४॥

\twolineshloka
{यत्फलं समवाप्नोति तं नत्वा गरुडध्वजम्}
{कृत्वापि बहुशः पापं नरो मोहसमन्वितः} %॥५॥

\twolineshloka
{न याति नरकं घोरं नत्वा पापहरं हरिम्}
{पृथिव्यां यानि तीर्थानि पुण्यान्यायतनानि च} %॥६॥

\twolineshloka
{तानि सर्वाण्यवाप्नोति विष्णोर्नामानुकीर्तनात्}
{देवं शार्ङ्गधरं विष्णु ये प्रपन्ना जनार्दनम्} %॥७॥

\twolineshloka
{न तेषां यमलोकश्च नृणां वै जायते कचित्}
{उपोष्यैकादशीमेको प्रसङ्गनापि मानवाः} %॥८॥

\twolineshloka
{न यान्ति यातनां यामी पापं कृत्वापि दारुणम्}
{वैष्णवः पुरुषो भूत्वा शिवनिन्दा करोति यः} %॥९॥

\twolineshloka
{यो निन्देवैष्णवं लोके स याति नरकं ध्रुवम्}
{अश्वमेधसहस्राणि राजसूयशतानि च} %॥१०॥

\twolineshloka
{एकादश्युपवासस्य कलां नाहन्ति षोडशीम्}
{एकादशीसमं पुण्यं किञ्चिल्लोके न विद्यते} %॥११॥

\twolineshloka
{नेदृशं पावनं किञ्चित्रिषु लोकेषु विद्यते}
{यादृशं पद्मनाभस्य दिन पातकहानिदम्} %॥१२॥


\threelineshloka
{तावत्पापानि तिष्ठन्ति दहेऽस्मिन् मनुजाधिप}
{यावनोपोष्यते भक्त्या एद्मनाभदिनं शुभम्}
{व्याजेनोपोषितमपि न दर्शयति भास्करिम्} %॥१३॥

\twolineshloka
{स्वर्गमोक्षप्रदा ह्येषा शरीरारोग्यदायिनी}
{सुकलप्रदा ह्येषा धनधान्यप्रदायिनी} %॥१४॥

\twolineshloka
{न गङ्गा न गया राजन काशी न च पुष्करम्}
{न चापि कौरवं क्षेत्रं पुण्यं भूप हरेर्दिनात्} %॥१५॥

\twolineshloka
{रागे जागरणं कृत्वा समुपोष्य हरेर्दिनम्}
{अनायासेन भूपाल प्राप्यते वैष्णवं पदम्} %॥१६॥

\twolineshloka
{दश वै मातृके पक्षे दश राजेन्द्र पैतृके}
{प्रियाया दश पक्षे तु पुरुषानुद्धरेनरः} %॥१७॥

\twolineshloka
{चतुर्भुजा दिव्यरूपा नागारिकृतकेतनाः}
{स्रग्विणः पीतवस्त्राश्च प्रयान्ति हरिमन्दिरम्} %॥१८॥

\twolineshloka
{बालत्वे यौवने चैव वृद्धत्वेऽपि नृपोत्तम}
{उपोष्य द्वादशी नूनं नैति पापोऽपि दुर्गतिम्} %॥१९॥

\twolineshloka
{पाशाकुशामुपोष्यैव आश्विने चासितेतरे}
{सर्वपापविनिर्मुक्तो हरिलोकं स गच्छति} %॥२०॥

\twolineshloka
{दत्त्वा हेमतिलान् भूमि गामनमुदकं तथा}
{उपानद्वस्वच्छत्रादि न पश्यति यमं नरः} %॥२१॥

\twolineshloka
{यस्य पुण्यविहीनानि दिनान्यपगतानि च}
{स लोहकारभस्नेव श्वसनपि न जीवति} %॥२२॥

\twolineshloka
{अवन्ध्यं दिवसं कुर्यादरिद्रोऽपि नृपोत्तम}
{समाचरन्यथाशक्ति स्नानदानादिकाः क्रियाः} %॥२३॥

\twolineshloka
{तडागारामसौधानां सत्राणां पुण्यकर्मणाम्}
{कर्तारो नैव पश्यन्ति धीरास्तां यमयातनाम्} %॥२४॥

\twolineshloka
{दीर्घायुषो धनाढयाश्च कुलीना रोगवर्जिताः}
{दृश्यन्ते मारवा लोके पुण्यकर्त्तार ईदृशाः} %॥२५॥

\twolineshloka
{किमत्र बहुमोक्तेन यान्त्यधर्मेण दुर्गतिम्}
{आरोहन्ति दिवं धर्मेात्र कार्या विचारणा} %॥२६॥

\twolineshloka
{इति ते कथितं राजन् यत्पृष्टोऽहं त्वयानघ}
{पाशाङ्कुशाया माहात्म्यं किमन्यच्छोतुमिच्छसि} %॥२७॥

॥इति श्रीब्रह्माण्डपुराणे आश्विनशुक्लैकादश्याः पाशाङ्कुशाख्याया माहात्म्यं सम्पूर्णम्॥


\hyperref[sec:ekadashi_mahatmyam_vrata_raja]{\closesub}
\clearpage

\sect{कार्त्तिक-कृष्ण-रमा-एकादशी-माहात्म्यम्}
\label{sec:vrata-raja-karttika-krishna-rama}

\uvacha{युधिष्ठिर उवाच}

\twolineshloka
{कथयस्व प्रसादेन मम स्नेहाजनार्दन}
{कार्तिकस्यासिते पक्षे किन्नामेकादशी भवेत्} %॥१॥

\uvacha{श्रीकृष्ण उवाच}

\twolineshloka
{श्रूयतां राजशार्दूल कथयामि तवाग्रतः}
{कार्तिके कृष्णपक्षे तु रमानाम्नी सुशोभना} %॥२॥

\twolineshloka
{एकादशी समाख्याता महापापहरा परा}
{अस्याः प्रसअतो राजन् माहात्म्य प्रवदामि ते} %॥३॥

\twolineshloka
{मुचुकुन्द इति ख्यातो बभूव नृपतिः पुरा}
{देवेन्द्रेण समं यस्य मित्रत्वमभवनृप} %॥४॥

\twolineshloka
{यमेन वरुणेनैव कुबेरेण समं तथा}
{विभीषणेन चैतस्य सखित्वमभवत्सह} %॥५॥

\twolineshloka
{विष्णुभक्तः सत्यसन्धो बभूव नपतिः सदा}
{तस्यैव शासतो राजन राज्यं निहतकण्टकम्} %॥६॥

\twolineshloka
{बभूव दुहिता गेहे चन्द्रभागा सरिद्वरा}
{शोभनाय च सा दत्ता चन्द्रसेनसुताय वै} %॥७॥

\twolineshloka
{स कदाचित्समायानः श्वशुरस्य गृहे नृप}
{एकादशीव्रतमिदं समायातं सुपुण्यदम्} %॥८॥

\twolineshloka
{समागते व्रतदिने चन्द्रभागा त्वचिन्तयत्}
{किं भविष्यति देवेश मम भर्तातिदुर्बलः} %॥९॥

\twolineshloka
{क्षुधां सोढुं न शक्नोति पिता चैवोप्रशासनः}
{पटहस्तायने यस्य सम्प्राप्ते दशमीदिने} %॥१०॥

\twolineshloka
{न भोक्तव्यं न भोक्तव्यं न भोक्तव्यं हरेर्दिने}
{श्रुत्वा पटहनियों शोभनस्त्वब्रवीत्प्रियाम्} %॥११॥

\twolineshloka
{किं कर्तव्यं मया कान्ते ब्रूझुपायं सुशोभने}
{कृतेन येन में सम्यग्जीवितं न विनश्यति} %॥१२॥

\uvacha{चन्द्रभागोवाच}

\twolineshloka
{मत्पितुर्वेश्मनि विभो भोक्तव्यं नापि केनचित्}
{गजैरश्वैस्तथा चोष्ट्रैरन्यैः पशुभिरेत च} %॥१३॥

\twolineshloka
{तृणमन्नं तथा वारि न भोक्तव्य हरेर्दिने}
{मानवैश्च कुतः कान्त भुज्यते हरिवासरे} %॥१४॥

\twolineshloka
{यदि त्वं भोक्ष्यसे कान्त ततो गेहात्प्रयास्यताम्}
{एवं विचार्य मनसा सुदृढं मानसं कुरु} %॥१५॥

\uvacha{शोभन उवाच}

\twolineshloka
{सत्यमेतत्त्वया चोक्तं करिष्येऽहमुपोषणम्}
{देवेन विहितं यद्वै तत्तथैव भविष्यति} %॥१६॥

\twolineshloka
{इति दिष्टे मति कृत्वा चकार व्रतमुत्तमम्}
{क्षुत्तृषापीडितततुः स बभूवातिदुःखितः} %॥१७॥

\twolineshloka
{एवं व्याकुलित तस्मिन्त्रादित्योऽस्तमगा द्विरिम्}
{वैष्णवानां नराणां सा निशा हर्षविवर्धिनी} %॥१८॥

\twolineshloka
{हरिपूजारतानां च जागरासक्तचेतसाम्}
{बभूव नृपशार्दूल शोभनस्यातिदुःसहा} %॥१९॥

\twolineshloka
{रवेरुदयवेलायां शोभनः पञ्चतां गतः}
{दाहयामास राजा तं राजयोग्यैश्च दारुभिः} %॥२०॥

\twolineshloka
{चन्द्रभागा नात्मदेहं ददाह पिटवारिता}
{कृत्वौर्ध्वदेहिकं तस्य तस्थौ जनकवेश्मनि} %॥२१॥

\twolineshloka
{शोभनेन नृपश्रेष्ठ रमाव्रतप्रभावतः}
{प्राप्तं देवपुरं रम्यं मन्दराचलसानुनि} %॥२२॥

\twolineshloka
{अत् त्तममनाधृष्यमसङ्ख्येयगुणान्वितम्}
{हेमस्तम्भमयैः सौधे रत्नवैदूर्यमण्डितैः} %॥२३॥

\twolineshloka
{स्फाटिकार्वविधाकारोर्वचित्रैरुपशोभितम्}
{सिंहासनसमारूढः सुश्वेतच्छत्रचामरः} %॥२४॥

\twolineshloka
{किरीटकुण्डलयुतो हारकेयूरभूषितः}
{स्तूयमानश्च गन्धर्वैरप्सरोगणसवितः} %॥२५॥

\twolineshloka
{शोभः शोभते तत्र देवराडपरो यथा}
{सोमशमॆति विख्यातो मुचुकुन्दपुरे वसन्} %॥२६॥

\twolineshloka
{तीर्थयात्राप्रसङ्गेन भ्रमन् विप्रो ददर्श तम्}
{नृपजामातरं ज्ञात्वा तत्समीपं जगाम सः} %॥२७॥


\threelineshloka
{आसनादुत्थितः शीघ्रं नम चक्रे द्विजोत्तमम्}
{चकार कुशलप श्रेश्वशुरस्य नपश्य च}
{कान्तायाश्चन्द्रभागायास्तथैव नगरस्य च} %॥२८॥

\uvacha{सोमशर्मोवाच}

\twolineshloka
{कुशलं वर्तते राजञ्छशुरस्य गृहे तव}
{चन्द्रभागा कुशलिनी सर्वतः कुशलं पुरे} %॥२९॥

\twolineshloka
{स्ववृतं कथ्यतां राजन्नाश्चर्य परमं मम}
{पुरं विचित्रं रुचिरं न दृष्ट केनचित्क्वचित्} %॥३०॥

\onelineshloka*
{एतदाचक्ष्व नृपते कुतः प्राप्तमिदं त्वया}

\uvacha{शोभन उवाच}

\onelineshloka
{कार्तिकस्यासिते पक्षे नाना चैकादशी रमा} %॥३१॥

\twolineshloka
{तामुपोष्प मया प्राप्तं द्विजेन्द्रपुरमध्रुवम्}
{ध्रुवं भवति येनैव तत्कुरुष्व द्विजोत्तम} %॥३२॥

\uvacha{द्विजेन्द्र उवाच}

\twolineshloka
{कथमध्रुवमेतद्धि कथं हि भवति ध्रुवम्}
{तत्त्वं कथय राजेन्द्र तत्करिष्यामि नान्यथा} %॥३३॥

\uvacha{शोभन उवाच}

\twolineshloka
{मयतद्विहितं विप्र श्रद्धाहीनं व्रतोत्तमम्}
{तेनेदमधुवम् मन्ये ध्रुवं भवति तच्छृणु} %॥३४॥

\twolineshloka
{मुचुकुन्दस्य दुहिता चन्द्रभागा सुशोभना}
{तस्यै कथय वृत्तान्तं ध्रुवमेतद्भविष्यति} %॥३५॥

\twolineshloka
{तच्छृत्वाथ द्विजवरस्तस्यै सर्व न्यवेदयत्}
{श्रुत्वाथ सा द्विजवचो विस्मयोत्फुल्ललोचना} %॥३६॥

\onelineshloka*
{प्रत्यक्षमथवा स्वप्नस्त्वयैतत्कथ्यते द्विज}

\uvacha{सोमशर्मोवाच}

\onelineshloka
{प्रत्यक्षं पुत्रि ते कान्तो मया दृष्टो महावने} %॥३७॥

\twolineshloka
{देवतुल्यमनाधृष्यं दृष्टं तस्य पुरं मया}
{अध्रुवं तेन तत्प्रोक्तं ध्रुवं भवति तत्कुरु} %॥३८॥

\uvacha{चन्द्रभागोवाच}

\twolineshloka
{तत्र मां नय विप्रर्षे पतिदर्शनलालसाम्}
{आत्मनो व्रतपुण्येन करिप्यामि पुरं ध्रुवम्} %॥३९॥

\twolineshloka
{आवयोर्द्विज संयोगो यथा भवति तत्कुरु}
{प्राप्यते हि महत्पुण्यं कृतं योगे विमुक्तयोः} %॥४०॥

\twolineshloka
{इति श्रुत्वा सह तया सोमशर्मा जगाम ह}
{आश्रमं वामदेवस्य मन्दराचलसन्निधौ} %॥४१॥

\twolineshloka
{वामदेवोऽशृणोत्सर्व वृत्तान्तं कथितं तयोः}
{अभ्यषिञ्चञ्चन्द्रभागा वेदमन्त्ररैथोज्ज्वलाम्} %॥४२॥

\twolineshloka
{ऋषिमन्त्रप्रभावेण विष्णुवासरसेवनात्}
{दिव्यदेहा बभूवासो दिव्यां गतिमवाप ह} %॥४३॥

\twolineshloka
{पत्युः समीपमगमत्प्रहर्षोत्फुल्ललोचना}
{सहर्षः शोभनोऽतीव दृष्ट्वा कान्तां समागताम्} %॥४४॥

\twolineshloka
{समाहूय स्वके वामे पार्श्वे तो सन्न्यवेशयत्}
{सा चोवाच मिय हर्षाञ्चन्द्रभागा प्रियं वचः} %॥४५॥

\twolineshloka
{शृणु कान्त हितं वाक्यं यत्पुण्यं विद्यते मयि}
{अष्टधिका जाता यदाहं पितृवेश्मनि} %॥४६॥

\twolineshloka
{मया ततःप्रभृति च कृतमेकादशीव्रतम्}
{यथोक्तविधिसंयुक्तं श्रद्धायुक्तेन चेतसा} %॥४७॥

\twolineshloka
{तेन पुण्यप्रभावेण भविष्यति पुरं ध्रुवम्}
{सर्वकामसमृद्धं च यावदाभूतसम्प्लवम्} %॥४८॥

\twolineshloka
{एवं सा नृपशार्दूल रमते पतिना सह}
{दिव्यभोगा दिव्यरूपा दिव्याभरणभूषिता} %॥४९॥

\twolineshloka
{शोभनोऽपि तया सार्द्ध रमते दिव्यविग्रहः}
{रमावतप्रभावेण मन्दराचलसानुनि} %॥५०॥

\twolineshloka
{चिन्तामणिसमा ह्येषा कामधेनुसमाथवा}
{रमाभिधाना नृपते तवाने कथिता मया} %॥५१॥

\twolineshloka
{ईदृशं च व्रतं राजन् ये कुर्वन्ति नरोत्तमाः}
{ब्रह्महत्यादिपापानि नाशं यान्ति न संशयः} %॥५२॥

\twolineshloka
{एकादश्या रमाख्याया माहात्म्यं शृणुयानरः}
{सर्वपापविनिर्मुक्तो विष्णुलोके महीयते} %॥५३॥

॥इति श्रीब्रह्माण्डपुराणे कार्तिककृष्णाया रमाख्याया माहात्म्यम्॥


\hyperref[sec:ekadashi_mahatmyam_vrata_raja]{\closesub}
\clearpage

\sect{कार्त्तिक-शुक्ल-प्रबोधिनी-एकादशी-माहात्म्यम्}
\label{sec:vrata-raja-karttika-shukla-prabodhini}

\uvacha{ब्रह्मोवाच}

\twolineshloka
{प्रबोधिन्याश्च माहात्म्यं पापघ्नं पुण्यवर्धनम्}
{मुक्तिप्रदं सुबुद्धीनां शृणुष्व मुनिसत्तम} %॥१॥

\twolineshloka
{तावद्गर्जति विप्रेन्द्र गङ्गा भागीरथी क्षितौ}
{यावन्नायाति पापघ्नी कार्तिके हरिबोधिनी} %॥२॥

\twolineshloka
{तावद्गर्जन्ति तीर्थानि ह्यासमुद्रं सरांसि च}
{यावत्प्रबोधिनी विष्णोस्तिथि याति कार्तिकी} %॥३॥

\twolineshloka
{अश्वमेधसहस्राणि राजसूयशतानि च}
{एकेनेवोपवासेन प्रबोधिन्या लभेन्नरः} %॥४॥

\uvacha{नारद उवाच}

\twolineshloka
{एकभक्ते च किं पुण्यं किं पुण्यं नक्तभोजने}
{उपवासे च किं पुण्यं तन्मे ब्रूहि पितामह} %॥५॥

\uvacha{ब्रह्मोवाच}

\twolineshloka
{एकभक्तेन जन्मोत्थं नक्तेन द्विजनुर्भवम्}
{सप्तजन्मभवं पापमुपवासेन नश्यति} %॥६॥

\twolineshloka
{यदुर्लभं यदप्राप्यं त्रैलोक्ये न तु गोचरम्}
{तदप्यप्रार्थितं पुत्रं ददाति हरिबोधिनी} %॥७॥

\twolineshloka
{मेरुमन्दरमात्राणि पापान्युप्राणि यानि तु}
{एकेनैवोपवासेन दहते पापहारिणी} %॥८॥

\twolineshloka
{पूर्वजन्मसहस्रेस्तु य कर्म ह्युपार्जितम्}
{जागरस्तत्प्रबोधिन्यां दहते चूलराशिवत्} %॥९॥

\twolineshloka
{उपवास प्रबोधिन्यां यः करोति स्वभावतः}
{विधिवन्मुनिशार्दूल यथोक्तं लमते फलम्} %॥१०॥

\twolineshloka
{यथोक्तं सुकृतं यस्तु विधिवत्कुरुते नरः}
{स्वल्पं मुनिवर श्रेष्ठ मेरुतुल्यं भवेञ्च तत्} %॥११॥

\twolineshloka
{विधिहीनं तु यः कुर्यात्सुकृतं मेरुमात्रकम्}
{अणुमात्रं न चाप्नोति फलं धर्मस्म नारद} %॥१२॥

\twolineshloka
{ये ध्यायन्ति मनोवृत्त्या करिष्यामः प्रबोधिनीम्}
{तेषां विलीयते पापं पूर्वजन्मशतोद्भवम्} %॥१३॥

\twolineshloka
{समतीतं भविष्यं च वर्तमानं कुलायुतम्}
{विष्णुलोकं नयत्याश प्रबोधिन्यां तु जागरात} %॥१४॥

\twolineshloka
{वसन्ति पितरो हृष्टा विष्णुलोकेत्यलङ्कृताः}
{विमुक्ता नारकेदुखेः पूर्वकर्मसमुद्भवैः} %॥१५॥

\twolineshloka
{कृत्वा तु पातकं घोरं ब्रह्महत्यादिकं नरः}
{कृत्वा तु जागरं विष्णोधौतपापो भवेन्मुने} %॥१६॥

\twolineshloka
{दुष्प्राप्यं यत्फलं विप्रैरश्वमेधादिभिर्मखैः}
{प्राप्यते तत्सुखेनैव प्रबोधिन्यां तु जागरात्} %॥१७॥

\twolineshloka
{आप्लुत्य सर्वतीर्थेषु दत्त्वा गाः काञ्चनं महीम्}
{न तत्फलमवाप्नोति यत्कृत्वा जागरं हेरेः} %॥१८॥

\twolineshloka
{जातः स एवं सुकृती कुलं तेनैव पावितम्}
{कार्तिके मुनिशार्दूल कृता येन प्रबोधिनी} %॥१९॥

\twolineshloka
{यानि कानि च तीर्थानि त्रैलोक्ये सम्भवन्ति च}
{तानि तस्य गृहे सम्यग्यः करोति प्रबोधिनीम्} %॥२॥

\twolineshloka
{सर्वकृत्यं परित्यज्य तुष्टयर्थ चक्रपाणिनः}
{उपोष्यकादशी रम्यां कार्तिके हरिबोधिनीम्} %॥२१॥

\twolineshloka
{स ज्ञानी स च योगी च स तपस्वी जितेन्द्रियः}
{विष्णुप्रियतरा ह्येषा धर्मसारस्य दायिनी} %॥२२॥

\twolineshloka
{सकृदेनामुपोष्यैव मुक्तिभाक् च भवेन्नरः}
{प्रबोधिनीमुपोषित्वा न गर्भ विशते नरः} %॥२३॥

\twolineshloka
{कर्मणा मनसा वाचा पाएं यत्समुपार्जितम्}
{तत्क्षालयति गोविन्दः प्रबोधिन्यां तु जागरात्} %॥२४॥

\twolineshloka
{स्नानं दानं जपो होमः समुद्दिश्य जनार्दनम्}
{नरैर्यत् क्रियते वत्स प्रबोधिन्यां तदक्षयम्} %॥२५॥

\twolineshloka
{बनानेन देवेशं परितोष्य जनार्दनम्}
{विराजयन्दिशः सर्वाः प्रयाति भवनं हरेः} %॥२६॥

\twolineshloka
{बाल्ये यच्चाजितं वत्स यौवने वार्धके तथा}
{शतजन्मकृतं पापं स्वल्पं वा यदि वा बहु} %॥२७॥


\threelineshloka
{तत्क्षालयति गोविन्दो ह्यस्यामभ्यर्चितो मुने}
{चन्द्रसूर्योपरागे च यत्फलं परिकीर्तितम्}
{तत्सहस्रगुणं प्रोक्तं प्रबोधिन्यां तु जागरात} %॥२८॥

\twolineshloka
{जन्मप्रभृति यत्पुण्यं नरेणासादितं भवेत्}
{वृथा भवति तत्सर्वमकृते कार्तिकवते} %॥२९॥

\twolineshloka
{अकृत्वा नियमं विष्णोः कार्तिकं यः क्षिपेन्नरः}
{जन्मार्जितस्य पुण्यस्य फलं नाप्नोति नारद} %॥३०॥

\twolineshloka
{तस्मात्त्वया प्रयत्नेन देवदेवो जनार्दनः}
{उपासनीयो विप्रेन्द्र सर्वकामफलप्रदः} %॥३१॥

\twolineshloka
{परान्नं वर्जयेद्यस्तु कार्तिके विष्णुतत्परः}
{अवश्यं स नरो वत्स चान्द्रायणफलं लभेत्} %॥३२॥

\twolineshloka
{न तथा तुष्यते यज्ञैर्न दानैर्मुनिसत्तम}
{यथा शास्त्रकथालापैः कार्तिके मधुसूदनः} %॥३३॥

\twolineshloka
{ये कुर्वन्ति कथां विष्णोर्ये शृण्वन्ति समाहिताः}
{श्लोकाई श्लोकमेकं वा कार्तिके गोशतं फलम्} %॥३४॥

\twolineshloka
{श्रेयसे लोभबुद्धया वा यः करोति हरेः कथाम्}
{कार्तिके मुनिशार्दूल कुलानां तारयेच्छतम्} %॥३५॥

\twolineshloka
{नियमेन नरो यस्तु शृणुते वैष्णवीं कथाम्}
{कार्तिके तु विशेषेण गोसहस्रफलं लभेत्} %॥३६॥

\twolineshloka
{प्रबोधवासरे विष्णोः कुरुते यो हरेः कथाम्}
{सप्तद्वीपवतीदान फलं स लभते मुने} %॥३७॥

\twolineshloka
{कृत्वा विष्णुकथां दिव्यां येऽर्चयन्ति कथाविदम्}
{स्वशक्त्या सुनिशार्दूल तेषां लोकाः सनातनाः} %॥३८॥

\onelineshloka*
{ब्रह्मणो वचनं श्रुत्वा नारदः पुनरबवीत्}

\uvacha{नारद उवाच}

\onelineshloka
{विधानं ब्रूहि मे स्वामिन्नेकादश्याः सुरोत्तम} %॥३९॥

\twolineshloka
{चीर्णन येन भगवन्यादृशं फलमाप्नुयात्}
{नारदस्य वचः श्रुत्वा ब्रह्मा वचनमब्रवीत्} %॥४०॥

\uvacha{ब्रह्मोवाच}

\twolineshloka
{ब्राह्म मुहूर्ते चोत्थाय ह्येकादश्यां द्विजोत्तम}
{स्मानं चैव प्रकर्तव्यं दन्तधावनपूर्वकम्} %॥४१॥

\twolineshloka
{नद्या तडागे कूपे वा वाप्यां गेहे तथैव च}
{नियमार्थे महाभाग इमं मन्त्रमुदीरयेत्} %॥४२॥

\twolineshloka
{एकादश्यां निराहारः स्थित्वाऽहनि परे ह्यहम्}
{भोक्ष्यामि पुण्डरीकाक्ष शरणं मे भवाच्युत} %॥४३॥

\twolineshloka
{गृहीत्वानेन नियमं देवदेवं च चक्रिणम्}
{सम्पूज्य भक्त्या तुष्टात्मा झुपवासं समाचरेत्} %॥४४॥

\twolineshloka
{रात्री जागरणं कुर्यादेवदेवस्य सविधौ}
{गीतं नृत्यं च वाद्यं च तथा कृष्णकां मुने} %॥४५॥

\twolineshloka
{बहुपुष्पैर्बहुफलैः कर्पूरागुरुकुङ्कुमैः}
{हरेः पूजा विधातव्या कार्तिक्या बोधवासरे} %॥४६॥

\twolineshloka
{वित्तशाष्ठचं न कर्तव्यं सम्प्राप्त हरिवासरे}
{फलैर्नानाविर्दिव्यैः प्रबोधिन्यां तु भक्तितः} %॥४७॥

\twolineshloka
{शङ्खतोयं समादाय ह्यों देयो जनार्दने}
{यत्फलं सर्वतीर्थेषु सर्वदाने यत्कलम्} %॥४८॥

\twolineshloka
{तत्फलं कोटिगुणितं दत्तेऽधे बोधवासरे}
{अगस्त्य कुसुमैर्देवं पूजयेद्यो जनार्दनम्} %॥४९॥

\twolineshloka
{देवेन्द्रोऽपि तदने च करोति करसम्पुटम्}
{न तत्करोति विप्रेन्द्र तपसा तोषितो हरिः} %॥५०॥

\twolineshloka
{यत् करोति हृषीकेशो मुनिपुष्परलकृतः}
{बिल्वपत्रैश्च ये कृष्णं कार्तिके कलिवर्द्धन} %॥५१॥

\twolineshloka
{पूजयन्ति महाभक्त्या मुक्तिस्तेषां मयोदिता}
{तुलसीदलपुष्पैर्ये पूजयन्ति जनार्दनम्} %॥५२॥

\twolineshloka
{कार्तिके स दहेत्तेषां पापं जन्मायुतोद्भवम्}
{दृष्टा स्पृष्टाथवा ध्याता कीर्तिता नमिता स्तुता} %॥५३॥

\twolineshloka
{रोपिता सचिता नित्यं पूजिता तुलसी शुभा}
{नवधा सेविता भक्त्या कार्तिके यर्दिनदिने} %॥५४॥

\twolineshloka
{युगकोटिसहस्राणि ते वसन्ति हरेहे}
{रोपिता तुलसी यैस्तु व ते वसुधातले} %॥५५॥

\twolineshloka
{कुले तेषां तु ये जाता ये भविष्यन्ति ये गताः}
{आकल्पयुगसाहस्रं तेषां वासो हरेहे} %॥५६॥

\twolineshloka
{कदम्बकुसुमैर्देवं येऽर्चयन्ति जनार्दनम्}
{तेषां यमालयो नैव प्रसादाचकपाणिनः} %॥५७॥

\twolineshloka
{दृष्ट्वा कदम्बकुसुमं प्रीतो भवति केशवः}
{किं पुनः पूजितो विप्र सर्वकामप्रदो हरिः} %॥५८॥

\twolineshloka
{यःपुनः पाटलापुष्पैः कार्तिके गरुडध्वजम्}
{अर्चयेत्परया भक्त्या मुक्तिभागी भवेद्धि सः} %॥५९॥

\twolineshloka
{बकुलाशोककुसुक्र्येऽर्चयन्ति जगत्पतिम्}
{विशोकास्ते भविष्यन्ति यावचन्द्रदिवाकरौ} %॥६०॥

\twolineshloka
{येऽर्चयन्ति जगन्नाथं करवीरैः सितासितैः}
{तेषां सदा तु विप्रेन्द्र प्रीतो भवति केशवः} %॥६॥

\twolineshloka
{मञ्जरी सहकारस्य केशवोपरि ये नराः}
{यच्छन्ति ते महाभागा गोकोटिफलभागिनः} %॥६॥

\twolineshloka
{दूर्वाकुरेहरेर्यस्तु पूजाकाले प्रयच्छति}
{पूजाफलं शतगुणं सम्यगामोति मानवः} %॥६३॥

\twolineshloka
{शमीपत्रैस्तु ये देवं पूजयन्ति सुखप्रदम्}
{यममार्गो महाघोरो निस्तीर्णस्तैस्तु नारद} %॥६४॥

\twolineshloka
{वर्षाकाले तु देवेश कुसुमैश्चम्पकोद्भवैः}
{येऽर्चयन्ति न ते माः संसरेयुः पुनर्भवे} %॥६५॥

\twolineshloka
{सुवर्णकेतकीपुष्पं यो ददाति जनार्दने}
{कोटिजन्मार्जितं पापं दहते गरुडध्वजः} %॥६६॥

\twolineshloka
{कुङ्कुमारुणवर्णा च गन्धाच्या शतुपत्रिकाम्}
{यो ददाति जगनाथे श्वेतदीपालये वसेत्} %॥६७॥

\twolineshloka
{एवं सम्पूज्य रात्रौ च केशवं भुक्तिमुक्तिदम्}
{प्रातरुत्थाय च ब्रह्मन् गत्वा तु सजला नदीम्} %॥६८॥

\twolineshloka
{तत्र स्नात्वा जपित्वा च कृत्वा पौर्वाहिकी क्रियाः}
{गृहं गत्वा च सम्पूज्यः केशवो विधिवन्नरैः} %॥६९॥

\twolineshloka
{व्रतस्य पूरणार्थाय ब्राह्मणान्भोजयेत्सुधीः}
{क्षमापयेत्सुवचसा भक्तियुक्तेन चेतसा} %॥७०॥

\twolineshloka
{गुरुपूजा ततः कार्या भोजनाच्छादनाभिः}
{दक्षिणा गौश्च दातव्या तुष्टयर्थ चक्रपाणिनः} %॥७१॥

\twolineshloka
{भूयसी चैव दातव्या ब्राह्मणेभ्यः प्रयत्नतः}
{नियमश्चैव सन्त्याज्यो ब्राह्मणाने प्रयत्नतः} %॥७२॥

\twolineshloka
{कथयित्वा द्विजेभ्यस्तु दद्याच्छक्त्या च दक्षिणाम्}
{नक्तभोजी नरो राजन् ब्राह्मणान् भोजयेच्छुभान्} %॥७३॥

\twolineshloka
{अयाचिते बलीव सहिरण्यं प्रदापयेत्}
{अमांसाशी नरो यस्तु प्रददेद्रां सदक्षिणाम्} %॥७४॥

\twolineshloka
{धात्रीस्नायी मरो दद्यादधि माक्षिकमेव च}
{फिलानां नियमे राजन् फलदान समाचरेत्} %॥७५॥

\twolineshloka
{तेलस्थाने घृतं देयं घृतस्थाने पयः स्मृतम्}
{धान्यानां नियमे राजन् दीयन्ते शालिगण्डुलाः} %॥७६॥

\twolineshloka
{दद्याद्र्शयने शय्यां सतूला सपरिच्छदाम्}
{पत्रभोजी नरो दद्याद्भाजनं वृतसंयुतम्} %॥७७॥

\twolineshloka
{मौने घण्टा तिलांश्चैव सहिरण्यं प्रदापयेत्}
{धारणे तु स्वकेशानामादर्श दारयेद्बुधः} %॥७८॥

\twolineshloka
{उपानहौ प्रदातव्यावुपामत्परिवर्जनात्}
{लवणस्य च सन्त्यागे शर्करां च प्रदापयेत्} %॥७९॥

\twolineshloka
{नित्यं दीपप्रदो यस्तु विष्णोर्वा विबुधालये}
{सदीपं सवृतं तानं काञ्चनं वा दशायुतम्} %॥८०॥

\twolineshloka
{प्रदद्याद्विष्णुभक्ताय व्रतसम्पूर्तिहेतवे}
{एकान्तरोपवासे तु कुम्भानष्टौ प्रदापयेत्} %॥८१॥

\twolineshloka
{सवनाकाञ्चनोपेतान् सर्वान् सालङ्कृताञ्छुभान्}
{यथोक्तकरणे शक्तिर्यदि न स्यात्तदा मुने} %॥८२॥

\twolineshloka
{द्विजवाक्यं स्मृतं राजन् सम्पूर्णवतसिद्धिदम्}
{नत्वा विसर्जयेद्विप्रास्ततो भुनीत च स्वयम्} %॥८३॥

\twolineshloka
{यत्त्यक्तं चतुरो मासान् समाप्तिं तस्य चाचरेत्}
{एवं य आचरेत्पार्थ सोनन्तफलमाप्नुयात्} %॥८४॥

\twolineshloka
{अवसाने तु राजेन्द्र वासुदेवपुरं व्रजेत्}
{यश्चाविन्नं समाप्यैवं चातुर्मास्यव्रतं नृप} %॥८५॥

\twolineshloka
{स भवेत्कृतकृत्यस्तु न पुनर्मानुषो भवेत्}
{एतत्कृत्वा महीपाल परिपूर्ण व्रतं भवेत्} %॥८६॥

\threelineshloka
{व्रतवैकल्यमासाद्य ह्यन्धः कुष्ठी प्रजायते}
{एतत्ते सर्वमाख्यातं यत्पृष्टोऽहमिह त्वया}
{पठनाच्छवणाद्वापि लभेगोदान फलम्} %॥८७॥

॥इति श्रीस्कान्दे महापुराणे कार्त्तिकशुक्लप्रबोधिन्येकादाशीमाहात्म्यं सम्पूर्णम्॥


\hyperref[sec:ekadashi_mahatmyam_vrata_raja]{\closesub}
\clearpage

\sect{पुरुषोत्तम-मासस्य शुक्ल-कामदा-एकादशी-माहात्म्यम्}
\label{sec:vrata-raja-purushottama-shukla-kamada}

\uvacha{युधिष्ठिर उवाच}

\twolineshloka
{मलिन्लुचस्य मासस्य का वा एकादशी भवेत्}
{किं नाम को विधिस्तस्याः कथयस्व जनार्दन} %॥१॥

\uvacha{श्रीकृष्ण उवाच}

\twolineshloka
{मलमासस्य या पुण्या पोका नाम्ना च पशिनी}
{सोपोषिता प्रयत्नेन पद्मनाभपुरं नयेत्} %॥२॥

\twolineshloka
{मलमासे महापुण्या कीर्तिता कल्मषापहा}
{तस्याः फलं कथयितुं न शक्तश्चतुराननः} %॥३॥

\twolineshloka
{नारदाय पुरा प्रोक्तं विधिना व्रतमुत्तमम्}
{पद्मिन्याः पापराशिनं भुक्तिमुक्तिफलप्रदम्} %॥४॥

\twolineshloka
{श्रुत्वा वाक्यं मुरारेस्तु प्रोवाचातिमुदान्वितः}
{युधिष्ठिरो जगन्नाथं विधि पप्रच्छ धर्मवित्} %॥५॥

\twolineshloka
{श्रुत्वा राज्ञस्तु क्चनमुवाच मधुसूदनः}
{शृणु राजन्प्रवक्ष्यामि मुनीनामप्यगोचरम्} %॥६॥

\twolineshloka
{दशमीदिवसे प्राप्ते व्रतारम्भो विधीयते}
{कांस्यं मांसं मसूरांश्च चणकान्कोद्रवांस्तथा} %॥७॥

\twolineshloka
{शाकं मधु परान्नं च दशम्यामष्ट वर्जयेत्}
{हविष्यान्नं च भुञ्जीत अक्षारलवणं तथा} %॥८॥

\twolineshloka
{भूमिशायी ब्रह्मचारी भवेच दशमीदिने}
{एकादशीदिने प्राप्ते प्रातरुत्थाय सादरम्} %॥९॥

\twolineshloka
{विधाय च मलोत्सर्ग न कुर्यादन्तधावनम्}
{कृत्वा द्वादशगण्डूषाञ्छुचिर्भूत्वा समाहितः} %॥१०॥

\twolineshloka
{सूर्योदये शुभे तीर्थे स्नानार्थ प्रवजेत्तुधीः}
{गोमयं मृत्तिकां गृह्य तिलान्दर्भाञ्छुचिस्तथा} %॥११॥

\twolineshloka
{चूर्णैरामलकीभूतेविधिना स्नानमाचरेत्}
{उद्धृतासि वराहेण कृष्णेन शतबाहुना} %॥१२॥

\twolineshloka
{मृत्तिके ब्रह्मदत्तासि काश्यपेनाभिमन्त्रिता}
{हरिपूजनयोग्यं मां मृत्तिके कुरु ते नमः} %॥१३॥

\twolineshloka
{सर्वोषधिसमुत्पन्नं गवोदरमधिष्ठितम्}
{पवित्रकरणं भूमी पावयतु गोमयम्} %॥१४॥

\twolineshloka
{ब्रह्मष्ठीवनसम्भूता धात्री भुवनपावनी}
{संस्पृष्टा पावयाङ्ग मे निर्मलं कुरु ते नमः} %॥१५॥

\twolineshloka
{देव देव जगन्नाथ शङ्खचक्रगदाधर}
{देहि विष्णो ममानुज्ञा तव तीर्थावगाहने} %॥१६॥

\twolineshloka
{वारुणांश्च जपेन्मन्त्रान् स्नान कुर्याः द्विधानतः}
{गङ्गादितीर्थ संस्मृत्य यत्र कुत्र जलाशय} %॥१७॥

\twolineshloka
{पश्चात्सम्मार्जयेगात्रं विधिना नृपसत्तम}
{परिधायाहतं वासः शुक्लं शुचि ह्यखण्डितम्} %॥१८॥

\twolineshloka
{सन्ध्यामुपास्य विधिना तर्पयित्वा पितॄन्सुरान्}
{हरेर्मन्दिरमागम्य पूजयेत्कमलापतिम्} %॥१९॥

\twolineshloka
{स्वर्गमाषकृतं देवं राधिकासहितं हरिम्}
{पार्वत्या सहितं शम्भु पूजयेद्विधिपूर्वकम्} %॥२०॥

\twolineshloka
{धान्योपरि न्यसेत्कुम्भ तानं मुन्मयमेव वा}
{दिव्यवस्त्रसमायुक्तं दिव्यगन्धानुवासितम्} %॥२१॥

\twolineshloka
{तस्योपरि न्यसेत् पात्र तानं रौप्यं हिरण्मयम्}
{तस्मिन्संस्थापयेदेवं विधिना पूजयेत्ततः} %॥२२॥

\twolineshloka
{सन्त्राप्य सलिलै श्रेष्ठेर्गन्धधूपाधिवासितैः}
{चन्दनागुरुकरैः पूजयेदेवमीश्वरम्} %॥२३॥

\twolineshloka
{नानाकुसुमकस्तूरीकुङ्कुमन सिताम्बुजैः}
{तत्कालजातैः कुसुमैः पूजयेत्परमेश्वरम्} %॥२४॥

\twolineshloka
{नैवेद्यैर्विविधैः शक्त्या तथा नीराजनादिभिः}
{धूपर्दी पैः सकपूरैः पूजयेत्केशवं शिवम्} %॥२५॥

\twolineshloka
{नृत्यं गीतं तदने तु कुर्याद्भक्तिपुरःसरम्}
{नालपेत्पतितान्पापांस्तस्मिन्नहनि न स्पृशेत्} %॥२६॥

\twolineshloka
{नानृतं हि वदेद्वाक्यं सत्यपूतं वचो वदेत्}
{रजस्वला न स्पृशेच न निन्देवाह्मणं गुरुम्} %॥२७॥

\twolineshloka
{पुराणं पुरतो विष्णोः शृणुयात्सह वैष्णवैः}
{निर्जला सा प्रकर्तव्या या च शुक्के मलिम्लुचे} %॥२८॥

\twolineshloka
{जलपानेन वा कुर्याद् दुग्धाहारण नान्यथा}
{रात्री जागरणं कुर्याद्गीतवादित्रसंयुतम्} %॥२९॥

\twolineshloka
{प्रहरे प्रहरे पूजा कार्या विष्णोः शिवस्य च}
{प्रथमे प्रहर इद्यानारिकेलार्घमुत्तमम्} %॥३०॥

\twolineshloka
{द्वितीये श्रीकलैश्चैव तृतीये बीजपूरकैः}
{चतुर्थप्रहरे पूगै रिङ्गैश्च विशेषतः} %॥३१॥

\twolineshloka
{प्रथम प्रहरे पुण्यमाग्निष्टोनस्य जायते}
{द्वितीये वाजपेयस्य तृतीये हयमेधजम्} %॥३२॥

\twolineshloka
{चतुथें राजसूयस्य जाप्रतो जायते फलम्}
{नातः परतरं पुण्यं नातः परतरा मखाः} %॥३३॥

\twolineshloka
{नातः परतरा विद्या नातः परतरं तपः}
{पृथिव्यां यानि तीर्थानि क्षेत्राण्यायतनानि च} %॥३४॥

\twolineshloka
{तेन स्नातानि दृष्टानि येनाकारि हरेवतम्}
{एवं जागरणं कुर्याद्यावत्सूर्योदयो भवेत्} %॥३५॥

\twolineshloka
{सूर्योदये शुभे तीर्थे गत्वा स्नानं समाचरेत्}
{स्नात्वा चामत्य भवनं पूजयेदेवमीश्वरम्} %॥३६॥

\twolineshloka
{पूर्वोदितन विधिना भोजयेद्ब्राह्मणाञ्छुभान्}
{कुम्भादिकं च यत्सर्व प्रतिमा केशवस्य च} %॥३७॥

\twolineshloka
{पूजयित्वा विधानेन ब्राह्मणाय समर्पयेत्}
{एवंविधं व्रतं यो वै कुरुते भुवि मानवः} %॥३८॥

\twolineshloka
{सफलं जायते जन्म तस्य मुक्तिफलप्रदम्}
{एतत्ते सर्वमाख्यातं यत्पृष्टोऽहं त्वयानघ} %॥३९॥

\twolineshloka
{व्रतानि तेन चीर्णानि सर्वाणि नृपनन्दन}
{पद्मिन्याः प्रीतियुक्तो यः कुरुते व्रतमुत्तमम्} %॥४०॥

\twolineshloka
{अब ते कथयिष्यामि कथामेकां मनोरमाम्}
{नारदाय पुलस्त्येन विस्तरैण निवोदिताम्} %॥४१॥

\twolineshloka
{कार्तवीर्येण कारायां निक्षिप्तं वीक्ष्य रावणम्}
{विमोचितः पुलस्त्येन याचयित्वा महीपतिम्} %॥४२॥

\twolineshloka
{तदाश्चर्य तदा श्रुत्वा नारदो दिव्यदर्शनः}
{पप्रच्छ च यथाभक्त्या पुलस्त्यं मुनिपुङ्गवम} %॥४३॥

\uvacha{नारद उवाच}

\twolineshloka
{दशाननेन विजिताः सर्वे देवाः सवासवाः}
{कार्तवीर्येण विजिताः कथं रणविशारदः} %॥४४॥

\onelineshloka*
{नारदस्य वचः श्रुत्वा पुलस्त्यो मुनिरब्रवीत्}

\uvacha{पुलस्त्य उवाच}

\onelineshloka
{शृणु वत्स प्रवक्ष्यामि कार्तवीर्यसमुद्भवम्} %॥४५॥

\twolineshloka
{पुरा त्रेतायुगे राजन्माहिष्मत्यां बृहत्तरः}
{हैहयानां कुले जातः कृतवीर्यों महीपतिः} %॥४६॥

\twolineshloka
{सहस्रं प्रमदास्तस्य नृपस्य प्राणवल्लभाः}
{न तासां तनयं काचिल्लेभे राज्यधुरन्धरम्} %॥४७॥

\twolineshloka
{यजन् देवानिपतृन्सिद्धान्प्रातपूज्य महत्तरान्}
{कुर्वस्तदुदितं सर्व लब्धवांस्तनयं न सः} %॥४८॥

\twolineshloka
{सुतं विना तदा राज्यं न सुखाय महीपतेः}
{क्षुधितस्य यथा भोगा न भवन्ति सुखप्रदाः} %॥४९॥

\twolineshloka
{विचार्य चित्ते नृपतिस्तपस्तप्तुं मनो दधे}
{तपसैव सदा सिद्धिर्जायते मनसेप्सिता} %॥५०॥

\twolineshloka
{इत्युक्त्वा स हि धर्मात्मा चीरवासा जटाधरः}
{तपस्तप्तुं गतः सद्यो गृहे न्यस्य सुमन्त्रिणम्} %॥५१॥

\twolineshloka
{निर्गतं नृपतिं वीक्ष्य पद्मिनी प्रमदोत्तमा}
{हरिश्चन्द्रस्य तनया तपस्तप्तुं कृतोद्यमम्} %॥५२॥

\twolineshloka
{भूषणादि परित्यज्य चीरमेकं समाश्रयत्}
{जगाम पतिना सार्द्ध पर्वते गन्धमादने} %॥५३॥

\twolineshloka
{गत्वा तत्र तपस्तेपे वर्षाणामयुतं नृपः}
{न लेभेऽथापि तनयं ध्यायन्देवं गदाधरम्} %॥५४॥

\twolineshloka
{अस्थिस्नायुमयं कान्तं दृष्ट्वा सा प्रमदोत्तमा}
{अनसूयां महासाध्वीं पप्रच्छ विनयान्विता} %॥५५॥

\twolineshloka
{भर्तुः प्रतपतः साध्वि वर्षाणामयुतं गतम्}
{तथापि न प्रसन्नोऽभूत्केशवः कष्टनाशनः} %॥५६॥

\twolineshloka
{व्रतं मम महाभागे कथयस्व यथातथम्}
{येन प्रसन्नो भगवान्भविष्यति सदा मयि} %॥५७॥

\twolineshloka
{येन जायेत मे पुत्रश्चक्रवर्ती महत्तरः}
{श्रुत्वा तस्यास्तु वचनं पतिव्रतपरायणा} %॥५८॥

\twolineshloka
{तदा प्रोवाच संहृष्टा पद्मिनी पद्मलोचनाम् मासो}
{मलिम्लुचः सुभ्र मासद्वादशकाधिकः} %॥५९॥

\twolineshloka
{द्वात्रिंशद्भिर्गतै सरा याति स शुभानने}
{तन्मध्ये द्वादशीयुग्मं पद्मिनी परमा तथा} %॥६०॥

\twolineshloka
{उपोष्य तत्प्रकर्तव्यं विधिना जागरैः समम्}
{शीघ्र प्रसन्नो भगवान् भविष्यति सुतप्रदः} %॥६१॥

\twolineshloka
{इत्युक्त्वाकथयत् सर्व मया पूर्वोदितं नृप}
{विधिव्रतस्य विधिवत्प्रसन्ना कर्दमाङ्गजा} %॥६२॥

\twolineshloka
{श्रुत्वा व्रतविधिसर्व यथोक्तमनसूयया}
{चक्रे राज्ञी च तत्सर्व पुत्रप्राप्तिमभीप्सती} %॥६३॥

\twolineshloka
{एकादश्यां निराहारा सदा जाता च निर्जला}
{जागरेण युता रात्रौ गीतनृत्यसमन्विता} %॥६४॥

\twolineshloka
{पूर्णे व्रते च वै शीघ्र प्रसन्नः केशवः स्वयम्}
{बभाषे गरुडारूढो वरं वरय शोभने} %॥६५॥

\twolineshloka
{श्रुत्वा वाक्यं जगद्धातुः स्तुत्वा प्रीत्या शुचिस्मिता}
{ययाचे द्य वरं देहि मम भर्तुर्वृहत्तरम्} %॥६६॥

\twolineshloka
{पद्मिन्या स्तद्वचः श्रुत्वा प्रत्युवाच जनार्दनः}
{यथा मालम्लुचो मासो नान्यो मे प्रीतिदायकः} %॥६॥

\twolineshloka
{तन्मध्यैकादशी रम्या मम प्रीतिविवर्द्धनी}
{सा त्वयोपोषिता सुभ्र यथोक्तविधिना शुभे} %॥६८॥

\twolineshloka
{तेन त्वया प्रसन्नोऽहं कृतोऽस्मि सुभगानने}
{तव भर्तुः प्रदास्यामि वरं यन्मनसेप्सितम्} %॥६९॥

\twolineshloka
{इत्युक्त्वा नृपति प्राह विष्णुर्विश्वार्तिनाशनः}
{वरं वरय राजेन्द्र यत्ते मनसि काङ्क्षितम्} %॥७०॥

\twolineshloka
{सन्तोषितोऽहं प्रियया तव सिद्धिचिकीर्षया}
{श्रुत्वा तद्वचनं विष्णोः प्रसन्नो नृपसत्तमः} %॥७१॥

\twolineshloka
{पत्रे सुतं महाबाहुं सर्वलोकनमस्कृतम्}
{न देवैर्मानुषैर्ना गर्दैत्यदानवराक्षसैः} %॥७२॥

\twolineshloka
{जेतुं शक्यो जगन्नाथ विना त्वां मधुसूदन}
{इत्युक्तो भगवान् बाहमित्युक्त्वान्तरधीयत} %॥७३॥

\twolineshloka
{नृपोऽपि सुप्रसनात्मा हृष्टः पुष्टः प्रियायुतः}
{समायात् स्वपुरं रम्यं नरनारीमनोरमम्} %॥७४॥

\twolineshloka
{स पद्मिन्यां सुतं लेभे कार्तवीर्य महाबलम्}
{न तेन सदृशः कश्चित्रिषु लोकेषु मानवः} %॥७५॥

\twolineshloka
{तस्मात्पराजितःसङ्ख्ये रावणो दशकन्धरः}
{न तं जेतुं समर्थोऽस्ति त्रिषु लोकेषु कश्चन} %॥७६॥

\twolineshloka
{विना नारायणं देवं चक्रपाणि गदाधरम्}
{न त्वया विस्मयः कार्यो रावणस्य पराजये} %॥७७॥

\twolineshloka
{मलिम्लुचप्रसादेन पद्मिन्याश्चाप्युपोषणात्}
{दत्तो देवाधिदेवेन कार्तवीर्यो महाबलः} %॥७८॥

\onelineshloka*
{इत्युक्त्वा प्रययौ विप्रेः प्रसनेनान्तरात्मना}

\uvacha{श्रीकृष्ण उवाच}

\onelineshloka
{एतत्ते सर्वमाख्यातं यत्पृष्टोऽहं त्वयानघ} %॥७९॥

\twolineshloka
{मलिम्लुचस्य मासस्य शुक्लाया व्रतमुत्तमम्}
{ये करिष्यन्ति मनुजास्ते यास्यन्ति हरेः पदम्} %॥८०॥

\twolineshloka
{त्वमेवं कुरु राजेन्द्र यदि चेष्टमभीप्ससि}
{केशवस्य वचः श्रुत्वा धर्मराजोऽतिहर्षितः} %॥८१॥

\onelineshloka*
{चक्रे व्रतं विधानेन बन्धुमिः परिवारितः}

\uvacha{सूत उवाच}

\twolineshloka
{एतत्ते सर्वमाख्यातं यत्पृष्टोऽहं पुरा द्विज}
{पुण्यं पवित्रं परमं किं भूयः श्रोतुमिच्छसि} %॥८२॥

\twolineshloka
{एवंविधं येपि व्रतं मनुष्या भक्त्या करिष्यन्ति मलिम्लुचस्य}
{उपोष्य शुक्कामतिसौख्यदात्रीमे कादशी ते भुवि शन्यन्याः} %॥८३॥

\twolineshloka
{श्रोष्यन्ति ये तस्य विधि समयं तेऽप्यशभाजो मतुजा भवन्ति}
{ये वै पठिष्यन्ति कथां समयां ते वै गमिष्यन्ति हरेनिवासम्} %॥८४॥

॥इत्यधिकमासस्य शुक्लैकादशीमाहात्म्यं सम्पूर्णम्॥

\hyperref[sec:ekadashi_mahatmyam_vrata_raja]{\closesub}
\clearpage

\sect{पुरुषोत्तम-मासस्य कृष्ण-कमला-एकादशी-माहात्म्यम्}
\label{sec:vrata-raja-purushottama-krishna-kamala}

\uvacha{युधिष्ठिर उवाच}

\twolineshloka
{मलिम्लुचस्य मासस्य कृष्णा का कथ्यते विभो}
{किं नाम को विधिस्तम्याः कथयस्व जगत्पते} %॥१॥

\uvacha{श्रीकृष्ण उवाच}

\twolineshloka
{परमेति समाख्याता पवित्रा पापहारिका}
{भुक्तिमुक्ति दा नृणां स्त्रीणां चापि युधिष्ठिर} %॥२॥

\twolineshloka
{पूर्वोक्तविधिना कार्या कृष्णापि भुवि मानवैः}
{सम्पूज्य परया भक्त्या नाम्ना देवं नरोत्तम} %॥३॥

\twolineshloka
{अव ते कथयिष्यामि कथामेतां मनोरमाम्}
{काम्पिल्यनगरे जातां मुनीनामग्रतः श्रुताम्} %॥४॥

\twolineshloka
{आसीविजवरः कश्चित्तुमेधानाम धार्मिकः}
{तस्य पत्नी पवित्राख्या पातिव्रत्यपरायणा} %॥५॥

\twolineshloka
{कर्मणा केनचिद्विप्रो धनधान्यविवर्जितः}
{न क्वापि लभते भिक्षा याचनपि नरान्बहून्} %॥६॥

\twolineshloka
{न भोज्यं लभते तादृङ्न वस्त्रं नैव मण्डनम्}
{रूपयौवनमाधुर्या नारी शुश्रूषते पतिम्} %॥७॥

\twolineshloka
{अतिथि भोजयित्वा सा क्षुषितापि स्वयं गृहे}
{तिष्ठत्येव विशालाक्षी ह्यम्लानमुखपङ्कजा} %॥८॥

\twolineshloka
{न भर्तारं क्वचिदपि नास्त्यन्नमिति भाषते}
{विलोक्य भार्या सुदती कर्षती स्वकलेवरम्} %॥९॥

\twolineshloka
{विचार्य ब्राह्मणश्चित्ते भार्यायाः प्रेमबन्धनम्}
{निन्दन्भाग्यं स्वकं खिन्नः प्रोचे वाक्यं प्रियंवदाम्} %॥१०॥

\twolineshloka
{कान्ने करोमि किं कार्य न मया लभ्यते धनम्}
{याचामि च नरान्मव्यान्न यच्छन्ति च मे धनम्} %॥११॥

\twolineshloka
{किं करोमि व गच्छामि तन्मे कथय शोभने}
{विना धनेन सुश्रोणि गृहकार्य न सियति} %॥१२॥

\twolineshloka
{देशाज्ञां परदेशाय गच्छामि धनलब्धये}
{यस्मिन्देशे च यत्प्राप्यं भोग्यं तत्रैव लभ्यते} %॥१३॥

\twolineshloka
{उद्यमेन धिन सिद्धिः कर्मणां नोपलभ्यते}
{तस्माद्बुधाः प्रशंसन्ति सर्वथैव शुभोद्यमम्} %॥१४॥

\twolineshloka
{श्रुत्वा कान्तस्य वचनं लाश्रुनेत्रा विचक्षणा}
{प्रोवाच प्राञ्जलिर्भूत्वा विनयानतकन्धरा} %॥१५॥

\twolineshloka
{त्वत्तो नास्ति सुविज्ञाता त्वयाज्ञप्ता ब्रवीम्यहम्}
{हितैषिणो नरा ब्रूयुः शश्वत्साधु ह्यसाध्वपि} %॥१६॥

\twolineshloka
{पूर्वदत्तं हि लल्गेत यत्र कुत्र महीतले}
{विना दानं न लभ्येत मेरो कनकपर्वते} %॥१७॥

\twolineshloka
{पूर्वदत्ता हि या विद्या पूर्वदत्तं हि यद्धनम्}
{पूर्वइत्ता हि या भूमिरिह जन्मनि लभ्यते} %॥१८॥

\twolineshloka
{यात्रा लिखितं भाले तत्तथैव हि लभ्यते}
{विना दानेन तु क्वापि लभ्यते नैव किश्चन} %॥१९॥

\twolineshloka
{पूर्वजन्मानि विप्रेन्द्र न मया म त्वया क्वचित्}
{सत्पात्राणां करे दत्तं स्वरूपं भूर्यपि सद्धनम्} %॥२०॥

\twolineshloka
{इह देशे परे वापि दत्तं सर्वत्र लभ्यते}
{अन्नमात्र तु विश्वेशो विना दत्तेन यच्छति} %॥२१॥

\twolineshloka
{तस्मादत्रैव विप्राग्य स्थातव्यं भवता मया}
{त्वां विनाहं न तिष्ठामि क्षणमात्र महामुने} %॥२२॥

\twolineshloka
{न माता न पिता भ्राता न श्वश्रूः श्वशुरो जनः}
{न सत्कुर्वन्ति केऽपि स्त्री स्वजनाश्च परे कुतः} %॥२३॥

\twolineshloka
{भी वियुक्तां निन्दन्ति दुर्भगेति वदन्ति च}
{तस्मादत्र स्थिरो भूत्वा विहरस्व यथासुखम्} %॥२४॥

\twolineshloka
{भवतो भाग्ययोगेन प्राप्तिश्चात्र भविष्यति}
{श्रुत्वा तस्यास्तु वचनं स्थितस्तत्र विचक्षणः} %॥२५॥

\twolineshloka
{तावत्तत्र समायातः कौण्डिन्यो मुनिसत्तमः}
{दृष्ट्वा समागतं हृष्टः सुमेधा द्विजसत्तमः} %॥२६॥

\twolineshloka
{सभार्यः सहसोत्थाय ननाम शिरसाऽसकृत्}
{धन्योऽप्यनुगृहीतोऽस्मि सफलं जीवितं मम} %॥२७॥

\twolineshloka
{यदृष्टोसि महाभाग्यादित्युवाच मुनीश्वरम्}
{दत्त्वा सुविष्ठरं तस्मै पूजयामास तं द्विजम्} %॥२८॥

\twolineshloka
{भोजयित्वा विधानेन पप्रच्छ प्रमदोत्तमा}
{विद्वन्केन प्रकारेण दारिद्यस्य क्षयो भवेत्} %॥२९॥

\twolineshloka
{विना दत्तं कथं लभ्येद्धनं विद्या कुटुम्बिनी}
{मां मे भर्ता परित्यज्य गन्तुकामोऽद्य वर्तते} %॥३०॥

\twolineshloka
{अन्यदेशं परॉल्लोकान्याचितुं परपत्तने}
{सम्प्रार्थ्य तु मया विद्वन् हेतुवाक्यमहत्तरैः} %॥३१॥

\twolineshloka
{नादत्तं लभ्यते किञ्चिदित्युक्त्वा स निवारितः}
{मम भाग्यान्मुनीन्द्राद्य त्वमत्रैव समागतः} %॥३२॥

\twolineshloka
{दारिद्यं त्वत्प्रसादान्मे शीघ्रं नश्यत्यसंशयम्}
{केनोपायेन विपेन्द्र दारिद्यं नश्यति ध्रुवम्} %॥३३॥

\twolineshloka
{कथयस्व कृपासिन्धो व्रतं तीर्थ तपादिकम्}
{श्रुत्वा तस्याः सुशीलाया भाषितं मुनिपुङ्गवः} %॥३४॥

\twolineshloka
{प्रोवाच प्रवरं चित्ते विचार्य व्रतमुत्तमम्}
{सर्वपापौघशमनं दुःखदारिद्यनाशनम्} %॥३५॥

\twolineshloka
{परमानाम विख्याता विष्णोस्तिथिरनुत्तमा}
{मलिम्लुचे तु या कृष्णा भुक्तिमुक्तिफलप्रदा} %॥३६॥

\twolineshloka
{तस्यामुपोषणं कृत्वा धनधान्ययुतो भवेत्}
{विधिना जागरैः साकं गीतवादित्रसंयुतम्} %॥३७॥

\twolineshloka
{धनदेन यदाचीर्ण व्रतमेतत्सुशोभनम्}
{तदा हृष्टेन रुद्रेण धनानामधिपः कृतः} %॥३८॥

\twolineshloka
{हरिश्चन्द्रेण च कृतं पुरा क्रीतसुतेन वै}
{पुनः प्राप्ता प्रिया तेन राज्यं निहतकण्टकम्} %॥३९॥

\twolineshloka
{तस्मात्कुरु विशालाक्षि व्रतमेतत्सुशोभनम्}
{यथोक्तविधिना भद्रे समं जागरणेन च} %॥४०॥

\twolineshloka
{इत्युक्त्वा तद्विधि सर्व कथयामास वाडवः}
{पुनः प्रोवाच तं विप्रं पञ्चरात्रिव्रतं शुभम्} %॥४१॥

\twolineshloka
{यस्यानुष्ठानमात्रेण भुक्तिर्मुक्तिश्च प्राप्यते}
{परमादिवसे प्रातः कृत्वा पौर्वाहिकं विधिम्} %॥४२॥

\twolineshloka
{कुर्यात् मुनियमाञ्छक्त्या पञ्चराविव्रतादरात्}
{प्रातः स्नात्वा निराहारो यस्तिष्ठेदिनपञ्चकम्} %॥४३॥

\twolineshloka
{स गच्छेद्वैष्णवं स्थानं पितृमातृप्रियायुतः}
{एकाशनस्तु यो भूयादिनानां पञ्चकं नरः} %॥४४॥

\twolineshloka
{सर्वपापविनिर्मुक्तः स्वर्गलोके महीयते}
{स्नात्वा यो भोजयेद्विप्रं दिनानां पञ्चकं नरः} %॥४५॥

\twolineshloka
{भोजितं तेन हि जगत्सदेवासुरमानुषम्}
{पूर्ण कुम्भं सुतोयेन यो ददाति द्विजातये} %॥४६॥

\twolineshloka
{दत्तं तेनैव सकलं ब्रह्माण्डं सचराचरम्}
{तिलपात्रं तु यो दद्याद्राह्मणाय विपश्चिते} %॥४७॥

\twolineshloka
{तिलसङ्ख्यासमाः साध्वि स वसेन्नाकमण्डले}
{घृतपात्रं तु यो दद्यात्स्नात्वा पञ्चदिनं नरः} %॥४८॥

\twolineshloka
{स भुक्त्वा विपुलाभोगान्सूर्यलोके महीयते}
{ब्रह्मचर्येण यस्तिष्ठेदिनानां पञ्चकं नरः} %॥४९॥

\twolineshloka
{भुनक्ति स स्वर्गभोगान्स्वर्वेश्याभिः समं मुदा}
{एवंविधं व्रतं साध्वि कुरु त्वं पतिना शुभे} %॥५०॥

\twolineshloka
{धनधान्ययुता भूत्वा स्वर्ग यास्यसि सुव्रते}
{इत्युक्ता सा व्रतं चक्रे कौण्डिन्येन यथोदितम्} %॥५१॥

\twolineshloka
{भर्ना समं भावयुता स्नात्वा मासि मलिम्लुचे}
{पञ्चरात्रव्रते पूर्णे परायाः प्रियसंयुता} %॥५२॥

\twolineshloka
{सापश्यद्राजभवनादायान्तं नृपनन्दनम्}
{स दत्त्वा नव्यभवनं भव्यवस्तुसमन्वितम्} %॥५३॥

\twolineshloka
{वासयामास विधिना विधिना प्रेरितः स्वयम्}
{दत्त्वा ग्रामं वृत्तिकरं ब्राह्मणाय सुमेधसे} %॥५४॥

\twolineshloka
{प्रसन्नस्तपसा राजा तं स्तुत्वा स्वगृहं ययौ}
{मलिम्लुचस्य मासस्य पराख्यायाः परादरात्} %॥५५॥

\twolineshloka
{उपोषणात्स कृष्णायाः पञ्चरात्रव्रतेन च}
{सर्वपापविनिर्मुक्तः सर्वसौख्यसमन्वितः} %॥५६॥

\onelineshloka*
{भुक्त्वा भोगास्त्रिया सार्द्धमन्ते विष्णुपुरं ययौ}

\uvacha{श्रीकृष्ण उवाच}

\onelineshloka
{पञ्चरात्रभवं पुण्यं मया वक्तुं न शक्यते} %॥५७॥

\twolineshloka
{तथापि किञ्चिद्वक्ष्यामि येन चीर्ण पराव्रतम्}
{स्नातानि पुष्कराद्यानि गङ्गाद्याः सरितस्तथा} %॥५८॥

\twolineshloka
{धेनुमुख्यानि दानानि तेन चीर्णानि सर्वथा}
{गयाश्राद्धं कृतं तेन पितरः परितोषिताः} %॥५९॥

\twolineshloka
{व्रतानि तेन चीर्णानि व्रतखण्डोदितानि वै}
{द्विपदां ब्राह्मणः श्रेष्ठो गौर्वरिष्ठा चतुस्पदाम्} %॥६०॥

\twolineshloka
{देवानां वासवः श्रेष्ठस्तथा मासो मलिम्लुचः}
{मलिम्लुचे पञ्चरात्रं महापापहरं स्मृतम्} %॥६१॥

\twolineshloka
{पञ्चरात्रे च परमा पद्मिनी पापशोषिणी}
{सैकाप्यशक्तैः कर्तव्याऽवश्यं भक्त्या विचक्षणैः} %॥६२॥

\twolineshloka
{मानुषं जनुरासाद्य न स्नातो यैर्मलिम्लुचः}
{ते जन्मघातिनो नूनं नोपोष्य हरिवासरे} %॥६३॥

\twolineshloka
{योनीभ्रमद्भिश्चवरशीतिलक्षाणि मानवैः}
{प्राप्यते मानुषं जन्म दुर्लभं पुण्यसञ्चयः} %॥६४॥

\onelineshloka*
{तस्मात्कार्य प्रयलेन परमाया व्रतं शुभम्}

\uvacha{श्रीकृष्ण उवाच}

\onelineshloka
{एतत्ते सर्वमाख्यातं यत्पृष्टोऽहं त्वयानघ} %॥६५॥

\twolineshloka
{मलिम्लुचस्य मासस्य परमायाः शुभं व्रतम्}
{तत्सर्व ते समाख्यातं कुरुप्यावहितो नृप} %॥६६॥

\twolineshloka
{ये त्वेवं भुवि परमा व्रतं चरन्ति सद्भक्त्या शुभविधिना मलिम्लुचे वै}
{ते भुक्त्वा दिवि विभवं सुरेन्द्रतुल्यं गच्छेयुत्रिभुवननन्दितस्य गेहम्} %॥६७॥

॥इत्यधिककृष्णैकादश्याः परमाख्याया माहात्म्यं सम्पूर्णम्॥

\hyperref[sec:ekadashi_mahatmyam_vrata_raja]{\closesub}
\end{center}
}{}
