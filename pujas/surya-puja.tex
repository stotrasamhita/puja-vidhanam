% !TeX program = XeLaTeX
% !TeX root = pUjA.tex
\chapt{श्री-सूर्य-पूजा}

\dnsub{पूर्वाङ्ग-विघ्नेश्वर-पूजा}

\graphicspath{{purvanga/}{../purvanga/}}

\centerline{\includegraphics[width=1cm]{ganesha.pdf}}

(आचम्य)

\twolineshloka*
{शुक्लाम्बरधरं विष्णुं शशिवर्णं चतुर्भुजम्}
{प्रसन्नवदनं ध्यायेत् सर्वविघ्नोपशान्तये}
 
प्राणान्  आयम्य।

(अप उपस्पृश्य, पुष्पाक्षतान् गृहीत्वा)\\

\twolineshloka*
{तदेव लग्नं सुदिनं तदेव ताराबलं चन्द्रबलं तदेव}
{विद्याबलं दैवबलं तदेव लक्ष्मीपतेरङ्घ्रियुगं स्मरामि}
 
ममोपात्त-समस्त-दुरित-क्षयद्वारा \\
श्री-परमेश्वर-प्रीत्यर्थं करिष्यमाणस्य कर्मणः\\
अविघ्नेन परिसमाप्त्यर्थम् आदौ विघ्नेश्वरपूजां करिष्ये।

(अप उपस्पृश्य)

\ifbool{veda}{
\twolineshloka*
{ॐ ग॒णानां᳚ त्वा ग॒णप॑तिꣳ हवामहे क॒विं क॑वी॒नामु॑प॒मश्र॑वस्तमम्}
{ज्ये॒ष्ठ॒राजं॒ ब्रह्म॑णां ब्रह्मणस्पत॒ आ नः॑ शृ॒ण्वन्नू॒तिभिः॑ सीद॒ साद॑नम्}
}{}

\twolineshloka*
{अगजानानपद्मार्कं गजाननमहर्निशम्}
{अनेकदं तं भक्तानाम् एकदन्तमुपास्महे}

\ifbool{veda}{भूर्भुवः॒ सुव॒रोम्।}{} अस्मिन् हरिद्राबिम्बे सुमुखं महागणपतिं ध्यायामि, आवाहयामि।\\

\renewcommand{\devAya}{\OMshri महागणपतये}

\devAya{} नमः, आसनं समर्पयामि।\\
\devAya{} नमः, पादयोः पाद्यं समर्पयामि।\\
\devAya{} नमः, अर्घ्यं समर्पयामि।\\
\devAya{} नमः, आचमनीयं समर्पयामि।\\
\devAya{} नमः, मधुपर्कं समर्पयामि।\\
\ifbool{veda}{ॐ भूर्भुवः॒ सुवः॑।}{}
\devAya{} नमः, शुद्धोदकस्नानं समर्पयामि। स्नानानन्तरमाचमनीयं समर्पयामि।\\
\devAya{} नमः, वस्त्रार्थमक्षतान् समर्पयामि।\\
\devAya{} नमः, यज्ञोपवीताभरणार्थे अक्षतान् समर्पयामि।\\
\devAya{} नमः, दिव्यपरिमलगन्धान् धारयामि। गन्धस्योपरि हरिद्राकुङ्कुमं समर्पयामि। \\
\devAya{} नमः, अक्षतान् समर्पयामि। \\
\devAya{} नमः, पुष्पमालिकां समर्पयामि। पुष्पैः पूजयामि।

\dnsub{अर्चना}
\begin{enumerate}
\begin{minipage}{0.475\linewidth}   
  \item सुमुखाय नमः
  \item एकदन्ताय नमः
  \item कपिलाय नमः
  \item गजकर्णकाय नमः
  \item लम्बोदराय नमः
  \item विकटाय नमः
  \item विघ्नराजाय नमः
  \item विनायकाय नमः
\end{minipage}
\begin{minipage}{0.525\linewidth}
  \item धूमकेतवे नमः
  \item गणाध्यक्षाय नमः
  \item फालचन्द्राय नमः
  \item गजाननाय नमः
  \item वक्रतुण्डाय नमः
  \item शूर्पकर्णाय नमः
  \item हेरम्बाय नमः
  \item स्कन्दपूर्वजाय नमः
\end{minipage}
\end{enumerate}
\devAya{} नमः, नानाविधपरिमलपत्रपुष्पाणि समर्पयामि॥\\
\devAya{} नमः, धूपमाघ्रापयामि।\\
अलङ्कारदीपं सन्दर्शयामि।\\
% नैवेद्यम्।\\
\ifbool{veda}{ॐ भूर्भुवः॒ सुवः॑। + ब्र॒ह्मणे॒ स्वाहा᳚।}{}
\devAya{} नमः, \blank{} (नैवेद्यं) निवेदयामि। 
\ifbool{veda}{अ॒मृ॒ता॒पि॒धा॒नम॑सि॥}{}
निवेदनान्तरम् आचमनीयं समर्पयामि।\\
\devAya{} नमः, ताम्बूलं समर्पयामि।\\
\devAya{} नमः, कर्पूरनीराजनं दर्शयामि। कर्पूरनीराजनानन्तरमाचमनीयं समर्पयामि।\\
\devAya{} नमः, मन्त्रपुष्पं समर्पयामि। स्वर्णपुष्पं समर्पयामि।\\

\twolineshloka*
{अभीप्सितार्थसिद्ध्यर्थं पूजितो यः सुरैरपि}
{सर्वविघ्नच्छिदे तस्मै गणाधिपतये नमः}

\twolineshloka*
{गजाननं भूतगणादिसेवितं कपित्थ-जम्बूफल-सार-भक्षितम्}
{उमासुतं शोकविनाशकारणं नमामि विघ्नेश्वरपादपङ्कजम्}

\centerline{अनन्तकोटिप्रदक्षिणनमस्कारान् समर्पयामि।}

\centerline{छत्त्रचामरादिसमस्तोपचारान् समर्पयामि।}

\twolineshloka*
{वक्रतुण्डमहाकाय कोटिसूर्यसमप्रभ}
{अविघ्नं कुरु मे देव सर्वकार्येषु सर्वदा}

\twolineshloka*
{सुमुखश्चैकदन्तश्च कपिलो गजकर्णकः}
{लम्बोदरश्च विकटो विघ्नराजो गणाधिपः}

\twolineshloka*
{धूमकेतुर्गणाध्यक्षो फालचन्द्रो गजाननः}
{वक्रतुण्डः शूर्पकर्णो हेरम्भः स्कन्दपूर्वजः}

\threelineshloka*
{षोडशैतानि नामानि यः पठेच्छृणुयादपि}
{विद्यारम्भे विवाहे च प्रवेशे निर्गमे तथा}
{सङ्ग्रामे सर्वकार्येषु विघ्नस्तस्य न जायते}

\centerline{प्रार्थनाः समर्पयामि।}

\closesub

\sect{प्रधान-पूजा — श्री-सूर्य-पूजा}

\shuklambaradharam

\pranayama

\renewcommand{\prityartham}{श्री-छाया-सुवर्चलाम्बा-समेत-श्री-सूर्यनारायण-प्रीत्यर्थं}
\renewcommand{\additionalSankalpa}{श्री-छाया-सुवर्चलाम्बा-समेत-श्री-सूर्यनारायण-प्रसाद-सिद्ध्यर्थं श्री-सूर्यनारायण-प्रसादेन अरोगदृढगात्रता-सिद्ध्यर्थं}
\renewcommand{\pujaam}{श्री-छाया-सुवर्चलाम्बा-समेत-श्री-सूर्यनारायण-पूजां}
\renewcommand{\kaale}{\blank-पुण्यकाले}
\renewcommand{\prakaarena}{कल्पोक्त-प्रकारेण}

\sankalpa{}

\vighneshvaraYathasthanam

\dnsub{आसन-पूजा}
\centerline{पृथिव्या  मेरुपृष्ठ  ऋषिः।  सुतलं  छन्दः।  कूर्मो  देवता॥}
\twolineshloka*
{पृथ्वि  त्वया  धृता  लोका  देवि  त्वं  विष्णुना  धृता}
{त्वं  च  धारय  मां  देवि  पवित्रं  चाऽऽसनं  कुरु}


\dnsub{घण्टा-पूजा}

\twolineshloka*
{आगमार्थं तु देवानां गमनार्थं तु रक्षसाम्}
{घण्टारवं करोम्यादौ देवताऽऽह्वानकारणम्}


\dnsub{कलशपूजा}
ॐ कलशाय नमः दिव्यगन्धान् धारयामि।\\
ॐ गङ्गायै नमः। ॐ यमुनायै नमः। ॐ गोदावर्यै नमः।  ॐ सरस्वत्यै नमः। ॐ नर्मदायै नमः। ॐ सिन्धवे नमः। ॐ कावेर्यै नमः।\\
ॐ सप्तकोटिमहातीर्थान्यावाहयामि।\\[-0.25ex]

(अथ कलशं स्पृष्ट्वा जपं कुर्यात्) \\
आपो॒ वा इ॒द सर्वं॒ विश्वा॑ भू॒तान्याप॑ प्रा॒णा वा आप॑ प॒शव॒ आपो\-ऽन्न॒मापोऽमृ॑त॒माप॑ स॒म्राडापो॑ वि॒राडाप॑ स्व॒राडाप॒श्\-छन्दा॒स्यापो॒ ज्योती॒ष्यापो॒ यजू॒ष्याप॑ स॒त्यमाप॒ सर्वा॑ दे॒वता॒ आपो॒ भूर्भुव॒ सुव॒राप॒ ओम्॥\\

\twolineshloka* 
{कलशस्य मुखे विष्णुः कण्ठे रुद्रः समाश्रितः}
{मूले तत्र स्थितो ब्रह्मा मध्ये मातृगणाः स्मृताः}
\threelineshloka* 
{कुक्षौ तु सागराः सर्वे सप्तद्वीपा वसुन्धरा}
{ऋग्वेदोऽथ यजुर्वेदः सामवेदोऽप्यथर्वणः}
{अङ्गैश्च सहिताः सर्वे कलशाम्बुसमाश्रिताः}
\twolineshloka* 
{गङ्गे च यमुने चैव गोदावरि सरस्वति}
{नर्मदे सिन्धुकावेरि जलेऽस्मिन् सन्निधिं कुरु}
\twolineshloka*
{सर्वे समुद्राः सरितः तीर्थानि च ह्रदा नदाः}
{आयान्तु देवपूजार्थं दुरितक्षयकारकाः}

\centerline{ॐ भूर्भुवः॒ सुवो॒ भूर्भुवः॒ सुवो॒ भूर्भुवः॒ सुवः॑।}

(इति कलशजलेन सर्वोपकरणानि आत्मानं च प्रोक्ष्य।)


\dnsub{आत्मपूजा}
ॐ आत्मने नमः, दिव्यगन्धान् धारयामि।
\begin{multicols}{2}
१. ॐ आत्मने नमः\\
२. ॐ अन्तरात्मने नमः\\
३. ॐ योगात्मने नमः\\
४. ॐ जीवात्मने नमः\\
५. ॐ परमात्मने नमः\\
६. ॐ ज्ञानात्मने नमः
\end{multicols}
समस्तोपचारान् समर्पयामि।

\twolineshloka*
{देहो देवालयः प्रोक्तो जीवो देवः सनातनः}
{त्यजेदज्ञाननिर्माल्यं सोऽहं भावेन पूजयेत्}


\begin{minipage}{\linewidth}
\dnsub{पीठ-पूजा}

\begin{multicols}{2}
\begin{enumerate}
\item आधारशक्त्यै नमः
\item मूलप्रकृत्यै नमः
\item आदिकूर्माय नमः 
\item आदिवराहाय नमः
\item अनन्ताय नमः
\item पृथिव्यै नमः
\item रत्नमण्डपाय नमः
\item रत्नवेदिकायै नमः
\item स्वर्णस्तम्भाय नमः
\item श्वेतच्छत्त्राय नमः
\item कल्पकवृक्षाय नमः
\item क्षीरसमुद्राय नमः 
\item सितचामराभ्यां नमः
\item योगपीठासनाय नमः
\end{enumerate}
\end{multicols}

\end{minipage}

\dnsub{गुरु-ध्यानम्}

\twolineshloka*
{गुरुर्ब्रह्मा गुरुर्विष्णुर्गुरुर्देवो महेश्वरः}
{गुरुः साक्षात् परं ब्रह्म तस्मै श्री-गुरवे नमः}


\sect{षोडशोपचार-पूजा}
\renewcommand{\devAya}{श्री-छाया-सुवर्चलाम्बा-समेत-श्री-सूर्यनारायणाय}

\begin{center}

\fourlineindentedshloka*
{सूर्यं सुन्दरलोकनाथममृतं वेदान्तसारं शिवं}
{ज्ञानं ब्रह्ममयं सुरेशममलं लोकैकचित्तं प्रभुम्}
{इन्द्रादित्यनराधिपं सुरगुरुं त्रैलोक्यचूडामणिं}
{विष्णुब्रह्मशिवस्वरूपहृदयं वन्दे सदा भास्करम्}

\fourlineindentedshloka*
{ध्यायेत् सूर्यमनन्तशक्तिकिरणं तेजोऽमलं भास्करं}
{भक्तानामभयप्रदं दिनकरं ज्योतिर्मयं शङ्करम्}
{आदित्यं जगदीशमच्युतमजं त्रैलोक्यचूडामणिं}
{भक्ताभीष्टवरप्रदं दिनमणिं मार्ताण्डमाद्यं शुभम्}

\twolineshloka*
{द्विभुजं पद्महस्तं च वरदं मकुटान्वितम्}
{ध्यायेद् दिवाकरं देवं सर्वाभीष्टप्रदायकम्} 

अस्मिन् बिम्बे/कुम्भे श्री-छाया-सुवर्चलाम्बा-समेत-श्री-सूर्यनारायणं ध्यायामि। 

\ifbool{veda}{\dnsub{प्राण-प्रतिष्ठा}
\begin{center}
अस्य श्रीप्राणप्रतिष्ठामहामन्त्रस्य ब्रह्मविष्णुमहेश्वरा ऋषयः।\\
ऋग्यजुस्सामाथर्वाणि छन्दांसि। प्राणशक्तिः परा देवता।

आं बीजम् - ह्रीं शक्तिः - क्रों कीलकम्। प्राणप्रतिष्ठापने विनियोगः॥

\begin{minipage}{\linewidth}
\begin{multicols}{2}
    \centering
    \textbf{॥करन्यासः॥}
    
    \begin{tabular}{lll}
        आं & अङ्गुष्ठाभ्यां & नमः।\\
        ह्रीं & तर्जनीभ्यां & नमः।\\
        क्रों & मध्यमाभ्यां & नमः।\\
        आं & अनामिकाभ्यां & नमः।\\
        ह्रीं & कनिष्ठिकाभ्यां & नमः।\\
        क्रों & करतलकरपृष्ठाभ्यां & नमः।\\
         & \\
    \end{tabular}

    \columnbreak
    
    \textbf{॥अङ्गन्यासः॥}

    \begin{tabular}{lll}
        आं & हृदयाय & नमः।\\
        ह्रीं & शिरसे & स्वाहा।\\
        क्रों & शिखायै & वषट्।\\
        आं & कवचाय & हुम्।\\
        ह्रीं & नेत्रत्रयाय & वौषट्।\\
        क्रों & अस्त्राय & फट्।\\
        \multicolumn{2}{l}{भूर्भुवस्सुवरोमिति} & दिग्बन्धः॥\\
    \end{tabular}
    

\end{multicols}
\end{minipage}

\textbf{॥ध्यानम्॥}

\fourlineindentedshloka*
{रक्ताम्भोधिस्थपोतोल्लसदरुणसरोजाधिरूढा कराब्जैः}
{पाशं कोदण्डमिक्षूद्भवमळिगुणमप्यङ्कुशं पञ्चबाणान्}
{बिभ्राणासृक् कपालं त्रिनयनलसिता पीनवक्षोरुहाढ्या}
{देवी बालार्कवर्णा भवतु सुखकरी प्राणशक्तिः परा नः}


आं ह्रीं क्रों, क्रों ह्रीं आं, य र ल व श ष स हों,\\
हंसः सोऽहं सोऽहं हंसः। अस्यां मूर्तौ प्राणास्तिष्ठन्तु। जीवस्तिष्ठतु। 

अस्यां मूर्तौ सर्वेन्द्रियाणि मनस्त्वक्\-चक्षुश्श्रोत्र\-जिह्वा\-घ्राण\-वाक्\-पाणि\-पाद\-पायूपस्थाख्यानि प्राणापान\-व्यानोदान\-समानाश्चात्रागत्य सुखं चिरं तिष्ठन्तु स्वाहा।

असु॑नीते॒ पुन॑र॒स्मासु॒ चक्षुः॒ पुनः॑ प्रा॒णमि॒ह नो॑ धेहि॒ भोग॑म्।\\
ज्योक्प॑श्येम॒ सूर्य॑मु॒च्चर॑न्त॒मनु॑मते मृ॒ळया॑ नः स्व॒स्ति॥\rlap{ऋक्~१०.५९.६॥}

प्राणान् प्रतिष्ठापयामि।

\end{center}}{}

\twolineshloka*
{आवाहयामि देवेड्यमाश्रिताभीष्टदायिनम्}
{आदित्यमखिलाराध्यमार्तत्राणपरायणम्}

\ifbool{veda}{
आ स॒त्येन॒ रज॑सा॒ वर्त॑मानो निवे॒शय॑न्न॒मृतं॒ मर्त्यं॑ च।\\
हि॒र॒ण्यये॑न सवि॒ता रथे॒नाऽदे॒वो या॑ति॒ भुव॑ना वि॒पश्यन्॑॥\\}{}

अस्मिन् बिम्बे/कुम्भे श्री-छाया-सुवर्चलाम्बा-समेत-श्री-सूर्यनारायणम् आवाहयामि।

\aavaahitobhava

\twolineshloka*
{आसनं संगृहाणेदं द्वादशात्मन् दिवाकर}
{छायानायक लोकेश मया भक्त्या समर्पितम्}
\textbf{\devAya{} नमः, आसनं समर्पयामि।}

\twolineshloka*
{पाद्यं ददामि पद्मादिसुगन्धिकुसुमैर्युतम्}
{पद्मबन्धो गृहाण त्वं प्रभाकर नमोऽस्तु ते}
\textbf{\devAya{} नमः, पाद्यं समर्पयामि।}

\twolineshloka*
{अर्घ्यं गृह्णीष्व भगवन् गन्धदूर्वादिमिश्रितम्}
{अघनाशक देवेश हरिदश्व रवे मुदा}
\textbf{\devAya{} नमः, अर्घ्यं समर्पयामि।}

\twolineshloka*
{ददाम्याचमनं तुभ्यं चराचरविबोधक}
{भक्त्याऽहं भास्करायाद्य जगद्ध्वान्तविनाशक}
\textbf{\devAya{} नमः, आचमनीयं समर्पयामि।}

\twolineshloka*
{मध्वाज्यदधिसंयुक्तं मधुपर्कं प्रभाकर}
{लोकनाथ मया दत्तं गृहाण मुनिसेवित}
\textbf{\devAya{} नमः, मधुपर्कं समर्पयामि।}

\twolineshloka*
{मध्वाज्यशर्करायुक्तं फलक्षीरसमन्वितम्}
{पञ्चामृतं प्रदास्यामि सर्वलोकसुपूजित}
\textbf{\devAya{} नमः, पञ्चामृतस्नानं समर्पयामि।}

\twolineshloka*
{वेदात्मने नमस्तुभ्यं जगदानन्ददायिने}
{नदीविमलतोयैस्त्वां स्नापयेऽहं विभावसो}
\textbf{\devAya{} नमः, शुद्धोदकस्नानं समर्पयामि। स्नानानन्तरं आचमनीयं समर्पयामि।}

\twolineshloka*
{दिव्याम्बरयुगं सूक्ष्मं हेमसूत्रविराजितम्}
{स्वीकुरुष्व नवं वस्त्रं दिवाकर मयाऽर्पितम्}
\textbf{\devAya{} नमः, वस्त्रम् समर्पयामि।}

\twolineshloka*
{ब्रह्मविष्णुमहेशानवपुषे विश्वचक्षुषे}
{साक्षिणे कर्मणां तुभ्यमुपवीतं ददाम्यहम्}
\textbf{\devAya{} नमः, उपवीतम् समर्पयामि।}

\twolineshloka*
{गृह्णीष्व भूषणं पूषन् मुदा देवगणैर्नुत}
{हारकुण्डलकेयूरकिरीटवलयादिकम्}
\textbf{\devAya{} नमः, आभरणम् समर्पयामि।}

\twolineshloka*
{श्रीगन्धं कुङ्कुमोपेतं कर्पूरेण समन्वितम्}
{गृहाण करुणावार्धे लोकबन्धो दिवाकर}
\textbf{\devAya{} नमः दिव्यपरिमलगन्धान् धारयामि। गन्धस्योपरि हरिद्राकुङ्कुमं समर्पयामि।}

\twolineshloka*
{प्राणिकर्मसाक्षिभूत भक्तानामभयप्रद}
{अक्षतान् प्रतिगृह्णीष्व ग्रहाणामीश भास्कर}
\textbf{\devAya{} नमः, अक्षतान् समर्पयामि।}

\twolineshloka*
{पङ्केरुहप्रियं पुष्पैर्मल्लिकावकुलादिभिः}
{सेवन्तिकापारिजातैः पूजयामि मुदा रविम्}
\textbf{\devAya{} नमः, पुष्पाणि समर्पयामि।}

\dnsub{अङ्ग-पूजा}
\begin{longtable}{rl@{ — }l}
१. & पावनाय नमः & पादौ पूजयामि\\
२. & ग्रहपतये नमः & गुल्फौ पूजयामि\\
३. & जगदानन्दनाय नमः & जानुनी पूजयामि\\
४. & सूर्याय नमः & ऊरू पूजयामि\\
५. & कर्मसाक्षिणे नमः & कटिं पूजयामि\\
६. & छायानाथाय नमः & नाभिं पूजयामि\\
७. & अरुणसारथये नमः & उदरं पूजयामि\\
८. & विश्वचक्षुषे नमः & वक्षः पूजयामि\\
९. & अहस्कराय नमः & हस्तौ पूजयामि\\
१०. & कालात्मने नमः & कण्ठं पूजयामि\\
११. & मार्ताण्डाय नमः & मुखं पूजयामि\\
१२. & लोकबन्धवे नमः & कपोलौ पूजयामि\\
१३. & पद्माक्षाय नमः & अक्षिणी पूजयामि\\
१४. & विकर्तनाय नमः & कर्णौ पूजयामि\\
१५. & मरीचिमालिने नमः & ललाटं पूजयामि\\
१६. & सप्ताश्वाय नमः & शिरः पूजयामि\\
\end{longtable}


\dnsub{अर्चना}

\begin{longtable}{rllll}
१. &  \ifbool{veda}{ॐ ह्राम्।& उ॒द्यन्न॒द्य मि॑त्रमहः।}{&} & मित्राय & नमः\\
२. &  \ifbool{veda}{ॐ ह्रीम्।& आ॒रोह॒न्नुत्त॑रां॒ दिवम्᳚।}{&} & रवये & नमः\\
३. &  \ifbool{veda}{ॐ ह्रूम्।& हृ॒द्रो॒गं मम॑ सूर्य।}{&} & सूर्याय & नमः\\
४. &  \ifbool{veda}{ॐ ह्रैम्।& ह॒रि॒माणं॑ च नाशय।}{&} & भानवे & नमः\\
५. &  \ifbool{veda}{ॐ ह्रौम्।& शुके॑षु मे हरि॒माणम्᳚।}{&} & खगाय & नमः\\
६. &  \ifbool{veda}{ॐ ह्रः।& रो॒प॒णाका॑सु दध्मसि।}{&} & पूष्णे & नमः\\
७. &  \ifbool{veda}{ॐ ह्राम्।& अथो॑ हारिद्र॒वेषु॑ मे।}{&} & हिरण्यगर्भाय & नमः\\
८. &  \ifbool{veda}{ॐ ह्रीम्।& ह॒रि॒माणं॒ नि द॑ध्मसि।}{&} & मरीचये & नमः\\
९. &  \ifbool{veda}{ॐ ह्रूम्।& उद॑गाद॒यमा॑दि॒त्यः।}{&} & आदित्याय & नमः\\
१०. & \ifbool{veda}{ॐ ह्रैम्।& विश्वे॑न॒ सह॑सा स॒ह।}{&} & सवित्रे & नमः\\
११. & \ifbool{veda}{ॐ ह्रौम्।& द्वि॒षन्तं॒ मम॑ र॒न्धयन्॑।}{&} & अर्काय & नमः\\
१२. & \ifbool{veda}{ॐ ह्रः।& मो अ॒हं द्वि॑ष॒तो र॑धम्।}{&} & भास्कराय & नमः\\
\end{longtable}

\begingroup
\setlength{\columnseprule}{1pt}
\let\chapt\sect
\input{../namavali-manjari/100/Surya_108.tex}
\endgroup

श्री-छाया-सुवर्चलाम्बा-समेत-श्री-सूर्यनारायण-स्वामिने नमः नानाविध\-परिमल\-पत्र\-पुष्पाणि समर्पयामि।

\dnsub{उत्तराङ्ग-पूजा}

\twolineshloka*
{दशाङ्गवासितो धूपः सुगन्धः सुरसेवित}
{गृह्यतामर्पितो भक्त्या मया सूर्य दिवाकर}
\textbf{\devAya{} नमः, धूपम् आघ्रापयामि।}

\twolineshloka*
{साज्यवर्तिसमायुक्तमन्धकारनिवारक}
{दीपं गृहाण घुमणे तेजोराशे गुणार्णव}
\textbf{\devAya{} नमः, अलङ्कारदीपं सन्दर्शयामि।}

\ifbool{veda}{ॐ भूर्भुवः॒ सुवः॑। + ब्र॒ह्मणे॒ स्वाहा᳚।}{}
\twolineshloka*
{नैवेद्यं षड्रसोपेतमिक्षुदण्डसमन्वितम्}
{पायसान्नं च सम्प्रीत्या गृह्यतां ग्रहनायक}
\textbf{\devAya{} नमः, गुडान्नं मुद्गान्नं माषापूपानि इक्षुदण्डं हरिद्रागुच्छं नैवेद्यं निवेदयामि।}
मध्ये मध्ये पानीयं समर्पयामि।
\ifbool{veda}{अ॒मृ॒ता॒पि॒धा॒नम॑सि॥}{}
निवेदनान्तरम् आचमनीयं समर्पयामि।

\twolineshloka*
{पूगीफलसमायुक्तं नागवल्लीदलैर्युतम्}
{कर्पूरचूर्णसंयुक्तं ताम्बूलं प्रतिगृह्याम्}
\textbf{\devAya{} नमः, ताम्बूलं समर्पयामि।}

\twolineshloka*
{कर्पूरखण्डकलितं नीराजनमिदं मुदा}
{नभोमणे गृहाण त्वं गाढध्वान्तविनाशक}
\textbf{\devAya{} नमः, कर्पूरनीराजनं दर्शयामि।} कर्पूरनीराजनानन्तरमाचमनीयं समर्पयामि।\\

\twolineshloka*
{नमो वेदस्वरूपाय भानवे कर्मसाक्षिणे}
{पुष्पाञ्जलिं प्रदास्यामि भक्त्या तुभ्यं दिवाकर}
\textbf{\devAya{} नमः, वेदोक्तमन्त्रपुष्पाञ्जलिं समर्पयामि। स्वर्णपुष्पं समर्पयामि।}


\twolineshloka*
{यानि कानि च पापानि जन्मान्तरकृतानि च}
{तानि तानि विनश्यन्ति प्रदक्षिण-पदे पदे}
\textbf{प्रदक्षिणं कृत्वा।}

\twolineshloka*
{धृतपद्मद्वयं भानुं तेजोमण्डलमध्यगम्}
{सर्वाधिव्याधिशमनं छायाश्लिष्टतनुं भजे}

\twolineshloka*
{जयाय जयभद्राय हर्यश्वाय नमो नमः}
{नमो नमः सहस्रांशो आदित्याय नमो नमः}

\twolineshloka*
{नम उग्राय वीराय सारङ्गाय नमो नमः}
{नमः पद्मप्रबोधाय मार्तण्डाय नमो नमः}

\end{center}

\ifbool{veda}{
\centerline{\normalsize (तैत्तिरीयब्राह्मणे अष्टकं – ३/प्रश्नः – ७/अनुवाकः – ६/ पञ्चादयः ७६-७७)}

% ॐ ह्रां ह्रीं ह्रूं ह्रैं ह्रौं ह्रः। ॐ ह्रां ह्रीं ह्रूं ह्रैं ह्रौं ह्रः।
उ॒द्यन्न॒द्य मि॑त्रमहः।
आ॒रोह॒न्नुत्त॑रां॒ दिवम्᳚।
हृ॒द्रो॒गं मम॑ सूर्य।
ह॒रि॒माणं॑ च नाशय।
शुके॑षु मे हरि॒माणम्᳚।
रो॒प॒णाका॑सु दध्मसि॥
अथो॑ हारिद्र॒वेषु॑ मे।
ह॒रि॒माणं॒ नि द॑ध्मसि।
उद॑गाद॒यमा॑दि॒त्यः।
विश्वे॑न॒ सह॑सा स॒ह।
द्वि॒षन्तं॒ मम॑ र॒न्धयन्॑।
मो अ॒हं द्वि॑ष॒तो र॑धम्।
मित्र-रवि-सूर्य-भानु-खग-पूष-हिरण्यगर्भ-मरीच्यादित्य-सवित्रर्क-भास्करेभ्यो नमः॥

\dnsub{आदित्यमण्डले परब्रह्मोपासनम्}
\centerline{\normalsize (तैत्तिरीयारण्यके प्रश्नः – १० (महानारयणोपनिषत्))}
आ॒दि॒त्यो वा ए॒ष ए॒तन्म॒ण्डलं॒ तप॑ति॒ तत्र॒ ता ऋच॒स्तदृ॒चा म॑ण्डल॒ꣳ॒ स ऋ॒चां लो॒कोऽथ॒ य ए॒ष ए॒तस्मि॑न्म॒ण्डले॒ऽर्चिर्दी॒प्यते॒ तानि॒ सामा॑नि॒ स सा॒म्नां म॒ण्डल॒ꣳ॒ स सा॒म्नां लो॒कोऽथ॒ य ए॒ष ए॒तस्मि॑न्म॒ण्डले॒ऽर्चिषि॒ पुरु॑ष॒स्तानि॒ यजूꣳ॑षि॒ स यजु॑षा मण्डल॒ꣳ॒ स यजु॑षां लो॒कः सैषा त्र॒य्येव॑ वि॒द्या त॑पति॒ य ए॒षो᳚ऽन्तरा॑दि॒त्ये हि॑र॒ण्मयः॒ पुरु॑षः॥३१॥
%६.१४.०

\dnsub{आदित्यपुरुषस्य सर्वात्मकत्वप्रदर्शनम्}
\centerline{\normalsize (तैत्तिरीयारण्यके प्रश्नः – १० (महानारयणोपनिषत्))}
आ॒दि॒त्यो वै तेज॒ ओजो॒ बलं॒ यश॒श्चक्षुः॒ श्रोत्र॑मा॒त्मा मनो॑ म॒न्युर्मनु॑र्मृ॒त्युः स॒त्यो मि॒त्रो वा॒युरा॑का॒शः प्रा॒णो लो॑कपा॒लः कः किं कं तथ्स॒त्यमन्न॑म॒मृतो॑ जी॒वो विश्वः॑ कत॒मः स्व॑य॒म्भु ब्रह्मै॒तदमृ॑त ए॒ष पुरु॑ष ए॒ष भू॒ताना॒मधि॑पति॒र्ब्रह्म॑णः॒ सायु॑ज्यꣳ सलो॒कता॑माप्नोत्ये॒तासा॑मे॒व दे॒वता॑ना॒ꣳ॒ सायु॑ज्यꣳ सा॒र्ष्टिताꣳ॑ समानलो॒कता॑माप्नोति॒ य ए॒वं वेदे᳚त्युप॒निषत्॥३२॥
%६.१५.०
}{}

\begin{center}

\twolineshloka*
{सौरमण्डलमध्यस्थं साम्बं संसारभेषजम्}
{नीलग्रीवं विरूपाक्षं नमामि शिवमव्ययम्}

\fourlineindentedshloka*
{ध्येयः सदा सवितृमण्डल-मध्यवर्ती}
{नारायणः सरसि-जासन-सन्निविष्टः}
{केयूरवान् मकरकुण्डलवान् किरीटी}
{हारी हिरण्मयवपुर्धृतशङ्खचक्रः}

\twolineshloka*
{शङ्ख-चक्र-गदापाणे द्वारकानिलयाच्युत}
{गोविन्द पुण्डरीकाक्ष रक्ष मां शरणागतम्}

\twolineshloka*
{आकाशात् पतितं तोयं यथा गच्छति सागरम्}
{सर्वदेवनमस्कारः केशवं प्रतिगच्छति}

श्री-केशवं प्रतिगच्छत्यों नम इति॥

\textbf{\devAya नमः, नमस्कारान् समर्पयामि।}


\dnsub{प्रार्थना}

\twolineshloka*
{भानो भास्कर मार्तण्ड चण्डरश्मे दिवाकर}
{आयुरारोग्यमैश्वर्यं श्रियं पुत्रांश्च देहि मे}

\twolineshloka*
{धृतपद्मद्वयं भानुं तेजोमण्डलमध्यगम्}
{सर्वाधिव्याधिशमनं छायाश्लिष्टतनुं भजे}

\fourlineindentedshloka*
{पद्मासनः पद्मकरो द्विबाहुः}
{पद्मद्युतिः सप्ततुरङ्गवाहः}
{दिवाकरो लोकगुरुः किरीटी}
{मयि प्रसादं विदधातु देवः}

\ifbool{veda}{भा॒स्क॒राय॑ वि॒द्महे॑ महद्युतिक॒राय॑ धीमहि।\\
तन्नो॑ आदित्यः प्रचो॒दया᳚त्।}{}

\textbf{प्रार्थनाः समर्पयामि।}

\textbf{\devAya{} नमः, छत्त्रचामरादिसमस्तोपचारान् समर्पयामि।}

\twolineshloka*
{आवाहनं न जानामि न जानामि विसर्जनम्}
{पूजाविधिं न जानामि क्षमस्वारुणसारथे}

\twolineshloka*
{अन्यथा शरणं नास्ति त्वमेव शरणं मम}
{तस्मात् कारुण्यभावेन रक्ष रक्ष दिवाकर}

\dnsub{अर्घ्यम्}
\resetShloka

\shuklambaradharam

अद्य-पूर्वोक्त एवं गुण-विशेषण-विशिष्टायाम् अस्याम् \textbf{\tithau{}} शुभतिथौ\\ 
ममोपात्त-समस्त-दुरित-क्षयद्वारा श्री-परमेश्वर-प्रीत्यर्थं\\
\prityartham{} श्रीसूर्य-नारायणपूजान्ते अर्घ्यप्रदानं करिष्ये।

\twolineshloka
{नमो भगवते तुभ्यं नमस्ते जातवेदसे}
{दत्तमर्घ्यं मया भानो गृहाण त्वं नमोऽस्तु ते}

\textbf{\devAya{} नमः, इदमर्घ्यं इदमर्घ्यं इदमर्घ्यम्।}

\twolineshloka
{एहि सूर्य सहस्रांशो तेजोमालिन् जगत्पते}
{अनुकम्पय मां देव गृहाणार्ध्यं नमोऽस्तु ते}

\textbf{\devAya{} नमः, इदमर्घ्यं इदमर्घ्यं इदमर्घ्यम्।}

\twolineshloka
{पुनरर्घ्यं ददाम्यद्य भानवे लोकसाक्षिणे}
{आदित्य जगदाधार सुप्रीतो मां समुद्धर}

\textbf{\devAya{} नमः, इदमर्घ्यं इदमर्घ्यं इदमर्घ्यम्।}

अनेन अर्घ्यप्रदानेन श्री-छाया-सुवर्चलाम्बा-समेत-श्री-सूर्यनारायणः सुप्रीतो वरदो भवतु॥

श्रीसूर्यनारायणस्वरूपस्य ब्राह्मणस्येदमासनम्। अमी ते गन्धाः। सकलाराधनैः स्वर्चितम्।

\hiranyagarbha

इदमुपायनं सदक्षिणाकं सताम्बूलं श्रीसूर्यनारायणप्रीतिं कामयमानस्तुभ्यमहं सम्प्रददे॥


\fourlineindentedshloka*
{कायेन वाचा मनसेन्द्रियैर्वा}
{बुद्‌ध्याऽऽत्मना वा प्रकृतेः स्वभावात्}
{करोमि यद्यत् सकलं परस्मै}
{नारायणायेति समर्पयामि}

अनया पूजया श्री-छाया-सुवर्चलाम्बा-समेत-श्री-सूर्यनारायणः प्रीयताम्॥

ॐ तत्सद्ब्रह्मार्पणमस्तु॥

\closesub

\begingroup
\let\chapt\sect
\begin{center}
    \input{../stotra-sangrahah/stotras/navagraha/AdityaHrdayam.tex}
    \input{../stotra-sangrahah/stotras/navagraha/DwadasharyaSuryaStuti.tex}
\end{center}
\endgroup

\end{center}

\closesection
