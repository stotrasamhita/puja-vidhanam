% !TeX program = XeLaTeX
% !TeX root = pUjA.tex

\setlength{\parindent}{0pt}
\chapt{श्री-तुलसी-पूजा}

\input{purvanga/vighneshwara-puja}

\sect{प्रधान-पूजा — तुलसी-पूजा}

\twolineshloka*
{शुक्लाम्बरधरं विष्णुं शशिवर्णं चतुर्भुजम्}
{प्रसन्नवदनं ध्यायेत् सर्वविघ्नोपशान्तये}
 
प्राणान्  आयम्य।  ॐ भूः + भूर्भुवः॒ सुव॒रोम्।

\dnsub{सङ्कल्पः}

ममोपात्त-समस्त-दुरित-क्षयद्वारा श्री-परमेश्वर-प्रीत्यर्थं शुभे शोभने मुहूर्ते अद्य ब्रह्मणः
द्वितीयपरार्धे श्वेतवराहकल्पे वैवस्वतमन्वन्तरे अष्टाविंशतितमे कलियुगे प्रथमे पादे
जम्बूद्वीपे भारतवर्षे भरतखण्डे मेरोः दक्षिणे पार्श्वे शकाब्दे अस्मिन् वर्तमाने व्यावहारिकाणां प्रभवादीनां षष्ट्याः संवत्सराणां मध्ये \mbox{(~~~)}\see{app:samvatsara_names} नाम संवत्सरे दक्षिणायने
शरद-ऋतौ  वृश्चिकमासे शुक्ल पक्षे द्वादश्यां शुभतिथौ
(इन्दु / भौम / बुध / गुरु / भृगु / स्थिर / भानु) वासरयुक्तायाम्
\mbox{(~~~)}\see{app:nakshatra_names} नक्षत्र \mbox{(~~~)}\see{app:yoga_names} नाम  योग  \mbox{(~~~)} करण युक्तायां च एवं गुण विशेषण विशिष्टायाम्
अस्याम् (एकादश्यां / द्वादश्यां) शुभतिथौ
अस्माकं सहकुटुम्बानां क्षेमस्थैर्य-धैर्य-वीर्य-विजय-आयुरारोग्य-ऐश्वर्याभिवृद्ध्यर्थम्
 धर्मार्थकाममोक्ष\-चतुर्विधफलपुरुषार्थसिद्ध्यर्थं पुत्रपौत्राभि\-वृद्ध्यर्थम् इष्टकाम्यार्थसिद्ध्यर्थम्
मम इहजन्मनि पूर्वजन्मनि जन्मान्तरे च सम्पादितानां ज्ञानाज्ञानकृतमहा\-पातकचतुष्टय-व्यतिरिक्तानां रहस्यकृतानां प्रकाशकृतानां सर्वेषां पापानां सद्य अपनोदनद्वारा सकल-पापक्षयार्थं श्रीमहाविष्णु-तुलसी-प्रीत्यर्थं यावच्छक्ति ध्यानावाहनादि
षोडशोपचार श्रीमहाविष्णु-तुलसी-पूजां करिष्ये। तदङ्गं कलशपूजां च करिष्ये।


श्रीविघ्नेश्वराय नमः यथास्थानं प्रतिष्ठापयामि।\\
(गणपति-प्रसादं शिरसा गृहीत्वा)


\dnsub{घण्टापूजा}

\twolineshloka*
{आगमार्थं तु देवानां गमनार्थं तु रक्षसाम्}
{घण्टारवं करोम्यादौ देवताऽऽह्वानकारणम्}

\dnsub{कलशपूजा}
ॐ कलशाय नमः दिव्यगन्धान् धारयामि।\\

\twolineshloka*
{गङ्गे च यमुने चैव गोदावरि सरस्वति}
{नर्मदे सिन्धुकावेरि जलेऽस्मिन् सन्निधिं कुरु}

ॐ गङ्गायै नमः। ॐ यमुनायै नमः। ॐ गोदावर्यै नमः।  ॐ सरस्वत्यै नमः। ॐ नर्मदायै नमः। ॐ सिन्धवे नमः। ॐ कावेर्यै नमः।\\
 ॐ सप्तकोटिमहातीर्थान्यावाहयामि। \\

(अथ कलशं स्पृष्ट्वा जपं कुर्यात्) \\
आपो॒ वा इ॒दꣳ सर्वं॒ विश्वा॑ भू॒तान्यापः॑ प्रा॒णा वा आपः॑ प॒शव॒ आपो\-ऽन्न॒मापोऽमृ॑त॒मापः॑ स॒म्राडापो॑ वि॒राडापः॑ स्व॒राडाप॒श्\-छन्दा॒ꣴ॒स्यापो॒ ज्योती॒ꣴ॒ष्यापो॒ यजू॒ꣴ॒ष्यापः॑ स॒त्यमापः॒ सर्वा॑ दे॒वता॒ आपो॒ भूर्भुवः॒ सुव॒राप॒ ओम्॥\\

\twolineshloka*
{कलशस्य मुखे विष्णुः कण्ठे रुद्रः समाश्रितः}
{मूले तत्र स्थितो ब्रह्मा मध्ये मातृगणाः स्मृताः}

\threelineshloka*
{कुक्षौ तु सागराः सर्वे सप्तद्वीपा वसुन्धरा}
{ऋग्वेदोऽथ यजुर्वेदः सामवेदोऽप्यथर्वणः}
{अङ्गैश्च सहिताः सर्वे कलशाम्बुसमाश्रिताः}

\twolineshloka*
{सर्वे समुद्राः सरितः तीर्थानि च ह्रदा नदाः}
{आयान्तु विष्णुपूजार्थं दुरितक्षयकारकाः}

\centerline{ॐ भूर्भुवः॒ सुवो॒ भूर्भुवः॒ सुवो॒ भूर्भुवः॒ सुवः॑।}

(इति कलशजलेन सर्वोपकरणानि आत्मानं च प्रोक्ष्य।)

\dnsub{आत्मपूजा}
ॐ आत्मने नमः, दिव्यगन्धान् धारयामि।
\begin{multicols}{2}
१. ॐ आत्मने नमः\\
२. ॐ अन्तरात्मने नमः\\
३. ॐ योगात्मने नमः\\
४. ॐ जीवात्मने नमः\\
५. ॐ परमात्मने नमः\\
६. ॐ ज्ञानात्मने नमः
\end{multicols}
समस्तोपचारान् समर्पयामि।\\

देहो देवालयः प्रोक्तो जीवो देवः सनातनः।\\
त्यजेदज्ञाननिर्माल्यं सोऽहं भावेन पूजयेत्॥\\

\dnsub{पीठपूजा}
\begin{multicols}{2}
\begin{enumerate}
\item ॐ आधारशक्त्यै नमः
\item ॐ मूलप्रकृत्यै नमः
\item ॐ आदिकूर्माय नमः
\item ॐ आदिवराहाय नमः
\item ॐ अनन्ताय नमः
\item ॐ पृथिव्यै नमः
\item ॐ रत्नमण्डपाय नमः
\item ॐ रत्नवेदिकायै नमः
\item ॐ स्वर्णस्तम्भाय नमः
\item ॐ श्वेतच्छत्त्राय नमः
\item ॐ कल्पकवृक्षाय नमः
\item ॐ क्षीरसमुद्राय नमः
\item ॐ सितचामराभ्यां नमः
\item ॐ योगपीठासनाय नमः
\end{enumerate}
\end{multicols}

\dnsub{गुरु-ध्यानम्}

\twolineshloka*
{गुरुर्ब्रह्मा गुरुर्विष्णुर्गुरुर्देवो महेश्वरः}
{गुरुः साक्षात् परं ब्रह्म तस्मै श्री-गुरवे नमः}

\sect{षोडशोपचार-पूजा}
\centering

\twolineshloka*
{ध्यायेच्च तुलसीं देवीं श्यामां कमललोचनाम्}
{प्रसन्नां पद्मकल्हारां वरदाभ्यां चतुर्भुजाम्}

\threelineshloka*
{किरीटहारकेयूरकुण्डलाद्यैर्विभूषिताम्}
{धवलां शुकसंवीतां पद्मासननिषेविताम्}
{देवीं त्रैलोक्यजननीं सर्वलोकैकपावनीम्}
अस्मिन् बिम्बे श्री-महाविष्णुं तुलसीं च ध्यायामि।

\twolineshloka*
{सर्वदेवमयि देवि सर्वदे विष्णुवल्लभे}
{आगच्छ मम गेहेऽस्मिन् नित्यं सन्निहिता भव}
अस्मिन् बिम्बे श्री-महाविष्णुं तुलसीं च आवाहयामि।
\medskip

\medskip
\twolineshloka*
{रत्नसिंहासनं चारु भुक्तिमुक्ति फलप्रदे}
{मया दत्तं महादेवि सङ्गृहाण सुरार्चिते}
 आसनं समर्पयामि।\medskip

\twolineshloka*
{नानागन्धसुपुष्पैश्च वासितं सुरवन्दिते}
{पाद्यं गृहाण देवि त्वं सर्वकामफलप्रदे}
 पाद्यं समर्पयामि।\medskip

\twolineshloka*
{गङ्गोदकं समानीतं सुवर्णकलशस्थितम्}
{तुलसि त्वं गृहाणेदं अर्घ्यमैश्वर्यदायकम्}
 अर्घ्यं समर्पयामि।\medskip

\twolineshloka*
{सर्वमङ्गलदेवेशि नवरत्नैर्विभूषिते}
{मया दत्तमिदं तोयं गृहाणाऽऽचमनीयकम्}
 आचमनीयं समर्पयामि।\medskip

\twolineshloka*
{पूर्णेन्दुबिम्ब सदृशं दधिखण्ड घृतं मधु}
{मधुपर्कं मयाऽऽनीतं गृहाण परमेश्वरि}
मधुपर्कं समर्पयामि।\medskip

 \twolineshloka*
{दधिक्षीरघृतं चैव शर्कराफलसंयुतम्}
{पञ्चामृतं गृहाण त्वं लोकानुग्रहकारिणि}
पञ्चामृतं समर्पयामि।\medskip


 \twolineshloka*
{गङ्गायमुनयोस्तोयैरानीतं निर्मलं शुभम्}
{शुद्दोदकिमिदं देवि स्नानार्थं प्रतिगृह्यताम्}
 शुद्धोदकस्नानं समर्पयामि। स्नानानन्तरम् आचमनीयं समर्पयामि।\medskip

 \twolineshloka*
 {क्षौमं काञ्चनसङ्काशमिदं शुभ्रं मनोहरम्}
 {वस्त्रयुग्मं शुभे देवि स्वीकुरुष्वाम्बुजेक्षणे}
 वस्त्रं समर्पयामि।\medskip

\twolineshloka*
{राजतं ब्रह्मसूत्रं च काञ्चनं चोत्तरीयकम्}
{गृहाण सर्ववरदे पद्मपत्रनिभेक्षणे}
 यज्ञोपवीतं समर्पयामि।\medskip

\twolineshloka*
{किरीटहारकटकान् केयूरान् कुण्डलान् शुभान्}
{नूपुरोदरबद्धं च भूषणानि समर्पये}
आभारणानि समर्पयामि।\medskip
 
\twolineshloka*
{चन्दनागरुकस्तूरी कर्पूरेण च संयुतम्}
{विलेपनं ददामि त्वं गृहाण तुलसि शुभे}
 दिव्यपरिमलगन्धान् धारयामि। गन्धस्योपरि हरिद्राकुङ्कुमं समर्पयामि। 

\twolineshloka*
{अक्षतान् धवलान् दिव्यान् शालीयान्स्तण्डुलान् शुभान्}
{अक्षतान् चार्पये देवि बृन्दसानोद्भवेश्वरि}
अक्षतान् समर्पयामि।\medskip

\twolineshloka*
{मल्लिका-कुन्द-मन्दार-जाजी-वकुल-चम्पकैः}
{शतपत्रैश्च कल्हारैः पूजयामि महेश्वरि}
 पुष्पाणि समर्पयामि।  पुष्पैः पूजयामि।

\dnsub{अङ्ग-पूजा}
\begin{longtable}{ll@{— }l}
१.& ॐ तुलसीदेव्यै नमः & पादौ पूजयामि \\
२.& बृन्दावनस्थायै नमः & गुल्फौ पूजयामि\\
३.& पद्मपत्रनिभेक्षणायै नमः & जङ्घे पूजयामि  \\
४.& पद्मकोटिसमप्रभायै नमः & ऊरू पूजयामि\\
५.& हरिप्रियायै नमः & जानुनी पूजयामि   \\
६.& कुङ्कुमाङ्कितगात्रायै नमः & कटिं पूजयामि \\
७.& सुरवन्दितायै नमः & नाभिं पूजयामि        \\
८.& लोकानुग्रहकारिण्यै नमः & उदरं पूजयामि\\
९.& त्रैलोक्यजनन्यै नमः & हृदयं पूजयामि    \\
१०.& पद्मप्रियायै नमः & हस्तान् पूजयामि\\
११.& इन्दिराख्यायै नमः & भुजान् पूजयामि\\
१२.& कम्बुकण्ठ्यै नमः & कण्ठं पूजयामि\\
१३.& कल्मषघ्न्यै नमः & कर्णौ पूजयामि  \\
१४.& वरप्रदायै नमः & मुखं  पूजयामि\\
१५.& आश्रितरक्षकायै नमः & शिरः पूजयामि\\
१६.& अभीष्टदायै नमः & सर्वाण्यङ्गानि पूजयामि   \\
\end{longtable}

\dnsub{चतुर्विंशति नामपूजा}
\begin{multicols}{2}
\begin{enumerate}
\item ॐ केशवाय नमः
\item ॐ नारायणाय नमः
\item ॐ माधवाय नमः
\item ॐ गोविन्दाय नमः
\item ॐ विष्णवे नमः
\item ॐ मधुसूदनाय नमः
\item ॐ त्रिविक्रमाय नमः
\item ॐ वामनाय नमः
\item ॐ श्रीधराय नमः
\item ॐ हृषीकेशाय नमः
\item ॐ पद्मनाभाय नमः
\item ॐ दामोदराय नमः
\item ॐ सङ्कर्षणाय नमः
\item ॐ वासुदेवाय नमः
\item ॐ प्रद्युम्नाय नमः
\item ॐ अनिरुद्धाय नमः
\item ॐ पुरुषोत्तमाय नमः
\item ॐ अधोक्षजाय नमः
\item ॐ नृसिंहाय नमः
\item ॐ अच्युताय नमः
\item ॐ जनार्दनाय नमः
\item ॐ उपेन्द्राय नमः
\item ॐ हरये नमः
\item ॐ श्रीकृष्णाय नमः
\end{enumerate}
\end{multicols}


\dnsub{श्री-कृष्णाष्टोत्तरशतनामावलिः}
\begin{multicols}{2}\setlength{\columnseprule}{1pt}
\begin{flushleft}
ॐ श्रीकृष्णाय नमः\\
ॐ कमलानाथाय नमः\\
ॐ वासुदेवाय नमः\\
ॐ सनातनाय नमः\\
ॐ वसुदेवात्मजाय नमः\\
ॐ पुण्याय नमः\\
ॐ लीलामानुषविग्रहाय नमः\\
ॐ श्रीवत्सकौस्तुभधराय नमः\\
ॐ यशोदावत्सलाय नमः\\
ॐ हरये नमः\hfill\devanumber{10}\\
ॐ चतुर्भुजात्तचक्रासि\-गदाशङ्खाम्बुजायुधाय नमः\\
ॐ देवकीनन्दनाय नमः\\
ॐ श्रीशाय नमः\\
ॐ नन्दगोपप्रियात्मजाय नमः\\
ॐ यमुनावेगसंहारिणे नमः\\
ॐ बलभद्रप्रियानुजाय नमः\\
ॐ पूतनाजीवितहराय नमः\\
ॐ शकटासुरभञ्जनाय नमः\\
ॐ नन्दव्रजजनानन्दिने नमः\\
ॐ सच्चिदानन्दविग्रहाय नमः\hfill\devanumber{20}\\
ॐ नवनीतविलिप्ताङ्गाय नमः\\
ॐ नवनीतनटाय नमः\\
ॐ अनघाय नमः\\
ॐ नवनीतनवाहाराय नमः\\
ॐ मुचुकुन्दप्रसादकाय नमः\\
ॐ षोडशस्त्रीसहस्रेशाय नमः\\
ॐ त्रिभङ्गीमधुराकृतये नमः\\
ॐ शुकवागमृताब्धीन्दवे नमः\\
ॐ गोविन्दाय नमः\\
ॐ योगिनां पतये नमः\hfill\devanumber{30}\\
ॐ वत्सवाटचराय नमः\\
ॐ अनन्ताय नमः\\
ॐ धेनुकासुरमर्दनाय नमः\\
ॐ तृणीकृततृणावर्ताय नमः\\
ॐ यमलार्जुनभञ्जनाय नमः\\
ॐ उत्तालतालभेत्रे नमः\\
ॐ तमालश्यामलाकृतये नमः\\
ॐ गोपगोपीश्वराय नमः\\
ॐ योगिने नमः\\
ॐ कोटिसूर्यसमप्रभाय नमः\hfill\devanumber{40}\\
ॐ इलापतये नमः\\
ॐ परस्मै ज्योतिषे नमः\\
ॐ यादवेन्द्राय नमः\\
ॐ यदूद्वहाय नमः\\
ॐ वनमालिने नमः\\
ॐ पीतवाससे नमः\\
ॐ पारिजातापहारकाय नमः\\
ॐ गोवर्धनाचलोद्धर्त्रे नमः\\
ॐ गोपालाय नमः\\
ॐ सर्वपालकाय नमः\hfill\devanumber{50}\\
ॐ अजाय नमः\\
ॐ निरञ्जनाय नमः\\
ॐ कामजनकाय नमः\\
ॐ कञ्जलोचनाय नमः\\
ॐ मधुघ्ने नमः\\
ॐ मथुरानाथाय नमः\\
ॐ द्वारकानायकाय नमः\\
ॐ बलिने नमः\\
ॐ बृन्दावनान्तसञ्चारिणे नमः\\
ॐ तुलसीदामभूषणाय नमः\hfill\devanumber{60}\\
ॐ स्यमन्तकमणेर्हर्त्रे नमः\\
ॐ नरनारायणात्मकाय नमः\\
ॐ कुब्जाकृष्णाम्बरधराय नमः\\
ॐ मायिने नमः\\
ॐ परमपूरुषाय नमः\\
ॐ मुष्टिकासुरचाणूर\-मल्लयुद्ध\-विशारदाय नमः\\
ॐ संसारवैरिणे नमः\\
ॐ कंसारये नमः\\
ॐ मुरारये नमः\\
ॐ नरकान्तकाय नमः\hfill\devanumber{70}\\
ॐ अनादिब्रह्मचारिणे नमः\\
ॐ कृष्णाव्यसनकर्षकाय नमः\\
ॐ शिशुपालशिरश्छेत्रे नमः\\
ॐ दुर्योधनकुलान्तकाय नमः\\
ॐ विदुराक्रूरवरदाय नमः\\
ॐ विश्वरूपप्रदर्शकाय नमः\\
ॐ सत्यवाचे नमः\\
ॐ सत्यसङ्कल्पाय नमः\\
ॐ सत्यभामारताय नमः\\
ॐ जयिने नमः\hfill\devanumber{80}\\
ॐ सुभद्रापूर्वजाय नमः\\
ॐ विष्णवे नमः\\
ॐ भीष्ममुक्तिप्रदायकाय नमः\\
ॐ जगद्गुरवे नमः\\
ॐ जगन्नाथाय नमः\\
ॐ वेणुनादविशारदाय नमः\\
ॐ वृषभासुरविध्वंसिने नमः\\
ॐ बाणासुरकरान्तकाय नमः\\
ॐ युधिष्ठिरप्रतिष्ठात्रे नमः\\
ॐ बर्हिबर्हावतंसकाय नमः\hfill\devanumber{90}\\
ॐ पार्थसारथये नमः\\
ॐ अव्यक्ताय नमः\\
ॐ गीतामृतमहोदधये नमः\\
ॐ कालीयफणिमाणिक्य\-रञ्जित\-श्री\-पदाम्बुजाय नमः\\
ॐ दामोदराय नमः\\
ॐ यज्ञभोक्त्रे नमः\\
ॐ दानवेन्द्रविनाशकाय नमः\\
ॐ नारायणाय नमः\\
ॐ परब्रह्मणे नमः\\
ॐ पन्नगाशनवाहनाय नमः\hfill\devanumber{100}\\
ॐ जलक्रीडासमासक्त\-गोपी\-वस्त्रापहारकाय~नमः\\
ॐ पुण्यश्लोकाय नमः\\
ॐ तीर्थपादाय नमः\\
ॐ वेदवेद्याय नमः\\
ॐ दयानिधये नमः\\
ॐ सर्वतीर्थात्मकाय नमः\\
ॐ सर्वग्रहरूपिणे नमः\\
ॐ परात्पराय नमः\hfill\devanumber{108}\\
\end{flushleft}
\end{multicols}

\dnsub{तुलस्यष्टोत्तरशतनामावलिः}
\begin{multicols}{2}
\begin{flushleft}
ॐ तुलस्यै~नमः\\
ॐ पावन्यै~नमः\\
ॐ पूज्यायै~नमः\\
ॐ वृन्दावननिवासिन्यै~नमः\\
ॐ ज्ञानदात्र्यै~नमः\\
ॐ ज्ञानमय्यै~नमः\\
ॐ निर्मलायै~नमः\\
ॐ सर्वपूजितायै~नमः\\
ॐ सत्यै~नमः\\
ॐ पतिव्रतायै~नमः\hfill\devanumber{10}‌\\
ॐ वृन्दायै~नमः\\
ॐ क्षीराब्धिमथनोद्भवायै~नमः\\
ॐ कृष्णवर्णायै~नमः\\
ॐ रोगहन्त्र्यै~नमः\\
ॐ त्रिवर्णायै ~नमः\\
ॐ सर्वकामदायै~नमः\\
ॐ लक्ष्मीसख्यै~नमः\\
ॐ नित्यशुद्धायै~नमः\\
ॐ सुदत्यै~नमः\\
ॐ भूमिपावन्यै~नमः\hfill\devanumber{20}‌\\
ॐ हरिद्रान्नैकनिरतायै~नमः\\
ॐ हरिपादकृतालयायै~नमः\\
ॐ पवित्ररूपिण्यै~नमः\\
ॐ धन्यायै~नमः\\
ॐ सुगन्धिन्यै~नमः\\
ॐ अमृतोद्भवायै~नमः\\
ॐ सुरूपायै आरोग्यदायै~नमः\\
ॐ तुष्टायै~नमः\\
ॐ शक्तित्रितयरूपिण्यै~नमः\\
ॐ देव्यै~नमः\hfill\devanumber{30}‌\\
ॐ देवर्षिसंस्तुत्यायै~नमः\\
ॐ कान्तायै~नमः\\
ॐ विष्णुमनःप्रियायै~नमः\\
ॐ भूतवेतालभीतिघ्न्यै~नमः\\
ॐ महापातकनाशिन्यै~नमः\\
ॐ मनोरथप्रदायै~नमः\\
ॐ मेधायै~नमः\\
ॐ कान्त्यै~नमः\\
ॐ विजयदायिन्यै~नमः\\
ॐ शङ्खचक्रगदापद्मधारिण्यै~नमः\hfill\devanumber{40}‌\\
ॐ कामरूपिण्यै~नमः\\
ॐ अपवर्गप्रदायै~नमः\\
ॐ श्यामायै~नमः\\
ॐ कृशमध्यायै~नमः\\
ॐ सुकेशिन्यै~नमः\\
ॐ वैकुण्ठवासिन्यै~नमः\\
ॐ नन्दायै~नमः\\
ॐ बिम्बोष्ठ्यै~नमः\\
ॐ कोकिलस्वरायै~नमः\\
ॐ कपिलायै~नमः\hfill\devanumber{50}‌\\
ॐ निम्नगाजन्मभूम्यै~नमः\\
ॐ आयुष्यदायिन्यै~नमः\\
ॐ वनरूपायै~नमः\\
ॐ दुःखनाशिन्यै~नमः\\
ॐ अविकारायै~नमः\\
ॐ चतुर्भुजायै~नमः\\
ॐ गरुत्मद्वाहनायै~नमः\\
ॐ शान्तायै~नमः\\
ॐ दान्तायै~नमः\\
ॐ विघ्ननिवारिण्यै~नमः\hfill\devanumber{60}‌\\
ॐ श्रीविष्णुमूलिकायै~नमः\\
ॐ पुष्ट्यै~नमः\\
ॐ त्रिवर्गफलदायिन्यै~नमः\\
ॐ महाशक्त्यै~नमः\\
ॐ महामायायै~नमः\\
ॐ लक्ष्मीवाणीसुपूजितायै~नमः\\
ॐ सुमङ्गल्यर्चनप्रीतायै~नमः\\
ॐ सौमङ्गल्यविवर्धिन्यै~नमः\\
ॐ चातुर्मास्योत्सवाराध्यायै~नमः\\
ॐ विष्णुसान्निध्यदायिन्यै~नमः\hfill\devanumber{70}‌\\
ॐ उत्थानद्वादशीपूज्यायै~नमः\\
ॐ सर्वदेवप्रपूजितायै~नमः\\
ॐ गोपीरतिप्रदायै~नमः\\
ॐ नित्यायै~नमः\\
ॐ निर्गुणायै~नमः\\
ॐ पार्वतीप्रियायै~नमः\\
ॐ अपमृत्युहरायै~नमः\\
ॐ राधाप्रियायै~नमः\\
ॐ मृगविलोचनायै~नमः\\
ॐ अम्लानायै~नमः\hfill\devanumber{80}‌\\
ॐ हंसगमनायै~नमः\\
ॐ कमलासनवन्दितायै~नमः\\
ॐ भूलोकवासिन्यै~नमः\\
ॐ शुद्धायै~नमः\\
ॐ रामकृष्णादिपूजितायै~नमः\\
ॐ सीतापूज्यायै~नमः\\
ॐ राममनःप्रियायै~नमः\\
ॐ नन्दनसंस्थितायै~नमः\\
ॐ सर्वतीर्थमय्यै~नमः\\
ॐ मुक्तायै~नमः\hfill\devanumber{90}‌\\
ॐ लोकसृष्टिविधायिन्यै~नमः\\
ॐ प्रातर्दृश्यायै~नमः\\
ॐ ग्लानिहन्त्र्यै~नमः\\
ॐ वैष्णव्यै~नमः\\
ॐ सर्वसिद्धिदायै~नमः\\
ॐ नारायण्यै~नमः\\
ॐ सन्ततिदायै~नमः\\
ॐ मूलमृद्धारिपावन्यै~नमः\\
ॐ अशोकवनिकासंस्थायै~नमः\\
ॐ सीताध्यातायै~नमः\hfill\devanumber{100}‌\\
ॐ निराश्रयायै~नमः\\
ॐ गोमतीसरयूतीररोपितायै~नमः\\
ॐ कुटिलालकायै~नमः\\
ॐ अपात्रभक्ष्यपापघ्न्यै~नमः\\
ॐ दानतोयविशुद्धिदायै~नमः\\
ॐ श्रुतिधारणसुप्रीतायै~नमः\\
ॐ शुभायै~नमः\\
ॐ सर्वेष्टदायिन्यै~नमः\hfill\devanumber{108}\\
\end{flushleft}
\end{multicols}
श्री-महाविष्णु-तुलसी-देव्यै नमः नानाविध परिमल-पत्र-पुष्पाणि समर्पयामि।


\sect{उत्तराङ्ग-पूजा}
गुग्गुलुर्गोघृतं चैव दशाङ्गं सुमनोहरम्।
धूपं गृहाण वरदे सर्वाभीष्टफलप्रदे॥
श्री-महाविष्णु-तुलसी-देव्यै नमः धूपमाघ्रापयामि।\\

उद्दी᳚प्यस्व जातवेदोऽप॒घ्नन्निर्ऋ॑तिं॒ मम॑।\\
 प॒शूꣳश्च॒ मह्य॒माव॑ह॒ जीव॑नं च॒ दिशो॑ दिश॥ \\
मा नो॑ हिꣳसीज्जातवेदो॒ गामश्वं॒ पुरु॑षं॒ जग॑त्।\\
अबि॑भ्र॒दग्न॒ आग॑हि श्रि॒या मा॒ परि॑पातय॥ \\
साज्यं त्रिवर्तिसंयुक्तं वह्निना योजितं मया।\\
गृहाण मङ्गलं दीपं त्रैलोक्यतिमिरापहम्॥ \\
श्री-महाविष्णु-तुलसी-देव्यै नमः अलङ्कारदीपं सन्दर्शयामि।\\

नैवेद्यम् षड्रसोपेतं फलसूपसमन्वितम्।\\
सघृतं समधुक्षीरं गृहाण तुलसीश्वरि॥\\
- श्री-महाविष्णु-तुलसी-देव्यै नमः (	) निवेदयामि, \\
अमृतापिधानमसि। निवेदनानन्तरम् आचमनीयं समर्पयामि।\\

पूगीफलसमायुक्तं नागवल्लीदलैर्युतम्।\\
कर्पूरचूर्णसंयुक्तं ताम्बूलं प्रतिगृह्यताम्॥\\
श्री-महाविष्णु-तुलसी-देव्यै नमः कर्पूरताम्बूलं समर्पयामि।\\

नीराजनं सुमाङ्गल्यं दिव्यज्योतिसमन्वितम्।\\
अज्ञानघ्ने गृहाण त्वं ज्ञानमार्गप्रदायिनि॥\\
श्री-महाविष्णु-तुलसी-देव्यै नमः समस्त-अपराध-क्षमापनार्थं कर्पूरनीराजनं दर्शयामि।\\
कर्पूरनीरजनानन्तरम् आचमनीयं समर्पयामि।\\

यो॑ऽपां पुष्पं॒ वेद॑। पुष्प॑वान् प्र॒जावा᳚न् पशु॒मान् भ॑वति।\\
च॒न्द्रमा॒ वा अ॒पां पुष्पम्᳚। पुष्प॑वान् प्र॒जावा᳚न् पशु॒मान् भ॑वति।\\
य ए॒वं वेद॑। यो॑ऽपामा॒यत॑नं॒ वेद॑। आ॒यत॑नवान् भवति।\\

ओं᳚ तद्ब्र॒ह्म। ओं᳚ तद्वा॒युः। ओं᳚ तदा॒त्मा। ओं᳚ तथ्स॒त्यम्‌।\\
ओं᳚ तथ्सर्वम्᳚‌। ओं᳚ तत्पुरो॒र्नमः॥\\

अन्तश्चरति॑ भूते॒षु॒ गुहायां वि॑श्वमू॒र्तिषु। \\
त्वं यज्ञस्त्वं वषट्कारस्त्वमिन्द्रस्त्वꣳ रुद्रस्त्वं विष्णुस्त्वं ब्रह्म त्वं॑ प्रजा॒पतिः। \\
त्वं त॑दाप॒ आपो॒ ज्योती॒ रसो॒ऽमृतं॒ ब्रह्म॒ भूर्भुव॒स्सुव॒रोम्‌॥\\

श्री-महाविष्णु-तुलसी-देव्यै नमः वेदोक्तमन्त्रपुष्पाञ्जलिं समर्पयामि।\\

सुवर्णरजतैर्युक्तं चामीकरविनिर्मितम्।\\
स्वर्णपुष्पं प्रदास्यामि गृह्यतां मधुसूदन॥ - स्वर्णपुष्पं समर्पयामि\\

प्रकृष्टपापनाशाय प्रकृष्टफलसिद्धये।\\
प्रदक्षिणं करोमि त्वां सर्वाभीष्टफलप्रदे॥\\

आयुरारोग्यमैश्वर्यं विद्या ज्ञानं यशः सुखम्।\\
देहि देहि ममाभीष्टं प्रदक्षिणकृतोत्तमे॥\\

नमस्ते तुलसि देवि सर्वाभीष्टफलप्रदे।\\
नमस्ते त्रिजगद्वन्द्ये नमस्ते लोकरक्षिके॥\\

सर्वमङ्गलमाङ्गल्ये शिवे सर्वार्थसाधिके।\\
शरण्ये त्र्यम्बिके गौरि नारायणि नमोऽस्तु ते॥\\

यानि कानि च पापानि जन्मान्तरकृतानि च।\\
तानि तानि विनश्यन्ति प्रदक्षिण-पदे पदे॥\\
\textbf{प्रदक्षिणं कृत्वा।}
\medskip

- श्री-महाविष्णु-तुलसी-देव्यै नमः अनन्तकोटिप्रदक्षिणनमस्कारान् समर्पयामि।\\
छत्त्रं समर्पयामि। चामरं समर्पयामि। गीतं श्रावयामि। व्यजनं वीजयामि।\\
नृत्यं दर्शयामि। वाद्यं घोषयामि। आन्दोलिकां समर्पयामि।\\
अश्वान् आरोपयमि। गजान् आरोपयामि।\\
समस्त राजोपचारान् समर्पयामि।\\


\dnsub{अर्घ्यप्रदानम्}
ममोपात्त-समस्त-दुरित-क्षयद्वारा श्रीपरमेश्वरप्रीत्यर्थम् अद्य पूर्वोक्त एवङ्गुण विशेषेण विशिष्टायाम् अस्यां द्वादश्यां शुभतिथौ अद्य कृत महाविष्णु-तुलसी-पूजान्ते क्षीरार्घ्यप्रदानं पायसपात्रदानं च करिष्ये॥\\

तुलस्यै तु नमस्तुभ्यं नमस्ते फलदायिनि।\\
इदमर्घ्यं प्रदास्यामि सुप्रीता वरदा भव॥\\
-तुलस्यै नमः इदमर्घ्यमिदमर्घ्यमिदमर्घ्यम्॥\\
लक्ष्मीपतिप्रिया देवी तुलसी दिव्यरूपिणी।\\
इदमर्घ्यं प्रदास्यामि सुप्रीता वरदा भव॥\\
-तुलस्यै नमः इदमर्घ्यमिदमर्घ्यमिदमर्घ्यम्।\\
सर्वपापहरे देवि आश्रिताभीष्टदायिनि।\\
मया दत्तं गृहाणेदं सुप्रीता वरदा भव॥\\
-तुलस्यै नमः इदमर्घ्यमिदमर्घ्यमिदमर्घ्यम्॥\\
अनेन अर्घ्यप्रदानेन श्री-महाविष्णु-तुलसी प्रीयेताम्।\\

\dnsub{प्रार्थना}

\twolineshloka*
{देहि मे विजयं देहि विद्यां देहि महेश्वरि}
{त्वामिति प्रार्थये नित्यं शीघ्रमेव फलं कुरु}

\dnsub{पायसपात्रदानम्}
श्री-महाविष्णु-तुलसी-स्वरूपस्य ब्राह्मणस्य इदमासनम्। गन्धादि सकलाराधनैः स्वर्चितम्।

हिरण्यगर्भगर्भस्थं हेमबीजं विभावसोः।\\
अनन्तपुण्यफलदम् अतः शान्तिं प्रयच्छ मे॥\\

बृन्दावन-द्वादशी-पुण्यकालेऽस्मिन् मया क्रियमाण महाविष्णु-तुलसी-पूजायां यद्देयमुपायनदानं तत्प्रत्याम्नायार्थं
हिरण्यं इदम् सपायसं पात्रम् सदक्षिणाकं सताम्बूलं श्री-महाविष्णु-तुलसी-प्रीतिम्
कामयमानः मनसोद्दिष्टाय ब्राह्मणाय सम्प्रददे नमः न मम।
अनया पूजया श्री-महाविष्णु-तुलसी प्रीयेताम्।

\dnsub{विसर्जनम्}
यस्य स्मृत्या च नामोक्त्या तपः-पूजा-क्रियादिषु।\\
न्यूनं सम्पूर्णतां याति सद्यो वन्दे तमच्युतम्॥ \\
इदं व्रतं मया देव कृतं प्रीत्यै तव प्रभो।\\
न्यूनं सम्पूर्णतां यातु त्वत्प्रसादाज्जनार्द्दन॥\\

अस्मात् बिम्बात् श्री-महाविष्णु-तुलसीं यथास्थानं प्रतिष्ठापयामि\\
(अक्षतानर्पित्वा देवमुत्सर्जयेत्।)\\
अनया पूजया श्री-महाविष्णु-तुलसी प्रीयेताम्। \\
ॐ तत्सद्ब्रह्मार्पणमस्तु।