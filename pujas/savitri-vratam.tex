% !TeX program = XeLaTeX
% !TeX root = ../pujavidhanam.tex
\chapt{श्री-सावित्री-व्रतम् — कामाक्षी-पूजा}


\dnsub{पूर्वाङ्ग-विघ्नेश्वर-पूजा}

\graphicspath{{purvanga/}{../purvanga/}}

\centerline{\includegraphics[width=1cm]{ganesha.pdf}}

(आचम्य)

\twolineshloka*
{शुक्लाम्बरधरं विष्णुं शशिवर्णं चतुर्भुजम्}
{प्रसन्नवदनं ध्यायेत् सर्वविघ्नोपशान्तये}
 
प्राणान्  आयम्य।

(अप उपस्पृश्य, पुष्पाक्षतान् गृहीत्वा)\\

\twolineshloka*
{तदेव लग्नं सुदिनं तदेव ताराबलं चन्द्रबलं तदेव}
{विद्याबलं दैवबलं तदेव लक्ष्मीपतेरङ्घ्रियुगं स्मरामि}
 
ममोपात्त-समस्त-दुरित-क्षयद्वारा \\
श्री-परमेश्वर-प्रीत्यर्थं करिष्यमाणस्य कर्मणः\\
अविघ्नेन परिसमाप्त्यर्थम् आदौ विघ्नेश्वरपूजां करिष्ये।

(अप उपस्पृश्य)

\ifbool{veda}{
\twolineshloka*
{ॐ ग॒णानां᳚ त्वा ग॒णप॑तिꣳ हवामहे क॒विं क॑वी॒नामु॑प॒मश्र॑वस्तमम्}
{ज्ये॒ष्ठ॒राजं॒ ब्रह्म॑णां ब्रह्मणस्पत॒ आ नः॑ शृ॒ण्वन्नू॒तिभिः॑ सीद॒ साद॑नम्}
}{}

\twolineshloka*
{अगजानानपद्मार्कं गजाननमहर्निशम्}
{अनेकदं तं भक्तानाम् एकदन्तमुपास्महे}

\ifbool{veda}{भूर्भुवः॒ सुव॒रोम्।}{} अस्मिन् हरिद्राबिम्बे सुमुखं महागणपतिं ध्यायामि, आवाहयामि।\\

\renewcommand{\devAya}{\OMshri महागणपतये}

\devAya{} नमः, आसनं समर्पयामि।\\
\devAya{} नमः, पादयोः पाद्यं समर्पयामि।\\
\devAya{} नमः, अर्घ्यं समर्पयामि।\\
\devAya{} नमः, आचमनीयं समर्पयामि।\\
\devAya{} नमः, मधुपर्कं समर्पयामि।\\
\ifbool{veda}{ॐ भूर्भुवः॒ सुवः॑।}{}
\devAya{} नमः, शुद्धोदकस्नानं समर्पयामि। स्नानानन्तरमाचमनीयं समर्पयामि।\\
\devAya{} नमः, वस्त्रार्थमक्षतान् समर्पयामि।\\
\devAya{} नमः, यज्ञोपवीताभरणार्थे अक्षतान् समर्पयामि।\\
\devAya{} नमः, दिव्यपरिमलगन्धान् धारयामि। गन्धस्योपरि हरिद्राकुङ्कुमं समर्पयामि। \\
\devAya{} नमः, अक्षतान् समर्पयामि। \\
\devAya{} नमः, पुष्पमालिकां समर्पयामि। पुष्पैः पूजयामि।

\dnsub{अर्चना}
\begin{enumerate}
\begin{minipage}{0.475\linewidth}   
  \item सुमुखाय नमः
  \item एकदन्ताय नमः
  \item कपिलाय नमः
  \item गजकर्णकाय नमः
  \item लम्बोदराय नमः
  \item विकटाय नमः
  \item विघ्नराजाय नमः
  \item विनायकाय नमः
\end{minipage}
\begin{minipage}{0.525\linewidth}
  \item धूमकेतवे नमः
  \item गणाध्यक्षाय नमः
  \item फालचन्द्राय नमः
  \item गजाननाय नमः
  \item वक्रतुण्डाय नमः
  \item शूर्पकर्णाय नमः
  \item हेरम्बाय नमः
  \item स्कन्दपूर्वजाय नमः
\end{minipage}
\end{enumerate}
\devAya{} नमः, नानाविधपरिमलपत्रपुष्पाणि समर्पयामि॥\\
\devAya{} नमः, धूपमाघ्रापयामि।\\
अलङ्कारदीपं सन्दर्शयामि।\\
% नैवेद्यम्।\\
\ifbool{veda}{ॐ भूर्भुवः॒ सुवः॑। + ब्र॒ह्मणे॒ स्वाहा᳚।}{}
\devAya{} नमः, \blank{} (नैवेद्यं) निवेदयामि। 
\ifbool{veda}{अ॒मृ॒ता॒पि॒धा॒नम॑सि॥}{}
निवेदनान्तरम् आचमनीयं समर्पयामि।\\
\devAya{} नमः, ताम्बूलं समर्पयामि।\\
\devAya{} नमः, कर्पूरनीराजनं दर्शयामि। कर्पूरनीराजनानन्तरमाचमनीयं समर्पयामि।\\
\devAya{} नमः, मन्त्रपुष्पं समर्पयामि। स्वर्णपुष्पं समर्पयामि।\\

\twolineshloka*
{अभीप्सितार्थसिद्ध्यर्थं पूजितो यः सुरैरपि}
{सर्वविघ्नच्छिदे तस्मै गणाधिपतये नमः}

\twolineshloka*
{गजाननं भूतगणादिसेवितं कपित्थ-जम्बूफल-सार-भक्षितम्}
{उमासुतं शोकविनाशकारणं नमामि विघ्नेश्वरपादपङ्कजम्}

\centerline{अनन्तकोटिप्रदक्षिणनमस्कारान् समर्पयामि।}

\centerline{छत्त्रचामरादिसमस्तोपचारान् समर्पयामि।}

\twolineshloka*
{वक्रतुण्डमहाकाय कोटिसूर्यसमप्रभ}
{अविघ्नं कुरु मे देव सर्वकार्येषु सर्वदा}

\twolineshloka*
{सुमुखश्चैकदन्तश्च कपिलो गजकर्णकः}
{लम्बोदरश्च विकटो विघ्नराजो गणाधिपः}

\twolineshloka*
{धूमकेतुर्गणाध्यक्षो फालचन्द्रो गजाननः}
{वक्रतुण्डः शूर्पकर्णो हेरम्भः स्कन्दपूर्वजः}

\threelineshloka*
{षोडशैतानि नामानि यः पठेच्छृणुयादपि}
{विद्यारम्भे विवाहे च प्रवेशे निर्गमे तथा}
{सङ्ग्रामे सर्वकार्येषु विघ्नस्तस्य न जायते}

\centerline{प्रार्थनाः समर्पयामि।}

\closesub

\sect{प्रधान-पूजा — श्री-कामाक्षी-पूजा}

\shuklambaradharam

\renewcommand{\ayane}{उत्तरायणे}
\renewcommand{\rtu}{शिशिर}
\renewcommand{\masa}{कुम्भ}
\renewcommand{\prityartham}{}
\renewcommand{\pujaam}{श्रीकामाक्षी-पूजां}
\renewcommand{\regularSankalpa}{}
\renewcommand{\additionalSankalpa}{श्रीपरमेश्वर-प्रीत्यर्थं कामाक्ष्याः प्रीत्यर्थं कामाक्ष्याः प्रसादेन मम दीर्घ सौमाङ्गल्य-अवाप्त्यर्थं मम भर्तुश्च अन्योन्यप्राप्त्यर्थम् अवियोगार्थं}

\sankalpa{}

\vighneshvaraYathasthanam

\dnsub{आसन-पूजा}
\centerline{पृथिव्या  मेरुपृष्ठ  ऋषिः।  सुतलं  छन्दः।  कूर्मो  देवता॥}
\twolineshloka*
{पृथ्वि  त्वया  धृता  लोका  देवि  त्वं  विष्णुना  धृता}
{त्वं  च  धारय  मां  देवि  पवित्रं  चाऽऽसनं  कुरु}


\dnsub{घण्टा-पूजा}

\twolineshloka*
{आगमार्थं तु देवानां गमनार्थं तु रक्षसाम्}
{घण्टारवं करोम्यादौ देवताऽऽह्वानकारणम्}


\dnsub{कलशपूजा}
ॐ कलशाय नमः दिव्यगन्धान् धारयामि।\\
ॐ गङ्गायै नमः। ॐ यमुनायै नमः। ॐ गोदावर्यै नमः।  ॐ सरस्वत्यै नमः। ॐ नर्मदायै नमः। ॐ सिन्धवे नमः। ॐ कावेर्यै नमः।\\
ॐ सप्तकोटिमहातीर्थान्यावाहयामि।\\[-0.25ex]

(अथ कलशं स्पृष्ट्वा जपं कुर्यात्) \\
आपो॒ वा इ॒द सर्वं॒ विश्वा॑ भू॒तान्याप॑ प्रा॒णा वा आप॑ प॒शव॒ आपो\-ऽन्न॒मापोऽमृ॑त॒माप॑ स॒म्राडापो॑ वि॒राडाप॑ स्व॒राडाप॒श्\-छन्दा॒स्यापो॒ ज्योती॒ष्यापो॒ यजू॒ष्याप॑ स॒त्यमाप॒ सर्वा॑ दे॒वता॒ आपो॒ भूर्भुव॒ सुव॒राप॒ ओम्॥\\

\twolineshloka* 
{कलशस्य मुखे विष्णुः कण्ठे रुद्रः समाश्रितः}
{मूले तत्र स्थितो ब्रह्मा मध्ये मातृगणाः स्मृताः}
\threelineshloka* 
{कुक्षौ तु सागराः सर्वे सप्तद्वीपा वसुन्धरा}
{ऋग्वेदोऽथ यजुर्वेदः सामवेदोऽप्यथर्वणः}
{अङ्गैश्च सहिताः सर्वे कलशाम्बुसमाश्रिताः}
\twolineshloka* 
{गङ्गे च यमुने चैव गोदावरि सरस्वति}
{नर्मदे सिन्धुकावेरि जलेऽस्मिन् सन्निधिं कुरु}
\twolineshloka*
{सर्वे समुद्राः सरितः तीर्थानि च ह्रदा नदाः}
{आयान्तु देवपूजार्थं दुरितक्षयकारकाः}

\centerline{ॐ भूर्भुवः॒ सुवो॒ भूर्भुवः॒ सुवो॒ भूर्भुवः॒ सुवः॑।}

(इति कलशजलेन सर्वोपकरणानि आत्मानं च प्रोक्ष्य।)


\dnsub{आत्मपूजा}
ॐ आत्मने नमः, दिव्यगन्धान् धारयामि।
\begin{multicols}{2}
१. ॐ आत्मने नमः\\
२. ॐ अन्तरात्मने नमः\\
३. ॐ योगात्मने नमः\\
४. ॐ जीवात्मने नमः\\
५. ॐ परमात्मने नमः\\
६. ॐ ज्ञानात्मने नमः
\end{multicols}
समस्तोपचारान् समर्पयामि।

\twolineshloka*
{देहो देवालयः प्रोक्तो जीवो देवः सनातनः}
{त्यजेदज्ञाननिर्माल्यं सोऽहं भावेन पूजयेत्}


\begin{minipage}{\linewidth}
\dnsub{पीठ-पूजा}

\begin{multicols}{2}
\begin{enumerate}
\item आधारशक्त्यै नमः
\item मूलप्रकृत्यै नमः
\item आदिकूर्माय नमः 
\item आदिवराहाय नमः
\item अनन्ताय नमः
\item पृथिव्यै नमः
\item रत्नमण्डपाय नमः
\item रत्नवेदिकायै नमः
\item स्वर्णस्तम्भाय नमः
\item श्वेतच्छत्त्राय नमः
\item कल्पकवृक्षाय नमः
\item क्षीरसमुद्राय नमः 
\item सितचामराभ्यां नमः
\item योगपीठासनाय नमः
\end{enumerate}
\end{multicols}

\end{minipage}

\dnsub{गुरु-ध्यानम्}

\twolineshloka*
{गुरुर्ब्रह्मा गुरुर्विष्णुर्गुरुर्देवो महेश्वरः}
{गुरुः साक्षात् परं ब्रह्म तस्मै श्री-गुरवे नमः}


\sect{षोडशोपचार-पूजा}

\renewcommand{\devAya}{कामाक्ष्यै नमः,}

\begin{center}

\twolineshloka*
{एकाम्रनाथ-दयितां कामाक्षीं भुवनेश्वरीम्}
{ध्यायामि हृदये देवीं वाञ्छितार्थप्रदायिनीम्}
\textbf{कामाक्षीं ध्यायामि।}
\medskip

\twolineshloka*
{सर्वमङ्गल-माङ्गल्ये शिवे सर्वार्थदायिनि}
{आवाहयामि कुम्भेऽस्मिन् मम माङ्गल्य-सिद्धये}
\textbf{कामाक्षीम् आवाहयामि।}
\medskip

\twolineshloka*
{कामाक्षि वरदे देवि काञ्चनेन विनिर्मितम्}
{भक्त्या दास्ये स्वीकुरुष्व वरदा भव चासनम्}
\textbf{\devAya{} आसनं समर्पयामि।}
\medskip

\twolineshloka*
{गङ्गादि-सर्व-तीर्थेभ्यः नदीभ्यश्च समाहृतम्}
{पाद्यं सम्प्रददे देवि गृहाण त्वं शिवप्रिये}
\textbf{\devAya{} पाद्यं समर्पयामि।}
\medskip

\twolineshloka*
{कामाक्षि स्वर्ण-कलशेनाहृतं च मया शिवे}
{मधुकैटभ-हन्त्रि त्वं ददाम्यर्घ्यं गृहाण भोः}
\textbf{\devAya{} अर्घ्यं समर्पयामि।}
\medskip

\twolineshloka*
{आचम्यतां महादेवि एलोशीर-सुवासितम्}
{ददामि तीर्थममलं गृहीत्वा लोकरक्षके}
\textbf{\devAya{} आचमनीयं समर्पयामि।}
\medskip

\twolineshloka*
{मधुपर्कं मया देवि काञ्चीपुर-निवासिनि}
{स्वीकृत्य दयया देहि चिरं मह्यं तु मङ्गलम्}
\textbf{\devAya{} मधुपर्कं समर्पयामि।}
\medskip

\twolineshloka*
{पञ्चामृतमिदं दिव्यं पञ्चपातक-नाशनम्}
{पञ्चभूतात्मके देवि पाहि स्वीकृत्य शङ्करि}
\textbf{\devAya{} पञ्चामृत-स्नानं समर्पयामि।}
\medskip

\twolineshloka*
{स्नास्यतां पापनाशाय या प्रवृत्ता सुरापगा}
{मयाऽर्पिता त्वं गृह्णीष्व प्रीता भव दयानिधे}
\textbf{\devAya{} स्नानं समर्पयामि।}
\medskip

\twolineshloka*
{दुकूलान्यम्बराणीह वस्त्राणि विविधानि च}
{ददामि हरदेवीशि विद्याधिष्टान-पीठिके}
\textbf{\devAya{} वस्त्रं समर्पयामि।}
\medskip

\twolineshloka*
{उपवीतं मया प्रीत्यै काञ्चनेन विनिर्मितम्}
{गृहीत्वा तव मे भक्तिं प्रयच्छ करुणानिधे}
\textbf{\devAya{} यज्ञोपवीतं समर्पयामि।}
\medskip

\twolineshloka*
{गन्धं सुवासितं रत्नं कुङ्कुमान्वितम् अम्बिके}
{गङ्गानुजे देहि मह्यं दीर्घमङ्गल-सूत्रकम्}
\textbf{\devAya{} गन्धान् धारयामि। हरिद्राकुङ्कुमं समर्पयामि।}
\medskip

\twolineshloka*
{कार्पास-सूत्रं दास्यामि सुवर्णमणि-संयुतम्}
{भूषणार्थं मयाऽऽनीतं देहि मे वरमुत्तमम्}
\textbf{\devAya{} मङ्गलसूत्रं समर्पयामि।}
\medskip

\twolineshloka*
{जातीचम्पक-पुन्नाग-केतकी-वकुलानि च}
{मयाऽर्पितानि सुभगे गृहाण जननि मम}
\textbf{\devAya{} पुष्पाणि समर्पयामि।}
\medskip

\end{center}

\section{अङ्ग-पूजा}

\begin{longtable}{ll@{— }l}
१. & कामाक्ष्यै नमः & पादौ पूजयामि\\
२. & कल्मषघ्न्यै नमः & गुल्फौ पूजयामि\\
३. & विद्याप्रदायिन्यै नमः & जङ्घे पूजयामि\\
४. & करुणामृत-सागरायै नमः & जानुनी पूजयामि\\
५. & वरदायै नमः & ऊरू पूजयामि\\
६. & काञ्चीनगर-वासिन्यै नमः & कटिं पूजयामि\\
७. & कन्दर्प-जनन्यै नमः & नाभिं पूजयामि\\
८. & पुरमथन-पुण्यकोट्यै नमः& वक्षः पूजयामि\\
९. & महाज्ञान-दायिन्यै नमः & स्तनौ पूजयामि\\
१०. & लोकमात्रे नमः & कण्ठं पूजयामि\\
११. & मायायै नमः & नेत्रे पूजयामि\\
१२. & मधुरवेणी-सहोदर्यै नमः & ललाटं पूजयामि\\
१३. & एकाम्र-नाथायै नमः & कर्णौ पूजयामि\\
१४. & कामकोटि-निलयायै नमः & शिरः पूजयामि\\
१५. & कामेश्वर्यै नमः & चिकुरं पूजयामि\\
१६. & कामितार्थ-दायिन्यै नमः & धम्मिल्लं पूजयामि\\
१७. & कामाक्ष्यै नमः & सर्वाण्यङ्गानि पूजयामि\\
\end{longtable}

\begingroup
\centering
\setlength{\columnseprule}{1pt}
\let\chapt\sect
\needspace{6em}
\input{../namavali-manjari/100/Kamakshi_108.tex}

\endgroup

\sect{उत्तराङ्ग-पूजा}

\begin{center}

\twolineshloka*
{एकाम्रनाथ-दयिते काञ्चीपुर-निवासिनि}
{धूपं गृहाण देवि त्वं सर्वाभीष्ट-प्रदायिनि}
\textbf{\devAya{} धूपम् आघ्रापयामि।}
\medskip

\twolineshloka*
{घृतवर्ति-समायुक्तं सर्वलोक-प्रकाशकम्}
{दीपं गृह्णीष्व सुभगे वाञ्छितार्थ-प्रदायिनि}
\textbf{\devAya{} दीपं सन्दर्शयामि।}
\medskip

\twolineshloka*
{गुडापूपत्रयं देवि साढकं प्रददाम्यहम्}
{नवनीतयुतं देवि मोदकापूपसंयुतम्}
\twolineshloka*
{पायसं सघृतं दद्यां सफलं लड्डुकान्वितम्}
{मम भर्तुस्सदा देवि गृहीत्वा प्रीतिदा भव}
\textbf{\devAya{} नैवेद्यं निवेदयामि।}
\medskip

\twolineshloka*
{पूगीफल-समायुक्तं नागवल्लिदलैर्युतम्}
{कर्पूरचूर्णसंयुक्तं ताम्बूलं प्रतिगृह्यताम्}
\textbf{\devAya{} कर्पूरताम्बूलं निवेदयामि।}
\medskip

\twolineshloka*
{कर्पूरदीपं सुभगे सर्वमङ्गल-वर्धनम्}
{सर्वव्याधिहरं देवि गृह्यताम् अम्बिके शिवे}
\textbf{\devAya{} मम दीर्घसौमाङ्गल्यता-सिद्ध्यर्थं कर्पूर-नीराञ्जनं सन्दर्शयामि।}
\medskip

\fourlineindentedshloka*
{कान्ता कामदुघा करीन्द्रगमना कामारिवामाङ्कगा}
{कल्याणी कलितावतारसुभगा कस्तूरिकाचर्चिता}
{कम्पातीररसालमूलनिलया कारुण्यकल्लोलिनी}
{कल्याणानि करोतु मे भगवती काञ्चीपुरीदेवता}

\twolineshloka*
{मङ्गले मङ्गलाधारे माङ्गल्ये मङ्गलप्रदे}
{मङ्गलाढ्ये मङ्गलेशे मङ्गलं देहि मे भवे}

\twolineshloka*
{नमो देव्यै महादेव्यै लोकमात्रे नमो नमः}
{शिवायै शिवरूपिण्यै भक्ताभीष्टप्रदा भव}

\twolineshloka*
{कामाक्षि काञ्चिनिलये मम माङ्गल्यवृद्धये}
{नमस्करोमि देवेशि मह्यं कुरु दयां शिवे}
\textbf{\devAya{} अनन्तकोटि-प्रदक्षिण-नमस्कारान् समर्पयामि।}
\medskip

कामाक्षी स्वरूपस्य ब्राह्मणस्य इदमासनम् — सकलाराधनैः स्वर्चितम्।

\twolineshloka*
{कामाक्षी काम-वृद्ध्यर्थं मम माङ्गल्य-सिद्धये}
{उपायनं प्रदास्यामि ददामोघं वरं मम}

इदं हिरण्यं सदक्षिणाकं सताम्बूलं कामाक्षी-स्वरूपाय ब्राह्मणाय सम्प्रददे न मम॥

\bigskip

% \tameng{சரடு கட்டிக்கொள்ளும் போது சொல்ல வேண்டிய ஶ்லோகம்}{While tying the saradu one should recite the following shlokam}
\dnsub{दोर-बन्धनम्}

\twolineshloka*
{दोरं गृह्णामि सुभगे सहारिद्रं धराम्यहम्}
{भर्तुरायुष्य-सिद्ध्यर्थं सुप्रीता भव सर्वदा}

\end{center}

\fourlineindentedshloka*
{कायेन वाचा मनसेन्द्रियैर्वा}
{बुद्ध्यात्मना वा प्रकृतेः स्वभावात्}
{करोमि यद्यत् सकलं परस्मै}
{नारायणायेति समर्पयामि।}

\closesub

\sect{श्री-कामाक्षी-चूर्णिका}

श्री-चन्द्र-मौलीश्वराय नमः। श्री-कामाक्षी-देव्यै नमः।

जय जय श्री-काम-गिरीन्द्र-निलये!

जय जय श्री-कामकोटि-पीठ-स्थिते!

जय जय श्री-त्रिचत्वारिंशत्-कोण-श्रीचक्रान्तराल-बिन्दु-पीठोपरि-लसत्-पञ्च-ब्रह्म-मय-मञ्च-मध्य-स्थ-श्री-शिव-कामेश-वामाङ्क-निलये!

जय जय श्री-विधि-हरि-हर-सुर-गण-वन्दित-चरणारविन्द-युगले!

जय जय श्री-रमा-वाणीन्द्राणी-प्रमुख-रमणी-कर-कमल-समर्पित-चरण-कमले!

जय जय श्री-निखिल-निगमागम-सकल-संवेद्यमान-विविध-वस्त्रालङ्कृत-हेम-निर्मित-अनर्घ-भूषण-भूषित-दिव्य-मूर्ते!

जय जय श्रीमद्-अनवरताभिषेक-धूप-दीप-नैवेद्यादि-नाना-विधोपचारैः परिशोभिते!

जय जय श्री-काञ्ची-नगर्यां द्वात्रिंशद्-धर्म-प्रतिपादनार्थ-स्थापित-हेम-ध्वजालङ्कृते!

जय जय श्री-सकल-मन्त्र-तन्त्र-यन्त्र-मय-परा-बिलाकाश-स्वरूपे!

जय जय श्री-काञ्ची-नगर्यां कामाक्षीति प्रख्यात-नामाङ्किते!

जय जय श्री-महात्रिपुरसुन्दरि बहु पराक्!

\kshama

\closesub

\begingroup
\ifbool{katha}{\input{kathas/pativrata-mahatyma-parva.tex}}{}
\endgroup

\closesection
