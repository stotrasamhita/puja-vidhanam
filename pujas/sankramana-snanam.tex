% !TeX program = XeLaTeX
% !TeX root = ../pujavidhanam.tex

\setlength{\parindent}{0pt}
\chapt{सङ्क्रमण-पुण्यकाल-स्नान-सङ्कल्पः}

\begin{center}
(आचम्य) 
\twolineshloka*
{शुक्लाम्बरधरं विष्णुं शशिवर्णं चतुर्भुजम्}
{प्रसन्नवदनं ध्यायेत् सर्वविघ्नोपशान्तये}
 
प्राणान्  आयम्य।  ॐ भूः + भूर्भुवः॒ सुव॒रोम्।

आचमनम्। शुक्लाम्बरधरं + शान्तये। प्राणायामः।


\twolineshloka*
{तदेव लग्नं सुदिनं तदेव ताराबलं चन्द्रबलं तदेव}
{विद्याबलं दैवबलं तदेव लक्ष्मीपतेरङ्घ्रियुगं स्मरामि}

\twolineshloka*
{अपवित्रः पवित्रो वा सर्वावस्थागतोऽपि वा}
{यः स्मरेत्पुण्डरीकाक्षं स बाह्याभ्यन्तरः शुचिः}

\twolineshloka*
{मानसं वाचिकं पापं कर्मणा समुपार्जितम्}
{श्रीरामः स्मरणेनैव व्यपोहति न संशयः}

श्रीराम राम राम।

\twolineshloka*
{तिथिर्विष्णुस्तथा वारो नक्षत्रं विष्णुरेव च}
{योगश्च करणं चैव सर्वं विष्णुमयं जगत्}

श्रीहरे गोविन्द गोविन्द गोविन्द।

\end{center}

ममोपात्त-समस्त-दुरित-क्षयद्वारा श्रीपरमेश्वरप्रीत्यर्थम्, अद्य –\\
श्रीभगवतः विष्णोः नारायणस्य अचिन्त्यया अपरिमितया शक्त्या भ्रियमाणस्य महाजलौघस्य मध्ये 
परिभ्रमताम् \textbf{अनेक\-कोटि\-ब्रह्माण्डा\-नाम् एकतमे}
पृथिवी-अप्-तेजो-वायु-आकाश-अहङ्कार-महद्-अव्यक्तैः आवरणैः आवृते अस्मिन् महति ब्रह्माण्डकरण्डमध्ये
चतुर्दशभुवनान्तर्गते भूमण्डले जम्बू-प्लक्ष-शाक-शाल्मलि-कुश-क्रौञ्च-पुष्कराख्य-सप्तद्वीपमध्ये
\textbf{जम्बूद्वीपे} भारत-किम्पुरुष-हरि-इलावृत-रम्यक-हिरण्मय-कुरु-भद्राश्व-केतुमाल-नववर्षमध्ये \textbf{भारतवर्षे}
इन्द्र-चेरु-ताम्र-गभस्ति-नाग-सौम्य-गन्धर्व-चारण-भरत-नवखण्डमध्ये \textbf{भरतखण्डे}
सुमेरु-निषद-हेमकूट-हिमाचल-माल्यवत्-पारियात्रक-गन्धमादन-कैलास-विन्ध्याचलादि-\textbf{अनेक\-पुण्य\-शैलानां मध्ये}
दण्डकारण्य-चम्पकारण्य-विन्ध्यारण्य-वीक्षारण्य-श्वेतारण्य-वेदारण्यादि-\textbf{अनेक\-पुण्या\-रण्यानां मध्ये}
कर्मभूमौ राम\-सेतु\-केदारयोः मध्ये
भागीरथी-यमुना-नर्मदा-त्रिवेणी-मलापहारिणी-गौतमी-कृष्णवेणी-तुङ्गभद्रा-कावेर्यादि-\textbf{अनेक\-पुण्य\-नदी-विराजिते}
इन्द्रप्रस्थ-यमप्रस्थ-अवन्तिका\-पुरी-हस्तिना\-पुरी-अयोध्या\-पुरी-द्वारका-मथुरा\-पुरी-माया\-पुरी-काशी\-पुरी-काञ्ची\-पुर्यादि-\textbf{अनेक\-पुण्य\-पुरी-विराजिते} –\\
सकलजगत्स्रष्टुः परार्धद्वयजीविनः \textbf{ब्रह्मणः द्वितीयपरार्धे} 
पञ्चाशद्-अब्दादौ प्रथमे वर्षे प्रथमे मासे प्रथमे पक्षे प्रथमे दिवसे अह्नि द्वितीये यामे तृतीये मुहूर्ते
स्वायम्भुव-स्वारोचिष-उत्तम-तामस-रैवत-चाक्षुषाख्येषु षट्सु मनुषु अतीतेषु सप्तमे \textbf{वैवस्वतमन्वन्तरे}
अष्टाविंशतितमे कलियुगे प्रथमे पादे अस्मिन् वर्तमाने व्यावहारिकाणां प्रभवादीनां षष्ट्याः संवत्सराणां मध्ये

(   )\see{app:samvatsara_names} नाम संवत्सरे उत्तरायणे / दक्षिणायने 
(ग्रीष्म / वर्ष / शरद् / हेमन्त / शिशिर / वसन्त) ऋतौ  (मेष / वृषभ / मिथुन / कर्कटक / सिंह / कन्या / तुला / 
वृश्चिक / धनुर् / मकर / कुम्भ / मीन) मासे (शुक्ल / कृष्ण) पक्षे ( ) शुभतिथौ
(इन्दु / भौम / बुध / गुरु / भृगु / स्थिर / भानु) वासरयुक्तायाम्
(  )\see{app:nakshatra_names} नक्षत्र (  )\see{app:yoga_names} नाम  योग  (  ) करण युक्तायां च 

एवं\-गुण\-विशेषण\-विशिष्टायाम् अस्याम् ( ) शुभतिथौ–

अनादि-अविद्या-वासनया प्रवर्तमाने अस्मिन् महति संसारचक्रे विचित्राभिः कर्मगतिभिः विचित्रासु योनिषु
पुनःपुनः अनेकधा जनित्वा केनापि पुण्यकर्मविशेषेण इदानीन्तन-मानुष-द्विजजन्म-विशेषं प्राप्तवतः मम –\\
जन्माभ्यासात् जन्मप्रभृति एतत्क्षणपर्यन्तं बाल्ये कौमारे यौवने मध्यमे वयसि वार्धके च
जागृत्-स्वप्न-सुषुप्ति-अवस्थासु मनो-वाक्-कायाख्य-त्रिकरणचेष्टया कर्मेन्द्रिय-ज्ञानेन्द्रिय-व्यापारैः
सम्भावितानाम् इह जन्मनि जन्मान्तरे च ज्ञानाज्ञानकृतानां महापातकानां महापातक-अनुमन्तृत्वादी\-नां
समपातकानाम् उपपातकानां मलिनी\-करणानां गर्ह्यधन-आदान-उपजीवनादीनाम् अपात्रीकरणानां जातिभ्रंश\-कराणां
विहित\-कर्म\-त्याग-निन्दित\-समाचरणादीनां ज्ञानतः सकृत् कृतानाम् अज्ञानतः असकृत् कृतानां सर्वेषां पापानां
सद्यः अपनोदनार्थं –\\
महागणपत्यादिसमस्तवैदिकदेवतासन्निधौ

( )-पुण्यकाल-स्नानमहं करिष्ये। अप उपस्पृश्य।

\twolineshloka*
{गङ्गा गङ्गेति यो ब्रूयाद्योजनानां शतैरपि}
{मुच्यते सर्वपापेभ्यो विष्णुलोकं स गच्छति}%  ॥ २८॥

\twolineshloka*
{गङ्गे च यमुने चैव गोदावरि सरस्वति}
{नर्मदे सिन्धु कावेरि जलेऽस्मिन् सन्निधिं कुरु}

\twolineshloka*
{अतिक्रूर महाकाय कल्पान्तदहनोपम}
{भैरवाय नमस्तुभ्यम् अनुज्ञां दातुम् अर्हसि}

\centerline{(प्रोक्षण-मन्त्राः/स्नान-मन्त्राः)} 

स्नात्वा वस्त्रं धृत्वा कुलाचारवत् पुण्ड्रधारणं च कृत्वा आचम्य।

\closesection