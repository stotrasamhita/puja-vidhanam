% !TeX program = XeLaTeX
% !TeX root = ../pujavidhanam.tex

\setlength{\parindent}{0pt}
\chapt{श्री-लक्ष्मी-नृसिंह-जयन्ती-पूजा}

\input{purvanga/vighneshwara-puja}

\sect{प्रधान-पूजा — श्री-लक्ष्मी-नृसिंहपूजा}

\twolineshloka*
{शुक्लाम्बरधरं विष्णुं शशिवर्णं चतुर्भुजम्}
{प्रसन्नवदनं ध्यायेत् सर्वविघ्नोपशान्तये}
 
प्राणान्  आयम्य।  ॐ भूः + भूर्भुवः॒ सुव॒रोम्।

\dnsub{सङ्कल्पः}

ममोपात्त-समस्त-दुरित-क्षयद्वारा श्री-परमेश्वर-प्रीत्यर्थं शुभे शोभने मुहूर्ते अद्य ब्रह्मणः
द्वितीयपरार्धे श्वेतवराहकल्पे वैवस्वतमन्वन्तरे अष्टाविंशतितमे कलियुगे प्रथमे पादे
जम्बूद्वीपे भारतवर्षे भरतखण्डे मेरोः दक्षिणे पार्श्वे शकाब्दे अस्मिन् वर्तमाने व्यावहारिकाणां प्रभवादीनां षष्ट्याः संवत्सराणां मध्ये (	)\see{app:samvatsara_names} नाम संवत्सरे उत्तरायणे / दक्षिणायने 
वसन्तऋतौ  मेषमासे शुक्लपक्षे चतुर्दश्यां शुभतिथौ
(इन्दु / भौम / बुध / गुरु / भृगु / स्थिर / भानु) वासरयुक्तायाम्
(स्वाती/?)\see{app:nakshatra_names} नक्षत्र \mbox{(~~~)}\see{app:yoga_names} नाम  योग  \mbox{(~~~)} करण युक्तायां च एवं गुण विशेषण विशिष्टायाम्
अस्यां	चतुर्दश्यां शुभतिथौ 
अस्माकं सहकुटुम्बानां क्षेमस्थैर्य-धैर्य-वीर्य-विजय-आयुरारोग्य-ऐश्वर्याभिवृद्ध्यर्थम्
 धर्मार्थकाममोक्ष\-चतुर्विधफलपुरुषार्थसिद्ध्यर्थं पुत्रपौत्राभि\-वृद्ध्यर्थम् इष्टकाम्यार्थसिद्ध्यर्थम्
मम इहजन्मनि पूर्वजन्मनि जन्मान्तरे च सम्पादितानां ज्ञानाज्ञानकृतमहा\-पातकचतुष्टय-व्यतिरिक्तानां रहस्यकृतानां प्रकाशकृतानां सर्वेषां पापानां सद्य अपनोदनद्वारा सकल-पापक्षयार्थं श्री~नृसिंह-जयन्ती-पुण्यकाले   यथाशक्ति-ध्यान-आवाहनादि-षोडशो\-पचारैः श्री-नृसिंह-पूजां करिष्ये। तदङ्गं कलशपूजां च करिष्ये।


श्रीविघ्नेश्वराय नमः यथास्थानं प्रतिष्ठापयामि।\\
(गणपति-प्रसादं शिरसा गृहीत्वा)
\renewcommand{\devaName}{विष्णु}

\input{purvanga/aasana-puja}

\input{purvanga/ghanta-puja}

\input{purvanga/kalasha-puja}

\input{purvanga/aatma-puja}

\input{purvanga/pitha-puja}

\input{purvanga/guru-dhyanam}

\begin{center}

\sect{षोडशोपचार-पूजा}

\twolineshloka*
{ध्यायामि देवदेवं तं शङ्खचक्रगदाधरम्}
{नृसिंहं भीषणं भद्रं लक्ष्मीयुक्तं विभूषितम्}

अस्मिन् बिम्बे श्री-लक्ष्मी-नृसिंहं ध्यायामि।
\medskip

\twolineshloka*
{स॒हस्र॑शीर्‌षा॒ पुरु॑षः। स॒ह॒स्रा॒क्षः स॒हस्र॑पात्}
{स भूमिं॑ वि॒श्वतो॑ वृ॒त्वा। अत्य॑तिष्ठद्दशाङ्गु॒लम्}
अस्मिन् बिम्बे श्री-लक्ष्मी-नृसिंहम् आवाहयामि।
\medskip

 \twolineshloka*
 {पुरु॑ष ए॒वेदꣳ सर्वम्᳚। यद्भू॒तं यच्च॒ भव्यम्᳚}
 {उ॒तामृ॑त॒त्वस्येशा॑नः। यदन्ने॑नाति॒रोह॑ति}
 आसनं समर्पयामि।\medskip

\twolineshloka*
{ए॒तावा॑नस्य महि॒मा। अतो॒ ज्यायाꣴ॑श्च॒ पूरु॑षः}
{पादो᳚ऽस्य॒ विश्वा॑ भू॒तानि॑। त्रि॒पाद॑स्या॒मृतं॑ दि॒वि}
 पाद्यं समर्पयामि।\medskip
 
\twolineshloka*
{त्रि॒पादू॒र्ध्व उदै॒त्पुरु॑षः। पादो᳚ऽस्ये॒हाऽऽभ॑वा॒त्पुनः॑}
{ततो॒ विश्व॒ङ्व्य॑क्रामत्। सा॒श॒ना॒न॒श॒ने अ॒भि}
 अर्घ्यं समर्पयामि।\medskip

\twolineshloka*
{तस्मा᳚द्वि॒राड॑जायत। वि॒राजो॒ अधि॒ पूरु॑षः}
{स जा॒तो अत्य॑रिच्यत। प॒श्चाद्भूमि॒मथो॑ पु॒रः}
 आचमनीयं समर्पयामि।\medskip

\twolineshloka*
{यत्पुरु॑षेण ह॒विषा᳚। दे॒वा य॒ज्ञमत॑न्वत}
{व॒स॒न्तो अ॑स्याऽऽसी॒दाज्यम्᳚। ग्री॒ष्म इ॒ध्मः श॒रद्ध॒विः}
मधुपर्कं समर्पयामि।\medskip

 \twolineshloka*
 {स॒प्तास्या॑ऽऽसन्  परि॒धयः॑। त्रिः स॒प्त स॒मिधः॑ कृ॒ताः}
 {दे॒वा यद्य॒ज्ञं त॑न्वा॒नाः। अब॑ध्न॒न् पु॑रुषं प॒शुम्}
 शुद्धोदकस्नानं समर्पयामि। स्नानानन्तरम् आचमनीयं समर्पयामि।\medskip

 \twolineshloka*
 {तं य॒ज्ञं ब॒र्{}हिषि॒ प्रौक्षन्॑। पुरु॑षं जा॒तम॑ग्र॒तः}
 {तेन॑ दे॒वा अय॑जन्त। सा॒ध्या ऋष॑यश्च॒ ये}
 वस्त्रं समर्पयामि।\medskip

\twolineshloka*
{तस्मा᳚द्य॒ज्ञाथ्स॑र्व॒हुतः॑। सम्भृ॑तं पृषदा॒ज्यम्}
{प॒शूꣴस्ताꣴश्च॑क्रे वाय॒व्यान्॑। आ॒र॒ण्यान्ग्रा॒म्याश्च॒ ये}
 यज्ञोपवीतं समर्पयामि।\medskip

\twolineshloka*
{तस्मा᳚द्य॒ज्ञाथ्स॑र्व॒हुतः॑। ऋचः॒ सामा॑नि जज्ञिरे}
{छन्दासि जज्ञिरे॒ तस्मा᳚त्। यजु॒स्तस्मा॑दजायत}
 दिव्यपरिमलगन्धान् धारयामि। गन्धस्योपरि हरिद्राकुङ्कुमं समर्पयामि। अक्षतान् समर्पयामि।\medskip

\twolineshloka*
{तस्मा॒दश्वा॑ अजायन्त। ये के चो॑भ॒याद॑तः}
{गावो॑ ह जज्ञिरे॒ तस्मा᳚त्। तस्मा᳚ज्जा॒ता अ॑जा॒वयः॑}
 पुष्पाणि समर्पयामि।  पुष्पैः पूजयामि।

\dnsub{अङ्ग-पूजा}
\begin{longtable}{ll@{— }l}
१.&	ॐ अनघाय नमः & पादौ पूजयामि	\\
२.&	वामनाय नमः & गुल्फौ पूजयामि\\
३.&	शौरये   नमः & जङ्घे पूजयामि	\\
४.&	वैकुण्ठवासिने नमः & ऊरू पूजयामि	\\
५.&	पुरुषोत्तमाय   नमः & मेढ्रं पूजयामि		\\
६.&	वासुदेवाय   नमः & कटिं पूजयामि	\\
७.&	हृषीकेशाय   नमः & नाभिं पूजयामि\\
८.&   माधवाय नमः & हृदयं पूजयामि\\
९.& मधुसूदनाय   नमः & कण्ठं पूजयामि\\
१०.&	वराहाय   नमः & बाहून् पूजयामि	\\
११.& नृसिंहाय   नमः & हस्तान् पूजयामि	\\
१३.& दैत्यसूदनाय   नमः & मुखं पूजयामि	\\
१६.& दामोदराय   नमः & नासिकां पूजयामि	\\
१४.&	 पुण्डरीकाक्षाय   नमः & नेत्रे पूजयामि\\
१५.& गरुडध्वजाय   नमः & श्रोत्रे पूजयामि	\\
१६.& गोविन्दाय   नमः & ललाटं पूजयामि	\\
१७.& अच्युताय   नमः & शिरः पूजयामि\\
१८.& श्री-नृसिंहाय   नमः &   सर्वाणि अङ्गानि पूजयामि	\\
\end{longtable}

\dnsub{चतुर्विंशति नामपूजा}
\begin{multicols}{2}
\begin{enumerate}
\item ॐ केशवाय नमः
\item ॐ नारायणाय नमः
\item ॐ माधवाय नमः
\item ॐ गोविन्दाय नमः
\item ॐ विष्णवे नमः	
\item ॐ मधुसूदनाय नमः
\item ॐ त्रिविक्रमाय नमः
\item ॐ वामनाय नमः
\item ॐ श्रीधराय नमः
\item ॐ हृषीकेशाय नमः
\item ॐ पद्मनाभाय नमः
\item ॐ दामोदराय नमः
\item ॐ सङ्कर्षणाय नमः
\item ॐ वासुदेवाय नमः
\item ॐ प्रद्युम्नाय नमः
\item ॐ अनिरुद्धाय नमः
\item ॐ पुरुषोत्तमाय नमः
\item ॐ अधोक्षजाय नमः
\item ॐ नृसिंहाय नमः
\item ॐ अच्युताय नमः
\item ॐ जनार्दनाय नमः
\item ॐ उपेन्द्राय नमः 
\item ॐ हरये नमः
\item ॐ श्रीकृष्णाय नमः
\end{enumerate}
\end{multicols}

\begingroup
\centering
\setlength{\columnseprule}{1pt}
\let\chapt\sect
\input{../namavali-manjari/100/LakshmiNrisimha_108.tex}
\endgroup

  
\sect{उत्तराङ्ग-पूजा}

\twolineshloka*
{यत्पुरु॑षं॒ व्य॑दधुः। क॒ति॒धा व्य॑कल्पयन्}
{मुखं॒ किम॑स्य॒ कौ बा॒हू। कावू॒रू पादा॑वुच्येते}
श्री-लक्ष्मी-नृसिंहाय नमः धूपमाघ्रापयामि।\medskip
 
\twolineshloka*
{ब्रा॒ह्म॒णो᳚ऽस्य॒ मुख॑मासीत्। बा॒हू रा॑ज॒न्यः॑ कृ॒तः}
{ऊ॒रू तद॑स्य॒ यद्वैश्यः॑। प॒द्भ्याꣳ शू॒द्रो अ॑जायत}
उद्दी᳚प्यस्व जातवेदोऽप॒घ्नन्निर्ऋ॑तिं॒ मम॑।\\
 प॒शूꣴश्च॒ मह्य॒माव॑ह॒ जीव॑नं च॒ दिशो॑ दिश॥ \\
मा नो॑ हिꣳसीज्जातवेदो॒ गामश्वं॒ पुरु॑षं॒ जग॑त्।\\
अबि॑भ्र॒दग्न॒ आग॑हि श्रि॒या मा॒ परि॑पातय॥ \\
श्री-लक्ष्मी-नृसिंहाय नमः अलङ्कारदीपं सन्दर्शयामि।\medskip

ॐ भूर्भुवः॒ सुवः॑। + ब्र॒ह्मणे॒ स्वाहा᳚।
 \twolineshloka*
 {च॒न्द्रमा॒ मन॑सो जा॒तः। चक्षोः॒ सूर्यो॑ अजायत}
 {मुखा॒दिन्द्र॑श्चा॒ग्निश्च॑। प्रा॒णाद्वा॒युर॑जायत}
श्री-लक्ष्मी-नृसिंहाय नमः (	) पानकं   च   निवेदयामि, \\
अमृतापिधानमसि। निवेदनानन्तरम् आचमनीयं समर्पयामि।\medskip

\twolineshloka*
{नाभ्या॑ आसीद॒न्तरि॑क्षम्। शी॒र्ष्णो द्यौः सम॑वर्तत}
{प॒द्भ्यां भूमि॒र्दिशः॒ श्रोत्रा᳚त्। तथा॑ लो॒काꣳ अ॑कल्पयन्}

\twolineshloka*
{पूगीफलसमायुक्तं नागवल्लीदलैर्युतम्}
{कर्पूरचूर्णसंयुक्तं ताम्बूलं प्रतिगृह्यताम्}
श्री-लक्ष्मी-नृसिंहाय नमः ताम्बूलं समर्पयामि।\medskip

\twolineshloka*
{वेदा॒हमे॒तं पुरु॑षं म॒हान्तम्᳚। आ॒दि॒त्यव॑र्णं॒ तम॑स॒स्तु पा॒रे}
{सर्वा॑णि रू॒पाणि॑ वि॒चित्य॒ धीरः॑। नामा॑नि कृ॒त्वाऽभि॒वद॒न्॒ यदास्ते᳚}
श्री-लक्ष्मी-नृसिंहाय नमः समस्त-अपराध-क्षमापनार्थं कर्पूरनीराजनं दर्शयामि।\\
कर्पूरनीरजनानन्तरम् आचमनीयं समर्पयामि।\medskip

\twolineshloka*
 {धा॒ता पु॒रस्ता॒द्यमु॑दाज॒हार॑। श॒क्रः प्रवि॒द्वान्  प्र॒दिश॒श्चत॑स्रः}
 {तमे॒वं वि॒द्वान॒मृत॑ इ॒ह भ॑वति। नान्यः पन्था॒ अय॑नाय विद्यते}

 यो॑ऽपां पुष्पं॒ वेद॑। पुष्प॑वान् प्र॒जावा᳚न् पशु॒मान् भ॑वति।\\
च॒न्द्रमा॒ वा अ॒पां पुष्पम्᳚। पुष्प॑वान् प्र॒जावा᳚न् पशु॒मान् भ॑वति।\\
य ए॒वं वेद॑। यो॑ऽपामा॒यत॑नं॒ वेद॑। आ॒यत॑नवान् भवति।\medskip

ओं᳚ तद्ब्र॒ह्म। ओं᳚ तद्वा॒युः। ओं᳚ तदा॒त्मा।\\ ओं᳚ तथ्स॒त्यम्‌।
ओं᳚ तथ्सर्वम्᳚‌। ओं᳚ तत्पुरो॒र्नमः॥\medskip

अन्तश्चरति॑ भूते॒षु॒ गुहायां वि॑श्वमू॒र्तिषु। \\
त्वं यज्ञस्त्वं वषट्कारस्त्वमिन्द्रस्त्वꣳ\\ रुद्रस्त्वं विष्णुस्त्वं ब्रह्म त्वं॑ प्रजा॒पतिः। \\
त्वं त॑दाप॒ आपो॒ ज्योती॒ रसो॒ऽमृतं॒ ब्रह्म॒ भूर्भुवः॒ सुव॒रोम्‌॥\medskip

\medskip

श्री-लक्ष्मी-नृसिंहाय नमः वेदोक्तमन्त्रपुष्पाञ्जलिं समर्पयामि।\medskip

\twolineshloka*
{सुवर्णरजतैर्युक्तं चामीकरविनिर्मितम्}
{स्वर्णपुष्पं प्रदास्यामि गृह्यतां मधुसूदन}
स्वर्णपुष्पं समर्पयामि।\medskip

\twolineshloka*
{प्रदक्षिणं करोम्यद्य पापानि नुत माधव}
{मयार्पितान्यशेषाणि परिगृह्य कृपां कुरु}

\twolineshloka*
{यानि कानि च पापानि जन्मान्तरकृतानि च}
{तानि तानि विनश्यन्ति प्रदक्षिण-पदे पदे}
\textbf{प्रदक्षिणं कृत्वा।}
\medskip

\twolineshloka*
{नमस्ते देवदेवेश नमस्ते भक्तवत्सल}
{नमस्ते पुण्डरीकाक्ष वासुदेवाय ते नमः}

\twolineshloka*
{नमः सर्वहितार्थाय जगदाधाररूपिणे}
{साष्टाङ्गोऽयं प्रणामोऽस्तु जगन्नाथ मया कृतः}
अनन्तकोटिप्रदक्षिणनमस्कारान् समर्पयामि।\medskip

\twolineshloka*
{य॒ज्ञेन॑ य॒ज्ञम॑यजन्त दे॒वाः। तानि॒ धर्मा॑णि प्रथ॒मान्या॑सन्}
{ते ह॒ नाकं॑ महि॒मानः॑ सचन्ते। यत्र॒ पूर्वे॑ सा॒ध्याः सन्ति॑ दे॒वाः}
- छत्त्रचामरादिसमस्तोपचारान् समर्पयामि।\medskip


\begin{center}
\input{../stotra-sangrahah/stotras/vishnu/LakshmiNrisimhaKarunarasaStotram}
\input{../stotra-sangrahah/stotras/vishnu/LakshmiNrisimhaPancharatnam}
\sect{श्रीमद्भागवते महापुराणे सप्तमस्कन्धे नवमोऽध्यायः}

\uvacha{श्रीनारद उवाच}

\twolineshloka
{एवं सुरादयः सर्वे ब्रह्मरुद्रपुरः सराः}
{नोपैतुमशकन्मन्यु संरम्भं सुदुरासदम्} %1

\twolineshloka
{साक्षात्श्रीः प्रेषिता देवैर्दृष्ट्वा तं महदद्भुतम्}
{अदृष्टाश्रुतपूर्वत्वात्सा नोपेयाय शङ्किता} %2

\twolineshloka
{प्रह्रादं प्रेषयामास ब्रह्मावस्थितमन्तिके}
{तात प्रशमयोपेहि स्वपित्रे कुपितं प्रभुम्} %3

\twolineshloka
{तथेति शनकै राजन्महाभागवतोऽर्भकः}
{उपेत्य भुवि कायेन ननाम विधृताञ्जलिः} %4

\fourlineindentedshloka
{स्वपादमूले पतितं तमर्भकं}
{विलोक्य देवः कृपया परिप्लुतः}
{उत्थाप्य तच्छीर्ष्ण्यदधात्कराम्बुजं}
{कालाहिवित्रस्तधियां कृताभयम्} %5

\fourlineindentedshloka
{स तत्करस्पर्शधुताखिलाशुभः}
{सपद्यभिव्यक्तपरात्मदर्शनः}
{तत्पादपद्मं हृदि निर्वृतो दधौ}
{हृष्यत्तनुः क्लिन्नहृदश्रुलोचनः} %6

\twolineshloka
{अस्तौषीद्धरिमेकाग्र मनसा सुसमाहितः}
{प्रेमगद्गदया वाचा तन्न्यस्तहृदयेक्षणः} %7

\uvacha{श्रीप्रह्राद उवाच}


\fourlineindentedshloka
{ब्रह्मादयः सुरगणा मुनयोऽथ सिद्धाः}
{सत्त्वैकतानगतयो वचसां प्रवाहैः}
{नाराधितुं पुरुगुणैरधुनापि पिप्रुः}
{किं तोष्टुमर्हति स मे हरिरुग्रजातेः} %8

\fourlineindentedshloka
{मन्ये धनाभिजनरूपतपःश्रुतौजस्}
{तेजःप्रभावबलपौरुषबुद्धियोगाः}
{नाराधनाय हि भवन्ति परस्य पुंसो}
{भक्त्या तुतोष भगवान्गजयूथपाय} %9

\fourlineindentedshloka
{विप्राद्द्विषड्गुणयुतादरविन्दनाभ}
{पादारविन्दविमुखात्श्वपचं वरिष्ठम्}
{मन्ये तदर्पितमनोवचनेहितार्थ}
{प्राणं पुनाति स कुलं न तु भूरिमानः} %10

\fourlineindentedshloka
{नैवात्मनः प्रभुरयं निजलाभपूर्णो}
{मानं जनादविदुषः करुणो वृणीते}
{यद्यज्जनो भगवते विदधीत मानं}
{तच्चात्मने प्रतिमुखस्य यथा मुखश्रीः} %11

\fourlineindentedshloka
{तस्मादहं विगतविक्लव ईश्वरस्य}
{सर्वात्मना महि गृणामि यथा मनीषम्}
{नीचोऽजया गुणविसर्गमनुप्रविष्टः}
{पूयेत येन हि पुमाननुवर्णितेन} %12

\fourlineindentedshloka
{सर्वे ह्यमी विधिकरास्तव सत्त्वधाम्नो}
{ब्रह्मादयो वयमिवेश न चोद्विजन्तः}
{क्षेमाय भूतय उतात्मसुखाय चास्य}
{विक्रीडितं भगवतो रुचिरावतारैः} %13

\fourlineindentedshloka
{तद्यच्छ मन्युमसुरश्च हतस्त्वयाद्य}
{मोदेत साधुरपि वृश्चिकसर्पहत्या}
{लोकाश्च निर्वृतिमिताः प्रतियन्ति सर्वे}
{रूपं नृसिंह विभयाय जनाः स्मरन्ति} %14

\fourlineindentedshloka
{नाहं बिभेम्यजित तेऽतिभयानकास्य}
{जिह्वार्कनेत्रभ्रुकुटीरभसोग्रदंष्ट्रात्}
{आन्त्रस्रजःक्षतजकेशरशङ्कुकर्णान्}
{निर्ह्रादभीतदिगिभादरिभिन्नखाग्रात्} %15

\fourlineindentedshloka
{त्रस्तोऽस्म्यहं कृपणवत्सल दुःसहोग्र}
{संसारचक्रकदनाद्ग्रसतां प्रणीतः}
{बद्धः स्वकर्मभिरुशत्तम तेऽङ्घ्रिमूलं}
{प्रीतोऽपवर्गशरणं ह्वयसे कदा नु} %16

\fourlineindentedshloka
{यस्मात्प्रियाप्रियवियोगसंयोगजन्म}
{शोकाग्निना सकलयोनिषु दह्यमानः}
{दुःखौषधं तदपि दुःखमतद्धियाहं}
{भूमन्भ्रमामि वद मे तव दास्ययोगम्} %17

\fourlineindentedshloka
{सोऽहं प्रियस्य सुहृदः परदेवताया}
{लीलाकथास्तव नृसिंह विरिञ्चगीताः}
{अञ्जस्तितर्म्यनुगृणन्गुणविप्रमुक्तो}
{दुर्गाणि ते पदयुगालयहंससङ्गः} %18

\fourlineindentedshloka
{बालस्य नेह शरणं पितरौ नृसिंह}
{नार्तस्य चागदमुदन्वति मज्जतो नौः}
{तप्तस्य तत्प्रतिविधिर्य इहाञ्जसेष्टस्}
{तावद्विभो तनुभृतां त्वदुपेक्षितानाम्} %19

\fourlineindentedshloka
{यस्मिन्यतो यर्हि येन च यस्य यस्माद्}
{यस्मै यथा यदुत यस्त्वपरः परो वा}
{भावः करोति विकरोति पृथक्स्वभावः}
{सञ्चोदितस्तदखिलं भवतः स्वरूपम्} %20

\fourlineindentedshloka
{माया मनः सृजति कर्ममयं बलीयः}
{कालेन चोदितगुणानुमतेन पुंसः}
{छन्दोमयं यदजयार्पितषोडशारं}
{संसारचक्रमज कोऽतितरेत्त्वदन्यः} %21

\fourlineindentedshloka
{स त्वं हि नित्यविजितात्मगुणः स्वधाम्ना}
{कालो वशीकृतविसृज्यविसर्गशक्तिः}
{चक्रे विसृष्टमजयेश्वर षोडशारे}
{निष्पीड्यमानमुपकर्ष विभो प्रपन्नम्} %22

\fourlineindentedshloka
{दृष्टा मया दिवि विभोऽखिलधिष्ण्यपानाम्}
{आयुः श्रियो विभव इच्छति यान्जनोऽयम्}
{येऽस्मत्पितुः कुपितहासविजृम्भितभ्रू}
{विस्फूर्जितेन लुलिताः स तु ते निरस्तः} %23

\fourlineindentedshloka
{तस्मादमूस्तनुभृतामहमाशिषोऽज्ञ}
{आयुः श्रियं विभवमैन्द्रियमाविरिञ्च्यात्}
{नेच्छामि ते विलुलितानुरुविक्रमेण}
{कालात्मनोपनय मां निजभृत्यपार्श्वम्} %24

\fourlineindentedshloka
{कुत्राशिषः श्रुतिसुखा मृगतृष्णिरूपाः}
{क्वेदं कलेवरमशेषरुजां विरोहः}
{निर्विद्यते न तु जनो यदपीति विद्वान्}
{कामानलं मधुलवैः शमयन्दुरापैः} %25

\fourlineindentedshloka
{क्वाहं रजःप्रभव ईश तमोऽधिकेऽस्मिन्}
{जातः सुरेतरकुले क्व तवानुकम्पा}
{न ब्रह्मणो न तु भवस्य न वै रमाया}
{यन्मेऽर्पितः शिरसि पद्मकरः प्रसादः} %26

\fourlineindentedshloka
{नैषा परावरमतिर्भवतो ननु स्याज्}
{जन्तोर्यथात्मसुहृदो जगतस्तथापि}
{संसेवया सुरतरोरिव ते प्रसादः}
{सेवानुरूपमुदयो न परावरत्वम्} %27

\fourlineindentedshloka
{एवं जनं निपतितं प्रभवाहिकूपे}
{कामाभिकाममनु यः प्रपतन्प्रसङ्गात्}
{कृत्वात्मसात्सुरर्षिणा भगवन्गृहीतः}
{सोऽहं कथं नु विसृजे तव भृत्यसेवाम्} %28

\fourlineindentedshloka
{मत्प्राणरक्षणमनन्त पितुर्वधश्च}
{मन्ये स्वभृत्यऋषिवाक्यमृतं विधातुम्}
{खड्गं प्रगृह्य यदवोचदसद्विधित्सुस्}
{त्वामीश्वरो मदपरोऽवतु कं हरामि} %29

\fourlineindentedshloka
{एकस्त्वमेव जगदेतममुष्य यत्त्वम्}
{आद्यन्तयोः पृथगवस्यसि मध्यतश्च}
{सृष्ट्वा गुणव्यतिकरं निजमाययेदं}
{नानेव तैरवसितस्तदनुप्रविष्टः} %30

\fourlineindentedshloka
{त्वम्वा इदं सदसदीश भवांस्ततोऽन्यो}
{माया यदात्मपरबुद्धिरियं ह्यपार्था}
{यद्यस्य जन्म निधनं स्थितिरीक्षणं च}
{तद्वैतदेव वसुकालवदष्टितर्वोः} %31

\fourlineindentedshloka
{न्यस्येदमात्मनि जगद्विलयाम्बुमध्ये}
{शेषेत्मना निजसुखानुभवो निरीहः}
{योगेन मीलितदृगात्मनिपीतनिद्रस्}
{तुर्ये स्थितो न तु तमो न गुणांश्च युङ्क्षे} %32

\fourlineindentedshloka
{तस्यैव ते वपुरिदं निजकालशक्त्या}
{सञ्चोदितप्रकृतिधर्मण आत्मगूढम्}
{अम्भस्यनन्तशयनाद्विरमत्समाधेर्}
{नाभेरभूत्स्वकणिकावटवन्महाब्जम्} %33

\fourlineindentedshloka
{तत्सम्भवः कविरतोऽन्यदपश्यमानस्}
{त्वां बीजमात्मनि ततं स बहिर्विचिन्त्य}
{नाविन्ददब्दशतमप्सु निमज्जमानो}
{जातेऽङ्कुरे कथमुहोपलभेत बीजम्} %34

\fourlineindentedshloka
{स त्वात्मयोनिरतिविस्मित आश्रितोऽब्जं}
{कालेन तीव्रतपसा परिशुद्धभावः}
{त्वामात्मनीश भुवि गन्धमिवातिसूक्ष्मं}
{भूतेन्द्रियाशयमये विततं ददर्श} %35

\fourlineindentedshloka
{एवं सहस्रवदनाङ्घ्रिशिरःकरोरु}
{नासाद्यकर्णनयनाभरणायुधाढ्यम्}
{मायामयं सदुपलक्षितसन्निवेशं}
{दृष्ट्वा महापुरुषमाप मुदं विरिञ्चः} %36

\fourlineindentedshloka
{तस्मै भवान्हयशिरस्तनुवं हि बिभ्रद्}
{वेदद्रुहावतिबलौ मधुकैटभाख्यौ}
{हत्वानयच्छ्रुतिगणांश्च रजस्तमश्च}
{सत्त्वं तव प्रियतमां तनुमामनन्ति} %37

\fourlineindentedshloka
{इत्थं नृतिर्यगृषिदेवझषावतारैर्}
{लोकान्विभावयसि हंसि जगत्प्रतीपान्}
{धर्मं महापुरुष पासि युगानुवृत्तं}
{छन्नः कलौ यदभवस्त्रियुगोऽथ स त्वम्} %38

\fourlineindentedshloka
{नैतन्मनस्तव कथासु विकुण्ठनाथ}
{सम्प्रीयते दुरितदुष्टमसाधु तीव्रम्}
{कामातुरं हर्षशोकभयैषणार्तं}
{तस्मिन्कथं तव गतिं विमृशामि दीनः} %39

\fourlineindentedshloka
{जिह्वैकतोऽच्युत विकर्षति मावितृप्ता}
{शिश्नोऽन्यतस्त्वगुदरं श्रवणं कुतश्चित्}
{घ्राणोऽन्यतश्चपलदृक्क्व च कर्मशक्तिर्}
{बह्व्यः सपत्न्य इव गेहपतिं लुनन्ति} %40

\fourlineindentedshloka
{एवं स्वकर्मपतितं भववैतरण्याम्}
{अन्योन्यजन्ममरणाशनभीतभीतम्}
{पश्यन्जनं स्वपरविग्रहवैरमैत्रं}
{हन्तेति पारचर पीपृहि मूढमद्य} %41

\fourlineindentedshloka
{को न्वत्र तेऽखिलगुरो भगवन्प्रयास}
{उत्तारणेऽस्य भवसम्भवलोपहेतोः}
{मूढेषु वै महदनुग्रह आर्तबन्धो}
{किं तेन ते प्रियजनाननुसेवतां नः} %42

\fourlineindentedshloka
{नैवोद्विजे पर दुरत्ययवैतरण्यास्}
{त्वद्वीर्यगायनमहामृतमग्नचित्तः}
{शोचे ततो विमुखचेतस इन्द्रियार्थ}
{मायासुखाय भरमुद्वहतो विमूढान्} %43

\fourlineindentedshloka
{प्रायेण देव मुनयः स्वविमुक्तिकामा}
{मौनं चरन्ति विजने न परार्थनिष्ठाः}
{नैतान्विहाय कृपणान्विमुमुक्ष एको}
{नान्यं त्वदस्य शरणं भ्रमतोऽनुपश्ये} %44

\fourlineindentedshloka
{यन्मैथुनादिगृहमेधिसुखं हि तुच्छं}
{कण्डूयनेन करयोरिव दुःखदुःखम्}
{तृप्यन्ति नेह कृपणा बहुदुःखभाजः}
{कण्डूतिवन्मनसिजं विषहेत धीरः} %45

\fourlineindentedshloka
{मौनव्रतश्रुततपोऽध्ययनस्वधर्म}
{व्याख्यारहोजपसमाधय आपवर्ग्याः}
{प्रायः परं पुरुष ते त्वजितेन्द्रियाणां}
{वार्ता भवन्त्युत न वात्र तु दाम्भिकानाम्} %46

\fourlineindentedshloka
{रूपे इमे सदसती तव वेदसृष्टे}
{बीजाङ्कुराविव न चान्यदरूपकस्य}
{युक्ताः समक्षमुभयत्र विचक्षन्ते त्वां}
{योगेन वह्निमिव दारुषु नान्यतः स्यात्} %47

\fourlineindentedshloka
{त्वं वायुरग्निरवनिर्वियदम्बु मात्राः}
{प्राणेन्द्रियाणि हृदयं चिदनुग्रहश्च}
{सर्वं त्वमेव सगुणो विगुणश्च भूमन्}
{नान्यत्त्वदस्त्यपि मनोवचसा निरुक्तम्} %48

\fourlineindentedshloka
{नैते गुणा न गुणिनो महदादयो ये}
{सर्वे मनः प्रभृतयः सहदेवमर्त्याः}
{आद्यन्तवन्त उरुगाय विदन्ति हि त्वाम्}
{एवं विमृश्य सुधियो विरमन्ति शब्दात्} %49

\fourlineindentedshloka
{तत्तेऽर्हत्तम नमः स्तुतिकर्मपूजाः}
{कर्म स्मृतिश्चरणयोः श्रवणं कथायाम्}
{संसेवया त्वयि विनेति षडङ्गया किं}
{भक्तिं जनः परमहंसगतौ लभेत} %50

\uvacha{श्रीनारद उवाच}


\twolineshloka
{एतावद्वर्णितगुणो भक्त्या भक्तेन निर्गुणः}
{प्रह्रादं प्रणतं प्रीतो यतमन्युरभाषत} %51

\uvacha{श्रीभगवानुवाच}


\twolineshloka
{प्रह्राद भद्र भद्रं ते प्रीतोऽहं तेऽसुरोत्तम}
{वरं वृणीष्वाभिमतं कामपूरोऽस्म्यहं नृणाम्} %52

\twolineshloka
{मामप्रीणत आयुष्मन्दर्शनं दुर्लभं हि मे}
{दृष्ट्वा मां न पुनर्जन्तुरात्मानं तप्तुमर्हति} %53

\twolineshloka
{प्रीणन्ति ह्यथ मां धीराः सर्वभावेन साधवः}
{श्रेयस्कामा महाभाग सर्वासामाशिषां पतिम्} %54

\uvacha{श्रीनारद उवाच}

\twolineshloka
{एवं प्रलोभ्यमानोऽपि वरैर्लोकप्रलोभनैः}
{एकान्तित्वाद्भगवति नैच्छत्तानसुरोत्तमः} %॥५५॥\\

॥इति श्रीमद्भागवते महापुराणे पारमहंस्यां संहितायां सप्तमस्कन्धे नवमोऽध्यायः॥


\sect{श्रीमद्भागवते महापुराणे सप्तमस्कन्धे नवमोऽध्यायः}

\uvacha{श्रीनारद उवाच}

\twolineshloka
{एवं सुरादयः सर्वे ब्रह्मरुद्रपुरः सराः}
{नोपैतुमशकन्मन्यु संरम्भं सुदुरासदम्} %1

\twolineshloka
{साक्षात्श्रीः प्रेषिता देवैर्दृष्ट्वा तं महदद्भुतम्}
{अदृष्टाश्रुतपूर्वत्वात्सा नोपेयाय शङ्किता} %2

\twolineshloka
{प्रह्रादं प्रेषयामास ब्रह्मावस्थितमन्तिके}
{तात प्रशमयोपेहि स्वपित्रे कुपितं प्रभुम्} %3

\twolineshloka
{तथेति शनकै राजन्महाभागवतोऽर्भकः}
{उपेत्य भुवि कायेन ननाम विधृताञ्जलिः} %4

\twolineshloka
{स्वपादमूले पतितं तमर्भकं विलोक्य देवः कृपया परिप्लुतः}
{उत्थाप्य तच्छीर्ष्ण्यदधात्कराम्बुजं कालाहिवित्रस्तधियां कृताभयम्} %5

\twolineshloka
{स तत्करस्पर्शधुताखिलाशुभः सपद्यभिव्यक्तपरात्मदर्शनः}
{तत्पादपद्मं हृदि निर्वृतो दधौ हृष्यत्तनुः क्लिन्नहृदश्रुलोचनः} %6

\twolineshloka
{अस्तौषीद्धरिमेकाग्र मनसा सुसमाहितः}
{प्रेमगद्गदया वाचा तन्न्यस्तहृदयेक्षणः} %7

\uvacha{श्रीप्रह्राद उवाच}


\fourlineindentedshloka
{ब्रह्मादयः सुरगणा मुनयोऽथ सिद्धाः}
{सत्त्वैकतानगतयो वचसां प्रवाहैः}
{नाराधितुं पुरुगुणैरधुनापि पिप्रुः}
{किं तोष्टुमर्हति स मे हरिरुग्रजातेः} %8

\fourlineindentedshloka
{मन्ये धनाभिजनरूपतपःश्रुतौजस्-}
{तेजःप्रभावबलपौरुषबुद्धियोगाः}
{नाराधनाय हि भवन्ति परस्य पुंसो}
{भक्त्या तुतोष भगवान्गजयूथपाय} %9

\fourlineindentedshloka
{विप्राद्द्विषड्गुणयुतादरविन्दनाभ}
{पादारविन्दविमुखात्श्वपचं वरिष्ठम्}
{मन्ये तदर्पितमनोवचनेहितार्थ}
{प्राणं पुनाति स कुलं न तु भूरिमानः} %10

\fourlineindentedshloka
{नैवात्मनः प्रभुरयं निजलाभपूर्णो}
{मानं जनादविदुषः करुणो वृणीते}
{यद्यज्जनो भगवते विदधीत मानं}
{तच्चात्मने प्रतिमुखस्य यथा मुखश्रीः} %11

\fourlineindentedshloka
{तस्मादहं विगतविक्लव ईश्वरस्य}
{सर्वात्मना महि गृणामि यथा मनीषम्}
{नीचोऽजया गुणविसर्गमनुप्रविष्टः}
{पूयेत येन हि पुमाननुवर्णितेन} %12

\fourlineindentedshloka
{सर्वे ह्यमी विधिकरास्तव सत्त्वधाम्नो}
{ब्रह्मादयो वयमिवेश न चोद्विजन्तः}
{क्षेमाय भूतय उतात्मसुखाय चास्य}
{विक्रीडितं भगवतो रुचिरावतारैः} %13

\fourlineindentedshloka
{तद्यच्छ मन्युमसुरश्च हतस्त्वयाद्य}
{मोदेत साधुरपि वृश्चिकसर्पहत्या}
{लोकाश्च निर्वृतिमिताः प्रतियन्ति सर्वे}
{रूपं नृसिंह विभयाय जनाः स्मरन्ति} %14

\fourlineindentedshloka
{नाहं बिभेम्यजित तेऽतिभयानकास्य}
{जिह्वार्कनेत्रभ्रुकुटीरभसोग्रदंष्ट्रात्}
{आन्त्रस्रजःक्षतजकेशरशङ्कुकर्णान्}
{निर्ह्रादभीतदिगिभादरिभिन्नखाग्रात्} %15

\fourlineindentedshloka
{त्रस्तोऽस्म्यहं कृपणवत्सल दुःसहोग्र}
{संसारचक्रकदनाद्ग्रसतां प्रणीतः}
{बद्धः स्वकर्मभिरुशत्तम तेऽङ्घ्रिमूलं}
{प्रीतोऽपवर्गशरणं ह्वयसे कदा नु} %16

\fourlineindentedshloka
{यस्मात्प्रियाप्रियवियोगसंयोगजन्म}
{शोकाग्निना सकलयोनिषु दह्यमानः}
{दुःखौषधं तदपि दुःखमतद्धियाहं}
{भूमन्भ्रमामि वद मे तव दास्ययोगम्} %17

\fourlineindentedshloka
{सोऽहं प्रियस्य सुहृदः परदेवताया}
{लीलाकथास्तव नृसिंह विरिञ्चगीताः}
{अञ्जस्तितर्म्यनुगृणन्गुणविप्रमुक्तो}
{दुर्गाणि ते पदयुगालयहंससङ्गः} %18

\fourlineindentedshloka
{बालस्य नेह शरणं पितरौ नृसिंह}
{नार्तस्य चागदमुदन्वति मज्जतो नौः}
{तप्तस्य तत्प्रतिविधिर्य इहाञ्जसेष्टस्-}
{तावद्विभो तनुभृतां त्वदुपेक्षितानाम्} %19

\fourlineindentedshloka
{यस्मिन्यतो यर्हि येन च यस्य यस्माद्}
{यस्मै यथा यदुत यस्त्वपरः परो वा}
{भावः करोति विकरोति पृथक्स्वभावः}
{सञ्चोदितस्तदखिलं भवतः स्वरूपम्} %20

\fourlineindentedshloka
{माया मनः सृजति कर्ममयं बलीयः}
{कालेन चोदितगुणानुमतेन पुंसः}
{छन्दोमयं यदजयार्पितषोडशारं}
{संसारचक्रमज कोऽतितरेत्त्वदन्यः} %21

\fourlineindentedshloka
{स त्वं हि नित्यविजितात्मगुणः स्वधाम्ना}
{कालो वशीकृतविसृज्यविसर्गशक्तिः}
{चक्रे विसृष्टमजयेश्वर षोडशारे}
{निष्पीड्यमानमुपकर्ष विभो प्रपन्नम्} %22

\fourlineindentedshloka
{दृष्टा मया दिवि विभोऽखिलधिष्ण्यपानाम्}
{आयुः श्रियो विभव इच्छति यान्जनोऽयम्}
{येऽस्मत्पितुः कुपितहासविजृम्भितभ्रू}
{विस्फूर्जितेन लुलिताः स तु ते निरस्तः} %23

\fourlineindentedshloka
{तस्मादमूस्तनुभृतामहमाशिषोऽज्ञ}
{आयुः श्रियं विभवमैन्द्रियमाविरिञ्च्यात्}
{नेच्छामि ते विलुलितानुरुविक्रमेण}
{कालात्मनोपनय मां निजभृत्यपार्श्वम्} %24

\fourlineindentedshloka
{कुत्राशिषः श्रुतिसुखा मृगतृष्णिरूपाः}
{क्वेदं कलेवरमशेषरुजां विरोहः}
{निर्विद्यते न तु जनो यदपीति विद्वान्}
{कामानलं मधुलवैः शमयन्दुरापैः} %25

\fourlineindentedshloka
{क्वाहं रजःप्रभव ईश तमोऽधिकेऽस्मिन्}
{जातः सुरेतरकुले क्व तवानुकम्पा}
{न ब्रह्मणो न तु भवस्य न वै रमाया}
{यन्मेऽर्पितः शिरसि पद्मकरः प्रसादः} %26

\fourlineindentedshloka
{नैषा परावरमतिर्भवतो ननु स्याज्}
{जन्तोर्यथात्मसुहृदो जगतस्तथापि}
{संसेवया सुरतरोरिव ते प्रसादः}
{सेवानुरूपमुदयो न परावरत्वम्} %27

\fourlineindentedshloka
{एवं जनं निपतितं प्रभवाहिकूपे}
{कामाभिकाममनु यः प्रपतन्प्रसङ्गात्}
{कृत्वात्मसात्सुरर्षिणा भगवन्गृहीतः}
{सोऽहं कथं नु विसृजे तव भृत्यसेवाम्} %28

\fourlineindentedshloka
{मत्प्राणरक्षणमनन्त पितुर्वधश्च}
{मन्ये स्वभृत्यऋषिवाक्यमृतं विधातुम्}
{खड्गं प्रगृह्य यदवोचदसद्विधित्सुस्-}
{त्वामीश्वरो मदपरोऽवतु कं हरामि} %29

\fourlineindentedshloka
{एकस्त्वमेव जगदेतममुष्य यत्त्वम्}
{आद्यन्तयोः पृथगवस्यसि मध्यतश्च}
{सृष्ट्वा गुणव्यतिकरं निजमाययेदं}
{नानेव तैरवसितस्तदनुप्रविष्टः} %30

\fourlineindentedshloka
{त्वम्वा इदं सदसदीश भवांस्ततोऽन्यो}
{माया यदात्मपरबुद्धिरियं ह्यपार्था}
{यद्यस्य जन्म निधनं स्थितिरीक्षणं च}
{तद्वैतदेव वसुकालवदष्टितर्वोः} %31

\fourlineindentedshloka
{न्यस्येदमात्मनि जगद्विलयाम्बुमध्ये}
{शेषेऽऽत्मना निजसुखानुभवो निरीहः}
{योगेन मीलितदृगात्मनिपीतनिद्रस्-}
{तुर्ये स्थितो न तु तमो न गुणांश्च युङ्क्षे} %32

\fourlineindentedshloka
{तस्यैव ते वपुरिदं निजकालशक्त्या}
{सञ्चोदितप्रकृतिधर्मण आत्मगूढम्}
{अम्भस्यनन्तशयनाद्विरमत्समाधेर्-}
{नाभेरभूत्स्वकणिकावटवन्महाब्जम्} %33

\fourlineindentedshloka
{तत्सम्भवः कविरतोऽन्यदपश्यमानस्-}
{त्वां बीजमात्मनि ततं स बहिर्विचिन्त्य}
{नाविन्ददब्दशतमप्सु निमज्जमानो}
{जातेऽङ्कुरे कथमुहोपलभेत बीजम्} %34

\fourlineindentedshloka
{स त्वात्मयोनिरतिविस्मित आश्रितोऽब्जं}
{कालेन तीव्रतपसा परिशुद्धभावः}
{त्वामात्मनीश भुवि गन्धमिवातिसूक्ष्मं}
{भूतेन्द्रियाशयमये विततं ददर्श} %35

\fourlineindentedshloka
{एवं सहस्रवदनाङ्घ्रिशिरःकरोरु}
{नासाद्यकर्णनयनाभरणायुधाढ्यम्}
{मायामयं सदुपलक्षितसन्निवेशं}
{दृष्ट्वा महापुरुषमाप मुदं विरिञ्चः} %36

\fourlineindentedshloka
{तस्मै भवान्हयशिरस्तनुवं हि बिभ्रद्}
{वेदद्रुहावतिबलौ मधुकैटभाख्यौ}
{हत्वानयच्छ्रुतिगणांश्च रजस्तमश्च}
{सत्त्वं तव प्रियतमां तनुमामनन्ति} %37

\fourlineindentedshloka
{इत्थं नृतिर्यगृषिदेवझषावतारैर्}
{लोकान्विभावयसि हंसि जगत्प्रतीपान्}
{धर्मं महापुरुष पासि युगानुवृत्तं}
{छन्नः कलौ यदभवस्त्रियुगोऽथ स त्वम्} %38

\fourlineindentedshloka
{नैतन्मनस्तव कथासु विकुण्ठनाथ}
{सम्प्रीयते दुरितदुष्टमसाधु तीव्रम्}
{कामातुरं हर्षशोकभयैषणार्तं}
{तस्मिन्कथं तव गतिं विमृशामि दीनः} %39

\fourlineindentedshloka
{जिह्वैकतोऽच्युत विकर्षति मावितृप्ता}
{शिश्नोऽन्यतस्त्वगुदरं श्रवणं कुतश्चित्}
{घ्राणोऽन्यतश्चपलदृक्क्व च कर्मशक्तिर्}
{बह्व्यः सपत्न्य इव गेहपतिं लुनन्ति} %40

\fourlineindentedshloka
{एवं स्वकर्मपतितं भववैतरण्याम्}
{अन्योन्यजन्ममरणाशनभीतभीतम्}
{पश्यन्जनं स्वपरविग्रहवैरमैत्रं}
{हन्तेति पारचर पीपृहि मूढमद्य} %41

\fourlineindentedshloka
{को न्वत्र तेऽखिलगुरो भगवन्प्रयास}
{उत्तारणेऽस्य भवसम्भवलोपहेतोः}
{मूढेषु वै महदनुग्रह आर्तबन्धो}
{किं तेन ते प्रियजनाननुसेवतां नः} %42

\fourlineindentedshloka
{नैवोद्विजे पर दुरत्ययवैतरण्यास्-}
{त्वद्वीर्यगायनमहामृतमग्नचित्तः}
{शोचे ततो विमुखचेतस इन्द्रियार्थ}
{मायासुखाय भरमुद्वहतो विमूढान्} %43

\fourlineindentedshloka
{प्रायेण देव मुनयः स्वविमुक्तिकामा}
{मौनं चरन्ति विजने न परार्थनिष्ठाः}
{नैतान्विहाय कृपणान्विमुमुक्ष एको}
{नान्यं त्वदस्य शरणं भ्रमतोऽनुपश्ये} %44

\fourlineindentedshloka
{यन्मैथुनादिगृहमेधिसुखं हि तुच्छं}
{कण्डूयनेन करयोरिव दुःखदुःखम्}
{तृप्यन्ति नेह कृपणा बहुदुःखभाजः}
{कण्डूतिवन्मनसिजं विषहेत धीरः} %45

\fourlineindentedshloka
{मौनव्रतश्रुततपोऽध्ययनस्वधर्म}
{व्याख्यारहोजपसमाधय आपवर्ग्याः}
{प्रायः परं पुरुष ते त्वजितेन्द्रियाणां}
{वार्ता भवन्त्युत न वात्र तु दाम्भिकानाम्} %46

\fourlineindentedshloka
{रूपे इमे सदसती तव वेदसृष्टे}
{बीजाङ्कुराविव न चान्यदरूपकस्य}
{युक्ताः समक्षमुभयत्र विचक्षन्ते त्वां}
{योगेन वह्निमिव दारुषु नान्यतः स्यात्} %47

\fourlineindentedshloka
{त्वं वायुरग्निरवनिर्वियदम्बु मात्राः}
{प्राणेन्द्रियाणि हृदयं चिदनुग्रहश्च}
{सर्वं त्वमेव सगुणो विगुणश्च भूमन्}
{नान्यत्त्वदस्त्यपि मनोवचसा निरुक्तम्} %48

\fourlineindentedshloka
{नैते गुणा न गुणिनो महदादयो ये}
{सर्वे मनः प्रभृतयः सहदेवमर्त्याः}
{आद्यन्तवन्त उरुगाय विदन्ति हि त्वाम्}
{एवं विमृश्य सुधियो विरमन्ति शब्दात्} %49

\fourlineindentedshloka
{तत्तेऽर्हत्तम नमः स्तुतिकर्मपूजाः}
{कर्म स्मृतिश्चरणयोः श्रवणं कथायाम्}
{संसेवया त्वयि विनेति षडङ्गया किं}
{भक्तिं जनः परमहंसगतौ लभेत} %50

\uvacha{श्रीनारद उवाच}


\twolineshloka
{एतावद्वर्णितगुणो भक्त्या भक्तेन निर्गुणः}
{प्रह्रादं प्रणतं प्रीतो यतमन्युरभाषत} %51

\uvacha{श्रीभगवानुवाच}


\twolineshloka
{प्रह्राद भद्र भद्रं ते प्रीतोऽहं तेऽसुरोत्तम}
{वरं वृणीष्वाभिमतं कामपूरोऽस्म्यहं नृणाम्} %52

\twolineshloka
{मामप्रीणत आयुष्मन्दर्शनं दुर्लभं हि मे}
{दृष्ट्वा मां न पुनर्जन्तुरात्मानं तप्तुमर्हति} %53

\twolineshloka
{प्रीणन्ति ह्यथ मां धीराः सर्वभावेन साधवः}
{श्रेयस्कामा महाभाग सर्वासामाशिषां पतिम्} %54

\uvacha{श्रीनारद उवाच}

\twolineshloka
{एवं प्रलोभ्यमानोऽपि वरैर्लोकप्रलोभनैः}
{एकान्तित्वाद्भगवति नैच्छत्तानसुरोत्तमः} %॥५५॥\\

॥इति श्रीमद्भागवते महापुराणे पारमहंस्यां संहितायां सप्तमस्कन्धे नवमोऽध्यायः॥


\end{center}
\markboth{उत्तराङ्ग-पूजा}{उत्तराङ्ग-पूजा}


\dnsub{अर्घ्यप्रदानम्}
ममोपात्त-समस्त-दुरित-क्षयद्वारा श्रीपरमेश्वरप्रीत्यर्थम्  नृसिंह-जयन्ती-पुण्यकाले  श्री-लक्ष्मी-नृसिंह-पूजान्ते क्षीरार्घ्यप्रदानं करिष्ये॥
\medskip

\threelineshloka*
{हिरण्याक्षवधार्थाय   भूभारोत्तरणाय   च}
{परित्राणाय   साधूनां   जातो   विष्णुर्नृकेसरी}
{गृहाणार्घ्यं   मया   दत्तं   सलक्ष्मी-नृहरे   स्वयम्}
	श्री-लक्ष्मी-नृसिंहाय नमः इदमर्घ्यमिदमर्घ्यमिदमर्घ्यम्॥\medskip

अनेन अर्घ्यप्रदानेन भगवान् सर्वात्मकः\\ श्री-लक्ष्मी-नृसिंहः प्रीयताम्।\medskip

\twolineshloka*
{हिरण्यगर्भगर्भस्थं हेमबीजं विभावसोः}
{अनन्तपुण्यफलदम् अतः शान्तिं प्रयच्छ मे}

श्री-लक्ष्मी-नृसिंह-जयन्ती-पुण्यकाले अस्मिन् मया क्रियमाण\\
महाविष्णुपूजायां यद्देयमुपायनदानं तत्प्रत्याम्नायार्थं हिरण्यं\\
श्री-लक्ष्मी-नृसिंह-प्रीतिं कामयमानः\\
मनसोद्दिष्टाय ब्राह्मणाय सम्प्रददे नमः न मम।\\ 
अनया पूजया श्री-लक्ष्मी-नृसिंहः प्रीयताम्। 
 
\dnsub{विसर्जनम्}

\twolineshloka*
{यस्य स्मृत्या च नामोक्त्या तपः-पूजा-क्रियादिषु}
{न्यूनं सम्पूर्णतां याति सद्यो वन्दे तमच्युतम्}

\twolineshloka*
{इदं व्रतं मया देव कृतं प्रीत्यै तव प्रभो}
{न्यूनं सम्पूर्णतां यातु त्वत्प्रसादाज्जनार्द्दन}

\twolineshloka*
{मद्वंशे   ये   नरा   जाता   ये   जनिष्यन्ति   चापरे}
{तांस्त्वमुद्धर   देवेश   दुःसहाद्भवसागरात्}

\twolineshloka*
{पातकार्णवमग्नस्य   व्याधिदुःखाम्बुवारिभिः}
{तीव्रैश्च  परिभूतस्य   मोहदुःखगतस्य   मे}

\twolineshloka*
{करावलम्बनं   देहि   शेषशायिन्   जगत्पते}
{श्रीनृसिंह   रमाकान्त   भक्तानां   भयनाशन}

\twolineshloka*
{क्षीराब्धिनिवासिन्   त्वं   चक्रपाणे   जनार्दन}
{व्रतेनानेन   देवेश   भुक्तिमुक्तिप्रदो   भव}

\medskip

अस्मात् बिम्बात् श्री-लक्ष्मी-नृसिंहं यथास्थानं प्रतिष्ठापयामि (अक्षतानर्पित्वा देवमुत्सर्जयेत्।)\\
अनया पूजया श्री-लक्ष्मी-नृसिंहः प्रीयताम्।\medskip

\fourlineindentedshloka*
{कायेन वाचा मनसेन्द्रियैर्वा}
{बुद्‌ध्याऽऽत्मना वा प्रकृतेः स्वभावात्}
{करोमि यद्यत् सकलं परस्मै}
{नारायणायेति समर्पयामि}


ॐ तत्सद्ब्रह्मार्पणमस्तु।\medskip

\twolineshloka* 
{सालग्रामशिलावारि पापहारि शरीरिणाम्}
{आजन्मकृतपापानां प्रायश्चित्तं दिने दिने}

\twolineshloka*
{अकालमृत्युहरणं सर्वव्याधिनिवारणम्}
{सर्वपापक्षयकरं विष्णुपादोदकं शुभम्}
 इति तीर्थं पीत्वा शिरसि प्रसादं धारयेत्।

\end{center}

\closesub

\begin{center}
\ifbool{katha}{\input{kathas/nrisimha-jayanti/nrisimha-jayanti-vrata-katha}}{}
\end{center}

\closesection
