% !TeX program = XeLaTeX
% !TeX root = ../pujavidhanam.tex

\setlength{\parindent}{0pt}
\chapt{श्री-लक्ष्मी-नृसिंह-जयन्ती-पूजा}

\dnsub{पूर्वाङ्ग-विघ्नेश्वर-पूजा}

\graphicspath{{purvanga/}{../purvanga/}}

\centerline{\includegraphics[width=1cm]{ganesha.pdf}}

(आचम्य)

\twolineshloka*
{शुक्लाम्बरधरं विष्णुं शशिवर्णं चतुर्भुजम्}
{प्रसन्नवदनं ध्यायेत् सर्वविघ्नोपशान्तये}
 
प्राणान्  आयम्य।

(अप उपस्पृश्य, पुष्पाक्षतान् गृहीत्वा)\\

\twolineshloka*
{तदेव लग्नं सुदिनं तदेव ताराबलं चन्द्रबलं तदेव}
{विद्याबलं दैवबलं तदेव लक्ष्मीपतेरङ्घ्रियुगं स्मरामि}
 
ममोपात्त-समस्त-दुरित-क्षयद्वारा \\
श्री-परमेश्वर-प्रीत्यर्थं करिष्यमाणस्य कर्मणः\\
अविघ्नेन परिसमाप्त्यर्थम् आदौ विघ्नेश्वरपूजां करिष्ये।

(अप उपस्पृश्य)

\ifbool{veda}{
\twolineshloka*
{ॐ ग॒णानां᳚ त्वा ग॒णप॑तिꣳ हवामहे क॒विं क॑वी॒नामु॑प॒मश्र॑वस्तमम्}
{ज्ये॒ष्ठ॒राजं॒ ब्रह्म॑णां ब्रह्मणस्पत॒ आ नः॑ शृ॒ण्वन्नू॒तिभिः॑ सीद॒ साद॑नम्}
}{}

\twolineshloka*
{अगजानानपद्मार्कं गजाननमहर्निशम्}
{अनेकदं तं भक्तानाम् एकदन्तमुपास्महे}

\ifbool{veda}{भूर्भुवः॒ सुव॒रोम्।}{} अस्मिन् हरिद्राबिम्बे सुमुखं महागणपतिं ध्यायामि, आवाहयामि।\\

\renewcommand{\devAya}{\OMshri महागणपतये}

\devAya{} नमः, आसनं समर्पयामि।\\
\devAya{} नमः, पादयोः पाद्यं समर्पयामि।\\
\devAya{} नमः, अर्घ्यं समर्पयामि।\\
\devAya{} नमः, आचमनीयं समर्पयामि।\\
\devAya{} नमः, मधुपर्कं समर्पयामि।\\
\ifbool{veda}{ॐ भूर्भुवः॒ सुवः॑।}{}
\devAya{} नमः, शुद्धोदकस्नानं समर्पयामि। स्नानानन्तरमाचमनीयं समर्पयामि।\\
\devAya{} नमः, वस्त्रार्थमक्षतान् समर्पयामि।\\
\devAya{} नमः, यज्ञोपवीताभरणार्थे अक्षतान् समर्पयामि।\\
\devAya{} नमः, दिव्यपरिमलगन्धान् धारयामि। गन्धस्योपरि हरिद्राकुङ्कुमं समर्पयामि। \\
\devAya{} नमः, अक्षतान् समर्पयामि। \\
\devAya{} नमः, पुष्पमालिकां समर्पयामि। पुष्पैः पूजयामि।

\dnsub{अर्चना}
\begin{enumerate}
\begin{minipage}{0.475\linewidth}   
  \item सुमुखाय नमः
  \item एकदन्ताय नमः
  \item कपिलाय नमः
  \item गजकर्णकाय नमः
  \item लम्बोदराय नमः
  \item विकटाय नमः
  \item विघ्नराजाय नमः
  \item विनायकाय नमः
\end{minipage}
\begin{minipage}{0.525\linewidth}
  \item धूमकेतवे नमः
  \item गणाध्यक्षाय नमः
  \item फालचन्द्राय नमः
  \item गजाननाय नमः
  \item वक्रतुण्डाय नमः
  \item शूर्पकर्णाय नमः
  \item हेरम्बाय नमः
  \item स्कन्दपूर्वजाय नमः
\end{minipage}
\end{enumerate}
\devAya{} नमः, नानाविधपरिमलपत्रपुष्पाणि समर्पयामि॥\\
\devAya{} नमः, धूपमाघ्रापयामि।\\
अलङ्कारदीपं सन्दर्शयामि।\\
% नैवेद्यम्।\\
\ifbool{veda}{ॐ भूर्भुवः॒ सुवः॑। + ब्र॒ह्मणे॒ स्वाहा᳚।}{}
\devAya{} नमः, \blank{} (नैवेद्यं) निवेदयामि। 
\ifbool{veda}{अ॒मृ॒ता॒पि॒धा॒नम॑सि॥}{}
निवेदनान्तरम् आचमनीयं समर्पयामि।\\
\devAya{} नमः, ताम्बूलं समर्पयामि।\\
\devAya{} नमः, कर्पूरनीराजनं दर्शयामि। कर्पूरनीराजनानन्तरमाचमनीयं समर्पयामि।\\
\devAya{} नमः, मन्त्रपुष्पं समर्पयामि। स्वर्णपुष्पं समर्पयामि।\\

\twolineshloka*
{अभीप्सितार्थसिद्ध्यर्थं पूजितो यः सुरैरपि}
{सर्वविघ्नच्छिदे तस्मै गणाधिपतये नमः}

\twolineshloka*
{गजाननं भूतगणादिसेवितं कपित्थ-जम्बूफल-सार-भक्षितम्}
{उमासुतं शोकविनाशकारणं नमामि विघ्नेश्वरपादपङ्कजम्}

\centerline{अनन्तकोटिप्रदक्षिणनमस्कारान् समर्पयामि।}

\centerline{छत्त्रचामरादिसमस्तोपचारान् समर्पयामि।}

\twolineshloka*
{वक्रतुण्डमहाकाय कोटिसूर्यसमप्रभ}
{अविघ्नं कुरु मे देव सर्वकार्येषु सर्वदा}

\twolineshloka*
{सुमुखश्चैकदन्तश्च कपिलो गजकर्णकः}
{लम्बोदरश्च विकटो विघ्नराजो गणाधिपः}

\twolineshloka*
{धूमकेतुर्गणाध्यक्षो फालचन्द्रो गजाननः}
{वक्रतुण्डः शूर्पकर्णो हेरम्भः स्कन्दपूर्वजः}

\threelineshloka*
{षोडशैतानि नामानि यः पठेच्छृणुयादपि}
{विद्यारम्भे विवाहे च प्रवेशे निर्गमे तथा}
{सङ्ग्रामे सर्वकार्येषु विघ्नस्तस्य न जायते}

\centerline{प्रार्थनाः समर्पयामि।}

\closesub

\sect{प्रधान-पूजा — श्री-लक्ष्मी-नृसिंहपूजा}

\twolineshloka*
{शुक्लाम्बरधरं विष्णुं शशिवर्णं चतुर्भुजम्}
{प्रसन्नवदनं ध्यायेत् सर्वविघ्नोपशान्तये}
 
प्राणान्  आयम्य।  ॐ भूः + भूर्भुवः॒ सुव॒रोम्।

\dnsub{सङ्कल्पः}

ममोपात्त-समस्त-दुरित-क्षयद्वारा श्री-परमेश्वर-प्रीत्यर्थं शुभे शोभने मुहूर्ते अद्य ब्रह्मणः
द्वितीयपरार्द्धे श्वेतवराहकल्पे वैवस्वतमन्वन्तरे अष्टाविंशतितमे कलियुगे प्रथमे पादे
जम्बूद्वीपे भारतवर्षे भरतखण्डे मेरोः दक्षिणे पार्श्वे शकाब्दे अस्मिन् वर्तमाने व्यावहारिकाणां प्रभवादीनां षष्ट्याः संवत्सराणां मध्ये (	)\see{app:samvatsara_names} नाम संवत्सरे उत्तरायणे / दक्षिणायने 
वसन्तऋतौ  मेषमासे शुक्लपक्षे चतुर्दश्यां शुभतिथौ
(इन्दु / भौम / बुध / गुरु / भृगु / स्थिर / भानु) वासरयुक्तायाम्
(स्वाती/?)\see{app:nakshatra_names} नक्षत्र \mbox{(~~~)}\see{app:yoga_names} नाम  योग  \mbox{(~~~)} करण युक्तायां च एवं गुण विशेषण विशिष्टायाम्
अस्यां	चतुर्दश्यां शुभतिथौ 
अस्माकं सहकुटुम्बानां क्षेमस्थैर्य-धैर्य-वीर्य-विजय-आयुरारोग्य-ऐश्वर्याभिवृद्ध्यर्थम्
 धर्मार्थकाममोक्ष\-चतुर्विधफलपुरुषार्थसिद्ध्यर्थं पुत्रपौत्राभि\-वृद्ध्यर्थम् इष्टकाम्यार्थसिद्ध्यर्थम्
मम इहजन्मनि पूर्वजन्मनि जन्मान्तरे च सम्पादितानां ज्ञानाज्ञानकृतमहा\-पातकचतुष्टय-व्यतिरिक्तानां रहस्यकृतानां प्रकाशकृतानां सर्वेषां पापानां सद्य अपनोदनद्वारा सकल-पापक्षयार्थं श्री~नृसिंह-जयन्ती-पुण्यकाले   यथाशक्ति-ध्यान-आवाहनादि-षोडशो\-पचारैः श्री-नृसिंह-पूजां करिष्ये। तदङ्गं कलशपूजां च करिष्ये।


श्रीविघ्नेश्वराय नमः यथास्थानं प्रतिष्ठापयामि।\\
(गणपति-प्रसादं शिरसा गृहीत्वा)
\renewcommand{\devaName}{विष्णु}

\dnsub{आसन-पूजा}
\centerline{पृथिव्या  मेरुपृष्ठ  ऋषिः।  सुतलं  छन्दः।  कूर्मो  देवता॥}
\twolineshloka*
{पृथ्वि  त्वया  धृता  लोका  देवि  त्वं  विष्णुना  धृता}
{त्वं  च  धारय  मां  देवि  पवित्रं  चाऽऽसनं  कुरु}


\dnsub{घण्टा-पूजा}

\twolineshloka*
{आगमार्थं तु देवानां गमनार्थं तु रक्षसाम्}
{घण्टारवं करोम्यादौ देवताऽऽह्वानकारणम्}


\dnsub{कलशपूजा}
ॐ कलशाय नमः दिव्यगन्धान् धारयामि।\\
ॐ गङ्गायै नमः। ॐ यमुनायै नमः। ॐ गोदावर्यै नमः।  ॐ सरस्वत्यै नमः। ॐ नर्मदायै नमः। ॐ सिन्धवे नमः। ॐ कावेर्यै नमः।\\
ॐ सप्तकोटिमहातीर्थान्यावाहयामि।\\[-0.25ex]

(अथ कलशं स्पृष्ट्वा जपं कुर्यात्) \\
आपो॒ वा इ॒द सर्वं॒ विश्वा॑ भू॒तान्याप॑ प्रा॒णा वा आप॑ प॒शव॒ आपो\-ऽन्न॒मापोऽमृ॑त॒माप॑ स॒म्राडापो॑ वि॒राडाप॑ स्व॒राडाप॒श्\-छन्दा॒स्यापो॒ ज्योती॒ष्यापो॒ यजू॒ष्याप॑ स॒त्यमाप॒ सर्वा॑ दे॒वता॒ आपो॒ भूर्भुव॒ सुव॒राप॒ ओम्॥\\

\twolineshloka* 
{कलशस्य मुखे विष्णुः कण्ठे रुद्रः समाश्रितः}
{मूले तत्र स्थितो ब्रह्मा मध्ये मातृगणाः स्मृताः}
\threelineshloka* 
{कुक्षौ तु सागराः सर्वे सप्तद्वीपा वसुन्धरा}
{ऋग्वेदोऽथ यजुर्वेदः सामवेदोऽप्यथर्वणः}
{अङ्गैश्च सहिताः सर्वे कलशाम्बुसमाश्रिताः}
\twolineshloka* 
{गङ्गे च यमुने चैव गोदावरि सरस्वति}
{नर्मदे सिन्धुकावेरि जलेऽस्मिन् सन्निधिं कुरु}
\twolineshloka*
{सर्वे समुद्राः सरितः तीर्थानि च ह्रदा नदाः}
{आयान्तु देवपूजार्थं दुरितक्षयकारकाः}

\centerline{ॐ भूर्भुवः॒ सुवो॒ भूर्भुवः॒ सुवो॒ भूर्भुवः॒ सुवः॑।}

(इति कलशजलेन सर्वोपकरणानि आत्मानं च प्रोक्ष्य।)


\dnsub{आत्मपूजा}
ॐ आत्मने नमः, दिव्यगन्धान् धारयामि।
\begin{multicols}{2}
१. ॐ आत्मने नमः\\
२. ॐ अन्तरात्मने नमः\\
३. ॐ योगात्मने नमः\\
४. ॐ जीवात्मने नमः\\
५. ॐ परमात्मने नमः\\
६. ॐ ज्ञानात्मने नमः
\end{multicols}
समस्तोपचारान् समर्पयामि।

\twolineshloka*
{देहो देवालयः प्रोक्तो जीवो देवः सनातनः}
{त्यजेदज्ञाननिर्माल्यं सोऽहं भावेन पूजयेत्}


\begin{minipage}{\linewidth}
\dnsub{पीठ-पूजा}

\begin{multicols}{2}
\begin{enumerate}
\item आधारशक्त्यै नमः
\item मूलप्रकृत्यै नमः
\item आदिकूर्माय नमः 
\item आदिवराहाय नमः
\item अनन्ताय नमः
\item पृथिव्यै नमः
\item रत्नमण्डपाय नमः
\item रत्नवेदिकायै नमः
\item स्वर्णस्तम्भाय नमः
\item श्वेतच्छत्त्राय नमः
\item कल्पकवृक्षाय नमः
\item क्षीरसमुद्राय नमः 
\item सितचामराभ्यां नमः
\item योगपीठासनाय नमः
\end{enumerate}
\end{multicols}

\end{minipage}

\dnsub{गुरु-ध्यानम्}

\twolineshloka*
{गुरुर्ब्रह्मा गुरुर्विष्णुर्गुरुर्देवो महेश्वरः}
{गुरुः साक्षात् परं ब्रह्म तस्मै श्री-गुरवे नमः}


\begin{center}

\sect{षोडशोपचार-पूजा}

\twolineshloka*
{ध्यायामि देवदेवं तं शङ्खचक्रगदाधरम्}
{नृसिंहं भीषणं भद्रं लक्ष्मीयुक्तं विभूषितम्}

अस्मिन् बिम्बे श्री-लक्ष्मी-नृसिंहं ध्यायामि।
\medskip

\twolineshloka*
{स॒हस्र॑शीर्‌षा॒ पुरु॑षः। स॒ह॒स्रा॒क्षः स॒हस्र॑पात्}
{स भूमिं॑ वि॒श्वतो॑ वृ॒त्वा। अत्य॑तिष्ठद्दशाङ्गु॒लम्}
अस्मिन् बिम्बे श्री-लक्ष्मी-नृसिंहम् आवाहयामि।
\medskip

 \twolineshloka*
 {पुरु॑ष ए॒वेदꣳ सर्वम्᳚। यद्भू॒तं यच्च॒ भव्यम्᳚}
 {उ॒तामृ॑त॒त्वस्येशा॑नः। यदन्ने॑नाति॒रोह॑ति}
 आसनं समर्पयामि।\medskip

\twolineshloka*
{ए॒तावा॑नस्य महि॒मा। अतो॒ ज्यायाꣴ॑श्च॒ पूरु॑षः}
{पादो᳚ऽस्य॒ विश्वा॑ भू॒तानि॑। त्रि॒पाद॑स्या॒मृतं॑ दि॒वि}
 पाद्यं समर्पयामि।\medskip
 
\twolineshloka*
{त्रि॒पादू॒र्ध्व उदै॒त्पुरु॑षः। पादो᳚ऽस्ये॒हाऽऽभ॑वा॒त्पुनः॑}
{ततो॒ विश्व॒ङ्व्य॑क्रामत्। सा॒श॒ना॒न॒श॒ने अ॒भि}
 अर्घ्यं समर्पयामि।\medskip

\twolineshloka*
{तस्मा᳚द्वि॒राड॑जायत। वि॒राजो॒ अधि॒ पूरु॑षः}
{स जा॒तो अत्य॑रिच्यत। प॒श्चाद्भूमि॒मथो॑ पु॒रः}
 आचमनीयं समर्पयामि।\medskip

\twolineshloka*
{यत्पुरु॑षेण ह॒विषा᳚। दे॒वा य॒ज्ञमत॑न्वत}
{व॒स॒न्तो अ॑स्याऽऽसी॒दाज्यम्᳚। ग्री॒ष्म इ॒ध्मः श॒रद्ध॒विः}
मधुपर्कं समर्पयामि।\medskip

 \twolineshloka*
 {स॒प्तास्या॑ऽऽसन्  परि॒धयः॑। त्रिः स॒प्त स॒मिधः॑ कृ॒ताः}
 {दे॒वा यद्य॒ज्ञं त॑न्वा॒नाः। अब॑ध्न॒न् पु॑रुषं प॒शुम्}
 शुद्धोदकस्नानं समर्पयामि। स्नानानन्तरम् आचमनीयं समर्पयामि।\medskip

 \twolineshloka*
 {तं य॒ज्ञं ब॒र्{}हिषि॒ प्रौक्षन्॑। पुरु॑षं जा॒तम॑ग्र॒तः}
 {तेन॑ दे॒वा अय॑जन्त। सा॒ध्या ऋष॑यश्च॒ ये}
 वस्त्रं समर्पयामि।\medskip

\twolineshloka*
{तस्मा᳚द्य॒ज्ञाथ्स॑र्व॒हुतः॑। सम्भृ॑तं पृषदा॒ज्यम्}
{प॒शूꣴस्ताꣴश्च॑क्रे वाय॒व्यान्॑। आ॒र॒ण्यान्ग्रा॒म्याश्च॒ ये}
 यज्ञोपवीतं समर्पयामि।\medskip

\twolineshloka*
{तस्मा᳚द्य॒ज्ञाथ्स॑र्व॒हुतः॑। ऋचः॒ सामा॑नि जज्ञिरे}
{छन्दासि जज्ञिरे॒ तस्मा᳚त्। यजु॒स्तस्मा॑दजायत}
 दिव्यपरिमलगन्धान् धारयामि। गन्धस्योपरि हरिद्राकुङ्कुमं समर्पयामि। अक्षतान् समर्पयामि।\medskip

\twolineshloka*
{तस्मा॒दश्वा॑ अजायन्त। ये के चो॑भ॒याद॑तः}
{गावो॑ ह जज्ञिरे॒ तस्मा᳚त्। तस्मा᳚ज्जा॒ता अ॑जा॒वयः॑}
 पुष्पाणि समर्पयामि।  पुष्पैः पूजयामि।

\dnsub{अङ्ग-पूजा}
\begin{longtable}{ll@{— }l}
१.&	ॐ अनघाय नमः & पादौ पूजयामि	\\
२.&	वामनाय नमः & गुल्फौ पूजयामि\\
३.&	शौरये   नमः & जङ्घे पूजयामि	\\
४.&	वैकुण्ठवासिने नमः & ऊरू पूजयामि	\\
५.&	पुरुषोत्तमाय   नमः & मेढ्रं पूजयामि		\\
६.&	वासुदेवाय   नमः & कटिं पूजयामि	\\
७.&	हृषीकेशाय   नमः & नाभिं पूजयामि\\
८.&   माधवाय नमः & हृदयं पूजयामि\\
९.& मधुसूदनाय   नमः & कण्ठं पूजयामि\\
१०.&	वराहाय   नमः & बाहून् पूजयामि	\\
११.& नृसिंहाय   नमः & हस्तान् पूजयामि	\\
१३.& दैत्यसूदनाय   नमः & मुखं पूजयामि	\\
१६.& दामोदराय   नमः & नासिकां पूजयामि	\\
१४.&	 पुण्डरीकाक्षाय   नमः & नेत्रे पूजयामि\\
१५.& गरुडध्वजाय   नमः & श्रोत्रे पूजयामि	\\
१६.& गोविन्दाय   नमः & ललाटं पूजयामि	\\
१७.& अच्युताय   नमः & शिरः पूजयामि\\
१८.& श्री-नृसिंहाय   नमः &   सर्वाणि अङ्गानि पूजयामि	\\
\end{longtable}

\dnsub{चतुर्विंशति नामपूजा}
\begin{multicols}{2}
\begin{enumerate}
\item ॐ केशवाय नमः
\item ॐ नारायणाय नमः
\item ॐ माधवाय नमः
\item ॐ गोविन्दाय नमः
\item ॐ विष्णवे नमः	
\item ॐ मधुसूदनाय नमः
\item ॐ त्रिविक्रमाय नमः
\item ॐ वामनाय नमः
\item ॐ श्रीधराय नमः
\item ॐ हृषीकेशाय नमः
\item ॐ पद्मनाभाय नमः
\item ॐ दामोदराय नमः
\item ॐ सङ्कर्षणाय नमः
\item ॐ वासुदेवाय नमः
\item ॐ प्रद्युम्नाय नमः
\item ॐ अनिरुद्धाय नमः
\item ॐ पुरुषोत्तमाय नमः
\item ॐ अधोक्षजाय नमः
\item ॐ नृसिंहाय नमः
\item ॐ अच्युताय नमः
\item ॐ जनार्दनाय नमः
\item ॐ उपेन्द्राय नमः 
\item ॐ हरये नमः
\item ॐ श्रीकृष्णाय नमः
\end{enumerate}
\end{multicols}

\begingroup
\centering
\setlength{\columnseprule}{1pt}
\let\chapt\sect
\input{../namavali-manjari/100/LakshmiNrisimha_108.tex}
\endgroup

  
\sect{उत्तराङ्ग-पूजा}

\twolineshloka*
{यत्पुरु॑षं॒ व्य॑दधुः। क॒ति॒धा व्य॑कल्पयन्}
{मुखं॒ किम॑स्य॒ कौ बा॒हू। कावू॒रू पादा॑वुच्येते}
श्री-लक्ष्मी-नृसिंहाय नमः धूपमाघ्रापयामि।\medskip
 
\twolineshloka*
{ब्रा॒ह्म॒णो᳚ऽस्य॒ मुख॑मासीत्। बा॒हू रा॑ज॒न्यः॑ कृ॒तः}
{ऊ॒रू तद॑स्य॒ यद्वैश्यः॑। प॒द्भ्याꣳ शू॒द्रो अ॑जायत}
उद्दी᳚प्यस्व जातवेदोऽप॒घ्नन्निर्ऋ॑तिं॒ मम॑।\\
 प॒शूꣴश्च॒ मह्य॒माव॑ह॒ जीव॑नं च॒ दिशो॑ दिश॥ \\
मा नो॑ हिꣳसीज्जातवेदो॒ गामश्वं॒ पुरु॑षं॒ जग॑त्।\\
अबि॑भ्र॒दग्न॒ आग॑हि श्रि॒या मा॒ परि॑पातय॥ \\
श्री-लक्ष्मी-नृसिंहाय नमः अलङ्कारदीपं सन्दर्शयामि।\medskip

ॐ भूर्भुवः॒ सुवः॑। + ब्र॒ह्मणे॒ स्वाहा᳚।
 \twolineshloka*
 {च॒न्द्रमा॒ मन॑सो जा॒तः। चक्षोः॒ सूर्यो॑ अजायत}
 {मुखा॒दिन्द्र॑श्चा॒ग्निश्च॑। प्रा॒णाद्वा॒युर॑जायत}
श्री-लक्ष्मी-नृसिंहाय नमः (	) पानकं   च   निवेदयामि, \\
अमृतापिधानमसि। निवेदनानन्तरम् आचमनीयं समर्पयामि।\medskip

\twolineshloka*
{नाभ्या॑ आसीद॒न्तरि॑क्षम्। शी॒र्ष्णो द्यौः सम॑वर्तत}
{प॒द्भ्यां भूमि॒र्दिशः॒ श्रोत्रा᳚त्। तथा॑ लो॒काꣳ अ॑कल्पयन्}

\twolineshloka*
{पूगीफलसमायुक्तं नागवल्लीदलैर्युतम्}
{कर्पूरचूर्णसंयुक्तं ताम्बूलं प्रतिगृह्यताम्}
श्री-लक्ष्मी-नृसिंहाय नमः ताम्बूलं समर्पयामि।\medskip

\twolineshloka*
{वेदा॒हमे॒तं पुरु॑षं म॒हान्तम्᳚। आ॒दि॒त्यव॑र्णं॒ तम॑स॒स्तु पा॒रे}
{सर्वा॑णि रू॒पाणि॑ वि॒चित्य॒ धीरः॑। नामा॑नि कृ॒त्वाऽभि॒वद॒न्॒ यदास्ते᳚}
श्री-लक्ष्मी-नृसिंहाय नमः समस्त-अपराध-क्षमापनार्थं कर्पूरनीराजनं दर्शयामि।\\
कर्पूरनीरजनानन्तरम् आचमनीयं समर्पयामि।\medskip

\twolineshloka*
 {धा॒ता पु॒रस्ता॒द्यमु॑दाज॒हार॑। श॒क्रः प्रवि॒द्वान्  प्र॒दिश॒श्चत॑स्रः}
 {तमे॒वं वि॒द्वान॒मृत॑ इ॒ह भ॑वति। नान्यः पन्था॒ अय॑नाय विद्यते}

 यो॑ऽपां पुष्पं॒ वेद॑। पुष्प॑वान् प्र॒जावा᳚न् पशु॒मान् भ॑वति।\\
च॒न्द्रमा॒ वा अ॒पां पुष्पम्᳚। पुष्प॑वान् प्र॒जावा᳚न् पशु॒मान् भ॑वति।\\
य ए॒वं वेद॑। यो॑ऽपामा॒यत॑नं॒ वेद॑। आ॒यत॑नवान् भवति।\medskip

ओं᳚ तद्ब्र॒ह्म। ओं᳚ तद्वा॒युः। ओं᳚ तदा॒त्मा।\\ ओं᳚ तथ्स॒त्यम्‌।
ओं᳚ तथ्सर्वम्᳚‌। ओं᳚ तत्पुरो॒र्नमः॥\medskip

अन्तश्चरति॑ भूते॒षु॒ गुहायां वि॑श्वमू॒र्तिषु। \\
त्वं यज्ञस्त्वं वषट्कारस्त्वमिन्द्रस्त्वꣳ\\ रुद्रस्त्वं विष्णुस्त्वं ब्रह्म त्वं॑ प्रजा॒पतिः। \\
त्वं त॑दाप॒ आपो॒ ज्योती॒ रसो॒ऽमृतं॒ ब्रह्म॒ भूर्भुवः॒ सुव॒रोम्‌॥\medskip

\medskip

श्री-लक्ष्मी-नृसिंहाय नमः वेदोक्तमन्त्रपुष्पाञ्जलिं समर्पयामि।\medskip

\twolineshloka*
{सुवर्णरजतैर्युक्तं चामीकरविनिर्मितम्}
{स्वर्णपुष्पं प्रदास्यामि गृह्यतां मधुसूदन}
स्वर्णपुष्पं समर्पयामि।\medskip

\twolineshloka*
{प्रदक्षिणं करोम्यद्य पापानि नुत माधव}
{मयार्पितान्यशेषाणि परिगृह्य कृपां कुरु}

\twolineshloka*
{यानि कानि च पापानि जन्मान्तरकृतानि च}
{तानि तानि विनश्यन्ति प्रदक्षिण-पदे पदे}
\textbf{प्रदक्षिणं कृत्वा।}
\medskip

\twolineshloka*
{नमस्ते देवदेवेश नमस्ते भक्तवत्सल}
{नमस्ते पुण्डरीकाक्ष वासुदेवाय ते नमः}

\twolineshloka*
{नमः सर्वहितार्थाय जगदाधाररूपिणे}
{साष्टाङ्गोऽयं प्रणामोऽस्तु जगन्नाथ मया कृतः}
अनन्तकोटिप्रदक्षिणनमस्कारान् समर्पयामि।\medskip

\twolineshloka*
{य॒ज्ञेन॑ य॒ज्ञम॑यजन्त दे॒वाः। तानि॒ धर्मा॑णि प्रथ॒मान्या॑सन्}
{ते ह॒ नाकं॑ महि॒मानः॑ सचन्ते। यत्र॒ पूर्वे॑ सा॒ध्याः सन्ति॑ दे॒वाः}
- छत्त्रचामरादिसमस्तोपचारान् समर्पयामि।\medskip


\begin{center}
\input{../stotra-sangrahah/stotras/vishnu/LakshmiNrisimhaKarunarasaStotram}
\input{../stotra-sangrahah/stotras/vishnu/LakshmiNrisimhaPancharatnam}
\sect{श्रीमद्भागवते महापुराणे सप्तमस्कन्धे अष्टमोऽध्यायः}

\uvacha{श्रीनारद उवाच}


\twolineshloka
{अथ दैत्यसुताः सर्वे श्रुत्वा तदनुवर्णितम्}
{जगृहुर्निरवद्यत्वान्नैव गुर्वनुशिक्षितम्} %1

\twolineshloka
{अथाचार्यसुतस्तेषां बुद्धिमेकान्तसंस्थिताम्}
{आलक्ष्य भीतस्त्वरितो राज्ञ आवेदयद्यथा} %2

\twolineshloka
{श्रुत्वा तदप्रियं दैत्यो दुःसहं तनयानयम्}
{कोपावेशचलद्गात्रः पुत्रं हन्तुं मनो दधे} %3

\twolineshloka
{क्षिप्त्वा परुषया वाचा प्रह्रादमतदर्हणम्}
{आहेक्षमाणः पापेन तिरश्चीनेन चक्षुषा} %4

\twolineshloka
{प्रश्रयावनतं दान्तं बद्धाञ्जलिमवस्थितम्}
{सर्पः पदाहत इव श्वसन्प्रकृतिदारुणः} %5

\uvacha{श्रीहिरण्यकशिपुरुवाच}


\twolineshloka
{हे दुर्विनीत मन्दात्मन्कुलभेदकराधम}
{स्तब्धं मच्छासनोद्वृत्तं नेष्ये त्वाद्य यमक्षयम्} %6

\twolineshloka
{क्रुद्धस्य यस्य कम्पन्ते त्रयो लोकाः सहेश्वराः}
{तस्य मेऽभीतवन्मूढ शासनं किं बलोऽत्यगाः} %7

\uvacha{श्रीप्रह्राद उवाच}


\twolineshloka
{न केवलं मे भवतश्च राजन्स वै बलं बलिनां चापरेषाम्}
{परेऽवरेऽमी स्थिरजङ्गमा ये ब्रह्मादयो येन वशं प्रणीताः} %8

\twolineshloka
{स ईश्वरः काल उरुक्रमोऽसावोजः सहः सत्त्वबलेन्द्रियात्मा}
{स एव विश्वं परमः स्वशक्तिभिः सृजत्यवत्यत्ति गुणत्रयेशः} %9

\twolineshloka
{जह्यासुरं भावमिमं त्वमात्मनः समं मनो धत्स्व न सन्ति विद्विषः}
{ऋतेऽजितादात्मन उत्पथे स्थितात्तद्धि ह्यनन्तस्य महत्समर्हणम्} %10

\twolineshloka
{दस्यून्पुरा षण्न विजित्य लुम्पतो मन्यन्त एके स्वजिता दिशो दश}
{जितात्मनो ज्ञस्य समस्य देहिनां साधोः स्वमोहप्रभवाः कुतः परे} %11

\uvacha{श्रीहिरण्यकशिपुरुवाच}


\twolineshloka
{व्यक्तं त्वं मर्तुकामोऽसि योऽतिमात्रं विकत्थसे}
{मुमूर्षूणां हि मन्दात्मन्ननु स्युर्विक्लवा गिरः} %12

\twolineshloka
{यस्त्वया मन्दभाग्योक्तो मदन्यो जगदीश्वरः}
{क्वासौ यदि स सर्वत्र कस्मात्स्तम्भे न दृश्यते} %13

\twolineshloka
{सोऽहं विकत्थमानस्य शिरः कायाद्धरामि ते}
{गोपायेत हरिस्त्वाद्य यस्ते शरणमीप्सितम्} %14

\twolineshloka
{एवं दुरुक्तैर्मुहुरर्दयन्रुषा सुतं महाभागवतं महासुरः}
{खड्गं प्रगृह्योत्पतितो वरासनात्स्तम्भं तताडातिबलः स्वमुष्टिना} %15

\twolineshloka
{तदैव तस्मिन्निनदोऽतिभीषणो बभूव येनाण्डकटाहमस्फुटत्}
{यं वै स्वधिष्ण्योपगतं त्वजादयः श्रुत्वा स्वधामात्ययमङ्ग मेनिरे} %16

\twolineshloka
{स विक्रमन्पुत्रवधेप्सुरोजसा निशम्य निर्ह्रादमपूर्वमद्भुतम्}
{अन्तःसभायां न ददर्श तत्पदं वितत्रसुर्येन सुरारियूथपाः} %17

\twolineshloka
{सत्यं विधातुं निजभृत्यभाषितं व्याप्तिं च भूतेष्वखिलेषु चात्मनः}
{अदृश्यतात्यद्भुतरूपमुद्वहन्स्तम्भे सभायां न मृगं न मानुषम्} %18

\twolineshloka
{स सत्त्वमेनं परितो विपश्यन्स्तम्भस्य मध्यादनुनिर्जिहानम्}
{नायं मृगो नापि नरो विचित्रमहो किमेतन्नृमृगेन्द्ररूपम्} %19

\twolineshloka
{मीमांसमानस्य समुत्थितोऽग्रतो नृसिंहरूपस्तदलं भयानकम्}
{प्रतप्तचामीकरचण्डलोचनं स्फुरत्सटाकेशरजृम्भिताननम्} %20

\twolineshloka
{करालदंष्ट्रं करवालचञ्चल क्षुरान्तजिह्वं भ्रुकुटीमुखोल्बणम्}
{स्तब्धोर्ध्वकर्णं गिरिकन्दराद्भुत व्यात्तास्यनासं हनुभेदभीषणम्} %21

\twolineshloka
{दिविस्पृशत्कायमदीर्घपीवर ग्रीवोरुवक्षःस्थलमल्पमध्यमम्}
{चन्द्रांशुगौरैश्छुरितं तनूरुहैर्विष्वग्भुजानीकशतं नखायुधम्} %22

\twolineshloka
{दुरासदं सर्वनिजेतरायुध प्रवेकविद्रावितदैत्यदानवम्}
{प्रायेण मेऽयं हरिणोरुमायिना वधः स्मृतोऽनेन समुद्यतेन किम्} %23

\twolineshloka
{एवं ब्रुवंस्त्वभ्यपतद्गदायुधो नदन्नृसिंहं प्रति दैत्यकुञ्जरः}
{अलक्षितोऽग्नौ पतितः पतङ्गमो यथा नृसिंहौजसि सोऽसुरस्तदा} %24

\twolineshloka
{न तद्विचित्रं खलु सत्त्वधामनि स्वतेजसा यो नु पुरापिबत्तमः}
{ततोऽभिपद्याभ्यहनन्महासुरो रुषा नृसिंहं गदयोरुवेगया} %25

\twolineshloka
{तं विक्रमन्तं सगदं गदाधरो महोरगं तार्क्ष्यसुतो यथाग्रहीत्}
{स तस्य हस्तोत्कलितस्तदासुरो विक्रीडतो यद्वदहिर्गरुत्मतः} %26

\threelineshloka
{असाध्वमन्यन्त हृतौकसोऽमरा घनच्छदा भारत सर्वधिष्ण्यपाः}
{तं मन्यमानो निजवीर्यशङ्कितं यद्धस्तमुक्तो नृहरिं महासुरः}
{पुनस्तमासज्जत खड्गचर्मणी प्रगृह्य वेगेन गतश्रमो मृधे} % ॥२७॥\\

\twolineshloka
{तं श्येनवेगं शतचन्द्रवर्त्मभिश्चरन्तमच्छिद्रमुपर्यधो हरिः}
{कृत्वाट्टहासं खरमुत्स्वनोल्बणं निमीलिताक्षं जगृहे महाजवः} %28

\twolineshloka
{विष्वक्स्फुरन्तं ग्रहणातुरं हरिर्व्यालो यथाखुं कुलिशाक्षतत्वचम्}
{द्वार्यूरुमापत्य ददार लीलया नखैर्यथाहिं गरुडो महाविषम्} %29

\twolineshloka
{संरम्भदुष्प्रेक्ष्यकराललोचनो व्यात्ताननान्तं विलिहन्स्वजिह्वया}
{असृग्लवाक्तारुणकेशराननो यथान्त्रमाली द्विपहत्यया हरिः} %30

\twolineshloka
{नखाङ्कुरोत्पाटितहृत्सरोरुहं विसृज्य तस्यानुचरानुदायुधान्}
{अहन्समस्तान्नखशस्त्रपाणिभिर्दोर्दण्डयूथोऽनुपथान्सहस्रशः} %31

\twolineshloka
{सटावधूता जलदाः परापतन्ग्रहाश्च तद्दृष्टिविमुष्टरोचिषः}
{अम्भोधयः श्वासहता विचुक्षुभुर्निर्ह्रादभीता दिगिभा विचुक्रुशुः} %32

\twolineshloka
{द्यौस्तत्सटोत्क्षिप्तविमानसङ्कुला प्रोत्सर्पत क्ष्मा च पदाभिपीडिता}
{शैलाः समुत्पेतुरमुष्य रंहसा तत्तेजसा खं ककुभो न रेजिरे} %33

\twolineshloka
{ततः सभायामुपविष्टमुत्तमे नृपासने सम्भृततेजसं विभुम्}
{अलक्षितद्वैरथमत्यमर्षणं प्रचण्डवक्त्रं न बभाज कश्चन} %34

\twolineshloka
{निशाम्य लोकत्रयमस्तकज्वरं तमादिदैत्यं हरिणा हतं मृधे}
{प्रहर्षवेगोत्कलितानना मुहुः प्रसूनवर्षैर्ववृषुः सुरस्त्रियः} %35

\twolineshloka
{तदा विमानावलिभिर्नभस्तलं दिदृक्षतां सङ्कुलमास नाकिनाम्}
{सुरानका दुन्दुभयोऽथ जघ्निरे गन्धर्वमुख्या ननृतुर्जगुः स्त्रियः} %36

\twolineshloka
{तत्रोपव्रज्य विबुधा ब्रह्मेन्द्रगिरिशादयः}
{ऋषयः पितरः सिद्धा विद्याधरमहोरगाः} %37

\twolineshloka
{मनवः प्रजानां पतयो गन्धर्वाप्सरचारणाः}
{यक्षाः किम्पुरुषास्तात वेतालाः सहकिन्नराः} %38

\threelineshloka
{ते विष्णुपार्षदाः सर्वे सुनन्दकुमुदादयः}
{मूर्ध्नि बद्धाञ्जलिपुटा आसीनं तीव्रतेजसम्}
{ईडिरे नरशार्दुलं नातिदूरचराः पृथक्} % 39


\uvacha{श्रीब्रह्मोवाच}


\twolineshloka
{नतोऽस्म्यनन्ताय दुरन्तशक्तये विचित्रवीर्याय पवित्रकर्मणे}
{विश्वस्य सर्गस्थितिसंयमान्गुणैः स्वलीलया सन्दधतेऽव्ययात्मने} %40

\uvacha{श्रीरुद्र उवाच}


\twolineshloka
{कोपकालो युगान्तस्ते हतोऽयमसुरोऽल्पकः}
{तत्सुतं पाह्युपसृतं भक्तं ते भक्तवत्सल} %41

\uvacha{श्रीइन्द्र उवाच}


\fourlineindentedshloka
{प्रत्यानीताः परम भवता त्रायता नः स्वभागा}
{दैत्याक्रान्तं हृदयकमलं तद्गृहं प्रत्यबोधि}
{कालग्रस्तं कियदिदमहो नाथ शुश्रूषतां ते}
{मुक्तिस्तेषां न हि बहुमता नारसिंहापरैः किम्} %42

\uvacha{श्रीऋषय ऊचुः}


\fourlineindentedshloka
{त्वं नस्तपः परममात्थ यदात्मतेजो}
{येनेदमादिपुरुषात्मगतं ससर्क्थ}
{तद्विप्रलुप्तममुनाद्य शरण्यपाल}
{रक्षागृहीतवपुषा पुनरन्वमंस्थाः} %43

\uvacha{श्रीपितर ऊचुः}


\fourlineindentedshloka
{श्राद्धानि नोऽधिबुभुजे प्रसभं तनूजैर्}
{दत्तानि तीर्थसमयेऽप्यपिबत्तिलाम्बु}
{तस्योदरान्नखविदीर्णवपाद्य आर्च्छत्}
{तस्मै नमो नृहरयेऽखिलधर्मगोप्त्रे} %44

\uvacha{श्रीसिद्धा ऊचुः}


\twolineshloka
{यो नो गतिं योगसिद्धामसाधुरहार्षीद्योगतपोबलेन}
{नाना दर्पं तं नखैर्विददार तस्मै तुभ्यं प्रणताः स्मो नृसिंह} %45

\uvacha{श्रीविद्याधरा ऊचुः}


\twolineshloka
{विद्यां पृथग्धारणयानुराद्धां न्यषेधदज्ञो बलवीर्यदृप्तः}
{स येन सङ्ख्ये पशुवद्धतस्तं मायानृसिंहं प्रणताः स्म नित्यम्} %46

\uvacha{श्रीनागा ऊचुः}


\twolineshloka
{येन पापेन रत्नानि स्त्रीरत्नानि हृतानि नः}
{तद्वक्षःपाटनेनासां दत्तानन्द नमोऽस्तु ते} %47

\uvacha{श्रीमनव ऊचुः}


\twolineshloka
{मनवो वयं तव निदेशकारिणो दितिजेन देव परिभूतसेतवः}
{भवता खलः स उपसंहृतः प्रभो करवाम ते किमनुशाधि किङ्करान्} %48

\uvacha{श्रीप्रजापतय ऊचुः}


\twolineshloka
{प्रजेशा वयं ते परेशाभिसृष्टा न येन प्रजा वै सृजामो निषिद्धाः}
{स एष त्वया भिन्नवक्षा नु शेते जगन्मङ्गलं सत्त्वमूर्तेऽवतारः} %49

\uvacha{श्रीगन्धर्वा ऊचुः}


\twolineshloka
{वयं विभो ते नटनाट्यगायका येनात्मसाद्वीर्यबलौजसा कृताः}
{स एष नीतो भवता दशामिमां किमुत्पथस्थः कुशलाय कल्पते} %50

\uvacha{श्रीचारणा ऊचुः}


\twolineshloka
{हरे तवाङ्घ्रिपङ्कजं भवापवर्गमाश्रिताः}
{यदेष साधुहृच्छयस्त्वयासुरः समापितः} %51

\uvacha{श्रीयक्षा ऊचुः}


\fourlineindentedshloka
{वयमनुचरमुख्याः कर्मभिस्ते मनोज्ञैस्}
{त इह दितिसुतेन प्रापिता वाहकत्वम्}
{स तु जनपरितापं तत्कृतं जानता ते}
{नरहर उपनीतः पञ्चतां पञ्चविंश} %52

\uvacha{श्रीकिम्पुरुषा ऊचुः}


\twolineshloka
{वयं किम्पुरुषास्त्वं तु महापुरुष ईश्वरः}
{अयं कुपुरुषो नष्टो धिक्कृतः साधुभिर्यदा} %53

\uvacha{श्रीवैतालिका ऊचुः}


\twolineshloka
{सभासु सत्रेषु तवामलं यशो गीत्वा सपर्यां महतीं लभामहे}
{यस्तामनैषीद्वशमेष दुर्जनो द्विष्ट्या हतस्ते भगवन्यथामयः} %54

\uvacha{श्रीकिन्नरा ऊचुः}


\twolineshloka
{वयमीश किन्नरगणास्तवानुगा दितिजेन विष्टिममुनानुकारिताः}
{भवता हरे स वृजिनोऽवसादितो नरसिंह नाथ विभवाय नो भव} %55

\uvacha{श्रीविष्णुपार्षदा ऊचुः}

\twolineshloka
{अद्यैतद्धरिनररूपमद्भुतं ते दृष्टं नः शरणद सर्वलोकशर्म}
{सोऽयं ते विधिकर ईश विप्रशप्तस्तस्येदं निधनमनुग्रहाय विद्मः} %॥५६॥\\


॥इति श्रीमद्भागवते महापुराणे पारमहंस्यां संहितायां सप्तमस्कन्धे अष्टमोऽध्यायः॥


\input{kathas/nrisimha-jayanti/bhagavatam-07-09}
\end{center}
\markboth{उत्तराङ्ग-पूजा}{उत्तराङ्ग-पूजा}


\dnsub{अर्घ्यप्रदानम्}
ममोपात्त-समस्त-दुरित-क्षयद्वारा श्रीपरमेश्वरप्रीत्यर्थम्  नृसिंह-जयन्ती-पुण्यकाले  श्री-लक्ष्मी-नृसिंह-पूजान्ते क्षीरार्घ्यप्रदानं करिष्ये॥
\medskip

\threelineshloka*
{हिरण्याक्षवधार्थाय   भूभारोत्तरणाय   च}
{परित्राणाय   साधूनां   जातो   विष्णुर्नृकेसरी}
{गृहाणार्घ्यं   मया   दत्तं   सलक्ष्मी-नृहरे   स्वयम्}
	श्री-लक्ष्मी-नृसिंहाय नमः इदमर्घ्यमिदमर्घ्यमिदमर्घ्यम्॥\medskip

अनेन अर्घ्यप्रदानेन भगवान् सर्वात्मकः\\ श्री-लक्ष्मी-नृसिंहः प्रीयताम्।\medskip

\twolineshloka*
{हिरण्यगर्भगर्भस्थं हेमबीजं विभावसोः}
{अनन्तपुण्यफलदम् अतः शान्तिं प्रयच्छ मे}

श्री-लक्ष्मी-नृसिंह-जयन्ती-पुण्यकाले अस्मिन् मया क्रियमाण\\
महाविष्णुपूजायां यद्देयमुपायनदानं तत्प्रत्याम्नायार्थं हिरण्यं\\
श्री-लक्ष्मी-नृसिंह-प्रीतिं कामयमानः\\
मनसोद्दिष्टाय ब्राह्मणाय सम्प्रददे नमः न मम।\\ 
अनया पूजया श्री-लक्ष्मी-नृसिंहः प्रीयताम्। 
 
\dnsub{विसर्जनम्}

\twolineshloka*
{यस्य स्मृत्या च नामोक्त्या तपः-पूजा-क्रियादिषु}
{न्यूनं सम्पूर्णतां याति सद्यो वन्दे तमच्युतम्}

\twolineshloka*
{इदं व्रतं मया देव कृतं प्रीत्यै तव प्रभो}
{न्यूनं सम्पूर्णतां यातु त्वत्प्रसादाज्जनार्द्दन}

\twolineshloka*
{मद्वंशे   ये   नरा   जाता   ये   जनिष्यन्ति   चापरे}
{तांस्त्वमुद्धर   देवेश   दुःसहाद्भवसागरात्}

\twolineshloka*
{पातकार्णवमग्नस्य   व्याधिदुःखाम्बुवारिभिः}
{तीव्रैश्च  परिभूतस्य   मोहदुःखगतस्य   मे}

\twolineshloka*
{करावलम्बनं   देहि   शेषशायिन्   जगत्पते}
{श्रीनृसिंह   रमाकान्त   भक्तानां   भयनाशन}

\twolineshloka*
{क्षीराब्धिनिवासिन्   त्वं   चक्रपाणे   जनार्दन}
{व्रतेनानेन   देवेश   भुक्तिमुक्तिप्रदो   भव}

\medskip

अस्मात् बिम्बात् श्री-लक्ष्मी-नृसिंहं यथास्थानं प्रतिष्ठापयामि (अक्षतानर्पित्वा देवमुत्सर्जयेत्।)\\
अनया पूजया श्री-लक्ष्मी-नृसिंहः प्रीयताम्।\medskip

\fourlineindentedshloka*
{कायेन वाचा मनसेन्द्रियैर्वा}
{बुद्‌ध्याऽऽत्मना वा प्रकृतेः स्वभावात्}
{करोमि यद्यत् सकलं परस्मै}
{नारायणायेति समर्पयामि}


ॐ तत्सद्ब्रह्मार्पणमस्तु।\medskip

\twolineshloka* 
{सालग्रामशिलावारि पापहारि शरीरिणाम्}
{आजन्मकृतपापानां प्रायश्चित्तं दिने दिने}

\twolineshloka*
{अकालमृत्युहरणं सर्वव्याधिनिवारणम्}
{सर्वपापक्षयकरं विष्णुपादोदकं शुभम्}
 इति तीर्थं पीत्वा शिरसि प्रसादं धारयेत्।

\end{center}

\closesub

\sect{नृसिंह-जयन्ती-व्रत-कथा}

\uvacha{सूत उवाच}

\twolineshloka
{हिरण्यकशिपुं हत्वा देवदेवं जगद्गुरुम्}
{सुखासीनं च नृहरिं शान्तकोपं रमापतिम्} %॥१॥

\twolineshloka
{प्रह्लादो ज्ञानिनां श्रेष्ठः पालयन् राज्यमुत्तमम्}
{एकाकी च तदुत्सङ्गे प्रियं वचनमब्रवीत्} %॥२॥

\uvacha{प्रह्लाद उवाच}

\twolineshloka
{नमस्ते भगवन्विष्णो नृसिंहरूपिणे नमः}
{त्वद्भक्तोऽहं सुरेशैकं त्वां पृच्छामि तु तत्त्वतः} %॥३॥

\twolineshloka
{स्वामिस्त्वार्य ममाभिन्ना भक्तिर्जाता त्वनेकधा}
{कथं च ते प्रियो जातः कारणं मे वद प्रभो} %॥४॥

\uvacha{नृसिंह उवाच}

\twolineshloka
{कथयामि महाप्राज्ञ शृणुष्वैकाग्रमानसः}
{भक्तेर्यत्कारणं वत्स प्रियत्वस्य च कारणम्} %॥५॥

\twolineshloka
{पुरा काले ह्यभूद् विप्रः किञ्चित्त्वं नाप्यधीतवान्}
{नाना त्वं वासुदेवो हि वेश्यासंसक्तमानसः} %॥६॥


\twolineshloka
{तस्मिञ्जातु न चैव त्वं चकर्थ सुकृतं कियत्}
{कृतवान्मद्व्रतं चैकं वेश्यासङ्गतिलालसः} %॥७॥


{मद्व्रतस्य प्रभावेण भक्तिर्जाता तवानघ।}

\uvacha{प्रह्लाद उवाच}

\onelineshloka
{श्रीनृसिंहोच्यतां तावत्कस्य पुत्रश्च किं व्रतम्} %॥८॥

\twolineshloka
{वेश्यायां वर्तमानेन कथं तच्च कृतं मया}
{येन त्वत्प्रीतिमापन्नो वक्तुमर्हसि साम्प्रतम्} %॥९॥

\uvacha{नृसिंह उवाच}

\twolineshloka
{पुराऽवन्तीपुरे ह्यासीद्राह्मणो वेदपारगः}
{तस्य नाम सुशर्मेति बहुलोकेषु विश्रुतः} %॥१०॥

\twolineshloka
{नित्यहोमक्रियां चैव विदधाति द्विजोत्तमः}
{ब्राह्मक्रियासु नियतं सर्वासु किल तत्परः} %॥११॥

\twolineshloka
{अग्निष्टोमादिभिर्यज्ञैरिष्टाः सर्वे सुरोत्तमाः}
{तस्य भार्या सुशीलाभूद्विख्याता भुवनत्रये} %॥१२॥

\twolineshloka
{पतिव्रता सदाचारा पतिभक्तिपरायणा}
{जज्ञिरेस्या सुताः पञ्च तस्माद्द्विजवरात्तथा} %॥१३॥

\twolineshloka
{सदाचारेषु विद्वांसः पितृभक्तिपरायणाः}
{तेषां मध्ये कनिष्ठस्त्वं वेश्यासङ्गतितत्परः} %॥१४॥

\twolineshloka
{तया निषेध्यमानेन सुरापानं त्वया कृतम्}
{सुवर्ण चाप्यपहृतं चौरैः सार्ध त्वया बहु} %॥१५॥

\twolineshloka
{विलासिन्या समं चैव त्वया चीर्णमघं बहु}
{एकदा सद्गृहे चाऽऽसीन्म मन्कलिस्त्वया सह} %॥१६॥

\twolineshloka
{तेन कलहभावेन व्रतमेतत्त्वया कृतम्}
{अज्ञानान्मद्व्रतं जातं व्रतानामुत्तमं हि तत्} %॥१७॥

\twolineshloka
{तस्यां विहारयोगेन रात्रौ जागरणं कृतम्}
{वेश्याया वल्लम्भं किञ्चित्प्रजातं न त्वया सह} %॥१८॥

\twolineshloka
{रात्रौ जागरणं चीर्णं त्यक्तं भोग्यमनेकधा}
{व्रतेनानेन चीर्णेन मोदन्ति दिवि देवताः} %॥१९॥

\twolineshloka
{सृष्टयर्थे च पुरा ब्रह्मा चक्रे ह्येतदनुत्तमम्}
{मद्रतस्य प्रभावेण निर्मितं सचराचरम्} %॥२०॥

\twolineshloka
{ईश्वरेण पुरा चीर्णं वधार्थं त्रिपुरस्य च}
{माहात्म्येन व्रतस्याऽऽशु त्रिपुरस्तु निपातितः} %॥२१॥

\twolineshloka
{अन्यश्च बहुभिर्देवैर्ऋषिभिश्च पुराऽनघ}
{राजभिश्च महाप्राज्ञैर्विदितं व्रतमुत्तमम्} %॥२२॥

\twolineshloka
{एतद्वतप्रभावेण सर्वे सिद्धिमुपागताः}
{वेश्याऽपि मत्प्रिया जाता त्रैलोक्ये सुखचारिणी} %॥२३॥

\twolineshloka
{ईदृशं मद्व्रतं वत्स त्रैलोक्ये तु सुविश्रुतम्}
{कलहेन विलासिन्या व्रतमेतदुपस्थितम्} %॥२४॥

\twolineshloka
{प्रह्लाद तेन ते भक्तिर्मयि जाता ह्यनुत्तमा}
{धूर्तया च विलासिन्या ज्ञात्वा व्रतदिनं मम} %॥२५॥

\twolineshloka
{कलहश्च कृतो येन मद्व्रतं च कृतं भवेत्}
{सा वेश्या त्वप्सरा जाता भुक्त्वा भोगाननेकशः} %॥२६॥

\twolineshloka
{मुक्ता कर्मविलीना तु त्वं प्रसाद विशस्व माम्}
{कार्यार्थं च भवानास्ते मच्छरीरपृथक्तया} %॥२७॥

\twolineshloka
{विधाय सर्वकार्याणि शीघ्रं चैव गमिष्यसि}
{इदं व्रतमवश्यं ये प्रकरिष्यन्ति मानवाः} %॥२८॥

\twolineshloka
{न तेषां पुनरावृत्तिर्मत्तः कल्पशतैरपि}
{अपुत्रो लभते पुत्रान्मद्भक्तश्च सुवर्चसः} %॥२९॥

\twolineshloka
{दरिद्रो लभते लक्ष्मी धनदस्य च यादृशी}
{तेजःकामो लभत्तेजो राज्येच्छू राज्यमुत्तमम्} %॥३०॥

\twolineshloka
{आयुःकामो लभेदायुर्यादृशं च शिवस्य हि}
{स्त्रीणां व्रतमिदं साधुपुत्रदं भाग्यदं तथा} %॥३१॥

\twolineshloka
{अवैधव्यकरं तासां पुत्रशोकविनाशनम्}
{धनधान्यकरं चैव जातिश्रैष्ठ्यकरं शुभम्} %॥३२॥

\twolineshloka
{सार्वभौमसुखं तासां दिव्यं सौख्यं भवेत्ततः}
{स्त्रियो वा पुरुषाश्चापि कुर्वन्ति व्रतमुत्तमम्} %॥३३॥

\twolineshloka
{तेभ्योऽहं प्रददे सौख्यं भुक्तिमुक्ति-समन्वितम्}
{बहुनोक्तेन किं वत्स व्रतस्यास्य फलं महत्} %॥३४॥

\twolineshloka
{मद्व्रतस्य फलं वक्तुं नाहं शक्तो न शङ्करः}
{ब्रह्मा चतुर्भिर्वक्त्रैश्च न लभेन्महिमावधिम्} %॥३५॥

\uvacha{प्रह्लाद उवाच}
\twolineshloka
{भगवंस्त्वत्प्रसादेन श्रुतं व्रतमनुत्तमम्}
{व्रतस्यास्य फलं साधु त्वयि मे भक्तिकारणम्} %॥३६॥

\twolineshloka
{स्वामिञ्जातं विशेषण त्वत्तः पापनिकृन्तनम्}
{अधुना श्रोतुमिच्छामि व्रतस्यास्य विधिं परम्} %॥३७॥

\twolineshloka
{कस्मिन्मासे भवेदेतत्कस्मिन्वा तिथिवासरे}
{एतद्विस्तरतो देव वक्तुमर्हसि साम्प्रतम्} %॥३८॥

\twolineshloka
{विधिना येन वै स्वामिन् समग्रफलभुग्भवेत्}
{ममोपरि कृपां कृत्वा ब्रूहि त्वं सकलं प्रभो} %॥३९॥

\uvacha{नृसिंह उवाच}

\twolineshloka
{साधु साधु महाभाग व्रतस्यास्य विधिं परम्}
{सर्वं कथयतो मेऽद्य त्वमेकाग्रमनाः शृण} %॥४०॥

\twolineshloka
{वैशाखशुक्लपक्षे तु चतुर्दश्यां समाचरेत्}
{मज्जन्मसम्भवं पुण्यं व्रतं पापप्रणाशनम्} %॥४१॥


\twolineshloka
{वर्षे वर्षे तु कर्तव्यं मम सन्तुष्टिकारकम्}
{महापुण्यमिदं श्रेष्ठं मानुषैर्भवभीरुभिः} %॥४२॥

\twolineshloka
{तेनैव क्रियमाणेन सहस्रद्वादशीफलम्}
{जायते तद्व्रते वच्मि मानुषाणां महात्मनाम्} %॥४३॥

\twolineshloka
{स्वाती नक्षत्रयोगेन शनिवारेण संयुते}
{सिद्धियोगस्य संयोगे वणिजे करणे तथा} %॥४४॥

\twolineshloka
{पुण्यसौभाग्ययोगेन लभ्यते दैवयोगतः}
{सर्वैरेतैस्तु संयुक्तं हत्याकोटिविनाशनम्} %॥४५॥

\twolineshloka
{एतदन्यतरे योगे तद्दिनं पापनाशनम्}
{केवलेऽपि च कर्तव्यं मद्दिने व्रतमुत्तमम्} %॥४६॥

\twolineshloka
{अन्यथा नरकं याति यावच्चन्द्रदिवाकरौ}
{यथा यथा प्रवृत्तिः स्यात्पातकस्य कलौ युगे} %॥४७॥

\twolineshloka
{तथा तथा प्रणश्यन्ति सर्वे धर्मा न संशयः}
{एतद्व्रतप्रभावेण मद्भक्तिः स्याद्दुरात्मनाम्} %॥४८॥


\twolineshloka
{विचार्येत्थं प्रकर्तव्यं माधवे मासि तद्व्रतम्}
{नियमश्च प्रकर्तव्यो दन्तधावनपूर्वकम्} %॥४९॥


\twolineshloka
{श्रीनृसिंह महोग्रस्त्वं दयां कृत्वा ममोपरि}
{अद्याहं ते विधास्यामि व्रतं निर्विघ्नतां नय} %॥५०॥

इति नियममन्त्रः।

\twolineshloka
{व्रतस्थेन न कर्तव्या सङ्गतिः पापिभिः सह}
{मिथ्यालापो न कर्तव्यः समग्रफलकाङ्क्षिणा} %॥५१॥

\twolineshloka
{स्त्रीभिर्दुष्टैश्च आलापान्व्रतस्थो नैव कारयेत्}
{स्मर्तव्यं च महारूपं मद्दिने सकलं शुभे} %॥५२॥

\twolineshloka
{ततो मध्याह्नवेलायां नद्यादौ विमले जले}
{गृहे वा देवखाते वा तडागे विमले शुभे} %॥५३॥

\twolineshloka
{वैदिकेन च मन्त्रेण स्नानं कृत्वा विचक्षणः}
{मृत्तिकागोमयेनैव तथा धात्रीफलेन च} %॥५४॥

\twolineshloka
{तिलैश्च सर्वपापन्नः स्नानं कृत्वा महात्मभिः}
{परिधाय शुचिर्वासो नित्यकर्म समाचरेत्} %॥५५॥

\twolineshloka
{ततो गृहं समागत्य स्मरन् मां भक्तियोगतः}
{गोमयेन प्रलिप्याथ कुर्यादष्टदलं शुभम्} %॥५६॥

\twolineshloka
{कलशं तत्र संस्थाप्य ताम्रं रत्नसमन्वितम्}
{तस्योपरि न्यसेत् पात्रं वंशजं व्रीहिपूरितम्} %॥५७॥

\twolineshloka
{हैमी तत्र च मन्मूर्तिः स्थाप्या लक्षम्यास्तथैव च}
{पलेन वा तदर्धेन तदर्धार्धेन वा पुनः} %॥५८॥

\twolineshloka
{यथाशक्त्याऽथवा कार्या वित्तशाठ्यविवर्जितैः}
{पञ्चामृतेन संस्नाप्य पूजनं तु समाचरेत्} %॥५९॥

\twolineshloka
{ततो ब्राह्मणमाहूय तमाचार्यमलोलुपम्}
{सदाचारसमायुक्तं शान्तं दान्तं जितेन्द्रियम्} %॥६०॥

\twolineshloka
{आचार्यवचनाद्धीमान् पूजां कुर्याद्यथाविधि}
{मण्डपं कारयेत्तत्र पुष्पस्तबकशोभितम्} %॥६१॥

\threelineshloka
{ऋतुकालोद्भवैः पुष्पैः पूजयेत्स्वस्थमानसः}
{उपचारः षोडेशभिमन्त्रैर्वेदोद्भवैस्तथा} %॥६२॥
{शुभैः पौराणिकैर्मन्त्रैः पूजनीयो यथाविधि}

\twolineshloka
{चन्दनं शीतलं दिव्यं चन्द्रकुङ्कुममिश्रितम्}
{ददामि तव तुष्ट्यर्थं नृसिंह परमेश्वर} %॥६३॥

इति चन्दनम्।

\twolineshloka
{कालोद्भवानि पुष्पाणि तुलस्यादीनि वै प्रभो}
{सम्यक् गृहाण देवेश लक्ष्म्या सह नमोऽस्तु ते} %॥६४॥

इति पुष्पाणि।

\twolineshloka
{कृष्णागुरुमयं धूपं श्रीनृसिंह जगत्पते}
{तव तुष्ट्यै प्रदास्यामि सर्वदेव नमोऽस्तु ते} %॥६५॥

इति धूपम्।

\twolineshloka
{सर्वतेजोद्भवं तेजस्तस्माद्दीपं ददामि ते}
{श्रीनृसिंह महाबाहो तिमिरं मे विनाशय} %॥६६॥

इति दीपम्।

\twolineshloka
{नैवेद्यं सौख्यदं चारु भक्ष्यभोज्यसमन्वितम्}
{ददामि ते रमाकान्त सर्वपापक्षयं कुरु} %॥६७॥

इति नैवेद्यम्।

\twolineshloka
{नृसिंहाच्युत देवेश लक्ष्मीकान्त जगत्पते}
{अनेनार्घ्यप्रदानेन सफलाः स्युर्मनोरथाः} %॥६८॥

इति अर्घ्यम्।

\twolineshloka
{पीताम्बर महाबाहो प्रह्लादभयनाशन}
{यथाभूतेनार्चनेन यथोक्तफलदो भव} %॥६९॥

इति प्रार्थना॥
\twolineshloka
{रात्रौ जागरणं कार्यं गीतवादित्रनिःस्वनैः}
{पुराणश्रवणाद्यैश्च श्रोतव्याश्च कथाः शुभाः} %॥७०॥

\twolineshloka
{ततः प्रभातसमये स्नानं कृत्वा जितेन्द्रियः}
{पूर्वोक्तेन विधानेन पूजयेन्मां प्रयत्नतः} %॥७१॥

\twolineshloka
{वैष्णवान्प्रजपेन्मन्त्रान् मदग्रे स्वस्थमानसः}
{ततो दानानि देयानि वक्ष्यमाणानि चानघ} %॥७२॥

\twolineshloka
{पात्रेभ्यस्तु द्विजेभ्यो हि लोकद्वयजिगीषया}
{सिंहः स्वर्णमयो देयो मम सन्तोषकारकः} %॥७३॥

\twolineshloka
{गोभूतिलहिरण्यानि दयानि च फलेप्सुभिः}
{शय्या सतूलिका देया सप्तधान्यसमन्विता} %॥७४॥

\twolineshloka
{अन्यानि च यथाशक्त्या देयानि मम तुष्टये}
{वित्तशाठ्यं न कुर्वीत यथोक्तफलकाङ्क्षया} %॥७५॥

\twolineshloka
{ब्राह्मणान् भोजयेद्भक्त्या तेभ्यो दद्याञ्च दक्षिणाम्}
{निर्धनेनापि कर्तव्यं देयं शक्त्यनुसारतः} %॥७६॥


\twolineshloka
{सर्वेषामेव वर्णानामधिकारोऽस्ति मद्व्रते}
{मद्भक्तैस्तु विशेषेण कर्तव्यं मत्परायणैः} %॥७७॥


\twolineshloka
{तद्वंशे न भवेद्दुःखं न दोषो मत्प्रसादतः}
{मद्वंशे ये नरा जाता ये निष्पत्तिपरायणाः} %॥७८॥


\twolineshloka
{तान् समुद्धर देवेश दुस्तराद्भवसागरात्}
{पातकार्णवमग्नस्य व्याधिदुःखाम्बुवासिभिः} %॥७९॥


\twolineshloka
{जीवैस्तु परिभूतस्य मोहदुःखगतस्य मे}
{करावलम्बनं देहि शेषशायिञ्जगत्पते} %॥८०॥


\twolineshloka
{श्रीनृसिंह रमाकान्त भक्तानां भयनाशन}
{क्षीराम्बुनिधिवासिंस्त्वं चक्रपाणे जनार्दन} %॥८१॥


\twolineshloka
{व्रतेनानेन देवेश भुक्तिमुक्तिप्रदो भव}
{एवं प्रार्थ्य ततो देवं विसृज्य च यथाविधि} %॥८२॥

\threelineshloka
{उपहारादिकं सर्वमाचार्याय निवेदयेत्}
{दक्षिणाभिस्तु सन्तोष्य ब्राह्मणांस्तु विसर्जयेत्} %॥८३॥
{मध्याह्ने तु सुसंयत्तो भुञ्जीत सह बन्धुभिः॥}

\twolineshloka
{य इदं शृणुयाद्भक्त्या व्रतं पापप्रणाशनम्}
{तस्य श्रवणमात्रेण ब्रह्महत्या व्यपोहति} %॥८४॥

\twolineshloka
{पवित्रं परमं गुह्यं कीर्तयेद्यस्तु मानवः}
{सर्वान् कामानवाप्नोति व्रतस्यास्य फलं लभेत्}

इति हेमाद्रौ नृसिंहपुराणे नृसिंहचतुर्दशीव्रतकथा समाप्ता॥


\closesection
