% !TeX program = XeLaTeX
% !TeX root = pUjA.tex
\chapt{पञ्चाङ्ग-पूजा}

\input{purvanga/vighneshwara-puja}

\sect{प्रधान-पूजा — पञ्चाङ्ग-पूजा}

\twolineshloka*
{शुक्लाम्बरधरं विष्णुं शशिवर्णं चतुर्भुजम्}
{प्रसन्नवदनं ध्यायेत् सर्वविघ्नोपशान्तये}
 
प्राणान् आयम्य। ॐ भूः + भूर्भुवः॒ सुव॒रोम्।

\dnsub{सङ्कल्पः}

ममोपात्तसमस्तदुरितक्षयद्वारा श्रीपरमेश्वरप्रीत्यर्थं शुभे शोभने मुहूर्ते अद्य ब्रह्मणः
द्वितीयपरार्धे श्वेतवराहकल्पे वैवस्वतमन्वन्तरे अष्टाविंशतितमे कलियुगे प्रथमे पादे
जम्बूद्वीपे भारतवर्षे भरत\-खण्डे मेरोः दक्षिणे पार्श्वे अस्मिन् वर्तमाने व्यावहारिकाणां 
प्रभवादि-षष्टिसंवत्सराणां मध्ये \textbf{(~~~)}\see{app:samvatsara_names}-नाम-संवत्सरे \textbf{उत्तरायणे} \textbf{वसन्त}-ऋतौ \\

\textbf{चान्द्रमान-संवत्सरारम्भः}

\textbf{चैत्र}-मासे 	\textbf{शुक्ल}-पक्षे	\textbf{प्रथमायां}  शुभतिथौ	\textbf{(~~~)}\-वासर\-युक्तायां	\textbf{(~~~)}-नक्षत्र\-युक्तायां	\textbf{(~~~)}-योग\-युक्तायां	\textbf{बव}-करण (१२:४९; \textbf{बालव}-करण)\-युक्तायाम्	एवं-गुण-विशेषण-विशिष्टायाम् 	अस्यां \textbf{(~~~)} \\


\textbf{सौरमान-संवत्सरारम्भः}

\textbf{मेष}-\textbf{चैत्र}-मासे 	\textbf{(~~~)}-पक्षे	\textbf{प्रथमायां} शुभतिथौ	\textbf{(~~~)}\-वासर\-युक्तायां	\textbf{(~~~)}-नक्षत्र\-युक्तायां	\textbf{(~~~)}-योग\-युक्तायां	\textbf{(~~~)}-करण\-युक्तायाम्	एवं-गुण-विशेषण-विशिष्टायाम् 	अस्यां \textbf{(~~~)}\\


\noindent अस्माकं सहकुटुम्बानां क्षेमस्थैर्य-धैर्य-वीर्य-विजय-आयुर्-आरोग्य-ऐश्वर्याभि\-वृद्ध्यर्थं
धर्मार्थकाममोक्ष-चतुर्विध\-फल\-पुरुषार्थसिद्ध्यर्थं सत्सन्तानाभि\-वृद्ध्यर्थम् इष्टकाम्यार्थ\-सिद्ध्यर्थम्\\
महागणपत्यादि-त्रयस्त्रिंशत्-कोटि-देवता-प्रीत्यर्थं 
संवत्सर-अयन-ऋतु-मास-पक्ष-तिथि-वार-\\
नक्षत्र-योग-करण-देवता-प्रीत्यर्थम्
आदित्यादि-नवग्रह-देवता-प्रीत्यर्थं च यावच्छक्ति-ध्यान-आवाहनादि-षोडशोपचारपूजां करिष्ये। तदङ्गं कलशपूजां च करिष्ये।

श्रीविघ्नेश्वराय नमः यथास्थानं प्रतिष्ठापयामि।
(गणपति-प्रसादं शिरसा गृहीत्वा)

\input{purvanga/aasana-puja}

\input{purvanga/ghanta-puja}

\input{purvanga/kalasha-puja}

\input{purvanga/aatma-puja}

\input{purvanga/pitha-puja}

\input{purvanga/guru-dhyanam}


ध्यान-श्लोकः\see{app:samvatsara_names}


\newcommand{\devataaM}{(~~~)}
\newcommand{\devataam}{(~~~)}
\newcommand{\devataayai}{(~~~)}
\newcommand{\devataadi}{(~~~)}
\newcommand{\yname}{(~~~)}


अस्मिन् पञ्चाङ्ग-पुस्तके \yname-संवत्सर-देवतां \devataaM{} ध्यायामि। \devataam{} आवाहयामि।

उत्तरायण-देवतां ध्यायामि। उत्तरायण-देवताम् आवाहयामि।

दक्षिणायन-देवतां ध्यायामि। दक्षिणायन-देवताम् आवाहयामि।

वसन्तादि-ऋतु-देवताः ध्यायामि। वसन्तादि-ऋतु-देवताः आवाहयामि।

मेषादि-मास-देवताः ध्यायामि। मेषादि-मास-देवताः आवाहयामि।

शुक्ल-पक्ष-देवताम् इन्दुं ध्यायामि। शुक्ल-पक्ष-देवताम् इन्दुम् आवाहयामि।

कृष्ण-पक्ष-देवतां सूर्यं ध्यायामि। कृष्ण-पक्ष-देवतां सूर्यम् आवाहयामि।

तिथि-देवताः ध्यायामि। तिथि-देवताः आवाहयामि।

वार-देवताः ध्यायामि। वार-देवताः आवाहयामि।

कृत्तिकादि-अष्टाविंशति-नक्षत्र-देवताः ध्यायामि। कृत्तिकादि-अष्टाविंशति-नक्षत्र-देवताः आवाहयामि।

विष्कम्भादि-योग-देवताः ध्यायामि। विष्कम्भादि-योग-देवताः आवाहयामि।

बवादि-करण-देवताः ध्यायामि। बवादि-करण-देवताः आवाहयामि।

\twolineshloka*
{आदित्याय च सोमाय मङ्गलाय बुधाय च}
{गुरु-शुक्र-शनिभ्यश्च राहवे केतवे नमः}

आदित्यादि-नवग्रह-देवताः ध्यायामि। आदित्यादि-नवग्रह-देवताः आवाहयामि।\\
\devataadi-समस्त-देवताभ्यो नमः, आसनं समर्पयामि।\\
\devataadi-समस्त-देवताभ्यो नमः, पाद्यं समर्पयामि।\\
\devataadi-समस्त-देवताभ्यो नमः, अर्घ्यं समर्पयामि।\\
\devataadi-समस्त-देवताभ्यो नमः, आचमनीयं समर्पयामि।\\
\devataadi-समस्त-देवताभ्यो नमः, शुद्धोदकस्नानं समर्पयामि।\\
\devataadi-समस्त-देवताभ्यो नमः, स्नानानन्तरम् आचमनीयं समर्पयामि।\\
\devataadi-समस्त-देवताभ्यो नमः, वस्त्रार्थम् अक्षतान् समर्पयामि।\\
\devataadi-समस्त-देवताभ्यो नमः, यज्ञोपवीताभरणार्थे अक्षतान् समर्पयामि।\\
\devataadi-समस्त-देवताभ्यो नमः, दिव्यपरिमलगन्धान् धारयामि।\\ गन्धस्योपरि हरिद्राकुङ्कुमं समर्पयामि।\\
\devataadi-समस्त-देवताभ्यो नमः, अक्षतान् समर्पयामि। पुष्पैः पूजयामि।\\

\begin{multicols}{2}
\setlength\columnsep{0pt} 
\begin{enumerate}
\item \yname-संवत्सर-देवतायै \devataayai{}~नमः
\item उत्तरायण-देवतायै नमः 
\item दक्षिणायन-देवतायै नमः
\item वसन्तादि-ऋतु-देवताभ्यो नमः 
\item मेषादि-मास-देवताभ्यो नमः 
\item शुक्ल-पक्ष-देवतायै इन्दवे नमः
\item कृष्ण-पक्ष-देवतायै सूर्याय नमः 
\item तिथि-देवताभ्यो नमः 
\item वार-देवताभ्यो नमः
\item कृत्तिकादि-अष्टाविंशति-नक्षत्र-देवताभ्यो नमः 
\item विष्कम्भादि-योग-देवताभ्यो नमः 
\item बवादि-करण-देवताभ्यो नमः 
\item आदित्याय नमः
\item सोमाय नमः
\item मङ्गलाय नमः
\item बुधाय नमः
\item गुरवे नमः
\item शुक्राय नमः
\item शनैश्चराय नमः
\item राहवे नमः
\item केतवे नमः
\end{enumerate}
\end{multicols}

\devataadi-समस्त-देवताभ्यो नमः, नानाविध-परिमल-पत्र-पुष्पाणि समर्पयामि।
\devataadi-समस्त-देवताभ्यो नमः, धूपमाघ्रापयामि।\\
\devataadi-समस्त-देवताभ्यो नमः, दीपं दर्शयामि।\\
नैवेद्यम्। \devataadi-समस्त-देवताभ्यो नमः, निम्बकुसुमभक्षणं कदलीफलं च निवेदयामि। आचमनीयं समर्पयामि।\\
\devataadi-समस्त-देवताभ्यो नमः, कर्पूरताम्बूलं समर्पयामि। \devataadi-समस्त-देवताभ्यो नमः, कर्पूरनीराजनं दर्शयामि।\\
\devataadi-समस्त-देवताभ्यो नमः, समस्तोपचारान् समर्पयामि।\\


\devataadi-समस्त-देवताभ्यो नमः, प्रदक्षिणनमस्कारान् समर्पयामि।\\

प्रार्थनाः समर्पयामि।

\fourlineindentedshloka*
{कायेन वाचा मनसेन्द्रियैर्वा}
{बुद्‌ध्याऽऽत्मना वा प्रकृतेः स्वभावात्}
{करोमि यद्यत् सकलं परस्मै}
{नारायणायेति समर्पयामि}

अनेन पूजनेन अग्न्यादयः प्रीयन्ताम्। \\

ॐ तत्सद्ब्रह्मार्पणमस्तु।

\sect{लघु-पञ्चाङ्ग-पठन-पद्धतिः}

\twolineshloka*
{वागीशाद्याः सुमनसः सर्वार्थानामुपक्रमे}
{यं नत्वा कृतकृत्याः स्युस्तं नमामि गजाननम्}

\twolineshloka*
{तिथेश्च श्रियमाप्नोति वारादायुष्यवर्धनम्}
{नक्षत्राद्धरते पापं योगाद्रोगनिवारणम्}

\twolineshloka*
{करणात् कार्यसिद्धिः स्यात् पञ्चाङ्गफलमुत्तमम्}
{एतेषां श्रवणान्नित्यं गङ्गास्नानफलं भवेत्}

{\sffamily During the Panchanga reading, the following should be read:
The year’s Navanayaka phalam,
Makara Sankranti phalam,
Kandhaaya phalam,
Rashi phalam,
Grahanas, and
details on the Udaya and Astamana of the Grahas.
Then, the Chithirai 1st date Panchanga should be read twice.

\hspace{2ex} These details will be given in every Panchanga in different places. One should note them down the previous day and read them after the Panchanga Puja, while others listen.}


\twolineshloka*
{मङ्गलं कोसलेन्द्राय महनीयगुणात्मने}
{चक्रवर्तितनूजाय सार्वभौमाय मङ्गलम्}

\closesection 


\sect{निम्बकुसुमभक्षणम्}
\centerline{\textsf{Shloka for partaking the Nimba Kusuma Bhakshanam}}

\twolineshloka*
{शतायुर्वज्रदेहाय सर्वसम्पत्कराय च}
{सर्वानिष्टविनाशाय निम्बकन्दलभक्षणम्}


\sect{नवग्रहस्तोत्रम्}

\twolineshloka
{जपाकुसुमसङ्काशं काश्यपेयं महाद्युतिम्}
{तमोरिं सर्वपापघ्नं प्रणतोऽस्मि दिवाकरम्}
\hfill (सूर्यः)

\twolineshloka
{दधिशङ्खतुषाराभं क्षीरोदार्णवसम्भवम्}
{नमामि शशिनं सोमं शम्भोर्मुकुटभूषणम्}
\hfill (चन्द्रः)

\twolineshloka
{धरणीगर्भसम्भूतं विद्युत्कान्तिसमप्रभम्}
{कुमारं शक्तिहस्तं तं मङ्गलं प्रणमाम्यहम्}
\hfill (अङ्गारकः)

\twolineshloka
{प्रियङ्गुकलिकाश्यामं रूपेणाप्रतिमं बुधम्}
{सौम्यं सौम्यगुणोपेतं तं बुधं प्रणमाम्यहम्}
\hfill (बुधः)

\twolineshloka
{देवानां च ऋषीणां च गुरुं काञ्चनसन्निभम्}
{बुद्धिभूतं त्रिलोकेशं तं नमामि बृहस्पतिम्}

\hfill (बृहस्पतिः)

\twolineshloka
{हिमकुन्दमृणालाभं दैत्यानां परमं गुरुम्}
{सर्वशास्त्रप्रवक्तारं भार्गवं प्रणमाम्यहम्}
\hfill (शुक्रः)

\twolineshloka
{नीलाञ्जनसमाभासं रविपुत्रं यमाग्रजम्}
{छायामार्तण्डसम्भूतं तं नमामि शनैश्चरम्}

\hfill (शनैश्चरः)

\twolineshloka
{अर्धकायं महावीर्यं चन्द्रादित्यविमर्दनम्}
{सिंहिकागर्भसम्भूतं तं राहुं प्रणमाम्यहम्}
\hfill (राहुः)

\twolineshloka
{पलाशपुष्पसङ्काशं तारकाग्रहमस्तकम्}
{रौद्रं रौद्रात्मकं घोरं तं केतुं प्रणमाम्यहम्}
\hfill (केतुः)

\twolineshloka
{इति व्यासमुखोद्गीतं यः पठेत् सुसमाहितः}
{दिवा वा यदि वा रात्रौ विघ्नशान्तिर्भविष्यति}

\twolineshloka
{नरनारीनृपाणां च भवेद्दुःस्वप्ननाशनम्}
{ऐश्वर्यमतुलं तेषामारोग्यं पुष्टिवर्धनम्}

\twolineshloka
{ग्रहनक्षत्रजाः पीडास्तस्कराग्निसमुद्भवाः}
{ताः सर्वाः प्रशमं यान्ति व्यासो ब्रूते न संशयः}

\centerline{॥इति श्रीव्यासविरचितं नवग्रहस्तोत्रं सम्पूर्णम्॥}

\sect{प्रार्थना-मन्त्राः}

\twolineshloka
{ॐ नमो ब्रह्मणे तुभ्यं कामाय च महात्मने}
{नमस्तेऽस्तु निमेषाय त्रुटये च नमोऽस्तु ते}

\twolineshloka
{लवाय च नमस्तुभ्यं नमस्तेऽस्तु क्षणाय च}
{नमो नमस्ते काष्ठायै कलायै ते नमोऽस्तु ते}

\twolineshloka
{नाडिकायै सुसूक्ष्मायै मुहूर्ताय नमो नमः}
{नमो निशाभ्यः पुण्येभ्यो दिवसेभ्यश्च नित्यशः}

\twolineshloka
{पक्षाभ्यां चाथ मासेभ्य ऋतुभ्यः षड्भ्य एव च}
{अयनाभ्यां च पञ्चभ्यो वत्सरेभ्यश्च सर्वदा}

\twolineshloka
{नमः कृतयुगादिभ्यो ग्रहेभ्यश्च नमो नमः}
{अष्टाविंशतिसङ्ख्येभ्यो नक्षत्रेभ्यो नमो नमः}

\twolineshloka
{राशिभ्यः करणेभ्यश्च व्यतीपातेभ्य एव च}
{प्रतिवर्षाधिपेभ्यश्च विज्ञातेभ्यो नमः सदा}

\twolineshloka
{नमोऽस्तु कुलनागेभ्यः सानुयात्रेभ्य\footnotemark{} एव च}
{नमोऽस्तु सर्वदिग्भ्यश्च दिक्पालेभ्यो नमो नमः}
\footnotetext{सानुचरेभ्यः}

\twolineshloka
{नमश्चतुर्दशभ्यश्च मनुभ्यस्तु नमो नमः}
{नमः पुरन्दरेभ्यश्च तत्सङ्ख्येभ्यो नमो नमः}

\twolineshloka
{पञ्चाशते नमो नित्यं दक्षकन्याभ्य एव च}
{नमोऽदित्यै सुभद्रायै जयायै चाथ सर्वदा}

\twolineshloka
{सुशास्त्राय नमस्तुभ्यं सर्वास्त्रजनकाय च}
{नमस्ते बहुपुत्राय पत्नीभिः सहिताय च}

\twolineshloka
{नमो बुद्ध तथा वृद्धयै निद्रायै धनदाय च}
{नमः कुबेरपुत्राय\footnotemark{} गुह्यकस्वामिने नमः}
\footnotetext{नलकूबरयक्षाय}

\twolineshloka
{नमोऽस्तु शङ्खपद्माभ्यां निदिभ्यामथ नित्यशः}
{भद्रकाल्यै नमस्तुभ्यं सुरभ्यै च नमो नमः}

\twolineshloka
{वेदवेदाङ्गवेदान्तविद्यासंस्थेभ्य एव च}
{नागयक्षसुपर्णेभ्यो नमोऽस्तु गरुडाय च}

\twolineshloka
{सप्तभ्यश्च समुद्रेभ्यः सागरेभ्यश्च सर्वदा}
{उत्तरेभ्यः कुरुभ्यश्च नमो मेरुगताय\footnotemark{} च}
\footnotetext{हैरण्यताय}

\twolineshloka
{भद्राश्वकेतुमालाभ्यां नमः सर्वत्र सर्वदा}
{इलावृत्ता (त) य च नमो हरिवर्षाय चैव हि}

\twolineshloka
{नमः किंपुरुषेभ्यश्च भारताय नमो नमः}
{नमोभारतभेदेभ्यो\footnotemark{}  महद्भ्यश्चाथ सर्वदा}
\footnotetext{नवभ्य इति च पाठः}

\twolineshloka
{पातालेभ्यश्च सप्तभ्यो नरकेभ्यो नमो नमः}
{कालाग्निरुद्रशैवाभ्यां हरये क्रोडरूपिणे}

\twolineshloka
{सप्तभ्यस्त्वथ लोकेभ्यो महाभूतेभ्य एव च}
{नमः सम्बुद्धये चैव नमः प्रकृतये तथा}

\twolineshloka
{पुरुषायाभिमानाय नमस्त्वक्तमूर्तये}
{पौराणीभ्यश्च गङ्गाभ्यः सप्तभ्यश्च नमो नमः}

\twolineshloka
{नमोस्त्वादि मुनिभ्यश्च सप्तभ्यश्चाथ सर्वदा}
{नमोऽस्तु पुष्करादिभ्यस्तीर्थेभ्यश्च पुनः पुनः}

\twolineshloka
{निम्नगाभ्यो नमो नित्यं वितस्तादिभ्य एव च}
{चतुर्दशभ्यो दीर्घाभ्यो धरणीभ्यो नमो नमः}

\twolineshloka
{नमो धात्रेविधात्रे च च्छन्दोभ्यश्च नमो नमः}
{सुरभ्यैरावणाभ्यां च नमो भूयो नमो नमः}

\twolineshloka
{नमस्तथोच्चैःश्रवसे ध्रुवाय च नमो नमः}
{नमोऽस्तु धन्वन्तराये शस्त्रास्त्राभ्यां नमो नमः}

\twolineshloka
{विनायककुमाराभ्यां विघ्नेभ्यश्च नमः सदा}
{शाखाय च विशाखाय नैगमेयाय वै नमः}

\twolineshloka
{नमः स्कन्दग्रहेभ्यश्चस्कन्दमातृभ्य एव च}
{ज्वराय रोगपतये भस्मप्रहरणाय च}

\twolineshloka
{ऋषिभ्यो वालखिल्येभ्यः केशवाय नमः सदा}
{अगस्तये नारदाय व्यासादिभ्यो नमो नमः}

\twolineshloka
{अप्सरोभ्यः सोमपेभ्यो देवेभ्यश्च नमो नमः}
{असोमपेभ्यश्च नमस्तुषितेभ्यो नमः सदा}

\twolineshloka
{आदित्येभ्यो नमो नित्यं द्वादशभ्यश्च सर्वदा}
{एकादशभ्यो रुद्रेभ्यस्तपस्विभ्यो नमो नमः}

\twolineshloka
{नमो नासत्यदस्रायामश्विभ्यां नित्यमेव हि}
{साध्येभ्यो द्वादशभ्यश्च पौराणेभ्यो नमः सदा}

\twolineshloka
{एकोनपञ्चशते च मरुद्भयश्च नमो नमः}
{शिल्पाचार्याय देवाय नमस्ते विश्वकर्मणे}

\twolineshloka
{अष्टभ्यो लोकपालेभ्यः सानुगेभ्यश्च सर्वदा}
{आयुधेभ्यो वाहनेभ्यो वर्मभ्यश्च नमः सदा}

\twolineshloka
{आसनेभ्यो दुन्दुभिभ्यो देवेभ्यश्च नमः सदा}
{दैत्यराक्षसगन्धर्वपिशाचेभ्यश्च नित्यशः}

\twolineshloka
{पितृभ्यः सप्तभेदेभ्यः प्रेतेभ्यश्च नमः सदा}
{सुसूक्ष्मेभ्यश्च देवेभ्यो भावगम्येभ्य एव च}

\threelineshloka
{नमस्ते बहुरूपाय विष्णवे परमात्मने}
{अथ किं बहुनोक्तेन मन्त्रेणानेन वाऽर्चयेत्}
{प्राङ्मुखोदङ्मुखो विप्रान् देवानुद्दिश्य पूर्ववत्}

\closesection