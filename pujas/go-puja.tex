% !TeX program = XeLaTeX
% !TeX root = pUjA.tex
\chapt{गो-पूजा}

\dnsub{पूर्वाङ्ग-विघ्नेश्वर-पूजा}

\graphicspath{{purvanga/}{../purvanga/}}

\centerline{\includegraphics[width=1cm]{ganesha.pdf}}

(आचम्य)

\twolineshloka*
{शुक्लाम्बरधरं विष्णुं शशिवर्णं चतुर्भुजम्}
{प्रसन्नवदनं ध्यायेत् सर्वविघ्नोपशान्तये}
 
प्राणान्  आयम्य।

(अप उपस्पृश्य, पुष्पाक्षतान् गृहीत्वा)\\

\twolineshloka*
{तदेव लग्नं सुदिनं तदेव ताराबलं चन्द्रबलं तदेव}
{विद्याबलं दैवबलं तदेव लक्ष्मीपतेरङ्घ्रियुगं स्मरामि}
 
ममोपात्त-समस्त-दुरित-क्षयद्वारा \\
श्री-परमेश्वर-प्रीत्यर्थं करिष्यमाणस्य कर्मणः\\
अविघ्नेन परिसमाप्त्यर्थम् आदौ विघ्नेश्वरपूजां करिष्ये।

(अप उपस्पृश्य)

\ifbool{veda}{
\twolineshloka*
{ॐ ग॒णानां᳚ त्वा ग॒णप॑तिꣳ हवामहे क॒विं क॑वी॒नामु॑प॒मश्र॑वस्तमम्}
{ज्ये॒ष्ठ॒राजं॒ ब्रह्म॑णां ब्रह्मणस्पत॒ आ नः॑ शृ॒ण्वन्नू॒तिभिः॑ सीद॒ साद॑नम्}
}{}

\twolineshloka*
{अगजानानपद्मार्कं गजाननमहर्निशम्}
{अनेकदं तं भक्तानाम् एकदन्तमुपास्महे}

\ifbool{veda}{भूर्भुवः॒ सुव॒रोम्।}{} अस्मिन् हरिद्राबिम्बे सुमुखं महागणपतिं ध्यायामि, आवाहयामि।\\

\renewcommand{\devAya}{\OMshri महागणपतये}

\devAya{} नमः, आसनं समर्पयामि।\\
\devAya{} नमः, पादयोः पाद्यं समर्पयामि।\\
\devAya{} नमः, अर्घ्यं समर्पयामि।\\
\devAya{} नमः, आचमनीयं समर्पयामि।\\
\devAya{} नमः, मधुपर्कं समर्पयामि।\\
\ifbool{veda}{ॐ भूर्भुवः॒ सुवः॑।}{}
\devAya{} नमः, शुद्धोदकस्नानं समर्पयामि। स्नानानन्तरमाचमनीयं समर्पयामि।\\
\devAya{} नमः, वस्त्रार्थमक्षतान् समर्पयामि।\\
\devAya{} नमः, यज्ञोपवीताभरणार्थे अक्षतान् समर्पयामि।\\
\devAya{} नमः, दिव्यपरिमलगन्धान् धारयामि। गन्धस्योपरि हरिद्राकुङ्कुमं समर्पयामि। \\
\devAya{} नमः, अक्षतान् समर्पयामि। \\
\devAya{} नमः, पुष्पमालिकां समर्पयामि। पुष्पैः पूजयामि।

\dnsub{अर्चना}
\begin{enumerate}
\begin{minipage}{0.475\linewidth}   
  \item सुमुखाय नमः
  \item एकदन्ताय नमः
  \item कपिलाय नमः
  \item गजकर्णकाय नमः
  \item लम्बोदराय नमः
  \item विकटाय नमः
  \item विघ्नराजाय नमः
  \item विनायकाय नमः
\end{minipage}
\begin{minipage}{0.525\linewidth}
  \item धूमकेतवे नमः
  \item गणाध्यक्षाय नमः
  \item फालचन्द्राय नमः
  \item गजाननाय नमः
  \item वक्रतुण्डाय नमः
  \item शूर्पकर्णाय नमः
  \item हेरम्बाय नमः
  \item स्कन्दपूर्वजाय नमः
\end{minipage}
\end{enumerate}
\devAya{} नमः, नानाविधपरिमलपत्रपुष्पाणि समर्पयामि॥\\
\devAya{} नमः, धूपमाघ्रापयामि।\\
अलङ्कारदीपं सन्दर्शयामि।\\
% नैवेद्यम्।\\
\ifbool{veda}{ॐ भूर्भुवः॒ सुवः॑। + ब्र॒ह्मणे॒ स्वाहा᳚।}{}
\devAya{} नमः, \blank{} (नैवेद्यं) निवेदयामि। 
\ifbool{veda}{अ॒मृ॒ता॒पि॒धा॒नम॑सि॥}{}
निवेदनान्तरम् आचमनीयं समर्पयामि।\\
\devAya{} नमः, ताम्बूलं समर्पयामि।\\
\devAya{} नमः, कर्पूरनीराजनं दर्शयामि। कर्पूरनीराजनानन्तरमाचमनीयं समर्पयामि।\\
\devAya{} नमः, मन्त्रपुष्पं समर्पयामि। स्वर्णपुष्पं समर्पयामि।\\

\twolineshloka*
{अभीप्सितार्थसिद्ध्यर्थं पूजितो यः सुरैरपि}
{सर्वविघ्नच्छिदे तस्मै गणाधिपतये नमः}

\twolineshloka*
{गजाननं भूतगणादिसेवितं कपित्थ-जम्बूफल-सार-भक्षितम्}
{उमासुतं शोकविनाशकारणं नमामि विघ्नेश्वरपादपङ्कजम्}

\centerline{अनन्तकोटिप्रदक्षिणनमस्कारान् समर्पयामि।}

\centerline{छत्त्रचामरादिसमस्तोपचारान् समर्पयामि।}

\twolineshloka*
{वक्रतुण्डमहाकाय कोटिसूर्यसमप्रभ}
{अविघ्नं कुरु मे देव सर्वकार्येषु सर्वदा}

\twolineshloka*
{सुमुखश्चैकदन्तश्च कपिलो गजकर्णकः}
{लम्बोदरश्च विकटो विघ्नराजो गणाधिपः}

\twolineshloka*
{धूमकेतुर्गणाध्यक्षो फालचन्द्रो गजाननः}
{वक्रतुण्डः शूर्पकर्णो हेरम्भः स्कन्दपूर्वजः}

\threelineshloka*
{षोडशैतानि नामानि यः पठेच्छृणुयादपि}
{विद्यारम्भे विवाहे च प्रवेशे निर्गमे तथा}
{सङ्ग्रामे सर्वकार्येषु विघ्नस्तस्य न जायते}

\centerline{प्रार्थनाः समर्पयामि।}

\closesub

\sect{प्रधान-पूजा — गो-पूजा}

\shuklambaradharam

\pranayama

\renewcommand{\additionalSankalpa}{
\begin{itemize}
    \item अस्मिन् भारत-देशे गो-सम्पदः विशेषतः भारतीय-गो-जातीनाम् अभिवृद्ध्यर्थम्
    \item जनानां गो-विषये श्रद्धा-भक्ति-प्राप्त्यर्थम्
    \item गो-वध-निवृत्त्यर्थं तद्-विरुद्ध-शासन-नियम-सिद्ध्यर्थम्
    \item गवाम् अन्येषां च चतुष्पदां द्विपदां च सर्व-विध-व्याधि-परिहार-द्वारा अरोग-जीवन-अवाप्त्यर्थम्
    \item गवां समृद्ध-क्षीर-प्रदत्वार्थम्
    \item काले काले निकामं वृष्टि-प्राप्त्यर्थं, तद्-द्वारा धान्यानाम् अन्येषां च सप्रयोजनानां सस्यानां वृक्षाणां च अभिवृद्ध्यर्थम्
    \item जनानां परिसरस्य स्वच्छतया सुष्ठुतया परिपालने श्रद्धायाः उदयार्थम्
    \item तद्-द्वारा पञ्च-महाभूतानां प्रदूषण-परिहार-द्वारा पृथिव्याः ताप-शान्त्यर्थम्
    \item भगवतः गोपालस्य भगवतः वृषभध्वजस्य च अनुग्रहेण लोके सर्वत्र साधूनां परित्राणार्थं, दुष्कृतां च सद्बुद्धि-प्राप्त्यर्थं, धर्मस्य अभ्युत्थानार्थं जयार्थम्, अधर्मस्य ग्लान्यर्थं नाशार्थम्
    \item जनानां प्राणिषु परस्परं च सद्भावनार्थं, विश्वस्य कल्याणार्थं, गोमातुः जयार्थं
    \item जनानां सुख-शान्त-समृद्ध-जीवनार्थम्
\end{itemize}
}
\renewcommand{\pujaam}{सवत्स-गोमातृ-पूजां}

\sankalpa{}

\vighneshvaraYathasthanam

\dnsub{आसन-पूजा}
\centerline{पृथिव्या  मेरुपृष्ठ  ऋषिः।  सुतलं  छन्दः।  कूर्मो  देवता॥}
\twolineshloka*
{पृथ्वि  त्वया  धृता  लोका  देवि  त्वं  विष्णुना  धृता}
{त्वं  च  धारय  मां  देवि  पवित्रं  चाऽऽसनं  कुरु}


\dnsub{घण्टा-पूजा}

\twolineshloka*
{आगमार्थं तु देवानां गमनार्थं तु रक्षसाम्}
{घण्टारवं करोम्यादौ देवताऽऽह्वानकारणम्}


\dnsub{कलशपूजा}
ॐ कलशाय नमः दिव्यगन्धान् धारयामि।\\
ॐ गङ्गायै नमः। ॐ यमुनायै नमः। ॐ गोदावर्यै नमः।  ॐ सरस्वत्यै नमः। ॐ नर्मदायै नमः। ॐ सिन्धवे नमः। ॐ कावेर्यै नमः।\\
ॐ सप्तकोटिमहातीर्थान्यावाहयामि।\\[-0.25ex]

(अथ कलशं स्पृष्ट्वा जपं कुर्यात्) \\
आपो॒ वा इ॒द सर्वं॒ विश्वा॑ भू॒तान्याप॑ प्रा॒णा वा आप॑ प॒शव॒ आपो\-ऽन्न॒मापोऽमृ॑त॒माप॑ स॒म्राडापो॑ वि॒राडाप॑ स्व॒राडाप॒श्\-छन्दा॒स्यापो॒ ज्योती॒ष्यापो॒ यजू॒ष्याप॑ स॒त्यमाप॒ सर्वा॑ दे॒वता॒ आपो॒ भूर्भुव॒ सुव॒राप॒ ओम्॥\\

\twolineshloka* 
{कलशस्य मुखे विष्णुः कण्ठे रुद्रः समाश्रितः}
{मूले तत्र स्थितो ब्रह्मा मध्ये मातृगणाः स्मृताः}
\threelineshloka* 
{कुक्षौ तु सागराः सर्वे सप्तद्वीपा वसुन्धरा}
{ऋग्वेदोऽथ यजुर्वेदः सामवेदोऽप्यथर्वणः}
{अङ्गैश्च सहिताः सर्वे कलशाम्बुसमाश्रिताः}
\twolineshloka* 
{गङ्गे च यमुने चैव गोदावरि सरस्वति}
{नर्मदे सिन्धुकावेरि जलेऽस्मिन् सन्निधिं कुरु}
\twolineshloka*
{सर्वे समुद्राः सरितः तीर्थानि च ह्रदा नदाः}
{आयान्तु देवपूजार्थं दुरितक्षयकारकाः}

\centerline{ॐ भूर्भुवः॒ सुवो॒ भूर्भुवः॒ सुवो॒ भूर्भुवः॒ सुवः॑।}

(इति कलशजलेन सर्वोपकरणानि आत्मानं च प्रोक्ष्य।)


\dnsub{आत्मपूजा}
ॐ आत्मने नमः, दिव्यगन्धान् धारयामि।
\begin{multicols}{2}
१. ॐ आत्मने नमः\\
२. ॐ अन्तरात्मने नमः\\
३. ॐ योगात्मने नमः\\
४. ॐ जीवात्मने नमः\\
५. ॐ परमात्मने नमः\\
६. ॐ ज्ञानात्मने नमः
\end{multicols}
समस्तोपचारान् समर्पयामि।

\twolineshloka*
{देहो देवालयः प्रोक्तो जीवो देवः सनातनः}
{त्यजेदज्ञाननिर्माल्यं सोऽहं भावेन पूजयेत्}


\begin{minipage}{\linewidth}
\dnsub{पीठ-पूजा}

\begin{multicols}{2}
\begin{enumerate}
\item आधारशक्त्यै नमः
\item मूलप्रकृत्यै नमः
\item आदिकूर्माय नमः 
\item आदिवराहाय नमः
\item अनन्ताय नमः
\item पृथिव्यै नमः
\item रत्नमण्डपाय नमः
\item रत्नवेदिकायै नमः
\item स्वर्णस्तम्भाय नमः
\item श्वेतच्छत्त्राय नमः
\item कल्पकवृक्षाय नमः
\item क्षीरसमुद्राय नमः 
\item सितचामराभ्यां नमः
\item योगपीठासनाय नमः
\end{enumerate}
\end{multicols}

\end{minipage}

\dnsub{गुरु-ध्यानम्}

\twolineshloka*
{गुरुर्ब्रह्मा गुरुर्विष्णुर्गुरुर्देवो महेश्वरः}
{गुरुः साक्षात् परं ब्रह्म तस्मै श्री-गुरवे नमः}


\sect{षोडशोपचार-पूजा}
\renewcommand{\devAya}{गोमात्रे}

\begin{center}

\textbf{ध्यानम्।}

\twolineshloka*
{नमो गोभ्यः श्रीमतीभ्यः सौरभेयीभ्य एव च}
{नमो ब्रह्म-सुताभ्यश्च पवित्राभ्यो नमो नमः}

\twolineshloka*
{गवाम् अङ्गेषु तिष्ठन्ति भुवनानि चतुर्दश}
{यस्मात् तस्माच्छिवं मे स्यादिह लोके परत्र च}

% सर्व-गो-समष्टि-रूपां सुरभिं कामधेनुं ध्यायामि।
गोमातरं ध्यायामि।

\textbf{आवाहनम्।}
\twolineshloka*
{आवाहयाम्यहं देवीं गां त्वां त्रैलोक्य-मातरम्}
{यस्याः स्मरण-मात्रेण सर्व-पाप-प्रणाशनम्}

\twolineshloka*
{त्वं देवी त्वं जगन्माता त्वमेवासि वसुन्धरा}
{गायत्री त्वं च सावित्री गङ्गा त्वं च सरस्वती}

\twolineshloka*
{आगच्छ देवि कल्याणि शुभां पूजां गृहाण च}
{वत्सेन सहितां त्वां वै देवीमावाहयाम्यहम्}
% सुरभिमावाहयामि।
गोमातरम् आवाहयामि।

\end{center}


१. शृङ्गमूलयोः ब्रह्मविष्णुभ्यां नमः, ब्रह्मविष्णू आवाहयामि।

२. शृङ्गाग्रे सर्वतीर्थेभ्यो नमः, सर्वतीर्थानि आवाहयामि।

३. शिरोमध्ये महादेवाय नमः, महादेवम् आवाहयामि।

४. ललाटाग्रे गौर्यै नमः, गौरीम् आवाहयामि।

५. नासावंशे षण्मुखाय नमः, षण्मुखम् आवाहयामि।

६. नासापुटयोः कम्बलाश्वतराभ्यां नागाभ्यां नमः, कम्बलाश्वतरौ नागौ आवाहयामि।

७. कर्णयोः अश्विभ्यां नमः, अश्विनौ आवाहयामि।

८. नेत्रयोः शशिभास्कराभ्यां नमः, शशिभास्करौ आवाहयामि।

९. दन्तेषु वायुभ्यो नमः, वायून् आवाहयामि।

१०. जिह्वायां वरुणाय नमः, वरुणम् आवाहयामि।

११. हुङ्कारे सरस्वत्यै नमः, सरस्वतीम् आवाहयामि।

१२. गण्डयोः मासपक्षाभ्यां नमः, मासपक्षौ आवाहयामि।

१३. ओष्ठयोः सन्ध्या-द्वयाय नमः, सन्ध्याद्वयम् आवाहयामि।

१४. ग्रीवायाम् इन्द्राय नमः, इन्द्रम् आवाहयामि।

१५. कक्षदेशे रक्षोभ्यो नमः, रक्षांसि आवाहयामि।

१६. उरसि साध्येभ्यो नमः, साध्यान् आवाहयामि।

१७. जङ्घासु धर्माय नमः, धर्मम् आवाहयामि।

१८. खुराणां मध्येषु गन्धर्वेभ्यो नमः, गन्धर्वान् आवाहयामि।

१९. खुराणां पूर्वाग्रेषु पन्नगेभ्यो नमः, पन्नगान् आवाहयामि।

२०. खुराणां पश्चिमाग्रेषु अप्सरोभ्यो नमः, अप्सरस आवाहयामि।

२१. पृष्ठे एकादशरुद्रेभ्यो नमः, एकादशरुद्रान् आवाहयामि।

२२. सर्वसन्धिषु अष्टवसुभ्यो नमः, अष्टवसून् आवाहयामि।

२३. श्रोणीतटे पितृभ्यो नमः, पितॄन् आवाहयामि।

२४. पुच्छे सोमाय नमः, सोमम् आवाहयामि।

२५. अधोगात्रेषु द्वादशादित्येभ्यो नमः, द्वादशादित्यान् आवाहयामि।

२६. वालेषु सूर्यरश्मिभ्यो नमः, सूर्यरश्मीन् आवाहयामि।

२७. गोमूत्रे गङ्गायै नमः, गङ्गाम् आवाहयामि।

२८. गोमये यमुनायै नमः, यमुनाम् आवाहयामि।

२९. क्षीरे सरस्वत्यै नमः, सरस्वतीम् आवाहयामि।

३०. दधनि नर्मदायै नमः, नर्मदाम् आवाहयामि।

३१. घृते वह्नये नमः, वह्निम् आवाहयामि।

३२. रोमसु त्रयस्त्रिंशत्कोटिदेवेभ्यो नमः, त्रयस्\-त्रिंशत्\-कोटि\-देवान् आवाहयामि।

३३. उदरे पृथिव्यै नमः, पृथिवीम् आवाहयामि।

३४. स्तनेषु चतुर्भ्यः सागरेभ्यो नमः, चतुरः सागरान् आवाहयामि।

३५. सर्वशरीरे कामधेनवे नमः, कामधेनुम् आवाहयामि।

\begin{center}

\twolineshloka*
{नाना-रत्न-समायुक्तं कार्तस्वर-विभूषितम्}
{आसनं ते मया दत्तं गृहाण जगदम्बिके}

% \renewcommand{\devAya}{ नमःसुरभ्यै नमः}
\devAya{} नमः, आसनं समर्पयामि।

\twolineshloka*
{सौरभेयि सर्व-हिते पवित्रे पाप-नाशिनि}
{गृहाण त्वं मया दत्तं पाद्यं त्रैलोक्य-वन्दिते}

\devAya{} नमः, पाद्यं समर्पयामि।

\twolineshloka*
{देहे स्थिताऽसि रुद्राणि शङ्करस्य सदा प्रिया}
{धेनु-रूपेण सा देवी मम पापं व्यपोहतु}

\devAya{} नमः, अर्घ्यं समर्पयामि।

\twolineshloka*
{या लक्ष्मीः सर्व-भूतेषु या च देवेष्ववस्थिता}
{धेनु-रूपेण सा देवी मम पापं व्यपोहतु}
\devAya{} नमः, आचमनीयं समर्पयामि।

\twolineshloka*
{सर्व-देव-मयि मातः सर्व-देव-नमस्कृते}
{तोयमेतत् सुख-स्पर्शं स्नानार्थं गृह्ण\footnotemark[\value{footnote}] धेनुके}
\footnotetext{अयं शब्दप्रयोगः पौराणिकः अतः आर्षः इति भाति।}
\devAya{} नमः, स्नानं समर्पयामि।
स्नानोत्तरम् आचमनीयं समर्पयामि।

\twolineshloka*
{आच्छादनं गवे दद्यां सम्यक् शुद्धं सुशोभनम्}
{सुरभिर्वस्त्र-दानेन प्रीयतां परमेश्वरी}
\devAya{} नमः, वस्त्रं समर्पयामि।

\twolineshloka*
{सर्व-देव-प्रियं देवि चन्दनं चन्द्र-सन्निभम्}
{कस्तूरी-कुङ्कुमाढ्यं च सुगन्धं प्रतिगृह्यताम्}
\devAya{} नमः, दिव्य-परिमल-गन्धान् धारयामि। गन्धस्योपरि हरिद्रा-कुङ्कुमं समर्पयामि।\

\twolineshloka*
{अक्षताश्च सुर-श्रेष्ठे कुङ्कुमाक्ताः सुशोभिताः}
{मया निवेदिता भक्त्या गृहाण परमेश्वरि}
\devAya{} नमः, अक्षतान् समर्पयामि।

\devAya{} नमः, शृङ्ग-भूषणं, कण्ठ-भूषणं, दोहन-पात्रम्, अन्यच्च यथा-शक्ति अलङ्कार-द्रव्यं समर्पयामि।

\twolineshloka*
{पुष्प-मालां तथा जाती-पाटली-चम्पकानि च}
{पुष्पाणि गृह्ण धेनो त्वं सर्व-विघ्न-प्रणाशिनि}
\devAya{} नमः, पुष्पमालां  समर्पयामि।


\begingroup
\centering
\setlength{\columnseprule}{1pt}
\let\chapt\sect
\input{../namavali-manjari/100/Lakshmi_108.tex}
\endgroup

\devAya{} नमः, नाना-विध-परिमल-पत्र-पुष्पाणि समर्पयामि।\\

\dnsub{उत्तराङ्ग-पूजा}

\twolineshloka*
{देव-द्रुम-रसोद्भूतं गो-घृतेन समन्वितम्}
{प्रयच्छामि महाभागे धूपोऽयं प्रतिगृह्यताम्}
\devAya{} नमः, धूपम् आघ्रापयामि।

\twolineshloka*
{आनन्द-दः सुराणां च लोकानां सर्वदा प्रियः}
{गौस्त्वं पाहि जगन्मातः, दीपोऽयं प्रतिगृह्यताम्}
\devAya{} नमः, दीपं दर्शयामि।

\twolineshloka*
{सुरभिर्वैष्णवी माता नित्यं विष्णु-पदे स्थिता}
{ग्रासं गृह्णातु सा धेनुर्याऽस्ति त्रैलोक्य-वासिनी}
\devAya{} नमः, नैवेद्यं निवेदयामि।\\
गो-ग्रासं समर्पयामि। निवेदनानन्तरम् आचमनीयं समर्पयामि।

\twolineshloka*
{नीराजनं गृहाणेदं कर्पूरैः कलितं मया}
{कामधेनु-समुद्भूते सर्वाभीष्ट-फल-प्रदे}
\devAya{} नमः, नीराजनं दर्शयामि।

\twolineshloka*
{गोभ्यो यज्ञाः प्रवर्तन्ते गोभ्यो देवाः समुत्थिताः}
{गोभ्यो वेदाः समुत्कीर्णाः स-षडङ्ग-पद-क्रमाः}
\devAya{} नमः, पुष्पाञ्जलिं समर्पयामि।

\twolineshloka*
{यानि कानि च पापानि जन्मान्तर-कृतानि च}
{तानि नाशय धेनो त्वं प्रदक्षिणपदे पदे}


\section{प्रार्थना --- गोमती विद्या}

\addtocounter{shlokacount}{49}

\twolineshloka
{गावः सुरभयो नित्यं गावो गुग्गुलु-गन्धिकाः}
{गावः प्रतिष्ठा भूतानां गावः स्वस्त्ययनं महत्}

\twolineshloka
{अन्नमेव परं गावो देवानां हविरुत्तमम्}
{पावनं सर्व-भूतानां क्षरन्ति च वहन्ति च}% ॥

\twolineshloka
{हविषा मन्त्र-पूतेन तर्पयन्त्यमरान् दिवि}
{ऋषीणाम् अग्निहोत्रेषु, गावो होमे प्रतिष्ठिताः}% ॥

\threelineshloka
{सर्वेषामेव भूतानां गावः शरणमुत्तमम्}
{गावः पवित्रं परमं गावो मङ्गलमुत्तमम्}
{गावः स्वर्गस्य सोपानं गावो धन्याः सनातनाः}

\twolineshloka
{नमो गोभ्यः श्रीमतीभ्यः सौरभेयीभ्य एव च}
{नमो ब्रह्म-सुताभ्यश्च पवित्राभ्यो नमो नमः}% ॥

\twolineshloka
{ब्राह्मणाश्चैव गावश्च कुलमेकं द्विधा स्थितम्}
{एकत्र मन्त्रास्तिष्ठन्ति हविरेकत्र तिष्ठति}% ॥

\twolineshloka
{देव-ब्राह्मण-गो-साधु-साध्वीभिः सकलं जगत्}
{धार्यते वै सदा तस्मात् सर्वे पूज्यतमाः सदा}% ॥

\twolineshloka
{यत्र तीर्थे सदा गावः पिबन्ति तृषिता जलम्}
{उत्तरन्ति पथा येन स्थिता तत्र सरस्वती}% ॥५७॥

\fourlineindentedshloka
{गवां हि तीर्थे वसतीह गङ्गा}
{पुष्टिस्तथा तद्-रजसि प्रवृद्धा}
{लक्ष्मीः करीषे प्रणतौ च धर्मः}
{तासां प्रणामं सततं च कुर्यात्}% ॥

॥इति श्रीविष्णुधर्मोत्तरे द्वितीयखण्डे मार्कण्डेयवज्रसंवादे गोमाहात्म्ये गोमतीविद्या नाम द्विचत्वारिंशत्तमोऽध्यायः॥

\section{अन्ये प्रार्थना-श्लोकाः}
\twolineshloka
{गावो माम् उपतिष्ठन्तु हेम-शृङ्ग्यः पयोमुचः}
{सुरभ्यः सौरभेय्यश्च सरितः सागरं यथा}

\twolineshloka
{गा वै पश्यान्यहं नित्यं गावः पश्यन्तु मां सदा}
{गावोऽस्माकं वयं तासां यतो गावस्ततो वयम्}

\twolineshloka
{एवं रात्रौ दिवा वाऽपि समेषु विषमेषु च}
{भयेषु च नरो नित्यं कीर्तयन् मुच्यते भयात्}

\twolineshloka
{घृत-क्षीर-प्रदा गावो घृत-योन्यो घृतोद्भवाः}
{घृत-नद्यो घृतावर्तास्ता मे सन्तु सदा गृहे}

\twolineshloka
{घृतं मे हृदये नित्यं घृतं नाभ्यां प्रतिष्ठितम्}
{घृतं मे सर्वतश्चैव घृतं मे मनसि स्थितम्}

\twolineshloka
{गावो ममाग्रतः सन्तु गावो मे सन्तु पृष्ठतः}
{गावो मे हृदये सन्तु गवां मध्ये वसाम्यहम्}

\threelineshloka
{इत्याचम्य जपन् प्रातः सायं वा पुरुषस्तथा}
{यद् रात्र्या कुरुते पापं तद् रात्र्या प्रतिमुच्यते}
{यद् अह्ना कुरुते पापं तद् अह्ना प्रतिमुच्यते}

\devAya{} नमः, प्रार्थनाः समर्पयामि।

\fourlineindentedshloka*
{कायेन वाचा मनसेन्द्रियैर्वा}
{बुद्‌ध्याऽऽत्मना वा प्रकृतेः स्वभावात्}
{करोमि यद्यत् सकलं परस्मै}
{नारायणायेति समर्पयामि}

\medskip

\centerline{\textbf{अनेन पूजनेन गोमाता प्रीयताम्।}}

\medskip

ॐ तत् सद् ब्रह्मार्पणमस्तु।

\medskip

धर्म की जय हो! अधर्म का नाश हो! प्राणियों में सद्भावना हो!

विश्व का कल्याण हो! गौ हत्या बन्द हो! गौ माता की जय हो!

\hfill\small{(\textsf{---} पूज्य करपात्र स्वामी जी)}

\end{center}

\section{अनुबन्धः}
{\centering \textbf{गोशरीरे यत्र यत्र या या देवता वर्तन्ते\\ इत्यत्र प्रमाणश्लोकाः हेमाद्रौ भविष्ये\\
}}

(अत्र वसूनां रुद्राणाम् अश्विनोश्च उक्तत्वात् अनुक्ताः “अधोगात्रेषु द्वादशादित्याः” अवशिष्ट-पूरणाय गीता-मुद्रणालय-प्रामाण्येन पूर्वत्र योजिताः।)

\twolineshloka
{शृङ्गमूले गवां नित्यं ब्रह्मविष्णू समाश्रितौ}
{शृङ्गाग्रे सर्वतीर्थानि स्थावराणि चराणि च}

\twolineshloka
{शिरोमध्ये महादेवः सर्वदेवमयः स्थितः}
{ललाटाग्रे स्थिता गौरी नासावंशे च षण्मुखः}

\twolineshloka
{कम्बलाश्वतरौ नागौ नासापुटमुपाश्रितौ}
{कर्णयोरश्विनौ देवौ चक्षुषोः शशिभास्करौ}

\twolineshloka
{दन्तेषु वायवः सर्वे जिह्वायां वरुणः स्थितः}
{सरस्वती च हुङ्कारे मासपक्षौ च गण्डयोः}

\twolineshloka
{सन्ध्याद्वयं तथौष्ठाभ्यां ग्रीवामिन्द्रः समाश्रितः}
{रक्षांसि कक्षदेशे तु साध्याश्चोरसि संस्थिताः}

\twolineshloka
{चतुष्पात् सकलो धर्मः स्वयं जङ्घासु संस्थितः}
{खुरमध्ये तु गन्धर्वाः खुराग्रेषु च पन्नगाः}

\twolineshloka
{खुराणां पश्चिमाग्रेषु गणा ह्यप्सरसां स्थिताः}
{रुद्राश्चैकादश पृष्ठे वसवः सर्वसन्धिषु}

\twolineshloka
{श्रोणीतटस्थाः पितरः सोमो लाङ्गूलमाश्रितः}
{आदित्यरश्मयो वालाः पिण्डीभूता व्यवस्थिताः}

\twolineshloka
{साक्षाद् गङ्गा च गोमूत्रे गोमये यमुना स्थिता}
{क्षीरे सरस्वती देवी नर्मदा दध्नि संस्थिता}

\twolineshloka
{हुताशनः स्वयं सर्पिर्ब्राह्मणानां गुरुः परः}
{[अष्टाविंशति?]त्रयस्त्रिंशत् तु देवानां कोट्यो रोमसु संस्थिताः}

\twolineshloka
{उदरे पृथिवी ज्ञेया सशैलवनकानना}
{चत्वारः सागराः पूर्णा गवां ये तु पयोधराः}

\twolineshloka
{एतद् वः कथितं सर्वं यथा गोषु प्रतिष्ठितम्}
{जगद् वै देवशार्दूल सदेवासुरमानवम्}

\closesection
