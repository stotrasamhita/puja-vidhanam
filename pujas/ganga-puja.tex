% !TeX program = XeLaTeX
% !TeX root = ..\pujavidhanam.tex

\setlength{\parindent}{0pt}
\chapt{श्री-गङ्गापूजा}

\dnsub{पूर्वाङ्ग-विघ्नेश्वर-पूजा}

\graphicspath{{purvanga/}{../purvanga/}}

\centerline{\includegraphics[width=1cm]{ganesha.pdf}}

(आचम्य)

\twolineshloka*
{शुक्लाम्बरधरं विष्णुं शशिवर्णं चतुर्भुजम्}
{प्रसन्नवदनं ध्यायेत् सर्वविघ्नोपशान्तये}
 
प्राणान्  आयम्य।

(अप उपस्पृश्य, पुष्पाक्षतान् गृहीत्वा)\\

\twolineshloka*
{तदेव लग्नं सुदिनं तदेव ताराबलं चन्द्रबलं तदेव}
{विद्याबलं दैवबलं तदेव लक्ष्मीपतेरङ्घ्रियुगं स्मरामि}
 
ममोपात्त-समस्त-दुरित-क्षयद्वारा \\
श्री-परमेश्वर-प्रीत्यर्थं करिष्यमाणस्य कर्मणः\\
अविघ्नेन परिसमाप्त्यर्थम् आदौ विघ्नेश्वरपूजां करिष्ये।

(अप उपस्पृश्य)

\ifbool{veda}{
\twolineshloka*
{ॐ ग॒णानां᳚ त्वा ग॒णप॑तिꣳ हवामहे क॒विं क॑वी॒नामु॑प॒मश्र॑वस्तमम्}
{ज्ये॒ष्ठ॒राजं॒ ब्रह्म॑णां ब्रह्मणस्पत॒ आ नः॑ शृ॒ण्वन्नू॒तिभिः॑ सीद॒ साद॑नम्}
}{}

\twolineshloka*
{अगजानानपद्मार्कं गजाननमहर्निशम्}
{अनेकदं तं भक्तानाम् एकदन्तमुपास्महे}

\ifbool{veda}{भूर्भुवः॒ सुव॒रोम्।}{} अस्मिन् हरिद्राबिम्बे सुमुखं महागणपतिं ध्यायामि, आवाहयामि।\\

\renewcommand{\devAya}{\OMshri महागणपतये}

\devAya{} नमः, आसनं समर्पयामि।\\
\devAya{} नमः, पादयोः पाद्यं समर्पयामि।\\
\devAya{} नमः, अर्घ्यं समर्पयामि।\\
\devAya{} नमः, आचमनीयं समर्पयामि।\\
\devAya{} नमः, मधुपर्कं समर्पयामि।\\
\ifbool{veda}{ॐ भूर्भुवः॒ सुवः॑।}{}
\devAya{} नमः, शुद्धोदकस्नानं समर्पयामि। स्नानानन्तरमाचमनीयं समर्पयामि।\\
\devAya{} नमः, वस्त्रार्थमक्षतान् समर्पयामि।\\
\devAya{} नमः, यज्ञोपवीताभरणार्थे अक्षतान् समर्पयामि।\\
\devAya{} नमः, दिव्यपरिमलगन्धान् धारयामि। गन्धस्योपरि हरिद्राकुङ्कुमं समर्पयामि। \\
\devAya{} नमः, अक्षतान् समर्पयामि। \\
\devAya{} नमः, पुष्पमालिकां समर्पयामि। पुष्पैः पूजयामि।

\dnsub{अर्चना}
\begin{enumerate}
\begin{minipage}{0.475\linewidth}   
  \item सुमुखाय नमः
  \item एकदन्ताय नमः
  \item कपिलाय नमः
  \item गजकर्णकाय नमः
  \item लम्बोदराय नमः
  \item विकटाय नमः
  \item विघ्नराजाय नमः
  \item विनायकाय नमः
\end{minipage}
\begin{minipage}{0.525\linewidth}
  \item धूमकेतवे नमः
  \item गणाध्यक्षाय नमः
  \item फालचन्द्राय नमः
  \item गजाननाय नमः
  \item वक्रतुण्डाय नमः
  \item शूर्पकर्णाय नमः
  \item हेरम्बाय नमः
  \item स्कन्दपूर्वजाय नमः
\end{minipage}
\end{enumerate}
\devAya{} नमः, नानाविधपरिमलपत्रपुष्पाणि समर्पयामि॥\\
\devAya{} नमः, धूपमाघ्रापयामि।\\
अलङ्कारदीपं सन्दर्शयामि।\\
% नैवेद्यम्।\\
\ifbool{veda}{ॐ भूर्भुवः॒ सुवः॑। + ब्र॒ह्मणे॒ स्वाहा᳚।}{}
\devAya{} नमः, \blank{} (नैवेद्यं) निवेदयामि। 
\ifbool{veda}{अ॒मृ॒ता॒पि॒धा॒नम॑सि॥}{}
निवेदनान्तरम् आचमनीयं समर्पयामि।\\
\devAya{} नमः, ताम्बूलं समर्पयामि।\\
\devAya{} नमः, कर्पूरनीराजनं दर्शयामि। कर्पूरनीराजनानन्तरमाचमनीयं समर्पयामि।\\
\devAya{} नमः, मन्त्रपुष्पं समर्पयामि। स्वर्णपुष्पं समर्पयामि।\\

\twolineshloka*
{अभीप्सितार्थसिद्ध्यर्थं पूजितो यः सुरैरपि}
{सर्वविघ्नच्छिदे तस्मै गणाधिपतये नमः}

\twolineshloka*
{गजाननं भूतगणादिसेवितं कपित्थ-जम्बूफल-सार-भक्षितम्}
{उमासुतं शोकविनाशकारणं नमामि विघ्नेश्वरपादपङ्कजम्}

\centerline{अनन्तकोटिप्रदक्षिणनमस्कारान् समर्पयामि।}

\centerline{छत्त्रचामरादिसमस्तोपचारान् समर्पयामि।}

\twolineshloka*
{वक्रतुण्डमहाकाय कोटिसूर्यसमप्रभ}
{अविघ्नं कुरु मे देव सर्वकार्येषु सर्वदा}

\twolineshloka*
{सुमुखश्चैकदन्तश्च कपिलो गजकर्णकः}
{लम्बोदरश्च विकटो विघ्नराजो गणाधिपः}

\twolineshloka*
{धूमकेतुर्गणाध्यक्षो फालचन्द्रो गजाननः}
{वक्रतुण्डः शूर्पकर्णो हेरम्भः स्कन्दपूर्वजः}

\threelineshloka*
{षोडशैतानि नामानि यः पठेच्छृणुयादपि}
{विद्यारम्भे विवाहे च प्रवेशे निर्गमे तथा}
{सङ्ग्रामे सर्वकार्येषु विघ्नस्तस्य न जायते}

\centerline{प्रार्थनाः समर्पयामि।}

\closesub

\sect{प्रधान-पूजा — गङ्गापूजा}

\twolineshloka*
{शुक्लाम्बरधरं विष्णुं शशिवर्णं चतुर्भुजम्}
{प्रसन्नवदनं ध्यायेत् सर्वविघ्नोपशान्तये}

प्राणान् आयम्य। ॐ भूः + भूर्भुवः॒ सुव॒रोम्।

\dnsub{सङ्कल्पः}

ममोपात्त-समस्त-दुरित-क्षयद्वारा श्री-परमेश्वर-प्रीत्यर्थं शुभे शोभने मुहूर्ते अद्य ब्रह्मणः
द्वितीयपरार्धे श्वेतवराहकल्पे वैवस्वतमन्वन्तरे अष्टाविंशतितमे कलियुगे प्रथमे पादे
जम्बूद्वीपे भारतवर्षे भरतखण्डे मेरोः दक्षिणे पार्श्वे शकाब्दे अस्मिन् वर्तमाने व्यावहारिकाणां प्रभवादीनां षष्ट्याः संवत्सराणां मध्ये \mbox{(~~~)}\see{app:samvatsara_names} नाम संवत्सरे दक्षिणायने 
शरद्-ऋतौ तुला-मासे कृष्णपक्षे \mbox{(~~~)} शुभतिथौ
(इन्दु / भौम / बुध / गुरु / भृगु / स्थिर / भानु)-वासरयुक्तायाम्
\mbox{(~~~)}\see{app:nakshatra_names} नक्षत्र \mbox{(~~~)}\see{app:yoga_names}-योग \mbox{(~~~)}-करण-युक्तायां च एवं गुण विशेषण विशिष्टायाम्
अस्यां \mbox{(~~~)} शुभतिथौ 
अस्माकं सहकुटुम्बानां क्षेमस्थैर्य-धैर्य-वीर्य-विजय-आयुरारोग्य-ऐश्वर्याभिवृद्ध्यर्थम्
धर्मार्थकाममोक्ष\-चतुर्विधफलपुरुषार्थसिद्ध्यर्थं पुत्रपौत्राभि\-वृद्ध्यर्थम् इष्टकाम्यार्थसिद्ध्यर्थम्
मम इहजन्मनि पूर्वजन्मनि जन्मान्तरे च सम्पादितानां ज्ञानाज्ञानकृतमहा\-पातकचतुष्टय-व्यतिरिक्तानां रहस्यकृतानां प्रकाशकृतानां सर्वेषां पापानां सद्य अपनोदनद्वारा सकल-पापक्षयार्थं
गङ्गा-देवी-प्रीत्यर्थं
यावच्छक्ति-ध्यानावाहनादि-षोडशोपचार-गङ्गा-पूजां करिष्ये तदङ्गं कलशपूजां च करिष्ये।


श्रीविघ्नेश्वराय नमः यथास्थानं प्रतिष्ठापयामि।\\
(गणपति-प्रसादं शिरसा गृहीत्वा)

\dnsub{आसन-पूजा}
\centerline{पृथिव्या  मेरुपृष्ठ  ऋषिः।  सुतलं  छन्दः।  कूर्मो  देवता॥}
\twolineshloka*
{पृथ्वि  त्वया  धृता  लोका  देवि  त्वं  विष्णुना  धृता}
{त्वं  च  धारय  मां  देवि  पवित्रं  चाऽऽसनं  कुरु}


\dnsub{घण्टा-पूजा}

\twolineshloka*
{आगमार्थं तु देवानां गमनार्थं तु रक्षसाम्}
{घण्टारवं करोम्यादौ देवताऽऽह्वानकारणम्}


\dnsub{कलशपूजा}
ॐ कलशाय नमः दिव्यगन्धान् धारयामि।\\
ॐ गङ्गायै नमः। ॐ यमुनायै नमः। ॐ गोदावर्यै नमः।  ॐ सरस्वत्यै नमः। ॐ नर्मदायै नमः। ॐ सिन्धवे नमः। ॐ कावेर्यै नमः।\\
ॐ सप्तकोटिमहातीर्थान्यावाहयामि।\\[-0.25ex]

(अथ कलशं स्पृष्ट्वा जपं कुर्यात्) \\
आपो॒ वा इ॒द सर्वं॒ विश्वा॑ भू॒तान्याप॑ प्रा॒णा वा आप॑ प॒शव॒ आपो\-ऽन्न॒मापोऽमृ॑त॒माप॑ स॒म्राडापो॑ वि॒राडाप॑ स्व॒राडाप॒श्\-छन्दा॒स्यापो॒ ज्योती॒ष्यापो॒ यजू॒ष्याप॑ स॒त्यमाप॒ सर्वा॑ दे॒वता॒ आपो॒ भूर्भुव॒ सुव॒राप॒ ओम्॥\\

\twolineshloka* 
{कलशस्य मुखे विष्णुः कण्ठे रुद्रः समाश्रितः}
{मूले तत्र स्थितो ब्रह्मा मध्ये मातृगणाः स्मृताः}
\threelineshloka* 
{कुक्षौ तु सागराः सर्वे सप्तद्वीपा वसुन्धरा}
{ऋग्वेदोऽथ यजुर्वेदः सामवेदोऽप्यथर्वणः}
{अङ्गैश्च सहिताः सर्वे कलशाम्बुसमाश्रिताः}
\twolineshloka* 
{गङ्गे च यमुने चैव गोदावरि सरस्वति}
{नर्मदे सिन्धुकावेरि जलेऽस्मिन् सन्निधिं कुरु}
\twolineshloka*
{सर्वे समुद्राः सरितः तीर्थानि च ह्रदा नदाः}
{आयान्तु देवपूजार्थं दुरितक्षयकारकाः}

\centerline{ॐ भूर्भुवः॒ सुवो॒ भूर्भुवः॒ सुवो॒ भूर्भुवः॒ सुवः॑।}

(इति कलशजलेन सर्वोपकरणानि आत्मानं च प्रोक्ष्य।)


\dnsub{आत्मपूजा}
ॐ आत्मने नमः, दिव्यगन्धान् धारयामि।
\begin{multicols}{2}
१. ॐ आत्मने नमः\\
२. ॐ अन्तरात्मने नमः\\
३. ॐ योगात्मने नमः\\
४. ॐ जीवात्मने नमः\\
५. ॐ परमात्मने नमः\\
६. ॐ ज्ञानात्मने नमः
\end{multicols}
समस्तोपचारान् समर्पयामि।

\twolineshloka*
{देहो देवालयः प्रोक्तो जीवो देवः सनातनः}
{त्यजेदज्ञाननिर्माल्यं सोऽहं भावेन पूजयेत्}


\begin{minipage}{\linewidth}
\dnsub{पीठ-पूजा}

\begin{multicols}{2}
\begin{enumerate}
\item आधारशक्त्यै नमः
\item मूलप्रकृत्यै नमः
\item आदिकूर्माय नमः 
\item आदिवराहाय नमः
\item अनन्ताय नमः
\item पृथिव्यै नमः
\item रत्नमण्डपाय नमः
\item रत्नवेदिकायै नमः
\item स्वर्णस्तम्भाय नमः
\item श्वेतच्छत्त्राय नमः
\item कल्पकवृक्षाय नमः
\item क्षीरसमुद्राय नमः 
\item सितचामराभ्यां नमः
\item योगपीठासनाय नमः
\end{enumerate}
\end{multicols}

\end{minipage}

\dnsub{गुरु-ध्यानम्}

\twolineshloka*
{गुरुर्ब्रह्मा गुरुर्विष्णुर्गुरुर्देवो महेश्वरः}
{गुरुः साक्षात् परं ब्रह्म तस्मै श्री-गुरवे नमः}


\sect{षोडशोपचार-पूजा}
\renewcommand{\devAya}{गङ्गा-देव्यै नमः,}
(आचम्य)

[विघ्नेश्वरपूजां कृत्वा।]

\twolineshloka*
{शुक्लाम्बर-धरं विष्णुं शशि-वर्णं चतुर्भुजम्}
{प्रसन्न-वदनं ध्यायेत् सर्व-विघ्नोपशान्तये}

प्राणानायम्य। ममोपात्त + … ऋण-त्रय-विमोचनार्थं गङ्गा-भागीरथी-प्रसाद-सिद्ध्यर्थं गङ्गा-पूजां करिष्ये।

\addtolength{\shlokavskip}{1ex}

\begin{center}

\twolineshloka*
{चतुर्भुजां त्रि-नयनां शुद्ध-स्फटिक-सन्निभाम्}
{ध्यायेऽहं मकरारूढां शुभ्र-वस्त्रां शुचि-स्मिताम्}
गङ्गा-देवीं ध्यायामि॥

\twolineshloka*
{विष्णु-पादाब्ज-सम्भूते विश्वनाथ-शिरः-स्थिते}
{आवाहयामि गङ्गे त्वां भक्ताभीष्ट-फल-प्रदे}
गङ्गा-देवीम् आवाहयामि

\twolineshloka*
{मुक्ता-रत्न-सुवर्णादि-खचितं सुन्दरं शुभम्}
{सिंहासनं प्रदास्यामि गृहाण मकरासने }
गङ्गा-देव्यै नमः, आसनं समर्पयामि॥

\twolineshloka*
{सिन्ध्वादि-सरिदुद्भूतं गन्ध-पुष्प-समन्वितम्}
{पाद्यं ददाम्यहं देवि प्रसीद परमेश्वरि}
गङ्गा-देव्यै नमः, पाद्यं समर्पयामि॥

\twolineshloka*
{ब्रह्म-पात्र-समुद्भूते गङ्गे त्रि-पथ-गामिनि}
{गृहाणार्घ्यं प्रदास्यामि जह्नु-कन्ये नमोऽस्तु ते}
गङ्गा-देव्यै नमः, अर्घ्यं समर्पयामि॥

\twolineshloka*
{सुवर्ण-कलशानीतं नाना-गन्ध-सुवासितम्}
{आचम्यतां मया दत्तं गृहाणामृत-वर्षिणि}
गङ्गा-देव्यै नमः, आचमनीयं समर्पयामि॥

\twolineshloka*
{पयोदधि-घृत-क्षौद्र-रम्भा-फल-समन्वितम्}
{पञ्चामृतमिदं देवि स्वीकुरुष्व महेश्वरि}
गङ्गा-देव्यै नमः, पञ्चामृतस्नानं समर्पयामि॥

\twolineshloka*
{नर्मदा-यमुना-सिन्धु-गोदावर्याहृतैर्जलैः}
{स्नापयामि शिवे भक्त्या भागीरथि नमोऽस्तु ते}
गङ्गा-देव्यै नमः, स्नानं समर्पयामि॥
स्नानानन्तरम् आचमनीयं समर्पयामि॥

\twolineshloka*
{वैदूर्य-पद्मरागादि-खचितं मेखलान्वितम्}
{सुवर्ण-सूत्र-संयुक्तं क्षौमं दास्यामि गृह्यताम्}
गङ्गा-देव्यै नमः, वस्त्रं समर्पयामि॥

\twolineshloka*
{मलयाचल-सम्भूतं कस्तूरी-कुङ्कुमान्वितम्}
{कर्पूर-मिश्रितं गन्धं गृहाण परमेश्वरि}
गङ्गा-देव्यै नमः, गन्धान् धारयामि॥

\twolineshloka*
{अक्षतान् शालि-संभूतान् हरिद्रा-कुङ्कुमान्वितान्}
{पूजार्थं सङ्गृहाणेमान् अक्षय्य-फल-दायिनि}
गङ्गा-देव्यै नमः, अक्षतान् समर्पयामि॥

\twolineshloka*
{वज्र-वैदूर्य-माणिक्य-पद्मरागादि-निर्मितम्}
{कङ्कणं कर-शोभार्थं कामितार्थ-फल-प्रदे}
गङ्गा-देव्यै नमः, आभरणानि समर्पयामि॥

\twolineshloka*
{केतकी-तुलसी-बिल्व-मल्लिका-कमलादिभिः}
{पुन्नागैरर्चयामि त्वां गृहाणामर-वन्दिते}
गङ्गा-देव्यै नमः, पुष्पमालां समर्पयामि॥
पुष्पैः सम्पूजयामि।



\sect{अङ्ग-पूजा}

\begin{supertabular}{ll}
पाप-पर्वत-नाशिन्यै नमः & पादौ पूजयामि\\
भक्त-वत्सलायै नमः & गुल्फौ पूजयामि\\
जगद्धात्र्यै नमः & जङ्घे पूजयामि\\
जाह्नव्यै नमः & जानुनी पूजयामि\\
शैल-सुतायै नमः & ऊरू पूजयामि\\
समुद्र-गामिन्यै नमः & कटिं पूजयामि\\
मकरारूढायै नमः & गुह्यं पूजयामि\\
आनन्द-वर्धिन्यै नमः & जघनं पूजयामि\\
गङ्गायै नमः & नाभिं पूजयामि\\
जगत्-कुक्ष्यै नमः & उदरं पूजयामि\\
विशाल-वक्षसे नमः & वक्षः-स्थलं पूजयामि\\
ह्लादिन्यै नमः & हृदयं पूजयामि\\
सुस्तन्यै नमः & स्तनौ पूजयामि\\
तरङ्गिण्यै नमः & पार्श्वौ पूजयामि\\
उन्नत-कण्ठ्यै नमः & कण्ठं पूजयामि\\
त्रैलोक्य-सुन्दर्यै नमः & स्कन्धौ पूजयामि\\
अमृत-कलश-हस्तायै नमः & हस्तान् पूजयामि\\
लीला-शुक-धारिण्यै नमः & बाहून् पूजयामि\\
विद्या-प्रकाशिन्यै नमः & मुखं पूजयामि\\
त्रैलोक्य-वासिन्यै नमः & ललाटं पूजयामि\\
सुनासिकायै नमः & नासिकां पूजयामि\\
मकर-कुण्डल-धारिण्यै नमः & श्रोत्रे पूजयामि\\
बिम्बोष्ठ्यै नमः & ओष्ठे पूजयामि\\
अनाथ-रक्षिण्यै नमः & अधरं पूजयामि\\
चञ्चल-गत्यै नमः & जिह्वां पूजयामि\\
अलक-नन्दायै नमः & गण्ड-स्थलं पूजयामि\\
तिलक-धारिण्यै नमः & फालं पूजयामि\\
ज्ञान-रूपिण्यै नमः & चुबुकं पूजयामि\\
अमृत-बिम्बायै नमः & अलकान् पूजयामि\\
किरीट-धारिण्यै नमः & शिरः पूजयामि\\
भागीरथ्यै नमः & सर्वाण्यङ्गानि पूजयामि\\
\end{supertabular}

\end{center}

\begingroup
\centering
\setlength{\columnseprule}{1pt}
\let\chapt\sect
\input{../namavali-manjari/100/Ganga_108_2.tex}
\input{../namavali-manjari/100/Ganga_108.tex}

\endgroup

\begin{center}
गङ्गा-देव्यै नमः, नाना-विध-परिमल-पत्र-पुष्पाणि समर्पयामि॥

\twolineshloka*
{चन्दनागरु-मुस्तादि-घृत-गुग्गुलु-संयुतम्}
{दशाङ्ग-द्रव्य-संयुक्तं धूपोऽयं प्रतिगृह्यताम्}
गङ्गा-देव्यै नमः, धूपम् आघ्रापयामि॥

\twolineshloka*
{साज्यं त्रि-वर्ति-संयुक्तं वह्निना योजितं मया}
{गृहाण मङ्गलं दीपं त्रैलोक्य-तिमिरापहम्}
गङ्गा-देव्यै नमः, दीपं दर्शयामि॥

\twolineshloka*
{शाल्यन्नं व्यञ्जनैर्युक्तं सूपापूप-घृतान्वितम्}
{क्षीरान्नं लड्डुकोपेतं भुज्यताममृताशिनि}
गङ्गा-देव्यै नमः, नैवेद्यं समर्पयामि॥

\twolineshloka*
{पूगी-फल-समायुक्तं नाग-वल्ली-दलैर्युतम्}
{कर्पूर-चूर्ण-संयुक्तं ताम्बूलं प्रतिगृह्यताम्}
गङ्गा-देव्यै नमः, कर्पूर-ताम्बूलं समर्पयामि॥

\twolineshloka*
{तेजः-पुञ्ज-स्वरूपे ते तेजसा भासितं जगत्}
{नीराजयामि गङ्गे त्वां भक्ताभीष्ट-फल-प्रदे}
गङ्गा-देव्यै नमः, नीराजनं समर्पयामि॥

\twolineshloka*
{गङ्गे त्रि-पथ-गे दिव्ये जाह्नवि त्रिदिव-स्थिते}
{प्रदक्षिणं करोमि त्वां प्रणताघौघ-नाशिनि}
गङ्गा-देव्यै नमः, प्रदक्षिण-नमस्कारान् समर्पयामि॥

\fourlineindentedshloka*
{प्रालेयाचल-सम्भूते प्राचीनाब्धि-समागमे}
{प्राणिनां भव-रोग-घ्नि पूजा-सम्पूर्णतां कुरु}
{पुत्र-पौत्र-धरा-धान्य-पशु-पुण्य-फलोदयम्}
{देहि मे देवि भक्तिं ते त्वत्-पाद-कमले सदा}
गङ्गा-देव्यै नमः, प्रार्थनां समर्पयामि॥

पूजान्ते क्षीरार्घ्य-प्रदानं करिष्ये।

\resetShloka

\twolineshloka
{ब्रह्म-पात्र-समुद्भूते गङ्गे त्रि-पथ-गामिनि}
{त्रैलोक्य-वन्दिते देवि गृहाणार्घ्यं नमोऽस्तु ते}
गङ्गायै नमः इदमर्घ्यम् इदमर्घ्यम् इदमर्घ्यम्॥

\twolineshloka
{तपनस्य सुते देवि यम-ज्येष्ठे यशस्विनि}
{शुद्धानां शुद्धि-दे देवि गृहाणार्घ्यं नमोऽस्तु ते}
यमुनायै नमः इदमर्घ्यम् इदमर्घ्यम् इदमर्घ्यम्॥

\twolineshloka
{विरिञ्चि-तनये देवि ब्रह्मरन्ध्र-निवासिनि}
{सरस्वति जगन्मातर्गृहाणार्घ्यं नमोऽस्तु ते}
सरस्वत्यै नमः इदमर्घ्यम् इदमर्घ्यम् इदमर्घ्यम्॥

\twolineshloka
{गङ्गा-यमुनयोर्मध्ये यत्र गुप्ता सरस्वती}
{त्रैलोक्य-वन्दिते देवि त्रिवेण्यर्घ्यं नमोऽस्तु ते}
त्रिवेण्यै नमः इदमर्घ्यम् इदमर्घ्यम् इदमर्घ्यम्॥

\twolineshloka
{एकार्णवे महाकल्पे सुषुप्तौ माधव-प्रभोः}
{पर्यङ्क वट-राज त्वं गृहाणार्घ्यं नमोऽस्तु ते}
वट-राजाय नमः इदमर्घ्यम् इदमर्घ्यम् इदमर्घ्यम्॥

\twolineshloka
{वेणी-माधव सर्व-ज्ञ भक्तेप्सित-फल-प्रद}
{सफलां कुरु मे यात्रां वेणी-माधव ते नमः}
तीर्थ-राजाय नमः इदमर्घ्यम् इदमर्घ्यम् इदमर्घ्यम्॥

\twolineshloka*
{त्रि-वेणि त्र्यम्बके देवि त्रि-विधाघ-विनाशिनि}
{त्रि-मार्गे त्रि-गुणे त्राहि त्रि-वेणि शरणागतम्}
प्रार्थनां समर्पयामि ॥

\end{center}

\fourlineindentedshloka*
{कायेन वाचा मनसेन्द्रियैर्वा}
{बुद्‌ध्याऽऽत्मना वा प्रकृतेः स्वभावात्}
{करोमि यद्यत् सकलं परस्मै}
{नारायणायेति समर्पयामि}

\medskip

% \centerline{\textbf{अनेन पूजनेन गङ्गा-देवी प्रीयताम्।}}

\medskip

\centerline{ॐ तत्सद्ब्रह्मार्पणमस्तु।}

\centerline{॥इति गङ्गा-पूजा-कल्पः सम्पूर्णः॥}

\input{../stotra-sangrahah/stotras/nadi/Gangashtakam.tex}

\closesection