% !TeX program = XeLaTeX
% !TeX root = ..\pujavidhanam.tex
\chapt{श्री-कावेरी-पूजा}

\centerline{\small{(मूलम्—व्रतचूडामणिः)}}

\dnsub{पूर्वाङ्ग-विघ्नेश्वर-पूजा}

\graphicspath{{purvanga/}{../purvanga/}}

\centerline{\includegraphics[width=1cm]{ganesha.pdf}}

(आचम्य)

\twolineshloka*
{शुक्लाम्बरधरं विष्णुं शशिवर्णं चतुर्भुजम्}
{प्रसन्नवदनं ध्यायेत् सर्वविघ्नोपशान्तये}
 
प्राणान्  आयम्य।

(अप उपस्पृश्य, पुष्पाक्षतान् गृहीत्वा)\\

\twolineshloka*
{तदेव लग्नं सुदिनं तदेव ताराबलं चन्द्रबलं तदेव}
{विद्याबलं दैवबलं तदेव लक्ष्मीपतेरङ्घ्रियुगं स्मरामि}
 
ममोपात्त-समस्त-दुरित-क्षयद्वारा \\
श्री-परमेश्वर-प्रीत्यर्थं करिष्यमाणस्य कर्मणः\\
अविघ्नेन परिसमाप्त्यर्थम् आदौ विघ्नेश्वरपूजां करिष्ये।

(अप उपस्पृश्य)

\ifbool{veda}{
\twolineshloka*
{ॐ ग॒णानां᳚ त्वा ग॒णप॑तिꣳ हवामहे क॒विं क॑वी॒नामु॑प॒मश्र॑वस्तमम्}
{ज्ये॒ष्ठ॒राजं॒ ब्रह्म॑णां ब्रह्मणस्पत॒ आ नः॑ शृ॒ण्वन्नू॒तिभिः॑ सीद॒ साद॑नम्}
}{}

\twolineshloka*
{अगजानानपद्मार्कं गजाननमहर्निशम्}
{अनेकदं तं भक्तानाम् एकदन्तमुपास्महे}

\ifbool{veda}{भूर्भुवः॒ सुव॒रोम्।}{} अस्मिन् हरिद्राबिम्बे सुमुखं महागणपतिं ध्यायामि, आवाहयामि।\\

\renewcommand{\devAya}{\OMshri महागणपतये}

\devAya{} नमः, आसनं समर्पयामि।\\
\devAya{} नमः, पादयोः पाद्यं समर्पयामि।\\
\devAya{} नमः, अर्घ्यं समर्पयामि।\\
\devAya{} नमः, आचमनीयं समर्पयामि।\\
\devAya{} नमः, मधुपर्कं समर्पयामि।\\
\ifbool{veda}{ॐ भूर्भुवः॒ सुवः॑।}{}
\devAya{} नमः, शुद्धोदकस्नानं समर्पयामि। स्नानानन्तरमाचमनीयं समर्पयामि।\\
\devAya{} नमः, वस्त्रार्थमक्षतान् समर्पयामि।\\
\devAya{} नमः, यज्ञोपवीताभरणार्थे अक्षतान् समर्पयामि।\\
\devAya{} नमः, दिव्यपरिमलगन्धान् धारयामि। गन्धस्योपरि हरिद्राकुङ्कुमं समर्पयामि। \\
\devAya{} नमः, अक्षतान् समर्पयामि। \\
\devAya{} नमः, पुष्पमालिकां समर्पयामि। पुष्पैः पूजयामि।

\dnsub{अर्चना}
\begin{enumerate}
\begin{minipage}{0.475\linewidth}   
  \item सुमुखाय नमः
  \item एकदन्ताय नमः
  \item कपिलाय नमः
  \item गजकर्णकाय नमः
  \item लम्बोदराय नमः
  \item विकटाय नमः
  \item विघ्नराजाय नमः
  \item विनायकाय नमः
\end{minipage}
\begin{minipage}{0.525\linewidth}
  \item धूमकेतवे नमः
  \item गणाध्यक्षाय नमः
  \item फालचन्द्राय नमः
  \item गजाननाय नमः
  \item वक्रतुण्डाय नमः
  \item शूर्पकर्णाय नमः
  \item हेरम्बाय नमः
  \item स्कन्दपूर्वजाय नमः
\end{minipage}
\end{enumerate}
\devAya{} नमः, नानाविधपरिमलपत्रपुष्पाणि समर्पयामि॥\\
\devAya{} नमः, धूपमाघ्रापयामि।\\
अलङ्कारदीपं सन्दर्शयामि।\\
% नैवेद्यम्।\\
\ifbool{veda}{ॐ भूर्भुवः॒ सुवः॑। + ब्र॒ह्मणे॒ स्वाहा᳚।}{}
\devAya{} नमः, \blank{} (नैवेद्यं) निवेदयामि। 
\ifbool{veda}{अ॒मृ॒ता॒पि॒धा॒नम॑सि॥}{}
निवेदनान्तरम् आचमनीयं समर्पयामि।\\
\devAya{} नमः, ताम्बूलं समर्पयामि।\\
\devAya{} नमः, कर्पूरनीराजनं दर्शयामि। कर्पूरनीराजनानन्तरमाचमनीयं समर्पयामि।\\
\devAya{} नमः, मन्त्रपुष्पं समर्पयामि। स्वर्णपुष्पं समर्पयामि।\\

\twolineshloka*
{अभीप्सितार्थसिद्ध्यर्थं पूजितो यः सुरैरपि}
{सर्वविघ्नच्छिदे तस्मै गणाधिपतये नमः}

\twolineshloka*
{गजाननं भूतगणादिसेवितं कपित्थ-जम्बूफल-सार-भक्षितम्}
{उमासुतं शोकविनाशकारणं नमामि विघ्नेश्वरपादपङ्कजम्}

\centerline{अनन्तकोटिप्रदक्षिणनमस्कारान् समर्पयामि।}

\centerline{छत्त्रचामरादिसमस्तोपचारान् समर्पयामि।}

\twolineshloka*
{वक्रतुण्डमहाकाय कोटिसूर्यसमप्रभ}
{अविघ्नं कुरु मे देव सर्वकार्येषु सर्वदा}

\twolineshloka*
{सुमुखश्चैकदन्तश्च कपिलो गजकर्णकः}
{लम्बोदरश्च विकटो विघ्नराजो गणाधिपः}

\twolineshloka*
{धूमकेतुर्गणाध्यक्षो फालचन्द्रो गजाननः}
{वक्रतुण्डः शूर्पकर्णो हेरम्भः स्कन्दपूर्वजः}

\threelineshloka*
{षोडशैतानि नामानि यः पठेच्छृणुयादपि}
{विद्यारम्भे विवाहे च प्रवेशे निर्गमे तथा}
{सङ्ग्रामे सर्वकार्येषु विघ्नस्तस्य न जायते}

\centerline{प्रार्थनाः समर्पयामि।}

\closesub

\sect{षोडशोपचार-पूजा}

\twolineshloka*
{शुक्लाम्बरधरं विष्णुं शशिवर्णं चतुर्भुजम्}
{प्रसन्नवदनं ध्यायेत् सर्वविघ्नोपशान्तये}

प्राणान् आयम्य। ॐ भूः + भूर्भुवः॒ सुव॒रोम्।

\dnsub{सङ्कल्पः}

ममोपात्त-समस्त-दुरित-क्षयद्वारा श्री-परमेश्वर-प्रीत्यर्थं शुभे शोभने मुहूर्ते अद्य ब्रह्मणः
द्वितीयपरार्धे श्वेतवराहकल्पे वैवस्वतमन्वन्तरे अष्टाविंशतितमे कलियुगे प्रथमे पादे
जम्बूद्वीपे भारतवर्षे भरतखण्डे मेरोः दक्षिणे पार्श्वे शकाब्दे अस्मिन् वर्तमाने व्यावहारिकाणां
प्रभवादीनां षष्टिसंवत्सराणां मध्ये \blank\see{app:samvatsara_names} नाम संवत्सरे उत्तरायणे / दक्षिणायने
(ग्रीष्म / वर्ष / शरद् / हेमन्त / शिशिर / वसन्त)-ऋतौ (मेष / वृषभ / मिथुन / कर्कटक / सिंह / कन्या / तुला /
वृश्चिक / धनुर् / मकर / कुम्भ / मीन)-मासे (शुक्ल / कृष्ण)-पक्षे \blank{} शुभतिथौ
(इन्दु / भौम / सौम्य / गुरु / भृगु / स्थिर / भानु)-वासरयुक्तायाम्
\blank\see{app:nakshatra_names}-नक्षत्र \blank\see{app:yoga_names}-योग \blank-करण युक्तायां च एवं गुण विशेषण विशिष्टायाम्
अस्यां \blank{} शुभतिथौ
अस्माकं सहकुटुम्बानां क्षेमस्थैर्य-धैर्य-वीर्य-विजय-आयुरारोग्य-ऐश्वर्याभिवृद्ध्यर्थम्
धर्मार्थकाममोक्ष\-चतुर्विधफलपुरुषार्थसिद्ध्यर्थं पुत्रपौत्राभि\-वृद्ध्यर्थम् इष्टकाम्यार्थसिद्ध्यर्थम्
मम इहजन्मनि पूर्वजन्मनि जन्मान्तरे च सम्पादितानां ज्ञानाज्ञानकृतमहा\-पातकचतुष्टय-व्यतिरिक्तानां रहस्यकृतानां प्रकाशकृतानां सर्वेषां पापानां सद्य अपनोदनद्वारा सकल-पापक्षयार्थं
कावेरी-देवी-प्रीत्यर्थं यावच्छक्ति-ध्यानावाहनादि-षोडशोपचार-कावेरी-पूजां करिष्ये तदङ्गं कलशपूजां च करिष्ये।

श्रीविघ्नेश्वराय नमः यथास्थानं प्रतिष्ठापयामि।\\
(गणपति-प्रसादं शिरसा गृहीत्वा)

\dnsub{आसन-पूजा}
\centerline{पृथिव्या  मेरुपृष्ठ  ऋषिः।  सुतलं  छन्दः।  कूर्मो  देवता॥}
\twolineshloka*
{पृथ्वि  त्वया  धृता  लोका  देवि  त्वं  विष्णुना  धृता}
{त्वं  च  धारय  मां  देवि  पवित्रं  चाऽऽसनं  कुरु}


\dnsub{घण्टा-पूजा}

\twolineshloka*
{आगमार्थं तु देवानां गमनार्थं तु रक्षसाम्}
{घण्टारवं करोम्यादौ देवताऽऽह्वानकारणम्}


\dnsub{कलशपूजा}
ॐ कलशाय नमः दिव्यगन्धान् धारयामि।\\
ॐ गङ्गायै नमः। ॐ यमुनायै नमः। ॐ गोदावर्यै नमः।  ॐ सरस्वत्यै नमः। ॐ नर्मदायै नमः। ॐ सिन्धवे नमः। ॐ कावेर्यै नमः।\\
ॐ सप्तकोटिमहातीर्थान्यावाहयामि।\\[-0.25ex]

(अथ कलशं स्पृष्ट्वा जपं कुर्यात्) \\
आपो॒ वा इ॒द सर्वं॒ विश्वा॑ भू॒तान्याप॑ प्रा॒णा वा आप॑ प॒शव॒ आपो\-ऽन्न॒मापोऽमृ॑त॒माप॑ स॒म्राडापो॑ वि॒राडाप॑ स्व॒राडाप॒श्\-छन्दा॒स्यापो॒ ज्योती॒ष्यापो॒ यजू॒ष्याप॑ स॒त्यमाप॒ सर्वा॑ दे॒वता॒ आपो॒ भूर्भुव॒ सुव॒राप॒ ओम्॥\\

\twolineshloka* 
{कलशस्य मुखे विष्णुः कण्ठे रुद्रः समाश्रितः}
{मूले तत्र स्थितो ब्रह्मा मध्ये मातृगणाः स्मृताः}
\threelineshloka* 
{कुक्षौ तु सागराः सर्वे सप्तद्वीपा वसुन्धरा}
{ऋग्वेदोऽथ यजुर्वेदः सामवेदोऽप्यथर्वणः}
{अङ्गैश्च सहिताः सर्वे कलशाम्बुसमाश्रिताः}
\twolineshloka* 
{गङ्गे च यमुने चैव गोदावरि सरस्वति}
{नर्मदे सिन्धुकावेरि जलेऽस्मिन् सन्निधिं कुरु}
\twolineshloka*
{सर्वे समुद्राः सरितः तीर्थानि च ह्रदा नदाः}
{आयान्तु देवपूजार्थं दुरितक्षयकारकाः}

\centerline{ॐ भूर्भुवः॒ सुवो॒ भूर्भुवः॒ सुवो॒ भूर्भुवः॒ सुवः॑।}

(इति कलशजलेन सर्वोपकरणानि आत्मानं च प्रोक्ष्य।)


\dnsub{आत्मपूजा}
ॐ आत्मने नमः, दिव्यगन्धान् धारयामि।
\begin{multicols}{2}
१. ॐ आत्मने नमः\\
२. ॐ अन्तरात्मने नमः\\
३. ॐ योगात्मने नमः\\
४. ॐ जीवात्मने नमः\\
५. ॐ परमात्मने नमः\\
६. ॐ ज्ञानात्मने नमः
\end{multicols}
समस्तोपचारान् समर्पयामि।

\twolineshloka*
{देहो देवालयः प्रोक्तो जीवो देवः सनातनः}
{त्यजेदज्ञाननिर्माल्यं सोऽहं भावेन पूजयेत्}


\begin{minipage}{\linewidth}
\dnsub{पीठ-पूजा}

\begin{multicols}{2}
\begin{enumerate}
\item आधारशक्त्यै नमः
\item मूलप्रकृत्यै नमः
\item आदिकूर्माय नमः 
\item आदिवराहाय नमः
\item अनन्ताय नमः
\item पृथिव्यै नमः
\item रत्नमण्डपाय नमः
\item रत्नवेदिकायै नमः
\item स्वर्णस्तम्भाय नमः
\item श्वेतच्छत्त्राय नमः
\item कल्पकवृक्षाय नमः
\item क्षीरसमुद्राय नमः 
\item सितचामराभ्यां नमः
\item योगपीठासनाय नमः
\end{enumerate}
\end{multicols}

\end{minipage}

\dnsub{गुरु-ध्यानम्}

\twolineshloka*
{गुरुर्ब्रह्मा गुरुर्विष्णुर्गुरुर्देवो महेश्वरः}
{गुरुः साक्षात् परं ब्रह्म तस्मै श्री-गुरवे नमः}


\sect{षोडशोपचार-पूजा}
\renewcommand{\devAya}{कावेरी-देव्यै नमः,}

\begin{center}

\begin{center}

\fourlineindentedshloka
{अच्छस्वच्छलसद्दुकूलवसनां पद्मासनाद्‍ध्यासिनीं}
{हस्तन्यस्तवराभयाब्जकलशां राकेन्दुकोटिप्रभाम्}
{भास्वद्भूषणगन्धमाल्यरुचिरां चारुप्रसन्नाननां}
{श्री-गङ्गादिसमस्ततीर्थनिलयां ध्यायामि कावेरिकाम्}

\textbf{कावेरी-देवीं} ध्यायामि॥

\fourlineindentedshloka*
{श्रीकण्ठविश्वेश्वरसन्निभानि }
{लिङ्गानि यद्रोधसि लक्षकोट्यः}
{जलप्रवाहेपि च कोटिकोट्यः}
{कवेरजायाश्शिवमूर्तयस्स्युः}

\fourlineindentedshloka*
{पयांसि तीर्थानि शिलाश्च देवता}
{दिवौकसो वालुकतां प्रपन्नाः}
{अतो नदी सह्यगिरिप्रसूता}
{सरित्सुमुख्यामनुजैरलभ्या}

\textbf{कावेरी-देवीम्} आवाहयामि॥

\fourlineindentedshloka*
{आदावादि विधातृमानससुता पश्चात्कवेरात्मजा}
{भूयःकुम्भभवस्य तस्य दयिता तस्मान्नदीरूपिणी}
{श्रीरङ्गातुलकुम्भघोणविलसन्मायूरमध्यार्जुन-}
{श्वेतारण्यमुखस्थलेषु महिता सह्याद्रिजा दृश्यते}

\textbf{कावेरी-देव्यै नमः}, आसनं समर्पयामि॥

\twolineshloka*
{मरुद्वृधे महादेवी महाभागे मनोहरे}
{सर्वाभीष्टप्रदे लोकमातः पाद्यं ददामि ते}

\twolineshloka*
{मरुद्वृधे मान्यजलप्रवाहे कवेरकन्ये नमतां शरण्ये}
{मान्ये जगत्पूज्यतमप्रभावे कावेरि कावेरि मम प्रसीद}

\textbf{कावेरी-देव्यै नमः}, पाद्यं समर्पयामि॥


\twolineshloka*
{सह्यपादोद्भवे देवि श्रीरङ्गोत्सङ्गामिनि}
{कावेरीनामविख्याते गृहाणार्घ्यं नमोऽस्तुते}

\twolineshloka*
{प्राचीनवाक्कीर्तित-पुण्यकीर्ते कल्याणि भक्तेप्सितदानधुर्ये}
{कुम्भोद्भवप्रेरितगुम्भितार्थे कावेरि कावेरि मम प्रसीद}

\textbf{कावेरी-देव्यै नमः}, अर्घ्यं समर्पयामि॥

\twolineshloka*
{श्रीसह्यशैलतनये सर्वासह्यनिवारिणि}
{प्रसादं कुरु मे देवि प्रसन्ना भव सर्वदा}

\fourlineindentedshloka*
{संसारविस्रंसिनि संस्कृतानां}
{सर्वाघसंहारिणि सर्ववन्द्ये}
{समस्तलोकैकशरण्यमूर्ते}
{कावेरि कावेरि मम प्रसीद}

\textbf{कावेरी-देव्यै नमः}, आचमनीयं समर्पयामि॥

\twolineshloka*
{तुलामासे तु कावेरी सर्वतीर्थाश्रिता नदी}
{पञ्च-पातक-संहर्त्री वाजिमेधफलप्रदा}

\fourlineindentedshloka*
{भक्तानुकम्पे मुनिभाग्यलक्ष्मि}
{नित्ये जगन्मङ्गलदानशीले}
{निरञ्जने दक्षिणदेशगङ्गे}
{कावेरि कावेरि मम प्रसीद}

\textbf{कावेरी-देव्यै नमः}, पञ्चामृतस्नानं समर्पयामि॥

\twolineshloka*
{कावेरीतीरजन्मानः मृगपक्षिमहीरुहाः}
{तद्वारिशीतवातैश्च स्पृष्टा मुक्तिं प्रयान्ति वै}

\fourlineindentedshloka*
{मोक्षश्रियोपासित-पादपद्मे}
{नित्ये हरीशद्रुहिणस्वरूपे}
{सदाशिवे धातृवरप्रसादे}
{कावेरि कावेरि मम प्रसीद}

\textbf{कावेरी-देव्यै नमः}, शुद्धोदकस्नानं समर्पयामि॥
स्नानानन्तरम् आचमनीयं समर्पयामि॥

\twolineshloka*
{कवेरकन्ये कावेरि निम्नगानाथनायिके}
{वस्त्राणि वसभूवस्त्रे भुक्तिमुक्तिप्रदायिनि}

\fourlineindentedshloka*
{देवर्षिपूज्ये विमले नदीशे}
{परात्परे भावितनित्यपूर्णे}
{समस्तलोकोत्तमतीर्थमातः}
{कावेरि कावेरि मम प्रसीद}

\textbf{कावेरी-देव्यै नमः}, वस्त्रं समर्पयामि॥

\twolineshloka*
{सर्वयज्ञाश्रयतटे सर्वयज्ञसहायिनि}
{यज्ञाङ्गसम्भवे देवि सर्वयज्ञफलप्रदे}

\fourlineindentedshloka*
{कलिप्रभूताखिलदोषनाशे}
{विशुद्धविज्ञानजलप्रवाहे}
{कदम्बकल्हारकदम्बपूर्णे}
{कावेरि कावेरि मम प्रसीद}

\textbf{कावेरी-देव्यै नमः}, यज्ञोपवीतं समर्पयामि॥

\twolineshloka*
{जयदेवि जगन्मातर्लोपामुद्रे पुरातने}
{जयभद्रे भवोद्धारि मङ्गले मङ्गलप्रदे}

\fourlineindentedshloka*
{प्रसीद कारुण्य-गुणाभिरामे}
{प्रसीद कल्याणतरप्रवाहे}
{प्रसीद कामादिहरे पवित्रे}
{कावेरि कावेरि मम प्रसीद}

\textbf{कावेरी-देव्यै नमः}, मङ्गलद्रव्यं समर्पयामि। हरिद्रा-कुङ्कुमं समर्पयामि।

\twolineshloka*
{चन्दनागरुकस्तूरीहिमवालुककेसरैः}
{राङ्कवैस्साङ्कवैर्युक्तं गन्धं स्वीकुरु सह्यजे}

\fourlineindentedshloka*
{आद्ये परे चिन्मयपुण्यपादे}
{पचेलिमप्रौढकवेरभाग्ये}
{अनन्यसाधारणवैभवाढ्ये}
{कावेरि कावेरि मम प्रसीद}

\textbf{कावेरी-देव्यै नमः}, गन्धान् धारयामि॥

\twolineshloka*
{यस्यां सकृत्स्नानमात्रान्नरोऽक्षय्यफलं लभेत्}
{नक्षत्रमाल्यवच्छुभ्रामक्षतैरर्चयाम्यहम्}

\textbf{कावेरी-देव्यै नमः}, अक्षतान् समर्पयामि॥

\twolineshloka*
{किरीटहारकेयूरकुण्डलाङ्गदकङ्कणैः}
{हंसकैर्मेखलाद्यैश्च भूषये त्वां मरुद्वृधाम्}
\textbf{कावेरी-देव्यै नमः}, आभरणानि समर्पयामि॥

\twolineshloka*
{ताटङ्कं कण्ठसूत्रं च सिन्दूरं कज्जलादिकम्}
{सौमङ्गल्यप्रदे देवि गृहाणागस्त्यवल्लभे}

\textbf{कावेरी-देव्यै नमः}, ताटङ्कादिकान् समर्पयामि॥

\twolineshloka*
{तुलसीबिल्वमन्दारकुशाग्रशतपत्रकैः}
{इन्दीवरैः कोकनदैर्हल्लकैः कमलैरपि}

\textbf{कावेरी-देव्यै नमः}, पुष्पमालां समर्पयामि॥

\twolineshloka*
{कल्हारैः पुण्डरीकैश्च पुष्पैः सौगन्धिकादिभिः}
{जाती-चम्पक-पुन्नाग-मल्लिका-केतकादिभिः}

\twolineshloka*
{सुरभिद्रोणवासन्ती गन्धराज-कदम्बकैः}
{पुष्पैः सह्यसुते मातर्वेण्यलङ्करणं कुरु}

\twolineshloka*
{बहुजन्मकृतानेकवासनावासितात्मभिः}
{दिव्यैः सुमवरैः पूज्यां पूजयेत् पुष्पजातिभिः}

\textbf{कावेरी-देव्यै नमः}, पुष्पैः सम्पूजयामि।

\dnsub{अङ्ग-पूजा}

\begin{supertabular}{ll}
मरुद्वृधायै नमः & पादौ पूजयामि\\
महालक्ष्म्यै नमः & गुल्फौ पूजयामि\\
सह्यकन्यकायै नमः & जङ्घे पूजयामि\\
सरस्वत्यै नमः & जानुनी पूजयामि\\
अगस्त्यपत्न्यै नमः & मध्यम् पूजयामि\\
कावेर्यै नमः & नाभिम् पूजयामि\\
लोपामुद्रायै नमः & हृदयम् पूजयामि\\
वरप्रदायै नमः & स्तनौ पूजयामि\\
कमण्डलुसमुत्पन्नयै नमः & बाहू पूजयामि\\
सर्वतीर्थाधिदेवतायै नमः & कण्ठम् पूजयामि\\
विरजायै नमः & नासिकाम् पूजयामि\\
दक्षिणगङ्गायै नमः & श्रोत्रे पूजयामि\\
ब्रह्मविष्णुशिवात्मिकायै नमः & नेत्रे पूजयामि\\
चतुर्विधफलोद्धात्र्यै नमः & वक्त्रम् पूजयामि\\
चतुरानन कन्यकायै नमः & शिरः पूजयामि\\
सर्वाभीष्टप्रदात्र्यै नमः & सर्वाण्यङ्गानि पूजयामि\\
\end{supertabular}

\end{center}

\begingroup
\centering
\setlength{\columnseprule}{1pt}
\let\chapt\sect
\input{../namavali-manjari/100/Kaveri_108.tex}

\endgroup

\begin{center}
\textbf{कावेरी-देव्यै नमः}, नाना-विध-परिमल-पत्र-पुष्पाणि समर्पयामि॥

\sect{उत्तराङ्गपूजा}

\fourlineindentedshloka*
{रङ्गत्रयोत्सङ्गविराजमाने}
{ब्रह्माद्रिकूटाश्रिततीर्थगर्भे}
{समस्तसिद्धाश्रमरम्यतीर्थे}
{कावेरि कावेरि मम प्रसीद}

\twolineshloka*
{दशाङ्गो गुग्गुलोपेतस्सुगन्धो घ्राणतर्पणः}
{मरुद्वृधेऽम्बकावेरि धूपोयन्ते समर्पितः}

\textbf{कावेरी-देव्यै नमः}, धूपम् आघ्रापयामि॥

\fourlineindentedshloka*
{तुलागतेऽर्केतटिनि वराभिर्-}
{गङ्गादिभिः सेवितपादपद्मे}
{त्रिकोटितीर्थाश्रम-पुष्कराढ्ये}
{कावेरि कावेरि मम प्रसीद}

\twolineshloka*
{आपो ज्योतिस्त्वमस्यम्ब कुम्भसम्भवतेजसा}
{तेजःप्रदात्रि कावेरि दीपोऽयं प्रतिगृह्यताम्}

\textbf{कावेरी-देव्यै नमः}, दीपं दर्शयामि॥

\fourlineindentedshloka*
{ददासि तुल्याभिरनर्घधान्या-}
{न्यशेषचोलीयजनस्य देवि}
{भुक्तिप्रदे मुक्तिविधानदक्षे}
{कावेरि कावेरि मम प्रसीद}

\twolineshloka*
{अन्नं चतुर्विधं रुच्यं शाकव्यञ्जनसंयुतम्}
{सफलं सघृतं भुङ्क्ष्व कावेरि तटितां वरे}

\textbf{कावेरी-देव्यै नमः}, नैवेद्यं समर्पयामि॥ पानीयं समर्पयामि। उत्तरापोशनं समर्पयामि। निवेदनान्तरम् आचमनीयं समर्पयामि।

\fourlineindentedshloka*
{निजाभिधानश्रवणादृतानां}
{निश्शेषपापक्षयकारिणीत्वम्}
{निदाघतृष्णादि-निरासदक्षे}
{कावेरि कावेरि मम प्रसीद}

\textbf{कावेरी-देव्यै नमः}, फलं कर्पूर-ताम्बूलं च समर्पयामि॥

\fourlineindentedshloka*
{पराशरागस्त्यवसिष्ठमुख्यैर्-}
{महर्षिभिर्मान्यतपोभिरम्ब}
{त्वमाश्रिताकर्मठसिद्धिहेतुः}
{कावेरि कावेरि मम प्रसीद}

\twolineshloka*
{सर्वतीर्थैस्तुलामासे त्वन्नीराजनपादुका}
{नीराजयामि भक्त्या त्वां कावेर्यम्ब मरुद्वृधे}

\textbf{कावेरी-देव्यै नमः}, नीराजनं समर्पयामि॥

\fourlineindentedshloka*
{मायूरमध्यार्जुनकुम्भघोणश्-}
{रीरङ्गचुञ्चुस्थलघट्टगा त्वम्}
{सस्योद्भवे सिन्धुयुते स्वतन्त्रे}
{कावेरि कावेरि मम प्रसीद}

\twolineshloka*
{प्रदक्षिणं करोमि त्वां पुरुषार्थप्रदायिनि}
{दक्षिणावर्तकश्रीनिवासप्रीते मरुद्वृधे}

\textbf{कावेरी-देव्यै नमः}, प्रदक्षिणम् समर्पयामि॥

\fourlineindentedshloka*
{ब्रह्मादिभिर्देववरैस्त्रिसन्ध्यं}
{वसिष्ठपूर्वैर्मुनिभिर्वरिष्ठैः}
{नित्यं गजारण्यगतैर्निषेव्ये}
{कावेरि कावेरि मम प्रसीद}

\twolineshloka*
{नमस्ते तटितां मुख्ये निगमागमसंस्तुते}
{पाहि पाह्यम्ब कावेरि प्रपन्नं मां कृपादृशा}

\textbf{कावेरी-देव्यै नमः}, नमस्कारान् समर्पयामि॥

\dnsub{अर्घ्यप्रदानम्}

%\begin{center}

    \twolineshloka
    {मरुद्वृधे महाभागे सर्वलोकैकपावनि}
    {गृहाणार्घ्यं मया दत्तं पावनं कुरु मां सदा}
    कावेर्यै नमः - इदमर्घ्यम् इदमर्घ्यम् इदमर्घ्यम्॥

    \twolineshloka
    {विष्णुमाये महाकाये कवेरकुलसम्भवे}
    {सह्याचलसमुद्भूते गृहाणार्घ्यं वरप्रदे}
    कावेर्यै नमः - इदमर्घ्यम् इदमर्घ्यम् इदमर्घ्यम्॥

    \twolineshloka
    {कुम्भसम्भवकुम्भात् त्वं सह्यामलकपूजया}
    {शङ्खोदकेन सञ्जाता गृहाणार्घ्यं समुद्रगे}
    कावेर्यै नमः - इदमर्घ्यम् इदमर्घ्यम् इदमर्घ्यम्॥

%\end{center}

\centerline{प्रार्थना}
\resetShloka

\twolineshloka
{मरुद्वृधे महालक्ष्मीस्सह्यकन्या सरस्वती}
{अगस्त्यपत्नी कावेरी लोपामुद्रा वरप्रदा}

\twolineshloka
{कमण्डलुसमुत्पन्ना सर्वतीर्थाधिदेवता}
{विरजा दक्षिणा गङ्गा ब्रह्मविष्णुशिवात्मिका}

\threelineshloka
{चतुर्विधफलोद्धात्री चतुराननकन्यका}
{सर्वाभीष्टप्रदात्री च नाम्नां षोडशकं स्मृतम्}
{एभिर्नामपदैर्नित्यं पूजयेद्भक्तिमान्नरः}

\fourlineindentedshloka*
{सुवासिनीनां च निजाश्रितानां}
{पतिप्रियत्वं च सुतादिवर्गम्}
{दास्यस्युदारं बहुभोगभाग्यं}
{कावेरि कावेरि मम प्रसीद}

\fourlineindentedshloka*
{निजप्रवाहाप्रवलब्धपुण्य-}
{प्रसिद्धसत्कर्मफलोदयेन}
{नृणामभीष्टार्थविधायिनी त्वं}
{कावेरि कावेरि मम प्रसीद}

\twolineshloka*
{पापक्षयं पारिशुध्यमायुरारोग्यमेव च}
{सौभाग्यमपि सन्तानं ज्ञानं देहि मरुद्वृधे}

इति सम्प्रार्थ्य॥

\centerline{\textbf{सुवासिनीभ्यः वायनदानानि}}

\end{center}

\centerline {॥इति व्रतचूडामणौ कावेरी-पूजाविधिः॥}

\medskip

\centerline{ॐ तत् सद् ब्रह्मार्पणमस्तु।}

\closesection

\end{center}