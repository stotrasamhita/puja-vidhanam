% !TeX program = XeLaTeX
% !TeX root = pUjA.tex

\setlength{\parindent}{0pt}

\chapt{श्रीरामनवमी-पूजा}

\dnsub{पूर्वाङ्ग-विघ्नेश्वर-पूजा}

\graphicspath{{purvanga/}{../purvanga/}}

\centerline{\includegraphics[width=1cm]{ganesha.pdf}}

(आचम्य)

\twolineshloka*
{शुक्लाम्बरधरं विष्णुं शशिवर्णं चतुर्भुजम्}
{प्रसन्नवदनं ध्यायेत् सर्वविघ्नोपशान्तये}
 
प्राणान्  आयम्य।

(अप उपस्पृश्य, पुष्पाक्षतान् गृहीत्वा)\\

\twolineshloka*
{तदेव लग्नं सुदिनं तदेव ताराबलं चन्द्रबलं तदेव}
{विद्याबलं दैवबलं तदेव लक्ष्मीपतेरङ्घ्रियुगं स्मरामि}
 
ममोपात्त-समस्त-दुरित-क्षयद्वारा \\
श्री-परमेश्वर-प्रीत्यर्थं करिष्यमाणस्य कर्मणः\\
अविघ्नेन परिसमाप्त्यर्थम् आदौ विघ्नेश्वरपूजां करिष्ये।

(अप उपस्पृश्य)

\ifbool{veda}{
\twolineshloka*
{ॐ ग॒णानां᳚ त्वा ग॒णप॑तिꣳ हवामहे क॒विं क॑वी॒नामु॑प॒मश्र॑वस्तमम्}
{ज्ये॒ष्ठ॒राजं॒ ब्रह्म॑णां ब्रह्मणस्पत॒ आ नः॑ शृ॒ण्वन्नू॒तिभिः॑ सीद॒ साद॑नम्}
}{}

\twolineshloka*
{अगजानानपद्मार्कं गजाननमहर्निशम्}
{अनेकदं तं भक्तानाम् एकदन्तमुपास्महे}

\ifbool{veda}{भूर्भुवः॒ सुव॒रोम्।}{} अस्मिन् हरिद्राबिम्बे सुमुखं महागणपतिं ध्यायामि, आवाहयामि।\\

\renewcommand{\devAya}{\OMshri महागणपतये}

\devAya{} नमः, आसनं समर्पयामि।\\
\devAya{} नमः, पादयोः पाद्यं समर्पयामि।\\
\devAya{} नमः, अर्घ्यं समर्पयामि।\\
\devAya{} नमः, आचमनीयं समर्पयामि।\\
\devAya{} नमः, मधुपर्कं समर्पयामि।\\
\ifbool{veda}{ॐ भूर्भुवः॒ सुवः॑।}{}
\devAya{} नमः, शुद्धोदकस्नानं समर्पयामि। स्नानानन्तरमाचमनीयं समर्पयामि।\\
\devAya{} नमः, वस्त्रार्थमक्षतान् समर्पयामि।\\
\devAya{} नमः, यज्ञोपवीताभरणार्थे अक्षतान् समर्पयामि।\\
\devAya{} नमः, दिव्यपरिमलगन्धान् धारयामि। गन्धस्योपरि हरिद्राकुङ्कुमं समर्पयामि। \\
\devAya{} नमः, अक्षतान् समर्पयामि। \\
\devAya{} नमः, पुष्पमालिकां समर्पयामि। पुष्पैः पूजयामि।

\dnsub{अर्चना}
\begin{enumerate}
\begin{minipage}{0.475\linewidth}   
  \item सुमुखाय नमः
  \item एकदन्ताय नमः
  \item कपिलाय नमः
  \item गजकर्णकाय नमः
  \item लम्बोदराय नमः
  \item विकटाय नमः
  \item विघ्नराजाय नमः
  \item विनायकाय नमः
\end{minipage}
\begin{minipage}{0.525\linewidth}
  \item धूमकेतवे नमः
  \item गणाध्यक्षाय नमः
  \item फालचन्द्राय नमः
  \item गजाननाय नमः
  \item वक्रतुण्डाय नमः
  \item शूर्पकर्णाय नमः
  \item हेरम्बाय नमः
  \item स्कन्दपूर्वजाय नमः
\end{minipage}
\end{enumerate}
\devAya{} नमः, नानाविधपरिमलपत्रपुष्पाणि समर्पयामि॥\\
\devAya{} नमः, धूपमाघ्रापयामि।\\
अलङ्कारदीपं सन्दर्शयामि।\\
% नैवेद्यम्।\\
\ifbool{veda}{ॐ भूर्भुवः॒ सुवः॑। + ब्र॒ह्मणे॒ स्वाहा᳚।}{}
\devAya{} नमः, \blank{} (नैवेद्यं) निवेदयामि। 
\ifbool{veda}{अ॒मृ॒ता॒पि॒धा॒नम॑सि॥}{}
निवेदनान्तरम् आचमनीयं समर्पयामि।\\
\devAya{} नमः, ताम्बूलं समर्पयामि।\\
\devAya{} नमः, कर्पूरनीराजनं दर्शयामि। कर्पूरनीराजनानन्तरमाचमनीयं समर्पयामि।\\
\devAya{} नमः, मन्त्रपुष्पं समर्पयामि। स्वर्णपुष्पं समर्पयामि।\\

\twolineshloka*
{अभीप्सितार्थसिद्ध्यर्थं पूजितो यः सुरैरपि}
{सर्वविघ्नच्छिदे तस्मै गणाधिपतये नमः}

\twolineshloka*
{गजाननं भूतगणादिसेवितं कपित्थ-जम्बूफल-सार-भक्षितम्}
{उमासुतं शोकविनाशकारणं नमामि विघ्नेश्वरपादपङ्कजम्}

\centerline{अनन्तकोटिप्रदक्षिणनमस्कारान् समर्पयामि।}

\centerline{छत्त्रचामरादिसमस्तोपचारान् समर्पयामि।}

\twolineshloka*
{वक्रतुण्डमहाकाय कोटिसूर्यसमप्रभ}
{अविघ्नं कुरु मे देव सर्वकार्येषु सर्वदा}

\twolineshloka*
{सुमुखश्चैकदन्तश्च कपिलो गजकर्णकः}
{लम्बोदरश्च विकटो विघ्नराजो गणाधिपः}

\twolineshloka*
{धूमकेतुर्गणाध्यक्षो फालचन्द्रो गजाननः}
{वक्रतुण्डः शूर्पकर्णो हेरम्भः स्कन्दपूर्वजः}

\threelineshloka*
{षोडशैतानि नामानि यः पठेच्छृणुयादपि}
{विद्यारम्भे विवाहे च प्रवेशे निर्गमे तथा}
{सङ्ग्रामे सर्वकार्येषु विघ्नस्तस्य न जायते}

\centerline{प्रार्थनाः समर्पयामि।}

\closesub

\sect{प्रधान-पूजा - श्रीराम-पूजा}

\twolineshloka*
{शुक्लाम्बरधरं विष्णुं शशिवर्णं चतुर्भुजम्}
{प्रसन्नवदनं ध्यायेत् सर्वविघ्नोपशान्तये}
 
प्राणान्  आयम्य।  ॐ भूः + भूर्भुवः॒ सुव॒रोम्।

\dnsub{सङ्कल्पः}

ममोपात्तसमस्तदुरितक्षयद्वारा श्रीपरमेश्वरप्रीत्यर्थं शुभे शोभने मुहूर्ते अद्यब्रह्मणः
द्वितीयपरार्द्धे श्वेतवराहकल्पे वैवस्वतमन्वन्तरे अष्टाविंशतितमे कलियुगे प्रथमे पादे
जम्बूद्वीपे भारतवर्षे भरतखण्डे मेरोः दक्षिणेपार्श्वे शकाब्दे अस्मिन् वर्तमाने व्यावहारिके
 प्रभवादि षष्टिसंवत्सराणां मध्ये (	) नाम संवत्सरे उत्तरायणे वसन्त-ऋतौ  (मेष/मीन) मासे 
शुक्लपक्षे नवम्यां शुभतिथौ (इन्दु/भौम/बुध/गुरु/भृगु /स्थिर/भानु) वासरयुक्तायाम्
(आर्द्रा/पुनर्वसू/पुष्य) नक्षत्रयुक्तायां ()-योग ()-करण-युक्तायां च एवं गुण-विशेषण-विशिष्टायाम्
अस्याम् नवम्यां शुभतिथौ अस्माकं सहकुटुम्बानां क्षेमस्थैर्य-धैर्य-वीर्य-विजय आयुरारोग्य ऐश्वर्याभिवृद्ध्यर्थम्
 धर्मार्थकाममोक्ष\-चतुर्विधफलपुरुषार्थसिद्ध्यर्थं पुत्रपौत्राभिवृद्ध्यर्थम् इष्टकाम्यार्थसिद्ध्यर्थम्
मम इहजन्मनि पूर्वजन्मनि जन्मान्तरे च सम्पादितानां ज्ञानाज्ञानकृतमहा\-पातकचतुष्टय
व्यतिरिक्तानां रहस्यकृतानां प्रकाशकृतानां सर्वेषां पापानां सद्य अपनोदनद्वारा सकल 
पापक्षयार्थं 
श्रीसीतालक्ष्मणभरतशत्रुघ्नहनुमत्समेत श्रीरामचन्द्रप्रीत्यर्थं
श्रीरामनवमीपुण्यकाले कल्पोक्तप्रकारेण यथाशक्ति श्रीरामचन्द्रपूजां
करिष्ये
तदङ्गं कलशपूजां च करिष्ये।


श्रीविघ्नेश्वराय नमः यथास्थानं प्रतिष्ठापयामि।
(गणपति प्रसादं शिरसा गृहीत्वा)

\dnsub{घण्टापूजा}

\twolineshloka*
{आगमार्थं तु देवानां गमनार्थं तु रक्षसाम्}
{घण्टारवं करोम्यादौ देवताऽऽह्वानकारणम्}

\dnsub{कलशपूजा}
ॐ कलशाय नमः दिव्यगन्धान् धारयामि।\\
ॐ गङ्गायै नमः, ॐ यमुनायै नमः, ॐ गोदावर्यै नमः,  ॐ सरस्वत्यै नमः,\\ ॐ नर्मदायै नमः, ॐ सिन्धवे नमः, ॐ कावेर्यै नमः,\\
 ॐ सप्तकोटिमहातीर्थान्यावाहयामि। \\

(अथ कलशं स्पृष्ट्वा जपं कुर्यात्) \\
आपो॒ वा इ॒द सर्वं॒ विश्वा॑ भू॒तान्याप॑ प्रा॒णा वा आप॑ प॒शव॒ आपोऽन्न॒मापोऽमृ॑त॒माप॑ स॒म्राडापो॑ वि॒राडाप॑ स्व॒राडाप॒श्छन्दा॒स्यापो॒ ज्योती॒ष्यापो॒ यजू॒ष्याप॑ स॒त्यमाप॒ सर्वा॑ दे॒वता॒ आपो॒ भूर्भुव॒स्सुव॒राप॒ ओम्॥\\
 
कलशस्य मुखे विष्णुः कण्ठे रुद्रः समाश्रितः।\\
मूले तत्र स्थितो ब्रह्मा मध्ये मातृगणाः स्मृताः॥\\
कुक्षौ तु सागराः सर्वे सप्तद्वीपा वसुन्धरा।\\
ऋग्वेदोऽथ यजुर्वेदः सामवेदोऽप्यथर्वणः॥\\
अङ्गैश्च सहिताः सर्वे कलशाम्बुसमाश्रिताः।\\
गङ्गे च यमुने चैव गोदावरि सरस्वति।\\
नर्मदे सिन्धुकावेरि जलेऽस्मिन् सन्निधिं कुरु॥\\
सर्वे समुद्राः सरितः तीर्थानि च ह्रदा नदाः।\\
आयान्तु विष्णुपूजार्थं दुरितक्षयकारकाः॥\\
ॐ भूर्भुवः॒ सुवो॒ भूर्भुवः॒ सुवो॒ भूर्भुवः॒ सुवः॑।\\

(इति कलशजलेन सर्वोपकरणानि आत्मानं च प्रोक्ष्य।)

\dnsub{आत्मपूजा}
ॐ आत्मने नमः, दिव्यगन्धान् धारयामि।
\begin{multicols}{2}
१. ॐ आत्मने नमः\\
२. ॐ अन्तरात्मने नमः\\
३. ॐ योगात्मने नमः\\
४. ॐ जीवात्मने नमः\\
५. ॐ परमात्मने नमः\\
६. ॐ ज्ञानात्मने नमः
\end{multicols}
समस्तोपचारान् समर्पयामि।\\

देहो देवालयः प्रोक्तो जीवो देवः सनातनः।\\
त्यजेदज्ञाननिर्माल्यं सोऽहं भावेन पूजयेत्॥\\

\dnsub{पीठपूजा}
\begin{multicols}{2}
\begin{enumerate}
\item ॐ आधारशक्त्यै नमः
\item ॐ मूलप्रकृत्यै नमः
\item ॐ आदिकूर्माय नमः 
\item ॐ आदिवराहाय नमः
\item ॐ अनन्ताय नमः
\item ॐ पृथिव्यै नमः
\item ॐ रत्नमण्डपाय नमः
\item ॐ रत्नवेदिकायै नमः
\item ॐ स्वर्णस्तम्भाय नमः
\item ॐ श्वेतच्छत्त्राय नमः
\item ॐ कल्पकवृक्षाय नमः
\item ॐ क्षीरसमुद्राय नमः 
\item ॐ सितचामराभ्यां नमः
\item ॐ योगपीठासनाय नमः
\end{enumerate}
\end{multicols}
 
\sect{षोडशोपचारपूजा}
% \begin{center}

\fourlineindentedshloka*
{वैदेहीसहितं सुरद्रुमतले हैमे महामण्डपे}
{मध्ये पुष्पकमासने मणिमये वीरासने सुस्थितम्}
{अग्रे वाचयति प्रभञ्जनसुते तत्त्वं मुनिभ्यः परं}
{व्याख्यान्तं भरतादिभिः परिवृतं रामं भजे श्यामलम्}

\fourlineindentedshloka*
{वामे भूमिसुता पुरश्च हनुमान् पश्चात् सुमित्रासुतः}
{शत्रुघ्नो भरतश्च पार्श्वदलयोर्वाय्वादि कोणेषु च}
{सुग्रीवश्च विभीषणश्च युवराट् तारासुतो जाम्बवान्}
{मध्ये नीलसरोजकोमलरुचिं रामं भजे श्यामलम्}

श्री-सीता-लक्ष्मण-भरत-शत्रुघ्न-हनुमत्-समेत-श्री-रामचन्द्रं ध्यायामि।

(अथ प्राणप्रतिष्ठा)


आवाहयामि विश्वेशं वैदेहीवल्लभं विभुम्।\\
कौसल्यातनयं रामं पूर्णचन्द्रनिभाननम्॥

अस्मिन् बिम्बे श्री-सीता-लक्ष्मण-भरत-शत्रुघ्न-हनुमत्-समेत-श्री-रामचन्द्रम् आवाहयामि।

रत्नसिह्मासनारूढ सर्वभूपालवन्दित।\\
आसनं ते मया दत्तं प्रीतिं जनयतु प्रभो॥ - आसनं समर्पयामि।\\

पादाङ्गुष्ठसमुद्भूतगङ्गापावितविष्टप।\\
पाद्यार्थमुदकं राम ददामि परिगृह्यताम्॥ - पाद्यं समर्पयामि।\\

वालखिल्यादिभिर्विप्रैस्त्रिसन्ध्यं प्रयतात्मभिः।\\
अर्घ्यैराराधित विभो ममार्घ्यं राम गृह्यताम्॥ - अर्घ्यं समर्पयामि।\\

आचान्ताम्भोधिना राम मुनिना परिसेवित।\\
मया दत्तेन तोयेन कुर्वाचमनमीश्वर॥ - आचमनीयं समर्पयामि।\\

मधुदध्याज्यसंयुक्तं मधुसुदन राघव।\\
मधुपर्क मया दत्तं गृहाण रघुनायक॥ - मधुपर्कं समर्पयामि।\\

कामधेनु-समुद्भूतक्षीरेणेन्द्रेण राघव।\\
अभिषिक्ताखिलार्थाप्त्यै स्नाहि मद्दत्तदुग्धतः॥ - क्षीराभिषेकं समर्पयामि।\\

हनूमता मधुवनोद्भूतेन मधुना प्रभो।\\
प्रीत्याऽभिषेचिततनो मधुना स्नाहि मेऽद्य भोः॥ - मध्वभिषेकं समर्पयामि।\\

त्रैलोक्यतापहरणनामकीर्तन राघव।\\
मधूत्थतापशान्त्यर्थं स्नाहि क्षीरेण वै पुनः॥ - मध्वभिषेकान्ते पुनः क्षीराभिषेकं समर्पयामि।\\

नदीनदसमुद्रादितोयैर्मन्त्राभिसंस्कृतैः।\\
पट्टाभिषिक्त राजेन्द्र स्नाहि शुद्धजलेन मे॥ - शुद्धोदकस्नानं समर्पयामि।\\
स्नानानन्तरं आचमनीयं समर्पयामि।\\

हित्वा पीताम्बरं चीरकृष्णाजिनधराच्युत।\\
परिधत्स्वाद्य मे वस्त्रं स्वर्णसूत्रविनिर्मितम्॥ - वस्त्रं समर्पयामि।\\

राजर्षिवंशतिलक रामचन्द्र नमोऽस्तु ते।\\
यज्ञोपवीतं विधिना निर्मितं धत्स्व मे प्रभो॥ - उपवीतं समर्पयामि।\\

किरीटादीनि राजेन्द्र हंसकान्तानि राघव।\\
विभूषणानि धृत्वाऽद्य शोभस्व सह सीतया॥ - आभरणम् समर्पयामि।\\

सन्ध्यासमानरुचिना नीलाभ्रसमविग्रह।\\
लिम्पाभि तेऽङ्गकं राम चन्दनेन मुदा हृदि॥ - गन्धान् धारयामि।\\
गन्धस्योपरि हरिद्राकुङ्कुमं समर्पयमि।

अक्षतान् कुङ्कुमोन्मिश्रानक्षय्यफलदायक।\\
अर्पये तव पादाब्जे शालितण्डुलसम्भवान्॥ - अक्षतान् समर्पयामि।\\

चम्पकाशोकपुन्नागैर्जलजैस्तुलस्सीदलैः।\\
पूजयामि रघूत्तंस पूज्यं त्वां सनकादिभिः॥ - पुष्पाणि समर्पयामि।


\dnsub{अङ्गगुणपूजा}
\begin{supertabular}{p{0.525\linewidth}p{0.45\linewidth}}
ॐ अहल्योद्धारकाय नमः & श्रीरामपादरजः पूजयामि \\
ॐ शरणागतरक्षकाय नमः & पादकान्तिं पूजयामि \\
ॐ गङ्गानदीप्रवर्तनपराय नमः & पादनखान् पूजयामि \\
ॐ सीतासंवाहितपदाय नमः & पादतलं पूजयामि \\
ॐ विनतकल्पद्रुमाय नमः & गुल्फौ पूजयामि \\
ॐ दुन्दुभिकायविक्षेपकाय नमः & पादाङ्गुष्ठं पूजयामि \\
ॐ दण्डकारण्य-गमन-जङ्घालाय नमः & जङ्घे पूजयामि \\
ॐ जानुन्यस्तकराम्बुजाय नमः & जानुनी पूजयामि \\
ॐ वीरासनाध्यासिने नमः & ऊरू पूजयामि \\
ॐ पीताम्बरालङ्कृताय नमः & कटिं पूजयामि \\
ॐ आकाशमध्यगाय नमः & मध्यं पूजयामि \\
ॐ अरिनिग्रहपराय नमः & कटिलम्बितमासं पूजयामि \\
ॐ अब्धिमेखलापतये नमः & मध्यलम्बितमेखलादामं पूजयामि \\
ॐ उदरस्थितब्रह्माण्डाय नमः & उदरं पूजयामि \\
ॐ जगत्त्रयगुरवे नमः & वलित्रयं पूजयामि \\
ॐ सीतानुलेपित-काश्मीर-चन्दनाय नमः & वक्षः पूजयामि \\
ॐ अभयप्रदानशौण्डाय नमः & दक्षिणबाहुदण्डं पूजयामि \\
ॐ वितरणजितकल्पद्रुमाय नमः & दक्षिणकरतलं पूजयामि \\
ॐ आशरनिरसनपराय नमः & दक्षिणकरस्थितशरं पूजयामि \\
ॐ ज्ञानविज्ञानभासकाय नमः & चिन्मुद्रां पूजयामि \\
ॐ मुनिसङ्घार्पितदिव्यपदाय नमः & वामभुजदण्डं पूजयामि \\
ॐ दशाननकालरूपिणे नमः & वामहस्तस्थितकोदण्डं पूजयामि \\
ॐ शतमखदत्तशतपुष्करस्रजे नमः & अंसौ पूजयामि \\
ॐ कृत्तदशाननकिरीटकूटाय नमः & अंसलम्बिनिषङ्गद्वयं पूजयामि \\
ॐ सीताबाहुलतालिङ्गिताय नमः & कण्ठं पूजयामि \\
ॐ स्मितभाषिणे नमः & स्मितं पूजयामि \\
ॐ नित्यप्रसन्नाय नमः & मुखप्रसादं पूजयामि \\
ॐ सत्यवाचे नमः & वाचं पूजयामि \\
ॐ कपालिपूजिताय नमः & कपोलौ पूजयामि \\
ॐ चक्षुश्रवः प्रभुपूजिताय नमः & श्रवसी पूजयामि \\
ॐ अनासादितपापगन्धाय नमः & घ्राणं पूजयामि \\
ॐ पुण्डरीकाक्षाय नमः & अक्षिणी पूजयामि \\
ॐ अपाङ्गस्यन्दिकरुणाय नमः & अरुणापाङ्गद्वयं पूजयामि \\
ॐ विनाकृतरुषे नमः & अनाथ-रक्षक-कटाक्षं पूजयामि \\
ॐ कस्तूरीतिलकाङ्किताय नमः & फालं पूजयामि \\
ॐ राजाधिराजवेषाय नमः & किरीटं पूजयामि \\
ॐ मुनिमण्डलपूजिताय नमः & जटामण्डलं पूजयामि \\
ॐ मोहितमुनिजनाय नमः & पुंसां मोहनं रूपं पूजयामि \\
ॐ जानकीव्यजनवीजिताय नमः & विद्युद्विद्योतितकालाभ्रसदृशकान्तिं पूजयामि \\
ॐ हनुमदर्पितचूडामणये नमः & करुणारसोद्वेलितकटाक्षधारां पूजयामि \\
ॐ सुमन्त्रानुग्रहपराय नमः & तेजोमयरूपं पूजयामि \\
ॐ कम्पिताम्भोधये नमः & आहार्यकोपं पूजयामि \\
ॐ तिरस्कृतलङ्केश्वराय नमः & धये पूजयामि \\
ॐ दूराद्वन्दितजनकाय नमः & विनयं पूजयामि \\
ॐ सम्मानितत्रिजटाय नमः & अतिमानुषसौलभ्यं पूजयामि \\
ॐ गन्धर्वराजप्रतिमाय नमः & लोकोत्तरसौन्दर्य पूजयामि \\
ॐ असहाय-हत-खर-दूषणादि-चतुर्दश-सहस्र-राक्षसाय नमः & पराक्रमं पूजयामि \\
ॐ आलिङ्गिताञ्जनेयाय नमः & भक्तवात्सल्यं पूजयामि \\
ॐ लब्धराज्यपरित्यक्त्रे नमः & धर्मं पूजयामि \\
ॐ दर्भशायिने नमः & लोकानुवर्तनं पूजयामि \\
ॐ सर्वेश्वराय नमः & सर्वाण्यङ्गानि सर्वांश्च गुणान् पूजयामि \\
\end{supertabular}

\begingroup
\setlength{\columnseprule}{1pt}
\let\chapt\sect
\input{../namavali-manjari/100/Rama_108.tex}
\input{../namavali-manjari/100/Sita_108.tex}
\input{../namavali-manjari/100/Anjaneya_108.tex}
\endgroup

\dnsub{उत्तराङ्गपूजा}

\newcommand{\devAya}{श्री-सीता-लक्ष्मण-भरत-शत्रुघ्न-हनुमत्-समेत-श्री-रामचन्द्र-परब्रह्मणे नमः}

\twolineshloka*
{वनस्पतिरसोद्भूतः सुगन्धः सुमनोहरः}
{रामचन्द्र कृपाराशे धूपोऽयं प्रतिगृह्यताम्}
\devAya{} धूपमाघ्रापयामि।

\twolineshloka*
{सूर्यवंशसुदीपस्त्वं साज्यवर्तिसमन्वितम्}
{गृहाण मङ्गळं दीपं दीनबन्धो दयानिधे}
\devAya{} अलङ्कारदीपं सन्दर्शयामि।

ओं भूर्भुवस्सुवः + ब्रह्मणे स्वाहा।
\twolineshloka*
{नैवेद्यं षड्सोपेतं घृतसूपसमन्वितम्}
{फलभक्ष्यसमायुक्तं गृह्यतां रघुपुङ्गव}

\devAya{} नैवेद्यं निवेदयामि। मध्ये
मध्ये पानीयं समर्पयामि। अमृतापिधानमसि। नैवेद्यानन्तरं आचमनीयं समर्पयामि।


\twolineshloka*
{ताम्बूलं च सकर्पूरं पूगीफलसमन्वितम्}
{नागवल्लीदलैर्युक्तं गृह्यतां रघुनायक}
\devAya{} कर्पूरताम्बूलं समर्पयामि।


\twolineshloka*
{विभीषणाय भक्ताय लङ्काराज्यप्रदायक}
{नरिजनं गृहाणेदं मया भक्त्या समर्पितम्}
\devAya{} समस्त अपराध क्षमापनार्थं कर्पूरनीराजनं दर्शयामि। कर्पूरनीरजनानन्तरम् आचमनीयं समर्पयामि।

\twolineshloka*
{कल्पवृक्षसमुद्भूतैः पुरुहूतादिभिः सुमैः}
{पुष्पाञ्जलिं ददाम्यद्य पूजितायाशरद्विषे}

यो॑ऽपां पुष्पं॒ वेद॑। पुष्प॑वान् प्र॒जावान् पशु॒मान् भ॑वति।\\
च॒न्द्रमा॒ वा अ॒पां पुष्पम्। पुष्प॑वान् प्र॒जावान् पशु॒मान् भ॑वति।\\
य ए॒वं वेद॑। यो॑ऽपामा॒यत॑नं॒ वेद॑। आ॒यत॑नवान् भवति।\medskip

ओं तद्ब्र॒ह्म। ओं तद्वा॒युः। ओं तदा॒त्मा।\\ ओं᳚ तथ्स॒त्यम्‌।
ओं᳚ तथ्सर्वम्᳚‌। ओं तत्पुरो॒र्नमः॥\medskip

अन्तश्चरति॑ भूते॒षु॒ गुहायां वि॑श्वमू॒र्तिषु। \\
त्वं यज्ञस्त्वं वषट्कारस्त्वमिन्द्रस्त्व\\ रुद्रस्त्वं विष्णुस्त्वं ब्रह्म त्वं॑ प्रजा॒पतिः। \\
त्वं त॑दाप॒ आपो॒ ज्योती॒ रसो॒ऽमृतं॒ ब्रह्म॒ भूर्भुवः॒ सुव॒रोम्‌॥

\devAya{} वेदोक्तमन्त्रपुष्पाञ्जलिं समर्पयामि।

\twolineshloka*
{मन्दाकिनीसमुद्भूतकाञ्चनाब्जस्रजा विभो}
{सम्मानिताय शक्रेण स्वर्णपुष्पं ददामि ते}
स्वर्णपुष्पम् समर्पयामि।

\twolineshloka*
{यानि कानि च पापानि + प्रदक्षिणपदेपदे}
{प्रकृष्टपापनाशाय + प्रसीद पुरुषोत्तम}

\twolineshloka*
{चराचरं व्याप्नुवन्तमपि त्वां रघुनन्दन}
{प्रदक्षिणं करोग्यद्य मदते मूर्तिसंयुतम्}

\hfill (प्रदक्षिणम्)

\fourlineindentedshloka*     
{ध्येयं सदा परिभवघ्नमभीष्टदोहं}
{तीर्थास्पदं शिवविशिश्नुतं शरण्यम्}
{भृत्यार्तिहं प्रणतपाल-भवाब्धिपोतं}
{वन्दे महापुरुष ते चरणारविन्दम्}


\fourlineindentedshloka*     
{त्यक्त्वा सुदुस्त्यजसुरेप्सितराज्यलक्ष्मीं}
{धर्मिष्ठ आर्यवचसा यद्गादरण्यम्}
{मायामृगं दयितयेप्सितमन्वधावत्}
{वन्दे महापुरुष ते चरणारविन्दम्}

\twolineshloka*
{साङ्गोपाङ्गाय साराय जगतां सनकादिभिः}
{वन्दिताय वरेण्याय राघवाय नमो नमः } 
\devAya{} नमस्काराः समर्पयामि।

\sect{हनुमत्कृतं श्रीसीतारामस्तोत्रम्}

\twolineshloka
{अयोध्यापुरनेतारं मिथिलापुरनायिकाम्}
{इक्ष्वाकूणामलंकारं वैदेहानामलंक्रियाम्} % ॥ १ ॥

\twolineshloka
{रघूणां कुलदीपं च निमीनां कुलदीपिकाम्}
{सूर्यवंशसमुद्भूतं सोमवंशसमुद्भवाम्} % ॥ २ ॥

\twolineshloka
{पुत्रं दशरथस्यापि पुत्रीं जनकभूपतेः}
{वसिष्ठानुमताचारं शतानन्दमतानुगाम्} % ॥ ३ ॥

\twolineshloka
{कौसल्यागर्भसंभूतं वेदिगर्भोदितां स्वयम्}
{कालमेघनिभं रामं कार्तस्वरविभूषिताम्} % ॥ ४ ॥

\twolineshloka
{चन्द्रकान्ताननाम्भोजं चन्द्रबिम्बोपमाननाम्}
{पुण्डरीकविशालाक्षं स्फुरदिन्दीवरेक्षणम्} % ॥ ५ ॥

\twolineshloka
{मत्तमातङ्गगमनं मत्तसारसगामिनीम्}
{तालीदलश्यामलाङ्गं तप्तचामीकरप्रभाम्} % ॥  ६ ॥

\twolineshloka
{चन्दनार्द्रभुजामध्यं कुङ्कुमाक्तभुजान्तराम्}
{चापालंकृत हस्ताब्जं पद्मालंकृतपाणिकाम्} % ॥ ७ ॥

\twolineshloka
{शरणागतगोप्तारं प्रणिपातप्रसादिकाम्}
{सर्वलोकविधातारं सर्वलोकविधायिनीम्} % ॥ ८ ॥

\twolineshloka
{लोकाभिरामं श्रीराममभिरामां च मैथिलीम्}
{दिव्यसिह्मासनारूढं दिव्यत्रग्वस्त्रभूषणाम्} % ॥ ९ ॥

\threelineshloka
{अनुक्षणं कटाक्षाभ्यामन्योन्येक्षणकाङ्क्षिणौ} 
{अन्योन्यसदृशावेतो त्रैलोक्यगृहदम्पती}
{इमौ युवां प्रणम्याहं भजाम्यद्य कृतार्थताम्} % ॥ १० ॥

\twolineshloka
{अनया स्तोति यः स्तुत्या रामं सीतां च भक्तितः}
{तस्य तो तनुतां प्रीतो संपदः सकला अपि} % ॥ ११ ॥

\threelineshloka
{इतीदं रामचन्द्रस्य जानक्याश्च विशेषतः}
{कृतं हनुमता पुण्यं स्तोत्रं सद्योविमुक्तिदम्}
{यः पठेत्प्रातरुत्थाय सर्वान् कामानवाप्नुयात्} % ॥ १२ ॥

\hfill (इति स्तोत्रम्)



\twolineshloka*
{एकातपत्रच्छायायां शासिताशेषभूमिक}
{मम छत्रमिदं रत्नजालकं राम गृह्यताम्}
\hfill छत्रम् समर्पयामि।


\twolineshloka*
{रक्षोराजानुजाभ्यां ते कृतं चामरसेवया}
{वीजयेऽहं कराभ्यां ते चामरद्वयमादरात्}
\hfill चामरम् वीजयामि।


\twolineshloka*
{रामायणं साधु गीतं सुताभ्यां श्रुतवानसि}
{मयाऽपि गीयमानं ते स्तोत्रं चित्ताय रोचताम्}
\hfill गीतम् गायामि।


\twolineshloka*
{वीणावेणुमृदङ्गादिवाद्यैस्त्वां प्रीणयाम्यहम्}
{मददम्भाहङ्कृतीनां नाशको भव राघव}
\hfill वाद्यम् घोषयामि।


\twolineshloka*
{आरुह्य सीतया सार्धं दत्तामान्दोलिकां मया}
{विभाहि भूषितो राम मत्कृते पूजनोत्सवे}
\hfill आन्दोलिकां समर्पयामि।


\twolineshloka*
{मया कल्पितपल्याणं महान्तं मम घोटकम्}
{मदंसे चरणं न्यस्य मुदाऽऽरोह रघूत्तम}
\hfill अश्वान् आरोहयामि।


\twolineshloka*
{गजेन महताऽऽयान्तमाकांक्षन्ति स्म नागराः}
{द्रष्टुं त्वां मगजे भाहि दृष्ट्वा नन्देयमप्यहम्}
\hfill गजान् आरोहयामि।

समस्तराजोपचारदेवोपचारपूजाः समर्पयामि।

\sect{प्रार्थना}
\resetShloka
\threelineshloka
{त्वमक्षरोऽसि भगवन् व्यक्ताव्यक्तस्वरूपधृत्}
{यथा त्वं रावणं हत्वा यज्ञविघ्नकरं खलम्}
{लोकान् रक्षितवान् राम तथा मन्मानसाश्रयम्}

\twolineshloka
{रजस्तमञ्च निर्हृत्य त्वत्पूजालस्यकारकम्}
{सत्त्वमुद्रेकय विभो त्वत्पूजादरसिद्धये}

\twolineshloka
{विभूतिं वर्धय गृहे पुत्रपौत्राभिवृद्धिकृत्}
{कल्याणं कुरु मे नित्यं कैवल्यं दिश चान्ततः}

\twolineshloka
{विधितोऽविधितो वाऽपि या पूजा क्रियते मया}
{तां त्वं सन्तुष्टहृदयो यथावद्विहितामिव}

\twolineshloka
{स्वीकृत्य परमेशान मात्रा मे सह सीतया}
{लक्ष्मणादिभिरप्यत्र प्रसादं कुरु मे सदा}

\twolineshloka
{मनसा वचसा कायेनागसां शतमन्वहम्}
{धियाऽधिया च रचये क्षमस्व सहजक्षम}

\twolineshloka
{आवाहनं न जानामि न जानामि विसर्जनम्}
{पूजाविधिं न जानामि क्षमस्व पुरुषोत्तम}


\dnsub{अर्घ्य-प्रदानम्}

\twolineshloka*
{शुक्लाम्बरधरं विष्णुं शशिवर्णं चतुर्भुजम्}
{प्रसन्नवदनं ध्यायेत् सर्वविघ्नोपशान्तये}


प्राणान्  आयम्य।  ॐ भूः + भूर्भुवः॒ सुव॒रोम्।

ममोपात्त + प्रीत्यर्थम् अद्य पूर्वोक्त + शुभतिथौ श्रीरामचन्द्रपूजान्ते अर्घ्यप्रदानं करिष्ये (इति सङ्कल्प्य)।

\twolineshloka*
{राम रात्रिञ्चराराते क्षीरमध्वाज्यकल्पितम्}
{पूजान्तेऽर्घ्यं मया दत्तं स्वीकृत्य वरदो भव}

\devAya{} इदमर्घ्यं इदमर्घ्यं इदमर्घ्यम्॥

अनेनार्ध्यप्रदानेन श्री-सीता-लक्ष्मण-भरत-शत्रुघ्न-हनुमत्-समेत-श्री-रामचन्द्रः प्रीयताम्।


हिरण्यगर्भगर्भस्थं हेमबीजं विभावसोः।\\
अनन्तपुण्यफलदम् अतः शान्तिं प्रयच्छ मे॥\\

श्री-रामनवमी-पुण्यकाले अस्मिन् मया क्रियमाण श्रीरामपूजायां यद्देयमुपायनदानं तत्प्रतिनिधित्वेन हिरण्यं श्री-सीता-लक्ष्मण-भरत-शत्रुघ्न-हनुमत्-समेत-श्री-रामचन्द्र-प्रीतिं 
कामयमानः मनसोद्दिष्टाय ब्राह्मणाय सम्प्रददे नमः न मम। 

अनया पूजया श्री-सीता-लक्ष्मण-भरत-शत्रुघ्न-हनुमत्-समेत-श्री-रामचन्द्रः प्रीयताम्। 
 
\dnsub{विसर्जनम्}
\twolineshloka*
{यस्य स्मृत्या च नामोक्त्या तपः पूजा क्रियादिषु}
{न्यूनं सम्पूर्णतां याति सद्यो वन्दे तमच्युतम्} 

\twolineshloka*
{इदं व्रतं मया देव कृतं प्रीत्यै तव प्रभो}
{न्यूनं सम्पूर्णतां यातु त्वत्प्रसादाज्जनार्द्दन}

अस्मात् बिम्बात् श्री-सीता-लक्ष्मण-भरत-शत्रुघ्न-हनुमत्-समेत-श्री-रामचन्द्रं यथास्थानं प्रतिष्ठापयामि।\\
(अक्षतानर्पित्वा देवमुत्सर्जयेत्।)\\

\fourlineindentedshloka*
{कायेन वाचा मनसेन्द्रियैर्वा}
{बुद्‌ध्याऽऽत्मना वा प्रकृतेः स्वभावात्}
{करोमि यद्यत् सकलं परस्मै}
{नारायणायेति समर्पयामि}

अनया पूजया श्री-सीता-लक्ष्मण-भरत-शत्रुघ्न-हनुमत्-समेत-श्री-रामचन्द्रः प्रीयताम्। \\
ॐ तत्सद्ब्रह्मार्पणमस्तु।
 
\sect{कथा}
\uvacha{अगस्त्य उवाच}
\twolineshloka
{रहस्यं कथयिष्यामि सुतीक्ष्ण मुनिसत्तम}
{चैत्रे नवम्यां प्राक्पक्षे दिवापुण्ये पुनर्वसौ}%॥ १ ॥

\twolineshloka
{उदये गुरुगौरांशे स्वोच्चस्थे ग्रहपञ्चके}
{मेष पूषणि सम्प्राप्ते लग्ने कर्कटकाह्वये}%॥ २ ॥

\twolineshloka
{आविरासीत्स कलया कौसल्यायां परः पुमान्}
{तस्मिन्दिने तु कर्तव्यमुपवासव्रतं सदा}%॥ ३ ॥

\twolineshloka
{तत्र जागरणं कुर्याद्रघुनाथपुरो भुवि}% भुवीतिखट्वादिव्यावृत्त्यर्थम्
{प्रतिमायां यथाशक्ति पूजा कार्या यथाविधि}%॥ ४ ॥

\twolineshloka
{प्रातर्दशम्यांस्नात्वैव कृत्वा सन्ध्यादिकाः क्रियाः}
{सम्पूज्य विधिवद् रामं भक्त्या वित्तानुसारतः}%॥ ५ ॥

\twolineshloka
{ब्राह्मणान् भोजयेत् सम्यक् दक्षिणाभिश्च तोषयेत्}
{गोभूतिलहिरण्याद्यैर्वस्त्रालङ्करणैस्तथा}%॥ ६ ॥

\twolineshloka
{रामभक्तान्प्रयत्नेन प्रीणयेत्परया मुदा}
{एवं यः कुरुते भक्त्या श्रीरामनवमीव्रतम्}%॥ ७ ॥

\twolineshloka
{अनेकजन्मासिद्धानि पापानि सुबहूनि च}
{भस्मीकृत्य व्रजत्येव तद्विष्णोः परमं पदम्}%॥ ८ ॥

\threelineshloka
{सर्वेषामप्ययं धर्मो भुक्तिमुक्त्येकसाधनः}
{अशुचिर्वाऽपि पापिष्ठः कृत्वेदं व्रतमुत्तमम्}
{पूज्यः स्यात्सर्वभूतानां यथा रामस्तथैव सः}%॥ ९ ॥

\twolineshloka
{यस्तु रामनवम्यां वै भुक्ङ्ते स तु नराधमः}
{कुम्भीपाकेषु घोरेषु गच्छत्येव न संशयः}%॥ १० ॥

\twolineshloka
{अकृत्वा रामनवमीव्रतं सर्वव्रतोत्तमम्}
{व्रतान्यन्यानि कुरुते न तेषां फलभाग्भवेत्}%॥ ११ ॥

\twolineshloka
{रहस्यकृतपापानि प्रख्यातानि बहून्यपि}
{महान्ति च प्रणश्यन्ति श्रीरामनवमीव्रतात्}%॥ १२ ॥

\twolineshloka
{एकामपि नरो भक्त्या श्रीरामनवमीं मुने}
{उपोष्य कृतकृत्यः स्यात्सर्वपापैः प्रमुच्यते}%॥ १३ ॥

\twolineshloka
{नरो रामनवम्यां तु श्रीरामप्रतिमाप्रदः}
{विधानेन मुनिश्रेष्ठ स मुक्तो नात्र संशयः}%॥ १४ ॥

\uvacha{सुतीक्ष्ण उवाच}
\twolineshloka
{श्रीरामप्रतिमादानविधानं वा कथं मुने}
{कथय त्वं हि रामेऽपि भक्तम्य मम विस्तरात्}%॥ १५ ॥

\uvacha{अगस्त्य उवाच}
\onelineshloka{कथायिष्यामि तद्विद्वन् प्रतिमादानमुत्तमम्}%॥ १६ ॥

\twolineshloka
{विधानं चापि यत्नेन यतस्त्वं वैष्णवोत्तमः}
{अष्टम्यां चैत्रमासे तु शुक्लपक्षे जितेन्द्रियः}%॥ १७ ॥

\twolineshloka
{दन्तधावनपूर्वं तु प्रातः स्नायाद्यथाविधि}
{नद्यां तडागे कूपे वा ह्रदे प्रस्रवणेऽपि वा}%॥ १८ ॥

\twolineshloka
{ततः सन्ध्यादिका कार्याः संस्मरन् राघवं हृदि}
{गृहमासाद्य विप्रेन्द्र कुर्यादौपासनादिकम्}%॥ १९ ॥

\twolineshloka
{दान्तं कुटुम्बिनं विप्रं वेदशास्त्रपरं सदा}
{श्रीरामपूजानिरतं सुशीलं दम्भवर्जितम्}%॥ २० ॥

\twolineshloka
{विधिज्ञं राममन्त्राणां राममन्त्रैकसाधनम्}
{आहूय भक्त्या सम्पूज्य वृणुयात्प्रार्थयन्निति}%॥ २१ ॥

\twolineshloka
{श्रीरामप्रतिमादानं करिष्येऽहं द्विजोत्तम}
{तत्राचार्यो भव प्रीतः श्रीरामोऽसि त्वमेव च}%॥ २२ ॥

\twolineshloka
{इत्युक्त्वा पूज्य विप्रं तं स्नापयित्वा ततः परम्}
{तैलेनाभ्यज्य पयसा चिन्तयन्राघवं हृदि}%॥ २३ ॥

\twolineshloka
{श्वेताम्बरधरः श्वेतगन्धमाल्यानि धारयेत्}
{अर्चितो भूषितश्चैव कृतमाध्याह्निकक्रियः}%॥ २४ ॥

\twolineshloka
{आचार्यभोजयेद् भक्त्या सात्त्विकान्नैः सुविस्तरम्}
{भुञ्जीत स्वयमप्येवं हृदि राममनुस्मरन्}%॥ २५ ॥

\twolineshloka
{एकभक्तव्रती तत्र सहाचार्यो जितन्द्रियः}
{शृण्वन्रामकथां दिव्यामहःशेषं नयेन्मुने}%॥ २६ ॥

\twolineshloka
{सायं सन्ध्यादिकाः कुर्यात्क्रिया राममनुस्मरन्}
{आचार्यसहितो रात्रावधःशायी जितेन्द्रियः}%॥ २७ ॥

\twolineshloka
{वसेत्स्वयं न चैकान्ते श्रीरामार्पितमानसः}
{ततः प्रातः समुत्थाय स्नात्वा सन्ध्यां यथाविधि}%॥ २८ ॥

\twolineshloka
{प्रातः सर्वाणि कर्माणि शीघ्रमेव समापयेत्}
{ततः स्वस्थमना भूत्वा विद्वद्भिः सहितोऽनघ}%॥ २९ ॥

\twolineshloka
{स्वगृहे चोत्तरे देशे दानस्योज्ज्वलमण्डपम्}%स्वगृहे स्वगृहसमीपे॥
{चतुरं पताकाढयं सवितानं सतोरणम्}%॥ ३० ॥

\twolineshloka
{मनोहरं महोत्सेधं पुष्पाद्यैः समलङ्कृतम्}
{शङ्खचक्रहनूमाद्भिः प्रारद्वारे समलङ्कृतम्}%॥ ३१ ॥

\twolineshloka
{गरुत्मच्छार्ङ्गबाणैश्च दक्षिणे समलकृतम्}
{गदाखड्गाङ्गदैश्चैव पश्चिमे च विभूषितम्}%॥ ३२ ॥

\twolineshloka
{पद्मस्वस्तिकनीलैश्च कौबेर्यां समलङ्कृतम्}
{मध्यहस्तचतुष्काढ्यवेदिकायुक्तमायतम्}%॥ ३३ ॥

\twolineshloka
{प्रविश्य गीतनृत्यैश्च वाद्यैश्चापि समन्वितम्}
{पुण्याहं वाचयित्वा च विद्वद्भिः प्रीतमानसः}%॥ ३४ ॥

\twolineshloka
{ततः सङ्कल्पयेद्देवं राममेव स्मरन्मुने}
{अस्यां रामनवम्यां तु रामाराधनतत्परः}%॥ ३५ ॥

\twolineshloka
{उपोष्याष्टसु यामेषु पूजयित्वा यथाविधि}
{इमां स्वर्णमयीं रामप्रतिमां तु प्रयत्नतः}%॥ ३६ ॥

\twolineshloka
{श्रीरामप्रीतये दास्ये रामभक्ताय धीमते}
{प्रीतो रामो हरत्वाशु पापानि सुबहूनि मे}%॥ ३७ ॥

\twolineshloka
{अनेकजन्मसंसिद्धान्यभ्यस्तानि महान्ति च}
{विलिखेत्सर्वतोभद्रं वेदिकोपरि सुन्दरम्}%॥ ३८ ॥

\twolineshloka
{मध्ये तीर्थोदकैर्युक्तं पात्र संस्थाप्य चार्चितम्}
{सौवर्णे राजते ताम्रे पात्रे षट्कोणमालिखेत्}%॥ ३९ ॥

\twolineshloka
{ततः स्वर्णमयीं रामप्रतिमां पलमात्रतः}
{निर्मितां द्विभुजां रम्यां वामाङ्कस्थितजानकीम्}%॥ ४० ॥

\twolineshloka
{बिभ्रतीं दक्षिणे हस्ते ज्ञानमुद्रां महामुने}
{वामेनाधःकरेणाराद्देवीमालिंङ्ग्य संस्थिताम्}%॥ ४१ ॥

\twolineshloka
{सिंहासने राजते च पलद्वयविनिर्मिते}
{पञ्चामृतस्नानपूर्वं सम्पूज्य विधिवत्ततः}%॥ ४२ ॥

\twolineshloka
{मूलमन्त्रेण नियतो न्यासपूर्वमतन्द्रितः}
{दिवैवं विधिवत् कृत्वा रात्रौ जागरणं ततः}%॥ ४३ ॥

\twolineshloka
{दिव्यां रामकथां श्रुत्वा रामभक्तिसमन्वितः}
{गीतनृत्यादिभिश्चैव रामस्तोत्रैरनेकधा}%॥ ४४ ॥

\twolineshloka
{रामाष्टकैश्च संस्तुत्य गन्धपुष्पाक्षतादिभिः}
{कर्पूरागुरुकस्तूरीकह्लाराद्यैरनेकधा}%॥ ४५ ॥

\twolineshloka
{सम्पूज्य विधिवद् भक्त्या दिवारात्रं नयेद्बुधः}
{ततः प्रातः समुत्थाय स्नानसन्ध्यादिकाः क्रियाः}%॥ ४६ ॥

\twolineshloka
{समाप्य विधिवद्रामं पूजयेद्विधिवन्मुने}
{ततो होम प्रकुर्वीत मूलमन्त्रेण मन्त्रवित्}%॥ ४७ ॥

\twolineshloka
{पूर्वोक्त पद्मकुण्डे वा स्थण्डिले वा समाहितः}
{लौकिकाग्नौ विधानेन शतमष्टोत्तरं मुने}%॥ ४८ ॥

\twolineshloka
{साज्येन पायसेनैव स्मरन्राममनन्यधीः}
{ततो भक्त्या सुसन्तोष्य आचार्यं पूजयेन्मुने}%॥ ४९ ॥

\twolineshloka
{कुण्डलाभ्यां सरनाभ्यामगुलीयैरनेकधा}
{गन्धपुष्पाक्षतैर्वस्त्रैर्विचित्रैस्तु मनोहरैः}%॥ ५० ॥

\twolineshloka
{ततो रामं स्मरन्दद्यादिमं मन्त्रमुदीरयेत्}
{इमां स्वर्णमयीं रामप्रतिमां समलङ्कृताम्}%॥ ५१ ॥

\twolineshloka
{चित्रवस्त्रयुगच्छन्नरामोऽहं राघवाय ते}
{श्रीरामप्रीतये दास्ये तुष्टो भवतु राघवः}%॥ ५२ ॥

\twolineshloka
{इति दत्वा विधानेन दद्याद्वै दक्षिणां ध्रुवम्}
{अन्नेभ्यश्च यथाशक्त्या गोहिरण्यादि भक्तितः}%॥ ५३ ॥

\twolineshloka
{दद्याद्वासोयुगं धान्यं तथाऽलङ्करणानि च}
{एवं यः कुरुते रामप्रतिमादानमुत्तमम्}%॥ ५४ ॥

\twolineshloka
{ब्रह्महत्यादिपापेभ्यो मुच्यते नात्र संशयः}
{तुलापुरुषदानादिफलमाप्नोति सुव्रत}%॥ ५५ ॥

\twolineshloka
{अनेकजन्मसंसिद्धपापेभ्यो मुच्यते ध्रुवम्}
{बहुनाऽत्र किमुक्तेन मुक्तिस्तस्य करे स्थिता}%॥ ५६ ॥

\threelineshloka
{कुरुक्षेत्रे महापुण्ये सूर्यपर्वण्यशेषतः}
{तुलापुरुषदानाद्यैः कृतैर्यल्लभते फलम्}
{तत्फलं लभते मर्त्यो दानेनानेन सुव्रत}% ॥५७॥

\uvacha{सुतीक्ष्ण उवाच}
\twolineshloka
{प्रायेण हि नराः सर्वे दरिद्राः कृपणा मुने}
{कैः कर्तव्यं कथमिदं व्रतं ब्रूहि महामुने}%॥ ५८ ॥

\uvacha{अगस्त्य उवाच}
\onelineshloka
{दरिद्रश्च महाभाग स्वस्य वित्तानुसारतः}%॥ ५९ ॥

\twolineshloka
{पलार्धेन तदर्धेन तदर्धार्धेन वा पुनः}
{वित्तशाठ्यमकृत्वैव कुर्यादेवं व्रतं मुने}%॥ ६० ॥

\twolineshloka
{यदि घोरतरं दुष्टं पातकं नेहते क्वचित्}
{अकिञ्चनोऽपि यत्नेन उपोष्य नवमीदिने}%॥ ६१ ॥

\twolineshloka
{एकचित्तोऽपि विधिवत्सर्वपापैः प्रमुच्यते}
{प्रातःस्नानं च विधिवत्कृत्वा सन्ध्यादिकाः क्रियाः}%॥ ६२ ॥

\twolineshloka
{गोभूतिलहिरण्यादि दद्याद्वित्तानुसारतः}
{श्रीरामचन्द्रभक्तेभ्यो विद्वद्भयः श्रद्धयान्वितः}%॥ ६३ ॥

\twolineshloka
{पारणं त्वथ कुर्वीत ब्राह्मणैश्च स्वबन्धुभिः}
{एवं यः कुरुते भक्त्या सर्वपापैः प्रमुच्यते}%॥ ६४ ॥

\twolineshloka
{प्राप्ते श्रीरामनवमीदिने मर्त्यो विमूढधीः}
{उपोषणं न कुरुते कुम्भीपाकेषु पच्यते}%॥ ६५ ॥

\twolineshloka
{यत्किञ्चिद्राममुदिक्ष्य क्रियते न स्वशक्तितः}
{रौरवे स तु मूढात्मा पच्यते नात्र संशयः}%॥ ६६ ॥

\uvacha{सुतीक्ष्ण उवाच}
\twolineshloka
{यामाष्टके तु पूजा वै तत्र चोक्ता महामुने}
{मूलमन्त्रेणं संयुक्ता तां कथां वद सुव्रत}%॥ ६७ ॥

\uvacha{अगस्त्य उवाच}
\twolineshloka
{सर्वेषां राममन्त्राणां मन्त्रराज षडक्षरम्} %इदं तु स्कान्दे मोक्षखण्डे श्रीरामं प्रतिरुद्रगीतायां रुद्रवाक्यम्
{मुमूर्मणिकान्ते अर्धोदकनिवासिनः}%॥ ६८ ॥

\twolineshloka
{अहं दिशामि ते मन्त्रं तारकस्योपदेशतः}
{श्रीराम राम रामेति एतत्तारकमुच्यते}%॥ ६९ ॥

\twolineshloka
{अतस्त्वं जानकीनाथपरं ब्रह्माभिधीयसे}
{तारकं ब्रह्म चेत्युक्तं तेन पूजा प्रशस्यते}%॥ ७० ॥

\twolineshloka
{पीठाङ्गदेवतानां तु आवृत्तीनां तथैव च}
{आदावेव प्रकुर्वीत देवस्य प्रीतमानस}%॥ ७१ ॥

\twolineshloka
{उपचारैःषोडशभिः पूजाकार्या यथाविधि}
{आवाहनं स्थापनं च सम्मुखीकरणं तथा}%॥ ७२ ॥

\twolineshloka
{एवं मुद्रां प्रार्थनां च पूजामुद्रां प्रयत्नतः}
{शङ्खपूजां प्रकुर्वीत पूर्वोक्तविधिना ततः}%॥ ७३ ॥

\twolineshloka
{कलशं वामभागे च पूजाद्रव्याणि चादरात्}
{पीठे सम्पूज्य यत्नेन आत्मानं मन्त्रमुच्चरेत्}%॥ ७४ ॥

\twolineshloka
{पात्रासादनमप्येवं कुर्याद्यामेष्वतन्द्रितः}
{पीताम्बराणि देवाय प्रार्पयन्नर्चयेत्सुधीः}%॥ ७५ ॥

\twolineshloka
{स्वर्णयज्ञोपवीतानि दद्याद्देवाय भक्तितः}
{नानारत्नविचित्राणि दद्यादाभरणानि च}%॥ ७६ ॥

\twolineshloka
{हिमाम्बुघृष्टं रुचिरं घनसारमनोहरम्}
{क्रमात्तु मूलमन्त्रेण उपचारान्प्रकल्पयेत्}%॥ ७७ ॥

\twolineshloka
{कह्लारैः केतकैर्जात्यैः पुन्नागाद्यैः प्रपूजयेत्}
{चम्पकैः शतपत्रैश्च सुगन्धैः सुमनोहरैः}%॥ ७८ ॥

\twolineshloka
{पाद्यचन्दनधूपैश्च तत्तन्मन्त्रैः प्रपूजयेत्}
{भक्ष्यभोज्यादिकं भक्त्या देवाय विधिनार्ऽपयेत्}%॥ ७९ ॥

\twolineshloka
{येन सोपस्करं देवं दत्वा पापैः प्रमुच्यते}
{जन्मकोटिकृतैर्घोरैर्नानारूपैश्च दारुणः}%॥ ८० ॥

\twolineshloka
{विमुक्तः स्यात्क्षणादेव राम एव भवेन्मुने}
{श्रद्दधानस्य दातव्यं श्रीरामनवमीव्रतम्}%॥ ८१ ॥

\twolineshloka
{सर्वलोकहितायेदं पवित्रं पापनाशनम्}
{लोहेन निर्मितं वाऽपि शिलया दारुणाऽपि वा}%॥ ८२ ॥

\twolineshloka
{एकेनैव प्रकारेण यस्मै कस्मै च वा मुने}
{कृतं सर्वं प्रयत्नेन यत्किञ्चिदपि भक्तितः}%॥ ८३ ॥

\twolineshloka
{जपेदेकान्तमासीनो यावत्स दशमीदिनम्}
{अनेन स्यात्पुनः पूजा दशम्यां भोजयेद् द्विजान्}%॥ ८४ ॥

\twolineshloka
{भक्त्या भोज्यैर्बहुविधैर्दद्याद् भक्त्या च दक्षिणाम्}
{कृतकृत्यो भवेत्तेन सद्यो रामः प्रसीदति}%॥ ८५ ॥

\twolineshloka
{तूष्णीं तिष्ठन्नरो वाऽपि पुनरावृत्तिवर्जितः}
{द्वादशाब्दे कृतेनापि यत्पापं चापि मुच्यते}%॥ ८६ ॥
 
\twolineshloka
{विलयं याति तत्सर्वं श्रीरामनवमीव्रतम्}
{जपं च रामनन्त्राणां यो न जानाति तस्य वै}%॥ ८७ ॥

\twolineshloka
{उपोष्य संस्मरेद्रामं न्यासपूर्वमतन्द्रितः}
{गुरोर्लब्धमिमं मन्त्रं न्यसेन्न्यासपुरःसरम्}%॥ ८८ ॥

\threelineshloka
{यामे यामे च विधिना कुर्यात्पूजां समाहितः}
{मुमुक्षुश्च सदा कुर्याच्छ्रीरामनवमीत्रतम्}
{मुच्यते सर्वपापेभ्यो याति ब्रह्म सनातनम्}%॥ ८९ ॥

॥इति श्रीस्कन्दपुराणे अगस्त्य संहितायामगस्तिसुतीक्ष्णसंवादे रामनवमी\-व्रत\-विधिः सम्पूर्णः॥
