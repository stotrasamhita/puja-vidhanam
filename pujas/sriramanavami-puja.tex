% !TeX program = XeLaTeX
% !TeX root = pUjA.tex

\setlength{\parindent}{0pt}

\chapt{श्री-रामनवमी-पूजा}

\input{purvanga/vighneshwara-puja}

\sect{प्रधान-पूजा — श्रीराम-पूजा}

\twolineshloka*
{शुक्लाम्बरधरं विष्णुं शशिवर्णं चतुर्भुजम्}
{प्रसन्नवदनं ध्यायेत् सर्वविघ्नोपशान्तये}
 
प्राणान्  आयम्य।  ॐ भूः + भूर्भुवः॒ सुव॒रोम्।

\dnsub{सङ्कल्पः}

ममोपात्त-समस्त-दुरित-क्षयद्वारा श्री-परमेश्वर-प्रीत्यर्थं शुभे शोभने मुहूर्ते अद्य ब्रह्मणः
द्वितीयपरार्धे श्वेतवराहकल्पे वैवस्वतमन्वन्तरे अष्टाविंशतितमे कलियुगे प्रथमे पादे
जम्बूद्वीपे भारतवर्षे भरतखण्डे मेरोः दक्षिणे पार्श्वे शकाब्दे अस्मिन् वर्तमाने व्यावहारिकाणां प्रभवादीनां षष्ट्याः संवत्सराणां मध्ये (	) नाम संवत्सरे उत्तरायणे वसन्त-ऋतौ  (मेष/मीन) मासे 
शुक्लपक्षे नवम्यां शुभतिथौ (इन्दु/भौम/बुध/गुरु/भृगु/स्थिर/भानु) वासरयुक्तायाम्
(आर्द्रा/पुनर्वसू/पुष्य) नक्षत्रयुक्तायां ()-योग ()-करण-युक्तायां च एवं गुण-विशेषण-विशिष्टायाम्
अस्याम् नवम्यां शुभतिथौ 

\begin{itemize}

    \item भारतीयस्य वैदिकस्य अस्माकं सनातन-धर्मस्य सर्वत्र विजय-सिद्ध्यर्थं
    
    \item तद्-अवलम्बनेन भारतीयानां महाजनानां विघ्न-निवृत्ति-पूर्वक-सत्कार्य-प्रवृत्ति-द्वारा सर्वविध-ऐहिक-आमुष्मिक-योग-क्षेम-अभिवृद्ध्यर्थं, असत्कार्येभ्यः निवृत्त्यर्थं
    
    \item सनातन-धर्मस्य विरोधिनाम् उपशमनार्थं
    
    \item भारतीय-सत्-सन्तान-समृद्ध्यर्थं, भारतीयानां सन्ततेः अपि अस्मिन् सनातन-धर्म-सम्प्रदाये श्रद्धा-भक्त्योः अभिवृद्ध्यर्थं
    
    \item विभिन्न-सम्प्रदाय-स्थानां सनातन-धर्मावलम्बिनां परस्परं समन्वयेन अविरोधेन साधन-अनुष्ठानेन फल-सिद्ध्यर्थम्
    
    \item सर्वत्र भूमण्डले विशेषतः भारते च राम-नाम-महिम्नः प्रसारार्थं राम-राज्यस्य सिद्ध्यर्थं
    
    \item अयोध्यायां भगवतः श्रीरामस्य जन्म-भूमि-मन्दिरस्य मूल-स्थानं परितः महतः मन्दिर-परिसरस्य सम्यग् अभिवृद्ध्यर्थम्
    
    \item अस्माकं सह-कुटुम्बानां क्षेम-स्थैर्य-धैर्य-वीर्य-विजय-आयुः-आरोग्य-ऐश्वर्याणाम् अभिवृद्ध्यर्थं धर्मार्थ-काम-मोक्ष-चतुर्विध-पुरुषार्थ-सिद्ध्यर्थं
    
    \item मम इहजन्मनि पूर्वजन्मनि जन्मान्तरे च सम्पादितानां ज्ञानाज्ञानकृतमहा\-पातकचतुष्टय-व्यतिरिक्तानां रहस्यकृतानां प्रकाशकृतानां सर्वेषां पापानां सद्य अपनोदनद्वारा सकल-पापक्षयार्थं 

    \item श्री-सीता-लक्ष्मण-भरत-शत्रुघ्न-हनुमत्-समेत-श्री-रामचन्द्र-प्रीत्यर्थम्
    
    \end{itemize}

श्रीरामनवमीपुण्यकाले कल्पोक्तप्रकारेण यथाशक्ति श्रीरामचन्द्रपूजां
करिष्ये।
तदङ्गं कलशपूजां च करिष्ये।


श्रीविघ्नेश्वराय नमः यथास्थानं प्रतिष्ठापयामि।
(गणपति-प्रसादं शिरसा गृहीत्वा)

\input{purvanga/aasana-puja}

\input{purvanga/ghanta-puja}

\input{purvanga/kalasha-puja}

\input{purvanga/aatma-puja}

\input{purvanga/pitha-puja}

\input{purvanga/guru-dhyanam}
 
\begin{center}

\sect{षोडशोपचार-पूजा}
\renewcommand{\devAya}{सपरिवाराय श्रीरामाय नमः,}

\fourlineindentedshloka*
{वैदेही-सहितं सुर-द्रुम-तले हैमे महामण्डपे}
{मध्येपुष्पकमासने मणिमये वीरासने सुस्थितम्}
{अग्रे वाचयति प्रभञ्जन-सुते तत्त्वं मुनिभ्यः परं}
{व्याख्यान्तं भरतादिभिः परिवृतं रामं भजे श्यामलम्}

\fourlineindentedshloka*
{वामे भूमि-सुता पुरश्च हनुमान् पश्चात् सुमित्रा-सुतः}
{शत्रुघ्नो भरतश्च पार्श्व-दलयोर्वाय्वादि-कोणेषु च}
{सुग्रीवश्च विभीषणश्च युवराट् तारा-सुतो जाम्बवान्}
{मध्ये नील-सरोज-कोमल-रुचिं रामं भजे श्यामलम्}

\textbf{श्री-सीता-लक्ष्मण-भरत-शत्रुघ्न-हनुमत्-समेत-श्री-रामचन्द्रं ध्यायामि।}

(अथ प्राणप्रतिष्ठा)


आवाहयामि विश्वेशं वैदेही-वल्लभं विभुम्।\\
कौसल्या-तनयं विष्णुं श्री-रामं प्रकृतेः परम्॥\\
\textbf{श्रीरामाय नमः – आवाहयामि।}

वामे सीताम् आवाहयामि।
पुरस्तात् हनुमन्तम् आवाहयामि ।
पश्चात् लक्ष्मणम् आवाहयामि।
उत्तरस्यां शत्रुघ्नमवाहयामि ।
दक्षिणस्यां दिशि भरतम् आवाहयामि ।
वायव्यायां सुग्रीवम् आवाहयामि ।
ऐशान्यां विभीषणम् आवाहयामि ।
आग्नेय्याम् अङ्गदम् आवाहयामि ।
नैर्ऋत्यां जाम्बवन्तम् आवाहयामि ॥

\twolineshloka*
{रत्न-सिंहासनारूढ सर्व-भूपाल-वन्दित}
{आसनं ते मया दत्तं प्रीतिं जनयतु प्रभो}
\textbf{\devAya{} आसनं समर्पयामि।\\}

\twolineshloka*
{पादाङ्गुष्ठ-समुद्भूत-गङ्गा-पावित-विष्टप}
{पाद्यार्थमुदकं राम ददामि परिगृह्यताम्}
\textbf{\devAya{} पाद्यं समर्पयामि।\\}

\twolineshloka*
{वालखिल्यादिभिर्विप्रैस्त्रिसन्ध्यं प्रयतात्मभिः}
{अर्घ्यैराराधित विभो ममार्घ्यं राम गृह्यताम्}
\textbf{\devAya{} अर्घ्यं समर्पयामि।\\}

\twolineshloka*
{आचान्ताम्भोधिना राम मुनिना परिसेवित}
{मया दत्तेन तोयेन कुर्वाचमनमीश्वर}
\textbf{\devAya{} आचमनीयं समर्पयामि।\\}

\twolineshloka*
{नमः श्री-वासुदेवाय तत्त्व-ज्ञान-स्वरूपिणे }
{मधुपर्कं गृहाणेमं जानकीपतये नमः}
\textbf{\devAya{} मधुपर्कं समर्पयामि।\\}

\twolineshloka*
{कामधेनु-समुद्भूत-क्षीरेणेन्द्रेण राघव}
{अभिषिक्त अखिलार्थाप्त्यै स्नाहि मद्-दत्त-दुग्धतः}
\textbf{\devAya{} क्षीराभिषेकं समर्पयामि।\\}

\twolineshloka*
{हनूमता मधुवनोद्भूतेन मधुना प्रभो}
{प्रीत्याऽभिषेचित-तनो मधुना स्नाहि मेऽद्य भोः}
\textbf{\devAya{} मध्वभिषेकं समर्पयामि।\\}

\twolineshloka*
{त्रैलोक्य-ताप-हरण-नाम-कीर्तन राघव}
{मधूत्थ-ताप-शान्त्यर्थं स्नाहि क्षीरेण वै पुनः}
\textbf{\devAya{} मध्वभिषेकान्ते पुनः क्षीराभिषेकं समर्पयामि।\\}

\twolineshloka*
{नदी-नद-समुद्रादि-तोयैर्मन्त्राभिसंस्कृतैः}
{पट्टाभिषिक्त राजेन्द्र स्नाहि शुद्ध-जलेन मे}
\textbf{\devAya{} शुद्धोदक-स्नानं समर्पयामि।\\}
स्नानोत्तरम् आचमनीयं समर्पयामि।\\

\twolineshloka*
{हित्वा पीताम्बरं चीर-कृष्णाजिन-धराच्युत}
{परिधत्स्वाद्य मे वस्त्रं स्वर्ण-सूत्र-विनिर्मितम्}
\textbf{\devAya{} वस्त्रं समर्पयामि।\\}

\twolineshloka*
{राजर्षि-वंश-तिलक रामचन्द्र नमोऽस्तु ते}
{यज्ञोपवीतं विधिना निर्मितं धत्स्व मे प्रभो}
\textbf{\devAya{} उपवीतं समर्पयामि।\\}

\twolineshloka*
{किरीटादीनि राजेन्द्र हंसकान्तानि राघव}
{विभूषणानि धृत्वाऽद्य शोभस्व सह सीतया}
\textbf{\devAya{} आभरणम् समर्पयामि।\\}

\twolineshloka*
{सन्ध्या-समान-रुचिना नीलाभ्र-सम-विग्रह}
{लिम्पामि तेऽङ्गकं राम चन्दनेन मुदा हृदि}
\textbf{\devAya{} गन्धान् धारयामि।\\}
गन्धस्योपरि हरिद्रा-कुङ्कुमं समर्पयमि।\\

\twolineshloka*
{अक्षतान् कुङ्कुमोन्मिश्रान् अक्षय्य-फल-दायक}
{अर्पये तव पादाब्जे शालि-तण्डुल-सम्भवान्}
\textbf{\devAya{} अक्षतान् समर्पयामि।\\}

\twolineshloka*
{चम्पकाशोक-पुन्नागैर्जलजैस्तुलसी-दलैः}
{पूजयामि रघूत्तंस पूज्यं त्वां सनकादिभिः}
\textbf{\devAya{} पुष्पाणि समर्पयामि।}

\dnsub{अङ्ग-गुण-पूजा}
\begin{supertabular}{p{0.525\linewidth}p{0.45\linewidth}}
अहल्या-उद्धारकाय नमः & पाद-रजः पूजयामि \\
शरणागत-रक्षकाय नमः & पाद-कान्तिं पूजयामि \\
गङ्गा-नदी-प्रवर्तन-पराय नमः & पाद-नखान् पूजयामि \\
सीता-संवाहित-पदाय नमः & पाद-तलं पूजयामि \\
दुन्दुभि-काय-विक्षेपकाय नमः & पादाङ्गुष्ठं पूजयामि \\
विनत-कल्प-द्रुमाय नमः & गुल्फौ पूजयामि \\
दण्डकारण्य-गमन-जङ्घालाय नमः & जङ्घे पूजयामि \\
जानु-न्यस्त-कराम्बुजाय नमः & जानुनी पूजयामि \\
वीरासन-अध्यासिने नमः & ऊरू पूजयामि \\
पीताम्बर-अलङ्कृताय नमः & कटिं पूजयामि \\
आकाश-मध्यगाय नमः & मध्यं पूजयामि \\
अरि-निग्रह-पराय नमः & कटि-लम्बितम् असिं पूजयामि \\
अब्धि-मेखला-पतये नमः & मध्य-लम्बित-मेखला-दाम पूजयामि \\
उदर-स्थित-ब्रह्माण्डाय नमः & उदरं पूजयामि \\
जगत्-त्रय-गुरवे नमः & वलि-त्रयं पूजयामि \\
सीतानुलेपित-काश्मीर-चन्दनाय नमः & वक्षः पूजयामि \\
अभय-प्रदान-शौण्डाय नमः & दक्षिण-बाहु-दण्डं पूजयामि \\
वितरण-जित-कल्पद्रुमाय नमः & दक्षिण-कर-तलं पूजयामि \\
आशर-निरसन-पराय नमः & दक्षिण-कर-स्थित-शरं पूजयामि \\
ज्ञान-विज्ञान-भासकाय नमः & चिन्मुद्रां पूजयामि \\
मुनि-सङ्घार्पित-दिव्य-पदाय नमः & वाम-भुज-दण्डं पूजयामि \\
दशानन-काल-रूपिणे नमः & वाम-हस्त-स्थित-कोदण्डं पूजयामि \\
शत-मख-दत्त-शत-पुष्कर-स्रजे नमः & अंसौ पूजयामि \\
कृत्त-दशानन-किरीट-कूटाय नमः & अंस-लम्बित-निषङ्ग-द्वयं पूजयामि \\
सीता-बाहु-लतालिङ्गिताय नमः & कण्ठं पूजयामि \\
स्मित-भाषिणे नमः & स्मितं पूजयामि \\
नित्य-प्रसन्नाय नमः & मुख-प्रसादं पूजयामि \\
सत्य-वाचे नमः & वाचं पूजयामि \\
कपालि-पूजिताय नमः & कपोलौ पूजयामि \\
चक्षुःश्रवः-प्रभु-पूजिताय नमः & श्रोत्रे पूजयामि \\
अनासादित-पाप-गन्धाय नमः & घ्राणं पूजयामि \\
पुण्डरीकाक्षाय नमः & अक्षिणी पूजयामि \\
अपाङ्ग-स्यन्दि-करुणाय नमः & अरुणापाङ्ग-द्वयं पूजयामि \\
विना-कृत-रुषे नमः & अनाथ-रक्षक-कटाक्षं पूजयामि \\
कस्तूरी-तिलकाङ्किताय नमः & फालं पूजयामि \\
राजाधिराज-वेषाय नमः & किरीटं पूजयामि \\
मुनि-मण्डल-पूजिताय नमः & जटा-मण्डलं पूजयामि \\
मोहित-मुनि-जनाय नमः & पुंसां मोहनं रूपं पूजयामि \\
जानकी-व्यजन-वीजिताय नमः & विद्युद्-विद्योतित-कालाभ्र-सदृश-कान्तिं पूजयामि \\
हनुमदर्पित-चूडामणये नमः & करुणारस-उद्वेल्लित-कटाक्ष-धारां पूजयामि \\
सुमन्त्रानुग्रह-पराय नमः & तेजोमयरूपं पूजयामि \\
कम्पिताम्भोधये नमः & आहार्य-कोपं पूजयामि \\
तिरस्कृत-लङ्केश्वराय नमः & धैर्यं पूजयामि \\
वन्दित-जनकाय नमः & विनयं पूजयामि \\
सम्मानित-त्रिजटाय नमः & अतिमानुष-सौलभ्यं पूजयामि \\
गन्धर्व-राज-प्रतिमाय नमः & लोकोत्तर-सौन्दर्यं पूजयामि \\
असहाय-हत-खर-दूषणादि-चतुर्दश-सहस्र-राक्षसाय नमः & पराक्रमं पूजयामि \\
आलिङ्गित-आञ्जनेयाय नमः & भक्त-वात्सल्यं पूजयामि \\
लब्ध-राज्य-परित्यक्त्रे नमः & धर्मं पूजयामि \\
दर्भ-शायिने नमः & लोकानुवर्तनं पूजयामि \\
सर्वेश्वराय नमः & सर्वाण्यङ्गानि सर्वांश्च गुणान् पूजयामि \\
\end{supertabular}

\begingroup
\setlength{\columnseprule}{1pt}
\let\chapt\sect
\centering
\input{../namavali-manjari/100/Rama_108.tex}
\input{../namavali-manjari/100/Rama_Ramarahasya_108.tex}
\input{../namavali-manjari/100/Sita_Ramarahasya_108.tex}
\input{../namavali-manjari/100/Sita_108.tex}
\input{../namavali-manjari/100/Anjaneya_108.tex}
\input{../namavali-manjari/100/Anjaneya_Ramarahasya_108.tex}
\endgroup

\dnsub{उत्तराङ्ग-पूजा}\markboth{उत्तराङ्ग-पूजा}{उत्तराङ्ग-पूजा}

\renewcommand{\devAya}{श्री-सीता-लक्ष्मण-भरत-शत्रुघ्न-हनुमत्-समेत-श्री-रामचन्द्र-परब्रह्मणे नमः}



\twolineshloka*
{वनस्पति-रसोद्भूतो गन्धाढ्यो धूप उत्तमः}
{रामचन्द्र महीपाल धूपोऽयं प्रतिगृह्यताम्}
\textbf{\devAya{} धूपम् आघ्रापयामि।}

\twolineshloka*
{ज्योतिषां पतये तुभ्यं नमो रामाय वेधसे}
{गृहाण मङ्गलं दीपं त्रैलोक्य-तिमिरापहम्}
\textbf{\devAya{} अलङ्कारदीपं सन्दर्शयामि।}

ओं भूर्भुवस्सुवः + ब्रह्मणे स्वाहा।
\twolineshloka*
{इदं दिव्यान्नम् अमृतं रसैः षड्भिः समन्वितम्}
{रामचन्द्रेश नैवेद्यं सीतेश प्रतिगृह्यताम्}

\textbf{\devAya{} नैवेद्यं निवेदयामि। मध्ये मध्ये पानीयं समर्पयामि। निवेदनोत्तरम् आचमनीयं समर्पयामि।}

\twolineshloka*
{नागवल्ली-दलैर्युक्तं पूगी-फल-समन्वितम्}
{ताम्बूलं गृह्यतां राम कर्पूरादि-समन्वितम्}
\textbf{\devAya{} कर्पूरताम्बूलं समर्पयामि।}

\twolineshloka*
{मङ्गलार्थं महीपाल नीराजनमिदं हरे}
{सङ्गृहाण जगन्नाथ रामचन्द्र नमोऽस्तु ते}
\textbf{\devAya{} समस्त-अपराध-क्षमापणार्थंं समस्त-दुरित-उपशान्त्यर्थं समस्त-सन्मङ्गल-अवाप्त्यर्थं कर्पूर-नीराजनं दर्शयामि। रक्षां धारयामि।}


\twolineshloka*
{कल्पवृक्ष-समुद्भूतैः पुरुहूतादिभिः सुमैः}
{पुष्पाञ्जलिं ददाम्यद्य पूजिताय आशर-द्विषे}

यो॑ऽपां पुष्पं॒ वेद॑। पुष्प॑वान् प्र॒जावा᳚न् पशु॒मान् भ॑वति।\\
च॒न्द्रमा॒ वा अ॒पां पुष्पम्᳚। पुष्प॑वान् प्र॒जावा᳚न् पशु॒मान् भ॑वति।\\
य ए॒वं वेद॑। यो॑ऽपामा॒यत॑नं॒ वेद॑। आ॒यत॑नवान् भवति।\medskip

ओं᳚ तद्ब्र॒ह्म। ओं᳚ तद्वा॒युः। ओं᳚ तदा॒त्मा।\\ ओं᳚ तथ्स॒त्यम्‌।
ओं᳚ तथ्सर्वम्᳚‌। ओं᳚ तत्पुरो॒र्नमः॥\medskip

अन्तश्चरति॑ भूते॒षु॒ गुहायां वि॑श्वमू॒र्तिषु। \\
त्वं यज्ञस्त्वं वषट्कारस्त्वमिन्द्रस्त्वꣳ\\ रुद्रस्त्वं विष्णुस्त्वं ब्रह्म त्वं॑ प्रजा॒पतिः। \\
त्वं त॑दाप॒ आपो॒ ज्योती॒ रसो॒ऽमृतं॒ ब्रह्म॒ भूर्भुवः॒ सुव॒रोम्‌॥

\textbf{\devAya{} वेदोक्तमन्त्रपुष्पाञ्जलिं समर्पयामि।}

\twolineshloka*
{मन्दाकिनी-समुद्भूत-काञ्चनाब्ज-स्रजा विभो}
{सम्मानिताय शक्रेण स्वर्ण-पुष्पं ददामि ते}
\textbf{\devAya{} स्वर्ण-पुष्पम् समर्पयामि।}

\twolineshloka*
{चराचरं व्याप्नुवन्तम् अपि त्वां रघु-नन्दन}
{प्रदक्षिणं करोम्यद्य मदग्रे मूर्ति-संयुतम्}

\textbf{\devAya{} प्रदक्षिणं करोमि।}

\fourlineindentedshloka*
{ध्येयं सदा परिभव-घ्नम् अभीष्ट-दोहं}
{तीर्थास्पदं शिव-विरिञ्चि-नुतं शरण्यम्}
{भृत्यार्ति-हं प्रणत-पाल-भवाब्धि-पोतं}
{वन्दे महापुरुष ते चरणारविन्दम्}

\fourlineindentedshloka*     
{त्यक्त्वा सुदुस्त्यज-सुरेप्सित-राज्य-लक्ष्मीं}
{धर्मिष्ठ आर्य-वचसा यदगाद् अरण्यम्}
{माया-मृगं दयितयेप्सितम् अन्वधावत्}
{वन्दे महापुरुष ते चरणारविन्दम्}

\twolineshloka*
{साङ्गोपाङ्गाय साराय जगतां सनकादिभिः}
{वन्दिताय वरेण्याय राघवाय नमो नमः } 

\textbf{\devAya{} नमस्कारान् समर्पयामि।}

\sect{प्रार्थना}
\resetShloka
\threelineshloka
{त्वमक्षरोऽसि भगवन् व्यक्ताव्यक्त-स्वरूप-धृत्}
{यथा त्वं रावणं हत्वा यज्ञ-विघ्न-करं खलम्}
{लोकान् रक्षितवान् राम तथा मन्मानसाश्रयम्}

\twolineshloka
{रजस्तमश्च निर्हत्य त्वत्पूजालस्य-कारकम्}
{सत्त्वम् उद्रेचय विभो त्वत्पूजादर-सिद्धये}

\twolineshloka
{विभूतिं वर्षय गृहे पुत्रपौत्राभिवृद्धिकृत्}
{कल्याणं कुरु मे नित्यं कैवल्यं दिश चान्ततः}

\twolineshloka
{विधितोऽविधितो वाऽपि या पूजा क्रियते मया}
{तां त्वं सन्तुष्टहृदयो यथावद् विहितामिव}

\twolineshloka
{स्वीकृत्य परमेशान मात्रा मे सह सीतया}
{लक्ष्मणादिभिरप्यत्र प्रसादं कुरु मे सदा}

प्रार्थनाः समर्पयामि॥

\begin{center}
\input{../stotra-sangrahah/stotras/rama/SitaRamaStotram.tex}
\end{center}
\markboth{उत्तराङ्ग-पूजा}{उत्तराङ्ग-पूजा}

\twolineshloka*
{एकातपत्रच्छायायां शासिताशेषभूमिक}
{मम छत्रमिदं रत्नजालकं राम गृह्यताम्}
\textbf{\hfill छत्रम् समर्पयामि।}


\twolineshloka*
{रक्षोराजानुजाभ्यां ते कृतं चामरसेवया}
{वीजयेऽहं कराभ्यां ते चामरद्वयमादरात्}
\textbf{\hfill चामरम् वीजयामि।}


\twolineshloka*
{रामायणं साधु गीतं सुताभ्यां श्रुतवानसि}
{मयाऽपि गीयमानं ते स्तोत्रं चित्ताय रोचताम्}
\textbf{\hfill गीतम् गायामि।}


\twolineshloka*
{वीणावेणुमृदङ्गादिवाद्यैस्त्वां प्रीणयाम्यहम्}
{मददम्भाहङ्कृतीनां नाशको भव राघव}
\textbf{\hfill वाद्यम् घोषयामि।}


\twolineshloka*
{आरुह्य सीतया सार्धं दत्तामान्दोलिकां मया}
{विभाहि भूषितो राम मत्कृते पूजनोत्सवे}
\textbf{\hfill आन्दोलिकां समर्पयामि।}


\twolineshloka*
{मया कल्पितपल्याणं महान्तं मम घोटकम्}
{मदंसे चरणं न्यस्य मुदाऽऽरोह रघूत्तम}
\textbf{\hfill अश्वान् आरोहयामि।}


\twolineshloka*
{गजेन महताऽऽयान्तमाकाङ्क्षन्ति स्म नागराः}
{द्रष्टुं त्वां मगजे भाहि दृष्ट्वा नन्देयमप्यहम्}
\textbf{\hfill गजान् आरोहयामि।}

\textbf{समस्तराजोपचारदेवोपचारपूजाः समर्पयामि।}


\twolineshloka*
{मनसा वचसा कायेनागसां शतमन्वहम्}
{धियाऽधिया च रचये क्षमस्व सहजक्षम}

\twolineshloka*
{आवाहनं न जानामि न जानामि विसर्जनम्}
{पूजाविधिं न जानामि क्षमस्व पुरुषोत्तम}


\dnsub{अर्घ्य-प्रदानम्}

\twolineshloka*
{शुक्लाम्बरधरं विष्णुं शशिवर्णं चतुर्भुजम्}
{प्रसन्नवदनं ध्यायेत् सर्वविघ्नोपशान्तये}


प्राणान्  आयम्य।  ॐ भूः + भूर्भुवः॒ सुव॒रोम्।

ममोपात्त + प्रीत्यर्थम् अद्य पूर्वोक्त + शुभतिथौ श्रीरामचन्द्रपूजान्ते अर्घ्यप्रदानं करिष्ये (इति सङ्कल्प्य)।

\twolineshloka*
{राम रात्रिञ्चराराते क्षीरमध्वाज्यकल्पितम्}
{पूजान्तेऽर्घ्यं मया दत्तं स्वीकृत्य वरदो भव}

\devAya{} इदमर्घ्यं इदमर्घ्यं इदमर्घ्यम्॥

अनेनार्ध्यप्रदानेन श्री-सीता-लक्ष्मण-भरत-शत्रुघ्न-हनुमत्-समेत-श्री-रामचन्द्रः प्रीयताम्।

\twolineshloka*
{हिरण्यगर्भगर्भस्थं हेमबीजं विभावसोः}
{अनन्तपुण्यफलदम् अतः शान्तिं प्रयच्छ मे}

श्री-रामनवमी-पुण्यकाले अस्मिन् मया क्रियमाण-श्रीरामपूजायां यद्देयमुपायनदानं तत्प्रतिनिधित्वेन हिरण्यं श्री-सीता-लक्ष्मण-भरत-शत्रुघ्न-हनुमत्-समेत-श्री-रामचन्द्र-प्रीतिं 
कामयमानः मनसोद्दिष्टाय ब्राह्मणाय सम्प्रददे नमः न मम। 

अनया पूजया श्री-सीता-लक्ष्मण-भरत-शत्रुघ्न-हनुमत्-समेत-श्री-रामचन्द्रः प्रीयताम्। 
 
\dnsub{विसर्जनम्}
\twolineshloka*
{यस्य स्मृत्या च नामोक्त्या तपः-पूजा-क्रियादिषु}
{न्यूनं सम्पूर्णतां याति सद्यो वन्दे तमच्युतम्} 

\twolineshloka*
{इदं व्रतं मया देव कृतं प्रीत्यै तव प्रभो}
{न्यूनं सम्पूर्णतां यातु त्वत्प्रसादाज्जनार्द्दन}

अस्मात् बिम्बात् श्री-सीता-लक्ष्मण-भरत-शत्रुघ्न-हनुमत्-समेत-श्री-रामचन्द्रं यथास्थानं प्रतिष्ठापयामि।\\
(अक्षतानर्पित्वा देवमुत्सर्जयेत्।)\\

\fourlineindentedshloka*
{कायेन वाचा मनसेन्द्रियैर्वा}
{बुद्‌ध्याऽऽत्मना वा प्रकृतेः स्वभावात्}
{करोमि यद्यत् सकलं परस्मै}
{नारायणायेति समर्पयामि}

अनया पूजया श्री-सीता-लक्ष्मण-भरत-शत्रुघ्न-हनुमत्-समेत-श्री-रामचन्द्रः प्रीयताम्। \\
ॐ तत्सद्ब्रह्मार्पणमस्तु।

\end{center}

\closesub

\ifbool{katha}{\input{kathas/sriramanavami-vrata-katha}}{}

\closesection