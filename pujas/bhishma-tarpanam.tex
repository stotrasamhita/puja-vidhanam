% !TeX program = XeLaTeX
% !TeX root = pUjA.tex
\chapt{भीष्म-तर्पणम्}

{\sffamily Shri Bhishma Pitamaha whose very nature was dharma undertook a vow of Brahmacharya and lived as a Brahmachari throughout his life. Hence, he did not have any descendants. He had the boon of icchamrtyu (choosing when he could leave his body). So he waited until Uttarayanam and shed his mortal on Shukla Ashtami in Magha masa. Therefore, all of us should offer Tarpanam and Arghyadanam for the sake of Bhishma on Magha masa Shukla Ashtami. Laugakshi Smrti says that those who offer this Bhishma Tarpanam gets as much punya as feeding a thousand brahmanas.}

\twolineshloka*
{अष्टमीदिवसे चैव भीष्मतर्पणमाचरेत्}
{दद्यात् प्रदद्याद् भीष्माय तर्पणं प्रतिवत्सरम्}

\twolineshloka*
{तेन तर्पणमात्रेण सहस्रद्विजभोजने}
{यत्फलं कथितं सद्भिस्तदवाप्नोत्यसंशयः}

\twolineshloka*
{शुक्लाष्टम्यां तु माघस्य दद्याद्भीष्माय यो जलम्}
{संवत्सरकृतं पापं तत्क्षणादेव नश्यति}
 
\twolineshloka*
{ब्राह्मणाद्याश्च ये वर्णाः दद्युर्भीष्माय नो जलम्}
{संवत्सरकृतं तेषां पुण्यं नश्यति सत्तम}

{\sffamily Padma Puranam also warns that those who do not not offer this Tarpana to Bhishma on Bhishmashtami, lose the punya accumulated during the entire year. Hence we should all offer Tarpanam and Arghyadanam for the sake of Pitamaha Bhishma and strive to become those who perform their duties well.}

\faClockO{} {\sffamily After completion of the anushthanas of nityakarmas such as morning Snanam, Sandhyavandanam everyone should do Tarpanam and Arghyadanam for Bhishma.}

\shuklambaradharam

\pranayama

% \renewcommand{\prityartham}{श्री-छाया-सुवर्चलाम्बा-समेत-श्री-सूर्यनारायण-प्रीत्यर्थं}
% \renewcommand{\additionalSankalpa}{श्री-छाया-सुवर्चलाम्बा-समेत-श्री-सूर्यनारायण-प्रसाद-सिद्ध्यर्थं श्री-सूर्यनारायण-प्रसादेन अरोगदृढगात्रता-सिद्ध्यर्थं}


ममोपात्त-समस्त-दुरित-क्षयद्वारा श्री-परमेश्वर-प्रीत्यर्थं शुभे शोभने मुहूर्ते अद्य ब्रह्मणः
द्वितीयपरार्धे श्वेतवराहकल्पे वैवस्वतमन्वन्तरे अष्टाविंशतितमे कलियुगे प्रथमे पादे
जम्बूद्वीपे भारतवर्षे भरतखण्डे मेरोः दक्षिणे पार्श्वे शकाब्दे अस्मिन् वर्तमाने व्यावहारिकाणां 
प्रभवादीनां षष्ट्याः संवत्सराणां मध्ये

\textbf{\blank\see{app:samvatsara_names}} नाम संवत्सरे
\textbf{उत्तरायणे}
\textbf{शिशिर}-ऋतौ
\textbf{मकर-माघ}-मासे 
\textbf{शुक्ल}पक्षे
\textbf{अष्टम्यां} शुभतिथौ
\textbf{\vasara}-वासरयुक्तायां
\textbf{\nakshatra}-नक्षत्र-%
\textbf{\yoga}-योग-%
\textbf{भद्रा/बव}-करण-युक्तायां
च एवं गुण-विशेषण-विशिष्टायाम् अस्याम्\\
\textbf{अष्टम्यां} शुभतिथौ 

ममोपात्त-समस्त-दुरितक्षयद्वारा श्रीपरमेश्वर-प्रीत्यर्थं भीष्माष्टमी-पुण्यकाले भीष्मतर्पणम् अर्घ्यदानं च करिष्ये।


\centerline{\bfseries जीवत्पिताऽपि कुर्वीत तर्पणं यमभीष्मयोः}\nobreak

\centerline{\sffamily \emph{Those whose father is living must also offer Tarpanam for Yama and Bhishma}}

{\sffamily According to this vachana, those whose father is alive should also offer Tarpanam to Bhishma. Those who do not have a father should wear the Yajnopavitam on the left and offer tarpanam via the Pitr Tirtha with water mixed with tila (sesame). Those whose father is alive should offer Tarpanam with just as in Pitr Tarpanam in Brahmayajna (with Yajnopavitam upto the left wrist and with only water) with Pitr Tirtha.}

उदकदानमन्त्रः—
\threelineshloka*
{वैयाघ्रपादगोत्राय साङ्कृत्यप्रवराय च}
{गङ्गापुत्राय भीष्माय प्रदास्येऽहं तिलोदकम्}
{अपुत्राय ददाम्येतत् सलिलं भीष्मवर्मणे}
भीष्मं तर्पयामि। भीष्मं तर्पयामि। भीष्मं तर्पयामि॥

{\sffamily With the following Shlokas, everyone should offer Arghyadanam with water three times similar to how Arghyadanam is done in Sandhyavandanam.}

\twolineshloka*
{सत्यव्रताय शुचये गाङ्गेयाय महात्मने}
{अर्घ्यं ददामि भीष्माय सोमवंशोद्भवाय च}

\centerline{भीष्माय नमः इदमर्घ्यम्। भीष्माय नमः इदमर्घ्यम्। भीष्माय नमः इदमर्घ्यम्।}

\twolineshloka*
{वसूनामवताराय शन्तनोरात्मजाय च}
{अर्घ्यं ददामि भीष्माय आजन्मब्रह्मचारिणे}

\centerline{भीष्माय नमः इदमर्घ्यम्। भीष्माय नमः इदमर्घ्यम्। भीष्माय नमः इदमर्घ्यम्।}


\dnsub{प्रार्थना}

\twolineshloka*
{भीष्मः शान्तनवो वीरः सत्यवादी जितेन्द्रियः}
{आभिरद्भिरवाप्नोतु पुत्रपौत्रोचिताः क्रियाः}

{\sffamily With this shloka, offer Prarthana to Bhishma.}

\fourlineindentedshloka*
{कायेन वाचा मनसेन्द्रियैर्वा}
{बुद्‌ध्याऽऽत्मना वा प्रकृतेः स्वभावात्}
{करोमि यद्यत् सकलं परस्मै}
{नारायणायेति समर्पयामि}

अनेन मया कृतेन \emph{भीष्माष्टमी-पुण्यकाले भीष्मतर्पणेन} परमात्मा सुप्रीतः सुप्रसन्नो वरदो भवतु॥

\centerline{ॐ तत्सद्ब्रह्मार्पणमस्तु॥}

\closesub

\begingroup
\let\chapt\sect
\begin{center}
    \input{../stotra-sangrahah/stotras/krishna/BhishmaStuti.tex}
    \input{../stotra-sangrahah/stotras/krishna/BhishmaStavaraja.tex}
\end{center}
\endgroup


\closesection