% !TeX program = XeLaTeX
% !TeX root = ../pujavidhanam.tex

\setlength{\parindent}{0pt}
\chapt{श्री-लक्ष्मी--कुबेर-पूजा}

\input{purvanga/vighneshwara-puja}
 
\sect{प्रधान-पूजा — श्रीमहालक्ष्मी-पूजा}

\twolineshloka*
{शुक्लाम्बरधरं विष्णुं शशिवर्णं चतुर्भुजम्}
{प्रसन्नवदनं ध्यायेत् सर्वविघ्नोपशान्तये}
 
प्राणान्  आयम्य।  ॐ भूः + भूर्भुवः॒ सुव॒रोम्।

\dnsub{सङ्कल्पः}

ममोपात्त-समस्त-दुरित-क्षयद्वारा श्री-परमेश्वर-प्रीत्यर्थं शुभे शोभने मुहूर्ते अद्य ब्रह्मणः
द्वितीयपरार्धे श्वेतवराहकल्पे वैवस्वतमन्वन्तरे अष्टाविंशतितमे कलियुगे प्रथमे पादे
जम्बूद्वीपे भारतवर्षे भरतखण्डे मेरोः दक्षिणे पार्श्वे शकाब्दे अस्मिन् वर्तमाने व्यावहारिकाणां प्रभवादीनां षष्ट्याः संवत्सराणां मध्ये \mbox{(~~~)}\see{app:samvatsara_names} नाम संवत्सरे दक्षिणायने 
शरद्-ऋतौ तुला-मासे कृष्णपक्षे अमावास्यायां शुभतिथौ
(इन्दु / भौम / बुध / गुरु / भृगु / स्थिर / भानु) वासरयुक्तायाम्
\mbox{(~~~)}\see{app:nakshatra_names} नक्षत्र \mbox{(~~~)}\see{app:yoga_names} नाम  योग  (चतुष्पात्/नागव) करण युक्तायां च एवं गुण विशेषण विशिष्टायाम्
अस्याम् अमावास्यायां शुभतिथौ 
अस्माकं सहकुटुम्बानां क्षेमस्थैर्य-धैर्य-वीर्य-विजय-आयुरारोग्य-ऐश्वर्याभिवृद्ध्यर्थम्
 धर्मार्थकाममोक्ष\-चतुर्विधफलपुरुषार्थसिद्ध्यर्थं पुत्रपौत्राभि\-वृद्ध्यर्थम् इष्टकाम्यार्थसिद्ध्यर्थम्
मम इहजन्मनि पूर्वजन्मनि जन्मान्तरे च सम्पादितानां ज्ञानाज्ञानकृतमहा\-पातकचतुष्टय-व्यतिरिक्तानां रहस्यकृतानां प्रकाशकृतानां सर्वेषां पापानां सद्य अपनोदनद्वारा सकल-पापक्षयार्थं
श्रीमहालक्ष्मी-प्रीत्यर्थं श्रीमहालक्ष्मी-पूजां करिष्ये। तदङ्गं मातृगणपूजां नवग्रहपूजां लोकपाल-पूजां च करिष्ये। 
तदङ्गं कलशपूजां च करिष्ये। 


श्रीविघ्नेश्वराय नमः यथास्थानं प्रतिष्ठापयामि। शोभनार्थे क्षेमाय पुनरागमनाय च।\\
(गणपति-प्रसादं शिरसा गृहीत्वा)

\input{purvanga/aasana-puja}

\input{purvanga/ghanta-puja}

\input{purvanga/kalasha-puja}

\input{purvanga/aatma-puja}

\input{purvanga/pitha-puja}

\input{purvanga/guru-dhyanam}

\input{purvanga/matrugana-puja}

\input{purvanga/navagraha-puja}

\input{purvanga/lokapala-puja}

\sect{षोडशोपचार-पूजा}
\begin{center}

\fourlineindentedshloka*
{अरुणकमलसंस्था तद्रजःपुञ्जवर्णा}
{करकमलधृतेष्टाऽभीतियुग्माम्बुजा च}
{मणिमकुटविचित्रालङ्कृता कल्पजातैः}
{भवतु भुवनमाता सन्ततं श्रीः श्रियै नः}

हिर॑ण्यवर्णां॒ हरि॑णीं सुव॒र्णर॑जत॒स्रजाम्।\\
च॒न्द्रां॒ हि॒रण्म॑यीं ल॒क्ष्मीं॒ जात॑वेदो म॒ आव॑ह॥१॥

अस्मिन् बिम्बे श्रीमहालक्ष्मीं ध्यायामि।

\twolineshloka*
{आवाहये महालक्ष्मि चैतन्यस्तन्यदायिनि}
{विष्णुपत्नि जगन्मातः पूजां गृह्णीष्व ते नमः}
श्रीमहालक्ष्मीम् आवाहयामि।

तां म॒ आव॑ह॒ जात॑वेदो ल॒क्ष्मीमन॑पगा॒मिनी᳚म्।\\
यस्यां॒ हिर॑ण्यं वि॒न्देयं॒ गामश्वं॒ पुरु॑षान॒हम्॥२॥


\twolineshloka*
{तप्तकाञ्चनवर्णाभं मुक्तामणिविराजितम्}
{अमलं कमलं दिव्यम् आसनं प्रतिगृह्यताम्}
 आसनं समर्पयामि।\medskip

अ॒श्व॒पू॒र्वां र॑थम॒ध्यां॒ ह॒स्तिना॑दप्र॒बोधि॑नीम्।\\
श्रियं॑ दे॒वीमुप॑ह्वये॒ श्रीर्मा॑दे॒वीर्जु॑षताम्॥३॥


\twolineshloka*
{गङ्गातीर्थ-समुद्भूतं गन्ध-पुष्पादिभिर्युतम्}
{पाद्यं ददाम्यहं देवि गृहाणाऽऽशु नमोऽस्तु ते}
 पाद्यं समर्पयामि।\medskip

कां॒ सो॒ऽ॒स्मि॒तां हिर॑ण्यप्राकारामा॒र्द्रां ज्वल॑न्तीं तृ॒प्तां त॒र्पय॑न्तीम्।
प॒द्मे॒ स्थि॒तां प॒द्मव॑र्णां॒ तामि॒होप॑ह्वये॒ श्रियम्॥४॥

\twolineshloka*
{एलागन्धसमायुक्तं स्वर्णपात्रे प्रपूरितम्}
{अर्घ्यं गृहाण मद्दत्तं प्रसीद त्वं महेश्वरि}
 अर्घ्यं समर्पयामि।\medskip

च॒न्द्रां प्र॑भा॒सां य॒शसा॒ ज्वल॑न्तीं॒ श्रियं॑ लो॒के दे॒वजु॑ष्टामुदा॒राम्।
तां प॒द्मिनी॑मीं॒ शर॑णम॒हं प्रप॑द्येऽल॒क्ष्मीर्मे॑ नश्यतां॒ त्वां वृ॑णे॥५॥


\twolineshloka*
{सर्वलोकस्य या शक्तिः ब्रह्मरुद्रादिभिः स्तुता}
{ददाम्याचमनं तस्यै महालक्ष्म्यै मनोहरम्}
 आचमनीयं समर्पयामि।\medskip

 आ॒दि॒त्यव॑र्णे॒ तप॒सोऽधि॑जा॒तो वन॒स्पति॒स्तव॑ वृ॒क्षोऽथ बि॒ल्वः।
तस्य॒ फला॑नि॒ तप॒सा नु॑दन्तु मा॒यान्त॑रा॒याश्च॑ बा॒ह्या अ॑ल॒क्ष्मीः॥६॥


घृतेन स्नपयामि। पुनः शुद्धोदकं समर्पयामि।\\
पयसा स्नपयामि। पुनः शुद्धोदकं समर्पयामि।\\
दध्ना स्नपयामि। पुनः शुद्धोदकं समर्पयामि।\\
मधुना स्नपयामि। पुनः शुद्धोदकं समर्पयामि।\\
पञ्चामृतेन स्नपयामि। पुनः शुद्धोदकं समर्पयामि।

(कलशजलेन श्री-सूक्तं जप्य) शुद्धोदकस्नानं समर्पयामि।
स्नानानन्तरम् आचमनीयं समर्पयामि।\medskip

उपै॑तु॒ मां दे॑वस॒खः की॒र्तिश्च॒ मणि॑ना स॒ह।
प्रा॒दु॒र्भू॒तोऽस्मि॑ राष्ट्रे॒ऽ॒स्मि॒न्॒ की॒र्तिमृद्धिं॑ ददा॒तु मे॥७॥


\twolineshloka*
{दिव्याम्बरयुगं सूक्ष्मं कञ्चुकं च मनोहरम्}
{महालक्ष्मि महादेवि गृहाणेदं मयाऽर्पितम्}
 वस्त्रं समर्पयामि।\medskip

 क्षुत्पि॑पा॒साम॑लां ज्ये॒ष्ठा॒मल॒क्ष्मीं ना॑शया॒म्यहम्।
अभू॑ति॒\-मस॑मृद्धिं॒ च सर्वां॒ निर्णु॑द मे॒ गृहात्॥८॥


\twolineshloka*
{माङ्गल्यमणिसंयुक्तं मुक्ताविद्रुमसंयुतम्}
{दत्तं मङ्गलसूत्रं च गृहाण हरिवल्लभे}
कण्ठसूत्रं समर्पयामि।\medskip


\twolineshloka*
{रत्नकङ्कणवैडूर्य-मुक्ताहारादिकानि च}
{सुप्रसन्नेन मनसा दत्तानि त्वं गृहाण मे}
आभरणानि समर्पयामि।\medskip


ग॒न्ध॒द्वा॒रां दु॑राध॒र्॒‌षां॒ नि॒त्यपु॑ष्टां करी॒षिणी᳚म्।
ई॒श्वरीं᳚ सर्व॑भूता॒नां॒ तामि॒होप॑ह्वये॒ श्रियम्॥९॥

\twolineshloka*
{सिन्दूरारुणवर्णा च सिन्दूरतिलकप्रिया}
{अतो दत्तं मया देवि सिन्दूरं प्रतिगृह्यताम्}
 तिलकं समर्पयामि। \medskip


मन॑सः॒ काम॒माकू॑तिं वा॒चः स॒त्यम॑शीमहि।
प॒शू॒नां रू॒पमन्न॑स्य॒ मयि॒ श्रीः श्र॑यतां॒ यशः॑॥१०॥

\twolineshloka*
{मन्दार-पारिजाताद्याः पाटली केतकी तथा}
{माकन्दं कुरवं चैव गृहाणाऽऽशु नमोऽस्तु ते}
  पुष्पमालां धारयामि। 
\end{center}
\dnsub{अङ्ग-पूजा}
\begin{longtable}{ll@{— }l}
१.& ॐ चपलायै नमः & पादौ पूजयामि \\
२.& चञ्चलायै नमः & जानुनी पूजयामि\\
३.& कमलायै नमः & कटिं पूजयामि  \\
४.& कात्यायन्यै नमः & नाभिं पूजयामि\\
५.& जगन्मात्रे नमः & जठरं पूजयामि   \\
६.& विश्ववल्लभायै नमः & वक्षःस्थलं पूजयामि \\
७.& कमलवासिन्यै नमः & हस्तौ पूजयामि        \\
८.& पद्माननायै नमः & मुखं पूजयामि\\
९.& कमलपत्राक्ष्यै नमः & नेत्रत्रयं पूजयामि    \\
१०.& श्रियै नमः & शिरः पुजयामि\\
११.& महालक्ष्म्यै नमः & सर्वाणि अङ्गानि पूजयामि   \\
\end{longtable}

\dnsub{अष्टलक्ष्मी-अर्चना}
(प्राच्याम् आरभ्य अष्टदिक्षु प्रदक्षिणेन)

\begin{multicols}{2}
\begin{enumerate}
\item ॐ आद्यलक्ष्म्यै नमः
\item ॐ विद्यालक्ष्म्यै नमः
\item ॐ सौभाग्यलक्ष्म्यै नमः
\item ॐ अमृतलक्ष्म्यै नमः
\item ॐ कामलक्ष्म्यै नमः 
\item ॐ सत्यलक्ष्म्यै नमः
\item ॐ भोगलक्ष्म्यै नमः
\item ॐ योगलक्ष्म्यै नमः
\end{enumerate}
\end{multicols}

आद्यादिलक्ष्मीनां षोडशोपचार-पूजार्थे पुष्पाणि समर्पयामि।

\begingroup
\setlength{\columnseprule}{1pt}
\let\chapt\sect
\input{../namavali-manjari/100/Lakshmi_108.tex}
\endgroup


\sect{उत्तराङ्ग-पूजा}

\begin{center}

क॒र्दमे॑न प्र॑जाभू॒ता॒ म॒यि॒ सम्भ॑व क॒र्दम।
श्रियं॑ वा॒सय॑ मे कु॒ले मा॒तरं॑ पद्ममा॒लिनीम्॥११॥

\twolineshloka*
{वनस्पति-रसोत्पन्नो गन्धाढ्यो गन्ध उत्तमः}
{आघ्रेयः सर्वदेवानां धूपोऽयं प्रतिगृह्यताम्}
श्री-महालक्ष्म्यै नमः धूपमाघ्रापयामि।\\

आपः॑ सृ॒जन्तु॑ स्निग्धा॒नि॒ चिक्ली॒त व॑स मे॒ गृहे।
नि च॑ दे॒वीं मा॒तरं॒ श्रियं॑ वा॒सय॑ मे कु॒ले॥१२॥

\twolineshloka*
{कार्पासवर्तिसंयुक्तं घृतयुक्तं मनोहरम्}
{तमोनाशकरं दीपं गृहाण परमेश्वरि}
श्री-महालक्ष्म्यै नमः अलङ्कारदीपं सन्दर्शयामि।\\


ॐ भूर्भुवः॒ सुवः॑। + ब्र॒ह्मणे॒ स्वाहा᳚।

आ॒र्द्रां पु॒ष्करि॑णीं पु॒ष्टिं॒ सु॒व॒र्णां हे॑ममा॒लिनीम्।
सू॒र्यां हि॒रण्म॑यीं ल॒क्ष्मीं॒ जात॑वेदो म॒ आव॑ह॥१३॥

\twolineshloka*
{नैवेद्यं गृह्यतां लक्ष्मि भक्ष्य-भोज्य-समन्वितम्}
{षड्रसैर्रचितं दिव्यं लक्ष्मीदेवि नमोऽस्तु ते}
नैवेद्यम्\\
- श्री-महालक्ष्म्यै नमः (	) निवेदयामि, \\
अमृतापिधानमसि। निवेदनानन्तरम् आचमनीयं समर्पयामि।\\

पूगीफलसमायुक्तं नागवल्लीदलैर्युतम्।\\
कर्पूरचूर्णसंयुक्तं ताम्बूलं प्रतिगृह्यताम्॥\\
श्री-महालक्ष्म्यै नमः कर्पूरताम्बूलं समर्पयामि।\\

श्री-महालक्ष्म्यै नमः समस्त-अपराध-क्षमापनार्थं कर्पूरनीराजनं दर्शयामि।\\
कर्पूरनीरजनानन्तरम् आचमनीयं समर्पयामि।\\

 यो॑ऽपां पुष्पं॒ वेद॑। पुष्प॑वान् प्र॒जावा᳚न् पशु॒मान् भ॑वति।\\
च॒न्द्रमा॒ वा अ॒पां पुष्पम्᳚। पुष्प॑वान् प्र॒जावा᳚न् पशु॒मान् भ॑वति।\\
य ए॒वं वेद॑। यो॑ऽपामा॒यत॑नं॒ वेद॑। आ॒यत॑नवान् भवति।\\

ओं᳚ तद्ब्र॒ह्म। ओं᳚ तद्वा॒युः। ओं᳚ तदा॒त्मा। ओं᳚ तथ्स॒त्यम्‌।\\
ओं᳚ तथ्सर्वम्᳚‌। ओं᳚ तत्पुरो॒र्नमः॥\\

अन्तश्चरति॑ भूते॒षु॒ गुहायां वि॑श्वमू॒र्तिषु। \\
त्वं यज्ञस्त्वं वषट्कारस्त्वमिन्द्रस्त्वꣳ रुद्रस्त्वं विष्णुस्त्वं ब्रह्म त्वं॑ प्रजा॒पतिः। \\
त्वं त॑दाप॒ आपो॒ ज्योती॒ रसो॒ऽमृतं॒ ब्रह्म॒ भूर्भुव॒स्सुव॒रोम्‌॥\\

श्री-महालक्ष्म्यै नमः वेदोक्तमन्त्रपुष्पाञ्जलिं समर्पयामि।\\

स्वर्णपुष्पं समर्पयामि\\
 
अनन्तकोटिप्रदक्षिणनमस्कारान् समर्पयामि\\

छत्त्रचामरादिसमस्तोपचारान् समर्पयामि\\

\end{center}

\sect{ईशानादि पूजा}

ॐ ईशानाय नमः\\
ॐ शचिने नमः\\
ॐ मरुद्भ्यो नमः\\
ॐ प्रजापतये नमः\\
ॐ विश्वेभ्यो देवेभ्यो नमः\\
ॐ अमरराजाय नमः\\
ॐ सूर्याय नमः\\
ॐ विश्वकर्मणे नमः\\
ॐ गुरवे नमः\\
ॐ अथर्वाङ्गिरोभ्यां नमः\\
ॐ अश्विभ्यां नमः\\
ॐ मित्रावरुणाभ्यां नमः\\
ॐ विष्णवे नमः\\
ॐ ईशानादिभ्यो नमः\\

षोडशोपचार-पूजार्थे पुष्पाणि समर्पयामि।



\sect{कुबेर पूजा}

\twolineshloka*
{धनदाय नमस्तुभ्यं निधिपद्माय ते नमः}
{भवन्तु त्वत्प्रसादान्मे धनधान्यानि सम्पदः}

\twolineshloka*
{कुबेरं पुष्पकगतं निधिभिर्नवभिर्युतम्}
{सुवर्णवर्णं पिङ्गाक्षं मनसा भावयाम्यहम्}

\onelineshloka*
{नरवाहन यक्षेश सर्वपुण्यजनेश्वर}

कुबेराय नमः, षोडशोपचार-पूजां करिष्ये।
कुबेराय नमः, आवाहयामि।


कुबेराय नमः, आसनं समर्पयामि।
कुबेराय नमः, पाद्यं समर्पयामि।
कुबेराय नमः, अर्घ्यं समर्पयामि।
कुबेराय नमः, आचमनीयं समर्पयामि।
कुबेराय नमः, शुद्धोदकस्नानं समर्पयामि। स्नानानन्तरम् आचमनीयं समर्पयामि।
कुबेराय नमः, वस्त्रं समर्पयामि।
कुबेराय नमः, दिव्यपरिमलगन्धान् धारयामि। गन्धस्योपरि हरिद्राकुङ्कुमं समर्पयामि।
कुबेराय नमः, अक्षतान् समर्पयामि।
कुबेराय नमः, पुष्पैः पूजयामि।
कुबेराय नमः, धूपमाघ्रापयामि।
कुबेराय नमः, अलङ्कारदीपं सन्दर्शयामि।
कुबेराय नमः, कदलीफलानि निवेदयामि, \\
कुबेराय नमः, अमृतापिधानमसि। निवेदनानन्तरम् आचमनीयं समर्पयामि।
कुबेराय नमः, कर्पूरताम्बूलं समर्पयामि।
कुबेराय नमः, कर्पूरनीराजनं दर्शयामि।
कुबेराय नमः, कर्पूरनीरजनानन्तरम् आचमनीयं समर्पयामि।
कुबेराय नमः, समस्तोपचारान् समर्पयामि।


% \dnsub{कुबेराष्टोत्तरशतनामावलिः}

\fourlineindentedshloka*
{मनुजबाह्यविमानवरस्तुतं}
{गरुडरत्ननिभं निधिनायकम्}
{शिवसखं मुकुटादिविभूषितं}
{वररुचिं तमहमुपास्महे सदा}

\twolineshloka*
{अगस्त्य देवदेवेश मर्त्यलोकहितेच्छया}
{पूजयामि विधानेन प्रसन्नसुमुखो भव}

\begingroup
\centering
\setlength{\columnseprule}{1pt}
\let\chapt\sect
\input{../namavali-manjari/100/Kubera_108.tex}

\endgroup


\dnsub{नमस्कारः}

\twolineshloka*
{नमस्ते देवदेवेशि नमस्ते ईफ्सितप्रदे}
{नमस्तेऽस्तु जगन्मातः नमस्ते केशवप्रिये}

महालक्ष्म्यै नमः, नमस्करोमि॥

\sect{प्रार्थना}

\twolineshloka*
{दामोदरि नमस्तेऽस्तु नमस्त्रैलोक्यमातृके}
{नमस्तेऽस्तु महालक्ष्मि त्राहि मां परमेश्वरि}

\twolineshloka*
{सर्वदा देहि मे द्रव्यं दानायापि च भुक्तये}
{धनधान्यं धरां हर्षं कीर्तिम् आयुश्च देहि मे}

\twolineshloka*
{यन्मया वाञ्छितं देवि तत्सर्वं सफलं कुरु}
{न बाधन्तां कुकर्माणि सङ्कटं मे निवारय}

\sect{अपराध-क्षमापनम्}

\twolineshloka*
{न्यूनं वाऽप्यगुणं वाऽपि यन्मया मोहितं कृतम्}
{सर्वं तदस्तु सम्पूर्णं त्वत्प्रसादान्महेश्वरि}

\twolineshloka*
{लक्ष्मि त्वत्कृपया नित्यं कृता पूजा तवाऽऽज्ञया}
{स्थिरा भव गृहे ह्यस्मिन् मम सन्तानकर्मणि}

हिरण्यगर्भगर्भस्थं हेमबीजं विभावसोः।\\
अनन्तपुण्यफलदम् अतः शान्तिं प्रयच्छ मे॥\\

आश्वयुज-अमावास्या-पुण्यकालेऽस्मिन् मया क्रियमाण-श्री\-महा\-लक्ष्मी-पूजायां
यद्देयमुपायन\-दानं तत्प्रति\-निधित्वेन हिरण्यं श्री-महा\-लक्ष्मी\-प्रीतिं
कामयमानः मनसोद्दिष्टाय ब्राह्मणाय सम्प्रददे नमः न मम। 
अनया पूजया श्री-महालक्ष्मीः प्रीयताम्। 



\fourlineindentedshloka*
{कायेन वाचा मनसेन्द्रियैर्वा}
{बुद्‌ध्याऽऽत्मना वा प्रकृतेः स्वभावात्}
{करोमि यद्यत् सकलं परस्मै}
{नारायणायेति समर्पयामि}

ॐ तत्सद्ब्रह्मार्पणमस्तु।