% !TeX program = XeLaTeX
% !TeX root = ..\pujavidhanam.tex

\setlength{\parindent}{0pt}
\chapt{श्री-धन्वन्तरि-पूजा}

\input{purvanga/vighneshwara-puja}

\sect{प्रधान-पूजा — धन्वन्तरिपूजा}

\twolineshloka*
{शुक्लाम्बरधरं विष्णुं शशिवर्णं चतुर्भुजम्}
{प्रसन्नवदनं ध्यायेत् सर्वविघ्नोपशान्तये}

प्राणान्  आयम्य।  ॐ भूः + भूर्भुवः॒ सुव॒रोम्।

\dnsub{सङ्कल्पः}

ममोपात्त-समस्त-दुरित-क्षयद्वारा श्री-परमेश्वर-प्रीत्यर्थं शुभे शोभने मुहूत्ते अद्य ब्रह्मणः
द्वितीयपरार्धे श्वेतवराहकल्पे वैवस्वतमन्वन्तरे अष्टाविंशतितमे कलियुगे प्रथमे पादे
जम्बूद्वीपे भारतवर्षे भरतखण्डे मेरोः दक्षिणे पार्श्वे शकाब्दे अस्मिन् वर्तमाने व्यावहारिकाणां प्रभवादीनां षष्ट्याः संवत्सराणां मध्ये \mbox{(~~~)}\see{app:samvatsara_names} नाम संवत्सरे दक्षिणायने 
शरद्-ऋतौ  तुला-मासे कृष्णपक्षे त्रयोदश्यां शुभतिथौ
(इन्दु / भौम / बुध / गुरु / भृगु / स्थिर / भानु) वासरयुक्तायाम्
\mbox{(~~~)}\see{app:nakshatra_names} नक्षत्र \mbox{(~~~)}\see{app:yoga_names} नाम  योग  \mbox{(~~~)} करण युक्तायां च एवं गुण विशेषण विशिष्टायाम्
अस्याम् त्रयोदश्यां शुभतिथौ 
अस्माकं सहकुटुम्बानां क्षेमस्थैर्य-धैर्य-वीर्य-विजय-आयुरारोग्य-ऐश्वर्याभिवृद्ध्यर्थम्
धर्मार्थकाममोक्ष\-चतुर्विधफलपुरुषार्थसिद्ध्यर्थं पुत्रपौत्राभि\-वृद्ध्यर्थम् इष्टकाम्यार्थसिद्ध्यर्थम्
मम इहजन्मनि पूर्वजन्मनि जन्मान्तरे च सम्पादितानां ज्ञानाज्ञानकृतमहा\-पातकचतुष्टय-व्यतिरिक्तानां रहस्यकृतानां प्रकाशकृतानां सर्वेषां पापानां सद्य अपनोदनद्वारा सकल-पापक्षयार्थं
श्रीधन्वन्तरि-देवता-प्रीत्यर्थं श्री-धन्वन्तरि-देवता-प्रीति-पूर्वकम् आयुष्य-आरोग्य-ऐश्वर्य-अभिवृद्ध्यर्थं 
यावच्छक्ति ध्यानावाहनादि 
षोडशोपचार धन्वन्तरि-पूजां करिष्ये तदङ्गं कलशपूजां च करिष्ये।


श्रीविघ्नेश्वराय नमः यथास्थानं प्रतिष्ठापयामि।\\
(गणपति-प्रसादं शिरसा गृहीत्वा)

\input{purvanga/aasana-puja}

\input{purvanga/ghanta-puja}

\input{purvanga/kalasha-puja}

\input{purvanga/aatma-puja}

\input{purvanga/pitha-puja}

\input{purvanga/guru-dhyanam}

\sect{षोडशोपचार-पूजा}
\renewcommand{\devAya}{श्री-धन्वन्तरये नमः,}
\begin{center}

\twolineshloka*
{चतुर्भुजं पीतवस्त्रं सर्वालङ्कारशोभितम्}
{ध्याये धन्वन्तरिं देवं सुरासुरनमस्कृतम्}

\twolineshloka*
{युवानं पुण्डरीकाक्षं सर्वाभरणभूषितम्}
{दधानममृतस्यैव कमण्डलुं श्रिया युतम्}

\twolineshloka*
{यज्ञ-भोग-भुजं देवं सुरासुरनमस्कृतम्}
{ध्याये धन्वन्तरिं देवं श्वेताम्बरधरं शुभम्}
\textbf{अस्मिन् बिम्बे श्री-धन्वन्तरिं ध्यायामि।}
\medskip

\twolineshloka*
{स॒हस्र॑शीर्‌षा॒ पुरु॑षः। स॒ह॒स्रा॒क्षः स॒हस्र॑पात्}
{स भूमिं॑ वि॒श्वतो॑ वृ॒त्वा। अत्य॑तिष्ठद्दशाङ्गु॒लम्}

\textbf{अस्मिन् बिम्बे श्री-धन्वन्तरिम् आवाहयामि।}
\medskip
\twolineshloka*
{पुरु॑ष ए॒वेदꣳ सर्वम्᳚। यद्भू॒तं यच्च॒ भव्यम्᳚}
{उ॒तामृ॑त॒त्वस्येशा॑नः। यदन्ने॑नाति॒रोह॑ति}
\textbf{\devAya{} आसनं समर्पयामि।}\medskip

\twolineshloka*
{ए॒तावा॑नस्य महि॒मा। अतो॒ ज्यायाꣴ॑श्च॒ पूरु॑षः}
{पादो᳚ऽस्य॒ विश्वा॑ भू॒तानि॑। त्रि॒पाद॑स्या॒मृतं॑ दि॒वि}
\textbf{\devAya{} पाद्यं समर्पयामि।}\medskip

\twolineshloka*
{त्रि॒पादू॒र्ध्व उदै॒त्पुरु॑षः। पादो᳚ऽस्ये॒हाऽऽभ॑वा॒त्पुनः॑}
{ततो॒ विश्व॒ङ्व्य॑क्रामत्। सा॒श॒ना॒न॒श॒ने अ॒भि}
\textbf{\devAya{} अर्घ्यं समर्पयामि।}\medskip

\twolineshloka*
{तस्मा᳚द्वि॒राड॑जायत। वि॒राजो॒ अधि॒ पूरु॑षः}
{स जा॒तो अत्य॑रिच्यत। प॒श्चाद्भूमि॒मथो॑ पु॒रः}
\textbf{\devAya{} आचमनीयं समर्पयामि।}\medskip

\twolineshloka*
{यत्पुरु॑षेण ह॒विषा᳚। दे॒वा य॒ज्ञमत॑न्वत}
{व॒स॒न्तो अ॑स्याऽऽसी॒दाज्यम्᳚। ग्री॒ष्म इ॒ध्मः श॒रद्ध॒विः}
\textbf{\devAya{} मधुपर्कं समर्पयामि।}\medskip

\twolineshloka*
{स॒प्तास्याऽऽ॑सन्  परि॒धयः॑। त्रिः स॒प्त स॒मिधः॑ कृ॒ताः}
{दे॒वा यद्य॒ज्ञं त॑न्वा॒नाः। अब॑ध्न॒न् पु॑रुषं प॒शुम्}
\textbf{\devAya{} शुद्धोदकस्नानं समर्पयामि। }\\
स्नानानन्तरम् आचमनीयं समर्पयामि।\medskip

\twolineshloka*
{तं य॒ज्ञं ब॒र्\mbox{}हिषि॒ प्रौक्षन्॑। पुरु॑षं जा॒तम॑ग्र॒तः}
{तेन॑ दे॒वा अय॑जन्त। सा॒ध्या ऋष॑यश्च॒ ये}
\textbf{\devAya{} वस्त्रं समर्पयामि।}\medskip

\twolineshloka*
{तस्मा᳚द्य॒ज्ञात्स॑र्व॒हुतः॑। सम्भृ॑तं पृषदा॒ज्यम्}
{प॒शूꣴस्ताꣴश्च॑क्रे वाय॒व्यान्॑। आ॒र॒ण्यान्ग्रा॒म्याश्च॒ ये}
\textbf{\devAya{} यज्ञोपवीतं समर्पयामि।}\medskip

\twolineshloka*
{तस्मा᳚द्य॒ज्ञात्स॑र्व॒हुतः॑। ऋचः॒ सामा॑नि जज्ञिरे}
{छन्दासि जज्ञिरे॒ तस्मा᳚त्। यजु॒स्तस्मा॑दजायत}
\textbf{\devAya{} दिव्यपरिमलगन्धान् धारयामि। \\
गन्धस्योपरि हरिद्राकुङ्कुमं समर्पयामि। अक्षतान् समर्पयामि।}\medskip

\twolineshloka*
{तस्मा॒दश्वा॑ अजायन्त। ये के चो॑भ॒याद॑तः}
{गावो॑ ह जज्ञिरे॒ तस्मा᳚त्। तस्मा᳚ज्जा॒ता अ॑जा॒वयः॑}
\textbf{\devAya{} पुष्पाणि समर्पयामि। } पुष्पैः पूजयामि।
\medskip

\end{center}

\dnsub{अङ्ग-पूजा}
\begin{longtable}{ll@{— }l}
१.& ॐ वराहाय नमः & पादौ पूजयामि \\
२.& सङ्कर्षणाय नमः & गुल्फौ पूजयामि\\
३.& कालात्मने नमः & जानुनी पूजयामि  \\
४.& विश्वरूपाय नमः & जङ्घे पूजयामि\\
५.& क्रोढाय नमः & ऊरू पूजयामि   \\
६.& भोक्त्रे नमः & कटिं पूजयामि \\
७.& विष्णवे नमः & मेढ्रं पूजयामि        \\
८.& हिरण्यगर्भाय नमः & नाभिं पूजयामि\\
९.& श्रीवत्सधारिणे नमः & कुक्षिं पूजयामि    \\
१०.& परमात्मने नमः & हृदयं पूजयामि\\
११.& सर्वास्त्रधारिणे नमः & वक्षः पूजयामि   \\
१२.& वनमालिने नमः & कण्ठं पूजयामि\\
१३.& सर्वात्मने नमः & मुखं पूजयामि  \\
१४.&     सहस्राक्षाय नमः & नेत्राणि पूजयामि\\
१५.& सुप्रभाय नमः & ललाटं पूजयामि   \\
१६.& चम्पकनासिकाय नमः & नासिकां पूजयामि \\
१७.& सर्वेशाय नमः & कर्णौ पूजयामि   \\
१८.& सहस्रशिरसे नमः & शिरः पूजयामि\\
१९.& नीलमेघनिभाय नमः & केशान् पूजयामि   \\
२०.& महापुरुषाय नमः & सर्वाणि अङ्गानि पूजयामि   \\
\end{longtable}

\dnsub{चतुर्विंशति नामपूजा}
\begin{multicols}{2}
\begin{enumerate}
\item ॐ केशवाय नमः
\item ॐ नारायणाय नमः
\item ॐ माधवाय नमः
\item ॐ गोविन्दाय नमः
\item ॐ विष्णवे नमः 
\item ॐ मधुसूदनाय नमः
\item ॐ त्रिविक्रमाय नमः
\item ॐ वामनाय नमः
\item ॐ श्रीधराय नमः
\item ॐ हृषीकेशाय नमः
\item ॐ पद्मनाभाय नमः
\item ॐ दामोदराय नमः
\item ॐ सङ्कर्षणाय नमः
\item ॐ वासुदेवाय नमः
\item ॐ प्रद्युम्नाय नमः
\item ॐ अनिरुद्धाय नमः
\item ॐ पुरुषोत्तमाय नमः
\item ॐ अधोक्षजाय नमः
\item ॐ नृसिंहाय नमः
\item ॐ अच्युताय नमः
\item ॐ जनार्दनाय नमः
\item ॐ उपेन्द्राय नमः 
\item ॐ हरये नमः
\item ॐ श्रीकृष्णाय नमः
\end{enumerate}
\end{multicols}

\begingroup
\centering
\setlength{\columnseprule}{1pt}
\let\chapt\sect
\input{../namavali-manjari/100/Dhanvantari_108.tex}

\endgroup

 
\sect{उत्तराङ्ग-पूजा}
\begin{center}

\twolineshloka*
{यत्पुरु॑षं॒ व्य॑दधुः। क॒ति॒धा व्य॑कल्पयन्}
{मुखं॒ किम॑स्य॒ कौ बा॒हू। कावू॒रू पादा॑वुच्येते}

\twolineshloka*
{दशाङ्गं गुग्गुलं धूपं सुगन्धं सुमनोहरम्}
{धूपं गृहाण देवेश सर्वभूत मनोहर}
\textbf{\devAya{} धूपमाघ्रापयामि।}
\medskip

\twolineshloka*
{ब्रा॒ह्म॒णो᳚ऽस्य॒ मुख॑मासीत्। बा॒हू रा॑ज॒न्यः॑ कृ॒तः}
{ऊ॒रू तद॑स्य॒ यद्वैश्यः॑। प॒द्भ्याꣳ शू॒द्रो अ॑जायत}
उद्दी᳚प्यस्व जातवेदोऽप॒घ्नन्निर्ऋ॑तिं॒ मम॑।\\
प॒शूꣳश्च॒ मह्य॒माव॑ह॒ जीव॑नं च॒ दिशो॑ दिश॥ \\
मा नो॑ हिꣳसीज्जातवेदो॒ गामश्वं॒ पुरु॑षं॒ जग॑त्।\\
अबि॑भ्र॒दग्न॒ आग॑हि श्रि॒या मा॒ परि॑पातय॥ \\
\textbf{\devAya{} अलङ्कारदीपं सन्दर्शयामि।}
\medskip

\twolineshloka*
{च॒न्द्रमा॒ मन॑सो जा॒तः। चक्षोः॒ सूर्यो॑ अजायत}
{मुखा॒दिन्द्र॑श्चा॒ग्निश्च॑। प्रा॒णाद्वा॒युर॑जायत}
\textbf{\devAya{} \mbox{(~~~)} निवेदयामि।} अमृतापिधानमसि।\\
निवेदनानन्तरम् आचमनीयं समर्पयामि।\medskip


\twolineshloka*
{नाभ्या॑ आसीद॒न्तरि॑क्षम्। शी॒र्ष्णो द्यौः सम॑वर्तत}
{प॒द्भ्यां भूमि॒र्दिशः॒ श्रोत्रा᳚त्। तथा॑ लो॒काꣳ अ॑कल्पयन्}

\twolineshloka*
{पूगीफलसमायुक्तं नागवल्लीदलैर्युतम्}
{कर्पूरचूर्णसंयुक्तं ताम्बूलं प्रतिगृह्यताम्}
\textbf{\devAya{} कर्पूरताम्बूलं समर्पयामि।}
\medskip

\twolineshloka*
{वेदा॒हमे॒तं पुरु॑षं म॒हान्तम्᳚। आ॒दि॒त्यव॑र्णं॒ तम॑स॒स्तु पा॒रे}
{सर्वा॑णि रू॒पाणि॑ वि॒चित्य॒ धीरः॑। नामा॑नि कृ॒त्वाऽभि॒वद॒न्॒ यदास्ते᳚}
\textbf{\devAya{} समस्त-अपराध-क्षमापनार्थं कर्पूरनीराजनं दर्शयामि।}\\
कर्पूरनीरजनानन्तरम् आचमनीयं समर्पयामि।
\medskip

\twolineshloka*
{धा॒ता पु॒रस्ता॒द्यमु॑दाज॒हार॑। श॒क्रः प्रवि॒द्वान्  प्र॒दिश॒श्चत॑स्रः}
{तमे॒वं वि॒द्वान॒मृत॑ इ॒ह भ॑वति। नान्यः पन्था॒ अय॑नाय विद्यते}
यो॑ऽपां पुष्पं॒ वेद॑। पुष्प॑वान् प्र॒जावा᳚न् पशु॒मान् भ॑वति।\\
च॒न्द्रमा॒ वा अ॒पां पुष्पम्᳚। पुष्प॑वान् प्र॒जावा᳚न् पशु॒मान् भ॑वति।\\
य ए॒वं वेद॑। यो॑ऽपामा॒यत॑नं॒ वेद॑। आ॒यत॑नवान् भवति।\\

ओं᳚ तद्ब्र॒ह्म। ओं᳚ तद्वा॒युः। ओं᳚ तदा॒त्मा। ओं᳚ तथ्स॒त्यम्‌।\\
ओं᳚ तथ्सर्वम्᳚‌। ओं᳚ तत्पुरो॒र्नमः॥\\

अन्तश्चरति॑ भूते॒षु॒ गुहायां वि॑श्वमू॒र्तिषु। \\
त्वं यज्ञस्त्वं वषट्कारस्त्वमिन्द्रस्त्वꣳ रुद्रस्त्वं विष्णुस्त्वं ब्रह्म त्वं॑ प्रजा॒पतिः। \\
त्वं त॑दाप॒ आपो॒ ज्योती॒ रसो॒ऽमृतं॒ ब्रह्म॒ भूर्भुव॒स्सुव॒रोम्‌॥\\

\textbf{\devAya{} वेदोक्तमन्त्रपुष्पाञ्जलिं समर्पयामि।}
\medskip

\twolineshloka*
{सुवर्णरजतैर्युक्तं चामीकरविनिर्मितम्}
{स्वर्णपुष्पं प्रदास्यामि गृह्यतां मधुसूदन}
\textbf{\devAya{} स्वर्णपुष्पं समर्पयामि।}
\medskip

\twolineshloka*
{प्रदक्षिणं करोम्यद्य पापानि नुत माधव}
{मयार्पितान्यशेषाणि परिगृह्य कृपां कुरु}


\twolineshloka*
{यानि कानि च पापानि जन्मान्तरकृतानि च}
{तानि तानि विनश्यन्ति प्रदक्षिण-पदे पदे}
\textbf{प्रदक्षिणं कृत्वा।}
\medskip

\twolineshloka*
{नमस्ते देवदेवेश नमस्ते भक्तवत्सल}
{नमस्ते पुण्डरीकाक्ष वासुदेवाय ते नमः}

\twolineshloka*
{नमः सर्वहितार्त्थाय जगदाधाररूपिणे}
{साष्टाङ्गोऽयं प्रणामोस्तु जगन्नाथ मया कृतः}
\textbf{\devAya{} अनन्तकोटिप्रदक्षिणनमस्कारान् समर्पयामि।}
 \medskip

\twolineshloka*
{य॒ज्ञेन॑ य॒ज्ञम॑यजन्त दे॒वाः। तानि॒ धर्मा॑णि प्रथ॒मान्या॑सन्}
{ते ह॒ नाकं॑ महि॒मानः॑ सचन्ते। यत्र॒ पूर्वे॑ सा॒ध्याः सन्ति॑ दे॒वाः}
\textbf{\devAya{} छत्त्रचामरादिसमस्तोपचारान् समर्पयामि।}
\medskip

हिरण्यगर्भगर्भस्थं हेमबीजं विभावसोः।\\
अनन्तपुण्यफलदम् अतः शान्तिं प्रयच्छ मे॥\\

धन्वन्तरिजयन्ती-पुण्यकालेऽस्मिन् मया क्रियमाण-धन्वन्तरिपूजायां यद्देयमुपायनदानं तत्प्रत्याम्नायार्थं
हिरण्यं श्री-धन्वन्त्रिप्रीतिं 
कामयमानः मनसोद्दिष्टाय ब्राह्मणाय सम्प्रददे नमः न मम। 
अनया पूजया श्री-धन्वन्तरिः प्रीयताम्। 

\dnsub{विसर्जनम्}
\twolineshloka*
{यस्य स्मृत्या च नामोक्त्या तपः-पूजा-क्रियादिषु}
{न्यूनं सम्पूर्णतां याति सद्यो वन्दे तमच्युतम्}

\twolineshloka*
{इदं व्रतं मया देव कृतं प्रीत्यै तव प्रभो}
{न्यूनं सम्पूर्णतां यातु त्वत्प्रसादाज्जनार्द्दन}

अस्मात् बिम्बात् श्री-धन्वन्तरिं यथास्थानं प्रतिष्ठापयामि\\
(अक्षतानर्पित्वा देवमुत्सर्जयेत्।)\\
अनया पूजया श्री-धन्वन्तरिः प्रीयताम्।\\

ॐ तत्सद्ब्रह्मार्पणमस्तु॥

\input{../stotra-sangrahah/stotras/vishnu/DhanvatariStotram.tex}

\closesection

\end{center}