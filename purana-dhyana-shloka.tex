%%% PURANA DHYANAM ETC.
\newcommand{\ganapatyadiDhyanam}{

\dnsub{श्री-महागणपति-प्रार्थना}

\twolineshloka*
{शुक्लाम्बरधरं विष्णुं शशिवर्णं चतुर्भुजम्}
{प्रसन्नवदनं ध्यायेत् सर्वविघ्नोपशान्तये}

\twolineshloka*
{वागीशाद्याः सुमनसः सर्वार्थानामुपक्रमे}
{यं नत्वा कृतकृत्याः स्युस्तं नमामि गजाननम्}

\dnsub{श्री-गुरु-प्रार्थना}

\twolineshloka*
{गुरुर्ब्रह्मा गुरुर्विष्णुर्गुरुर्देवो महेश्वरः}
{गुरुः साक्षात् परं ब्रह्म तस्मै श्री-गुरवे नमः}

\twolineshloka*
{सदाशिवसमारम्भां शङ्कराचार्यमध्यमाम्}
{अस्मदाचार्यपर्यन्तां वन्दे गुरुपरम्पराम्}

\dnsub{श्री-सरस्वती-प्रार्थना}
\fourlineindentedshloka*
{दोर्भिर्युक्ता चतुर्भिः स्फटिकमणिनिभैरक्षमालां दधाना}
{हस्तेनैकेन पद्मं सितमपि च शुकं पुस्तकं चापरेण}
{भासा कुन्देन्दुशङ्खस्फटिकमणिनिभा भासमानाऽसमाना}
{सा मे वाग्देवतेयं निवसतु वदने सर्वदा सुप्रसन्ना}

\dnsub{श्री-व्यास-नमस्क्रिया}
\resetShloka

\twolineshloka
{व्यासं वसिष्ठनप्तारं शक्तेः पौत्रमकल्मषम्}
{पराशरात्मजं वन्दे शुकतातं तपोनिधिम्}

\fourlineindentedshloka
{अभ्रश्यामः पिङ्गजटाबद्धकलापः}
{प्रांशुर्दण्डी कृष्णमृगत्वक्परिधानः}
{साक्षाल्लोकान्पावयमानः कविमुख्यः}
{पाराशर्यः पर्वसु रूपं विवृणोतु}

\twolineshloka
{जयति पराशरसूनुः सत्यवतीहृदयनन्दनो व्यासः}
{यस्यास्यकमलगलितं वाङ्मयममृतं जगत् पिबति}
}

\newcommand{\genMangalaShloka}{
\dnsub{मङ्गलश्लोकाः}
\fourlineindentedshloka*
{स्वस्ति प्रजाभ्यः परिपालयन्तां}
{न्यायेन मार्गेण महीं महीशाः}
{गोब्राह्मणेभ्यः शुभमस्तु नित्यं}
{लोकाः समस्ताः सुखिनो भवन्तु}

\twolineshloka*
{काले वर्षतु पर्जन्यः पृथिवी सस्यशालिनी}
{देशोऽयं क्षोभरहितो ब्राह्मणाः सन्तु निर्भयाः}

\twolineshloka*
{अपुत्राः पुत्रिणः सन्तु पुत्रिणः सन्तु पौत्रिणः}
{अधनाः सधनाः सन्तु जीवन्तु शरदां शतम्}

\twolineshloka*
{यदक्षरपदभ्रष्टं मात्राहीनं तु यद् भवेत्}
{तत् सर्वं क्षम्यतां देव नारायण नमोऽस्तुते}

\twolineshloka*
{विसर्गबिन्दुमात्राश्च पदपादाक्षराणि च}
{न्यूनानि चातिरिक्तानि क्षमस्व पुरुषोत्तम}

\twolineshloka*
{यज्ञेशाच्युत गोविन्द माधवानन्त केशव}
{कृष्ण विष्णो हृषीकेश वासुदेव नमोऽस्तुते}

\fourlineindentedshloka*
{कायेन वाचा मनसेन्द्रियैर्वा}
{बुद्‌ध्याऽऽत्मना वा प्रकृतेः स्वभावात्}
{करोमि यद्यत् सकलं परस्मै}
{नारायणायेति समर्पयामि}}

\newcommand{\lingaPuranam}{
\dnsub{लिङ्गपुराणम् — मङ्गलश्लोकः}
\twolineshloka*
{नमो रुद्राय हरये ब्रह्मणे परमात्मने}
{प्रधानपुरुषेशाय सर्गस्थित्यन्तकारिणे}
}

\newcommand{\padmaPuranam}{
\dnsub{पाद्ममहापुराणम् — मङ्गलश्लोकः}
\fourlineindentedshloka*
{स्वच्छं चन्द्रावदातं करिकर-मकर-क्षोभ-सञ्जात-फेनं}
{ब्रह्मोद्भूतिप्रसक्तैर्व्रतनियमपरैः सेवितं विप्रमुख्यैः}
{ओङ्कारालङ्कृतेन त्रिभुवनगुरुणा ब्रह्मणा दृष्टिभूतं}
{सम्भोगा भोग रम्यं जलमशुभहरं पौष्करो नः पुनातु }
}

\newcommand{\bhagavatam}{
\dnsub{श्रीमद्भागवतम् — मङ्गलश्लोकः}
\fourlineindentedshloka*
{जन्माद्यस्य यतोऽन्वयादितरतश्चार्थे स्वभिज्ञः स्वराट्}
{ते ने ब्रह्महृदा य आदिकवये मुह्यन्ति यत् सूरयः}
{तेजोवारिमृदां यथा विनिमयो यत्र त्रिसर्गोऽमृषा}
{धाम्ना स्वेन सदा निरस्तकुहकं सत्यं परं धीमहि}
}

\newcommand{\skandaPuranam}{
\dnsub{स्कन्दमहापुराणम् — मङ्गलश्लोकः}
\twolineshloka*
{यस्याऽऽज्ञया जगत्स्रष्टा विरञ्चिः पालको हरिः}
{संहर्ता कालरूपाख्यो नमस्तस्मै पिनाकिने}
}