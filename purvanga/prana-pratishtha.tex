\dnsub{प्राण-प्रतिष्ठा}
\begin{center}
अस्य श्रीप्राणप्रतिष्ठामहामन्त्रस्य ब्रह्मविष्णुमहेश्वरा ऋषयः।\\
ऋग्यजुस्सामाथर्वाणि छन्दांसि। प्राणशक्तिः परा देवता।

आं बीजम् - ह्रीं शक्तिः - क्रों कीलकम्। प्राणप्रतिष्ठापने विनियोगः॥

\begin{minipage}{\linewidth}
\begin{multicols}{2}
    \centering
    \textbf{॥करन्यासः॥}
    
    \begin{tabular}{lll}
        आं & अङ्गुष्ठाभ्यां & नमः।\\
        ह्रीं & तर्जनीभ्यां & नमः।\\
        क्रों & मध्यमाभ्यां & नमः।\\
        आं & अनामिकाभ्यां & नमः।\\
        ह्रीं & कनिष्ठिकाभ्यां & नमः।\\
        क्रों & करतलकरपृष्ठाभ्यां & नमः।\\
         & \\
    \end{tabular}

    \columnbreak
    
    \textbf{॥अङ्गन्यासः॥}

    \begin{tabular}{lll}
        आं & हृदयाय & नमः।\\
        ह्रीं & शिरसे & स्वाहा।\\
        क्रों & शिखायै & वषट्।\\
        आं & कवचाय & हुम्।\\
        ह्रीं & नेत्रत्रयाय & वौषट्।\\
        क्रों & अस्त्राय & फट्।\\
        \multicolumn{2}{l}{भूर्भुवस्सुवरोमिति} & दिग्बन्धः॥\\
    \end{tabular}
    

\end{multicols}
\end{minipage}

\textbf{॥ध्यानम्॥}

\fourlineindentedshloka*
{रक्ताम्भोधिस्थपोतोल्लसदरुणसरोजाधिरूढा कराब्जैः}
{पाशं कोदण्डमिक्षूद्भवमळिगुणमप्यङ्कुशं पञ्चबाणान्}
{बिभ्राणासृक् कपालं त्रिनयनलसिता पीनवक्षोरुहाढ्या}
{देवी बालार्कवर्णा भवतु सुखकरी प्राणशक्तिः परा नः}


आं ह्रीं क्रों, क्रों ह्रीं आं, य र ल व श ष स हों,\\
हंसः सोऽहं सोऽहं हंसः। अस्यां मूर्तौ प्राणास्तिष्ठन्तु। जीवस्तिष्ठतु। 

अस्यां मूर्तौ सर्वेन्द्रियाणि मनस्त्वक्\-चक्षुश्श्रोत्र\-जिह्वा\-घ्राण\-वाक्\-पाणि\-पाद\-पायूपस्थाख्यानि प्राणापान\-व्यानोदान\-समानाश्चात्रागत्य सुखं चिरं तिष्ठन्तु स्वाहा।

असु॑नीते॒ पुन॑र॒स्मासु॒ चक्षुः॒ पुनः॑ प्रा॒णमि॒ह नो॑ धेहि॒ भोग॑म्।\\
ज्योक्प॑श्येम॒ सूर्य॑मु॒च्चर॑न्त॒मनु॑मते मृ॒ळया॑ नः स्व॒स्ति॥\rlap{ऋक्~१०.५९.६॥}

प्राणान् प्रतिष्ठापयामि।

\end{center}